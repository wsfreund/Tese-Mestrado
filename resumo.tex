\begin{abstract}

A monitoração não-invasiva pode ser aplicada para garantir a qualidade
de energia, o diagnóstico de carga, identificação de aparelhos
defeituosos ou com consumo excessivo de energia e eficiência
energética. Este trabalho irá tratar do desenvolvimento da tecnologia
quando abordando o tema de eficiência energética no setor residencial,
aonde será abordado a etapa de detecção de eventos de transitórios
causados por aparelhos elétricos. Para isso, foi empregado um filtro
de núcleo de Gaussiana obtendo, de sua resposta, as regiões sensibilizadas
que indicam alteração de estado nos equipamentos. Os parâmetros da
abordagem foram otimizados através de um algoritmo genético. 
%Para
%auxiliar na compreensão do problema, empregou-se Mapas
%Auto-Organizáveis para explorar as características dos distúrbios
%devido à alteração de um estado de consumo de um aparelho.
%Estudou-se, também, a possibilidade de empregá-los como um detector de
%eventos. 
Obteve-se taxa de detecção superior a 83\% enquanto
com falso alarme inferior a 17\% em cenário de aplicação
prática. A taxa de falso alarme pode ser representada por valores
menores, dependendo de como se considera o problema.
% TODO escrever sobre motivo do FA alto.

\end{abstract}

