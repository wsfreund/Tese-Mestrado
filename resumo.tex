\begin{abstract}

Este trabalho foi desenvolvido em colaboração com o \acs{cepel} no
desenvolvimento da tecnologia conhecida como \acf{nilm}. 
A monitoração não-invasiva pode ser aplicada para garantir a qualidade
de energia, o diagnóstico de carga, identificação de aparelhos
defeituosos ou com consumo excessivo de energia e eficiência
energética. Este trabalho irá tratar do desenvolvimento da tecnologia
quando abordando o tema de eficiência energética no setor residencial,
aonde será abordado a etapa de detecção de eventos de transitórios
causados por aparelhos elétricos, o ponto que tem sido atacado pelo
\acs{cepel} e necessário para a obtenção dos traços informação
deixados por equipamentos durante a alteração de seu estado de
operação. A proposta do \acs{cepel} é a utilização de um filtro de
derivada de Gaussiana que gera candidatos a eventos de transitórios,
os mesmos ainda analisados por outros testes para eliminação de falsos
eventos em seguida. 

A contribuição do trabalho foi a sistematização da determinação dos
parâmetros necessários para o método proposto pelo \acs{cepel}, que
permite o ajuste automático dos mesmos para diferentes cenários. Para
isso, foi implementado uma adaptação de um algoritmo genético, que
permite dinâmica para alocação de maior esforço computacional para
configurações que melhor se adequam ao problema. Além disso, o
otimizador faz parte de um extenso ambiente de análise que trouxe
vantagens operacionais em termo de infraestrutura para o projeto.

Ao aplicar o algoritmo na base de dados, o mesmo mostrou capacidade de
generalização para as condições avaliadas. Obteve-se taxa de detecção
superior a 80\% para taxa de falso alarme inferior a 1 \% em cenários
com operação simultânea de diversos aparelhos e dinâmica de carga. Já
para os conjuntos com operações simples de equipamentos as taxas são
de 98,1~\% e 0,6~\%, respectivamente.

\end{abstract}

