\begin{abstract}

A monitoração não-invasiva pode ser aplicada para garantir a qualidade
de energia, o diagnóstico de carga, identificação de aparelhos
defeituosos ou com consumo excessivo de energia e eficiência
energética. Este trabalho irá tratar do desenvolvido da tecnologia
quando abordando o tema de eficiência energética no setor residencial,
trabalhando a etapa de detecção de eventos de transitórios causados
por aparelhos elétricos. Para isso, foi empregado um filtro de núcleo
de Gaussiana obtendo de sua resposta regiões
sensibilizadas que indicam alteração de estado nos equipamentos. Os
parâmetros da abordagem foram otimizados através de um algoritmo
genético. Para auxiliar na compreensão do problema, empregou-se Mapas
Auto-Organizáveis para explorar as características dos distúrbios
devido à alteração de um estado de consumo de um aparelho. Estudou-se,
também, a possibilidade de empregá-los para aperfeiçoar a
metodologia anterior. Obteve-se taxa de detecção superiores a 84\%
enquanto obtendo falso alarme na inferiores a 16\% em cenário de
aplicação prática.

\end{abstract}

