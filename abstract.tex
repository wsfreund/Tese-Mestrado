\begin{foreignabstract}

This work was done in collaboration with \acs{cepel} at the
development of the technology known as Non-Intrusive Appliance Load
Moniroring (NILM). Non-intrusive monitoring may be applied to power
quality, load diagnosis, faulty appliance detection or with excessive
energy consumption and energy efficiency. This work will address the
step of detecting transients events caused by appliance loads, which
is required for obtaining the traces of information left by the
appliance while changing its consumption status. The technique
proposed by \acs{cepel} is a Gaussian derivative filter which
generates the transient events candidates, also analyzed by other
checks that further eliminate false alarm events.

The contribution of this work is on the systematic determination of
the parameters required for the proposed \acs{cepel}'s method, which
allows their automatic adjustment for several scenarios. To this end,
it was implemented an adaptation of a genetic algorithm, that has as
its feature the dynamic allocation of more computational effort for
those configurations which are best suited for solving the problem.
Furthermore, the optimizer is part of an extensive analysis
environment which led to operational advantages in terms of
infrastructure for the project.

When applying the algorithm to distinct datasets, it showed
generalization properties for the conditions evaluated. The results
were compared with related recent work. Detection rate above 80~\%,
meanwhile false alarm rate less than 1~\%, were achieved for practical
scenario applications with several simultaneous appliances operation
and load dynamics. Yet for datasets with naive appliance operation
conditions, those rates were respectively of 98,1~\% and 0,6~\%.


\end{foreignabstract}

