\chapter{A Eficiência Energética no Setor Residencial}
\label{chap:ee_retorno}

O setor residencial representa um grande potencial para o alcance de uma melhor
\gls{ee}, em especial quando considerando apenas o uso da eletricidade. Isso se
tornará ainda mais relevante com o desenvolvimento do país. Este Capítulo detalha
o seu perfil de consumo e potencial para explorar a \gls{ee}, dando 
especial enfase a eletricidade (Sessão~\ref{sec:ee_setor_residencial}). Se faz
uma referência as novas tendências para explorar esse potencial no
mundo (Sessão~\ref{sec:ee_res_exp}), que podem utilizar das possibilidades 
fornecidas pelas redes inteligentes (\emph{smart grid}) e seus aparelhos de 
medição inteligentes (\emph{advanced/smart meters}).

\section{Eletricidade e Potencial de Eficiência Energética 
no Setor Residêncial}
\label{sec:ee_setor_residencial}

A eletricidade é o segundo meio energético de maior participação na matriz 
energética brasileira, representando 16,7\% da demanda, 
Figura~\ref{fig:matriz_bra_2011}. 
Há uma tendência de crescimento dessa parcela na matriz indicado na 
Figura~\ref{fig:matriz_bra_evo}, que deverá progredir devido à fatores 
como \cite{iea_weo2010}:

\begin{figure}[h!t]
    \label{fig:eletricidade_brasil}
    \begin{center}
%
        \subfigure[]{%
            \label{fig:matriz_bra_2011}
            \includegraphics[width=0.8\textwidth]{imagens/matriz_energetica_brasileira_2011.pdf}
        } \\ %\hspace{0.05\textwidth}
        \subfigure[]{%
            \label{fig:matriz_bra_evo}
            \includegraphics[width=0.8\textwidth]{imagens/evolucao_matriz_energetica_brasil.pdf}
        }
%
    \end{center}
    \caption[Matriz energética brasileira.]{A matriz elétrica brasileira
em 2011 (a) e seu histório (b). Baseado (a) e extraído (b) de \cite{ben2012}.}% 
\end{figure}

\begin{itemize}
\item substituição da biomassa para eletricidade como meio de iluminação 
e aquecimento
no setor residêncial;
\item nesse setor também há um aumento do número de eletrodomésticos como
reflexo do desenvolvimento e melhor distribuição da riqueza;
\item a expansão do setor comercial e de serviços utilizando uma quantidade
maiores de aparelhos elétricos (como ar condicionado, iluminação e equipamentos
de \acrshort{ti}); 
\item transformação do setor industrial, gradualmente substituindo o carvão e 
aumentando a presença de dispositivos elétricos.
\end{itemize}

No que diz respeito ao consumo de eletricidade, 
Figura~\ref{fig:eletricidade_por_setor}, 
o setor residencial possui uma posição de destaque no consumo, com uma demanda
de 111,97 T\acrshort{wh} e uma parcela equivalente à 26\% do total no ano de 2011, 
exprimindo a importância desse setor para explorar a capacidade de economia de 
eletricidade através de \gls{ee}. 

\begin{figure}[h!t]
\centering
\includegraphics[width=.6\textwidth]{imagens/consumo_por_setor.pdf}
\caption[Consumo de eletricidade por setor em 2011.]
{O consumo de eletricidade por setor em 2011. Baseado em \cite{ben2012}.}
\label{fig:eletricidade_por_setor}
\end{figure}

De acordo com o último estudo do \gls{beu}\footnote{O \gls{beu} é realizado em
intervalos decenais desde 1985.} com o ano base de 2004 \cite{beu}, o potencial de 
economia no consumo de eletricidade nesse setor chega a 13,60 T\acrshort{wh}, 
ou 1,17 milhões de \acrshort{toe}. Apenas como figura de mérito, quando considerando 
melhorias de eficiência em todas as fontes energéticas, o setor residencial
possui o terceiro maior potencial de economia, após o setor industrial e de
transporte, com uma capacidade de economia de 2,97 milhões de 
\acrshort{toe}. No entando, vale ressaltar que esses valores de potenciais de economia 
apresentados são aproximados e reduzidos em relação ao valor real, onde se
considera apenas a perda de energia na primeira transformação do processo
produtivo. Também não são consideradas possíveis alterações no consumo de fontes
energéticas, como a mencionada alteração de biomassa para eletricidade, uma
fonte menos poluente e mais eficiente. Finalmente, deve-se acrescentar que 
esse potencial é calculado utilizando o rendimento de equipamentos no estado 
da arte entre aqueles normalmente comercializados, e não aqueles possíveis de 
se alcançar quando considerando a literatura técnica.

Os dados utilizados pela \gls{epe} para os estudos de \gls{ee} no setor
residencial, assim como os apresentados no \gls{pne2030} e \gls{pde}, no entanto,
utilizam as informações obtidas da última \gls{pph} realizada no Brasil entre 2004--2006 
que possibilitam o estudo baseado em uma abordagem desagregada. Essa abordagem 
depende do número de domícilios, a posse média e hábito de consumo específico 
dos equipamentos eletrodomésticos, que estão implicitamente indicados na curva de carga
ilustrada na Figura~\ref{fig:curva_carga} para a região sudeste, e rendimento médio desses 
equipamentos no país, informação baseada nas tabelas do \gls{pbe}, coordenado pelo \gls{inmetro}. 
Também são utilizadas variáveis agregadas para o ajuste do modelo
utilizado, sendo elas a relação entre o número de consumidores residenciais e população 
(que permite a projeção do número de consumidores a partir da projeção da população), 
e consumo médio por consumidor residencial
\cite{epe_eficiencia_2012,pde_2012,pne30_eff_energ}. 

% Inserir a figura aqui de posse média
\begin{figure}[h!t]
\centering
\includegraphics[width=.9\textwidth]{imagens/curva_demanda_sudeste.pdf}
\caption[Curva de carga média para a região sudeste, ano base 2005.]{Curva de
carga média para a região sudeste, ano base 2005. Extraído de
\cite{result_procel_2005}.}
\label{fig:curva_carga}
\end{figure}

Esses estudos exploram a conservação de energia no ganho de eficiência 
de equipamentos eletrodomésticos, mas não consideram melhoras possíveis 
em mudanças nos hábitos de consumo. 
Isso pode ser justificado pela dificuldade de prever essas mudanças 
e assim o seu potencial, em especial devido a precariedade --- e talvez ausência
--- de estudos nesse sentido no Brasil. Mesmo não participando das
estimativas do potencial de conservação por \gls{ee}, esse potencial não é totalmente 
negligenciado no Brasil: \textlabel{existem programas do
\gls{procel}}{text:prog_cepel} no intuíto de insentivar e 
conscientizar a população, como o Procel Educação, cujas linhas de ``Educação para Eficiência
Energética na Educação Básica: O Procel nas Escolas'' e ``Educação para
Eficiência Energética na Formação Profissional: níveis técnico, superior e
pós-graduação'' trabalham no sentido de difundir a \gls{ee}
\cite{procel_resultados_2012}.

Entretanto, estudos no exterior mostram um potencial ainda a ser explorado de
economia de energia elétrica no setor residencial, através do retorno de
informação da utilização de energia para o consumidor. Conforme será visto na
próxima sessão, esse potencial depende de aspectos, como a quantidade 
de informação retornada ao consumidor, inerentemente ligado à quantidade de
investimento utilizada na tecnologia para permitir um maior retorno, mas ainda,
dependem do modo em que esse retorno é fornecido para o consumidor no intuíto de
motivar ações sustentaveis de energia, sendo um problema complexo dependente de
aspectos sociais, culturais e psicológicos. 

Por fim, outros setores também podem reduzir seu consumo com políticas de
\gls{ee} aplicadas para o setor residencial devido a sua natureza similar a esse
setor. Os setores público e comercial, por exemplo, possuem prédios com natureza
de consumo correspondente ao do setor residencial, de forma que estratégias 
desenvolvidas possam sinergeticamente apresentar um potencial maior de 
economia de energia na matriz brasileira, em especial no que concerne estudos
para melhor eficiência de equipamentos.

\section{Expandindo o Potencial de Eficiência Energética}
\label{sec:ee_res_exp}

% Invisibilidade da eletricidade para os consumidores residencias
As novas fontes energéticas, dentre elas a eletricidade e gás natural e 
que atendem a demanda dos consumidores residenciais para a vasta 
variedade de serviços nas quais são utilizadas 
--- desde cocção, condicionamento do ambiente, a lazer e 
entreterimento ---, fluem invisivel e silenciosamente para seus domicílios, sem
deixar qualquer traço notável de sua utilização além do efeito final desejado pelo
consumidor. Para eles, o único retorno de seu consumo é informado na conta
apresentada pela concessionária, fornecida em um longo período após o consumo,
no início do mês. As informações nas contas são precárias, não informando 
muito além do total de energia consumido e o preço de energia. 
Os usuários não tem como inferir quais são os meios de uso final que demandam 
maior energia, nem a que ponto possíveis mudanças podem afetar sua demanda, 
seja através da mudança de seus hábitos ou na escolha 
de aparelhos mais eficientes. Atualmente, os usuários estão cegos quanto a essas
mudanças, não é possível vizualisar a energia que consomem. Além disso, as
informações fornecidas não permitem o consumidor comparar seu consumo com o de
outros, de modo que ele não é capaz de criar uma referência social para seu consumo.
Sem uma referência, o consumidor tem dificuldades para determinar se o consumo é
excessivo ou moderado e se é necessário algum tipo de intervenção 
\cite{aceee_2010_estudos_feedback}.

% Modos de alterar o hábito de consumo
Estratégias para intervir no comportamento podem ser classificadas de dois modos
\cite{aceee_2010_estudos_feedback,2009_epri}:
\begin{enumerate}
\item \textbf{Antecedentes}, que envolvem esforços para influênciar o que define 
um comportamento antes de sua realização; 
\item \textbf{De consequência}, que buscam alterar o que determina o 
comportamento, após que ele tenha ocorrido. 
\end{enumerate}

Exemplos de estratégias antecedentes são campanhas de informação 
com o objetivo de aumentar o conhecimento público sobre o impacto de suas 
escolhas e das opções para econômia de energia disponíveis --- como 
as já citadas ações do \gls{procel} (ver página~\pageref{text:prog_cepel}) ---,
engajar o indivíduo com um compromisso de mudança, criar metas de mudança
comportamentais, ou modelar e demonstrar o comportamento desejados. Para
estratégias de consequência pode-se citar recompensas, punições ou o
retorno de informação \cite{aceee_2010_estudos_feedback,2009_epri}. 

Iniciativas utilizando o retorno de informação se mostraram altamente eficientes
em mudanças comportamentais com relação ao consumo energético \cite{
aceee_2010_estudos_feedback,2009_epri,2012_schleich__austria,
2011_zhifeng_smart_energy_savings,2006_darby,2009_nber_studies_us,
ucla_studies_1975_2011_usa}. O uso do retorno
de informação se basea em que tanto resultados positivos ou
negativos podem modelar o comportamento. Resultados atribuidos como positivos 
irão torná-los em comportamentos mais atraentes, enquanto a atribuição de
resultados negativos propiciam comportamentos ruins em menos desejáveis. 
Sempre que possível, a atribuição negativa 
deve ser evitada, pois ela tende a reduzir a motivação e não coloca nada no 
lugar do comportamento evitado \cite{2010_aspectos_psicologicos_usa}.

A questão, entretanto, não é apenas fornecer retorno do consumo ao usuário final 
--- a própria conta de energia pode ser encarada como um meio de retorno ---, 
mas como o retorno pode ser utilizado para efetivamente motivar pessoas 
para reduzir o seu consumo. Algumas considerações devem ser tomadas: primeiro, 
quais são os tipos de retornos disponíveis (Subsessão \ref{ssec:ret_tipos}) e, 
dentre eles, quais tem mostrado resultados mais eficientes na redução do consumo
energético (Subsessão \ref{ssec:ret_eff})? O que mais deve ser levado em
consideração quando preparando programas de \gls{ee}
(Subsessão~\ref{ssec:ret_outros})? Quais são as técnologias 
disponíveis para fornecer essa informação e 
suas tendências (Subsessão \ref{ssec:ret_tec})? 
Ainda, pessoas possuem diferentes atitudes, crenças e valores, sendo motivadas 
de modos distintos. Uma breve consideração sobre a perspectiva psicológica que
envolve mudança comportamnetal será realizada (Subsessão \ref{ssec:asp_psic})
uma vez que esse aspecto é de principal relevância para o sucesso dos programas. 
Do mesmo modo, a apresentação visual da informação também irá influênciar no 
êxito, e por isso o tema também sendo colocado em pauta
(Subsessão~\ref{ssec:asp_visuais}). De nenhuma maneira as breves considerações 
devem substituir a análise de profissionais dessas áreas, servindo apenas para 
chamar atenção para a importância, bem como introduzir os leitores, aos temas. 

\subsection{Tipos de Retorno}
\label{ssec:ret_tipos}

A categorização dos tipos de retorno que será apresentada iniciou-se em
\cite{2000_darby} e depois foi aprimorada por \cite{2009_epri}. A primeira divisão
toma em conta o modo no qual o retorno é fornecido, sendo possíveis o retorno 
direto ou indireto. O termo \emph{indireto} é
utilizado quando há alguma espécie de processamento antes de atingir o
consumidor, enquanto \emph{direto} determina o retorno instantaneamente entregue
ao usuário. Em seguida, realiza-se uma divisão em termos de frequência,
diferenciando quatro tipos de retorno \emph{indiretos}, e dois retornos
\emph{diretos}. A divisão é apresentada resumidamente em ordem crescente em 
termos de custo e 
quantidade de informação disponível, onde os dois retornos \emph{diretos} estão no
final da lista \cite{aceee_2010_estudos_feedback,2009_epri}:

\begin{enumerate}
\item \textbf{Retorno por Faturamento Simples}: conta de energia contendo 
k\acrshort{wh} 
consumido, o preço da tarifa unitária ($\text{R\$}/$k\acrshort{wh}), o custo 
total e outros possíveis ônus. Nessa forma de retorno normalmente carece 
estatísticas comparativas ou qualquer informação detalhada sobre os aspectos 
temporais do consumo;
\item \textbf{Retorno por Faturamento Aprimorado}: fornece informações mais 
detalhadas
sobre o padrão de consumo de energia, incluindo em alguns casos estatísticas
comparativas, tanto comparando o maior consumo do mes atual e sua despesa
aliados ao consumo histórico e/ou a comparação com outros domicílios
pertencentes ao grupo do consumidor;
\item \textbf{Retorno Estimado}: essa abordagem utiliza geralmente de técnicas
estatísticas para desagregar o total de energia baseado no tipo do
domicílio do consumidor, informação de aparelhos e dados de faturamento. O
retorno resultante fornece um relato detalhado do uso de eletricidade pelos
utensilios e dispositivos de maior importância. A forma mais comum é através de
ferramentas de auditoria de energia residencial baseadas na internet, oferecida
por um fornecedor de serviços a seus consumidores;
\item \textbf{Retorno Diário/Semanal}: esses relatórios utilizam a média de
dados e frequentemente incluem estudos de leitura dos medidores pelos próprios 
consumidores, assim como estudos nos quais individuos são providos com
relatórios mensais ou semanais do fornecedor de serviços ou entidade de
pesquisa;
\item \textbf{Retorno em Tempo-Real}: fornecido por dispositivos que exibem o
consumo em (pratiamente) tempo-real e informações de custo da energia em nível
agregado domiciliar;
\item \textbf{Retorno em Tempo-Real Desagregado}: nesse caso, as informações são
exibidas desagregadas ao nível dos utensílios.
\end{enumerate}

\subsection{Resultados por Tipo de Retorno}
\label{ssec:ret_eff}

Há uma farta quantidade de estudos envolvendo o tema, com estudos iniciando na
década de 1970 em resposta à Crise do Petróleo, que tiveram um declínio durante
a década posterior. O interesse retornou a pauta recentemente devido à crescente 
preocupação com o meio ambiente e mudanças climáticas assim 
como o aparecimento de novas possibilidades tecnológicas 
associadas a \gls{ict} \cite{aceee_2010_estudos_feedback}. É
possível encontrar pesquisas nas quais se compilam diversos estudos
de retorno de informação para consumidores residenciais no intuíto de generalizar 
resultados \cite{aceee_2010_estudos_feedback,2011_zhifeng_smart_energy_savings,
2006_darby,2009_nber_studies_us,ucla_studies_1975_2011_usa}.

A pesquisa que mais se destacou \cite{aceee_2010_estudos_feedback} 
revisou 57 estudos primários de retorno, realizado em países 
desenvolvidos incluindo \gls{eua} (58\% dos
estudos), quatro países da Europa Ocidental (Países Baixos, Finlândia,
Dinamarca e Reino Unido, com um total de 22\%), Canada (15\%), Japão (5\%) e 
Australia (1 estudo). Em termos de retorno, metade dos estudos envolvem retorno
\emph{indireto}, dentre os quais 11 envolvem Faturamento Aprimorado, três
estudos envolvem o uso de Retorno Estimado e 15 estudos consideram Retorno
Diário/Semanal. Os remanescentes envolvem retorno \emph{direto}, dos quais 23
exploram retorno agregado e seis estudos onde é fornecido a informação em tempo
real no nível dos utensílios.

%Realiza-se nela uma divisão quanto à data em que os estudos foram realizados,
%utilizando o termo \gls{ece} para estudos realizados entre 1974 e 1994 
%e \gls{emc} nos anos conseguintes até 2009. 
%Essa divisão é importante pois os novos estudos provêm de
%tecnologias mais recentes, em especial dispõe de novas tecnologias de \gls{ict}.
%Essas oferecem meios inovativos de aumentar o efeito de
%mecanismos de retorno de informação, assim como reduziram os custos associados
%com fornecer retorno frequente e confiável para consumidores residenciais.
%Utilizou-se dois terços dos estudos para a era mais recente, \gls{emc}. 

%Com o objetivo de explorar melhor as diferenças de resultados, 
%foram realizadas mais divisões qualitativas: 
%quanto ao tamanho do estudo, referindo-se à quantidade de domicílios 
%envolvidos, no qual considera o estudo como pequeno
%(32\%) quando utilizando menos de 100 participantes, e grande caso 
%contrário; quanto a duração do estudo, considerada curta (40\%) para períodos
%inferiores à seis meses, e longa caso contrário; e elementos motivacionais
%utilizados além de fatores economicos ou apelo ao meio ambiente, como atribuir
%metas, competições e engajamento, e normas sociais.

Um resumo dos resultados obtidos para os estudos realizados entre 1995 e 
2010 (cerca de dois terços dos revisados) estão na 
Figura~\ref{fig:potencial_consumo_retorno} para os tipos de retorno 
detalhados na Subsessão anterior. A redução do consumo mostrada leva em conta os
resultados globais, ou seja, considerando a taxa de adesão da população aos
programas de \gls{ee}, considerando que os mesmos terão participação voluntária. 
Identifica-se nos resultados que os tipos de retorno são tanto incrementais
em custo e complexidade, quanto nos resultados de economia energética. Assim, 
é natural a implementação do sistema de retorno ser realizada de 
maneira continua, aplicando sistemas já disponíveis enquanto se realiza 
investimento em tecnologia para o desenvolvimento do
próximo nível de informação. Os desenvolvedores devem
manter o sistema o mais flexível possível, sendo desenvolvidos 
sempre preparados para a mudança e considerando o surgimento de novos 
mecanismos de retorno com o avanço da tecnologia 
\cite{aceee_2010_estudos_feedback}. Por exemplo, por ora é possível obter
econômia de energia utilizando um sistema de baixo custo, como o Faturamento
Aprimorado, que informa melhor o consumidor, ou até mesmo, com mais ambição, 
fornecer o Retorno Diário/Semanal.

\begin{figure}[h!t]
\centering
\includegraphics[width=\textwidth]{imagens/estudo_economia_aceee.pdf}
\caption[O potencial de consumo para cada tipo de retorno]
{O potencial de economia de consumo para cada tipo de retorno. 
Adaptado de \cite{aceee_2010_estudos_feedback}.}
\label{fig:potencial_consumo_retorno}
\end{figure}


\subsection{Indo Além dos Resultados}
\label{ssec:ret_outros}

Ademais, outras considerações devem ser levadas quando no desenvolvimento 
de programas de \gls{ee} \cite{aceee_2010_estudos_feedback,2006_darby,2009_epri}:

\begin{itemize}
\item \textbf{Retorno Indireto versus Direito}: 
Os retornos indiretos são mais adequados que diretos 
para demonstrar o efeito de mudanças condicionamento do ambiente, 
composição domiciliar e o impacto de investimentos 
em medidas de eficiência ou utensílios de alto consumo. Já o retorno instantaneo 
se adequa, geralmente, no fornecimento do impacto do consumo de aparelhos com 
usos de energia menores;
%\item \textbf{Era do Programa/Estudo}: Estudos anteriores a 1995 (não utilizados
%para gerar os resultados apresentados na 
%Figura~\ref{fig:potencial_consumo_retorno}) 
%apresentam economia de energia maiores aos posteriores. 
%Assim, recomenda-se a sua não utilização com o objetivo de evitar espectativas 
%infladas sobre o potencial de economia atualmente;
\item \textbf{Participação Voluntária}: Programas nos quais os usuários tem de
optar por não participar (\emph{opt-out}) tiveram adesão significamente 
maior (75\%-85\%) do que aqueles nos quais os usuários escolhem em colaborar
(\emph{opt-in}, participações menores a 10\%), sendo assim recomendada essa 
abordagem para maximizar a participação dos consumidores;
\item \textbf{Elementos Motivacionais}: A utilização de outros elementos para
motivar a população aquém do financeiro e apelo ao meio ambiente mostram-se 
importantes para aumentar a eficiência dos programas de \gls{ee}. São citados
como exemplo criar metas, compromissos, competições e normais sociais 
(tanto descritivas quanto injutivas). A Subsessão~\ref{ssec:asp_psic} 
irá tratar do tema com mais detalhes;
\item \textbf{Contexto Regional}: Diferenças regionais e culturais afetam os 
resultados. Os resultados para a Europa Ocidental superam os obtidos nos
\gls{eua}, podendo possivelmente ser atribuidos as diferenças em como o 
discurso sobre as mudanças climáticas pelas lideranças políticas nas duas regiões 
é feito e assim a preocupação ao tema da população. Nesse caso, chama-se atenção
novamente aos antecedentes no intuíto de preparar a população para os programas e
maximizar os resultados. Outro aspecto importante é a necessidade de estudos
sobre o tema afim de especificar como o brasileiro irá reagir em tais programas;
\item \textbf{Duração do Estudo e Persitência dos Resultados}: Quando os 
estudos são de menor duração ($< 6$ meses) se obtém resultados mais 
eficientes (média de 10,1\% de economia) que estudos mais longos (7,7\%),
discrepância essa atribuida a inaptidão de estudos curtos em observar variações
sazionais na utilização de energia. Alguns estudos indicam que se faz necessário 
a presença do retorno em longo termo para que os resultados persistam, 
enquanto outros apontam a necessidade do retorno continuamente, enfatizando
assim a necessidade na extensão dos programas de \gls{ee};
\item \textbf{Tamanho do Estudo}: Estudos com grandes ($> 100$) amostragens
domiciliares tendem a ter resultados mais modestos. Como esses estudos tem uma
representatividade melhor das residenciais, isso indica que programas de
\gls{ee} em larga escala também devem apresentar resultados mais modestos que
aqueles apresentados na Subsessão~\ref{ssec:ret_eff}.
Ainda, esses estudos mostram-se menos suscetíveis às oscilações 
quanto a duração dos estudos;
\item \textbf{Resposta de Ponta e Demanda versus Economia Fora de Ponta}:
Reduções de pico e demanda são de particular interesse das concessionárias que
buscam atender essencialmente o mesmo nível de serviços mas com custos totais
menores. Há dois modos de obter tal efeito: com uma melhoria em \gls{ee} ou
através do deslocamento de parte do consumo no horário de ponta para fora da
ponta. O interesse em resposta de demanda, ou seja, em reduzir o
consumo durante os horários de ponta difere dos programas de \gls{ee} que focam
em ter reduções eficientes economicamente durante todo o ano. Ainda que não seja
deprezível, programas de resposta de demanda apresentam economia de energia
bastante baixos quando em comparação aos de \gls{ee}. Os consumidores
normalmente não percebem a diferença entre os dois programas, do mesmo modo 
que a integração dos programas é plausível e sinergética, 
onde estudam mostram que a junção causa melhores resultados tanto 
em econômia de energia quanto na 
redução de picos, por isso sendo interessantes tanto do lado do consumidor 
quanto para a consessionária. Desta forma, a abordagem ótima ao tema deveria 
ser conseguir todos os meios economicamente atraentes de reduzir o desperdício 
e ineficiências antes de procurar oportunidades restantes de reduzir cargas 
durante os picos;
\item \textbf{Hábitos, Escolhas e Estilos de Vida}: Dentre os tipos de
comportamentos de \gls{ee} e conservação os que aparecem mais frequentemente são 
investimentos em novos equipamentos e utensílios em populações mais ricas, sendo
geralmente empreendido em conjunto com mudança de residência ou melhoria no
estilo (referido em oposição a funcional) do domícilio. Para a maioria da
população, os domícilios obtém melhor \gls{ee} através da mudança de hábitos e 
rotina ou pela avaliação dos comportamentos relacionados a energia. Esses
comportamentos de \gls{ee} são motivados assim por uma variedade de fatores,
incluindo interesse próprio (financeiro) e outros motivos altruístas e
preocupações cívicas. Desta forma, programas de \gls{ee} que procuram apenas 
a instalação de equipamentos mais novos e eficientes irá
desperdiçar o potencial relacionado à mudança comportamental, assim como
programas que apelam apenas para o interesse financeiro não irão influenciar um
largo grupo de fatores que motivam as pessoas para agir;
\item \textbf{Segmentação Populacional}: Poucas pesquisas exploraram como o
potencial de redução de consumo é afetado pelas diferentes classes sociais.
Desses estudos, as descobertas sugerem grandes níves de economia tendem a estar
associados a alto nível educacional e renda, grandes residenciais e dentre elas
as com maior número de pessoas, consumidores jovens e/ou com grande tendência a
valores ambientais. Essas considerações indicam que os potenciais apontados
provavelmente estão inflados para a aplicação no Brasil como um todo.
\end{itemize} 

\subsection{Tecnologias e Tendências}
\label{ssec:ret_tec}

Como constatado, os medidores atualmente utilizadas para medição pelas 
concessionárias, os medidores analógicos e eletrônicos, permitem fornecer um 
retorno com baixo custo mas com um potencial melhor de economia de energia, 
o Faturamento Aprimorado. As contas de energia de companias como a Light,
Ampla, Cemig e Eletropaulo fornecem o histórico de consumo dos últimos 12 meses,
uma informação que pode auxiliar o consumidor, já podendo ser consideradas um 
Faturamento Aprimorado. Entretanto outras informações podem ser utilizadas, 
como referencias do consumo acumulado e, em especial, comparações do consumo 
com o de vizinhos ou grupo pertencente. Também é possível estimar o uso 
energético por uso-final utilizando os valores médios de consumo das residenciais
para cada uso no sentido de auxiliar o cliente. A ideia é
transformar a conta de energia em uma espécie de relatório do consumo energético,
com um visual mais atraente (ver Subsessão~\ref{ssec:asp_visuais}), 
contendo gráficos e informações no sentido de
atrair o consumidor a se preocupar com o tema, entretanto esse é apenas o
primeiro passo \cite{2009_epri}.

No entanto, o sistema elétrico atual está 
se tornando obsoleto para atender aos problemas de 
aumento de carga nos centros urbanos devido ao crescimento do setor 
de serviços e do consumo das residencias. Há uma presença cada vez maior
de cargas eletrônicas injetando harmônicos e a geração centralizada exige 
excessivamente da capacidade de transmissão e distribuição, 
sobrecarregando as linhas nesses grandes centros que nem sempre podem 
corresponder à necessidade de novas linhas. A falta de informações sobre o 
estado do sistema dificultam a operação e planejamento de uma rede cada vez mais 
sobrecarregada. As \gls{ict} revolucionaram as redes de telecomunicações e 
serão a tendência para a criação das redes elétricas inteligentes (\emph{smart 
grids}), o novo sistema elétrico que tem como objetivo responder à essas
dificuldades. Ainda não foram definidas todas as características desse sistema,
no entanto as principais características são o uso de comunicações em tempo real
para o controle e informação, o uso massivo de sensores e medidores para
monitoramento do sistema, faturamento com preços para o momento de uso, 
gestão pelo lado da demanda, a integração de 
componentes avançados como linhas de transmissão supercondutoras, armazenamento, 
eletrônica de potência, geração distribuida etc. 
\cite{dissert_caires,aceee_2010_estudos_feedback}

Os aparelhos eletrônicos de medição utilizados nas redes inteligentes, referidos
neste trabalho como medidores inteligentes (\emph{advanced/smart metering}), irão 
fornecer uma gama maior de informações em tempo-real para as concessionárias,
melhorando a operação e planejamento. Ao mesmo tempo, será possível a
concessionária se comunicar com o cliente, oferecendo incentivos (como
descontos) para reduções de carga durante os horários de ponta, 
outros planos de tarifação com preços dinâmicos de acordo com os horários,
aumentando a interação da concessionária com o cliente. 

Por outro lado, pelo ponto de vista da demanda (ou dos consumidores) essa 
informação também estará disponível, trazendo uma gama de novas oportunidades 
para os usuários participarem ativamente. Mais especificamente, na abordagem do
tema atual, os medidores inteligentes oferecem uma base
a ser explorada para fornecer o retorno em larga escala para os consumidores, 
tanto o direto quanto indireto. Nos medidores utilizados nos \gls{eua} foram 
apontadas algumas dificuldades
técnicas para esse fornecimento, sendo elas: a necessidade de grande quantidade
de energia para enviar um sinal frequente ao consumidor e o sinal ser enviado em
intervalos de 7 s. Com um custo adicional, um estudo na industria mostrou que
é possível de os medidores terem seu \emph{hardware} substituídos 
no futuro para que possam fornecer medições de pequena energia e \emph{chips} 
de comunicação para habilitar dados de utensílios específicos, assim como 
automação para grandes cargas, como unidades de condicionamento ambiental, 
bombas etc. Desta forma, sendo possível 
fornecer tanto a tecnologia para o Retorno em Tempo Real com a utilização de 
mostradores dentro do domicílio como o Retorno em Tempo Real Desagregado
\cite{aceee_2010_estudos_feedback}.

Algumas empresas se estabeleceram no novo mercado para informar o consumidor
sobre o seu uso de energia e em auxiliá-lo nas atitudes para reduzir o seu
consumo, antes mesmo que estivessem disponíveis os medidores inteligentes. 
Elas fornecem o retorno indireto em alguns países
desenvolvidos, dentre eles o \gls{eua}, Australia, Nova Zelândia, Reino Unido.
Dentre essas empresas, faz-se referência a \emph{Positive Energy} 
\cite{opower_site} e \emph{C3 Energy} \cite{c3_site} que disponibilizam seus 
serviços, organizados na Tabela~\ref{tab:servicos_ret_ind}, 
utilizando os dados da concessionária, tanto
quando presentes os medidores convencionais ou inteligêntes. É importante notar
que as abordagens utilizadas por essas empresas utilizarão analises mais
complicadas conforme a presença de dados mais detalhados, frequentes e
desagregados estejam disponíveis.

\begin{table}[h!t]
\resizebox{\textwidth}{!}{
\begin{tabular}{m{2.5cm}m{5cm}m{8cm}}
\hline \hline 
\textbf{Empresa} & \textbf{Tecnologia de Retorno} & 
\textbf{Principios Comportamentais} \\
\hline \hline
\textbf{Positive Energy} & 
Dependendo na concessionária envia correspondencias mensais ou trimestral
e/ou fornecem um portal na internet com novas redes sociais &
\emph{Tipo de Retorno}: Retorno indireto incluindo informação sobre o domicílio
e conselhos, auditorias de energia através do uso da \emph{web}, análise de 
faturamento, consumo estimado por aparelho, \gls{co2}, k\acrshort{wh} e \$.

\emph{Principios Comportamentais}: Comparações sociais, metas, comparações
pessoais e plano de ações. \\
\hline
\textbf{C3 Energy} & 
Portal de comunidade social com retorno de consumo de energia e água & 
\emph{Tipo de Retorno}: Retorno indireto incluindo informação sobre o domicílio
e conselhos, auditorias de energia através do uso da \emph{web}, análise de 
faturamento, consumo estimado por aparelho, \gls{co2}, k\acrshort{wh}, \$ e
outras unidades.

\emph{Principios Comportamentais}: Comparações sociais, metas, competições
redes sociais, comparações pessoais e plano de ações. \\
\hline \hline
\end{tabular}
}
\caption[Empresas utilizando informação da concessionária e as 
oportunidades e insentivo de economia de energia oferecidas.]
{Empresas utilizando informação da concessionária e as 
oportunidades e insentivo de economia de energia oferecidas. Adaptado e 
atualizado de \cite{aceee_2010_estudos_feedback}.}
\label{tab:servicos_ret_ind}
\end{table}

Já o retorno direto pode ser encontrado através de
mostradores de energia no domicílio. A Tabela~\ref{tab:servicos_ret_dir}
identifica alguns dos mostradores oferecidos atualmente e suas propriedades.
Muitas vezes as companias oferecem também análises e estimativas do consumo 
especifico de aparelhos, comparações sociais e outros principios para motivar os
consumidores a economizar energia. A informação de consumo de aparelhos
especifico ou é estimada ou realizada através de sensores nos aparelhos.

\begin{table}[h!t]
\resizebox{\textwidth}{!}{
\begin{tabular}{p{4cm}p{7cm}p{7cm}}
\hline \hline 
&
\multicolumn{1}{c}{\textbf{The Energy Detector TED} \cite{ted_site} }& 
\multicolumn{1}{c}{\textbf{Wattson}                 \cite{wattson_site}}\\
\hline \hline
\textbf{Descrição da \newline Tecnologia} & 
\emph{Software} de suporte, aplicativos para celular &
\emph{Software} de suporte com acesso a comunidades \\
\hline 
\textbf{Mecanismos de \newline Retorno} & 
Mostradores em tempo real de k\acrshort{watt}, \$/hr, \gls{co2}, consumo e gastos
diários, conta estimada em k\acrshort{wh} e \$, pico de consumo, voltagem
min/max e custo/demanda projetada &
Mostradores em tempo real aproximado do consumo em \acrshort{watt},
k\acrshort{watt}, conta estimada. Leituras entre 3 a 20 s. Brilha conforma o
consumo: azul para consumo baixo; roxo para médio; vermelho para alto. \\
\hline
{\multirow{5}{4cm}{\textbf{Principios Comportamentais}}} &
\multicolumn{2}{c}{\emph{Retorno de Informação:}}
\\
& & \\
& 
\multicolumn{2}{p{14cm}}{
Retorno direto incluindo conselhos, auditorias de energia baseadas na \emph{web},
análise do consumo, estimativa de consumo por utensílios, \gls{co2} e \$.
}
\\
& & \\
&
\multicolumn{2}{p{14cm}}{\emph{Motivações Oferecidas:}
\centering 
Comparações sociais, metas, comparações pessoais e etapas de ações.
}
\\
\hline \hline
& 
\multicolumn{1}{c}{\textbf{PowerCost Monitor} \cite{powercost_site}}& 
\multicolumn{1}{c}{\textbf{Efergy Elite}      \cite{efergy_site}}\\
\hline \hline
\textbf{Descrição da\newline Tecnologia} & 
\emph{Software} de suporte, aplicativos para celular &
\emph{Software} de suporte, aplicativos para celular \\
\hline
\textbf{Mecanismos de\newline Retorno} & 
Mostradores em tempo real aproximado do consumo em
k\acrshort{watt} e \$/hr, pico de consumo nas últimas 24 horas, contagem de
k/\acrshort{wh} (reiniciável), recurso para medição de aparelhos específicos. &
Mostradores em tempo real aproximado do consumo em k\acrshort{watt} e \$/hora
(leituras em 6, 12 ou 18 s), informação de consumo média por hora, semanal,
mensal. Alarmes para consumo alto. \\
\hline
{\multirow{5}{4cm}{\textbf{Principios Comportamentais}}} &
\multicolumn{2}{c}{\emph{Retorno de Informação:}} \\
& & \\
& 
Retorno direto incluindo conselhos, auditorias de energia baseadas na \emph{web},
análise do consumo, estimativa de consumo por utensílios, \gls{co2} e \$.  &
Retorno direto, análise de consumo, estimativa de consumo em \$.  \\
& & \\
&
\multicolumn{2}{p{14cm}}{\emph{Motivações Oferecidas:} 
\centering Metas e comparações pessoais}
\\
\hline \hline 
\end{tabular}
}
\caption[Especificações de mostradores domiciliares disponíveis.]{
Especificações de mostradores domiciliares disponíveis. Adaptado de 
\cite{aceee_2010_estudos_feedback}.}
\label{tab:servicos_ret_dir}
\end{table}

% TODO Atualizar a tabela

Uma outra maneira mais eficiente economicamente para fornecer o
Retorno em Tempo Real Desagregado é através do uso de \gls{nialm}. 
O \gls{nialm} (Capítulo~\ref{cap:nialm}) coloca o
peso da desagregação da informação no \emph{software}, reduzindo a necessidade
de investimento em sensores e \emph{hardware}, sendo assim um método
economicamente atraente para a implementação de programas de \gls{ee} fornecendo 
esse tipo de retorno. No Brasil, os esforços da \gls{aneel} em regulamentar as
bases para os novos medidores inteligentes exigem que o consumidor pode exigir
da concessionária acesso às medições de tensão e corrente de cada fase no art.
3$^o$ da Resolução Normativa N$^o$ 502 \cite{ren502}. Ainda não se especificou 
a taxa de amostragem na qual essas leituras serão disponiblizadas, no entanto,
pode-se aproveitar toda a infra-estrutura das medidas e comunição oferecida pelo
medidores inteligente no intuíto de maximizar o custo-benefício.  
Caso a amostragem seja baixa ou interessante para obter uma maior capacidade de
identificação dos aparelhos, o \gls{nialm} pode utilizar de um \emph{hardware} 
próprio de medição.

Percebe-se que ainda é incerto se os medidores inteligentes são a melhor 
alternativa de fornecer retorno de informação, contudo parece natural sua 
utilização. Diversas tecnologias podem ser utilizadas, envolvendo, ou não, as 
concessionárias. Apenas com o desenvolvimento delas será
possível determinar as limitações, custos e vantagens para definir o que é 
economicamente mais atraente.

Finalmente, uma outra tendência é o uso de automação da rede doméstica. A 
automação, além de melhorar a qualidade de vida dos consumidores, 
pode aumentar o potencial de consumo uma vez que terão a capacidade de 
administrar a sua demanda de energia sem grande esforço, 
facilitando a mudança de habitos e gerir o sistema de maneira mais econômica.

\subsection{Aspectos Psicológicos}
\label{ssec:asp_psic}

A tecnologia apenas concebe as possibilidades de informação a
serem repassadas ao consumidor, no entanto, a grande questão está em como
apresentar essa informação e motivar o usuário para a mudança. 
O conselho de profissionais nos campos de psicologia, sociologia,
\emph{marketing}, mudança e economia comportamental serão criticos para
motivar, habilitar e continuamente empreender consumidores na gestão de sistemas 
de energia residenciais \cite{aceee_2010_estudos_feedback}. 
Uma abordagem da psicologia motivacional foi realizada em
\cite{2010_aspectos_psicologicos_usa} com noções básicas para os desenvolvedores 
de tecnologia que visam motivar a mudança comportamental sustentável, mostrando
como elas podem ser aplicadas nas tecnologias de retorno de informação. O
objetivo é atingir as atitudes, crenças e valores individuais em cada estágio da
mudança comportamental. Na Dissertação de Mestrado será realizado um resumo com
mais detalhes sobre a abordagem nesta Subsessão.

Um exemplo prático de sucesso da aplicação de psicologia motivacional é a empresa
\emph{Positive Energy}.

% TODO Mudanças comportamentais, tipos de comportamentos

\subsection{Aspectos Visuais da Informação}
\label{ssec:asp_visuais}

A apresentação na qual a informação é realizada também irá afetar o entendimento
e engajamento do consumidor nos programas de \gls{ee}. Novamente, especialistas
e profissionais na área de \emph{design} da informação serão importantes nesse
intuíto, percebendo que o problema jaz além de apenas disponibilizar a
tecnologia para fornecer a informação: também será necessário tornar a
informação atrativa ao consumidor.

O processo do fluxo de informação ocorre entre o remetente e o receptor, onde
este analisa a informação necessaria e a apresenta na mensagem através de textos
e imagens, enquanto aquele realiza escolhas entre as informações disponíves nela
e opta se irá processá-las mentalmente. A visão é o sentimento mais 
importante para a compreensão e vivência humana do meio externo. 
Cerca de 70\% de nossas células sensoriais estão nos olhos, assim a visualizão 
é um meio bastante efetivo de realizar a comunicação da informação e de dados. 
Tornar a mensagem interessante visualmente depende de vários elementos 
\cite{2012_visualisation_sweden}. Alguns exemplos são: fornecer para a 
mensagem uma estrutura na qual serão guiados os princípios para
a inserção das representações; clareza para a simplicidade de compreensão e
legibilidade da informação; enfase para atrair, direcionar ou manter a atenção;
unidade da mensagem, com uma coerência e união global \cite{it_depends}.

Em \cite{2012_visualisation_sweden} apresentou-se a crianças de um colégio sueco
uma visualização do consumo de energia em um mostrador em tempo real na sala de 
aula. As crianças tiveram participação na criação da interface, onde desenharam
e escolheram uma imagem para informar se o consumo estava elevado, médio ou
baixo, assim como informaram o que entederam das informações nos mostradores.
Esse estudo revelou cinco aspectos importantes no desenvolvimento de sua
visualização nos mostradores:

\begin{enumerate}
\item Deve chamar a atenção dos usuários, realizando o uso de cores brilhantes,
contrastes e quando possível uma exibição dinâmica; 
\item Mostrar comparações entre o consumo de modo a deixar evidente aos usuários
resultados positivos de um esforço;
\item Fornecer o consumo em tempo real para estimular a mudança direta no
comportamento;
\item Deve conter um tom positivo e encorajador, potencializando a positividade
de um comportamento correto;
\item Ser explicativa, com pequenos textos instrutivos que fazem aos usuários
simples de entender o que eles estão vendo.
\end{enumerate}


%O \gls{nialm} pode ser utilizado diretamente na obtenção de 
%uma melhor \gls{ee}, visto que estudos no exterior
%\cite{2010_advanced_metering,liikkanen2009extreme,schleich2012does,darby2006effectiveness}
%indicam uma possível econômia de até 12\% no consumo residencial através do fornecimento de um
%retorno detalhado em tempo real ao usuário de sua utilização da eletricidade, 
%com informação do consumo por equipamento e recomendações no intuíto de reduzir
%o consumo de energia. Esses estudos mostram uma grande variação conforme a
%cultura e perfil social do consumidor, sendo necessário um estudo para
%determinar esse potencial no Brasil, entretanto, percebe-se que o maior
%potencial está nos consumidores pertencentes a classe média, que são os mais aptos a reduzirem seu
%consumo através do retorno de informação em detrimento aos mais pobres, que 
%raramente possuem mudanças que possam ser exploradas uma vez que seu consumo
%é vital e menor, e os mais ricos, que tem pouca sensibilidade aos gastos com
%energia elétrica.


