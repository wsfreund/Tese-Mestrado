\chapter{A Eficiência Energética no Setor Residencial}
\label{chap:ee_retorno}

Durante o estudo das técnicas aplicadas no \acs{nilm}, que serão
descritas no próximo capítulo, percebeu-se um apelo no exterior --- no
caso, em países desenvolvidos como os \acs{eua} e outros países da
Europa Ocidental --- para a capacidade de aplicação do \acs{nilm}
diretamente como uma ferramenta na obtenção de uma melhor \acs{ee}.
Estudos nesses países indicam uma capacidade de redução de consumo de eletricidade
acima de 10\% no setor residencial ao retornar a informação de
consumo em tempo real desagregada por equipamento. Na
Seção~\ref{sec:ee_setor_residencial}, será visto que no Brasil o
consumo nesse setor é também relevante, em especial quando levando em
conta apenas a eletricidade. Além de detalhar o perfil do consumo no
setor residencial, essa seção ainda se aprofundará na questão de
eficiência energética, em especial quanto a possibilidade de aplicação
do novo programa como um meio de expandir o potencial de \acs{ee} no
setor residencial.

Os estudos utilizados como referências para a capacidade de redução e
as considerações por eles levantadas são resumidos e comentados na
Seção~\ref{sec:ee_res_exp}. Uma tendência no exterior é aliar os
medidores inteligentes --- os medidores das redes elétricas
inteligentes --- ao \acs{nilm} para garantir o melhor
custo-benefício para o programa de \acs{ee}, garantindo o retorno da
informação desagregada por equipamento com baixo custo. Nota-se,
entretanto, que há limitações dessa tecnologia de fornecer essa
informação, que serão detalhadas no próximo capítulo. Para o programa
ser bem-sucedido, é necessário motivar o consumidor para reduzir seu
consumo e não apenas retornar a informação. Assim, além das
considerações feitas em torno da tecnologia necessária, chama-se
atenção, baseando-se em referências, aos aspectos psicológicos e
visuais da informação.

\section{Eletricidade e Potencial de Eficiência Energética
no Setor Residencial}
\label{sec:ee_setor_residencial}

A eletricidade é o segundo meio energético de maior participação na matriz
energética brasileira (Figura~\ref{fig:matriz_bra_2011}), representando 16,7\%
da demanda. Há uma tendência de crescimento dessa parcela na matriz
(Figura~\ref{fig:matriz_bra_evo}), que deverá progredir devido à fatores
como~\cite{iea_weo2010}:

\begin{figure}[h!t]
    \label{fig:eletricidade_brasil}
    \begin{center}
    \begin{subfigure}[c]{0.8\textwidth}
      \includegraphics[width=\textwidth]{imagens/matriz_energetica_brasileira_2011.pdf}
      \caption{}
      \label{fig:matriz_bra_2011}
    \end{subfigure}
    \hfill
    \begin{subfigure}[c]{0.8\textwidth}
      \includegraphics[width=\textwidth]{imagens/evolucao_matriz_energetica_brasil.pdf}
      \caption{}
      \label{fig:matriz_bra_evo}
    \end{subfigure}
  \end{center}
  \caption[Matriz energética brasileira.]{A matriz elétrica brasileira
em 2011 (a) e seu histórico (b). Baseado (a) e extraído (b) de \cite{ben2012}.}
\end{figure}

\begin{itemize}
\item substituição da biomassa para eletricidade como meio de iluminação
e aquecimento
no setor residencial;
\item nesse setor também há um aumento do número de eletrodomésticos como
reflexo do desenvolvimento e melhor distribuição da riqueza;
\item a expansão do setor comercial e de serviços utilizando uma quantidade
maiores de equipamentos elétricos (como ar condicionado, iluminação e equipamentos
de \acs{ti});
\item transformação do setor industrial, gradualmente substituindo o carvão e
aumentando a presença de dispositivos elétricos.
\end{itemize}

No que diz respeito ao consumo de eletricidade,
Figura~\ref{fig:eletricidade_por_setor},
o setor residencial possui uma posição de destaque no consumo, com uma demanda
de 111,97 T\acs{wh} e uma parcela equivalente à 26\% do total no ano de 2011,
exprimindo a importância desse setor para explorar a capacidade de economia de
eletricidade através de \gls{ee}.

\begin{figure}[h!t]
\centering
\includegraphics[width=.6\textwidth]{imagens/consumo_por_setor.pdf}
\caption[Consumo de eletricidade por setor em 2011.]
{O consumo de eletricidade por setor em 2011. Baseado em \cite[pp. 32]{ben2012}.}
\label{fig:eletricidade_por_setor}
\end{figure}

De acordo com o último estudo do \gls{beu}\footnote{O \gls{beu} é
realizado em intervalos decenais desde 1985.} com o ano base de 2004
\cite{beu}, o potencial de economia no consumo de eletricidade nesse
setor chega a 13,60 T\acs{wh}, ou 1,17 milhões de \acs{toe}.
Apenas como figura de mérito, quando considerando melhorias de
eficiência em todas as fontes energéticas, o setor residencial possui
o terceiro maior potencial de economia, após o setor industrial e de
transporte, com uma capacidade de economia de 2,97 milhões de
\acs{toe}. Por sua vez, ressalta-se que esses valores de
potenciais de economia apresentados são aproximados e reduzidos em
relação ao valor real, onde se considera apenas a perda de energia na
primeira transformação do processo produtivo. Também não são
consideradas possíveis alterações no consumo de fontes energéticas,
como a mencionada alteração de biomassa para eletricidade, uma fonte
menos poluente e mais eficiente. Finalmente, deve-se acrescentar que
esse potencial é calculado utilizando o rendimento de equipamentos no
estado da arte entre aqueles normalmente comercializados, e não os
possíveis de se alcançar quando considerando a literatura técnica. Os
dados  do \gls{beu} são normalmente utilizados para os estudos em
\gls{ee} para o setor industrial e comercial.

No entanto, os dados utilizados pela \gls{epe} para os estudos atuais
de \gls{ee} no setor residencial, como os \glspl{pde} e \gls{pne2030},
por sua vez, são das informações obtidas na última \gls{pph} realizada
no Brasil entre 2004--2006 que possibilitam o estudo baseado em uma
abordagem desagregada. Essa abordagem depende do número de domicílios,
da posse média e hábito de consumo específico dos equipamentos
eletrodomésticos --- que estão implicitamente indicados na curva de
carga ilustrada na Figura~\ref{fig:curva_carga} para a região sudeste
--- e do rendimento médio desses equipamentos no país, informação
baseada nas tabelas do \gls{pbe}, coordenado pelo \gls{inmetro}.
Também são utilizadas variáveis agregadas para o ajuste do modelo em
questão, sendo elas a relação entre o número de consumidores
residenciais e população (que permite a projeção do número de
consumidores a partir da projeção da população), e consumo médio por
consumidor residencial
\cite{epe_eficiencia_2012,pde_2020,pne30_eff_energ}.

% TODO Inserir a figura aqui de posse média
\begin{figure}[h!t]
\centering
\includegraphics[width=.9\textwidth]{imagens/curva_demanda_sudeste.pdf}
\caption[Curva de carga média para a região sudeste, ano base 2005.]{Curva de
carga média para a região sudeste, ano base 2005. Extraído de
\cite{result_procel_2005}.}
\label{fig:curva_carga}
\end{figure}

Esses estudos exploram a conservação de energia no ganho de eficiência
de equipamentos eletrodomésticos, mas não consideram melhorias possíveis
em mudanças nos hábitos de consumo. Isso se justifica uma vez que a abordagem
representa o ganho de conservação de energia através do progresso autônomo
ou tendencial, progresso esse ``que se dá por iniciativa do mercado, sem a
interferência de políticas públicas de forma espontânea, ou seja, através da
reposição natural do parque de equipamentos por similares novos e mais
eficientes ou tecnologias novas que produzem o mesmo serviço de forma mais
eficiente'' \cite[pp. 1]{pnef}, assim como por ``efeitos de programas e ações de
conservação já em execução no país'' \cite[pp. 247]{pde_2020}.

Por outro lado, o progresso induzido refere-se à ``instituição de programas e ações
adicionais orientados para determinados setores, refletindo políticas púbicas;
programas e mecanismos ainda não implantados no Brasil'' \cite[pp. 247]{pde_2020}.
As políticas de progresso induzido estão detalhadas no \gls{pnef}, e a
estratégia adotada no tocante às mudanças nos hábitos de consumo é através da
educação com os programas do \gls{procel} e \gls{conpet}, \textlabel{com as
linhas}{text:prog_cepel}: Eficiência Energética na Educação Básica; Eficiência
Energética na Formação Profissional; e Rede de Laboratórios e Centros de Pesquisa em Eficiência
Energética \cite[cap. 5]{pnef}. No caso do \gls{procel}, investiu-se um montante superior a
R\$~4,5 Mi em 2011 em projetos voltados para o desenvolvimento e
aperfeiçoamento dessas três linhas \cite{procel_resultados_2012}, trabalhando na
conscientização, sensibilização e informação para obter uma melhor \gls{ee},
atuando nos três níveis de educação.

Entretanto, estudos no exterior mostram um potencial ainda a ser explorado de
economia de energia elétrica no setor residencial, através do retorno de
informação da utilização de energia para o consumidor, uma estratégia de
progresso induzido que pode ser adicionada aos programas de \gls{ee} no Brasil.
Conforme será visto na próxima seção, esse potencial depende de aspectos, como a quantidade
de informação retornada ao consumidor, essa inerentemente ligada à quantidade de
investimento utilizada na tecnologia para permitir um maior retorno, mas ainda,
dependem do modo em que esse retorno é fornecido para o mesmo no intuito de
motivar ações sustentáveis de energia, sendo um problema complexo dependente de
aspectos sociais, culturais e psicológicos.

Por fim, outros setores também podem reduzir seu consumo com políticas de
\gls{ee} aplicadas para o setor residencial devido a sua natureza similar a esse
setor --- nesse caso, não se referindo somente à politica de \gls{ee} sugerida.
Os setores público e comercial, por exemplo, possuem prédios com natureza
de consumo correspondente ao do setor residencial, de forma que estratégias
desenvolvidas possam sinergeticamente apresentar um potencial maior de
economia de energia na matriz brasileira, em especial no que concerne estudos
para melhor eficiência de equipamentos.

\section{Expandindo o Potencial de Eficiência Energética através do Retorno de
Informação de Consumo}
\label{sec:ee_res_exp}

% Invisibilidade da eletricidade para os consumidores residencias
As novas fontes energéticas, dentre elas a eletricidade e gás natural,
que atendem a demanda dos consumidores residenciais para a vasta
variedade de serviços nas quais são utilizadas
--- desde cocção, condicionamento do ambiente, a lazer e
entretenimento ---, fluem invisível e silenciosamente para seus domicílios, sem
deixar qualquer traço notável de sua utilização além do efeito final desejado pelo
consumidor. Para eles, o único retorno de seu consumo é informado na conta
apresentada pela concessionária, fornecida em um longo período após o consumo
(mensalmente, por exemplo). As informações nas contas são precárias, não informando
muito além do total de energia consumido e o preço de energia.
Os usuários não tem como inferir quais são os meios de uso final que demandam
maior energia, nem a que ponto possíveis mudanças podem afetar sua demanda,
seja através da mudança de seus hábitos ou na escolha
de equipamentos mais eficientes. Atualmente, os usuários estão cegos quanto a essas
mudanças, não é possível visualizar a energia que consomem. Além disso, as
informações fornecidas não permitem o consumidor comparar seu consumo com o de
outros, de modo que ele não é capaz de criar uma referência social para seu consumo.
Sem uma referência, o consumidor tem dificuldades para determinar se o consumo é
excessivo ou moderado e se é necessário algum tipo de intervenção
\cite{aceee_2010_estudos_feedback}.

% Modos de alterar o hábito de consumo
Estratégias para intervir no comportamento podem ser classificadas de dois modos
\cite{aceee_2010_estudos_feedback,2009_epri}:
\begin{enumerate}
\item \textbf{Antecedentes}, que envolvem esforços para influenciar o que define
um comportamento antes de sua realização;
\item \textbf{De consequência}, que buscam alterar o que determina o
comportamento após a sua ocorrência.
\end{enumerate}

Exemplos de estratégias antecedentes são campanhas de informação
com o objetivo de aumentar o conhecimento público sobre o impacto de suas
escolhas e das opções para economia de energia disponíveis --- como
as já citadas ações do \gls{procel} e \gls{conpet} (ver pp.~\pageref{text:prog_cepel}) ---,
engajar o indivíduo com um compromisso de mudança, criar metas de mudança
comportamentais, ou modelar e demonstrar o comportamento desejado. Já para
estratégias de consequência se pode citar recompensas, punições ou o
retorno de informação \cite{aceee_2010_estudos_feedback,2009_epri}.

Iniciativas utilizando o retorno de informação mostraram-se altamente eficientes
em mudanças comportamentais com relação ao consumo energético \cite{
aceee_2010_estudos_feedback,2009_epri,2012_schleich__austria,
2011_zhifeng_smart_energy_savings,2006_darby,2009_nber_studies_us,
ucla_studies_1975_2011_usa,2010_nilm_melhorando_pph_usa_37}. O uso do
retorno de informação baseia-se em que tanto resultados positivos ou
negativos podem modelar o comportamento. Resultados atribuídos como
positivos os tornam comportamentos mais atraentes, enquanto a
atribuição de resultados negativos fazem de comportamentos ruins
menos desejáveis. Sempre que possível, a atribuição negativa deve ser
evitada, pois ela tende a reduzir a motivação e não coloca nada no
lugar do comportamento evitado \cite{2010_aspectos_psicologicos_usa}.

A questão, por outro lado, não é apenas fornecer retorno do consumo ao usuário
final --- a própria conta de energia pode ser encarada como um meio de retorno
---, mas como o retorno pode ser utilizado para efetivamente motivar pessoas
para reduzir o seu consumo. Algumas considerações devem ser tomadas: primeiro,
quais são os tipos de retornos disponíveis (Subseção \ref{ssec:ret_tipos}) e,
dentre eles, quais tem mostrado resultados mais eficientes na redução do
consumo energético (Subseção \ref{ssec:ret_eff})? O que mais deve ser levado
em consideração quando preparando tais programas e estudos de \gls{ee}
(Subseção~\ref{ssec:ret_outros})? Quais são as tecnologias disponíveis para
fornecer essa informação e suas tendências (Subseção \ref{ssec:ret_tec})?
Ainda, pessoas possuem diferentes atitudes, crenças e valores, sendo motivadas
de modos distintos. Uma breve consideração sobre a perspectiva psicológica que
envolve mudança comportamental será realizada (Subseção \ref{ssec:asp_psic})
uma vez que esse aspecto é de principal relevância para o sucesso dos
programas.  Do mesmo modo, a apresentação visual da informação também
influenciará no êxito, e por isso o tema também será colocado em pauta
(Subseção~\ref{ssec:asp_visuais}). De nenhuma maneira as breves considerações
realizadas nas subseções sobre os aspectos psicológicos e visuais devem
substituir a análise de profissionais dessas áreas, servindo apenas para chamar
atenção para a importância, bem como introduzir os leitores, aos temas.

\subsection{Tipos de Retorno}
\label{ssec:ret_tipos}

A categorização dos tipos de retorno que será apresentada se iniciou em
\cite{2000_darby} e depois foi aprimorada por \cite{2009_epri}. A primeira divisão
toma em conta o modo no qual o retorno é fornecido, sendo possíveis o retorno
direto ou indireto. O termo \emph{indireto} é
utilizado quando há alguma espécie de processamento antes de atingir o
consumidor, enquanto \emph{direto} determina o retorno
instantaneamente\footnote{O termo instantâneo é empregado ao considerar que o
atraso entre o retorno da informação e o consumo, normalmente na ordem de
segundos, é desprezível para os fins.} entregue
ao usuário. Em seguida, realiza-se uma divisão em termos de frequência,
diferenciando quatro tipos de retorno \emph{indiretos}, e dois retornos
\emph{diretos}. A divisão é apresentada resumidamente em ordem crescente em
termos de custo e
quantidade de informação disponível, onde os dois retornos \emph{diretos} estão no
final da lista \cite{aceee_2010_estudos_feedback,2009_epri}:

\begin{enumerate}
\item \textbf{Retorno por Faturamento Simples}: conta de energia contendo
k\acs{wh}
consumido, o preço da tarifa unitária ($\text{R\$}/$k\acs{wh}), o custo
total e outros possíveis ônus. Nessa forma de retorno, normalmente se carece
estatísticas comparativas ou qualquer informação detalhada sobre os aspectos
temporais do consumo;
\item \textbf{Retorno por Faturamento Aprimorado}: fornece informações mais
detalhadas
sobre o padrão de consumo de energia, incluindo em alguns casos estatísticas
comparativas, tanto comparando o maior consumo do mês atual e sua despesa
aliados ao consumo histórico e/ou a comparação com outros domicílios
pertencentes ao grupo do consumidor;
\item \textbf{Retorno Estimado}: essa abordagem utiliza geralmente técnicas
estatísticas para desagregar o total de energia baseado no tipo do
domicílio do consumidor, informação de equipamentos e dados de faturamento. O
retorno resultante fornece um relato detalhado do uso de eletricidade pelos
equipamentos e dispositivos de maior importância. A forma mais comum é através de
ferramentas de auditoria de energia residencial baseadas na internet, oferecida
por um fornecedor de serviços a seus consumidores;
\item \textbf{Retorno Diário/Semanal}: esses relatórios utilizam a média de
dados e frequentemente incluem estudos de leitura dos medidores pelos próprios
consumidores, assim como estudos nos quais indivíduos são providos com
relatórios mensais ou semanais do fornecedor de serviços ou entidade de
pesquisa;
\item \textbf{Retorno em Tempo-Real}: fornecido por dispositivos que exibem o
consumo (praticamente) em tempo-real e informações de custo da energia em nível
agregado domiciliar;
\item \textbf{Retorno em Tempo-Real Desagregado}: nesse caso, as informações são
exibidas desagregadas ao nível dos equipamentos.
\end{enumerate}

\subsection{Resultados por Tipo de Retorno}
\label{ssec:ret_eff}

No exterior, há uma farta quantidade de estudos envolvendo o tema, datando a
partir da década de 1970 como resposta à Crise do Petróleo e que tiveram um
declínio durante a década posterior. O interesse retornou a pauta recentemente
devido à crescente preocupação com o meio ambiente e mudanças climáticas, assim
como o aparecimento de novas possibilidades tecnológicas associadas a
\gls{ict}~\cite{aceee_2010_estudos_feedback}. É possível encontrar pesquisas nas
quais se compilam diversos estudos de retorno de informação para consumidores
residenciais no intuito de generalizar
resultados~\cite{aceee_2010_estudos_feedback,2011_zhifeng_smart_energy_savings,
2006_darby,2009_nber_studies_us,ucla_studies_1975_2011_usa}.
Infelizmente, o mesmo não pode ser dito para o Brasil, onde não se
encontrou pesquisas nesse sentido.

Por isso, este trabalho guiar-se-á
na pesquisa que mais se destacou com resultado de estudos no exterior
\cite{aceee_2010_estudos_feedback}. Nela se revisou 57 estudos
primários de retorno, realizados em países desenvolvidos incluindo
\gls{eua} (58\% dos estudos), quatro países da Europa Ocidental
(Países Baixos, Finlândia, Dinamarca e Reino Unido, com um total de
22\%), Canada (15\%), Japão (5\%) e Austrália (1 estudo). Em termos de
retorno, metade dos estudos envolvem retorno \emph{indireto}, dentre
os quais 11 envolvem Faturamento Aprimorado, três estudos envolvem o
uso de Retorno Estimado e 15 estudos consideram Retorno
Diário/Semanal. Os remanescentes envolvem retorno \emph{direto}, dos
quais 23 exploram retorno agregado e outros seis estudos onde são
fornecidos a informação em tempo real no nível dos equipamentos.
Salienta-se que pesquisas nesse sentido no Brasil são necessárias para
validar os resultados, tomando em posse as diferenças culturais,
sociais e econômicas quanto a nossa realidade.

%Realiza-se nela uma divisão quanto à data em que os estudos foram realizados,
%utilizando o termo \gls{ece} para estudos realizados entre 1974 e 1994
%e \gls{emc} nos anos conseguintes até 2009.
%Essa divisão é importante pois os novos estudos provêm de
%tecnologias mais recentes, em especial dispõe de novas tecnologias de \gls{ict}.
%Essas oferecem meios inovativos de aumentar o efeito de
%mecanismos de retorno de informação, assim como reduziram os custos associados
%com fornecer retorno frequente e confiável para consumidores residenciais.
%Utilizou-se dois terços dos estudos para a era mais recente, \gls{emc}.

%Com o objetivo de explorar melhor as diferenças de resultados,
%foram realizadas mais divisões qualitativas:
%quanto ao tamanho do estudo, referindo-se à quantidade de domicílios
%envolvidos, no qual considera o estudo como pequeno
%(32\%) quando utilizando menos de 100 participantes, e grande caso
%contrário; quanto a duração do estudo, considerada curta (40\%) para períodos
%inferiores à seis meses, e longa caso contrário; e elementos motivacionais
%utilizados além de fatores economicos ou apelo ao meio ambiente, como atribuir
%metas, competições e engajamento, e normas sociais.

Um resumo dos resultados obtidos nessa referência, para os estudos realizados entre 1995 e
2010 (cerca de dois terços dos revisados), estão na Figura~\ref{fig:potencial_consumo_retorno}.
A redução do consumo mostrada leva em conta os
resultados globais, ou seja, considerando a taxa esperada de adesão da população aos
programas de \gls{ee}, supondo que os mesmos terão participação voluntária.

Identifica-se nos resultados que os tipos de retorno são tanto incrementais
em custo e complexidade, quanto nos resultados de economia energética. Assim,
é natural a implementação do sistema de retorno ser realizada de
maneira continua, aplicando sistemas já disponíveis enquanto se realiza
investimento em tecnologia para o desenvolvimento do
próximo nível de informação. Os desenvolvedores devem
manter o sistema o mais flexível possível, sendo desenvolvidos
sempre preparados para a mudança e considerando o surgimento de novos
mecanismos de retorno com o avanço da tecnologia
\cite{aceee_2010_estudos_feedback}. Por exemplo, por ora é possível obter
economia de energia utilizando um sistema de baixo custo, como o Faturamento
Aprimora, que informa melhor o consumidor, ou até mesmo, com mais ambição,
fornecer o Retorno Diário/Semanal.

\begin{figure}[h!t]
\centering
\includegraphics[width=\textwidth]{imagens/estudo_economia_aceee.pdf}
\caption[O potencial de consumo para cada tipo de retorno]
{O potencial de economia de consumo para cada tipo de retorno. Estudos
no exterior e em países desenvolvidos. Adaptado de
\cite{aceee_2010_estudos_feedback}.}
\label{fig:potencial_consumo_retorno}
\end{figure}


\subsection{Indo Além dos Resultados}
\label{ssec:ret_outros}

Ademais, outras considerações devem ser levadas quando no desenvolvimento
de programas ou estudos de \gls{ee} através do retorno do consumo de energia
\cite{aceee_2010_estudos_feedback,2006_darby,2009_epri}:

\begin{itemize}
\item \textbf{Retorno Indireto versus Direto}:
Os retornos indiretos são mais adequados que diretos
para demonstrar o efeito de mudanças no condicionamento do ambiente,
composição domiciliar e o impacto de investimentos
em medidas de eficiência ou equipamentos de alto consumo. Já o retorno instantâneo
se adéqua, geralmente, no fornecimento do impacto do consumo de equipamentos com
usos de energia menores;
%\item \textbf{Era do Programa/Estudo}: Estudos anteriores a 1995 (não utilizados
%para gerar os resultados apresentados na
%Figura~\ref{fig:potencial_consumo_retorno})
%apresentam economia de energia maiores aos posteriores.
%Assim, recomenda-se a sua não utilização com o objetivo de evitar espectativas
%infladas sobre o potencial de economia atualmente;
\item \textbf{Participação Voluntária}: Programas nos quais os usuários tem de
optar por não participar (\emph{opt-out}) tiveram adesão significadamente
maior (75\%-85\%) do que aqueles nos quais os usuários escolhem em colaborar
(\emph{opt-in}, participações menores a 10\%), sendo assim recomendada
a primeira abordagem para maximizar a participação dos consumidores;
\item \textbf{Elementos Motivacionais}: A utilização de outros elementos para
motivar a população aquém do financeiro e apelo ao meio ambiente mostram-se
importantes para aumentar a eficiência dos programas de \gls{ee}. São citados
como exemplo criar metas, compromissos, competições e normas sociais
(tanto descritivas quanto injuntivas). A Subseção~\ref{ssec:asp_psic}
tratará do tema com mais detalhes;
\item \textbf{Contexto Regional}: Diferenças regionais e culturais afetam os
resultados. Os resultados para a Europa Ocidental superam os obtidos nos
\gls{eua}, podendo ser atribuídos as diferenças em como o
discurso sobre as mudanças climáticas pelas lideranças políticas nas duas regiões
é feito e assim a preocupação com o tema pela população. Nesse caso, chama-se atenção
novamente às estratégias antecedentes no intuito de preparar a população para os programas e
maximizar os resultados. Outro aspecto importante é a necessidade de estudos
sobre o tema a fim de especificar como o brasileiro reagirá em tais programas;
\item \textbf{Duração do Estudo e Persistência dos Resultados}: Quando os
estudos são de menor duração ($< 6$ meses) se obtém resultados mais
eficientes (média de 10,1\% de economia) que estudos mais longos (7,7\%),
discrepância essa atribuída a inaptidão de estudos curtos em observar variações
sazonais na utilização de energia. Alguns estudos indicam que se faz necessário
a presença do retorno em longo termo para que os resultados persistam,
enquanto outros apontam a necessidade do retorno continuamente, enfatizando
assim a necessidade na extensão dos programas de \gls{ee};
\item \textbf{Tamanho do Estudo}: Estudos com grandes ($> 100$) amostragens
domiciliares tendem a ter resultados mais modestos. Como esses estudos têm uma
representatividade melhor das residencias, isso indica que programas de
\gls{ee} em larga escala também devem apresentar resultados mais modestos que
aqueles apresentados na Subseção~\ref{ssec:ret_eff}.
Ainda, esses estudos mostram-se menos suscetíveis às oscilações
quanto a duração dos estudos;
\item \textbf{Resposta de Ponta e Demanda versus Economia Fora de
Ponta}: Reduções de pico e demanda são de particular interesse das
concessionárias que buscam atender essencialmente o mesmo nível de
serviços mas com custos totais menores. Há dois modos de obter tal
efeito: com uma melhoria em \gls{ee} ou através do deslocamento de
parte do consumo no horário de ponta para fora da ponta. O interesse
em resposta de demanda, ou seja, em reduzir o consumo durante os
horários de ponta difere dos programas de \gls{ee} que focam em ter
reduções eficientes economicamente durante todo o ano. Ainda que não
seja desprezível, programas de resposta de demanda apresentam economia
de energia bastante baixos quando em comparação aos de \gls{ee}. Além
disso, esses programas também são importantes para a questão
ambiental, considerando que essa energia normalmente será proveniente
de fontes fosseis. Por outro lado, os consumidores normalmente não
percebem a diferença entre os dois programas, do mesmo modo que a
integração dos programas é plausível e sinergética, onde estudos
mostram que a junção causa melhores resultados tanto em economia de
energia quanto na redução de picos, por isso, sendo interessantes
tanto do lado do consumidor quanto para a concessionária. Desta forma,
a abordagem ótima ao tema deve ser conseguir todos os meios
economicamente atraentes de reduzir o desperdício e ineficiências
antes de procurar oportunidades restantes de reduzir cargas durante os
picos;
\item \textbf{Hábitos, Escolhas e Estilos de Vida}: Dentre os tipos de
comportamentos de \gls{ee} e conservação, os que aparecem mais frequentemente são
investimentos em novos equipamentos e equipamentos em populações mais ricas, sendo
geralmente empreendido em conjunto com mudança de residência ou melhoria no
estilo (referido em oposição a funcional) do domicilio. Para a maioria da
população, os domicílios obtém melhor \gls{ee} através da mudança de hábitos e
rotina, ou pela avaliação dos comportamentos relacionados a energia. Esses
comportamentos de \gls{ee} são motivados assim por uma variedade de fatores,
incluindo interesse próprio (financeiro) e outros motivos altruístas e
preocupações cívicas. Desta forma, programas de \gls{ee} que procuram apenas
a instalação de equipamentos mais novos e eficientes
desperdiçam o potencial relacionado à mudança comportamental, assim como
programas que apelam apenas para o interesse financeiro não influenciam em um
largo grupo de fatores que motivam as pessoas para agir;
\item \textbf{Segmentação Populacional}: Poucas pesquisas exploraram como o
potencial de redução de consumo é afetado pelas diferentes classes sociais.
Desses estudos, as descobertas sugerem grandes nives de economia tendem estar
associados a alto nível educacional e renda, grandes residências e dentre elas
as com maior número de pessoas, consumidores jovens e/ou com grande tendência a
valores ambientais.
\end{itemize}

\subsection{Tecnologias e Tendências}
\label{ssec:ret_tec}

Como constatado, os medidores atualmente utilizados pelas
concessionárias, os medidores analógicos e eletrônicos, permitem fornecer um
retorno de baixo custo mas com um potencial melhor de economia de energia,
o Faturamento Aprimorado. As contas de energia de companhias como a Light,
Ampla, Cemig e Eletropaulo fornecem o histórico de consumo dos últimos 12 meses,
uma informação que pode auxiliar o consumidor, já podendo ser consideradas um
Faturamento Aprimorado. Entretanto outras informações podem ser utilizadas,
como referencias do consumo acumulado e, em especial, comparações do consumo
com o de vizinhos ou grupo pertencente. Também é possível estimar o uso
energético por uso-final utilizando os valores médios de consumo das residenciais
para cada uso no sentido de auxiliar o cliente. A ideia é
transformar a conta de energia em uma espécie de relatório do consumo energético,
com um visual mais atraente (ver Subseção~\ref{ssec:asp_visuais}),
contendo gráficos e informações no sentido de atrair o consumidor a se
preocupar com o tema, sendo esse o primeiro passo \cite{2009_epri}.

No entanto, o sistema elétrico atual está se tornando obsoleto para atender aos problemas de
aumento de carga nos centros urbanos devido ao crescimento do setor
de serviços e do consumo das residencias. Há uma presença cada vez maior
de cargas eletrônicas injetando harmônicos e a geração centralizada exige
excessivamente da capacidade de transmissão e distribuição,
sobrecarregando as linhas nesses grandes centros que nem sempre podem
corresponder à necessidade de novas linhas. A falta de informações sobre o
estado do sistema dificultam a operação e planejamento de uma rede cada vez mais
sobrecarregada. As \gls{ict} revolucionaram as redes de telecomunicações e
serão a tendência para a criação das redes elétricas inteligentes (\emph{smart
grids}), o novo sistema elétrico que tem como objetivo responder a essas
dificuldades. Ainda não foram definidas todas as características desse sistema,
contudo, as principais características são o uso de comunicações em tempo real
para o controle e informação, o uso massivo de sensores e medidores para
monitoração do sistema, faturamento com preços para o momento de uso,
gestão pelo lado da demanda, a integração de
componentes avançados como linhas de transmissão supercondutoras,
armazenamento de energia, eletrônica de potência, geração distribuída
etc.  \cite{dissert_caires,aceee_2010_estudos_feedback}

Os equipamentos eletrônicos de medição utilizados nas redes inteligentes,
referidos neste trabalho como medidores inteligentes
(\emph{advanced/smart metering}), fornecerão uma gama maior de
informações em tempo-real para as concessionárias, melhorando a
operação e planejamento. Ao mesmo tempo, será possível a
concessionária comunicar-se com o cliente, oferecendo incentivos (como
descontos) para reduções de carga durante os horários de ponta, outros
planos de tarifação com preços dinâmicos de acordo com os horários,
aumentando a interação da concessionária com o cliente. Essas
especificações estão regulamentadas na Resolução Normativa n$^o$~502
da \gls{aneel} \cite{ren502}.

Por outro lado, pelo ponto de vista da demanda (ou dos consumidores) essa
informação também estará disponível, trazendo uma gama de novas oportunidades
para os usuários participarem ativamente. Mais especificamente, na abordagem do
tema atual, os medidores inteligentes oferecem uma base
a ser explorada para fornecer o retorno em larga escala para os consumidores,
tanto o direto quanto indireto. Nos medidores utilizados nos \gls{eua}
foram apontadas algumas dificuldades técnicas para esse fornecimento,
sendo elas:

\begin{itemize}
\item a necessidade de grande quantidade de energia para enviar
um sinal frequente ao consumidor;
\item a atender a necessidade do sinal ser enviado em intervalos frequentes,
atendendo a taxa escolhida de 7 s.
\end{itemize}

Um estudo na industria estadunidense mostrou que é possível dos medidores terem
seu \emph{hardware} substituídos, com um custo adicional, no futuro para que
possam fornecer medições de pequena energia e \emph{chips} de comunicação para
habilitar dados de equipamentos específicos, assim como automação para grandes
cargas, como unidades de condicionamento ambiental, bombas etc. Desta forma,
seria possível fornecer tanto a tecnologia para o Retorno em Tempo Real com a
utilização de \emph{displays} dentro do domicílio, quanto o Retorno em Tempo
Real Desagregado \cite{aceee_2010_estudos_feedback}.

Algumas empresas se estabeleceram no novo mercado para informar o consumidor
sobre o seu uso de energia e em torná-la mais eficiente, antes mesmo que
estivessem disponíveis os medidores inteligentes.  Elas fornecem o retorno
indireto em alguns países desenvolvidos, dentre eles o \gls{eua}, Austrália,
Nova Zelândia, Reino Unido.  Dentre essas empresas, faz-se referência a
\emph{Positive Energy}~\cite{opower_site} e \emph{C3 Energy}~\cite{c3_site} que
disponibilizam seus serviços, organizados na Tabela~\ref{tab:servicos_ret_ind},
utilizando os dados da concessionária, independente quando presentes na
residência os medidores convencionais ou inteligentes.  É importante notar que
as abordagens utilizadas por empresas nesse ramo utilizarão análises mais
complexas conforme a presença de dados mais detalhados, frequentes e
desagregados estejam disponíveis.

\begin{table}[h!t]
\resizebox{\textwidth}{!}{
\begin{tabular}{m{2.5cm}m{5cm}m{8cm}}
\hline \hline
\centering{\textbf{Empresa}} & \textbf{Tecnologia de Retorno} &
\textbf{Princípios Comportamentais} \\
\hline \hline
\centerline{\textbf{Positive}}\centerline{\textbf{Energy}}\centerline{\cite{opower_site}} &
Dependendo da concessionária, envia correspondências mensais ou
trimestrais e/ou fornecem um portal na internet com novas redes sociais &
\emph{Tipo de Retorno}: Retorno indireto incluindo informação sobre o domicílio
e conselhos, auditorias de energia através do uso da \emph{web}, análise de
faturamento, consumo estimado por equipamento, \gls{co2}, k\acs{wh} e \$.

\emph{Princípios Comportamentais}: Comparações sociais, metas, comparações
pessoais e plano de ações. \\
\hline
\centerline{\textbf{C3}}\centerline{\textbf{Energy}}\centerline{\cite{c3_site}} &
Portal de comunidade social com retorno de consumo de energia e água &
\emph{Tipo de Retorno}: Retorno indireto incluindo informação sobre o domicílio
e conselhos, auditorias de energia através do uso da \emph{web}, análise de
faturamento, consumo estimado por equipamento, \gls{co2}, k\acs{wh}, \$ e
outras unidades.

\emph{Princípios Comportamentais}: Comparações sociais, metas, competições
redes sociais, comparações pessoais e plano de ações. \\
\hline \hline
\end{tabular}
}
\caption[Empresas utilizando informação da concessionária e as
oportunidades e incentivo de economia de energia oferecidas.]
{Empresas utilizando informação da concessionária e as
oportunidades e isentivo de economia de energia oferecidas. Extraído e
atualizado de \cite[tradução própria]{aceee_2010_estudos_feedback}.}
\label{tab:servicos_ret_ind}
\end{table}

Já o retorno direto pode ser encontrado através de
\emph{displays} de energia no domicílio. A Tabela~\ref{tab:servicos_ret_dir}
identifica alguns dos \emph{displays} oferecidos atualmente e suas propriedades.
Muitas vezes as companhias oferecem também análises e estimativas do consumo
especifico de equipamentos, comparações sociais e outros princípios para motivar os
consumidores a economizar energia. A informação de consumo de equipamentos
especifico pode ser estimada ou realizada através de sensores nos equipamentos.

\begin{table}[h!t]
\resizebox{\textwidth}{!}{
\begin{tabular}{p{4cm}p{7cm}p{7cm}}
\hline \hline
&
\multicolumn{1}{c}{\textbf{The Energy Detector TED} \cite{ted_site} }&
\multicolumn{1}{c}{\textbf{Wattson}                 \cite{wattson_site}}\\
\hline \hline
\textbf{Descrição da \newline Tecnologia} &
\emph{Software} de suporte, aplicativos para celular &
\emph{Software} de suporte com acesso a comunidades \\
\hline
\textbf{Mecanismos de \newline Retorno} &
Mostradores em tempo real de k\acs{watt}, \$/hr, \gls{co2}, consumo e gastos
diários, conta estimada em k\acs{wh} e \$, pico de consumo, voltagem
min/max e custo/demanda projetada &
Mostradores em tempo real aproximado do consumo em \acs{watt},
k\acs{watt}, conta estimada. Leituras entre 3 a 20 s. Brilha conforma o
consumo: azul para consumo baixo; roxo para médio; vermelho para alto. \\
\hline
{\multirow{5}{4cm}{\textbf{Princípios Comportamentais}}} &
\multicolumn{2}{c}{\emph{Retorno de Informação:}}
\\
& & \\
&
\multicolumn{2}{p{14cm}}{
Retorno direto incluindo conselhos, auditorias de energia baseadas na \emph{web},
análise do consumo, estimativa de consumo por equipamentos, \gls{co2} e \$.
}
\\
& & \\
&
\multicolumn{2}{p{14cm}}{\emph{Motivações Oferecidas:}
\centering
Comparações sociais, metas, comparações pessoais e etapas de ações.
}
\\
\hline \hline
&
\multicolumn{1}{c}{\textbf{PowerCost Monitor} \cite{powercost_site}}&
\multicolumn{1}{c}{\textbf{Efergy Elite}      \cite{efergy_site}}\\
\hline \hline
\textbf{Descrição da\newline Tecnologia} &
\emph{Software} de suporte, aplicativos para celular &
\emph{Software} de suporte, aplicativos para celular \\
\hline
\textbf{Mecanismos de\newline Retorno} &
Mostradores em tempo real aproximado do consumo em
k\acs{watt} e \$/hr, pico de consumo nas últimas 24 horas, contagem de
k/\acs{wh} (reiniciável), recurso para medição de equipamentos específicos. &
Mostradores em tempo real aproximado do consumo em k\acs{watt} e \$/hora
(leituras em 6, 12 ou 18 s), informação de consumo média por hora, semanal,
mensal. Alarmes para consumo alto. \\
\hline
{\multirow{5}{4cm}{\textbf{Princípios Comportamentais}}} &
\multicolumn{2}{c}{\emph{Retorno de Informação:}} \\
& & \\
&
Retorno direto incluindo conselhos, auditorias de energia baseadas na \emph{web},
análise do consumo, estimativa de consumo por equipamentos, \gls{co2} e \$.  &
Retorno direto, análise de consumo, estimativa de consumo em \$.  \\
& & \\
&
\multicolumn{2}{p{14cm}}{\emph{Motivações Oferecidas:}
\centering Metas e comparações pessoais}
\\
\hline \hline
\end{tabular}
}
\caption[Especificações de \emph{displays} domiciliares disponíveis.]{
Especificações de \emph{displays} domiciliares disponíveis. Extraído de
\cite[tradução própria]{aceee_2010_estudos_feedback}.}
\label{tab:servicos_ret_dir}
\end{table}

% TODO Atualizar a tabela

Uma outra maneira para fornecer o
Retorno em Tempo Real Desagregado é através do \gls{nilm}
(Capítulo~\ref{cap:nilm}). Essa técnica --- ainda em desenvolvimento,
mas de crescente interesse no mundo devido a todas questões aqui
levantadas, com ofensivas de pesquisas no tema por grandes empresas como
\emph{Intel} e \emph{Belkin} --- coloca o peso da desagregação da
informação no \emph{software}, reduzindo a necessidade de investimento
em sensores e \emph{hardware}, sendo assim um método potencialmente
mais favorável economicamente para a implementação de programas de
\gls{ee} fornecendo esse tipo de retorno, no entanto isso dependerá
de suas limitações e eficácia.  No Brasil, os esforços da \gls{aneel}
em regulamentar as bases para os novos medidores inteligentes dão
poder ao consumidor de exigir à concessionária acesso às medições de
tensão e corrente de cada fase, como rege no art.~3$^o$ da Resolução
Normativa n$^o$~502 \cite{ren502}. Ainda não se especificou a taxa de
amostragem na qual essas leituras serão disponibilizadas, por sua vez,
pode-se utilizar toda a infra-estrutura das colheitas de medidas e
comunicação oferecida pelos medidores inteligentes no intuito de
maximizar o custo-benefício. Caso a amostragem deles seja baixa, ou de
interesse aumentá-la para obter uma maior capacidade de identificação
dos equipamentos, o \gls{nilm} pode fazer mão de um \emph{hardware}
próprio de medição.

Percebe-se que ainda é incerto se os medidores inteligentes são a melhor
alternativa para fornecer retorno de informação, contudo parece natural sua
utilização. Diversas tecnologias podem ser utilizadas envolvendo, ou não, as
concessionárias. Apenas com o desenvolvimento dessas tecnologias será
possível determinar as limitações, custos e vantagens para definir o que é
economicamente mais atraente.

Finalmente, uma outra tendência é o uso de automação da rede doméstica. A
automação, além de melhorar a qualidade de vida dos consumidores, pode aumentar
o potencial de redução de consumo. Com uma maior capacidade de administração de
sua demanda sem grande esforço, facilita-se aos consumidores de realizarem a
mudança de hábitos no sentido de um comportamento sustentável, simplificando
a conduta do sistema elétrico de um modo mais econômico.

\subsection{Aspectos Psicológicos}
\label{ssec:asp_psic}

A tecnologia apenas concebe as possibilidades de informação a
serem repassadas ao consumidor, no entanto, a questão ainda está em como
apresentar essa informação e motivar o usuário para a mudança.
O conselho de profissionais nos campos de psicologia, sociologia,
\emph{marketing}, mudança e economia comportamental serão críticos para
motivar, habilitar e continuamente empreender consumidores na gestão de sistemas
de energia residenciais \cite{aceee_2010_estudos_feedback}.
Um exemplo é a empresa \emph{Positive Energy}, que utiliza psicólogos para
auxiliar no desenvolvimento de suas tecnologias, disponibilizando ferramentas
sociais e estratégias persuasivas para um engajamento maior de seus clientes.

Um levantamento das noções básicas da psicologia motivacional foi realizada em
\cite{2010_aspectos_psicologicos_usa}, assim como uma estrutura para os desenvolvedores
da tecnologia aplicada nos programas de retorno de informação no sentido de motivar a
mudança comportamental para uma melhor \gls{ee} e um mundo sustentável.
Será realizado um resumo desses tópicos a seguir guiado nessa referência, mas
vale enfatizar que os profissionais nessas áreas devem analisar o tema e
escolher a melhor abordagem a ser utilizada nas tecnologias desenvolvidas.

O objetivo é motivar o consumidor para a mudança através
de recomendações levando em conta o processo de mudança comportamental
do consumidor. Define-se a motivação como
\cite[pp. 927-928, tradução própria]{2010_aspectos_psicologicos_usa}:

\begin{quote}
Motivação é um questionamento ao porquê do comportamento. Ela é um estado
interno ou condição (as vezes descrita como uma necessidade, desejo ou querer)
que serve para ativar ou energizar o comportamento. Motivação está fortemente
ligada a processos emocionais. Emoções podem estar envolvidas na iniciação
comportamental (como a emoção de solidão pode motivar a ação de procurar
companhia). Em alternativa, o desejo para viver uma emoção em particular pode
também motivar para a ação (como a decisão de correr uma maratona pode ser
motivada pelo desejo de experimentar a sensação de realização de um feito).
\end{quote}

Ela é influenciada por ideais psicológicos que foram aprendidos pelos
indivíduos. Nota-se que diferentes indivíduos tem ideais psicológicos distintos,
estes estando apresentados em ordem crescente quanto a possibilidade de
sofrerem alteração:

\begin{itemize}
\item \emph{Atitudes} são predisposições aprendidas quanto a respostas
para uma pessoa, objeto ou ideia em um modo favorável ou desfavorável. Por
exemplo, o ato de tomar banho curtos devido a uma atitude favorável em respeito
ao meio ambiente;
\item \emph{Crenças} são os meios nos quais as pessoas estruturam seu
entendimento da realidade, refletindo a ideia do que é certo e o que é errado.
A maioria das crenças são baseadas em experiências passadas, como a reciclagem
faz bem ao meio ambiente.
\item \emph{Valores} são os fundamentos para o conceito de um indivíduo sobre si
mesmo. Podem ser conceituadas como ideais comportamentais ou preferências por
vivências. No caso dos primeiros, valores funcionam como conceitos duradouros de
bem e mal, certo e errado, enquanto para preferências por vivências os valores
guiam indivíduos para vivenciarem situações nas quais as proporcionam certos
tipos de emoções. A Tabela~\ref{tab:valores} contém um subgrupo de valores
definidos por Rokeach e Maslow, onde se propõe que pessoas possuem
uma estrutura hierárquica ou prioritária de valores individuais. Rokeach
acredita que as diferenças no comportamento ocorrem devido a diferenças na
classificação de importância de valores, enquanto Maslow fornece uma ordem de
valores em níveis que serão priorizados pelos indivíduos pelos níveis mais
baixos antes dos níveis superiores.
\end{itemize}

\begin{table}[h!t]
\resizebox{\textwidth}{!}{
\begin{tabular}{ccc}
\hline
\multicolumn{1}{|p{6cm}|}{\centering \textbf{Ideais Comportamentais} (Rokeach)} &
\multicolumn{1}{p{6cm}|}{\centering \textbf{Preferências por Experiências} (Rokeach)} &
\multicolumn{1}{p{6cm}|}{\centering \textbf{Preferências por Experiências}
(Maslow, níveis em ordem crescente)} \\
\hline \hline

\multicolumn{1}{|m{6cm}|}{
\textbf{Capaz}\newline Competente, eficiente. \newline
\textbf{Prestativo}\newline Trabalhando para o bem-estar dos outros. \newline
\textbf{Honestidade}\newline Sinceridade e confiável. \newline
\textbf{Imaginativo}\newline Ousado e criativo. \newline
\textbf{Independente}\newline Autoconfiante, autossuficiente. \newline
\textbf{Intelectual}\newline Inteligente e pensativo. \newline
\textbf{Lógico}\newline Consistente e racional. \newline
\textbf{Obediente}\newline Atencioso e respeitoso. \newline
\textbf{Responsável}\newline Fidedigno e confiável.
} &
\multicolumn{1}{m{6cm}|}{
\textbf{Vida Confortável}\newline Uma vida próspera. \newline
\textbf{Liberdade}\newline Independência e liberdade de escolha. \newline
\textbf{Saúde}\newline Bem-estar psicológico e físico. \newline
\textbf{Harmônia Interna}\newline Ausência de conflitos interiores. \newline
\textbf{Sentimento de Realização}\newline Uma contribuição duradoura. \newline
\textbf{Reconhecimento Social}\newline Respeito e admiração. \newline
\textbf{Sabedoria}\newline Um entendimento maduro da vida. \newline
\textbf{Um Mundo Belo}\newline Beleza da natureza e das artes.
} &
\multicolumn{1}{m{6cm}|}{
\textbf{Psicológico}\newline Homeostase e apetites. \newline
\textbf{Segurança}\newline Segurança do corpo, emprego, recursos, família, saúde,
propriedade. \newline
\textbf{Amor/Aceitação}\newline Afeição e ser aceito. \newline
\textbf{Estima}\newline Respeito próprio, autoestima, estima dos outros \newline
\textbf{Realização Pessoal}\newline Encontrar satisfação pessoal e compreender seu
potencial.
} \\
\hline
\end{tabular}
}
\caption[Valores propostos por Rokeach e Maslow.]
{Valores propostos por Rokeach e Maslow. Tradução própria de
\cite[pp. 928]{2010_aspectos_psicologicos_usa}.}
\label{tab:valores}
\end{table}

Outra questão importante é a persistência ou durabilidade do
comportamento, ou seja, da capacidade do comportamento manter-se, sem
a necessidade de intervenções.  Para atingir esse objetivo é
aconselhável motivação intrínseca, que é a realização de uma atividade
pelas satisfações que a tarefa oferece, enquanto o seu oposto, a
motivação extrínseca, está ligada à realização para obter uma
consequência separável. Exemplos para o primeiro são: curiosidade,
competência e satisfação; no outro caso: incentivos materiais e
reconhecimento social.

Utilizou-se nessa referência o Modelo Transteórico\footnote{A referência
levanta outros modelos comumente utilizados e seus prós e contras.
Também há uma preocupação quanto a modelar a mudança através de
estados discretos, que podem não representar bem a realidade do
processo. A utilização do modelo é justificada por seu valor
heurístico, usado como um modelo simplificado de uma mudança ideal.},
onde se considera que o processo da mudança ocorre em uma série de
estados, sendo a motivação a força motriz para se deslocar entre os
estágios. O objetivo em cada etapa, assim como recomendações para o
atingir estão a seguir:

\begin{itemize}
\item Estrutura Exemplo --- \emph{Etapa}: explicação.
\begin{enumerate}
\item Objetivo nessa etapa.
\begin{enumerate}
\item Recomendação para atingir o objetivo.
\end{enumerate}
\end{enumerate}
\item \emph{Pré-contemplação}: o indivíduo está desencorajado,
relutante em manter atitudes em prol da mudança, mal-informado ou
desconhece o problema comportamental. Não há previsão de ação no
futuro, esse medido normalmente como os próximos 6 meses.
\begin{enumerate}
\item Apresentar informação em moderação com o proposito de plantar a
semente no sentido dos indivíduos tomarem conhecimento de seus
comportamentos (de consumo de energia) atuais problemáticos. A
moderação é importante pois uma maior intensidade produzirá, geralmente,
menores resultados nessa etapa.
\begin{enumerate}
\item Fornecer retorno de informação personalizado notando tanto os
prós e contras de um comportamento \emph{não-sustentável}. Apresentar
os benefícios e consequências em relação aos valores pessoais, de modo
não tendencioso.
\item Utilizar normas sociais a respeito de comportamentos de consumo
sustentáveis, combinando o uso de normas injuntivas e descritivas. As
normas sociais são as regras ou expectativas por um comportamento
apropriado em uma situação social em particular. Elas ``podem levar
pessoas a dizer coisas que sabem que não são verdade, utilizar drogas
ilícitas ou deixar de reagir a uma ameaça eminente '' \cite[pp. 51,
tradução própria]{aceee_2010_estudos_feedback}.  Normas descritivas
são percepções de comportamentos normalmente realizados (ex. 85\% da
sua vizinhança reciclam), apelando ao valor de Maslow de
\emph{Amor/Aceitação}. Normas injuntivas são percepções e
comportamentos que são normalmente aceitos ou aprovados (ex. um sinal
de polegar positivo com o texto ``Gere menos resíduos''). Essas apelam
para o valor de Rokeach \emph{Obediente}.  Ao juntar ambas normas
descritivas e injuntivas, há uma chance ainda maior de sucesso quando
em comparação da aplicação delas isoladamente.
\item Fornecer uma variedade de pequenas ações de consumo que, se realizadas,
podem ter impactos positivos no meio ambiente. Isso trabalhará duas
barreiras para a motivação, que são: não se sentir competente e não acreditar
que suas ações levarão a um resultado positivo. Apresentar uma variedade de
ações apela ao valor de Rokeach de \emph{Liberdade} e aumenta o senso de
controle pessoal assim como a motivação intrínseca.
\end{enumerate}
\end{enumerate}
\item \emph{Contemplação}: há conhecimento de seu problema comportamental e
planeja-se uma mudança no futuro. Contempladores são abertos a informação sobre
o problema, ainda que estão distantes do compromisso de mudança devido ao
sentimento de ambivalência;
\begin{enumerate}
\item Pender a balança no sentido de mudança. Ambivalência\footnote{Presença de
sentimentos/pensamentos conflitantes perante uma coisa ou pessoa.} é o problema chave
que precisa ser resolvido, uma vez que a avaliação dos prós e contras tem mais
ou menos o mesmo peso.
\begin{enumerate}
\item Nessa etapa, deve-se fornecer o retorno de informação enfatizando os prós de um
comportamento sustentável e os contras do comportamento não-sustentável. É
importante auxiliar o consumidor em perceber os prós, uma vez que os mesmos
resistirão às mudanças enquanto percebam isso como um fator redutor de sua
qualidade de vida, em especial àquelas que salientam o sacrifício pessoal pelo
bem comum. Os contras devem enfatizar nos custos de comportamentos
não-sustentáveis, numa perspectiva de perda em detrimento de ganho. O foco nos
valores pessoais podem ser extremamente persuasivos nesse estágio.
\item Lembrar o individuo de sua atitude em favor do meio ambiente, informá-lo
da discrepância de suas atitudes e o comportamento correspondente, encorajar a
mudança. Essa técnica apela para a dissonância cognitiva\footnote{Um estado
desconfortável que ocorre quando a pessoa possui uma atitude e um comportamento
que são psicologicamente inconsistentes.}. Como as pessoas mudam de atitudes mais
fácil que de comportamento, é importante encorajar o comportamento sustentável.
\item Fornecer incentivo para pequenas mudanças de consumo (independente da
intenção do consumidor original era a utilização sustentável de energia) para
fomentar maiores mudanças no futuro.
\item Vincular a tecnologia de retorno a um portal de uma comunidade virtual e
incentivar o indivíduo para procurar e ler a informação de experiências de
outros usuários com consumos sustentáveis na comunidade. Isso apela para normas
sociais de um modo vivo e personalizado, explorando a abertura dos
contempladores ao tema, mas, ao mesmo tempo, sem forçar nenhum tipo de ação.
\end{enumerate}
\end{enumerate}
\item \emph{Preparação}: momento em que o indivíduo está pronto para ação no
futuro iminente (medido normalmente como 1 mês), e tem como objetivo desenvolver
e engajar-se a um plano. Pelo menos uma tentativa de mudança foi realizada no
último ano;
\begin{enumerate}
\item Ajudar os usuários a desenvolver um plano que seja aceitável, acessível e
efetivo. Esses planos podem se relacionar a ações extraordinárias (compra de um
refrigerador eficiente) ou diárias (tomar banhos mais curtos). Um objetivo é
definido como uma representação interna de um resultado desejado. Usuários na
fase de preparação podem ter objetivos abstratos mas não saber necessariamente
como os alcançar.
\begin{enumerate}
\item Ajudar na criação de metas pessoais especificas e quantitativas
(o nível de dificuldade deve se elevar conforme o sucesso, começando com metas
simples e partindo para mais complicadas). Metas difíceis, pessoais e
especificas tem maior engajamento quando comparadas a tarefas ``faça o seu
melhor'', fáceis ou que lhe foram atribuídas.
\item Desenvolver modos múltiplos para os consumidores atingirem suas metas e
incentivá-los para utilizar sua habilidade e experiência pessoal nesses planos.
\item Fornecer no portal da comunidade virtual aos usuários a opção de ter
um ``Conselheiro/Tutor''. Os conselheiros seriam pessoas exemplares na
fase de ação ou manutenção. Essa conexão fornece um maior nível de
engajamento.
\end{enumerate}
\end{enumerate}
\item \emph{Ação}: ocorre a manifestação de modo evidente da mudança
comportamental, usualmente dentro dos últimos 6 meses;
\begin{enumerate}
\item Reforçar positivamente ações sustentáveis de energia. O reforço positivo é
a técnica mais efetiva para motivar a maior ocorrência de um comportamento
desejado, ela tende a aumentar a motivação intrínseca. Técnicas como a punição ou
reforço negativo evitam um comportamento não desejado, mas não o substitui por
nada, além de reduzir a motivação intrínseca.
\begin{enumerate}
\item Fornecer o reforço positivo imediatamente após o comportamento desejado
ocorrer e em múltiplos modos com o intuito de aumentar sua eficácia.
\end{enumerate}
\item Desenvolver motivações intrínsecas para o comportamento sustentável.
\begin{enumerate}
\item Permitir uma exploração interativa, personalizável e anotações na
interface oferecida.
\end{enumerate}
\end{enumerate}
\item \emph{Manutenção, Recaída e Reciclagem}: trabalho no sentido de manter o
comportamento alterado e a luta para prevenir recaídas. Se aquela ocorrer, o
indivíduo regressa a um dos estágios anteriores e o processo recomeça.
\begin{enumerate}
\item Manter o comportamento sustentável permanente. Em algum momento as
mudanças tornar-se-ão sustentáveis por si próprias, sendo possível a saída
do indivíduo do ciclo de mudança. Enquanto isso, o objetivo deve ser fazer o
indivíduo apenas um pouco mais consciente e informado.
\begin{enumerate}
\item Apoiar ações sustentáveis para que elas virem hábitos, relembrando os
usuários para realizar ações específicas.
\item Fornecer a opção para usuários nessa etapa de se tornarem
``Conselheiros/Tutores'' para indivíduos na etapa de preparação. Essa técnica
aplica dissonância cognitiva, uma vez que indivíduos que tentaram persuadir
alguém racionalizarão internamente o seu comportamento, e assim estarão
propensos a intensificar seu engajamento. Esse método apela para o valor de
Rokeach \emph{Reconhecimento Pessoal} e \emph{Sabedoria}, e, em retorno, pode
gerar satisfações intrínsecas de competência e satisfação.
\item Encorajar os usuários para reforço e reflexões pessoais em suas
experiências através de um diário. A reflexão de suas atitudes em relação a
energia e percepção de seu progresso podem trazer a tona satisfações intrínsecas
de interesse, competência e satisfação. O reforço pessoal (na forma de orgulho
ou senso de realização) trazerá, também, satisfações intrínsecas, no caso de
competência, e ainda levar a percepções de auto-eficácia. Isso é
importante pois, para um indivíduo vivenciar sucesso de longo-termo, eles
precisam de auto-eficácia e atribuições intrínsecas do comportamento.
\item Manter um ciclo de motivação intrínseca de interesse, curiosidade, desafio
ótimo, competência e satisfação. A motivação intrínseca é um ciclo de dois
passos. Primeiro, estímulos como novidade, complexidade e mudança atraem a
atenção, curiosidade e interesse, o que convida para a investigação, exploração
e manipulação dos estímulos. Segundo, desempenhos de competência em tarefas são
desfrutados, enquanto o crescimento da satisfação aumenta tanto o desejo na
atividade quanto a capacidade de confrontar desafios parecidos no futuro.
\end{enumerate}
\end{enumerate}
\end{itemize}

\subsection{Aspectos Visuais da Informação}
\label{ssec:asp_visuais}

A apresentação na qual a informação é realizada também afetará o entendimento
e engajamento do consumidor nos programas de \gls{ee}. Novamente, os especialistas
e profissionais da área, no caso de \emph{design} e da informação,
serão importantes nesse intuito, frisando que o problema jaz além de
apenas disponibilizar a tecnologia para fornecer a informação: também
será necessário tornar a informação atrativa ao consumidor.

O processo do fluxo de informação ocorre entre o remetente e o
receptor, onde este analisa a informação necessária e a apresenta na
mensagem através de textos e imagens, enquanto aquele realiza escolhas
entre as informações disponíveis nela e opta se processa-las-á
mentalmente. A visão é o sentimento mais importante para a compreensão
e vivência humana do meio externo.  Cerca de 70\% de nossas células
sensoriais estão nos olhos, assim, a visualização é um meio bastante
efetivo de realizar a comunicação da informação e de dados.  Tornar a
mensagem interessante visualmente depende de vários elementos
\cite{2012_visualisation_sweden}. Alguns exemplos são: fornecer para a
mensagem uma estrutura na qual serão guiados os princípios para a
inserção das representações; clareza para a simplicidade de
compreensão e legibilidade da informação; enfase para atrair,
direcionar ou manter a atenção; unidade da mensagem, com uma coerência
e união global \cite{it_depends}.

Em \cite{2012_visualisation_sweden}, apresentou-se a jovens de um
colégio sueco uma visualização do consumo de energia em um
\emph{display} em tempo real na sala de aula. Os jovens tiveram
participação na criação da interface\footnote{A participação dos
usuários no desenvolvimento facilita à tecnologia atender às
especificações e necessidades dos mesmos. No caso, a participação dos
usuários utilizada está de acordo com a recomendação de
\cite{2009_extreme_user_filandia}, na qual há a descrição de uma
abordagem orientada ao usuário para o retorno do consumo de energia.
Nela, o usuário apenas fornece inspiração aos desenvolvedores
profissionais, que desenvolvem as soluções.}, onde desenharam e
escolheram uma imagem para informar se o consumo estava elevado, médio
ou baixo, assim como informaram o que entenderam das informações nos
\emph{displays}.  Esse estudo revelou cinco aspectos importantes no
desenvolvimento de sua visualização nos \emph{displays}:

\begin{enumerate}
\item Deve chamar a atenção dos usuários, realizando o uso de cores brilhantes,
contrastes e quando possível uma exibição dinâmica;
\item Mostrar comparações entre o consumo de modo a deixar evidente aos usuários
resultados positivos de um esforço;
\item Fornecer o consumo em tempo real para estimular a mudança direta no
comportamento;
\item Deve conter um tom positivo e encorajador, potencializando a positividade
de um comportamento correto;
\item Ser explicativa, com pequenos textos instrutivos que fazem aos usuários
simples de entender o que eles estão vendo.
\end{enumerate}

