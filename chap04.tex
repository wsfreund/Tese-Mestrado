\chapter[NIALM]{\acrlong{nialm}}
\label{cap:nialm}


O \gls{nialm} é um método para fornecer a informação de consumo de energia
elétrica desagregado por utensílio. Ao invés das técnicas normalmente utilizadas
que envolvem o uso de sensores em cada utensílio que enviam informações para uma
central que as decodifica de modo a identificar os utensílios que estão
demandando consumo na rede, o \gls{nialm} é um método em que não ocorre a
intrusão da propriedade do usuário, contendo geralmente apenas um medidor central no
fornecimento de energia dessa propriedade. O peso da identificação dos utensílios
é transferido da utilização de uma grande quantidade de sensores e
\emph{hardware} complexo para um \emph{software} e algoritmos de processamento
de sinais no intuíto de realizar uma análise profunda das medições e desagregar
as informações.

Ainda assim, a capacidade dos algoritmos implementados dependem da
disponibilidade das informações disponíveis, de forma que quanto mais avançado
for o \emph{hardware} de medição, enviando uma quantidade e/ou frequência maior de 
informação aos algoritmos, maior será a capacidade dos mesmos de desagregar o
consumo específico dos utensílios.

Aplicações importantes para esse dispositivo já foram detalhadas nos Capítulos
anteriores, sendo elas: auxiliar ou substituir as \glspl{pph} sem que seja
necessário causar incômodo ao usuário devido a presença de medidores na
residencia (Sessão~\ref{sec:ee_dificuldades}), baratear e disponibilizar a informação 
desagregada para o fornecimento do Retorno em Tempo Real Desagregado
(Sessão~\ref{sec:ee_res_exp}) em programas de \gls{ee}. Obviamente, não é
necessário que a informação seja fornecida apenas nesses casos, podendo ser de
interesse das concessionárias na busca por furtos de energia, obter com maior
precisão e rapidez a curva de demanda de seus clientes, ou de consumidores
que buscam essa informação para reduzir seu consumo ou obter algum tipo de
controle sobre seu consumo de energia.

Os estudos nessa área datam do inicio da década de 1980. No entanto, a
referência de base ao tema ocorreu em 1992 \cite{nilm_hart}. Nela, se
utiliza apenas medições da \gls{p} em uma amostragem de 1 Hz. O estudo se
limitou a identificar apenas cargas com potencia maior à 150
\acrshort{watt}, mas a sequência da abordagem ainda é utilizada em muitos modelos 
atuais do \gls{nialm}, sendo esta: 

\begin{enumerate}
\item Se detecta transientes de consumo na rede devido a mudança de estado de uma carga 
através de mudanças no consumo que devem superar um limiear específico; 
\item Esses eventos de alteração devido as alterações de estado de uma carga são agrupados por um 
algoritmo que irá gerar os centróides para cada mudança de estado possíveis causadas pelas
cargas. Se o equipamento apenas possuir dois estados (ligado e desligado) os
centróides terão o mesmo valor absoluto, podendo agrupá-los como uma mesma
carga. No entanto, se o equipamento tiver uma sequência de estados ainda é
necessário determiná-la, procurando as possíveis combinações que fechem um ciclo
de sequência em que todos os estados são únicos utilizando os centróides
disponíveis; 
\item Com os modelos das cargas realizados, coloca-se etiquetas de tempo
nos eventos, de forma a determinar a sequência correta das maquinas de estados.
\item Em seguida é feito um levantamento da estatística detalhando o comportamento de 
consumo, como o tempo ligado e desligado de cada equipamento;
\item As estatítiscas de comportamento e sua potência e sequência de estados são
utilizados para identificar o equipamento.
\end{enumerate}

A exata abordagem é aprimorada em \cite{nilm_cole_eventos_simultaneos}, tratando
problemas de eventos de duas cargas distintas ocorrendo simultaneamente. Esses
eventos irão se parecer com uma terceira carga, dificultando a identificação
correta do ligamento ou da sequência de operação dos estados, de forma que um
aparelho irá aparecer como ligado ou em um determinado estado por dias. Para
tratar desse problema, ele considera estatisticamente a probabilidade de ocorrer
um evento simultâneo, dependendo do número de aparelhos, quantidade de eventos e
taxa de amostragem.

Outra otimização é realizada nos métodos fornecidos por 
\cite{nilm_baranski_genetic_base,nilm_baraski_genetic_detalhado}.
A primeira contribuição está em utilizar ao invés de casar um único $+\Delta{P}$ 
com outro $-\Delta{P}$, se realiza o casamento
de um conjunto de possíveis $+\Delta{P}_i$ com um outro possível grupo de
$-\Delta{P}_i$, reduzindo assim as dificuldades de eventos simultâneos.
Para melhorar a eficiência, esse casamento foi otimizado através 
de um algoritmo de força bruta. Em seguida a abordagem foi aprimorada,
utilizando um algoritmo de agrupamento com lógica \emph{fuzzy} para a criação
dos grupos correspondentes aos eventos e um algoritmo genético para otimização
dos casamentos.

Essa abordagem é utilizada similarmente para a otimização das
possíveis maquinas de estado em \cite{nilm_distribuido}, contando também com 
algoritmo genético para a inicialização e otimização das possíveis máquinas de estado. 
Entretanto a contribuição desse trabalho é uma nova arquitetura, distruibuída, para o \gls{nialm}. 
Nesse caso, é utilizado um medidor inteligente para a colheita de dados e devido a
limitações de processamento, apenas a deteção e identificação dos eventos é 
realizado no mesmo. A geração das possíveis máquinas de estado e
aparelhos é realizada em uma central, com maior capacidade de processamento, 
no qual este envia os eventos de transição para que aquela gere as sequências de
estado de transição dos aparelhos com sua identificação em uma tabela.
Posteriormente a essa etapa, chamada de aprendizagem, a tabela é enviada 
ao medidor inteligente que poderá identificar os eventos sem grandes custos
compotacionais. Se necessário o processo pode ser novamente
realizado para minimizar erros introduzidos por novos aparelhos e/ou aparelhos
não identificados ou utilizados durante a última aprendizagem.

Há ainda a possibilidade de se explorar em baixa amostragem, mesmo que 
limitadamente, outras as características da onda, como o tempo do transiente
e seu valor de pico \cite{nilm_cole_extra_info_surge}. Essas informações podem
auxiliar e minimizar os erros dos casamentos de eventos anteriores. O uso de
outras grandezas como \gls{q}, \gls{d} etc. pode ser realizada com o mesmo
objetivo. Um exemplo é \cite{nilm_distribuido}, onde também se utilizou a 
\gls{q} quando disponível no medidor inteligente.

Indo além, as possibilidades também aumentam conforme se eleva a taxa de amostragem. 
Os estudos anteriores realizados no \gls{cepel} \cite{alvaro,bezerra,aguiar} 
utilizaram uma amostragem de 60 \acrshort{hz}. 
Com isso, possibiliza-se obter uma informação mais detalhada do
transiente, no caso a envoltória do consumo de corrente da residência, 
que é propagada em uma rede neural no intuíto de diferenciar 
os equipamentos em grupos pré-determinados de consumo. Devido à boa capacidade
de identificação dos grupos de cargas selecionados, esses métodos utilizaram
apenas a informação de transitório, mas seria possível acoplar mais informações para
auxiliar na identificação, se necessário. Nesses trabalhos o medidor foi
desenvolvido pelo próprio \gls{cepel}, de forma que \cite{aguiar} explorou, além dos
resultados de aplicação da rede neural, os aspectos necessários para manter a qualidade da
informação e compactá-la utilizando técnicas como \gls{pca}. Já \cite{bezerra} forneceu 
ao sistema a possibilidade de uma arquitetura distribuída de processamento,
similar àquela explicada em \cite{nilm_distribuido}. Nesse trabalho também foi
analisado o uso de \gls{pcd} em comparação ao pré-processamento por \gls{pca}.
Por fim, \cite{alvaro} implementou um detector automático de eventos de
transição de estados de cargas.

Taxas ainda maiores de amostragem podem ser utilizadas, permitindo a utilização
de outras técnicas. A estratégia em \cite{nilm_laughman} envolvem encontrar transientes e subtrair as
ondas pré e pós transiente de corrente em regimente permantente. Aplica-se a
Transformada de Fourier e compara-se com um banco de dados. 

Utilizou-se em \cite{nascimento} uma amostragem de 13,6
\acrshort{hz}. Nesse trabalho utiliza-se uma estrutura de árvore de decisão para
a discriminação. Uma pré-classificação é realizada utilizando o \emph{hardware}
para equipamentos com valores específicos de \gls{p} e \gls{q}. Caso não seja um
dos equipamentos de simples discriminação, modifica-se a informação a ser
discriminada nesta ordem: aplicada Transformada de Hilbert, Transformada Wavelet
e por último o Método de Burg. Essa informação é apresentada ao final classificador, que
realiza a discriminação em duas etapas, primeiro selecionando um grupo genérico
ao qual o equipamento pertence, para depois selecionar o grupo específico. A
seleção é feita através do erro médio quadrático aplicado entre o banco de dados
e o resultado do espectro de Burg para cada nível de detalhe utilizado da
Transformada Wavelet. O equipamento do banco de dados que tiver maior número de 
detalhes compatível com o sendo testado é o resultado do processo de
discriminação.









