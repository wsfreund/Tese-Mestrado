\chapter[Monitoração Não-Invasiva de Cargas Elétricas]{\acrfull{nilm}}
\label{cap:nilm}

Este capítulo trata em detalhe as particularidades envolvidas no
desenvolvimento da tecnologia conhecida como
\gls{nilm}\footnote{NIALM e NALM são outras abreviaturas utilizadas na
literatura.}, em especial quanto aos aspectos técnicos. A
Sessão~\ref{sec:nilm_aspec_gerais} irá introduzir os seus pontos
gerais, sendo um resumo da informação contida nos capítulos
anteriores. Em seguida, as diversas metodologias utilizadas no mundo
são utilizadas para um levantamento dos aspectos envolvidos no
desenvolvimento da tecnologia na Sessão~\ref{sec:nilm_mundo}.
Finalmente, será apresentado na Sessão~\ref{sec:nilm_cepel} a evolução
dessa tecnologia no escopo deste trabalho, desenvolvido em conjunto
com o \gls{cepel}.

\section{Aspectos Gerais}
\label{sec:nilm_aspec_gerais}

O \gls{nilm} é uma alternativa para fornecer a informação de consumo
de energia elétrica desagregado por utensílio. Ao invés das técnicas
normalmente utilizadas --- onde se faz mão de sensores dispostos em cada
utensílio, esses enviando informações para uma central encarregada de
decodificá-las e, assim, identificar os utensílios que estão demandando
consumo na rede ---, o \gls{nilm} é um método em que não ocorre a intrusão
da propriedade do usuário (ou intrusão em escala mínima), contendo
geralmente apenas um medidor central no fornecimento de energia dessa
propriedade. O peso da identificação dos utensílios é transferido da
utilização de uma grande quantidade de sensores e \emph{hardware}
complexo para um \emph{software} e algoritmos de processamento de
sinais no intuíto de realizar uma análise profunda das medições e
desagregar as informações.

Ainda que o peso da identificação esteja no \emph{software}, a
aptidão dos algoritmos implementados dependem da capacidade do
medidor de extrair informações da rede e das distorções causadas pelos
utensílios na mesma, de forma que quanto mais avançado for o
\emph{hardware} de medição, enviando uma quantidade e/ou frequência
maior de informação aos algoritmos encarregados de realizar a
discriminação dos rastros deixados na rede elétrica pelos aparelhos,
também maior será a capacidade dos mesmos de desagregar o consumo
específico dos utensílios.

Possíveis aplicações para esse dispositivo são: 

\begin{itemize}
\item auxiliar ou substituir as \glspl{pph} sem que seja necessário
causar incômodo ao consumidor devido a presença de medidores na
residencia, fornecendo assim dados com maior fidelidade em caracter
desagregado por utensílio e maior frequência para estudos de \gls{ee}
no consumo de eletricidade (Sessão~\ref{sec:ee_dificuldades}), em
especial para o setor residencial;
\item disponibilizar a informação desagregada para o fornecimento do
Retorno Indireto Diário/Semanal e/ou Retorno em Tempo Real Desagregado
em programa de \gls{ee}. Os programa de \gls{ee} utilizando retorno
de informação são possíveis fontes de redução de consumo nos grandes
centros urbanos, no entanto, são necessários estudos no Brasil para
determinar seu potencial (Sessão~\ref{sec:ee_res_exp}). O retorno de
informação desagregada por utensílio pode ser utilizado por empresas
que oferecem serviços de redução de consumo
(Subsessão~\ref{ssec:ret_tec}), ou mesmo por consumidores que precisam
ter algum tipo de controle sobre seu consumo de energia; \item no
interesse das concessionárias, a informação desagregada pode auxiliar
os clientes a identificarem consumo não essencial durante o horário de
ponta, auxiliando no deslocamento de carga (ver item \emph{Resposta de
Ponta e Demanda versus Economia Fora de Ponta} na
Subsessão~\ref{ssec:ret_outros}), assim como identificar clientes com
maior potencial de redução de consumo durante esses períodos para
oferecer incentivos nesse sentido (Subssessão~\ref{ssec:ret_tec}); 
\item outras possíveis aplicações: monitoramento da qualidade de
energia, diagnóstico de carga, identificação de aparelhos defeituosos
ou com consumo excessivo de energia;
\end{itemize}

As possibilidades de aplicação como um meio de melhoria da \gls{ee} e
redução da intensidade elétrica tem elevado o interesse nesse assunto,
em especial nos países desenvolvidos, como uma forma de aliviar
a pressão de consumo nos grandes centros urbanos e na redução de
emissões de \gls{co2} \cite{nilm_zeifman_review_2011}.
Isso tem levado a gigantes no setor eletrônico, como \emph{Intel} e
\emph{Belkin}, a investirem fortemente no desenvolvimento dessa
tecnologia. O crescente interesse na evolução por parte da academia
levou a organização do primeiro \emph{workshop} especificamente para o
tema em 2012 \cite{workshop_nilm}. Essa alternativa é mais simples
quando comparando com a automação residencial por não requerer
mudanças na produção dos eletrodomésticos --- a automação requer
comunicação nos dois sentidos (entre a interface e o aparelho), tal
como a capacidade de controle do aparelho, de forma que se faz
necessário adaptar aparelhos antigos e a produção dos novos
equipamentos com essas capacidades --- juntamente com o fato de haver
um relutância social quanto às residências automatizadas, apesar de
esforços governamentais e da mídia local \cite{Lipoff_Automation_2010}
(a referência estudou a falta de interesse na automação residencial
nos \gls{eua}). Não bastasse, também se pode citar o nascimento de
\emph{start-ups} procurando espaço na corrida por esse novo mercado,
como \emph{GetEmme} \cite{getemme_site} e \emph{Navetas}
\cite{navetas_site}.

\section{As Diversas Metodologias Utilizadas no Mundo}
\label{sec:nilm_mundo}

A ideia de desagregar a informação não é nova, sendo possível
encontrar pesquisas nesse sentido datando da década de 1980. Apesar
disso, um extenso levantamento bibliográfico
\cite{nilm_hart_1992_8,nilm_bouloutas_viterbi_ext_1991_11,
nilm_hart_fsm_viterbi_1993_12,nilm_norford_leeb_medianfilt_1996_13,
nilm_leeb_spectral_envelope_1995_23,
nilm_cole_data_extraction_1998_14,nilm_cole_extra_info_surge_1998_15,
nilm_powers_15minsamp_1991_16,nilm_farinaccio_16ssamp_1999_17,
nilm_marceau_16ssamp_improved_1999_18,nilm_baranski_genetic_base_2003_19,
nilm_baranski_genetic_detalhado_2004_20,nilm_baranski_summary_2004_21,
nilm_matthews_overview_2008_22,nilm_laughman_continuous_variables_2003_9,
nilm_lee_variable_speed_estimation_2005_24,
nilm_wichakool_2009_25,nilm_shaw_2008_26,nilm_srinivasan_nn_2006_27,
nilm_patel_2007_29,nilm_gupta_patel_2010_30,
nilm_sultanem_1991_10,nilm_chan_2000_31,nilm_lee_2004_32,nilm_lam_2007_33,
nilm_liang_pt1_2010_34,nilm_suzuki_2011_35,nilm_berges_2008_7,
nilm_berges_2009_36,2010_nilm_melhorando_pph_usa_37,
nilm_liang_pt2_2010_40} realizado em \citet*{nilm_zeifman_review_2011}
expõe que as técnicas aplicadas em \glspl{nilm} ou não são robustas no
sentido de atenderem especificamente a um grupo limitado de utensílios
estudados, ou apresentam acurácia marginal, mostrando que o processo
de desagregação da informação não é trivial. Indo além das
dificuldades técnicas, ressaltam-se os pontos enfatizados nas
Subsessões \ref{ssec:asp_psic} e \ref{ssec:asp_visuais}, nas quais
foram evidenciadas as características multidisciplinares desse método
quando aplicado como um meio de economia de energia e intensificação
da \gls{ee}. Outras referências que não constam em
\cite{nilm_zeifman_review_2011} citadas nesta sessão contendo
abordagens de \glspl{nilm} são
\cite{nilm_apresentacao_review_2011,
nilm_bergman_distribuido_2011,nilm_zeifman_vast_2011,
nilm_zeifman_vast_hisample_pdfmerge_2011,
nilm_zeifman_vastext_approach_2012,
nilm_zeifman_statistical_vastext_1stws_2012,
nilm_zeifman_statistical_naive_enduses_2013,
nilm_genetic_2013,nilm_patel_review_2011,
nilm_coppe_nascimento,nilm_itajuba_rodrigues,
wavelet_transients_2009,
seminilm_fhmm_empiricalnmeter_2013,
seminilm_ihome_tomek_2012,
seminilm_berges_multisensor_2010}.

Dividiu-se esta sessão da seguinte maneira. As primeiras subsessões,
\ref{ssec:modelos_carga} e \ref{ssec:metodologia_generica}, referem-se
aos modelos de carga utilizados para os aparelhos eletrodomésticos e
às etapas e abordagens usualmente envolvidas na desagregação da
informação de consumo quando utilizando \glspl{nilm} respectivamente.
Uma proposta de padronização para o cálculo de eficiência é
apresentado na Subsessão~\ref{ssec:nilm_eff_calc}. Em seguida, a
Subsessão~\ref{ssec:nilm_tecnicas} irá apresentar as diversas
abordagens utilizadas, sendo guiada na proposta de divisão das
técnicas e características feita por \cite{nilm_zeifman_review_2011}.
Essa subsessão contém levantamento bastante técnico, sendo de
interesse somente daqueles que desejam se aprofundar no tema.  A
discussão das informações levantas é realizada em
\ref{ssec:nilm_discussao}, sendo sua leitura suficiente para
aqueles que desejam as informações mais relevantes.

\subsection{Modelos de Carga}
\label{ssec:modelos_carga}

Os utensílios podem ser modelados devido às características de
comportamento de suas cargas. A seguir, encontram-se possíveis
características de carga elétrica dos eletrodomésticos. Os quatro
primeiros itens são modelos de cargas mutuamente exclusivos, enquanto
os dois seguintes podem ser incluídos, dependendo das propriedades dos
aparelhos ligados à rede, para caracterizar aparelhos potencialmente
dificultadores do processo de desagregação \cite[com adaptações]{
nilm_hart_1992_8,nilm_baranski_genetic_base_2003_19,
nilm_zeifman_review_2011,nilm_zeifman_vast_hisample_pdfmerge_2011,
nilm_apresentacao_review_2011,nilm_liang_pt2_2010_40,
nilm_liang_pt1_2010_34}:

\begin{itemize}
\item \textbf{\Gls{c1}}: aparelhos que permanecem
ligados 24~h/dia, 7~dias/semana, com consumo de energia praticamente
constante. Ex.: detectores de fumaça, fontes de alimentação
constantemente ligadas à rede;
\item \textbf{\gls{c2}}\footnote{Nas referências, não há separação na
categoria \gls{c2}, que é utilizada para descrever exclusivamente
aparelhos modelados como \acs{c2a}. Nelas, a categoria
\acs{c2b} é incluída em \acs{c4}, uma vez que em ambos os
casos as \glspl{fsm} podem alterar o seu estado de operação para
qualquer outro estado independente tanto de qual estado anterior ele estava
operando como o tempo de operação nesse estado. A separação das
\glspl{fsm} com estados discretos em duas categorias parece mais
familiar do que incluir \glspl{fsm} com patamares discretos e
aleatórios em uma categoria que permite patamares
continuos.\label{fn:subdivisao}}: essa categoria é utilizada
para identificar aparelhos que passam um conjunto de estados definidos
no qual os ciclos de mudança de estados são repetidos com frequência
suficiente nos eventos diários ou semanais. Mesmo equipamentos que são
apenas ligados/desligados pelo usuário podem acabar sendo modelados
como \glspl{c2} por mudar seus patamares de consumo enquanto estiver
ligado. Essa categoria pode ser dividida em dois conjuntos:
\begin{itemize}
\item \textbf{\gls{c2a}}\fnref{fn:subdivisao}: os estados do aparelho repetem-se em
padrões definidos temporalmente, garantindo que os seus ciclos serão
observados frequentemente durante intervalos diários ou semanais. Ex.:
máquina de lavar roupas, máquinas de lavar louças;
\item \textbf{\gls{c2b}}\fnref{fn:subdivisao}: nesses aparelhos não há
um padrão para os seus ciclos de operação. A operação por uma fonte
externa, como o consumidor, altera o seu padrão de consumo sem ser
possível encontrar uma regra operativa para o ciclo através da busca
de repetições de suas trocas de estado na rede, os estados mudam
aleatoriamente depois de quantidades de tempo também aleatórios. Ex.:
ventilador de múltiplas velocidades, liquidificador --- ambos
dependendo de como operados pelo consumidor: se apenas ligados e
desligados, irão comportar-se como \acs{c3}s, enquanto se
operados de modo padronizado, irão se comportar como \acs{c2a}s;
\end{itemize}
\item \textbf{\gls{c3}}: um caso partícular das \glspl{c2} ocorre
quando o aparelho pode ser modelado como tendo apenas dois estados:
ligado/desligado. Ex.: lâmpadas, torradeiras, bombas de água;
\item \textbf{\Gls{c4}}: uma generalização das \glspl{c2}, onde
há uma infinidade de estados para os quais o aparelho pode operar.
Essa categoria pressupõe que o aparelho irá estabilizar o seu consumo
em um patamar após um período de tempo. Sua operação pode ser dividida
em dois grupos: operação manual do operador, aparelhos
auto-controlados. Estes são de mais simples detecção quando comparados
com aqueles, uma vez que seus ciclos de mudanças de estado são
distribuidos uniformemente no tempo. Ainda assim, essa categoria é o
maior desafio para as técnicas empregadas nos \glspl{nilm}, sendo
raramente tratada por elas. Ex.: lâmpadas com \emph{dimmer},
ferramentas elétricas (furadeiras, serras etc.), bomba de aquário.
\item \textbf{\Gls{c5}}\footnote{As referências optaram por não
criar essas categorias uma vez que essas são apenas
características das cargas. Por sua vez, as mesmas são citadas, no
mínimo, em \cite{nilm_zeifman_review_2011,nilm_liang_pt2_2010_40}
como dificultadores no processo de desagregação e, por esse motivo,
preferiu-se adicionar diretamente essas categorias para
enfatizar e facilitar a identificação de cargas com essas
características.\label{fn:categoria_add}}: aparelhos que causam
distúrbios na rede continuamente devido à dinâmica durante sua
operação, se referindo a possíveis oscilações inerentes a
característica operativa do mesmo. Alguns exemplos observados pela
equipe do \gls{cepel}: televisores, onde a variabilidade de brilho,
cores e som, alteram seu consumo (também observado em
\cite{nilm_zeifman_statistical_naive_enduses_2013}); computadores, que
alteram sua potência conforme a demanda dos processadores e
\emph{coolers}, consumindo mais quando o usuário está realizando
tarefas como, por exemplo, executando algoritmos do mestrado,
escutando música etc.; alguns outros aparelhos como o ar condicionado
--- esse quando com o compressor ligado --- apresentam uma dinâmica
com oscilações em frequências sub-harmônicas, os mesmos são potenciais
dificultadores à identificação de rastros deixados por outros
aparelhos, particularmente os de menor consumo, caso estudado em
\cite{nilm_liang_pt2_2010_40}. A Figura~\ref{fig:ar_cond_dinamica}
demonstra a dinâmica sub-harmônica causada na envoltória para esse
aparelho. Outros exemplos de utensílios com motores que também geram
oscilações --- mas em ordem inferior ao ar condicionado --- são:
microondas, geladeira, desumidificador
\cite{nilm_liang_pt2_2010_40};
\item \textbf{\Gls{c6}}\fnref{fn:categoria_add}: apesar de não ser
uma característica de um utensílio \emph{per se} e nem constituir um
modelo de carga elétrica, estudos de performance de \glspl{nilm} podem
considerar quais equipamentos serão potencialmente vistos como se
fossem um mesmo equipamento por possuirem os mesmos padrões. Essa categoria
varia conforme os aparelhos presentes e quais são as
características sendo extraídas. Por exemplo, um computador e uma
lâmpada incandescente possuem consumos semelhantes quando procurando
padrões no plano \acs{dp}$\times$\acs{dq}
\cite{nilm_laughman_continuous_variables_2003_9}, enquanto
equipamentos com motores de potências distintas podem não ser
desagregados quando apenas olhando para seus transitórios --- em
especial quando normalizados, abordagem que seria utilizada se
utilizando \acrfull{rna}. \cite{nilm_liang_pt1_2010_34} propõe a
equação \ref{eq:similaridade}\footnote{Nota-se aqui que qualquer 
outra divergência representativa pode ser empregada com esse 
intuíto. Divergências são generalizações das métricas e medem a
quase-distância entre duas medidas não-negativas
\cite[cap. 2]{cichocki2009nonnegative}. No caso de algumas
divergências, como a métrica Euclidiana, é possível aplicar valores
negativos.} como uma maneira de medir a similaridade entre dois
utensílios, onde $s^j_{ia,ib}$ é a relação entre a j-ésima
característica entre os utensílios $ia$ e $ib$, $N$ é o tamanho da
dimensão dessa característica e $y_{k|(ia,j)}$ é o ponto k para o
utensílio $ia$ e j-ésima característica. Quanto mais próximo for
$s^j_{ia,ib}$ da unidade, mais dificil será desagregar $ia$ e $ib$
utilizando a característica $s^j$. Uma última consideração, a presença
de similaridade entre dois aparelhos não necessariamente é um
problema. Deseja-se para certos aparelhos, como por exemplo, lâmpadas
do mesmo tipo, ou equipamentos de marcas diferentes, que eles sejam
vistos como um mesmo equipamento pelo \gls{nilm}. Há certos casos em
que as características são tão discriminantes que dificultam a
aplicação do \gls{nilm} em larga escala (ver metodologia na
p.~\pageref{nilm:emi}), sendo necessário encontrar um equilibrio entre
a capacidade discriminante das características. No entanto, utiliza-se
o termo \glspl{c6} exclusivamente para identificar os casos em que a
similaridade dos equipamentos não é desejada causando erros de
classificação.
\end{itemize}

\begin{equation}\label{eq:similaridade}
s^j_{ia,ib} = \dfrac{
\sum^N_{k=1}y^2_{k|(ia,j)}}{
\left[\sum^N_{k=1}y^2_{k|(ia,j)} + \left(
\sum^N_{k=1}y^2_{k|(ia,j)} - \sum^N_{k=1}y^2_{k|(ib,j)}
\right)^2
\right]
}
\end{equation}

\begin{figure}[h!t]
\centering
\includegraphics[width=\textwidth]
{imagens/ArCondicionado-CargaDemandaDinamica_ComTextoImpr.pdf}
\caption[Exemplo de aparelho C5: ar condicionado]
{Exemplo de aparelho \acs{c5}: ar condicionado. Também é possível
observar uma alteração gradual no nível de consumo, que irá dificultar
o processo de reconstrução de energia.}
\label{fig:ar_cond_dinamica}
\end{figure}

\subsection{Metodologia Genérica}
\label{ssec:metodologia_generica}

O \gls{nilm} pode ser resumido em quatro etapas para o
tratamento da informação na rede elétrica e em três abordagens quanto
às técnicas empregadas: 

\subsubsection[Etapas]{Etapas \cite{nilm_matthews_overview_2008_22}}
\label{top:etapas}

\begin{enumerate}[label={Etapa} \arabic* - ,ref=\arabic*,align=left]
\item\label{itm:etapa1} \textbf{\gls{fex}}: são extraídas
informações das amostragens realizadas. A diversidade de
características que podem ser extraídas depende da capacidade do
sistema de aquisição de dados, em especial no medidor, mas podendo
sofrer devido a outras limitações (como capacidade de transferência de
dados etc.). As características são utilizadas nas etapas seguintes, podendo
haver reaproveitamento. Em alguns casos, a Etapa~\ref{itm:etapa2}
utiliza o sinal sem processamento e a \gls{fex} é realizada somente
para a Etapa~\ref{itm:etapa3}, o que permite reduzir o esforço de
processamento;
\item\label{itm:etapa2}\textbf{Detecção de eventos de transitório}:
identificar alterações causadas por utensílios na rede.  Essa etapa é
necessária para identificar alterações no consumo de equipamentos.
Pode-se empregar limiares estáticos ou dinâmicos. Os limiares
dinâmicos permitem o ajuste de operação, reduzindo ou aumentando a
sensibilidade do detector conforme a presença de equipamentos
\acs{c5}. Em algumas topologias, a Etapa~\ref{itm:etapa3} é
realizada ciclo a ciclo. Nesses estudos, os autores se reservaram
apenas a discriminar os estados operativos dos aparelhos, restando
ainda determinar como seria realizado a estimativa do consumo dos
aparelhos. Uma alternativa seria realizar a Etapa~\ref{itm:etapa2} em
cima da informação fornecida pela Etapa~\ref{itm:etapa3}, de formar
que a ordem dessas etapas estaria invertida;
\item\label{itm:etapa3}\textbf{Reconhecimento de padrões}: utilizar as
características pertinentes para o reconhecimento de padrões,
deduzindo, assim, qual foi o utensílio que causou o disturbio na rede
e qual seu novo consumo. É desejável que o algoritmo seja capaz de identificar
a ocorrência de novos padrões e reconhecê-los em suas próximas 
aparições pois a construção de um catálogo com todos os possíveis
eletrodomésticos é impraticável, se não impossível. Tal tarefa só será
possível com a capacidade dos \glspl{nilm} de incluirem novos
equipamentos ao catálogo. Diversas técnicas podem ser utilizadas em
conjunto para esta etapa;
\item\label{itm:etapa4}\textbf{Refinamento dos resultados}: após as
Etapas~\ref{itm:etapa2} e/ou \ref{itm:etapa3}, pode-se adicionar uma
etapa opcional para procurar por possíveis erros ou melhorias na
informação desagregada. Por exemplo, corrigir a informação de um
aparelho que remanesce consumindo energia da rede por dias enquanto
sua operação normalmente ocorre em intervalos curtos. Isso pode
ocorrer por falhas na Etapa~\ref{itm:etapa2}, onde o desligamento do
equipamento não foi encontrado, ou na Etapa~\ref{itm:etapa3}, na qual
o desligamento foi identificado como causado por outro equipamento.
Outra possível melhoria seria encontrar possíveis novos ciclos de
operações para aparelhos~\gls{c2}. As estratégias corretivas podem ser
meramente remediativas, ou seja, simplesmente ignorar alterações de
estados que permanecem em no mesmo patamar de consumo durante um grande
período de tempo para melhorar a resolução em energia do \gls{nilm},
aplicarem técnicas complementares para reanalizar a informação
parcial, realizar uma otimização complementar utilizando a informação
obtida nas etapas anteriores, ou simplesmente realizarem uma nova
análise através da Abordagem~\ref{itm:abordagem2}.
\end{enumerate}

\subsubsection[Abordagens]{Abordagens \cite[com
adaptações]{nilm_liang_pt1_2010_34,nilm_zeifman_review_2011}}
\label{top:abordagens}

\begin{enumerate}[label={Abordagem} \arabic* - ,ref=\arabic*,align=left]
\item\label{itm:abordagem1}\textbf{Abordagem por reconhecimento de
padrões}\footnote{A referência \cite{nilm_zeifman_review_2011}
trabalha a diferença das abordagens em termo da quantidade de dados
abordados, onde a Abordagem~\ref{itm:abordagem1} trata a informação
uma a uma, ou seja, para cada evento de transição, enquanto a
Abordagem~\ref{itm:abordagem2} opera com todo o período sendo
otimizado, o que não é necessariamente verdade. Geralmente ambos os casos passam
por um período de otimização antes de serem empregados e, depois de
otimizados, são utilizados para a detecção dos padrões dos aparelhos
na rede. A distinção está que o primeiro otimiza a capacidade de
discernir os padrões --- sendo a reconstrução consequência disso ---,
enquanto o segundo a capacidade de reconstruir com maior fidelidade
possível o sinal original --- obtendo os padrões como
resultado.\label{fn:diff_abordagens}}: as técnicas de reconhecimento
de padrões são treinadas (otimizadas) em conjuntos de dados similares
aos quais eles irão operar. O reconhecimento de padrões pode ocorrer
apenas para as respostas da Etapa~\ref{itm:etapa2} ou para cada ciclo
da rede. Algumas técnicas utilizadas nessa abordagem podem ser
robustas aos aparelhos desconhecidos, sendo capaz de destacar seu padrão
dos outros já conhecidos e adicioná-lo ao catálogo de padrões. Assim,
quando esses padrões ocorrerem novamente, eles serão identificados
como o mesmo aparelho --- chamado de aprendizado em tempo real. Em
\cite{nilm_matthews_overview_2008_22} é observado a importância dessa
estratégia para tornar possível o crescimento do catálogo, que, tendo
o novo aparelho nomeado pelo consumidor, torna possível a criação de
um catálogo universal de equipamentos. Essa tarefa é impraticável, se
não impossível, de ser realizada em laboratório;
\item\label{itm:abordagem2}\textbf{Abordagem
por otimização}\fnref{fn:diff_abordagens}: concentra a capacidade de
suas técnicas na otimização, onde é realizada a procura por uma
combinação de aparelhos cujo o sinal agregado resultante é a melhor
aproximação do possível do sinal observado. Em alguns casos,
utiliza-se a concentração dos dados em longos períodos de tempo para
identificar o consumo desagregado, retornando a operação dos diversos
equipamentos no final do processo. Nesses casos, a informação final
pode ser utilizada como padrões a serem identificados posteriomente.
Para manter os equipamentos atualizados, novos períodos (possivelmente
menores ao período inicial) podem ser utilizados para garantir a
resposta adequada a possíveis alterações na presença ou utilização de
aparelhos. Na outra possibilidade, a otimização é realizada no nível
de ciclo da rede, onde, sabendo o padrão dos possíveis aparelhos
presentes, se busca a melhor combinação operativa que reflitam o sinal
observado;
\end{enumerate}

\subsection{Cálculo da Eficiência}
\label{ssec:nilm_eff_calc}

\subsubsection{Padronização}
\label{top:nilm_padrao}

O estudo bibliográfico realizado por \cite{nilm_zeifman_review_2011}
teve dificuldades ao tentar comparar as diferentes técnicas aplicadas
nos \glspl{nilm}. O primeiro empecilho está na variedade das base
de dados utilizadas, possuindo aparelhos e estados de operações
bastante distintos, criando condições que podem previlegiar a
eficiência de um determinado \gls{nilm}. Para a unificação dos dados
estudados e permitir a comparação de performance entre os algoritmos
empregados nos \glspl{nilm}, foi disponibilizado por
\cite{nilm_dataset_blued_2012} um conjunto de dados públicos para
a análise, o \acs{blued}. Um outro conjunto de dados público
utilizado nas referências é o \acs{redd}
\cite{nilm_dataset_redd_2011}, onde há dados gravados tanto em alta
frequência (15~k\acs{hz}) e baixa frequência (0,5~\acs{hz} e
1~\acs{hz}). Os conjuntos de dados foram construidos para
representar a realidade de residências nos \gls{eua} e por isso podem
não corresponder a realidade brasileira. Apesar disso, o conjunto
de dados podem servir como base para comparação da perfomance dos
\glspl{nilm} aqui desenvolvidos com os do exterior, assim como nada
impede o emprego em paralelo de dados próprios mais representativos
para demonstrar a realidade de aplicação e capacidade da abordagem
utilizada.

Outra dificuldade foi o fato de autores utilizarem uma
medida própria para o cálculo das taxas de eficiência.  Além disso,
normalmente os autores não reportaram as taxas de falsos positivos na
Etapa~\ref{itm:etapa2}, apenas a capacidade dos algoritmos de
detectarem os eventos (excessões observadas são
\cite{nilm_marceau_16ssamp_improved_1999_18,nilm_liang_pt2_2010_40}).
Em outros casos, os autores concentraram-se apenas na capacidade dos
algoritmos da Etapa~\ref{itm:etapa3} de discriminarem equipamentos,
reportando medidas representativas para essa eficiência.

Por isso, \cite{nilm_zeifman_review_2011} recomenda a utilização das
medidas apresentadas por \cite{nilm_liang_pt1_2010_34}, no qual se
apresentou considerações metódicas para o tema. Foram apresentadas
três medidas. A primeira medida, \gls{det_eff}, considera a capacidade
do \gls{nilm} de desagregar a informação nos eventos que foram
detectados\footnote{Empregada por \cite{nilm_hart_1992_8} quando não
disponível a medição paralela de energia dos utensílios e por
\cite{nilm_gupta_patel_2010_30}.}. Quando interessado apenas em
estudar a capacidade do classificador para os eventos detectados, a
medida \gls{class_eff} deve ser utilizada. Finalmente, a
\gls{total_eff} é dada por \ref{eq:total_eff}, levando em conta apenas
a capacidade do \gls{nilm} de corretamente classificar os eventos
reais, causados pelos equipamentos na rede\footnote{Empregada por
\cite{nilm_patel_2007_29,nilm_berges_2009_36} pois a
Etapa~\ref{itm:etapa2} foi realizada manualmente
\label{fn:patel_manual} e, geralmente, por demais estudos que
estudaram apenas a Etapa~\ref{itm:etapa3}.}.

\begin{subequations}\label{eq:eff}
\begin{equation}\label{eq:det_eff}
\eta_{det} = \frac{N_{id}}{N_{real} + N_{fp} - N_{ni}}
\end{equation}
\begin{equation}\label{eq:class_eff}
\eta_{class} = \frac{N_{id}}{N_{real} - N_{ni}}
\end{equation}
\begin{equation}\label{eq:total_eff}
\eta_{total} = \frac{N_{id}}{N_{real}}
\end{equation}
\end{subequations}

\noindent onde:  

\begin{description}
\item[$N_{id}$] são eventos identificados, ou seja, corretamente
detectados e classificados pelo \gls{nilm}; 
\item[$N_{real}$] é o número de eventos realmente causados pelos
equipamentos na rede;
\item[$N_{fp}$] são eventos devido a falsos positivos, ou seja,
evento erroneamente identificados;
\item[$N_{ni}$] são eventos não identificados, ou perdas de alvo.
\end{description}

Segmenta-se \ref{eq:total_eff} para obter a eficiência do \gls{nilm}
por aparelho conforme:

\begin{equation}\label{eq:app_eff}
\eta_{total}^i\approx\frac{N_{id}^i}{N_{real}^i} ~~ \forall ~~ i =
1,2,...,N_{ap}
\end{equation}

\noindent onde $N_{id}^i$ e $N_{real}^i$ são os respectivos
$N_{id}$ e $N_{real}$ para o i-ésimo aparelho dos $N_{ap}$
disponíveis.

O grande favorecimento para essas medidas é sua simplicidade de serem
obtidas, porém algumas considerações podem ser feitas sobre elas.
Primeiro, a medida com maior sensibilidade à capacidade do \gls{nilm}
é a \ref{eq:det_eff}, uma vez que os valores por ela representados
levam em conta as perdas de alvo e os falsos positivos.
Segundo, as mesmas não levam em conta o consumo de energia dos
aparelhos, dando importância análoga para aparelhos com parcelas
pequenas ou grandes de consumo. Além disso, a correta identificação
dos $N_{id}$ não significa que a energia será corretamente
reconstruída, dependendo da capacidade do \gls{nilm} de unir essas
informações para gerar a informação do consumo desagregado.  Ainda,
como apontado por \cite{nilm_zeifman_review_2011}, elas apenas
representam a eficiência no ponto de operação, não sendo possível
observar como o \gls{nilm} se portaria para outros pontos. Indo além,
elas não permitem comparações de técnicas utilizadas exclusivamente para
as Etapas~\ref{itm:etapa2} e \ref{itm:etapa3}, impedindo a
contraposição de técnicas onde os autores se limitaram a uma dessas
etapas.

\subsubsection{Outras representações}
\label{top:outras_eff}

Por isso, além das medidas apontadas, outras maneiras de representar
a eficiência podem ser utilizadas para complementar o estudo do
comportamendo da abordagem utilizada. Uma técnica para representar o
compromisso entre a capacidade de detectar eventos e a quantidade de
falsos positivos encontrados é a curva \gls{roc}, também recomendada
por \cite{nilm_zeifman_review_2011}. A \gls{roc} além de ser utilizada
para expressar de maneira geral a capacidade do algoritmo de detectar
e identificar em função dos falsos positivos, pode ser utilizada para
estudar a eficiência específica da Etapa~\ref{itm:etapa2}. Já para a
Etapa~\ref{itm:etapa3}, a matriz de confusão permite entender quais
aparelhos ou classes de aparelhos são confundidos em outras classes,
assim como a eficiência de classificação em uma única representação.

As outras medidas utilizadas na literatura levantada são: o
percentual de classificações corretas por ciclo da rede, ou seja, a
correta classificação do estado de operação para cada ciclo dividido
pelo número total de ciclos
\cite{nilm_srinivasan_nn_2006_27,nilm_suzuki_2011_35}; porcentagem de
detecção de transitórios \cite{nilm_patel_2007_29}\footnote{O estudo
reportou eficiência para as Etapas~\ref{itm:etapa2} e \ref{itm:etapa3}
separadamente. Como foi dito na nota \ref{fn:patel_manual}, a
Etapa~\ref{itm:etapa3} utilizou eventos recortados manualmente.};
\gls{p_eff_i}
\cite{nilm_hart_1992_8,nilm_cole_data_extraction_1998_14,
nilm_cole_extra_info_surge_1998_15,nilm_farinaccio_16ssamp_1999_17,
nilm_marceau_16ssamp_improved_1999_18}; \gls{p_eff} 
\cite{2010_nilm_melhorando_pph_usa_37}; desvio
do tempo em que o aparelho foi identificado operando em relação ao
tempo que ele realmente estava operando
\cite{nilm_farinaccio_16ssamp_1999_17}\footnote{O estudo focou na
identificação de grandes cargas elétricas como ar condicionado e
aquecedores de água, modelados por \gls{c3}\label{fn:valc3}. Essa
medida seria limitada para outros modelos.}; porcentagem de detecção
de eventos de transição para ligado perdidos
\cite{nilm_farinaccio_16ssamp_1999_17}\fnref{fn:valc3}; erro médio
absoluto de reconstrução de energia e outras estatísticas por aparelho
\cite{nilm_powers_15minsamp_1991_16}.

A medida mais utilizada pelas referências, a \gls{p_eff_i}, embora
por elas não definida matematicamente, concebe-se que seja dada por:

\begin{subequations}
\begin{equation}\label{eq:frac_en_app}
\rho_{En}^i = \frac{E_{det}^i}{E_{real}^i} ~~ \forall ~~ 
i = 1,2,...,N_{ap}
\end{equation}
\begin{equation}\label{eq:frac_en}
\rho_{En} = \frac{\sum_{i}^{N_{ap}}E_{id}^i}{\sum_{i}^{N_{ap}}E_{real}^i} 
\end{equation}
\end{subequations}

\noindent onde $E_{det}^i$, $E_{real}^i$ é o consumo detectado e
consumo real do i-ésimo aparelho, respectivamente, o último sendo
obtido por submedição ou por um estimador. A \gls{p_eff_i}
pode ser generalizada para calcular a \gls{p_eff} através de
\ref{eq:frac_en}.  Essas medidas levam em consideração o consumo
detectado pelo \gls{nilm}, mas perdem a capacidade das medidas 
\ref{eq:eff} de representar a informação que foi corretamente
identificada. Por exemplo, se um equipamento é considerado como ligado
em um espaço de tempo em que o mesmo está desligado, isso irá
contribuir para corrigir possíveis erros que seriam atribuidos quando
o estado estimado e a operação estiverem na lógica oposta.

Assim, fica evidente que essas medidas precisam ser refinadas para
identificar os momentos nos quais a energia foi corretamente
reconstruída. Para isso, aqui se sugere o uso de \ref{eq:e_id_i} com
o intuíto de determinar a \gls{e_id_i}. A ideia
é representar que identificações do aparelho em outros estados, mas
com pequena diferença de energia, não irão afetar tanto na resolução
de energia, assim como resguardar que identificações em estados de
consumo maiores para os quais os aparelhos realmente operam não
arremeterão na conta de energia corretamente identificada.

\begin{equation}\label{eq:e_id_i}
E_{id}^i = E_{det}^i-\varepsilon^i
\end{equation}

A \gls{en_res} representa a ideia, em energia, para tanto falsos
positivos ou quanto identificações errôneas para estados de maior
consumo, ou seja, a parcela de $E_{det}^i$ que excede àquela obtida
através de um submedidor --- medição diretamente da energia do
equipamento --- ou estimada $E_{real}^i$. Ela pode ser descrita
por:

\begin{equation}\label{eq:en_res}
\varepsilon^i = \left\{\begin{array}{rl}
 E_{det}^i - E_{real}^i &\mbox{ se $E_{det}^i>E_{real}^i$} \\
 0 &\mbox{o.c.}
\end{array} \right. ~~ \forall ~~ i = 1,2,...,N_{ap}
\end{equation}

Isto posto, para obter a \gls{en_eff_i} e sua generalização,
\gls{en_eff}, basta empregar:

\begin{subequations}
\begin{equation}\label{eq:en_eff_i}
\eta_{En}^i = \frac{E_{id}^i}{E_{real}^i} ~~ \forall ~~ i =
1,2,...,N_{ap}
\end{equation}
\begin{equation}\label{eq:en_eff}
\eta_{En} = \frac{\sum_{i}^{N_{ap}}E_{id}^i}{\sum_{i}^{N_{ap}}E_{real}^i}
\end{equation}
\end{subequations}

E para as taxas de redundância:

\begin{subequations}
\begin{equation}\label{eq:p_red_i}
\rho_{red}^i = \frac{\varepsilon^i}{E_{real}^i} ~~ \forall ~~ i =
1,2,...,N_{ap}
\end{equation}
\begin{equation}\label{eq:p_red}
\rho_{red} = \frac{\sum_{i}^{N_{ap}}\varepsilon^i}{\sum_{i}^{N_{ap}}E_{real}^i}
\end{equation}
\end{subequations}

Posteriormente, descobriu-se que o próprio autor de
\cite{nilm_zeifman_review_2011} criou uma medida que, depois de
modificada por \cite{seminilm_fhmm_empiricalnmeter_2013}, permite
também exprimir a questão de precisão de reconstrução de energia, bem
como reflete, de outra maneira, a quantidade de energia redudante. 
A alteração de \cite{seminilm_fhmm_empiricalnmeter_2013} será referida
como \glslink{medidafenergia}{medida-eF (\acs{medidafenergia})}.
Ela pode ser descrita pelo quadrado da média geométrica normalizado
pela média aritmética (\ref{eq:fmeasure_en}) de duas grandezas: 

\begin{itemize}
\item parcela de energia atribuida ao aparelho que foi realmente
consumida pelo mesmo em relação ao seu consumo pelo \gls{nilm}
atribuído, um parâmetro que também reflete a ideia de energia
redundante, descrita por \ref{eq:en_recon};
\item parcela de energia que foi corretamente identificada em relação
ao consumo total do aparelho, a própria \gls{en_eff_i} aqui descrita
em \ref{eq:en_eff_i}.
\end{itemize}

\begin{equation}\label{eq:en_recon}
\eta_{En,prec}^i = \frac{E_{id}^i}{E_{det}^i}
\end{equation}

\begin{equation}\label{eq:fmeasure_en}
F_{en}^i=\frac{2 \;\; \eta_{En,prec}^i \;\; \eta_{En}^i}{\eta_{En,prec}^i+\eta_{En}^i}
\end{equation}

Entretanto, a \glslink{medidaf}{medida-F} (\ref{eq:fmeasure}) original
apresentada por \citet*{nilm_zeifman_vastext_approach_2012}, como
observado pelo próprio autor, não é ótima como medida de desagregação,
por não levar em consideração erros na detecção de energia. Ela fica
aqui referenciada por ter sido utilizada em diversos estudos desse
autor.

\begin{equation}\label{eq:fmeasure}
F^i=2\dfrac{\left(\dfrac{N_{id}}{N_{id}+N_{fp}}\right)\eta_{total}}
{\left(\dfrac{N_{id}}{N_{id}+N_{fp}}\right)+\eta_{total}}
\end{equation}


\subsection{Técnicas Aplicadas}
\label{ssec:nilm_tecnicas}

As abordagens aplicadas desde o início dos estudos ao tema e
utilizadas como referências fizeram mão de ostensivas técnicas para a
desagregação do consumo. Cada vertente buscou extrair características
ou inovar aplicando outras técnicas, de forma que é possível observar
uma grande diversidade de abordagens. As abordagens serão agrupadas em
relação à \gls{fex} realizada. A capacidade de extrair
características dos sinais é correlacionada com a frequência de
amostragem e, em vista disso, dividir-se-ão os métodos aplicados de
acordo com a taxa de amostragem utilizada. A ideia de subdivisão aqui
seguida foi de autoria da referência \cite{nilm_zeifman_review_2011}.

Em seguida, um outro tópico especifico para as topologias onde há
semi-intrusão da residência, utilizando submedição em pequenas
escalas, tanto energeticamente, quanto correlacionando a utilização de
energia com outros medidores, como temperatura, movimento, som etc.

\subsubsection{1. Medição com Baixa Amostragem}
\label{top:nilm_baixa_am}

A utilização de características mascroscópicas de consumo do aparelho,
como alterações no patamar de consumo da rede, foi a
primeira abordagem encontrada ao tema. As mesmas podem ser obtidas sem
grande granularidade na taxa de amostragem, por isso, esse tipo de
abordagem beneficia-se de medidores de baixo custo, amplamente
disponíveis no mercado. No entanto,
\cite{nilm_berges_2008_7,nilm_matthews_overview_2008_22} alertam
para discrepâncias entre medidores na ordem de 10\%-20\%, bem maiores
que aquelas alegadas, de 3\%. Os medidores testados no caso foram
\emph{Brand Meter I}, \emph{Watts Up? PRO} e \emph{EnerSure}.

A taxa de amostragem mais frequentemente utilizada é 1 Hz, entretanto
alguns estudos fizeram mão taxa de amostragem ainda menores por
desejarem identificar aparelhos que se ressaltam dentre os outros
devido ao seu relativo alto consumo, como ar condicionado, aquecedores
de água e geladeira. Exemplos de medidores utilizados no exterior são
\gls{ted} \cite{ted_site} e \emph{Watts up? PRO} \cite{wattsup_site},
o último sendo capaz de informar o consumo de \acl{q}.

\begin{enumerate}[label=\textbf{1.\arabic*},wide=\parindent]
\item \textbf{\Acrlong{p} e \Acrlong{q}}
\label{nilm:pot_real_reat}

\indent A referência inicial de grande destaque no tema,
\citet*{nilm_hart_1992_8}\footnote{\citet*{nilm_sultanem_1991_10}
publicou seu estudo em \citeyear{nilm_sultanem_1991_10}, antes de 
\citeauthor*{nilm_hart_1992_8} que o publicou em
\citeyear{nilm_hart_1992_8}. Aparentemente, nenhum dos autores teve
conhecimento do trabalho do outro. Apesar do trabalho de
\citeauthor*{nilm_sultanem_1991_10} ter sido publicado primeiro,
\citeauthor*{nilm_hart_1992_8} retrata em seu trabalho que sua descoberta
ocorreu em 1982, tendo diversos trabalhos realizados pelo seu grupo em
relação ao tema desde essa data, resultando em um material bem mais
extenso que o de \citeauthor*{nilm_sultanem_1991_10}.}, ocorreu em
1992. Nela, utilizam-se medições de
\gls{p} e \gls{q} com uma taxa de amostragem de 1 Hz. A abordagem
aplica uma normalização para reduzir flutuações no consumo devido a
alterações na tensão de acordo com \ref{eq:norm_hart} com o intuíto de
reduzir disperções nos dados. O estudo de \citeauthor*{nilm_hart_1992_8}
limitou-se a identificar apenas cargas com potência maior a 150
\acs{watt}. A essência da metodologia ainda pode ser encontrada
em \glspl{nilm} mais atuais, sendo esta: 

\begin{equation} \label{eq:norm_hart}
P_{\text{norm}}(t) = \left[ \frac{120}{V(t)} \right]^2 P(t)
\end{equation}

\begin{enumerate}[label=\arabic*]
\item Detectam-se transitórios de consumo na rede devido a mudança de
estado de um utensílio através de alterações no consumo que devem
superar um limiar específico (15 \acs{watt}/\acs{var}) para os sinais
normalizados como em \ref{eq:norm_hart} para \acs{p} e \acs{q}.
As amostragens dentro de um regime permanente são normalizadas para
sua média com o objetivo de tirar o ruído. Para a \gls{fex}, utiliza-se o
degrau entre o regime permanente posterior e anterior (já no valor de
suas médias) ao evento transitório para \gls{p} e \gls{q};
\item Os eventos de transitório são analizados por um algoritmo de
agrupamento que irá gerar os centróides das mudanças de estado
possíveis causadas pelos utensílios no plano
\acs{dp}$\times$\acs{dq};
% FIXME Tem que ver se ele cria C2, ou na verdade procura combinações
% para gerar C3.
\item \textlabel{Centróides com simétria em relação aos eixos são tomados em
pares e com eles são criados modelos \gls{c3}}{text:passo3}. Para os
centróides remanescentes, além de regras heurísticas como a junção de
centróides próximos que permitam o pareamento com um outro refletido
nos eixos --- etapa conhecida como resolução de conflitos ---,
determinam-se possíveis combinações de centróides que possam formar
uma \gls{c2} utilizando uma adaptação do algoritmo de \emph{Viterbi}
\cite{nilm_bouloutas_viterbi_ext_1991_11,
nilm_hart_fsm_viterbi_1993_12}\footnote{A versão original pode ser
encontrada em \cite{viterbi_alg}.}. Assim que é determinada uma combinação
que permite a criação de uma \gls{c2}, os centróides da mesma são
removidos, e o processo continua até que todas as \glspl{fsm} tenham
sido construídas. A adaptação utilizada
\cite{nilm_bouloutas_viterbi_ext_1991_11,
nilm_hart_fsm_viterbi_1993_12} permite várias operações para consertar
corrupções e retornar uma estimativa ótima das alterações de estados
operativos da \gls{fsm} original. Como a reconstrução depende da
estatística do processo, é necessário que as mudanças de estado das
\glspl{fsm} observadas tenham um comportamento para que a \gls{fsm}
original seja reconstruída, e por isso, restringe-se apenas às
\glspl{c2a}. As \glspl{c2b} podem ser reconstruídas se houver
conhecimento prévio da presença das mesmas, de modo que elas sejam
medidas operando em cada um de seus estados e então inseridas
manualmente no catálogo do \gls{nilm};
\item Em seguida é levantado o comportamento dos equipamentos,
montando o estados de consumo para cada aparelho. É utilizado um
algoritmo de força-bruta para corrigir ocorrências de dois ligamentos
ou desligamentos de um mesmo aparelho encontrados em seguida. A causa
desses erros é, geralmente, a ocorrência de um evento simultâneo de
dois equipamentos. Assim, o algoritmo busca por eventos não-usuais
cuja soma é o valor de dois outros eventos perdidos;
\item Por fim, é levantada a estatística detalhando o
comportamento de consumo, como o tempo ligado e desligado de cada
equipamento. Essa informação, junto com a potência do equipamento é
utilizada para auxiliar a identificar o equipamento. 
\end{enumerate}

O método é robusto para o desagregação de cargas \glspl{c3}
($> 150~$\acs{watt}) e as adaptações \cite{nilm_bouloutas_viterbi_ext_1991_11,
nilm_hart_fsm_viterbi_1993_12} do algoritmo de \emph{Viterbi}
parecem resolver o problema das \glspl{c2a}. Outro problema das
metodologias envolvendo algoritmos de agrupamento é a lenta alteração
da resitência conforme a operação do aparelho. Geralmente, ao
interromper a operação, o aparelho tem alterações no consumo na margem
de 5\%-10\% em relação ao inicio de operação
\cite{nilm_sultanem_1991_10}. \citeauthor*{nilm_hart_1992_8} observa
o degrau geralmente é menor em valor absoluto para os desligamentos
nos casos de equipamentos com motores, que reduzem o consumo conforme
seu aquecimento.

\textlabel{Uma estratégia bastante parecida é realizada por}{text:cole}
\citet*{nilm_cole_data_extraction_1998_14,
nilm_cole_extra_info_surge_1998_15}, 
onde são feitas considerações em relação as características de bordas e
inclinações. Aqueles são definidos como o auge atingido de potência
durante o acionamento e estes variações lentas de mudança no consumo.
Apesar de definir as bordas como o pico de potência, as referências
empregam as bordas apenas como os eventos de transição de consumo, não
empregando essa informação para classificação. A abordagem aplicada,
ao invés de agrupar os dados para depois procurar por possíveis
aparelhos como feito por \citeauthor*{nilm_hart_1992_8}, primeiro
busca temporalmente por ciclos fechados nos eventos de transição (ou
bordas, como na nomenclatura da referência), que depois serão
adicionadas aos centróides no espaço
\acs{dp}$\times$\acs{dq}. Se o centróide não existir, será
criado um candidato a centróide. Conforme a quantidade de ciclos dos
centróides aumenta, o mesmo irá se tornar um candidato a uma carga.
Para as cargas \glspl{c3}, a carga será aceita apenas se a detecção
das bordas ocorrerem repetidamente. Já para as cargas \glspl{c2a}, foi
realizado um estudo da probabilidade dela ter sido originada pela
sobreposição de duas bordas geradas por equipamentos distintos. A
conclusão foi que se forem encontrados mais de um ciclo de três bordas
em um período de 6 horas é suficiente para aceitá-lo como uma
\gls{c2a}. Finalmente, o envelope só foi considerado para a melhoria
em resolução de energia e, no entanto, a referência indica que a
utilização da média de consumo entre as bordas apresenta melhores
resoluções.

\item \textbf{\Acrlong{p}, \Acrlong{q} e Transitório}
\label{nilm:pot_real_trans}

Um trabalho paralelo ao de \citeauthor*{nilm_hart_1992_8} foi
realizado pelo mesmo instituto para aplicar o \gls{nilm} no setor
comercial e industrial, sendo realizado por
\citet*{nilm_norford_leeb_medianfilt_1996_13,
nilm_leeb_spectral_envelope_1995_23}.
No setor comercial são encontrados aparelhos com características
diferentes ao setor residencial, geralmente com transitórios mais
lentos (podendo chegar a cerca de centenas de segundos
\cite{nilm_norford_leeb_medianfilt_1996_13}), menor consumo reativo
devido às preocupações com a qualidade de energia e consequentemente
correção do fator de potência, e a presença de \gls{c5}, como exemplo,
na referência foi observada uma bomba com picos periódicos de 20
k\acs{watt}. Tipicamente há também uma maior presença de
equipamentos com cargas variáveis, \glspl{c4}, como motores de
velocidade variável. 

Assim, foi adicionado a informação do transitório
da envoltória em amostragens de 1~\acs{hz}, que por serem
maiores, ainda podem ser observados em amostragens baixas, suprindo,
ao mesmo tempo, a menor capacidade de discriminação da variável
\gls{q} nesse setor. Para redução dos ruídos, utilizou-se um filtro de
mediana com 11 pontos, esse sendo mais indicado para a eliminação dos
picos quando comparado aos filtros lineares, que terão dificuldades de
distinguir os picos e os degraus, uma vez que eles tem espectros de
frequência parecidos. Os trabalhos utilizam o conceito de sessões-v
(tradução própria de \emph{v-sections}), que são subdivisões do
transitório em segmentos de variação significante. Para isso, o grupo
utilizou uma janela para identificar mudanças na média e criar as
sessões-v, onde os valores ótimos são determinados por um processo de
treinamento pré-instalação ou em laboratório. É aplicada uma medida de
distância entre as sessões-v observadas e os transitórios
característicos que, ao estarem dentro de um limiar, serão
identificados como um determinado equipamento. Para o tratamento das
\glspl{c4}, a referência indica o emprego de variáveis de controle,
quando disponíveis, correlacionadas com o seu consumo para estimá-las,
como o caso para os equipamentos de \gls{avac} em geral.

\item \textbf{Unicamente \acl{p}}
\label{nilm:pot_real}

A medição de \acl{q} adiciona custo ao \gls{nilm} --- ainda que
não tão oneroso quanto medições em altas frequências --- e, para
detectar certos aparelhos com assinaturas de destaque na rede,
essa variável pode ser desnecessária. Em outros casos, medidores que
disponibilizam essa informação podem não estar disponíveis, sendo
possível operar apenas com a \acl{p}.

\begin{enumerate}[label*=.\textbf{\arabic*},wide=\parindent]
\item \textbf{Separação dos principais equipamentos por uso-final}

Exemplos do primeiro caso são os estudos de
\citet*{nilm_powers_15minsamp_1991_16,nilm_farinaccio_16ssamp_1999_17,
nilm_marceau_16ssamp_improved_1999_18},
para os quais os autores se preocuparam em identificar apenas
equipamentos de maior uso-final.
\cite{nilm_powers_15minsamp_1991_16,nilm_farinaccio_16ssamp_1999_17}
utilizaram somente regras heurísticas, enquanto
\cite{nilm_marceau_16ssamp_improved_1999_18} também empregou os
degraus em potencia real e um filtro para a detecção dos
ligamentos/desligamentos dos aparelhos.

Em \cite{nilm_powers_15minsamp_1991_16}, foram reportadas a capacidade
de reconstrução para ar condicionado e aquecedores de água. A
amostragem é realizada a cada 15 minutos e os arquivos são analisados
dia a dia. Por ser proprietário, as regras não são detalhadas (é
utilizada uma árvore de decisões, embora o estudo considere a aplicação
de redes neurais), mas o algoritmo procura por picos no consumo, assim
como sua duração, tempo e magnetude, que são utilizados pelas regras
para determinar se os mesmos foram utilizados para os usos-finais
cobiçados. Posteriomente, eles são ajustados conforme
verificações de consistência. Para o ar condicionado, é relatado que o
valor de pico estimado médio para as residências difere cerca de
apenas 5\% do valor original médio, enquanto o consumo fica na margem
de 10\% e observa-se boa capacidade de estimar os horários de consumo.
 
Já os estudos \cite{nilm_farinaccio_16ssamp_1999_17,
nilm_marceau_16ssamp_improved_1999_18}, realizados por outro
grupo, empregaram amostragem de \acs{p} a cada 16~segundos. Os
aparelhos estudados foram: geladeira, aquecedor de água e aquecedores
de ambiente (este somente em
\cite{nilm_marceau_16ssamp_improved_1999_18}\footnote{O algoritmo da
referência \cite{nilm_marceau_16ssamp_improved_1999_18}
também leva em consideração a máquina de lavar roupa, mas os
resultados focaram apenas nos outros três aparelhos.}).
\cite{nilm_zeifman_review_2011} expõe a arbitrariedade e não
intuitividade das regras utilizadas em
\cite{nilm_farinaccio_16ssamp_1999_17}, que precisam ser estudadas
para cada caso de aparelho. Foram determinadas 8 regras para cada
aparelho (algumas regras são reaproveitadas entre aparelhos),
divididas em duas etapas: determinar o conjunto de eventos de
transição e a duração do consumo.  Em seguida, a duração de consumo é
multiplicada pela demanda média do aparelho durante a fase de
treinamento para obter o consumo estimado. A fase de treinamento,
período em que há medição paralela dos equipamentos, e, por isso,
ocorrendo intrusão da propriedade do consumidor, é feita para um
período de uma semana. A reconstrução de energia diária para os
equipamentos é na margem de $-10,5\%$ a $15,9\%$.

O estudo em sequêcia aperfeiçoou o anterior com uma abordagem única
para determinar os aparelhos em operação. Ele compara, em ordem
decrescente em termos de demanda média operativa, se a magnetude
do evento é próxima à média do nível operativo de um dos aparelhos
almejados, empregando como limiar de corte dois desvios padrões. Ainda
assim, a referência emprega diversas regras de pré/pós-processamento
determinadas empiricamente para melhorar a resolução em energia, assim
como também necessita do período de treinamento através de medição
paralela de 1~semana, limitando a aplicabilidade do método para uma
gama maior de equipamentos. Por outro lado, o método serve para o seu
próposito, obtendo reconstruções na faixa de 10\% para a maioria das
análises realizadas.

\item \textbf{\Acrlong{q} não disponível}

\begin{itemize}[wide=\parindent]
\item \emph{A abordagem de 
\citeauthor*{nilm_baranski_genetic_base_2003_19}}

Com o intuíto de possibilitar a impregnação da aplicação de \glspl{nilm}
na Alemanha, \citet*{nilm_baranski_genetic_base_2003_19,
nilm_baranski_genetic_detalhado_2004_20,nilm_baranski_summary_2004_21}
recorreram a leitura ótica dos medidores eletromecânicos
(o trabalho \cite{nilm_baranski_genetic_base_2003_19}, realizado em
2003, indica que mais de 99\% dos medidores desse país possuem essa
configuração) para obter as medições com frequência de
1~\acs{hz}. Por isso, apenas \gls{p} estava disponível para esses
estudos. 

Apesar das limitações, essa abordagem é uma referência de destaque
devido às diversas contribuições feitas. Para melhorar a capacidade de
discriminação entre os equipamentos, além da potência ativa, o estudo
adicionou como característica o pico de consumo para o evento de
transição, bem como o período que o mesmo leva para estabilizar
(apresentado em \cite{nilm_baranski_genetic_detalhado_2004_20}),
diferente de \cite{nilm_cole_data_extraction_1998_14,
nilm_cole_extra_info_surge_1998_15} que observou essas propriedades,
mas empregou somente a última com o intuíto de melhorar a
capacidade de reconstrução de energia. Ainda, o autor contribuiu
transformando a estratégia aplicada por
\citeauthor*{nilm_hart_1992_8} em uma Abordagem~\ref{itm:abordagem2}.

Tratar-se-ão de detalhes das técnicas aplicadas pelos autores por dois
motivos:

\begin{itemize}
\item a técnica teve bons resultados apesar de utilização de pouca
informação, podendo ter melhores resultados quando alimentada com mais
informação e, por isso, sendo um possível caminho a ser percorrido;
\item outros autores \cite{nilm_bergman_distribuido_2011,
nilm_zeifman_vast_2011,nilm_zeifman_vast_hisample_pdfmerge_2011,
nilm_zeifman_vastext_approach_2012,
nilm_zeifman_statistical_vastext_1stws_2012,
nilm_zeifman_statistical_naive_enduses_2013} se basearam nessas
ideias;
\item os artigos não são de compreensão trivial, em especial para a
elucidação da adaptação do algoritmo de \emph{Viterbi}\footnote{Para os
leitores que desejarem se aprofundar, recomenda-se a leitura de
\cite{nilm_bergman_distribuido_2011} antes dos artigos de
\citeauthor*{nilm_baranski_genetic_detalhado_2004_20}.}. 
\end{itemize}

Os leitores não interessados em detalhes técnicos podem
seguir para a próxima abordagem mantendo em mente que as contribuições
das técnicas foram: 

\begin{enumerate}
\item criação das \gls{fsm} por algoritmos genéticos
para reduzir o tempo de otimização; 
\item em seguida essas são otimizadas por \gls{es} para obter os
parâmetros da \gls{fsm} (tempo e consumo em cada estado); 
\item e finalmente, os modelos utilizam lógica \emph{fuzzy}, permitindo que
dois ou mais modelos sejam criados para um mesmo distúrbio na rede,
sendo depois escolhido o modelo que melhor se aplica.
\end{enumerate}

A estratégia começa com o agrupamento dos dados em centróides,
limitando-se a degraus acima de um limiar mínimo de potência (valores
aplicados de 50 \acs{watt} em
\cite{nilm_baranski_genetic_base_2003_19} e 80 \acs{watt}
\cite{nilm_baranski_genetic_detalhado_2004_20}). As abordagens em
\cite{nilm_baranski_genetic_base_2003_19,
nilm_baranski_genetic_detalhado_2004_20} se basearam no agrupamento
utilizando lógica \emph{fuzzy}, mas na última referência
\cite{nilm_baranski_summary_2004_21}, além desse método, cita-se o
emprego de \gls{som} para essa etapa. A fim de reduzir a complexidade
do problema (a estimativa de eventos é de $16.000$ por dia), o autor
desconsidera os centróides com poucas ocorrências, limitando-se a
identificar apenas aparelhos com padrões recorrentes. 

Na primeira abordagem \cite{nilm_baranski_genetic_base_2003_19},
\citeauthor*{nilm_baranski_genetic_base_2003_19} segmentaram a etapa
de modelar os aparelhos. A primeira modela as \glspl{c3} simplesmente
encontrando pares de centróides no espaço. Para validar os modelos
\glspl{c3} encontrados, é gerado a matriz de correlação cruzada
utilizando o estado de operação para os aparelhos em cada instante de
tempo. Se o aparelho $i$ e o aparelho $j$ forem na realidade uma
\gls{fsm}, espera-se $r_{ij}\approx1$, onde $r_{ij}$ indica a
frequência de operação do aparelho $j$ quando $i$ está operando,
associando, assim, $j$ com a operação $i$ (o corte utilizado é de
0,8). Já para as \gls{c2}, são criados todos os modelos que
juntos somam aproximadamente zero e seus centróides tem frequência de
eventos também próximos. Esses modelos são então validados
temporalmente, e junto com as \glspl{c3} são comparados com um
possível catálogo antigo a fim de atualizá-lo. Em seguida, uma rede
neural é treinada com os padrões encontrados para os aparelhos (o
autor cita como exemplo: tempo médio de consumo, consumo médio, número
de estados) para encontrar esses padrões na residência.

Essa abordagem é aprimorada em
\cite{nilm_baranski_genetic_detalhado_2004_20,nilm_baranski_summary_2004_21},
que ao invés de encontrar todos possíveis modelos de \gls{fsm},
faz a otimização das possíveis máquinas através de algoritmo genético.
Nessa abordagem não há a discriminação para a criação de \gls{c2} ou
\gls{c3}, a abordagem única utiliza $N_{ap}$ (o número de aparelhos
deve ser maior que o número de centróides, no entanto, não é
especificado um bom valor a ser utilizado) \glspl{fsm} para os quais e o
algoritmo genético fica encarregado de alterar valores binários em uma
matriz $\mathbf{X}$ representando se um determinado centróide
pertence, ou não, à \gls{fsm}. É possível que um mesmo centróide
pertença a mais de uma \gls{fsm}. São utilizados três critérios para
otimização: 

\begin{itemize}
\item minimização do valor absoluto de potência da soma dos
centróides pertencentes a \gls{fsm}; 
\item o item anterior, mas levando em conta a frequência de
eventos em cada centróide; 
\item e minimização do número de centróides em
cada \gls{fsm} (priorizando aparelhos com menos estados).
\end{itemize}

Uma adaptação do algoritmo de \emph{Viterbi} é utilizada para encontrar os
modelos de \gls{fsm}. Com os modelos resultantes, é criado as
sequências de estado para elas supondo que as mesmas são \gls{c2a}.
Mais precisamente, os autores consideram que as sequências de estados
devem ser recorrentes com seus parâmetros em uma área limitada dentro
de seu valor esperado (os autores citam dois exemplos de parâmetros:
tempo de duração no determinado estado e a capacidade de reconstrução
de energia para o consumo estimado nos estados da \gls{fsm} em relação
ao consumido nos caminhos percorridos; mas não dá detalhes de quais
empregou) e apenas visitados uma vez em cada ciclo. Para isso, é
realizada uma otimização em dois tempos. Primeiro, encontra-se o
melhor caminho para aqueles que obedecem as restrições (consumo de
potência positiva e permanencia em um estado por um tempo não muito
longo), juntando os estados da máquina em um caminho de operação com a
melhor qualidade em relação aos parâmetros escolhidos. Os parâmetros
podem ser iniciados com os valores da mediana para todos os eventos
acoplados a \gls{fsm}. A qualidade é avaliada pela entropia de
\emph{Shannon}, \ref{eq:shannon}. Isso é repetido iterativamente
utilizando \gls{es} que irá alterar os parâmetros até a convergência
da qualidade.

\begin{equation}\label{eq:shannon}
Q_{shannon} = - \Delta{e_{i}} \log{|\Delta{e_{i}}|}
\end{equation}

Os melhores caminhos operativos para as \glspl{fsm} ainda precisam ser
resolvidos quanto aos centróides que pertencem a mais de um aparelho.
Para isso, \cite{nilm_baranski_summary_2004_21} cita resumidamente um
algoritmo de força bruta que irá investigar para cada sobreposição
qual caminho tem a melhor qualidade.

Os autores revelam que o método necessita de 5 a 10 dias para
encontrar os modelos dos utensílios típicos, enquanto dados diários
são suficientes para atualizar o catálogo de utensílios detectados em
cada residência. Os resultados mostram que os aparelhos de maiores
consumo, como geladeira, aquecedor elétrico (de fluxo) e fogão podem
ser detectados com eficiência.

\item \emph{\gls{dnilm}}

A abordagem de \citeauthor*{nilm_baranski_summary_2004_21} é a base
empregada para o trabalho de \citet*{nilm_bergman_distribuido_2011},
contando com a mesma sequência de criação através de algoritmo
genético e otimização das \gls{fsm}. Diferenças podem ser notadas
apenas para a técnica de agrupamento, que não utilizam as informações
de tempo nem o pico atingido no transitório, contudo, podem utilizar a
\gls{q} se o medidor da residência realizar essa medida. Além disso, o
agrupamento é realizado em tempo real, atualizando a média e desvio
padrão de cada centróide conforme os eventos ocorrem. Se o evento não
for atribuído a nenhum centróide, um novo centróide é criado. Ainda, o
método utiliza outra otimização para fazer a identificação do aparelho
que alterou o estado operativo, um algoritmo de otimização de mochila
(do inglês \emph{Knapsack problem}), enquanto a abordagem anterior
citou a utilização de redes neurais na referência
\cite{nilm_baranski_genetic_base_2003_19}\footnote{
\citeauthor*{nilm_baranski_genetic_base_2003_19}
praticamente não levaram o assunto em consideração, a descrição dos
artigos desses autores gira em torno do tema de modelagem dos
aparelhos, mas tirando a informação aqui citada, não há outra
informação sobre como se tratou o reconhecimento dos estados
operativos através dos modelos.}.

Entretanto a maior contribuição do trabalho
em questão é uma nova arquitetura, distruibuída, para o \gls{nilm}.
Nesse caso, diferente dos medidores eletromecânicos disponíveis para
\citeauthor*{nilm_baranski_summary_2004_21}, o trabalho opera com
medidores inteligentes (os medidores das redes inteligentes descritos
na Subsessão~\ref{ssec:ret_tec}) para a coleta de
dados. Os medidores inteligentes também servem como pontos de
processamento local, mas devido às limitações de processamento, apenas
a deteção e identificação dos eventos é realizado no mesmo. Por isso,
a geração das \glspl{fsm} é realizada em uma central, com maior
capacidade de processamento, no qual este envia os eventos de
transição para que aquela os processe e retorne um catálogo com as
\glspl{fsm} e seus padrões a serem encontrados. O catálogo é chamado
de tabela estática. Assim, o medidor fica encarregado apenas de
comparar, localmente, os distúrbios encontrados com o catálogo,
identificando assim os estados operativos dos aparelhos. O estados
operativos de cada aparelho é chamado de tabela dinâmica, e é
preenchida por um algoritmo adaptado para otimização do problema da
mochila. A etapa de criação da
tabela estática, chamada de aprendizagem, é realizada devido a
critérios do controlador (iniciação pró-ativa) ou do medidor
(reativamente). No primeiro caso, o controlador atualiza as tabelas do
medidor se as mesmas expirarem. Já o medidor inteligente pode
requisitar um novo treinamento de acordo com um dos critérios:

\begin{itemize}
\item a diferença absoluta entre a soma da demanda real e estimada
está superior a um patamar;
\item uma \gls{fsm} muda de estado frequentemente, onde os patamares
aplicados para determinar se a mudança de estado é frequente dependem
do aparelho (é mais aceitável observar mudanças frequentes no ar
condicionado ou aquecedor do que em um carregador de bateria
veicular);
\item mais de um determinado número de \glspl{fsm} alteram de estado
em um único evento.
\end{itemize}

Uma das dificuldades do projeto está em ajustar o fluxo de dados. A
referência considera armazenar os dados em períodos de maior atividade
nas residências, enviando as alterações de estado posteriormente
conforme a rede de comunicação não estiver congestionada. Diversas
outras considerações são feitas em relação ao processo de
aprendizagem.

Os resultados reportados são em comparação com um \gls{nilm}
centralizado. O trabalho reservou-se a detectar aparelhos com consumo
superior a 1000 \acs{watt}. O \gls{nilm} centralizado recebe uma
tabela estática otimizada para todo o período, enquanto o \gls{dnilm}
recebe uma tabela treinada para o primeiro dia, podendo atualizá-la de
acordo com o critérios anteriormente citados. As diferenças de
acurácia entre o \gls{dnilm} e o \gls{nilm} centralizado ficaram entre
60\% e 90\%.

\item \emph{Otimização dos estados operativos}

O problema para a otimização da tabela dinâmica --- construção
temporal dos estados operativos dos aparelhos --- foi abordado por
\citet*{nilm_genetic_2013} (aparentemente sem conhecimento do trabalho
de \citeauthor{nilm_bergman_distribuido_2011}). Ao invés de um
algoritmo de força bruta adaptado, aplicou-se um algoritmo genético
para realizar a otimização do problema da mochila. O artigo limitou-se
a estudar a performance do algoritmo em resolver o problema da mochila
e, por isso, considerou-se que são conhecidos os momentos de transição
e os consumos de cada aparelho.

A referência utilizou simulações de 2 horas, gerando aleatoriamente o
tempo de operação dos aparelhos e suas potências. Foram simuladas
diversas condições, variando o número de aparelhos, transições, a
presença de ruído e de equipamentos desconhecidos. As observações
foram: 

\begin{itemize}
\item quanto maior o \gls{nt} e \gls{nap}, menor é a eficiência de
detecção. Casos com pequenos números de \gls{nt} e \gls{nap} obteram
100\% de eficiência de detecção;
\item o algoritmo se comportou bem na presença de ruído, onde houve
pouca deterioração na eficiência de detecção;
\item no entanto, a presença de aparelhos desconhecidos ou aparelhos
com sobreposição de potências deterioram bruscamente a performance do
algoritmo.
\end{itemize}

\end{itemize}

\end{enumerate}

\item \textbf{\Acrlong{p} e estatística de uso}
\label{top:nilm_p_estatistica}

Também com o objetivo de possibilitar a impregnação do método, mas no
caso para os \gls{eua},
\citeauthor*{nilm_zeifman_vastext_approach_2012} propôs um novo método
\cite{nilm_zeifman_vast_2011,
nilm_zeifman_vast_hisample_pdfmerge_2011,
nilm_zeifman_vastext_approach_2012,
nilm_zeifman_statistical_vastext_1stws_2012,
nilm_zeifman_statistical_naive_enduses_2013} para abordar o tema em
cima de mostradores de energia domiciliares. Esses mostradores
normalmente disponibilizam apenas a \acl{p}, com uma taxa de
amostragem na faixa de 1~\acs{hz}. A fim de melhorar a capacidade
de desagregação do consumo que é limitada pela pouca informação
oferecida pela sistema de aquisição de dados, o estudo empregou o
conhecimento prévio de uso dos equipamentos através de uma abordagem
estatística para melhorar a capacidade de seu \gls{nilm}. O autor
observa que os estudos \cite{nilm_farinaccio_16ssamp_1999_17,
nilm_marceau_16ssamp_improved_1999_18} também utilizaram a estatística
de uso, servindo, provavelmente, de inspiração para o seu trabalho.
Será tratado da abordagem desse autor com detalhes pois essas
referências parecem ser as mais próximas de possibilidar o \gls{nilm}
em um programa de \gls{ee} em larga escala.

Devido à ausência da \acl{q}, a sobreposição dos centróides ocorrerá
com frequência maior, aumentando as ocorrências de erros decorrentes
de \gls{c6}. O autor criou o método \gls{vast} para empregar a informação
estatística no processo de discriminação, Etapa~\ref{itm:etapa3}.
Diferente das outras adaptações do algoritmo de \emph{Viterbi} que se
encontram no levantamento bibliográfico
\cite{nilm_bouloutas_viterbi_ext_1991_11,
nilm_hart_fsm_viterbi_1993_12,nilm_baranski_genetic_base_2003_19,
nilm_baranski_genetic_detalhado_2004_20,nilm_baranski_summary_2004_21},
onde o objetivo é modelar as \glspl{fsm} presentes na rede, a ideia é
modelar os estados operativos dos equipamentos como sendo dependentes,
além de seu próprio estado operativo, no estado operativo de outros
equipamentos. Indo além, criar essa dependência para auxiliar aonde a
informação é precária, ou seja, quando há sobreposição de potência.  O
estudo observa que, normalmente, isso só ocorre para trios de
equipamentos, ou seja, supondo que os aparelhos da residência estejam
distribuídos em ordem crescente de consumo, o aparelho $i$ só deverá
ter sobreposição de consumo com o aparelho $i-1$ e $i+1$. Assim, ao
invés do algoritmo de \emph{Viterbi} de força bruta que tornaria a
aplicação inviável conforme o número de aparelhos (ou estados
operativos) aumentasse, já que todos os estados dos outros aparelhos
podem influenciar no aparelho $i$, somente será utilizado os estados
dos aparelhos $i-1$ e $i+1$ para determinar o seu estado operativo.
Com isso, reduz-se a complexidade da versão original de
\emph{Viterbi}, que é exponencial conforme o número de estados
possíveis, para uma complexidade linear.

%Um exemplo dado pelo autor, supondo que para o trio de aparelhos com
%sobreposição, os aparelhos $i-1$ e $i+1$ tipicamente permanecem
%desligados durante 1 hora, enquanto essa duração para $i$ é de 10
%horas. Quando o aparelho $i$ está desligado, a probabilidade de um
%evento de transição positiva corresponder ao aparelho $i$ sendo ligado
%irá depender, quando apenas levando em consideração o tempo, se os
%outros dois aparelhos estão ligados ou desligados, e neste caso, por
%quanto tempo cada aparelho está desligado.

Para tornar mais simples a tarefa, considera-se que os aparelhos $i-1$
e $i+1$ também serão independente entre si, sendo necessário, assim,
determinar para cada aparelho apenas dois pares de cadeias de
\emph{Markov} (exclusive o aparelho de menor e maior consumo, que
terão apenas uma cadeia): $\{i-1,i\}$ e $\{i,i+1\}$. O preenchimento das
matrizes com as probabilidades de transição é feita com a informação
obtida \emph{a priori}, aplicando as estatísticas de tempo ligado e
desligado, distribuições dos momentos em que os aparelhos trocam de
estado operativo no dia e as distribuições dos eventos de transições
para cada aparelho. Outras características podem ser utilizadas, tais
como: as distribuições para as informações de tempo e amplitude para o
pico de ligamentos (influênciado por
\cite{nilm_baranski_genetic_base_2003_19,
nilm_baranski_genetic_detalhado_2004_20,nilm_baranski_summary_2004_21})
e características ``finas'' específicas de cada aparelho (o autor deu
exemplos em \cite{nilm_zeifman_statistical_naive_enduses_2013} e serão
citados mais adiante neste tópico). Essas probabilidades podem ainda
ser melhoradas considerando distribuições para períodos diurnos ou
noturnos e sazonalidade.

Como o ciclo de observações sofre de alterações como falta
de transições (no caso de perda de alvo para um evento de transição)
ou a presença de intrusos (tanto falso positivo, ou quanto a presença
da transição de um evento é causado na cadeia $\{i-1,i\}$ é causado
pelo um aparelho $i+1$, por exemplo), é necessário que a matriz de
probabilidade considere mudanças não esperadas para garantir a solução
do problema. Por exemplo, no caso de um intruso, é necessário dar a
possibilidade dos aparelhos continuarem no mesmo estado reconhecendo
sua presença, enquanto também é necessário dar a possibilidade de um aparelho
$i$, que estava desligado, modificar seu estado para ligado, enquanto
o evento de transição foi de desligamento, caso que ocorreria
excepcionalmente para um evento de transição perdido de ligamento para
o aparelho $i$ e o evento sendo julgado ser intruso. Obviamente essas
probabilidades serão pequenas, sendo utilizadas constantes para
preencher essas possibilidades. Essas constantes podem utilizar taxas
esperadas de perdas de alvo, falso positivos ou intrusão por outro
aparelho, entretanto a determinação dos valores parece estar deixada
em aberto.

O algoritmo calcula, então, a probabilidade máxima de observar um
estado após cada evento de transição para todas observações das
cadeias. Ao chegar no final delas, é escolhido a configuração com
maior probabilidade e então se rastreia a sequência de transições no
sentido oposto até chegar no estado inicial dos sistemas (algoritmo de
\emph{Viterbi}\footnote{A última versão do código utilizou um modelo
adaptado para a cadeia de \emph{Markov} de segunda ordem.}). A fusão
para as duas cadeias de \emph{Markov} de um mesmo aparelho $i$ é feita
utilizando o conceito de máxima verossimilhança, empregando a
configuração de maior probabilidade final para esse aparelho.

Para descobrir os aparelhos e suas estatísticas, o algoritmo realiza a
coleta de dados por um longo período de tempo (ex. duas semanas) e,
em seguida, realiza o agrupamento dos dados por centróides (foi
utilizado \acs{isodata}, mas o próprio autor pretende empregar
outras técnicas). Para corrigir possíveis problemas com os centróides,
são feitas operações de separação e agrupamento dos mesmos de acordo
com regras que indicam a necessidade das mesmas. Os centróides finais
positivos e negativos são casados com a mesma abordagem utilizada por
\citeauthor*{nilm_hart_1992_8} (ver o passo~3 desse autor na
p.~\pageref{text:passo3}, limitando-se a parte para \gls{c3}). A
montagem das \glslink{cdf}{cdfs} com as características de tempo é
realizada através do modelo empírico --- constrói a \acs{cdf}
através dos dados amostrados, sem ajuste ---, enquanto para as
\glslink{pdf}{pdfs} de tempo e potência, ajusta-se a distribuição
através da mistura de duas componentes de distribuições de
Laplace\footnote{A primeira versão realizou o ajuste com duas
componentes Gaussianas.}.

Por fim, após o processo de levantamento de estatística, o
algoritmo é aplicado em tempo real, limitando a janela de dados para um
tamanho aceitável, citando como exemplo 1~dia. O tamanho de janela
pode ser ajustado experimentalmente. A base de dados utilizadas para a
estatística também é atualizada conforme mais dados são recolhidos.
Assim, o método não necessita de treinamento, identificando os modelos
de carga e suas propriedades. O autor considera depois comparar a
informação encontrada com as estatísticas de uso e consumo típicas
para identificar o aparelho, mas havendo comunicação com o usuário, é
possível que o próprio realize a identificação, dando a comparação
apenas como sugestão. Outra consideração é trabalhar com trios de
cargas ao invés de duplas, já que alguns casos essa configuração
parece ser mais realista devido à sobreposição de vizinhos aquém
daqueles imediatamente subsequentes, reduzindo a presença de intrusos
nas cadeias de \emph{Markov}.

O autor empregou dados simulados e reais para provar a eficácia de seu
algoritmo comparando sua capacidade com um classificador \emph{Naïve
Bayes}, obtendo resultados em termos de \acs{medidaf}
(\ref{eq:fmeasure}) na ordem de 0,83 a 0,97 em relação à medição
paralela, em geral superiores ao classificador de \emph{Bayes}. Nos
dados simulados, o autor demonstra que as outras abordagens
normalmente empregadas para 1~\acs{hz} são deficientes para as
configurações de sobreposição de consumo, enquanto sua versão consegue
reconstrução praticamente perfeita. Entretanto, atualmente o algoritmo
só atende aparelhos modelados por \gls{c3}.

Já no estudo mais recente
\cite{nilm_zeifman_statistical_naive_enduses_2013}, o autor não
empregou o \gls{vast}. Como nas abordagens \cite{nilm_farinaccio_16ssamp_1999_17,
nilm_marceau_16ssamp_improved_1999_18}, limitou-se a identificar os
aparelhos de maior uso-final. No caso, os aparelhos de interesse foram:
\begin{enumerate*}[label=\itshape\alph*\upshape)]
\item ar condicionado;
\item aquecedores de ambiente;
\item aquecimento de água doméstica;
\item \label{itm:iluminacao} iluminação;
\item geladeira;
\item secadores de roupa elétricos;
\item \label{itm:aparelhoeletronico} aparelhos eletrônicos;
\end{enumerate*} que representam em média 80\% do consumo residencial,
no caso, para os \gls{eua}. Nesse configuração, utilizou-se o conceito
de máxima entropia para selecionar modelos estatísticos mais adequados
para os dados. Uma vez que a média e alcance das variáveis são conhecidos,
as distribuições Beta são as mais indicadas por esse conceito. Foram
feitas considerações em relação a iluminação, que tem dependência em
relação aos ambientes de uso. Ainda, para melhorar a performance,
adiciona-se como característica as assinaturas específicas dos
aparelhos. O exemplo é a televisão em comparação com uma lâmpada, onde
aquele varia o seu consumo conforme flutuações na imagem e som,
enquanto este tem o consumo bastante estável. Essas
características ``finas'' dos aparelhos podem ser modeladas
matematicamente e empregadas em conjunto com o conhecimento prévio de
utilização dos aparelhos. Já para o caso dos secadores de roupa
elétricos (uma \gls{fsm}), foi utilizado duas distribuições Beta, uma
para cada estado.

Os autores utilizaram o classificador \emph{Naïve Bayes} com base nas
distribuições conjuntas resultantes obtidas para os sete uso-finais
indicados, adicionados de uma classe ``outros''.  Os resultados
empregaram a \acs{medidaf}, obtendo valores de acurária para os
uso-finais mais desafiantes na ordem de 0,65 e 0,70, sendo os mesmos
os itens \ref{itm:iluminacao} e \ref{itm:aparelhoeletronico},
respectivamente. Já quando utilizando a informação de características
finais, essas mesmas acurácias se elevam para 0,92 e 0,90. Para os
outros aparelhos, os resultados ficaram na margem de 0,9, atingindo a
melhor marca para os aquecedores de água doméstica: 0,99.

\end{enumerate}

\subsubsection{2. Medição com Alta Amostragem}
\label{top:nilm_alta_am}

A abordagem para os \glspl{nilm} operando com aquisição de dados
obtidos com alta amostragem possibilita a obtenção de características
possivelmente mais discriminantes devido à alta resolução. A maior
granularidade na amostragem permite o acesso aos harmônicos e formas
de onda do sinal, chamadas de características microscópicas. Valhe
ressaltar aqui que a utilização de taxas de amostragem alta no
sistema de aquisição não implica na extração de características
microscópicas. Por exemplo, ao extrair a diferença de consumo de
potência ativa entre o pré e pós transitório com 100 M\acs{hz},
isso não define essa características como microscópica porque ela
ainda poderia ser obtida com menor granularidade.
 
Devido à maior amostragem, é inerente que os autores, em muitos casos,
buscaram empregar técnicas de processamento de sinais para o
tratamento dos dados, onde a bem conhecida \gls{fft} foi a primeira
abordagem. Outras abordagens vão além da \gls{fft}, procurando outras
maneiras de representar a informação, em alguns casos trabalhando até
mesmo com a informação crua, sem tratamento. A criatividade dos
autores leva a tona uma gama de características para descrever o sinal
e buscar através de diversas técnicas as assinataras dos aparelhos. As
características e técnicas empregadas pelos diversos autores para essa
amostragem é tratada a seguir.

Antes disso, considerações quanto aos requerimentos mínimos de
amostragem deve ser realizada \cite{nilm_matthews_overview_2008_22}.
Quando explorando as características no domínio da frequência, o
teorema de \emph{Nyquist-Shannon} formula que a frequência de
amostragem deve ser no mínimo duas vezes superior à frequência do
maior harmônico a ser estudado. Já no domínio do tempo, uma regra de
boa-prática formula que a amostragem deve ser no mínimo de cinco vezes
a frequência de interesse (no caso, o mínimo para representar a
fundamental no Brasil é utilizando amostragem em 300~\acs{hz}).
Entretanto, é preciso ter em mente que certos distúrbios só poderão
ser observados em frequências maiores, em especial a caracterização
dos regimes transitórios, sendo necessários nesses casos frequências
bem maiores. As frequências utilizadas pelos autores variaram nas
ordens de $\sim$100\acs{hz} até M\acs{hz}.

\begin{enumerate}[label=\textbf{2.\arabic*},wide=\parindent]
\item \textbf{Harmônicos e \gls{fft}}
\label{nilm:harmonic_fft}

A decomposição através de \gls{fft} fornece informação além daquelas
obtidas por \gls{p} e \gls{q}, cuja única informação presente é das
componentes fundamentais e sua defasagem. A informação presente na
decomposição auxilia na discriminação de cargas não-lineares,
separando as mesmas entre si e das cargas lineares. Por outro lado, é
importante notar que a decomposição não fornece mais informação
discriminante além da informação já contida em \gls{p} e \gls{q} para
desagregar cargas lineares.

\begin{itemize}[wide=\parindent]
\item \emph{Harmônicos para identificação de \acs{c4}}

O grupo que trabalhou com o intuíto de levar o \gls{nilm} para o setor
comercial e industrial \cite{nilm_norford_leeb_medianfilt_1996_13,
nilm_leeb_spectral_envelope_1995_23} utilizava amostragem em
frequências superiores a 8~k\acs{hz} para construir o envelope
espectral do sinal \cite{nilm_laughman_continuous_variables_2003_9}. O
grupo se referia dessa maneira ao envelope pois ele era construído
através da resposta da \gls{fft}. Como foi dito durante o levante
desses trabalhos, o grupo subdividia os transitórios em sessões-v que
eram utilizadas para classificar os utensílios. 

Uma vez que a decomposição harmônica já estava sendo extraída, os
trabalhos posteriores procuraram empregar essa informação para tratar
um problema presente em maior escala nesses setores, as \glspl{c4}.
Os trabalhos de \citet{nilm_lee_variable_speed_estimation_2005_24,
nilm_wichakool_2009_25,nilm_shaw_2008_26} avaliaram como identificar
possíveis maneiras de reconstruir o consumo de cargas variáveis
em cenários previamente conhecidos. Um dos cenários incluiu inclusive
a aplicação do \gls{nilm} em um carro \cite{nilm_shaw_2008_26}.
Foi observado nas \glspl{c4} estudadas a presença de correlação entre
sua demanda e os harmônicos de potência aparente. Assim, é ajustado
uma função polinômial utilizando \gls{mse} para determinar a sua
mudança de consumo. A função polinômial é utilizada e comparada com as
envelope espectral das sessões-v para determinar se a mudança foi
causada por alterações no estado operativo da \gls{c4}.

Apesar do método se comportar bem para as aplicações propostas,
há a necessidade de fazer o levantamento caso a caso, sendo
interessante apenas para a aplicação no setor comercial e industrial
de grande porte, onde dificilmente será possível aplicar um \gls{nilm}
genérico devido as diversas peculiaridades das cargas presentes. Além
disso, não se sabe como a presença de outros aparelhos irão afetar a
correlação, podendo limitar sua aplicação.

\item \emph{Extração harmônica ciclo a ciclo}
\label{nilm:harmonico_ciclo_ciclo}

A abordagem de \citet{nilm_srinivasan_nn_2006_27} alterou a estratégia
do \gls{nilm} ao não realizar a extração dos eventos de transitório.
Ao invés disso, ele realiza diretamente para cada ciclo a desagregação
da informação de consumo por aparelho. 

Para isso, utilizou-se técnicas de aprendizado
de máquina (\glspl{rna} com as arquiteturas \acs{mlp} e
\acs{rbf}, além de \acs{svm}) treinadas com a informação das
componentes harmônicas impares até o 15$^o$ harmônico. O caso
analisado continha oito aparelhos. Em cada possível combinação
operativa desses aparelhos ($2^{N_{ap}}=256$), aquisitou-se dados para
treinar e testar as técnicas aplicadas. Nessa configuração, o método
obteve reconstrução praticamente perfeita para a \gls{mlp} e
\gls{rbf}. Em seguida, para melhorar a aplicabilidade dos métodos,
utilizou-se as amostras coletadas com apenas um dos aparelhos ligados
por vez que foram, então, somadas para reproduzir os outros estados
operativos. Nessas condições, houve deterioração para quatro
equipamentos para taxas em torno de 70\%, enquanto os outros aparelhos
continuaram com taxas em próximas a 100\%, tanto para a \gls{mlp} e
\gls{rbf} (com algumas discrepâncias entre as duas). Também se testou
condições com 10 aparelhos. Para esse caso, como seriam necessárias
medições em 1024 situações, utilizou-se o método de obter apenas as
operações de cada aparelho sozinho, e somá-las para obter as outras
configurações. Outras configurações testadas incluiram adição de
ruído e teste para rede com topologia trifásica, onde se apresentou
que o método é robusto para essas configurações. A \gls{svm} não teve
respostas compatíveis com as da \gls{mlp} e \gls{rbf} para as diversas
configurações testadas.

Todavia, conforme o número de aparelhos cresce, a quantidade de
combinações necessárias para treinar a rede cresce exponencialmente.
Uma residência com cerca de 30 a 50 aparelhos (valores típicos
citados por \cite{nilm_zeifman_review_2011}) necessitariam do
treinamento para números na ordem de $10^7$ a $10^{15}$ configurações
--- isso sem considerar possíveis estados que serão adicionados por
aparelhos \glspl{c2}  ---, onde possivelmente diversas dessas configurações
seriam similares e, por isso, o método não conseguiria reproduzir as
capacidades para os exemplos citados. Além disso, é necessário saber
\emph{a priori} todos aparelhos pertencentes na residencia, assim como
o autor não levantou como as respostas seriam afetadas devido à
presença de aparelhos não-conhecidos. \textlabel{Mais adiante, também
é necessário estimar o consumo por 
aparelho}{text:transf_info_discr_energia}, assunto deixado em aberto
pelo autor.  Para isso, é possível realizar a identificação de
transitórios em cima da informação de discriminação, como considerando
uma janela para a qual se o aparelho mudar de estado operativo por um
tempo maior àquele da janela, então se confirma a mudança operativa e
calcula o consumo desse aparelho no novo estágio através da diferença
pós e pré-transitório. Possíveis adaptações podem ser feitas para
obter melhor resolução em energia. Uma abordagem mais elegante seria
realizar as duas etapas em paralelo e comparar as informações obtidas
em cada uma para validar se houve realmente a mudança de estado. Já
com o intuíto de tornar essa tarefa trivial, pode-se evitar a
utilização da Etapa~\ref{itm:etapa2} e simplesmente empregar um valor
de consumo estimado para cada equipamento.

\item \emph{Ruído espectral de tensão}
\label{nilm:emi}

\citet*{nilm_patel_2007_29,nilm_gupta_patel_2010_30} encontraram
características com capacidade discriminantes \emph{sui generis}. A
metodologia utilizada permite diferenciar aparelhos iguais mas em
diferentes cômodos. De acordo com os autores, em alguns casos é
possível identificar qual interruptor de uma lâmpada \emph{three-way}
teve seu estado alterado. Para isso, ao invés de utilizar o consumo
agregado, os autores mudaram o conceito de como realizar a aquisição
de dados. No caso, seu sistema pode ser conectado em qualquer tomada
da residência, aonde serão analisados a \gls{emi} causado por
chaveamentos, sendo necessário medir somente a \emph{tensão}. 

As características internas do aparelho, bem como a configuração da
rede elétrica entre o aparelho e o medidor, alteram em como o ruído
será observado. Os ruídos podem ser gerados continuamente ou apenas
durante os eventos de transitório. Exemplos de equipamentos que geram
ruídos contínuos são: equipamentos com funcionamentos a base de motores
(como secadores de cabelo, ventiladores) que irão gerar ruídos de
tensão em frequências síncronas àquelas da rede causados pelo
chaveamento das escovas dos motores (os aparelhos desse tipo que não
tiverem escova, não vão produzir \emph{emi} continuamente); e os
aparelhos eletrônicos com fontes baseadas a chaveamento, nesse caso
gerando ruído sincrono ao oscilador de sua fonte. Já eventos de ruído
\gls{emi} podem ser observados ao colocar um aparelho na tomada,
alterar o estado de um interruptor ou ligar uma televisão.

Na primeira abordagem \cite{nilm_patel_2007_29}, o objetivo foi
encontrar os ruídos transitórios de chaveamentos na rede. São gerados
2048 pontos igualmente espaçados com os resultados da decomposição da
\gls{fft} (apenas magnetude) no alcance de 0 a 50 k\gls{hz}. Essa
informação é repassada para uma janela deslisante que cálcula a média
para 1 $\mu$s de dados. Com os resultados das janelas, é tirada a
distância Euclidiana em relação a janela anterior, e se esse valor
superar um limiear considera o início de um transitório. O transitório
será registrado até a distância Euclidiana ter outra mudança drástica.
A resolução de 1 $\mu$s é importante pois alguns transitórios tem
duração de apenas alguns poucos $\mu$s. Tendo o transitório isolado,
obtém-se a média para os 2048 pontos em relação ao número de amostras
temporais. Essa informação alimenta uma \acs{svm}. A capacidade
de isolamento de transitórios (taxa de detecção apenas para a
Etapa~\ref{itm:etapa2}) obtida foi entre 88\% a 98\%, embora a taxa de
falsos positivos não tenha sido informada. Os acertos para a
quando considerando a Etapa~\ref{itm:etapa3} ficaram entre 85-90\%
quando a \acs{svm} é treinada com 5 instâncias para cada evento.
Caso seja coletado apenas 2 instâncias por evento, há uma deterioração
para cerca de 80\%.

Já a segunda abordagem tratou da abordagem para ruídos gerados
continuamente na rede \cite{nilm_gupta_patel_2010_30}. Os autores
perceberam que os transitórios contínuos não são dependentes da
configuração da rede, onde os equipamentos terão as mesmas
características independente das residenciais e tomadas em questão.
Nesse caso, o espectro análisado de \emph{Fourier} é de
36-500k\acs{hz} para a mesma quantidade de pontos igualmente
espaçados. A mesma ideia de janela móvel e corte através de um patamar
é realizada para essa abordagem, no entanto, a frequência de geração
de vetores é de 244 a cada segundo, sendo utilizados 25 deles para
calcular a média da janela deslisante (a frequência resultante é bem
menor que aquela utilizada para estudar eventos de transitório). 
A extração de características é realizada em cima da diferença entre o
espectro do ruído atual e anterior, aonde são ajustados uma Gaussiana.
As características são a média, amplitude e variância, que são
comparadas utilizando a técnica de vizinho mais próximo. A acurária
reportada é de 91,75\%.

Apesar das boas capacidades, ainda há uma série de dúvidas quanto a
capacidade de aplicar essa metodologia nos \glspl{nilm}. Em
\cite{nilm_gupta_patel_2010_30}, os autores levantam uma série de
limitações quanto à abordagem por detecção de transitórios, devido ao
custo computacional necessário para avaliar ruído transitório, a
necessidade de treinar para cada eletrodoméstico, a dependência das
características na configuração da rede elétrica do consumidor e os
eventos tem deposição de energia pequena devido a seu longo espectro
de frequência. Percebe-se que a grande sensibilidade das
características presentes nos transitórios traz um problema inverso
àqueles carregados pelas \glspl{c6}: ao simplesmente trocar
o aparelho de tomada altera como o mesmo é visto pelo \gls{nilm}.
Como levantado por \cite{nilm_zeifman_review_2011}, a segunda
abordagem é mais robusta, no entanto, diversas questões ficam em
aberto. Ela é limitada para aparelhos que tenham chaveamento, por
exemplo, fornos e secadores de roupas elétricos não puderam ser
identificados. Os autores não consideraram como a presença de
\gls{emi} nos vizinhos, em especial para o caso de apartamentos, irá
interferir no processo de detecção. Na construção do catálogo, é
necessário considerar todas as mudanças de fontes chaveadas que forem
realizadas pelos fabricantes, assim como as diferenças entre as fontes
utilizadas por eles. A sobreposição de assinaturas \gls{emi} também
irá ocorrer, sendo necessário tratar esse problema. Enfim, os
autores apenas se preocuparam em identificar os aparelhos, não sendo
obtido o consumo desagregado dos mesmos.

\item \emph{Fourier como função de ajuste para Vizinho mais Próximo}

Os trabalhos de \cite{nilm_berges_2008_7,nilm_berges_2009_36,
2010_nilm_melhorando_pph_usa_37} buscaram apenas reproduzir as
eficácias dos \glspl{nilm} presentes na literatura, não tendo como
objetivo apresentar melhorarias para as técnicas presentes nos
\gls{nilm}, mas apenas fornecer essa tecnologia para o consumidor
ter acesso em baixo custo para uma ferramenta que o permita
compreender suas despesas com energia. O projeto começou desde o
sistema de aquisição de dados, onde se observou problemas com
medidores disponíveis no mercado estadunidense que apresentavam
discrepâncias de 10-20\% entre si, enquanto a taxa reportada era de
3\%. O seu sistema de aquisição de dados atual coleta amostras na
frequência de 100 k\acs{hz} utilizados para gerar características
em 20 \gls{hz}. Devido às dificuldades para rotular os momentos
exatos para o treinamento das técnicas, eles criaram um detector
\emph{wireless} de eventos de transitório, diretamente colocado no
cabo do aparelho.

Ele utiliza um modelo probabilístico para a detecção de eventos
alimentando um discriminador de vizinho mais próximo utilizando função
de ajuste como a \gls{fft} na sua melhor configuração. As eficiências
de detecção obtida estão em torno de 67\% e 100\%. A discrepância em
energia do \gls{nilm} para o valor total de energia foi de 14,8\%.

\end{itemize}

\item \textbf{\gls{tw}}
\label{nilm:wavelet}

Uma das diferenças mais marcantes entre a \gls{fft} e a \gls{tw} é a
possibilidade que a última oferece de resolução tanto de frequência
como temporal. Para isso, a \gls{tw} reduz o tamanho da janela
conforme há o aumento das frequências para permitir resolução
temporal. A decomposição é realizada em funções chamadas
\emph{wavelets} ao invés de senóides. A \gls{tw} tem diversas
aplicações em Sistemas de Potências, o levantamento realizado por
\cite{wavelet_overview} em 2002 mostra que as aplicações mais comuns
eram proteção, qualidade de energia e detecção de transitório. A
larga aplicação da \gls{tw} em detecção de transitório em sistema de
potência mostra que elas podem ser utilizadas para, no mínimo,
realizar a Etapa~\ref{itm:etapa2} do \gls{nilm}, sendo um substituto
potencial para as métricas simples adicionados de um corte linear
comumente aplicadas pelos \glspl{nilm} nessa etapa.

Indo além, o levantamento de \cite{nilm_zeifman_review_2011} encontrou
um estudo (\citet*{nilm_chan_2000_31}) realizado em
\citeyear{nilm_chan_2000_31} que mostrou a capacidade
de aplicação da \gls{tw} para a identificação de cargas não-lineares.
Entretanto, esse estudo não se aprofundou no tema, apenas levantando a
capacidade da \gls{tw} de diferenciar três aparelhos, seguindo uma
ideia similar àquela explicada para
\citeauthor{nilm_srinivasan_nn_2006_27}\footnote{No caso, o estudo de
\citeauthor{nilm_chan_2000_31} foi realizado depois.}. Para isso, 
explorou-se os níveis de detalhes da \gls{tw} discreta para todas as
combinações operativas dos três aparelhos, salientando que essa
informação era discriminante e seria possível aplicar \acs{rna}
para discriminar essa informação, ainda que isso não tenha sido
realizado.

Um trabalho no Brasil, realizado por \citet*{nilm_itajuba_rodrigues},
seguiu exatamente essa metodologia. A taxa de amostragem empregada foi
de 15.360 k\acs{hz}, onde se testou duas configurações, uma aonde
os aparelhos elétricos operavam isolados na rede e outra com as
diversas combinações operativas. Os testes foram feitos com três e
quatro aparelhos e obteram capacidade de identificação de 100\%. Em
seguida, o trabalho mostrou a importância da \gls{tw} quando em
comparação com a \gls{fft}, onde a \gls{fft} obteve 90\% mesmo
utilizando 16 componentes do espectro (linearmente espaçado) de
frequência, enquanto a \gls{tw} foi decomposta para 7 níveis de
detalhes. Entretanto, essa abordagem sofre das mesmas desvantagens
referidas para o trabalho de \citeauthor*{nilm_srinivasan_nn_2006_27}
(ver p.~\pageref{nilm:harmonico_ciclo_ciclo}).

Apesar de no exterior ter sido encontrado apenas um estudo preliminar
mostrando a possibilidade de utilizar \gls{tw}
\cite{nilm_chan_2000_31}, um outro trabalho brasileiro também seguiu o
caminho do emprego de \gls{tw} para a identificação de aparelhos. O
trabalho de \citet{nilm_coppe_nascimento} utilizou uma amostragem em
256~\acs{hz}. É empregado um detector de eventos onde se calcula
o incremento de corrente em relação a amostra passada, no caso de o
valor ultrapassar o patamar de 32 m\acs{a}, é criado um evento de
transitório com 8 ciclos e um total de 2048 pontos a serem analisados.

Uma pré-classificação é realizada utilizando o \emph{hardware} para
equipamentos com valores específicos de \gls{dp} e \gls{dq}. Caso não
seja um dos equipamentos de simples discriminação, modifica-se a
informação em uma série de etapas, nesta ordem: aplica-se Transformada
de \emph{Hilbert} no módulo da corrente, \gls{tw} com cinco níveis e
por último o Método de Burg, esse adicionado para suavizar e
sintetizar a informação contida nos níveis de detalhes obtidos na
decomposição do sinal pela \gls{tw}. As características utilizadas são
os picos para os níveis de detalhes do Método de Burg adicionados do
\gls{fp}.

A discriminação é feita em duas etapas, primeiro se seleciona um grupo
genérico ao qual o equipamento pertence, para então selecionar o grupo
específico. A seleção é feita através do vizinho mais próximo com
métrica Euclidiana entre o valor no banco de dados e o resultado do
espectro de Burg para os níveis de detalhe da \gls{tw}. É realizada
uma votação (em cada etapa) para cada nível de detalhe, o equipamento
que tiver maior votação é o resultado da etapa do processo de
discriminação.

Os resultados para acionamentos individuais reportados pelo autor
foram de 99,7\% para a seleção na primeira etapa e 88,8\% na segunda.
Foi realizado um ensaio para verificar o comportamento do algoritmo em
casos empilhados, porém o mesmo foi trivial, apenas
adicionando uma lâmpada incandescente --- sem dinâmica de consumo --- 
antes de acionar os outros equipamentos.  Assim, não é possível
deduzir a capacidade do método para a aplicação em condições reais,
onde é necessário a robustez das técnicas aplicadas para ao
acúmulo de diversos equipamentos, injetando ruídos e dificultando
a desagregação da informação. O autor identificou, também, que é
necessário trabalhar no detector de eventos transitórios, que talvez
tenham sido o motivo para a deterioção da eficácia dos motores no
teste empregado. Ainda, as condições operativas normais em uma
residência podem revelar a presença de falsos positivos, sendo também
um levantamento necessário pelo trabalho. Por fim, os autores
preocuparam-se apenas com a capacidade de identificação dos aparelhos,
sendo necessário, pela perspectiva de aplicação do \gls{nilm}, ainda o
passo de transformar essa informação qualitativamente para termos de
consumo.

\item \textbf{Curvas I-V}
\label{nilm:curvas_iv}

As características geométricas das formas de onda para diversos
equipamentos foi explorado pelos estudos de
\citet{nilm_lee_2004_32,nilm_lam_2007_33}. Conforme
\cite{nilm_zeifman_review_2011}, a contribuição foi a investigação de
uma nova características, as curvas I-V\footnote{Provavelmente os
leitores já tiveram a oportunidade de observar essas curvas no
osciloscópio, caso oposto, diversas curvas podem ser observadas 
na referência \cite{nilm_lam_2007_33}.}. As curvas são formadas ao
distribuir as amostragens temporais de corrente contra tensão no plano
catersiano. No caso de cargas lineares, elas formam elipses, podendo
transformar-se em uma reta caso o aparelho seja puramente resistivo.
Nesse caso, o sentido de rotação da curva dependente se a carga é
indutiva ou capacitiva, mais precisamente estando relacionada ao
ângulo de carga. De acordo com a presença de harmônicos, pode ocorrer
distorções nas elipses de modo que as mesmas enlacem a si mesmas em n
pontos de interseção.  Assim, o estudo propos diversas características
a serem extraídas dessas curvas, cada uma representando
informações que permitam discernir as características dos aparelhos
entre si.

São utilizados os valores normalizados para que todos os aparelhos
tenham a mesma escala, removendo as discrepâncias causadas por
utensílios de diferentes marcas. A fim de estudar a capacidade
discriminativa, a metodologia empregada foi a geração de dendograma
para compreender a taxonomia resultante e compará-la com aquelas
normalmente obtidas quando utilizando variáveis padrões, no caso o
autor considerou: valor eficaz da corrente em um ciclo; potência
reativa; \gls{fp}; número total de harmônicos impares; número total de
harmônicos pares.  O estudo também comparou com a taxonomia obtida ao
realizar \gls{svd} da corrente de um período. Os
primeiros 24 autovetores foram utilizados como representação da
informação, contendo 99\% da energia, para serem formados os
agrupamentos.

Os resultados da separação taxonômica está na
Tabela~\ref{tab:taxonomias_lam}, (ver p.~\pageref{tab:taxonomias_lam},
estando mais adiante simplesmente para agrupar as taxonomias com as do
\gls{cepel} em um único lugar neste trabalho).  Constatou-se a
presença de grupos mais representativos para as curvas I-V em relação
aos outros dois casos, obtendo grupos bem definidos para os tipos
resistivos, motores e eletrônicos. A representação com variáveis
padrões teve dificuldade em distinguir entre os aparelhos eletrônicos.
Ocorreu também a união de aparelhos resistivos, eletrônicos e
aparelhos de operação através de motores. Já no caso da decomposição,
os aglomerados formados permitiram a classificação dos aparelhos em
relação ao formato de suas ondas, no entanto, não é possível inferir
como a afinidade dos aparelhos foi gerada.

\item \textbf{Ondas sem processamento}

A metodologia empregada por \citet{nilm_suzuki_2011_35} também opera
ciclo a ciclo na rede, como o caso das abordagens de
\cite{nilm_srinivasan_nn_2006_27,nilm_itajuba_rodrigues}, mas
alterando a lógica quando em comparação com esses outros métodos. Ao
tratar o problema como sendo de otimização, foi resolvido o caso 
do paradoxo da quantidade de configurações para o treinamento das
técnicas supervisionada. O próprio autor observa que outros autores
japôneses tentaram resolver o problema da mesma maneira --- através de
redes neurais --- ressaltando a quantidade de dados necessários por
essa abordagem. A técnica consiste-se em ajustar a melhor configuração
das ondas cruas de corrente --- sem pré-processamento --- que se
justaponha àquela observada pela medição centralizada. Isso é
realizado através de Programação por Inteiros (tradução própria de
\emph{Integer Programming}), modelando tanto \gls{c1}, \gls{c2} quanto
\gls{c3}, desde que se saiba da existência dos mesmos \emph{a priori}.
São obtidos 666 pontos por ciclo, resultando em uma frequência de
amostragem de 40 k\acs{hz}.

A abordagem foi avaliada em uma residência, onde os residentes
realizaram suas tarefas normalmente, mas anotando os momentos em que
eles utilizavam os aparelhos durante 6 dias. Os resultados mostraram
que a fase~2 com 7 aparelhos obteve taxas de detecção na ordem de
96,8\%, enquanto a fase~1 com 15 aparelhos, sendo eles diversas
lâmpadas, obteve para essa acurária, valor de 79,0\%. É esperado para
esse método a mesma dificuldade operativa das abordagens
\cite{nilm_srinivasan_nn_2006_27,nilm_itajuba_rodrigues}, já que a
quantidade de configurações possíveis cresce exponencialmente com o
número de estados possíveis nos aparelhos da residência. Como o
próprio autor observa, os erros na fase~2 foram causados por
configurações onde os estados de consumo agregado são semelhantes, no
caso ``Torradeira + Secador (Apenas Ventilação)'' $\approx$ ``Secador
de Cabelo (Com aquecimento no fraco)''. Segundo o crescimento de
aparelhos, é esperado que a ocorrência desses casos se torne maior. Os
autores planejam trabalhar em uma contramedida para possibilitar a
modelagem de \gls{c4}, bem como conseguir uma maneira automática para
unir aparelhos similares. Para isso, aqui se nota que utilizar um
patamar de corte em \ref{eq:similaridade} parece ser suficiente. Outra
questão ressaltada por \cite{nilm_zeifman_review_2011} é que a
utilização das ondas cruas pode ter reduzido a eficácia da abordagem,
considerando que a extração através de outras representações da mesma
são geralmente mais robustas. Por outro lado, nada impede o emprego de
Programação por Inteiros em cima de outras representações.

\item \textbf{Aplicação de múltiplas técnicas}
\label{nilm:multiplas_tecnicas}

Os estudos \citet*{nilm_liang_pt1_2010_34,nilm_liang_pt2_2010_40}
trataram o tema de desagregação do consumo de maneira metódica,
definindo matematicamente diversas grandezas que muitos autores
tentaram exprimir por palavras. No entanto, a maior contribuição foi
mostrar como combinar as técnicas presentes na vasta extensão de
abordagens já realizadas ao tema. A primeira questão já foi
explicitada na Subsessão~\ref{ssec:modelos_carga} quando se referindo
às \glspl{c6}, sendo o ponto de similaridade entre as características.
A seleção de características que não apresentem similaridade para os
mesmos aparelhos possibilita que as técnicas de discriminação possam
obter padrões relevantes para a solução do problema. O mesmo pode ser
dito quanto a capacidade de complementação das técnicas empregadas na
Etapa~\ref{itm:etapa3}, os autores propuseram a \gls{cr} em
\ref{eq:cr}, sendo $Z_a$ e $Z_b$ dois discriminadores quaisquer
empregados.

\begin{equation}\label{eq:cr}
CR_{z_a,z_b} = \dfrac{P\{Z_a=false|Z_b=true\}}{P\{Z_a=true\}}
\end{equation}

Naturalmente, essa abordagem requer mais um subpasso para combinar as
respostas dos diversos discriminadores para produzir uma resposta
única. Essa técnica é um problema conhecido tratado
através pela ideia de um \gls{cdm}, que dita as regras de como será
tratado o valor de cada decisão para realizar a fusão de informação.
\cite{nilm_liang_pt1_2010_34}, levanta algumas possibilidades:

\begin{description}
\item \textlabel{\gls{mco}}{text:uniao_tecnicas}: escolha do candidato
mais comum entre os membros da comissão. É o processo mais trivial de
ser executada computacionalmente, realizando apenas a contagem de
votos. Essa abordagem pode criar soluções não-únicas devido ao empate
na votação;
\item \gls{lur}: seleciona o melhor candidato através da escolha
daquele que obtém o menor valor \gls{lur}. Para isso, são definidos o
\gls{ir}, onde $y_{(k|j)}$ representa a característica desconhecida,
que é comparada com a assinatura conhecida
$\hat{y}_{(k|(i,j))}$. A grandeza \gls{ur} é representada por
\ref{eq:ur}, onde $M$ é o número de candidatos sendo consideradas.
E, finalmente, o \gls{lur} sendo definido por \ref{eq:lur}, onde
$\Theta$ é a população de candidados.

\begin{subequations}
\begin{equation}\label{eq:ir}
IR_{(i,j)} = \dfrac{
\left(\sum^N_{k=1}y_{(k|j)}-\hat{y}_{(k|(i,j))}\right)^2}{
\sum^N_{k=1}y^2_{(k|j)}}
\end{equation}
\begin{equation}\label{eq:ur}
UR_i = \prod_{j=1}^M IR_{(i,j)}
\end{equation}
\begin{equation}\label{eq:lur}
LUR = min \left(UR_i |\forall i \in \Theta\right)
\end{equation}
\end{subequations}

O \gls{lur} considera assim, qual candidato tem o menor resíduo em
relação às decisões tomadas pelas diversas técnicas. Esse \gls{cdm} é
mais pesado computacionalmente quando comparado com \gls{mco}, por
outro lado, sempre permite soluções, e frequentemente únicas.

\item \gls{mle}: outra maneira é utilizar o levantamento de
estatística. Ao simular diversas condições com casos conhecidos, é
possível determinar a resposta mais provável para a k-ésima
característica utilizando a j-ésima técnica, sendo ela
determinada pela função $K(l,j)$. Assim, calcula-se a probabilidade
marginal de ter $ib$ como resposta verdadeira enquanto a j-ésima
técnica com a k-ésima característica dão como candidato a resposta
$i_a$ ($\theta_{ia}$). Um conjunto de verossimilhanças é dado por
\begin{equation}
\rho(ib) = \prod_{j=1}^M\prod_{l=1}^L
p(\theta_{ia},ib,j,l|ia=K(l,j),\forall i_b)
\end{equation}
onde M e L são o número de características e técnicas,
respectivamente. A máxima verossimilhança pode ser determinada através de 
\begin{equation}
\Lambda(\rho)=max\{\rho(ib)|\forall ib\in\Theta\}
\end{equation}
A \gls{mle} é o método mais intensivo computacionalmente dos citados
pelo autor.
\end{description}

Já \cite{nilm_zeifman_review_2011} propõe a utilização de
técnicas adaptando a teoria de \emph{Dempster-Shafer} para realizar
tal tarefa, e cita \cite{information_fusion_basir_2007_40} como
exemplo. Outra abordagem possível é o emprego de uma \gls{rna}.

Em \cite{nilm_liang_pt2_2010_40} os autores criaram um simulador de
Monte-Carlo, capaz de gerar situações de operação simultânea de aparelhos,
ruído devido a dinâmica de carga (\gls{c5}) ou fontes externas de
ruído, dando uma capacidade maior para as possibilidades dos
testes das técnicas a serem empregadas pelos \gls{nilm}. Com base nos
dados simuladores por esse gerador, eles mostraram que o uso de
múltiplas características e múltiplas técnicas é benéfico para o
desagregador, aumentando a eficiência em cerca de 10\% quando
comparado com a melhor resultado de qualquer outra técnica quando
aplicada em isolado, obtendo valores na ordem de 90\% para a
\acs{class_eff}. A facilidade do simulador também permitiu uma
série de outros levantamentos:

\begin{enumerate}[label=\itshape\alph*\upshape)]
\item as \gls{cdm} são tanto incrementais em peso computacional quanto
em performance, onde a \gls{mle} obteve a melhor performance para
todos os casos simulados. A \gls{lur}, por outro lado, apresenta
praticamente o mesmo resultado, sendo talvez uma abordagem com melhor
custo benefício computacional;
\item a opção com múltiplas técnicas se apresentou superior a
ruído que as opções singulares;
\item as simulações para o ar condicionado mostraram que sua presença
deteriora a eficácia do desagregador;
\item há deterioção conforme a maior demanda na rede, bem como uma
presença maior de aparelhos operando;
\item quanto maior for o incremento de potência em relação à demanda
da rede, maior será a facilidade do \gls{nilm} desagregar essa
informação; e
\item a análise por similaridade permitiu identificar os aparelhos com
maior confusão.
\end{enumerate}

O autor de \cite{nilm_zeifman_review_2011}, motivado pelas
considerações de \cite{nilm_liang_pt2_2010_40}, realizou o estudo
\cite{nilm_zeifman_vast_hisample_pdfmerge_2011} mostrando a intensão
de unir o seu método para baixas amostragens 
\cite{nilm_zeifman_vast_2011} com a combinação de múltiplos métodos
para altas amostragens. Baseado nos estudos de seu levantamento
bibliográfico, o autor utilizou amostragens de 500~k\gls{hz} para
trabalhar com análise de transitórios. O detector de transitório
observa por mudanças estatisticamente significantes na forma de onda,
que ao serem encontradas causam a coleta de diversas formas de onda
antes e depois da mudança. Realiza-se a subtração entre o valor pós e
pré transitório para a geração da diferença no formato da onda
decorrente do transitório. Essa informação alimenta duas técnicas para
discriminação: uma utilizando os componentes harmônicos e outra os
dados crus para alimentar uma técnica multivariável. Calcula-se o
valor de \acs{pdf} para os dados de entrada em relações às
\acs{pdf}s obtidas experimentalmente para todos os aparelhos. A
classificação final de cada algoritmo é o valor máximo de
\acs{pdf}.

Um teste simples foi realizado para comparar os dois métodos. Foram
alterado estados operativos de quatro aparelhos, obtendo desagregação
perfeita para o caso das ondas cruas, e um erro para a decomposição
harmônica. O método combinado também obteve reconstrução perfeita.
Em seguida, coletaram-se dados para 8 horas com os quatro aparelhos,
obtendo novamente 1 erro para a decomposição harmônica e 100\% para os
outros dois casos. O autor retrata que seria ainda possível combinar
essa informação com o \gls{vast}, que também apresenta saída
probabilística. No entanto, parece que o autor descontinuou o projeto
em virtude da melhor capacidade prática do método desenvolvido para
baixas frequências que tem se mostrado capaz de cumprir com as
necessidades de projeto para o \gls{nilm} ser aplicado em um programa
de \gls{ee}.

\end{enumerate}

\subsubsection{3. Utilização de outros sensores}
\label{top:seminilm}

Diferente das propostas anteriores, alguns autores consideraram uma
abordagem não centralizada. A ideia é empregar outros sensores aquém
do medidor central na alimentação da residência para permitir maior
capacidade de desagregação, seja por eles já estarem disponíveis, ou,
em alguns casos, através de sua instalação. Mesmo que esses sensores
já estejam disponíveis, dificilmente se terá acesso a sua informação
sem intrusão da propriedade, constituindo assim uma abordagem
intrusiva na maioria dos casos.

Já se referiu a um caso onde os autores consideraram a utilização de
outros sensores. A abordagem de
\cite{nilm_norford_leeb_medianfilt_1996_13} para o setor comercial foi
bem sucedida ao empregar variáveis de controle para possibilitar a
obtenção do consumo de equipamentos \gls{c4}. 

Em \cite{seminilm_berges_multisensor_2010}, os autores mostram a
intensão de seguir uma nova abordagem. A ideia é utilizar a informação
de múltiplos sensores no ambiente, como iluminação, som etc. e
correlacionar a informação por eles obtidas automaticamente com o
consumo de aparelhos, permitindo, por exemplo, desagregar o consumo de
uma lâmpada na cozinha com a de um banheiro devido à um sensor de
iluminação nesse ambiente.

Um projeto na Suiça por \cite{seminilm_ihome_tomek_2012}, utiliza
diversos submedidores de energia conectados via \emph{wireless} a uma
central. A ideia é reduzir o acumulo de aparelhos no sinal a ser
desagregado ao colocar os medidores mais próximos ao consumo,
facilitando assim a identificação dos aparelhos. Cada módulo de
submedição realiza medições em 3,2~k\acs{hz} de tensão e corrente, que
são utilizados para gerar a \acl{p}, \acl{q} e os harmônicos impares
de corrente com indices de 1-11. Quando é detectado um evento de
transição, a informação é repassada para uma central, que procura no
catálogo o aparelho que melhor se corresponde com a transição.
Reporta-se eficiência de detecção de 95\% para submedidores com um
único aparelho. 

Uma outra abordagem bastante interessante realizada por
\citet*{seminilm_fhmm_empiricalnmeter_2013}, altera o conceito
para uma perspectiva semi-intrusiva, onde a intenção é
reduzir a necessidade monetária associada com custos em
sensores fazendo a seleção ótima do número de sensores ao mesmo tempo
que se maximiza a capacidade em desagregar a informação. Para isso,
eles criaram uma regra heurística para encontrar o conjunto ótimo de
aparelhos a serem monitorados por submedição. Os dados foram obtidos
na frequência de amostragem de 1~\acs{hz} para o medidor central,
e $\frac{1}{3}$~\acs{hz} para os medidores por aparelho. Essa
informação foi reduzida para uma taxa de 1 amostra a cada 20 segundos
através de um filtro de mediana. Também se utiliza o
algoritmo de \emph{Viterbi} para modelar as \gls{fsm}, no caso
utilizando \gls{hmm}. Conseguiu-se experimentalmente eficiências na
ordem de 95\% por aparelho em termos de \acs{medidafenergia}
(\ref{eq:fmeasure_en}). Em um trabalho futuro, os autores também
desejam utilizar outros sensores no ambiente, como em
\cite{seminilm_berges_multisensor_2010}.

\subsection{Discussão}
\label{ssec:nilm_discussao}

A primeira questão que chama atenção no levantamento bibliográfico é a
quesito de escala. O \gls{nilm} deve ser robusto para a aplicação
independente da quantidade de aparelhos disponíveis na residência do
consumidor. Naturalmente irá ocorrer a presença de aparelhos
desconhecidos, bem como de \gls{c5} que, conforme o acúmulo de
diversos aparelhos, irá tornar a rede um ambiente muito mais complexo
para operação das técnicas. É justamente nesse ambiente que as
abordagens devem ser analisadas, demonstrando como será o seu
comportamento nas condições adversas presentes nas redes elétricas
residenciais.

Geralmente os \glspl{nilm} utilizam eventos de transitório para buscar
por alterações no estado operativo dos aparelhos na rede ---
Etapa~\ref{itm:etapa2}. Essa estratégia reduz a análise de informação
pelo discriminador, uma vez que a busca por padrões só é realizada
quando distúrbios são encontrados na rede. Apesar disso, as técnicas
empregadas para encontrar os eventos devem ser capazes de separar
distúrbios causados por ruídos daqueles causados pelos aparelhos, uma
vez que a eficiência do \gls{nilm} será diretamente afetada pela sua
capacidade. Isso também evidência que os discriminadores precisam ser
robustos à presença de intrusos, tanto causado por ruídos ou por
aparelhos desconhecidos. Poucos estudos trataram a capacidade do
detector de transitório ou como um transitório falso é tratado pelo
discriminador.

Outra capacidade desejada nos \glspl{nilm} é a criação dos modelos
de consumo dos aparelhos de maneira automática. Diversos
autores fizeram mão de modificações do algoritmo de \emph{Viterbi}
\cite{nilm_bouloutas_viterbi_ext_1991_11,
nilm_hart_fsm_viterbi_1993_12,nilm_baranski_genetic_base_2003_19,
nilm_baranski_genetic_detalhado_2004_20,nilm_baranski_summary_2004_21,
nilm_bergman_distribuido_2011,nilm_zeifman_vast_2011,
nilm_zeifman_vastext_approach_2012,
nilm_zeifman_statistical_vastext_1stws_2012,
seminilm_fhmm_empiricalnmeter_2013}
operando em cima de técnicas de agrupamento de dados para a modelagem
de \glspl{c2} e \glspl{c3}, no entanto, por modelar utilizando o
comportamento mais provável, está técnica não permite observar
\gls{c2b} e \gls{c4}. É compreensível que tal tarefa não seja possível
de ser realizada cegamente, justamente por não existir um padrão a ser
identificado. Possivelmente esses aparelhos terão seus estados
parcialmente modelados por \glspl{c3} e/ou \gls{c2}, sendo necessário
intervenção humana para corrigir sua modelagem. As \glspl{c4}
representam um desafio ainda maior para os \glspl{nilm}, podendo ser
mais uma fonte de ruídos para aquelas configurações que realizam a
Etapa~\ref{itm:etapa2}, bem como um empecilho para a reconstrução de
energia por não alterarem seu consumo de maneira discreta. Deste modo,
até o momento as técnicas para criação de modelos trabalharam apenas
com as características macroscópicas e estatística de uso.
Normalmente, os autores utilizaram em conjunto com os modelos
construídos cegamente técnicas de otimização
\cite{nilm_bergman_distribuido_2011} ou discriminadores estatísticos
\cite{nilm_zeifman_vast_2011, nilm_zeifman_vastext_approach_2012,
nilm_zeifman_statistical_vastext_1stws_2012,
seminilm_fhmm_empiricalnmeter_2013}, o que seria a continuação natural
ao problema. Ainda assim, a extração dos modelos necessita de uma
grande quantidade de estatística --- 5 dias à 2 semanas ---,
possibilitando a utilização dessa estatística obtida na construção dos
modelos para alimentar diretamente o treinamento de técnicas
supervisionadas.

Além do mais, há uma discussão em relação a como transformar a
informação dos modelos construídos cegamente em aparelhos. Os casos
observados \cite{nilm_hart_1992_8,nilm_bergman_distribuido_2011,
nilm_zeifman_vastext_approach_2012} tendem a comparar as informações
obtidas com um catálogo, mas como observado por
\cite{nilm_matthews_overview_2008_22}, a melhor estratégia ao tema
seria ter uma plataforma de comunicação com o consumidor, deixando a
tarefa de identificação para o mesmo. O catálogo poderia ser utilizado
como valor inicial ou sugerido para o aparelho, e se nenhum valor
fosse encontrado no catálogo, um valor temporario como
``desconhecido01'' poderia ser utilizado até o usuário modificar seu
valor. Com todas as facilidades oferecidas pelas \glspl{ict}, a
comunicação com o usuário pode ser realizada por uma interface
\emph{web}, um aplicativo para \emph{smart phones} ou alterar os
mostradores domiciliares para permitir iteração com o consumidor.

Nesse caso também seria interessante a identificação de novos
aparelhos conforme a utilização da rede. A identificação de aparelhos
não mais em uso pode ser benéfica para técnicas que sofrem com 
\gls{c6}, reduzindo a possibilidade de confusão se um aparelho não
estiver mais sendo utilizado. Aproveitando a possibilidade de
comunicação com o consumidor, o projeto pode permitir do mesmo
alterar possíveis erros na modelagem dos aparelhos, unindo ou
separando estados de \gls{fsm} erroneamente identificados. Isso também
pode ser expandido para o caso de erros de discriminação dos
aparelhos. Já para o caso das abordagens onde o \gls{nilm} não tem a
aptidão para modelar automaticamente os aparelhos, fica evidente a
necessidade de priorizar os aparelhos de maior uso-final para, assim,
maximizar sua reconstrução de energia. Ainda assim, o sucesso da
técnica depende de alta eficiência, onde a intervenção do consumidor
não pode ser um ponto necessário no projeto. Em especial para o caso
da aplicação da técnica como um programa de \gls{ee}, onde a tarefa do
\gls{nilm}, além de desagregar energia é informar o consumidor sobre
como utilizar a energia de maneira sustentável através da redução do
consumo e resposta em demanda.

Do ponto de vista técnico, o emprego das abordagens devem ser
realizados tendo em mente que a tarefa principal do \gls{nilm}
é a desagregação da informação de energia, independente de como a
técnica empregada entenda o problema. Por exemplo, técnicas de separação
cega de fontes irão ver o problema como um problema de desmistura de
fontes, técnicas de aprendizado de máquina como a capacidade de
entender as diferenças das características apresentadas e técnicas de
otimização como obter a melhor qualidade/aptidão ou menor desvio com
aquilo que é oferecido.  As técnicas têm seus próprios parâmetros para
informar o quão bem realizaram suas tarefas, no entanto, pela
perspectiva do \gls{nilm} e do consumidor, a eficiência deve ser
informada em termos da capacidade de reconstruir a energia por
aparelho. Isso tornar ainda mais evidente quando o \gls{nilm} opera
em tempo real, de tal modo que sua eficiência deve analisar
\emph{temporalmente} o acerto de energia identificando erros devido à
redundância ou subestimação do consumo por aparelho como foi tratado
na Subsessão~\ref{ssec:nilm_eff_calc}.

Quanto à extração de características e técnicas empregadas, observam-se
três vertentes. Buscando a impregnação do método, uma vertente
limita-se ao emprego de técnicas em cima de características
macroscópicas obtidas com baixa amostragem 
\cite{nilm_hart_1992_8,nilm_cole_data_extraction_1998_14,
nilm_cole_extra_info_surge_1998_15,
nilm_norford_leeb_medianfilt_1996_13,
nilm_leeb_spectral_envelope_1995_23,
nilm_powers_15minsamp_1991_16,
nilm_farinaccio_16ssamp_1999_17,
nilm_marceau_16ssamp_improved_1999_18,
nilm_baranski_genetic_base_2003_19,
nilm_baranski_genetic_detalhado_2004_20,nilm_baranski_summary_2004_21,
nilm_bergman_distribuido_2011,nilm_genetic_2013,nilm_zeifman_vast_2011,
nilm_zeifman_vastext_approach_2012,
nilm_zeifman_statistical_vastext_1stws_2012,
nilm_zeifman_statistical_naive_enduses_2013}. Essa vertente apresenta
a configuração de menor embargo financeiro e pode aproveitar de uma
infraestrutura que está sendo disponibilizada pela rede elétrica
inteligente, os medidores inteligentes (no Brasil, sujeito a possíveis
restrições descritas na Subsessão~\ref{ssec:ret_tec}).  Os autores de
\cite{nilm_bergman_distribuido_2011} realizaram uma arquitetura
distribuida para permitir desagregar informação de consumo sem
qualquer alteração da estrutura fornecida nos \gls{eua}.
\cite{nilm_baranski_genetic_base_2003_19,
nilm_baranski_genetic_detalhado_2004_20,nilm_baranski_summary_2004_21}
mostraram que é possível realizar essa tarefa em medidores
eletromecânicos. No entanto, a presença apenas das informações
macroscópicas irão deteriorar a eficiência do \gls{nilm}, onde
\gls{c6} irão afetar a capacidade de discriminação dessa vertente. 

Por isso, as tendências nessa vertente são o emprego de estatística de
uso tanto de maneira cega \cite{nilm_zeifman_vast_2011,
nilm_zeifman_vastext_approach_2012,
nilm_zeifman_statistical_vastext_1stws_2012}, como através de
levantamentos de dados por institutos
\cite{nilm_zeifman_statistical_naive_enduses_2013}. No último caso, a
técnica irá se concentrar nos equipamentos de maior uso-final para
maximizar a capacidade de reconstrução de energia total. Por outro
lado, não se sabe no momento se é possível seguir esse caminho no
Brasil uma vez que são necessários dados recentes e representativos
das diversas regiões do país. Como foi articulado na
Sessão~\ref{sec:ee_dificuldades}, a concepção inicial para o
\gls{nilm} no Brasil foi justamente o intuíto de facilitar a coleta e
precisão de dados, reduzindo o peso das \glspl{pph}. Nada obstante, o
levantamento de estatística na própria residência, como mostrado por
\cite{nilm_zeifman_vastext_approach_2012,nilm_zeifman_statistical_vastext_1stws_2012},
parece ser suficiente para atender os pontos necessários para a
aplicação dessa tecnologia em um programa de \gls{ee}. A única questão
dessa abordagem seria como rotular os modelos encontrados com os
nomes dos aparelhos, assunto debatido previamente nesta subsessão.

Outra vertente, ao procurar por melhores capacidade de discriminação,
seguiu o caminho de elevar a taxa de amostragem \cite{
nilm_zeifman_vast_hisample_pdfmerge_2011,nilm_liang_pt1_2010_34,
nilm_liang_pt2_2010_40,nilm_patel_2007_29,nilm_gupta_patel_2010_30,
nilm_lee_variable_speed_estimation_2005_24,
nilm_wichakool_2009_25,nilm_shaw_2008_26,nilm_berges_2008_7,nilm_berges_2009_36,
2010_nilm_melhorando_pph_usa_37,nilm_chan_2000_31,nilm_coppe_nascimento,
nilm_lee_2004_32,nilm_lam_2007_33, nilm_srinivasan_nn_2006_27,
nilm_itajuba_rodrigues,nilm_suzuki_2011_35} para tornar possível a
avaliação de características microscópicas. A melhor capacidade
discriminativa das características a serem exploradas nesse universo
levaram aos autores geralmente a se contentarem em utilizar apenas uma
técnica para a desagregação do consumo. Em alguns casos \cite{
nilm_patel_2007_29,nilm_gupta_patel_2010_30}, a
resolução das características é tão fina que as próprias propriedades
da rede interferem nos padrões dos aparelhos. Logo, os discriminadores 
teriam a incoveniência de ter ser alimentados com a informação para
cada residência, bem como a utilização de um catalogo global seria
impossível, limitando a aplicação dessa abordagem se não houver
uma maneira automática de realizar esse processo. 

Os estudos realizados com características microscópicas geralmente
utilizaram do conhecimento prévio dos equipamentos presentes na rede e
analisaram a capacidade de desagregar a informação somente com esses
equipamentos presentes, ignorando a necessidade de escalabilidade do
projeto. Ainda, por analisar características com maior resolução,
essas técnicas são possivelmente mais sensíveis a ruído, já que os
mesmos distorcem com mais facilidade as características microscópicas
tornando as mesmas um padrão não similar àquele procurado. Por isso,
ainda é necessário descobrir se a maior capacidade de discriminação
também se aplica para essas condições.

A presença de informações discriminativas no nível de ciclo da rede
possibilitou abordagens onde a Etapa~\ref{itm:etapa2} não é realizada
antes da Etapa~\ref{itm:etapa3}. Uma das capacidades dos métodos que
exploram o regime operativo permanentemente é a identificação de
\glspl{c1}. Porém, eles sofrem com a necessidade de que todos os
padrões possíveis sejam conhecidos. No caso, só foram observadas três
abordagens que operam desse modo \cite{nilm_srinivasan_nn_2006_27,
nilm_itajuba_rodrigues,nilm_suzuki_2011_35}. As abordagens de
\cite{nilm_srinivasan_nn_2006_27,nilm_itajuba_rodrigues} trabalham com
todos os possíveis estados operativos da rede (combinação de todos os
estados operativos possíveis para os aparelhos), tornando essas
abordagens frágeis para a aplicação quando a quantidade de estados na
rede é grande uma vez que é necessário treinar as técnicas
supervisionadas (no caso \gls{rna}) com um número de padrões que
cresce exponencialmente proporcional ao número total de estados dos
aparelhos. Já a abordagem de \cite{nilm_suzuki_2011_35}, por trabalhar
o problema através de otimização, conseguiu uma maior possibilidade
prática para a aplicação de sua técnica, ainda estando sujeito à
possiveis problemas de patamares de consumo parecidos que possam ser
determinados por diversas combinações de aparelhos diferentes. Além
disso, não se sabe como a presença de aparelhos desconhecidos afetarão
essas abordagens. Ainda ficou em aberto como determinar o consumo dos
aparelhos através da informação retornada pela etapa de discriminação,
sugestões foram realizadas na p.~\pageref{text:transf_info_discr_energia}.
Finalmente, os resultados mostram que para configurações onde há
poucos aparelhos conhecidos essas técnicas solucionam o problema
bem-sucedidamente.

Uma possibilidade --- e \emph{quiçá} tendência --- para essa vertente é a
combinação de múltiplas características e técnicas para o
reconhecimento de padrões \cite{
nilm_zeifman_vast_hisample_pdfmerge_2011,nilm_liang_pt1_2010_34,
nilm_liang_pt2_2010_40}. A pluralidade de características
independentes é benéfica para a capacidade de discriminação, sendo
ainda um fato pouco explorado. Alguns autores chegaram a
utilizar/considerar as medidas macroscópicas para auxiliar no processo
discriminativo em conjunto com as medidas microscópicas. Por exemplo, 
\cite{nilm_coppe_nascimento} empregou uma etapa de pré-seleção para os
aparelhos que mais se destacavam nas características macroscópicas
facilitando no processo discriminativo. Nessa mesma abordagem, como se
utilizou apenas da magnetude da corrente como característica
microscópica, observou-se que a adição da variação do \gls{fp} era
benéfico para a discriminação. Porém, geralmente as características
microscópicas conseguem representar as características macroscópicas,
sendo necessário buscar por outras características microscópicas
representativas. Além disso, o emprego de multiplas técnicas
independentes em paralelo (em cima, ou não, das mesmas características)
permite uma melhor discriminação, já que poderá acontecer
complementação das lacunas individuais de cada técnica. Há uma
variedade de possibilidades para a fusão dos resultados apresentados em
paralelo pelos discriminadores, que além de \glspl{rna}, podem ser
realizadas como proposto por \cite{nilm_liang_pt1_2010_34} ou
\cite{nilm_zeifman_review_2011} (ver
p.~\pageref{text:uniao_tecnicas}). No entanto, é importante frisar que
isso também trará uma maior carga computacional ao \gls{nilm}, de
formar que a quantidade de características e técnicas aplicadas será
limitada pela capacidade de processamento disponível, particularmente
quando operando em tempo real. Ressalta-se aqui que, além da discussão
para aplicação de multiplas características microscópicas e técnicas,
o emprego de estatística do uso, característica até o momento
\emph{sui generis} na vertente explorando características
macroscópicas, também pode ser realizada em quaisquer topologias do
\gls{nilm}.

A última vertente quebra a ideia de um medidor central para desagregar
toda a informação de consumo, percebendo que a questão do projeto é
reduzir o custo com sensores de forma a torná-lo viável em larga
escala, o que não implica, necessariamente, na utilização de apenas um
medidor. Para essa versão, o \gls{nilm} deixa de ser não-intrusivo,
para ter uma configuração \emph{semi-intrusiva}
\cite{seminilm_ihome_tomek_2012,seminilm_fhmm_empiricalnmeter_2013,
seminilm_berges_multisensor_2010}. O objetivo para essa configuração
pode ser descrito como minimizar o custo com sensores sem deteriorar a
capacidade de desagregação. Essas abordagens podem optar por
submedidores mais próximos ao nível de consumo, reduzindo o
acúmulo de operação de aparelhos que, intrinsecamente, torna a tarefa mais
simples; utilizar outros sensores que deem sensibilidade para a
discriminação, como correlacionar o nível sonóro, de iluminação e
movimento com certos aparelhos; ou submedição diretamente nos
aparelhos que as técnicas empregadas tiverem maior dificuldade. Porém,
deve-se considerar a estética quando na utilização de submedidores,
onde os consumidores tenderão a rejeitar configurações que reduzam a
harmônia do ambiente em seus domicilios. Isso também é verdade para
qualquer configuração que reduza a privacidade ou façam-nos ter a
sensação de que isso possivelmente irá ocorrer. Uma outra percepção em
vista das dificuldades das abordagens que operam ciclo a ciclo 
\cite{nilm_srinivasan_nn_2006_27,
nilm_itajuba_rodrigues,nilm_suzuki_2011_35}, uma maneira de superar a
dificuldade do crescimento exponencial seria através da aplicação de
submedidores atuando em cima de uma informação agregada até um número
máximo de aparelhos.

Portanto, as técnicas devem ser robustas para aplicação em residências
com a presença de aparelhos desconhecidos e para os momentos onde
haverá maior operação, causando a operação simultânea de diversos aparelhos
que, possivelmente, injetarão ruídos na rede através de sua dinâmica.
Devido a presença de aparelhos deconhecidos, um aspecto desejável para
o \gls{nilm} é conseguir detectá-los de maneira automática e de algum
modo possibilitar a correlação dos modelos identificados com seus
nomes. Outra questão é como obter informação discriminante garantindo
que o projeto seja viável. A utilização de amostragem baixa deteriora
a capacidade de discriminação, por outro lado é possível que a
estrutura para essas abordagens já esteja disponível. Essa abordagem
parece ser a mais próxima de tornar-se realidade. Enquanto a elevação
da amostragem aumenta a eficiência, dificulta a aplicação do projeto
por necessitar de \emph{hardware} específico. Além disso, as técnicas
presentes atualmente para alta amostragem ainda precisam ser testadas
quanto a sua robustez. Todavia, um outro caminho para obter maiores
eficiências pode ser através da escolha ótima de submedidores que, ao
mesmo tempo, maximizam a eficiência e minimizam o custo do projeto.

\section[A Metodologia no CEPEL]{A Metodologia no \acs{cepel}}
\label{sec:nilm_cepel}

Esta sessão se dedica à descrição do projeto no \gls{cepel} e as
colaborações realizadas pela \acs{coppe} no mesmo. Os trabalhos de
autores precedentes ao presente trabalho é realizado na
Subsessão~\ref{ssec:cepel_anteriores}. A contribuição realizada pelo
grupo de engenheiros do \gls{cepel} posterior ao último trabalho em
colaboração com a \acs{coppe} encontra-se na
Subsessão~\ref{ssec:met_cepel}, sendo essa metodologia a base
para o trabalho atual. Na Subsessão~\ref{ssec:caracteristicas} é
realizada considerações quanto a escolha da \gls{fex} em relação a
discriminação, levantamento deixado para contribuir para a escolha das
mesmas em trabalhos futuros. Ver-se-á nesta sessão que há uma série de
dificuldades a serem abordadas antes de retornar a questão de
discriminação dos aparelhos.

\subsection{Estudos Anteriores}
\label{ssec:cepel_anteriores}

Os estudos envolvendo a aplicação de técnicas não-invasivas para a
desagregação do consumo no \gls{cepel} tiveram seu início com o estudo de
\citet*{nilm_cepel_aguiar}, em \citeyear{nilm_cepel_aguiar}. Esse
estudo, bem como os seus sucessores:
\citet*{nilm_cepel_bezerra} (\citeyear{nilm_cepel_bezerra}) e 
\citet*{nilm_cepel_alvaro} (\citeyear{nilm_cepel_alvaro}), utilizaram a
informação da envoltória da onda de corrente no regime transitório
para identificar o acionamento dos aparelhos domiciliares. Diferente
dos trabalhos realizados pelo mesmo grupo de
\citeauthor*{nilm_hart_1992_8} para o setor comercial e industria
\cite{nilm_norford_leeb_medianfilt_1996_13,
nilm_leeb_spectral_envelope_1995_23}\footnote{Esses são apenas alguns
dos trabalhos realizados, a literatura disponível pelo grupo é muito
mais extensa.} onde a extração da envoltória obtida através da decomposição
harmônica é realizada em amostras espaçadas em 1~\acs{hz}, os
trabalhos utilizaram a envoltória em 60~\acs{hz}. Essa diferença
é justificável pela diferença dos dois ambientes, onde os transitórios
de muitos equipamentos disponíveis para os trabalhos
\cite{nilm_norford_leeb_medianfilt_1996_13,
nilm_leeb_spectral_envelope_1995_23} tem características lentas (ordem
de segundos a poucos minutos), tanto que os autores tiveram que
explorar o conceito de sessões-v (ver
p.~\pageref{nilm:pot_real_trans}), enquanto os explorados no setor
residencial apresentam toda a informação relevante de
transitório em até 1~segundo (\cite{nilm_cepel_aguiar} observou para
seus aparelhos analisados 417~ms). Como foi notado por diversos autores
\cite{nilm_hart_1992_8,nilm_sultanem_1991_10,
nilm_cole_data_extraction_1998_14,nilm_cole_extra_info_surge_1998_15}
há um lento transitório em alguns equipamentos que tenuamente alteram
seu consumo --- por exemplo, devido ao aquecimento de motores ---, mas
aqui se refere essencialmente a informação discriminante do
transitório.

Todos os trabalhos utilizaram \gls{rna} para discriminar a envoltória
de corrente, entretanto algumas peculiaridades podem ser observadas
para cada estudo. Devido a maior limitação, \citet*{nilm_cepel_aguiar}
utilizou 16 pontos da envoltória (normalizados por 100~\acs{a})
espaçados não-linearmente para conseguir uma melhor representação da
informação do transitório --- distribuiu-se os pontos para obter uma
maior granularidade próxima ao transitório, que gradualmente vão se
esparçando conforme eles ficam distantes do centro ---, o valor de
corrente em regime permanente (normalizados por 50~\acs{a}) e
ângulo de carga. Apesar de todas as limitações técnicas, o método
conseguiu eficiência de classificação superiores à 90\%, mas sofrendo
deterioração para o grupo de eletrônicos. 

Já \citet*{nilm_cepel_bezerra}, trabalhou em cima de um sistema
de aquisição mais adequado, onde sua melhor configuração empregou 60 amostras
da envoltória que são transformados utilizando \gls{pcd}, resultando
em 98\% de eficiência de classificação.  Esse autor também realizou
outros estudos, sendo eles:
\begin{enumerate*}[label=\itshape\alph*\upshape)] 
\item consideração entre a utilização ou não da informação de regime
permanente para classificação, onde ela se mostrou desnecessária e
inclusive obtendo resultados piores que a sua melhor configuração; 
\item análise sobre como a sobreposição de transitórios afetaria a
classificação, mostrando que esses eventos são críticos para a
capacidade discriminativa; e 
\item implementação de diversos cenários possíveis para o projeto,
considerando inclusive a implementação de um \gls{dnilm}.  
\end{enumerate*}

Finalmente, o sistema de aquisição disponível para
\citet*{nilm_cepel_alvaro} possibilitou a análise de uma gama
maior de equipamentos, sendo capaz observar equipamentos com
consumo superior à 150~m\acs{a}, enquanto
\citet*{nilm_cepel_aguiar} operava somente para utensílios consumindo
mais de 1~\acs{a}. Houve também uma melhoria no
levantamento do banco de dados, havendo uma amostragem bastante
superior aos trabalhos anteriores, com um total de 1324 acionamentos
observados. O trabalho implementou dois detectores especialistas para
a identificação dos momentos de acionamentos (a
Etapa~\ref{itm:etapa2}, porém apenas para acionamentos), e um baseado
em correlação. Foram utilizadas 150 amostras da envoltória de
corrente, obtendo taxa de detecção de 86~\%.  Grande parte da
inexatidão ocorreu devido aos erros no grupo de eletrônicos, obtendo
taxas de classificação bastante inadequadas para esse grupo, na ordem
de 30~\%.

Apesar da evolução dos trabalhos, diversos pontos podem ser
levantados. Primeiro, a principal característica utilizada pelos
estudos só é discriminante para os acionamentos de eventos, não
podendo utilizar a mesma na identificação de padrões de desligamentos
ou mudanças para estados de menor consumo, que na maioria dos casos
são apenas uma queda abrupta no consumo.  Além disso, não é possível
inferir se essa informação também seria discriminante para mudanças de
estado positivas, já que os estudos não trataram o caso de \gls{c2}
--- sendo uma outra dificuldade a ser tratada. Não bastasse, os
estudos trataram apenas de casos de aparelhos sendo acionados
isoladamente. Como foi notado na Subsessão~\ref{ssec:nilm_discussao},
a maior dificuldade para os \glspl{nilm} ocorrem devido à
operação simultânea de equipamentos na rede, em especial quando eles
se constituem de \gls{c5}. Os esforços para realizar a
Etapa~\ref{itm:etapa2} envolveram apenas a detecção de acionamentos,
bem como o trabalho de \cite{nilm_cepel_alvaro} não considera a
questão de classificação do desligamento. Indo adiante, a taxonomia
levantada pelos autores realiza a discriminação em grupos bastante
genéricos de equipamentos.  Não se sabe até que ponto essa informação
pode contribuir para a desagregação do consumo de energia residencial,
em especial para a aplicação em programa de \gls{ee}, onde os
consumidores desejam saber o consumo por equipamento. Continuando, a
técnica se mostrou ineficiente para um grupo de equipamentos cada vez
mais presente nas residências, os eletrônicos, sem ser possível saber
o porquê isso ocorre. Por fim, não se tratou o tema de aparelhos
desconhecidos, sendo ainda necessário considerar como adaptar o método
para identificar novos padrões recorrentes.

Em vista das dificuldades de tornar a informação da classificação em
informação relevante para os \glspl{nilm}, o grupo do \gls{cepel}
empenhou-se no sentido de melhorar a capacidade de detecção de eventos
de transitório, como será visto a seguir. Quanto à questão do
problema da envoltória como característica para discriminação, serão
realizadas considerações mais adiante, na
Subsessão~\ref{ssec:caracteristicas}.

\subsection{Metodologia de Partida do Trabalho}
\label{ssec:met_cepel}

A equipe do \gls{cepel} comprometeu-se desde o trabalho de
\citeauthor*{nilm_cepel_alvaro} para tornar o projeto mais próximo das
necessidades do \gls{nilm}. Reparou-se na necessidade de um
método para a detecção de eventos de transitório que fosse capaz de
identificar tanto mudanças para estados através do acréscimo de
energia, quanto no decréscimo. Os detectores fornecidos por esse autor
são especialistas para detecção de acionamentos, sendo dois deles
determinados empiricamente e um outro utilizando correlação entre o
sinal e formas de onda pré-determinadas. Exceto o último, que não teve
bons resultados, eles não são aplicáveis para detecção de decréscimo
de demanda causado pelos equipamentos, sendo necessário adaptar os
detectores empíricos para os casos de desligamento, ou criar um novo
filtro. 

\begin{figure}[h!t]
\centering
\includegraphics[width=.8\textwidth]
{imagens/cepel_transitorio.pdf}
\caption[Esboço da metodologia empregada pelo CEPEL]{Esboço da
metodologia empregada pelo \acs{cepel}, ponto de partida para este
trabalho.}
\label{fig:cepel_transitorio}
\end{figure}


A opção tomada pelo grupo do \acs{cepel} foi a segunda, onde
\citet*{rel_cepel_detevt} decidiram utilizar a derivada de Gaussiana
baseando-se na literatura de processamento de imagens para a detecção
de bordas. Um esboço da metodologia empregada pelo \acs{cepel} está
disposto na Figura~\ref{fig:cepel_transitorio}. No caso, a diferença é
que ao invés de operar em cima de \emph{pixels}, os valores são
amostras de corrente, bem como em apenas uma dimensão.  Para realizar
a convolução, aplicou-se um \acs{fir} com uma janela inicialmente com
o mesmo número de amostras que havia sendo utilizado, de 150 amostras
em 60~\acs{hz}, o que implica em 2,5~s.  Valhe ressaltar que na
verdade é o desvio padrão da Gaussiana que determina o tamanho
relevante da janela, já que ao se distanciar do centro os valores irão
tender a zero --- o mesmo sendo verdade para sua derivada ---, sendo
assim, o número de 150 amostras é apenas um corte para a quantidade
máxima de informação contida no filtro e utilizadas para a convolução.
O valor de resposta alimenta um corte linear no qual, se a resposta
for superior em absoluto, irá gerar uma região sensibilizada para ser
analisada até a resposta do filtro normalizar, ou continuar
sensibilizando mas com sinal oposto, nesse caso já criando uma nova
região sensibilizada. Para determinar o centro do transitório,
simplesmente é realizado o cálculo do ponto de inflexão da resposta do
filtro, e se houver mais de um, o ponto de inflexão com maior valor
absoluto é utilizado.

%Foi proposto uma evolução gradual de dificuldade para a avaliação do
%método e uma sequência de etapas para evoluir no sentido de obter um
%\gls{nilm} capaz de desagregar o consumo, e não apenas identificar
%equipamentos. A configuração mais simples incluiu a coleta de dados de
%aparelhos operando individualmente, para depois adicionar
%configurações com empilhamentos dois a dois, seguindo para
%configurações mais complexas, adicionando \gls{c4}, \gls{c5} e
%\gls{c6} de maneira gradual até chegar em condições reais. A validação
%do filtro de derivada de Gaussiana foi realizada em cima de
%configurações mais simples, havendo acúmulo em configurações
%simples e dados limpos. 

A fim de avaliar a eficácia, o relatório utilizou de novos dados, onde
algumas configurações já apresentavam operação justaposta de aparelhos
operando em diversos estados, no entanto sem ruído. A ideia aplicada
foi de gradualmente aumentar a dificuldade a ser estudada e adaptar a
abordagem até chegar em condições similares àquelas encontradas nas
residências brasileiras. A análise para as condições límpidas revelou
que a derivada de Gaussiana obteve melhor eficiência que os filtros
apresentados no trabalho anterior, sendo a técnica escolhida desde
então para a detecção de eventos de transitório.  A versão do
algoritmo que foi o ponto de partida deste trabalho ainda considerou
mais alguns detalhes. Depois de ser determinado o centro, é calculado
o pré e o pós-transitório para as variáveis macroscópicas disponíveis.
No caso, elas são \acf{di}, \acf{dp}, \acf{dq}, \acf{dd} e \acf{ds},
obtidos através de \ref{eq:deltasMacro}.

Os valores determinados inicialmente pelo \gls{cepel} para determinar
os valores de pré e pós transitório são a média de 10 amostras
afastados de 140 pontos em relação ao centro do transitório. A
variável \acs{di} é empregada para realizar um corte linear para um
valor mínimo em que o candidato a evento --- o centro selecionado da
região sensibilizada pelo filtro --- com o intuíto de remover
possíveis regiões sensibilizadas por ruído. Também se realiza um outro
corte levando em conta a proximidade de candidatos a evento de
transitório. Caso os mesmos estejam próximos entre si abaixo de um
valor mínimo de amostras, os eventos temporalmente posteriores ao
primeiro evento são rejeitados. Ainda foi considerado se o corte por
ruído deveria ser realizado antes do corte temporal, caminho (I) na
Figura~\ref{fig:cepel_transitorio}; vice-versa, caminho (II); ou
apenas um dos cortes, caminhos (III) e (IV). O \gls{cepel} conseguiu
excelentes resultados para a configuração sem ruído, entretanto, havia
uma grande dificuldade quanto a como determinar os valores de corte,
bem como visualizar o comportamento do algoritmo para entender suas
nuances. Ao testar em configurações ruídosas, não era possível
determinar exatamente o que estava ocorrendo nem como saber quais
outros valor poderia amezinariam os problemas.

\begin{subequations} \label{eq:deltasMacro}
\begin{equation} \label{eq:dI}
\Delta I = I_{\text{pós}} - I_{\text{pré}}
\end{equation}
\begin{equation} \label{eq:dP}
\Delta P = P_{\text{pós}} - P_{\text{pré}}
\end{equation}
\begin{equation} \label{eq:dQ}
\Delta Q = Q_{\text{pós}} - Q_{\text{pré}}
\end{equation}
\begin{equation} \label{eq:dD}
\Delta D = D_{\text{pós}} - D_{\text{pré}}
\end{equation}
\begin{equation} \label{eq:dS}
\Delta S = S_{\text{pós}} - S_{\text{pré}}
\end{equation}
\end{subequations}


Cabe definir o que são as características representadas em
\ref{eq:deltasMacro} com mais detalhes. As mesmas são bastante
conhecidas e amplamente utilizadas em análise de sistemas elétricos
operando em condições não-senoidais \cite{akagi2007instantaneous},
sendo definidas para sistemas \emph{monofásicos} no domínio da
frequência por \ref{eq:ipqds}. Também se define uma grandeza já
mencionada anteriormente no texto, o \gls{fp}, através de \ref{eq:fp}.
Uma observação importante quanto ao \acs{dd}, o mesmo é obtido
exatamente como a equação \ref{eq:dD} formula, ou seja, ele não é uma
variável amostrada, e sim \emph{estimada}. É importante notar que de
acordo com \ref{eq:d} essa variável só pode assumir valores positivos,
diferente de \acs{dq} que assume valores em ambas direções, sendo na
realidade a única característica que apresenta essa característica.
Obviamente, erros de medição irão ocorrer, de modo que o valor de
\acs{d} pode acabar sendo pertencente ao conjunto de números complexo.
Para evitar isso, nesses casos o valor de \acs{d} é determinado como
zero.

Apesar de se ter conhecimento das deficiências para as variáveis em
\ref{eq:q}--\ref{eq:d} de representar grandezas físicas \cite[cap.
2]{akagi2007instantaneous}, não se levará o assunto em consideração
pois: 

\begin{enumerate}
\item mesmo que a representação física dessas variáveis não seja clara, 
não se está interessado na mesma, apenas na capacidade delas de
servirem como característica para a detecção de eventos e
discriminação, este trabalho focando na primeira parte;
%\item a rede elétrica de uma residência dificilmente irá se configurar
%no pior caso de representação das variáveis em questão, um sistema
%trifásico desbalanceado e não-senoidal; e
%\item a rede elétrica de uma residência pode ser estudada como redes
%monofásicas independentes, salve equipamentos bifásicos que
%geralmente são equilibrados, podendo serem modelados como duas cargas
%em paralelo iguais nas duas redes;
\item essas variáveis são representações possíveis de serem obtidas com
medidores de baixo custo. 
\end{enumerate}

\begin{subequations} \label{eq:ipqds}
\begin{eqnarray}\label{eq:v}
v(t) = \sqrt{2}V_1 \text{sen}(\omega_1 t + \phi_{1v}) +
\sqrt{2}V_2 \text{sen}(\omega_2 t + \phi_{2v}) +  \nonumber \\
\sqrt{2}V_3 \text{sen}(\omega_3 t + \phi_{3v}) + \dots
\end{eqnarray}
\begin{eqnarray}\label{eq:i}
i(t)=\sqrt{2}I_1 \text{sen}(\omega_1 t + \phi_{1v} - \theta_{1}) +
\sqrt{2}I_2 \text{sen}(\omega_2 t + \phi_{2v} - \theta_{2}) +
\nonumber \\
\sqrt{2}I_3 \text{sen}(\omega_3 t + \phi_{3v} - \theta_{3}) + \dots
\end{eqnarray}
\begin{equation}\label{eq:pn}
P_n=V_nI_n\cos{\theta_n} ~~,~~ P=\sum_{n=1}^{\infty}P_n
\end{equation}
\begin{equation}\label{eq:q}
Q_n=V_nI_n\text{sen}{\theta_n} ~~,~~ Q=\sum_{n=1}^{\infty}Q_n
\end{equation}
\begin{equation}\label{eq:s}
S=VI
\end{equation}
\begin{equation}\label{eq:d}
D^2=S^2-P^2-Q^2
\end{equation}
\end{subequations}

\noindent onde:

\begin{description}
\item[$V_n$] é a n-ésima componente da decomposição harmônica em valor
eficaz de tensão;
\item[$I_n$] é a n-ésima componente da decomposição harmônica em valor
eficaz de corrente; 
\item[$\omega_n$] é velocidade angular da n-ésima componente
harmônica, podendo ser escrita como $2n\pi f$ ($f$ frequência da
fundamental);
\item[$\phi_n$] é o ângulo de defasagem para a n-ésima componente de
tensão;
\item[$\theta_n$] é o ângulo de defasagem entre tensão e corrente para 
n-ésima componente da decomposição harmônica;
\item[$P_n$] é a potência ativa para o n-ésima harmônico;
\item[$Q_n$] é a potência reativa para o n-ésima harmônico;
\end{description}

\begin{equation} \label{eq:fp}
FP = \frac{P}{S}
\end{equation}

Em especial quanto ao segundo item, há um interesse por parte do
\gls{cepel} de procurar uma solução menos complexa, mesmo
que isso deteriore --- até niveis aceitáveis --- a capacidade do
\gls{nilm}, mas que resulte em uma resposta mais rápida em tempo de
projeto. Por esse mesmo motivo, o \gls{cepel} considera a utilização
de medidores presentes no mercado, dando preferência para os de baixo
custo para facilitar o projeto, já que o desenvolvimento do
\emph{hardware} adiciona peso e reduz a velocidade do projeto. Por
enquanto, dois medidores estão sendo utilizados, um proprietário do
\acs{cepel} e um medidor da empresa \emph{Yokogawa}
\cite{yokogawa_medidor}, estando alguns dados disponíveis em uma
configuração, e outros na outra.

As variáveis anteriormente citadas --- adicionadas de suas envoltórias
--- são as que estão disponíveis atualmente para serem utilizadas (em
amostragem de 60~\acs{hz}).
Este trabalho foca na detecção de eventos, buscando nessa informação
para identificá-los, e não entrando em detalhes sobre a classificação.
Na subsessão a seguir serão feitas algumas considerações com o intuíto
de auxiliar trabalhos futuros na escolha de características. Como é possível
perceber, há uma série de dúvidas a serem tratadas antes de chegar no
mérito da discriminação. Assim, focou-se em fornecer uma
plataforma para dar maleabilidade ao projeto e laborar a generalização
do detector de eventos para as condições de ruído que irão estar
presentes nas redes elétricas domicialiares. Em especial, pretende-se
que isso seja realizado de maneira automática, encontrando uma boa
configuração sem ser necessário realizar exaustivos testes em busca da
mesma, independente da estrutura dos dados sendo testados. A
plataforma se consiste de um ambiente de análise \emph{a posteriori},
estando descrito no Capítulo~\ref{chap:framework}. 
%Em seguida, serão
%descritos a metodologia (Capítulo~\ref{chap:metodologia}), a base de
%dados (Sessão~\ref{sec:base_de_dados}) e os resultados
%(Capítulo~\ref{chap:resultados}).
Os resultados para a aplicação do ambiente de análise pode ser
encontrada no Capítulo~\ref{chap:resultados}.

\subsection{Quanto à Escolha das Características Discriminantes}
\label{ssec:caracteristicas}

Como foi observado anteriormente, há dúvidas quanto a utilização da
envoltória de corrente como característica discriminante. A primeira
consideração tratada leva em conta a taxonomia empregada para
discriminação, sendo essa muito abrangente. Esta subsessão começa
tratando exatamente esse assunto.

O levantamento da taxonomia permite ter uma concepção do comportamento
da distribuição dos dados nas características utilizadas. Elas
oferecem um ponto de partida para evitar erros causados por \gls{c6},
ou seja, é possível estimar quais aparelhos terão padrões parecidos
utilizando tais características, começando assim, com uma configuração
onde esses aparelhos representam apenas uma única informação a ser
discriminada. Obviamente, a aglutinação de partículas não resolve o
problema quanto à discriminação específica para os elementos
aglutinados, mas possibilita obter uma informação com qualidade, ainda
que mais genérica. Se essa generalização for um problema, será
necessário procurar por características mais finas ou utilizar um
discriminador especializado para realizar a segregação posteriormente.
Em outros casos, a aglutinação pode ser realizada apenas para reduzir
a dimensão do problema, procurando apenas encontrar assinaturas que
permitam generalizar os aparelhos elétricos em grupos mais
abrangentes que são suficientes para a abordagem em questão. No ponto
de vista do \gls{nilm} quando aplicado com o objetivo de explorar a
\gls{ee} e/ou melhorar a precisão das informações das \glspl{pph}, a
generalização não é desejada já que é necessário a informação por
aparelho.

A taxonomia encontrada pelos estudos do \gls{cepel}
foi determinada empíricamente observando a forma das envoltórias e
estão na Tabela~\ref{tab:taxonomias_cepel}. Apenas como referência
para outros autores, foi adicionado também um outro levantamento
empírico realizado por \citet*{nilm_cepel_aguiar}, baseando nas
considerações de \cite{nilm_sultanem_1991_10}, para o plano
$I\times\theta$ (sendo $\theta$ o ângulo de carga). As principais
diferenças nas taxonomias entre os estudos no \gls{cepel} foram devido
às mudanças no sistema de aquisição de dados\footnote{A aglutinação
realizada por \citeauthor*{nilm_cepel_bezerra} para aparelhos
resistivos e de ventilação foi realizada com o intuíto de comparar o
seu método com outro utilizado pelo \gls{cepel}.}. Percebe-se, que
apesar de haver avanços na capacidade discriminante dos grupos, eles
ainda permanecem bastante genéricos, em especial para a aplicação do
\gls{nilm} em um programa de \gls{ee}. Avanços poderiam ser realizados
ao empregar discriminadores especializados, no entanto, não é possível
inferir qual seria a capacidade de discriminação para os diversos
grupos, nem até que ponto a envoltória pode ser discriminante para
disponibilizar informação desagregada de consumo.

Indo além da questão da taxonomia empregada nas envoltórias, as mesmas 
só fornecem informação discriminante para os acionamentos. Outro fato que
aqui deve ser levado em conta é o aspecto levantado na
Subsessão~\ref{ssec:cepel_anteriores}, onde se percebeu a dificuldade
da abordagem para discriminação do grupo de aparelhos eletrônicos.
As considerações feitas por \citeauthor*{nilm_cepel_alvaro} revelam
que há dissimilaridade entre os padrões encontrados nos aparelhos.
Infelizmente o autor não detalhou a causa para a dissimilaridade, não
podendo saber se isso foi devido à aglutinação dos aparelhos
eletrônicos ser indevida por eles terem padrões diferentes, dificultando o
discriminador de encontrar um padrão para esse grupo, ou se os
acionamentos desses aparelhos ocorrem de maneira não bem-definida, por
isso não sendo possível encontrar um padrão. O único fato revelado é
que há dissimilaridade entre os acionamentos nesse grupo.

Independente da resposta para a questão anterior, as envoltórias podem
ser consideradas como características macroscópicas quando levando em
questão a \emph{informação contida em um período de onda}, sendo um nível
intermediário entre a informação de alteração no nível de consumo e
aquela contida dentro dos ciclos de onda. Percebe-se que os aparelhos
eletrônicos, por serem não-lineares, tem grande parte de sua
informação discriminante dentro do ciclo de onda. Essas
características --- na literatura foram utilizados harmônicos obtidos da
decomposição por \gls{fft}, níveis de detalhes das \glspl{tw}, curvas
I-V e, até mesmo, as ondas cruas --- poderiam ser utilizadas para
explorar a capacidade de discriminação dos aparelhos eletrônicos e
outras cargas não-lineares. Essa informação também estará disponível
independente de ser um acionamento ou não, sendo
necessário apenas extrair a informação da diferença causada pela
mudança de estado dos aparelhos em pós e pré-transitório para as
características utilizadas, servindo como fatores discriminantes em
todos os eventos de transitório. Um exemplo seria a própria taxonomia
levantada por \citet{nilm_lam_2007_33} (para mais detalhes ver
p.~\pageref{nilm:curvas_iv}) onde se percebe a distribuição dos
aparelhos eletrônicos em um grupo bem definido (Grupo~7) e alguns
outros aparelhos com grupos únicos para si. Valhe ter em mente que os
autores não estavam interessados em generalizar por aparelho, somente
mostrar que é possível encontrar grupos de aparelhos com
características elétricas que façam sentido fisicamente, e mesmo
assim, alguns aparelhos tiveram assinaturas \emph{sui generis}.

\begin{table}[p]
\resizebox{\textwidth}{!}{
\begin{tabular}{p{1cm}p{15cm}}
\hline \hline \hline
\multicolumn{1}{c}{\textbf{Grupo}} &
\multicolumn{1}{c}{\textbf{Equipamento}} \\

\hline \hline
\parbox[c]{16cm}{\centering\emph{Envoltória de corrente 
por \citet*{nilm_cepel_aguiar}} } \\ 
\hline \hline

\multicolumn{1}{c}{1} & Equipamentos de refrigeração em geral - refrigeradores,
freezers e condicionadores de ar \\
\multicolumn{1}{c}{2} & Equipamentos para aquecimento (resistivos) - chuveiros,
boilers, ferros de passar roupas, fornos em geral \\
\multicolumn{1}{c}{3} & Equipamentos de agitação/movimento/mistura -
liquidificadores, batedeiras, máquinas de lavar, furadeiras \\
\multicolumn{1}{c}{4} & Equipamentos de ventilação - ventiladores e
circuladores \\
\multicolumn{1}{c}{5} & Equipamentos eletrônicos - televisores, microcomputadores,
aparelhos de som, videocassetes \\
\multicolumn{1}{c}{6} & Lâmpadas incandescentes \\
\multicolumn{1}{c}{7} & Lâmpadas fluorescentes \\

\hline \hline 
\parbox[c]{16cm}{\centering \emph{Envoltória de corrente 
por \citet*{nilm_cepel_bezerra}}} \\
\hline \hline 
\multicolumn{1}{c}{1} & Equipamentos de refrigeração em geral - refrigeradores,
freezers e condicionadores de ar \\
\multicolumn{1}{c}{2} & Equipamentos resistivos (aquecimento) e
ventilação \\
\multicolumn{1}{c}{3} & Equipamentos de agitação/movimento/mistura -
liquidificadores, batedeiras, máquinas de lavar, furadeiras \\
\multicolumn{1}{c}{4} & Equipamentos eletrônicos - televisores, microcomputadores,
aparelhos de som, videocassetes \\
\multicolumn{1}{c}{5} & Lâmpadas incandescentes \\
\multicolumn{1}{c}{6} & Lâmpadas fluorescentes \\
\multicolumn{1}{c}{7} & Microondas \\

\hline \hline 
\parbox[c]{16cm}{\centering\emph{Envoltória de corrente 
por \citet*{nilm_cepel_alvaro}}} \\
\hline \hline 
\multicolumn{1}{c}{1} & Equipamentos de refrigeração em geral - refrigeradores,
freezers e condicionadores de ar \\
\multicolumn{1}{c}{2} & Equipamentos para aquecimento (resistivos) - chuveiros,
boilers, ferros de passar roupas, fornos em geral \\
\multicolumn{1}{c}{3} & Equipamentos de agitação/movimento/mistura -
liquidificadores, batedeiras, máquinas de lavar, furadeiras \\
\multicolumn{1}{c}{4} & Equipamentos de ventilação - ventiladores e
circuladores \\
\multicolumn{1}{c}{5} & Equipamentos eletrônicos - televisores, microcomputadores,
aparelhos de som, videocassetes \\
\multicolumn{1}{c}{6} & Lâmpadas incandescentes \\
\multicolumn{1}{c}{7} & Lâmpadas fluorescentes de reator eletrônico \\
\multicolumn{1}{c}{8} & Lâmpadas fluorescentes de reator
eletromagnético \\
\multicolumn{1}{c}{9} & Máquinas de lavar \\
\multicolumn{1}{c}{10} & Monitores CRT \\
\multicolumn{1}{c}{11} & Monitores LCD \\


\hline \hline 
\parbox[c]{16cm}{\centering \emph{Plano I-$\theta$ por
\citet*{nilm_cepel_aguiar} baseado em 
\cite{nilm_sultanem_1991_10}}} \\
\hline \hline 
\multicolumn{1}{c}{1} & Refrigeradores e \emph{freezers} \\
\multicolumn{1}{c}{2} & Equipamentos resistivos (aquecimento) \\
\multicolumn{1}{c}{3} & Lâmpadas incandescentes \\
\multicolumn{1}{c}{4} & Motores universais (liquidificadores, batedeiras, aspiradores de
pó etc.) \\
\multicolumn{1}{c}{5} & Ventiladores e circuladores \\
\multicolumn{1}{c}{6} & Equipamentos eletrônicos e lâmpadas
fluorescentes \\
\multicolumn{1}{c}{7} & Ar condicionado \\
\hline \hline \hline

\end{tabular} }
\caption[Taxonomias utilizadas por autores de estudos anteriores
no CEPEL]{Taxonomias encontradas em estudos anteriores no \gls{cepel}.
Todas foram determinadas empiricamente.}
\label{tab:taxonomias_cepel}
\end{table}

\begin{table}[p]
\resizebox{\textwidth}{!}{
\begin{tabular}{p{1.3cm}p{17.7cm}}
\hline \hline \hline
\multicolumn{1}{c}{\textbf{Grupo}} &
\multicolumn{1}{c}{\textbf{Equipamento}} \\
\hline \hline
\parbox[c]{19cm}{\centering\emph{Variáveis padrões}}\\ 
\hline \hline 
\multicolumn{1}{c}{1} & Aparelhos eletrônicos em modo \emph{stand by}
- computadores \emph{desktop}, monitores \emph{LCD} \\
\multicolumn{1}{c}{2} & \emph{CD players}, geladeiras, carregadores de
bateria, desumidificadores \\
\multicolumn{1}{c}{3} & Mistura de aparelhos resistivos, eletrônicos e
com motores - jarros elétrico, lâmpadas fluorescentes de reator
eletrônico, lâmpadas incandescentes, ventiladores, computadores
portateis, projetores, televisão de plasma, liquidificadores,
exaustores \\
\multicolumn{1}{c}{4} & Lâmpadas fluorescentes de reator
eletromecânicos, desumidificador \\
\multicolumn{1}{c}{5} & Cargas resistivas de potência média - secadores
de cabelo (potência média), aquecedores, ar condicionado, fogão de
indução (potência média) \\
\multicolumn{1}{c}{6} & Cargas resistivas de potência elevada e outros
aparelhos ligeiramente não-lineares - ferro de passar, aquecedor de
cabelo (potência alta), fogão de indução (potência alta) \\
\multicolumn{1}{c}{7} & Aspiradores de pó e forno microondas \\
\multicolumn{1}{c}{8} & Aspiradores de pó \\
\multicolumn{1}{c}{9} & Termoventilador (alta potência) \\
\multicolumn{1}{c}{10} & Computadores \emph{desktop} e impressora a jato
de tinta em \emph{stand by} \\
\multicolumn{1}{c}{11} & Computadores \emph{desktop} e projetores \\
\multicolumn{1}{c}{12} & Lâmpadas de econômia de energia, computadores
\emph{desktop}, impressora de tinta a jato em \emph{stand by},
televisores, \emph{video cassette}, \emph{scanner}, impressora a
\emph{laser} em \emph{stand by}, \emph{DVD player} ativo, televisores
\emph{stand by} \\
\multicolumn{1}{c}{Outros} & Secadores de cabelo operando na metade de
consumo e televisores em \emph{stand by} \\ 

\hline \hline 
\parbox[c]{19cm}{\centering\emph{Curvas I-V}} \\ 
\hline \hline 

\multicolumn{1}{c}{1} & Lâmpadas incandescentes, ventiladores,
aspiradores de pó, aquecedores, ar condicionado \\
\multicolumn{1}{c|}{1.1} & Lâmpadas incandescentes, ar condicionado \\
\multicolumn{1}{c|}{1.2} & Ventiladores \\
\multicolumn{1}{c|}{1.3} & Aspiradores de pó \\
\multicolumn{1}{c|}{1.4} & Televisores LCD \\
\multicolumn{1}{c}{2} & Lâmpadas fluorescentes com reator
eletromagnético, \emph{CD player}, carregadores de bateria,
geladeiras, desumidificadores \\
\multicolumn{1}{c|}{2.1} & Lâmpadas fluorescentes com reator
eletromagnético, geladeiras, desumidificadores \\
\multicolumn{1}{c|}{2.2} & \emph{CD player}, carregadores de bateria \\
\multicolumn{1}{c}{3} & \emph{CD player} e televisores LCD em
\emph{stand by} \\
\multicolumn{1}{c}{4} & Secadores de cabelo operando em baixa potência
\\
\multicolumn{1}{c}{5} & Forno microondas\\
\multicolumn{1}{c}{6} & Lâmpadas fluorescentes com reator eletrônico,
computadores portateis, forno de indução \\
\multicolumn{1}{c|}{6.1} & Lâmpadas fluorescentes com reator eletrônico \\
\multicolumn{1}{c|}{6.2} & Computadores portáteis \\
\multicolumn{1}{c|}{6.3} & Fornos de indução \\
\multicolumn{1}{c}{7} & Computadores \emph{desktop}, televisores,
\emph{video cassette}, \emph{scanner}, impressoras a \emph{laser},
carregadores de bateria de telefones celulares \\
\multicolumn{1}{c}{8} & Aparelhos em modo \emph{stand by} -
computadores \emph{desktop}, monitores LCD, forno de indução \\
\multicolumn{1}{c}{Outros} & projetores, máquinas de lavar \\

\hline \hline 
\parbox[c]{19cm}{\centering\gls{svd}} \\ 
\hline \hline 

\multicolumn{1}{c}{1} & Mistura de aparelhos resistivos, indutivos e
outras ligeiramente não-lineares \\
\multicolumn{1}{c|}{1.1} & Lâmpadas incandescente, aquecedores, ar
condicionado \\
\multicolumn{1}{c|}{1.2} & Fornos microondas \\
\multicolumn{1}{c|}{1.3} & Secadores de cabelo (potência baixa) \\
\multicolumn{1}{c|}{1.4} & Lâmpadas fluorescente com reator
eletromagnético, geladeiras, desumidificadores \\
\multicolumn{1}{c|}{1.5} & Ventiladores, exaustor \\
\multicolumn{1}{c|}{1.6} & \emph{CD player} em \emph{stand by},
carregadores de bateria \\
\multicolumn{1}{c|}{1.7} & \emph{CD player} \\
\multicolumn{1}{c}{2} & Carregadores de bateria para telefones
celulares \\
\multicolumn{1}{c}{3} & Computadores, monitores LCD, televisão,
\emph{scanner}, impressora de tinta a jato \\
\multicolumn{1}{c}{4} & Aparelhos eletrônicos em modo \emph{stand by}
- computadores, monitores LCD, impressora de tinta a jato \\
\hline \hline \hline
\end{tabular} } 
\caption[Taxonomias determinadas por LAM et al.: curvas I-V,
decomposiçao SVD e variáveis padrões 
]{Taxonomias determinadas através de dendograma por
\citet{nilm_lam_2007_33} (ver p.~\pageref{nilm:curvas_iv}), tradução
própria. O grupo \emph{Outros} determina equipamentos com um grupo
único para si. Em alguns casos, o autor realizou a separação em
subgrupos, aqui indicados por uma barra vertical para facilitar a
identificação visual dos mesmos.}
\label{tab:taxonomias_lam}
\end{table}
 
Pode ser do interesse do \acs{cepel} alterar a amostragem dos
medidores, dependendo, por exemplo, se for possível utilizar os
medidores inteligentes para garantir a impregnação do método, ou de
ser necessário melhores eficácias, nesse caso, sendo necessário
aumentar a frequência de amostragem. No primeiro caso,
ter-se-á acesso apenas a variáveis macroscópicas, como \acs{di}, \acs{dp}
e \acs{ds}, possivelmente outras variáveis podem ser empregadas, como
\acs{dq} e \acs{dd}, entretanto isso dependerá da capacidade do
medidor utilizado. Já para alta amostragem, 
frequências maiores à 60~\acs{hz} permitem a obtenção da informação
contida dentro do ciclo, possibilitando o acesso às informações
descritas no parágrafo anterior e, assim, uma maior capacidade de
discriminação. Nota-se que, como levantado na
p.~\pageref{nilm:multiplas_tecnicas}, a escolha não precisa se limitar
a uma única característica ou técnica, podendo abranger-se para um
conjunto delas para obter melhor capacidade discriminativa. Nessa
alternativa, pode-se também explorar a envoltória das ondas uma vez que
elas mostram uma capacidade de, no mínimo, auxiliar na discriminação
de cargas lineares para os eventos de transitório de aparelhos durante
o acionamento. Essa capacidade pode extender-se para as
cargas não-lineares, porém não será tratado neste trabalho se isso
é verdadeiro, limitando-se aqui em apenas mostrar que a questão
está aberta. Além disso, a informação de estatística de uso, como foi
visto na discussão realizada para o levantamento bibliográfico (ver
Subsessão~\ref{ssec:nilm_discussao}) é bastante discriminante e pode
auxiliar tanto na vertente de baixas amostragens, onde seu uso chega a
ser praticamente uma necessidade, ou quanto ao optar por uso de
diversas características, que mesmo não sendo necessária, a sua
utilização é fortemente indicada tomando como argumento os resultados
mostrados em
\cite{nilm_zeifman_vast_2011,nilm_zeifman_vastext_approach_2012,
nilm_zeifman_statistical_vastext_1stws_2012,
nilm_zeifman_vast_hisample_pdfmerge_2011,
nilm_zeifman_statistical_naive_enduses_2013}. Finalmente, existe a
possibilidade de utilizar submedidores, seja para reduzir a operação
justaposta dos aparelhos a uma quantidade aceitável, ou para obter
diretamente a informação desagregada de aparelhos complexos e
importantes em termos de uso-final
\cite{seminilm_ihome_tomek_2012,seminilm_fhmm_empiricalnmeter_2013,
seminilm_berges_multisensor_2010}.



