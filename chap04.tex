\chapter[Monitoramento Não-Invasivo de Cargas Elétricas]{\acrfull{nilm}}
\label{cap:nilm}

Este capítulo trata em detalhe as particularidades envolvidas no
desenvolvimento da tecnologia conhecida como
\gls{nilm}\footnote{NIALM e NALM são outras abreviaturas utilizadas na
literatura.}, em especial quanto aos aspectos técnicos envolvidos. A
Sessão~\ref{sec:nilm_aspec_gerais} é a única que não leva em conta os
aspectos técnicos, onde são apresentados seus pontos gerais. Em
seguida, as diversas metodologias utilizadas no mundo são utilizadas
para um levantamento dos aspectos envolvidos no desenvolvimento da
tecnologia na Sessão~\ref{sec:nilm_mundo}. Finalmente, será
apresentado na Sessão~\ref{sec:nilm_cepel} a evolução dessa tecnologia
no escopo deste trabalho, desenvolvido em conjunto com o \gls{cepel}.

\section{Aspectos Gerais}
\label{sec:nilm_aspec_gerais}

O \gls{nilm} é uma alternativa para fornecer a informação de consumo
de energia elétrica desagregado por utensílio. Ao invés das técnicas
normalmente utilizadas --- onde se faz mão de sensores dispostos em cada
utensílio, esses enviando informações para uma central encarregada de
decodificá-las e, assim, identificar os utensílios que estão demandando
consumo na rede ---, o \gls{nilm} é um método em que não ocorre a intrusão
da propriedade do usuário (ou intrusão em escala mínima), contendo
geralmente apenas um medidor central no fornecimento de energia dessa
propriedade. O peso da identificação dos utensílios é transferido da
utilização de uma grande quantidade de sensores e \emph{hardware}
complexo para um \emph{software} e algoritmos de processamento de
sinais no intuíto de realizar uma análise profunda das medições e
desagregar as informações.

Ainda que o peso da identificação esteja no \emph{software}, a
aptidão dos algoritmos implementados dependem da capacidade do
medidor de extrair informações da rede e das distorções causadas pelos
utensílios na mesma, de forma que quanto mais avançado for o
\emph{hardware} de medição, enviando uma quantidade e/ou frequência
maior de informação aos algoritmos encarregados de realizar a
discriminação dos rastros deixados na rede elétrica pelos aparelhos,
também maior será a capacidade dos mesmos de desagregar o consumo
específico dos utensílios.

Possíveis aplicações para esse dispositivo são: 

\begin{itemize}
\item auxiliar ou substituir as \glspl{pph} sem que seja necessário
causar incômodo ao consumidor devido a presença de medidores na
residencia, fornecendo assim dados com maior fidelidade em caracter
desagregado por utensílio e maior frequência para estudos de \gls{ee}
no consumo de eletricidade (Sessão~\ref{sec:ee_dificuldades}), em
especial para o setor residencial;
\item disponibilizar a informação desagregada para o fornecimento do
Retorno Indireto Diário/Semanal e/ou Retorno em Tempo Real Desagregado
em programas de \gls{ee}. Os programas em \gls{ee} utilizando retorno
de informação são possíveis fontes de redução de consumo nos grandes
centros urbanos, no entanto, são necessários estudos no Brasil para
determinar seu potencial (Sessão~\ref{sec:ee_res_exp}). O retorno de
informação desagregada por utensílio pode ser utilizado por empresas
que oferecem serviços de redução de consumo
(Subsessão~\ref{ssec:ret_tec}), ou mesmo por consumidores que precisam
ter algum tipo de controle sobre seu consumo de energia; \item no
interesse das concessionárias, a informação desagregada pode auxiliar
os clientes a identificarem consumo não essencial durante o horário de
ponta, auxiliando no deslocamento de carga (ver item \emph{Resposta de
Ponta e Demanda versus Economia Fora de Ponta} na
Subsessão~\ref{ssec:ret_outros}), assim como identificar clientes com
maior potencial de redução de consumo durante esses períodos para
oferecer incentivos nesse sentido (Subssessão~\ref{ssec:ret_tec}); 
\item monitoramento da qualidade de energia, diagnóstico de carga,
identificação de aparelhos defeituosos ou com consumo excessivo de
energia;
\end{itemize}

As possibilidades de aplicação como um meio de melhoria da \gls{ee} e
redução da intensidade elétrica tem elevado o interesse nesse assunto,
em especial nos países desenvolvidos, como uma forma de aliviar
a pressão de consumo nos grandes centros urbanos e na redução de
emissões de \gls{co2} \cite{nilm_zeifman_review_2011}.
As possibilidades de aplicação dessa tecnologia tem levado a gigantes
no setor eletrônico, como \emph{Intel} e \emph{Belkin}, a investirem
fortemente no desenvolvimento dessa tecnologia. O crescente interesse
na evolução por parte da academia levou a organização do primeiro
\emph{workshop} especificamente para o tema em 2012
\cite{workshop_nilm}. Essa alternativa é mais simples quando
comparando com a automação residencial por não requerer mudanças na
produção dos eletrodomésticos --- a automação requer comunicação nos
dois sentidos (entre a interface e o aparelho), tal como a capacidade
de controle do aparelho, de forma que se faz necessário adaptar
aparelhos antigos e a produção dos novos equipamentos com essas
capacidades --- juntamente com o fato de haver um relutância social
quanto às residências automatizadas, apesar de esforços governamentais
e da mídia local \cite{Lipoff_Automation_2010} (a referência estudou a
falta de interesse na automação residencial nos \gls{eua}). Não
bastasse, também se pode citar o nascimento de \emph{start-ups}
procurando espaço na corrida por esse novo mercado, como
\emph{GetEmme} \cite{getemme_site} e \emph{Navetas}
\cite{navetas_site}.

\section{As diversas metodologias utilizadas no mundo}
\label{sec:nilm_mundo}

A ideia de desagregar a informação não é nova, sendo possível
encontrar pesquisas nesse sentido datando da década de 1980. Apesar
disso, um extenso levantamento bibliográfico
\cite{nilm_hart_1992_8,nilm_bouloutas_viterbi_ext_1991_11,
nilm_hart_fsm_viterbi_1993_12,nilm_norford_leeb_medianfilt_1996_13,
nilm_cole_data_extraction_1998_14,nilm_cole_extra_info_surge_1998_15,
nilm_powers_15minsamp_1991_16,nilm_farinaccio_16ssamp_1999_17,
nilm_marceau_16ssamp_improved_1999_18,nilm_baranski_genetic_base_2003_19,
nilm_baranski_genetic_detalhado_2004_20,nilm_baranski_summary_2004_21,
nilm_matthews_overview_2008_22,nilm_laughman_continuous_variables_2003_9,
nilm_leeb_spectral_envelope_1995_23,nilm_lee_variable_speed_estimation_2005_24,
nilm_wichakool_2009_25,nilm_shaw_2008_26,nilm_srinivasan_nn_2006_27,
nilm_akbar_2007_28,nilm_patel_2007_29,nilm_gupta_patel_2010_30,
nilm_sultanem_1991_10,nilm_chan_2000_31,nilm_lee_2004_32,nilm_lam_2007_33,
nilm_liang_pt1_2010_34,nilm_suzuki_2011_35,nilm_berges_2008_7,
nilm_berges_2009_36,2010_nilm_melhorando_pph_usa_37,
nilm_liang_pt2_2010_40} realizado em \citet*{nilm_zeifman_review_2011}
expõe que as técnicas aplicadas em \glspl{nilm} ou não são robustas no
sentido de atenderem especificamente a um grupo limitado de utensílios
estudados, ou apresentam acurácia marginal, mostrando que o processo
de desagregação da informação não é trivial. Indo além das
dificuldades técnicas, ressalta-se os pontos enfatizados nas
Subsessões \ref{ssec:asp_psic} e \ref{ssec:asp_visuais}, nas quais
foram evidenciadas as características multidisciplinares desse método
quando aplicado como um meio de economia de energia e intensificação
da \gls{ee}. Outras referências utilizadas que não constam em
\cite{nilm_zeifman_review_2011} são
\cite{nilm_zeifman_statistical_approach_resumo_2013,
nilm_bergman_distribuido_2011,
nilm_genetic_2013,nilm_zeifman_2011,nilm_patel_2011,
nilm_zeifman_nonintrusive_2011,nilm_ihome_tomek_2012,
wavelet_transients_2009,nilm_berges_multisensor_2010,
nilm_coppe_nascimento,nilm_itajuba_rodrigues}.

Dividiu-se esta sessão da seguinte maneira. As primeiras subsessões,
\ref{ssec:modelos_carga} e \ref{ssec:metodologia_generica}, referem-se
aos modelos de carga utilizados para os aparelhos eletrodomésticos e às
etapas usualmente envolvidas na desagregação da informação de
consumo quando utilizando \glspl{nilm} respectivamente. Uma
proposta de padronização para o cálculo de eficiência é apresentado na
Subsessão~\ref{ssec:nilm_eff_calc}. Em seguida, a
Subsessão~\ref{ssec:nilm_tecnicas} irá apresentar as diversas
abordagens utilizadas, sendo guiada na proposta de divisão das
técnicas e características feita por \cite{nilm_zeifman_review_2011}.
Finalmente, a discussão das informações levantas é realizada em
\ref{ssec:nilm_mundo_padroes}.

\subsection{Modelos de Carga}
\label{ssec:modelos_carga}

Os utensílios podem ser modelados devido às características de
comportamento de suas cargas. A seguir encontram-se possíveis
características de carga elétrica dos eletrodomésticos. Os quatro
primeiros itens são modelos de cargas mutuamente exclusivos, enquanto
os dois seguintes podem ser incluídos, dependendo das propriedades dos
aparelhos ligados à rede, para caracterizar aparelhos potencialmente
dificultadores do processo de desagregação \cite[com adaptações]{
nilm_hart_1992_8,nilm_baranski_genetic_base_2003_19,
nilm_zeifman_review_2011,nilm_zeifman_nonintrusive_2011,
nilm_apresentacao_review_2011,nilm_liang_pt2_2010_40}:

\begin{itemize}
\item \textbf{\Gls{c1}}: aparelhos que permanecem
ligados 24~h/dia, 7~dias/semana, com consumo de energia praticamente
constante. Ex.: detectores de fumaça, fontes de alimentação
constantemente ligadas à rede;
\item \textbf{\gls{c2}}\footnote{Nas referências, não há separação na
categoria \gls{c2}, que é utilizada para descrever exclusivamente
aparelhos modelados como \acrshort{c2a}. Nelas, a categoria
\acrshort{c2b} é incluída em \acrshort{c4}, uma vez que em ambos os
casos as \glspl{fsm} podem alterar o seu estado de operação para
qualquer outro estado independente tanto de qual estado anterior ele estava
operando como o tempo de operação nesse estado. A separação das
\glspl{fsm} com estados discretos em duas categorias parece mais
familiar do que incluir \glspl{fsm} com patamares discretos e
aleatórios em uma categoria que permite patamares
continuos.\label{fn:subdivisao}}: essa categoria é utilizada
para identificar aparelhos que passam um conjunto de estados definidos
no qual os ciclos de mudança de estados são repetidos com frequência
suficiente nos eventos diários ou semanais. Mesmo equipamentos que são
apenas ligados/desligados pelo usuário podem acabar sendo modelados
como \glspl{c2} por mudar seus patamares de consumo enquanto estiver
ligado. Essa categoria pode ser dividida em dois conjuntos:
\begin{itemize}
\item \textbf{\gls{c2a}}\fnref{fn:subdivisao}: os estados do aparelho repetem-se em
padrões definidos temporalmente, garantindo que os seus ciclos serão
observados frequentemente durante intervalos diários ou semanais. Ex.:
máquina de lavar roupas, máquinas de lavar louças;
\item \textbf{\gls{c2b}}\fnref{fn:subdivisao}: nesses aparelhos não há
um padrão para os seus ciclos de operação. A operação por uma fonte
externa, como o consumidor, altera o seu padrão de consumo sem ser
possível encontrar uma regra operativa para o ciclo através da busca
de repetições de suas trocas de estado na rede, os estados mudam
aleatoriamente depois de quantidades de tempo também aleatórios. Ex.:
ventilador de múltiplas velocidades, liquidificador --- ambos
dependendo de como operados pelo consumidor: se apenas ligados e
desligados, irão comportar-se como \acrshort{c3}s, enquanto se
operados de modo padronizado, irão se comportar como \acrshort{c2a}s;
\end{itemize}
\item \textbf{\gls{c3}}: um caso partícular das \glspl{c2} ocorre
quando o aparelho pode ser modelado como tendo apenas dois estados:
ligado/desligado. Ex.: lampadas, torradeiras, bombas de água;
\item \textbf{\Gls{c4}}: uma generalização das \glspl{c2}, onde
há uma infinidade de estados para os quais o aparelho pode operar.
Essa categoria pressupõe que o aparelho irá estabilizar o seu consumo
em um patamar após um período de tempo. Sua operação pode ser dividida
em dois grupos: operação manual do operador, aparelhos
auto-controlados. Estes são de mais simples detecção quando comparados
com aqueles, uma vez que seus ciclos de mudanças de estado são
distribuidos uniformemente no tempo. Ainda assim, essa categoria é o
maior desafio para as técnicas empregadas nos \glspl{nilm}, sendo
raramente tratada por elas. Ex.: lampadas com \emph{dimmer},
ferramentas elétricas (furadeiras, serras etc.), bomba de aquário.
\item \textbf{\Gls{c5}}\footnote{As referências optaram por não
criar essas categorias uma vez que essas são apenas
características das cargas. Por sua vez, as mesmas são citadas, no
mínimo, em \cite{nilm_zeifman_review_2011,nilm_liang_pt2_2010_40}
como dificultadores no processo de desagregação e, por esse motivo,
preferiu-se adicionar diretamente essas categorias para
enfatizar e facilitar a identificação de cargas com essas
características.\label{fn:categoria_add}}: aparelhos que causam
distúrbios na rede continuamente devido a flutuações no seu nível de
consumo. Ex. (observados pela equipe do \gls{cepel}): televisores,
onde a variabilidade de brilho, cores e som, alteram seu consumo
(também observado em 
\cite{nilm_zeifman_statistical_approach_resumo_2013});
computadores, que alteram sua potência conforme a demanda dos
processadores e \emph{coolers}, consumindo mais quando o usuário está
realizando tarefas como, por exemplo, executando algoritmos do
mestrado, escutando música etc.; cargas de demanda dinâmica tais como
o ar condicionado --- esse quando com o compressor ligado --- alteram
seu consumo para baleancear a demanda através de resposta em
frequência.  Dependendo da grandeza dos disturbios, os mesmos são
potenciais dificultadores à identificação de rastros deixados por
outros aparelhos, particularmente os de menor consumo, caso estudado 
em \cite{nilm_liang_pt2_2010_40}. A Figura~\ref{fig:ar_cond_dinamica}
demonstra a demanda dinâmica para esse aparelho. Outros exemplos de
utensílios com motores que também geram oscilações --- mas em ordem
inferior ao ar condicionado --- são: microondas, geladeira,
desumidificador \cite{nilm_liang_pt2_2010_40};
\item \textbf{\Gls{c6}}\fnref{fn:categoria_add}: apesar de não ser
uma característica de um utensílio \emph{per se} e nem constituir um
modelo de carga elétrica, estudos de performance de \glspl{nilm} podem
considerar quais equipamentos serão potencialmente vistos como se
fossem um mesmo equipamento por possuirem os mesmos padrões. Essa categoria
varia conforme os aparelhos ligados à rede e quais são as
características sendo extraídas. Por exemplo, um computador e uma
lampada incandescente possuem consumos semelhantes quando procurando
padrões no plano \acrshort{dp}$\times$\acrshort{dq}
\cite{nilm_laughman_continuous_variables_2003_9}, enquanto
equipamentos com motores de potências distintas podem não ser
desagregados quando apenas olhando para seus transitórios --- em
especial quando normalizados, abordagem que seria utilizada se
utilizando \acrfull{rna}.
\end{itemize}

\begin{figure}[h!t]
\centering
\includegraphics[width=\textwidth]
{imagens/ArCondicionado-CargaDemandaDinamica_ComTextoImpr.pdf}
\caption[Exemplo de carga com demanda dinâmica: Ar Condicionado]
{Exemplo de carga com demanda dinâmica: Ar Condicionado. No caso, as
pequenas oscilações vistas são causadas pela demanda dinâmica do ar
condicionado que, ao detectar oscilações na frequência, responde
alterando seu consumo para manter o balanço na rede.}
\label{fig:ar_cond_dinamica}
\end{figure}

\subsection{Metodologia genérica}
\label{ssec:metodologia_generica}

O \gls{nilm} pode ser resumido em quatro etapas para o
tratamento da informação na rede elétrica e em três abordagens quanto
às técnicas empregadas: 

\subsubsection[Etapas]{Etapas \cite{nilm_matthews_overview_2008_22}}
\label{top:etapas}

\begin{enumerate}[label={Etapa} \arabic* - ,ref=\arabic*,align=left]
\item\label{itm:etapa1} \textbf{\gls{fex}}: são extraídas
informações das amostragens realizadas. A diversidade de
características que podem ser extraídas depende da capacidade do
medidor. As características serão utilizadas tanto para a detecção de
eventos de transitório quanto na identificação de padrões dos
equipamentos. Em alguns casos, para reduzir o esforço de processamento
ou reduzir a necessidade de armazenamento, a \gls{fex} para a
identificação de equipamentos pode ser realizada ou armazenada somente
quando identificados os eventos na Etapa~\ref{itm:etapa2};
\item\label{itm:etapa2}\textbf{Detecção de eventos de
transitório}: identificar alterações causadas por utensílios na rede.
Essa etapa é necessária para identificar alterações no consumo de
equipamentos. Pode-se empregar limiares estáticos ou dinâmicos. Os
limiares dinâmicos permitem o ajuste de operação, reduzindo ou
aumentando a sensibilidade do detector conforme a presença de
equipamentos \acrshort{c5}. Em algumas topologias, essa etapa não é
realizada. Nesses casos, a Etapa~\ref{itm:etapa3} retorna para cada
intervalo de tempo estudado (normalmente um período completo na frequência
da rede, 50/60~\acrshort{hz}) o estado de operação dos equipamentos,
sendo necessário levantar nessa informação o estado de consumo de cada
aparelho para cada instante de tempo;
\item\label{itm:etapa3}\textbf{Reconhecimento de padrões}: utilizar as
características pertinentes para o reconhecimento de padrões,
deduzindo, assim, qual foi o utensílio que causou o disturbio na rede
e qual seu novo consumo. É desejável que o algoritmo seja capaz de identificar
a ocorrência de novos padrões e reconhecê-los em suas próximas 
aparições pois a construção de um catálogo com todos os possíveis
eletrodomésticos é impraticável, se não impossível. Tal tarefa só será
possível com a capacidade dos \glspl{nilm} de incluirem novos
equipamentos ao catálogo. Diversas técnicas podem ser utilizadas em
conjunto para esta etapa;
\item\label{itm:etapa4}\textbf{Refinamento dos resultados}: após as
Etapas~\ref{itm:etapa2} e/ou \ref{itm:etapa3}, pode-se adicionar uma
etapa opcional para procurar por possíveis erros ou melhorias na
informação desagregada. Por exemplo, corrigir a informação de um
aparelho que remanesce consumindo energia da rede por dias enquanto
sua operação normalmente ocorre em intervalos curtos. Isso pode
ocorrer por falhas na Etapa~\ref{itm:etapa2}, onde o desligamento do
equipamento não foi encontrado, ou na Etapa~\ref{itm:etapa3}, na qual
o desligamento foi identificado como causado por outro equipamento.
Outra possível melhoria seria encontrar possíveis novos ciclos de
operações para aparelhos~\gls{c2}. As estratégias corretivas podem ser
meramente remediativas, ou seja, simplesmente ignorar alterações de
estados que permanecem em no mesmo patamar de consumo durante um grande
período de tempo para melhorar a resolução em energia do \gls{nilm},
aplicarem técnicas complementares para reanalizar a informação
parcial, realizar uma otimização complementar utilizando a informação
obtida nas etapas anteriores, ou simplesmente realizarem uma nova
análise através da Abordagem~\ref{itm:abordagem2}.
\end{enumerate}

\subsubsection[Abordagens]{Abordagens \cite[com
adaptações]{nilm_zeifman_review_2011}}
\label{top:abordagens}

\begin{enumerate}[label={Abordagem} \arabic* - ,ref=\arabic*,align=left]
\item\label{itm:abordagem1}\textbf{Abordagem por reconhecimento de
padrões}\footnote{A referência considera
que a maior diferença entre as abordagens é o tempo de resposta, onde
a Abordagem~\ref{itm:abordagem1} responderia em tempo real e a
Abordagem~\ref{itm:abordagem2} para cada período otimizado, o que pode
não ser verdade. Geralmente ambos os casos passam por um período de
otimização antes de serem empregados e, depois de otimizados, são
utilizados para a detecção dos padrões dos aparelhos na rede. A
distinção está que o primeiro otimiza a capacidade de
discernir os padrões --- sendo a reconstrução consequência disso ---,
enquanto o segundo a capacidade de reconstruir com maior fidelidade
possível o sinal original --- obtendo os padrões como
resultado.\label{fn:diff_abordagens}}: as técnicas de reconhecimento
de padrões são treinadas (otimizadas) em conjuntos de dados similares
aos quais eles irão operar. O reconhecimento de padrões pode ocorrer
apenas para as respostas da Etapa~\ref{itm:etapa2} ou para cada ciclo
da rede. Algumas técnicas utilizadas nessa abordagem podem ser
robustas aos aparelhos desconhecidos, sendo capaz de destacar seu padrão
dos outros já conhecidos e adicioná-lo ao catálogo de padrões. Assim,
quando esses padrões ocorrerem novamente, eles serão identificados
como o mesmo aparelho --- chamado de aprendizado em tempo real. Em
\cite{nilm_matthews_overview_2008_22} é observado a importância dessa
estratégia para tornar possível o crescimento do catálogo, que, tendo
o novo aparelho nomeado pelo consumidor, torna possível a criação de
um catálogo universal de equipamentos. Essa tarefa é impraticável, se
não impossível, de ser realizada em laboratório;
\item\label{itm:abordagem2}\textbf{Abordagem
por otimização}\fnref{fn:diff_abordagens}: concentra a capacidade de
suas técnicas na otimização, onde é realizada a procura por uma
combinação de aparelhos cujo o sinal agregado resultante é a melhor
aproximação do possível do sinal observado. Em alguns casos,
utiliza-se a concentração dos dados em longos períodos de tempo para
identificar o consumo desagregado, retornando a operação dos diversos
equipamentos no final do processo. Nesses casos, a informação final
pode ser utilizada como padrões a serem identificados posteriomente.
Para manter os equipamentos atualizados, novos períodos (possivelmente
menores ao período inicial) podem ser utilizados para garantir a
resposta adequada a possíveis alterações na presença ou utilização de
aparelhos. Na outra possibilidade, a otimização é realizada no nível
de ciclo da rede, onde, sabendo o padrão dos possíveis aparelhos
presentes, se busca a melhor combinação operativa que reflitam o sinal
observado;
\end{enumerate}

\subsection{Cálculo da eficiência}
\label{ssec:nilm_eff_calc}

\subsubsection{Padronização}
\label{top:nilm_padrao}

O estudo bibliográfico realizado por \cite{nilm_zeifman_review_2011}
teve dificuldades ao tentar comparar as diferentes técnicas utilizadas
nos \glspl{nilm}. O primeiro empecilho está na variedade das base
de dados utilizadas, possuindo aparelhos e estados de operações
bastante distintos, criando condições que podem previlegiar a
eficiência de um determinado \gls{nilm}. Para a unificação dos dados
estudados e permitir a comparação de performance entre os algoritmos
empregados nos \glspl{nilm}, foi disponibilizado por
\cite{nilm_dataset_blued_2012} um conjunto de dados públicos para
a análise, sendo aqui sugerida a sua utilização. O conjunto de dados
foi construido para representar a realidade de residências nos
\gls{eua} e por isso podem não corresponder a realidade brasileira,
mas o conjunto de dados servem como base para comparação com a
perfomance dos \glspl{nilm} aqui desenvolvidos com os do exterior,
assim como nada impede da utilização em paralelo de dados próprios.

Outra dificuldade foi o fato de autores utilizarem uma
medida própria para o cálculo das taxas de eficiência.  Além disso,
normalmente os autores não reportaram as taxas de falsos positivos na
Etapa~\ref{itm:etapa2}, apenas a capacidade dos algoritmos de
detectarem os eventos (excessões observadas são
\cite{nilm_marceau_16ssamp_improved_1999_18,nilm_liang_pt2_2010_40}).
Em outros casos, os autores concentraram-se apenas na capacidade dos
algoritmos da Etapa~\ref{itm:etapa3} de discriminarem equipamentos,
reportando medidas representativas para essa eficiência.

Por isso, \cite{nilm_zeifman_review_2011} recomenda a utilização das
medidas apresentadas por \cite{nilm_liang_pt1_2010_34}, no qual se
apresentou considerações metódicas para o tema. Foram apresentadas
três medidas. A primeira medida, \gls{det_eff}, considera a
capacidade do \gls{nilm} de desagregar a informação nos eventos que
foram detectados\footnote{Empregada por \cite{nilm_hart_1992_8} quando não
disponível a medição paralela de energia dos utensílios e por
\cite{nilm_gupta_patel_2010_30}.}. Quando
interessado apenas em estudar a capacidade do classificador para os
eventos detectados, a medida \gls{class_eff} deve ser 
utilizada. Finalmente, a \gls{total_eff} é
dada por \ref{eq:total_eff}, levando em conta apenas a capacidade do
\gls{nilm} de corretamente classificar os eventos
reais, causados pelos equipamentos na rede\footnote{Empregada por
\cite{nilm_patel_2007_29,nilm_berges_2009_36} pois a
Etapa~\ref{itm:etapa2} foi realizada
manualmente \label{fn:patel_manual} e, geralmente, por demais estudos
que estudaram apenas a Etapa~\ref{itm:etapa3}.}.

\begin{subequations}\label{eq:eff}
\begin{equation}\label{eq:det_eff}
\eta_{det} = \frac{N_{id}}{N_{real} + N_{fp} - N_{ni}}
\end{equation}
\begin{equation}\label{eq:class_eff}
\eta_{class} = \frac{N_{id}}{N_{real} - N_{ni}}
\end{equation}
\begin{equation}\label{eq:total_eff}
\eta_{total} = \frac{N_{id}}{N_{real}}
\end{equation}
\end{subequations}

\noindent onde:  

\begin{description}
\item[$N_{id}$] são eventos identificados, ou seja, corretamente
detectados e classificados pelo \gls{nilm}; 
\item[$N_{real}$] são os disturbios causados pelos equipamentos
na rede;
\item[$N_{fp}$] são eventos devido a falsos positivos, ou seja,
evento erroneamente identificados;
\item[$N_{ni}$] são eventos não identificados, ou perdas de alvo.
\end{description}

Segmenta-se \ref{eq:total_eff} para obter a eficiência do \gls{nilm}
por aparelho conforme:

\begin{equation}\label{eq:app_eff}
\eta_{total}^i\approx\frac{N_{id}^i}{N_{real}^i} ~~ \forall ~~ i =
1,2,...,N_{ap}
\end{equation}

\noindent onde $N_{id}^i$ e $N_{real}^i$ são os respectivos
$N_{id}$ e $N_{real}$ para o i-ésimo aparelho dos $N_{ap}$
disponíveis.

O grande favorecimento para essas medidas é sua simplicidade de serem
obtidas, porém algumas considerações podem ser feitas sobre elas.
Primeiro, a medida com maior sensibilidade à capacidade do \gls{nilm}
é a \ref{eq:det_eff}, uma vez que os valores por ela representados
levam em conta as perdas de alvo e os falsos positivos.
Segundo, as mesmas não levam em conta o consumo de energia dos
aparelhos, dando importância análoga para aparelhos com parcelas
pequenas ou grandes de consumo. Além disso, a correta identificação
dos $N_{id}$ não significa que a energia será corretamente
reconstruída, dependendo da capacidade do \gls{nilm} de unir essas
informações para gerar a informação do consumo desagregado.  Ainda,
como apontado por \cite{nilm_zeifman_review_2011}, elas apenas
representam a eficiência no ponto de operação, não sendo possível
observar como o \gls{nilm} se portaria para outros pontos. Indo além,
elas não permitem comparações de técnicas utilizadas exclusivamente para
as Etapas~\ref{itm:etapa2} e \ref{itm:etapa3}, impedindo a
contraposição de técnicas onde os autores se limitaram a uma dessas
etapas.

\subsubsection{Outras representações}
\label{top:outras_eff}

Por isso, além das medidas apontadas, outras maneiras de representar
a eficiência podem ser utilizadas para complementar o estudo do
comportamendo da abordagem utilizada. Uma técnica para representar o
compromisso entre a capacidade de detectar eventos e a quantidade de
falsos positivos encontrados é a curva \gls{roc}, também recomendada
por \cite{nilm_zeifman_review_2011}. A \gls{roc} além de ser utilizada
para expressar de maneira geral a capacidade do algoritmo de detectar
e identificar em função dos falsos positivos, pode ser utilizada para
estudar a eficiência específica da Etapa~\ref{itm:etapa2}. Já para a
Etapa~\ref{itm:etapa3}, a matriz de confusão permite entender quais
aparelhos ou classes de aparelhos são confundidos em outras classes,
assim como a eficiência de classificação em uma única representação.

As outras medidas utilizadas na literatura levantada são: o
percentual de classificações corretas por ciclo da rede, ou seja, a
correta classificação do estado de operação para cada ciclo dividido
pelo número total de ciclos
\cite{nilm_srinivasan_nn_2006_27,nilm_suzuki_2011_35}; porcentagem de
detecção de transitórios \cite{nilm_patel_2007_29}\footnote{O estudo
reportou eficiência para as Etapas~\ref{itm:etapa2} e \ref{itm:etapa3}
separadamente. Como foi dito na nota \ref{fn:patel_manual}, a
Etapa~\ref{itm:etapa3} utilizou eventos recortados manualmente.};
\gls{p_eff_i}
\cite{nilm_hart_1992_8,nilm_cole_data_extraction_1998_14,
nilm_cole_extra_info_surge_1998_15,nilm_farinaccio_16ssamp_1999_17,
nilm_marceau_16ssamp_improved_1999_18}; \gls{p_eff} 
\cite{2010_nilm_melhorando_pph_usa_37}; desvio
do tempo em que o aparelho foi identificado operando em relação ao
tempo que ele realmente estava operando
\cite{nilm_farinaccio_16ssamp_1999_17}\footnote{O estudo focou na
identificação de grandes cargas elétricas como ar condicionado e
aquecedores de água, modelados por \gls{c3}\label{fn:valc3}. Essa
medida seria limitada para outros modelos.}; porcentagem de detecção
de eventos de transição para ligado perdidos
\cite{nilm_farinaccio_16ssamp_1999_17}\fnref{fn:valc3}; erro médio
absoluto de reconstrução de energia e outras estatísticas por aparelho
\cite{nilm_powers_15minsamp_1991_16}.

A medida mais utilizada pelas referências, a \gls{p_eff_i}, embora
por elas não definida matematicamente, concebe-se que seja dada por:

\begin{subequations}
\begin{equation}\label{eq:frac_en_app}
\rho_{En}^i = \frac{E_{det}^i}{E_{real}^i} ~~ \forall ~~ 
i = 1,2,...,N_{ap}
\end{equation}
\begin{equation}\label{eq:frac_en}
\rho_{En} = \frac{\sum_{i}^{N_{ap}}E_{id}^i}{\sum_{i}^{N_{ap}}E_{real}^i} 
\end{equation}
\end{subequations}

\noindent onde $E_{det}^i$, $E_{real}^i$ é o consumo detectado e
consumo real do i-ésimo aparelho, respectivamente, o último sendo
obtido por submedição ou por um estimador. A \gls{p_eff_i}
pode ser generalizada para calcular a \gls{p_eff} através de
\ref{eq:frac_en}.  Essas medidas levam em consideração o consumo
detectado pelo \gls{nilm}, mas perdem a capacidade das medidas 
\ref{eq:eff} de representar a informação que foi corretamente
identificada. Por exemplo, se um equipamento é considerado como ligado
em um espaço de tempo em que o mesmo está desligado, isso irá
contribuir para corrigir possíveis erros que seriam atribuidos quando
o estado estimado e a operação estiverem na lógica oposta.

Assim, fica evidente que essas medidas precisam ser refinadas para
identificar os momentos nos quais a energia foi corretamente
reconstruída. Para isso, aqui se sugere o uso de \ref{eq:e_id_i} com
o intuíto de determinar a \gls{e_id_i}. A ideia
é representar que identificações do aparelho em outros estados, mas
com pequena diferença de energia, não irão afetar tanto na resolução
de energia, assim como resguardar que identificações em estados de
consumo maiores para os quais os aparelhos realmente operam não
arremeterão na conta de energia corretamente identificada:

\begin{equation}\label{eq:e_id_i}
E_{id}^i = E_{det}^i-\varepsilon^i
\end{equation}

A \gls{en_res} representa a ideia, em energia, para tanto falsos
positivos ou quanto identificações errôneas para estados de maior
consumo, ou seja, a parcela de $E_{det}^i$ que excede àquela lida por
um medidor externo ou estimada $E_{real}^i$. Ela pode ser descrita
por:

\begin{equation}\label{eq:en_res}
\varepsilon^i = \left\{\begin{array}{rl}
 E_{det}^i - E_{real}^i &\mbox{ se $E_{det}^i>E_{real}^i$} \\
 0 &\mbox{o.c.}
\end{array} \right. ~~ \forall ~~ i = 1,2,...,N_{ap}
\end{equation}

Isto posto, para obter a \gls{en_eff_i} e sua generalização,
\gls{en_eff}, basta empregar:

\begin{subequations}
\begin{equation}\label{eq:en_eff_i}
\eta_{En}^i = \frac{E_{id}^i}{E_{real}^i} ~~ \forall ~~ i =
1,2,...,N_{ap}
\end{equation}
\begin{equation}\label{eq:en_eff}
\eta_{En} = \frac{\sum_{i}^{N_{ap}}E_{id}^i}{\sum_{i}^{N_{ap}}E_{real}^i}
\end{equation}
\end{subequations}

E para as taxas de redundância:

\begin{subequations}
\begin{equation}\label{eq:p_red_i}
\rho_{red}^i = \frac{\varepsilon^i}{E_{real}^i} ~~ \forall ~~ i =
1,2,...,N_{ap}
\end{equation}
\begin{equation}\label{eq:p_red}
\rho_{red} = \frac{\sum_{i}^{N_{ap}}\varepsilon^i}{\sum_{i}^{N_{ap}}E_{real}^i}
\end{equation}
\end{subequations}

Posteriormente, descobriu-se que o próprio autor de
\cite{nilm_zeifman_review_2011} criou uma medida que permite também
exprimir a questão de energia redudante e corretamente detectada, e
uniu-as através da chamada medida-F (tradução própria de
\emph{F-measure}) \cite{nilm_zeifman_statistical_approach_2012}. A
mesma, \ref{eq:fmeasure}, é o quadrado da média geométrica normalizado
pela média aritmética entre duas outras grandezas: 

\begin{itemize}
\item parcela de energia atribuida ao aparelho que foi realmente
consumida pelo mesmo, um parâmetro parecido com a ideia de energia
redundante descrita por \ref{eq:en_recon};
\item parcela de energia que foi corretamente identificada em relação
ao consumo total do aparelho, a própria \gls{en_eff_i} aqui descrita
em \ref{eq:en_eff_i}.
\end{itemize}

\begin{equation}\label{eq:en_recon}
\eta_{En,prec}^i = \frac{E_{id}^i}{E_{det}^i}
\end{equation}

\begin{equation}\label{eq:fmeasure}
F^i=\frac{2 \eta_{En,prec}^i \eta_{En}^i}{\eta_{En,prec}^i+\eta_{En}^i}
\end{equation}

\subsection{Técnicas aplicadas}
\label{ssec:nilm_tecnicas}

As abordagens aplicadas desde o início dos estudos ao tema e
utilizadas como referências fizeram mão de ostensivas técnicas para a
desagregação do consumo. Cada vertente buscou extrair características
ou inovar aplicando outras técnicas, de forma que é possível observar
uma grande diversidade de abordagens. As abordagens serão agrupadas em
relação à \gls{fex} realizada. A capacidade de extrair
características dos sinais é correlacionada com a frequência de
amostragem e, em vista disso, dividir-se-á os métodos aplicados de
acordo com a taxa de amostragem utilizada. A ideia de subdivisão aqui
seguida foi de autoria da referência \cite{nilm_zeifman_review_2011}.

\subsubsection{1. Medição com Baixa Amostragem}
\label{top:nilm_baixa_am}

A utilização de características mascroscópicas de consumo do aparelho,
como alterações no patamar de consumo da rede, foi a
primeira abordagem encontrada ao tema. As mesmas podem ser obtidas sem
grande granularidade na taxa de amostragem, por isso, esse tipo de
abordagem beneficia-se de medidores de baixo custo, amplamente
disponíveis no mercado. No entanto, \cite{nilm_berges_2008_7} alerta
para discrepâncias entre medidores na ordem de 10\%-20\%, bem maiores
que aquelas alegadas, de 3\%. Os medidores testados no caso foram
\emph{Brand Meter I}, \emph{Watts Up? PRO} e \emph{EnerSure}.

A taxa de amostragem mais frequentemente utilizada é 1 Hz, entretanto
alguns estudos fizeram mão taxa de amostragem ainda menores por
desejarem identificar aparelhos que se ressaltam dentre os outros
devido ao seu relativo alto consumo, como ar condicionado, aquecedores
de água e geladeira. Exemplos de medidores utilizados no exterior são
\gls{ted} \cite{ted_site} e \emph{Watts up? PRO} \cite{wattsup_site},
o último sendo capaz de informar o consumo de \acrlong{q}.

\begin{enumerate}[label=\textbf{1.\arabic*},wide=\parindent]
\item \textbf{\Acrlong{p} e \Acrlong{q}}
\label{nilm:pot_real_reat}

\indent A referência inicial de grande destaque no tema,
\citet*{nilm_hart_1992_8}, ocorreu em 1992. Nela, utiliza-se medições de
\gls{p} e \gls{q} com uma taxa de amostragem de 1 Hz. A abordagem
aplica uma normalização para reduzir flutuações no consumo devido a
alterações na tensão de acordo com \ref{eq:norm_hart} com o intuíto de
reduzir disperções nos dados. O estudo de \citeauthor*{nilm_hart_1992_8}
limitou-se a identificar apenas cargas com potência maior a 150
\acrshort{watt}. A essência da metodologia ainda pode ser encontrada
em \glspl{nilm} mais atuais, sendo esta: 

\begin{equation} \label{eq:norm_hart}
P_{\text{norm}}(t) = \left[ \frac{120}{V(t)} \right]^2 P(t)
\end{equation}

\begin{enumerate}[label=\arabic*]
\item Detecta-se transitórios de consumo na rede devido a mudança de
estado de um utensílio através de alterações no consumo que devem
superar um limiar específico (15 \acrshort{watt}/\acrshort{var}) para os sinais
normalizados como em \ref{eq:norm_hart} para \acrshort{p} e \acrshort{q}.
As amostragens dentro de um regime permanente são normalizadas para
sua média com o objetivo de tirar o ruído. Para a \gls{fex}, usa-se o
degrau entre o regime permanente posterior e anterior (já no valor de
suas médias) ao evento transitório para \gls{p} e \gls{q};
\item Os eventos de transitório são analizados por um algoritmo de
agrupamento que irá gerar os centróides das mudanças de estado
possíveis causadas pelos utensílios no plano
\acrshort{dp}$\times$\acrshort{dq};
\item Centróides com simétria em relação aos eixos são tomados em
pares e com eles são criados modelos \gls{c3}. Para os centróides
remanescentes, além de regras heurísticas como a junção de centróides
próximos que permitam o pareamento com um outro refletido nos eixos,
determina-se possíveis combinações de centróides que possam formar uma
\gls{c2} utilizando uma adaptação do algoritmo de Viterbi
\cite{nilm_bouloutas_viterbi_ext_1991_11,
nilm_hart_fsm_viterbi_1993_12}. Assim que é determinada uma combinação
que permite a criação de uma \gls{c2}, os centróides da mesma são
removidos, e o processo continua até que todas as \glspl{fsm} tenham
sido construídas. A adaptação utilizada permite várias operações para
consertar corrupções e retornar uma estimativa ótima da \gls{fsm}
original. Como a reconstrução depende da estatística do processo, é
necessário que as mudanças de estado das \glspl{fsm} observadas tenham
um comportamento para que a \gls{fsm} original seja reconstruída, e
por isso, restringe-se apenas às \glspl{c2a}. As \glspl{c2b} podem ser
reconstruídas se houver conhecimento prévio da presença das mesmas, de
modo que elas sejam medidas operando em cada um de seus estados e
então inseridas manualmente no catálogo do \gls{nilm};
\item Em seguida é levantado o comportamento dos equipamentos,
montando o estados de consumo para cada aparelho. É utilizado um
algoritmo de força-bruta para corrigir ocorrências de dois ligamentos
ou desligamentos em sequência. A causa desses erros é, geralmente, a
ocorrência de um evento simultâneo de dois equipamentos. Assim, o
algoritmo busca por eventos não-usuais cuja soma é o valor de dois
outros eventos perdidos;
\item Finalmente, é levantada a estatística detalhando o
comportamento de consumo, como o tempo ligado e desligado de cada
equipamento. Essa informação, junto com a potência do equipamento é
utilizada para auxiliar a identificar o equipamento. 
\end{enumerate}

O método é robusto para o desagregamento de cargas \glspl{c3}
($> 150~$\acrshort{watt}) e as adaptações \cite{nilm_bouloutas_viterbi_ext_1991_11,
nilm_hart_fsm_viterbi_1993_12} do algoritmo de Viterbi
parecem resolver o problema das \glspl{c2a}, no entanto, essas
referências para o tratamento das \glspl{c2a} fundamentam a
matemática envolvida, mas não dão detalhes mais práticos da
implementação\footnote{Outras referências nesses artigos
não foram consultadas, podendo possuir essas informações.}.
Outro problema das metodologias envolvendo
algoritmos de agrupamento é a lenta alteração da resitência conforme a
operação do aparelho. Geralmente, ao interromper a operação, o
aparelho tem alterações no consumo na margem de 5\%-10\% em relação ao
inicio de operação \cite{nilm_sultanem_1991_10}.
\citeauthor*{nilm_hart_1992_8} observa o degrau geralmente é
menor em valor absoluto para os desligamentos nos casos de
equipamentos com motores, que reduzem o consumo conforme seu
aquecimento.

Uma estratégia bastante parecida é realizada por
\citet*{nilm_cole_data_extraction_1998_14,nilm_cole_extra_info_surge_1998_15}, 
onde são feitas considerações em relação as características de bordas e
inclinações. Aqueles são definidos como o auge atingido de potência
durante o acionamento e estes variações lentas de mudança no consumo.
Apesar de definir as bordas como o pico de potência, as referências
empregam as bordas apenas como os eventos de transição de consumo, não
empregando essa informação para classificação. A abordagem aplicada,
ao invés de agrupar os dados para depois procurar por possíveis
aparelhos como feito por \citeauthor*{nilm_hart_1992_8}, primeiro
busca temporalmente por ciclos fechados nos eventos de transição (ou
bordas, como na nomenclatura da referência), que depois serão
adicionadas aos centróides no espaço
\acrshort{dp}$\times$\acrshort{dq}. Se o centróide não existir, será
criado um candidato a centróide. Conforme a quantidade de ciclos dos
centróides aumenta, o mesmo irá se tornar um candidato a uma carga.
Para as cargas \glspl{c3}, a carga será aceita apenas se a detecção
das bordas ocorrerem repetidamente. Já para as cargas \glspl{c2a}, foi
realizado um estudo da probabilidade dela ter sido originada pela
sobreposição de duas bordas geradas por equipamentos distintos. A
conclusão foi que se forem encontrados mais de um ciclo de três bordas
em um período de 6 horas é suficiente para aceitá-lo como uma
\gls{c2a}. Finalmente, o envelope só foi considerado para a melhoria
em resolução de energia e, no entanto, a referência indica que a
utilização da média de consumo entre as bordas apresenta melhores
resoluções.

\item \textbf{\Acrlong{p}, \Acrlong{q} e Transitório}
\label{nilm:pot_real_trans}

Um trabalho paralelo ao de \citeauthor*{nilm_hart_1992_8} foi
realizado pelo mesmo instituto para aplicar o \gls{nilm} no setor
comercial, sendo realizado por
\citet*{nilm_norford_leeb_medianfilt_1996_13}. No setor comercial
são encontrados aparelhos com características diferentes ao setor
residencial, geralmente com transitórios mais lentos (podendo chegar a
cerca de centenas de segundos
\cite{nilm_norford_leeb_medianfilt_1996_13}), menor consumo reativo
devido às preocupações com a qualidade de energia e consequentemente
correção do fator de potência, e a presença de \gls{c5}, como exemplo,
na referência foi observada uma bomba com picos periódicos de 20
k\acrshort{watt}. Tipicamente há também uma maior presença de
equipamentos com cargas variáveis, \glspl{c4}, como motores de velocidade
variável. Assim, foi adicionado a informação do transitório da
envoltória em amostragens de 1~\acrshort{hz}, que por serem maiores,
ainda podem ser observados em amostragens baixas, suprindo, ao mesmo
tempo, a menor capacidade de discriminação da variável \gls{q} nesse
setor. Para redução dos ruídos, utilizou-se um filtro de mediana
com 11 pontos, esse sendo mais indicado para a eliminação dos picos
quando comparado aos filtros lineares, que terão dificuldades de
distinguir os picos e os degraus, uma vez que eles tem espectros de
frequência parecidos. É aplicada uma medida de distância entre as
sessões dos transitórios observados e os transitórios característicos
que, ao estarem dentro de um limiar, serão identificados como um
determinado equipamento.  Para o tratamento das \glspl{c4}, a
referência indica o emprego de variáveis de controle, quando
disponíveis, correlacionadas com o seu consumo para estimá-las, como o
caso para os equipamentos de \gls{avac} em geral.

\item \textbf{Unicamente \acrlong{p}}
\label{nilm:pot_real}

A medição de \acrlong{q} adiciona custo ao \gls{nilm} --- ainda que
não tão oneroso quanto medições em altas frequências --- e, para
detectar certos aparelhos com assinaturas de destaque na rede,
essa variável pode ser desnecessária. Em outros casos, medidores que
disponibilizam essa informação podem não estar disponíveis, sendo
possível operar apenas com a \acrlong{p}.

\begin{enumerate}[label*=.\textbf{\arabic*},wide=\parindent]
\item \textbf{Separação dos principais equipamentos por uso-final}

Exemplos do primeiro caso são os estudos de
\citet*{nilm_powers_15minsamp_1991_16,nilm_farinaccio_16ssamp_1999_17,
nilm_marceau_16ssamp_improved_1999_18,
nilm_zeifman_statistical_approach_resumo_2013},
para os quais os autores se preocuparam em identificar apenas
equipamentos de maior uso-final.
\cite{nilm_powers_15minsamp_1991_16,nilm_farinaccio_16ssamp_1999_17}
utilizaram somente regras heurísticas, enquanto
\cite{nilm_marceau_16ssamp_improved_1999_18} também empregou os
degraus em potencia real e um filtro para a detecção dos
ligamentos/desligamentos dos aparelhos. Finalmente,
\cite{nilm_zeifman_statistical_approach_resumo_2013} abordou o problema
estatísticamente.

\begin{itemize}[wide=\parindent]
\item \emph{Regras empíricas}

Em \cite{nilm_powers_15minsamp_1991_16}, foram reportadas a capacidade
de reconstrução para ar condicionado e aquecedores de água. A
amostragem é realizada a cada 15 minutos e os arquivos são analisados
dia a dia. Por ser proprietário, as regras não são detalhadas (é
utilizada uma árvore de decisões, embora o estudo considere a aplicação
de redes neurais), mas o algoritmo procura por picos no consumo, assim
como sua duração, tempo e magnetude, que são utilizados pelas regras
para determinar se os mesmos foram utilizados para os usos-finais
cobiçados. Posteriomente, eles são ajustados conforme
verificações de consistência. Para o ar condicionado, é relatado que o
valor de pico estimado médio para as residências difere cerca de
apenas 5\% do valor original médio, enquanto o consumo fica na margem
de 10\% e observa-se boa capacidade de estimar os horários de consumo.
 
Já os estudos \cite{nilm_farinaccio_16ssamp_1999_17,
nilm_marceau_16ssamp_improved_1999_18}, realizados por outro
grupo, empregaram amostragem de \acrshort{p} a cada 16~segundos. Os
aparelhos estudados foram: geladeira, aquecedor de água e aquecedores
de ambiente (este somente em
\cite{nilm_marceau_16ssamp_improved_1999_18}\footnote{O algoritmo da
referência \cite{nilm_marceau_16ssamp_improved_1999_18}
também leva em consideração a máquina de lavar roupa, mas os
resultados focaram apenas nos outros três aparelhos.}).
\cite{nilm_zeifman_review_2011} expõe a arbitrariedade e não
intuitividade das regras utilizadas em
\cite{nilm_farinaccio_16ssamp_1999_17}, que precisam ser estudadas
para cada caso de aparelho. Foram determinadas 8 regras para cada
aparelho (algumas regras são reaproveitadas entre aparelhos),
divididas em duas etapas: determinar o conjunto de eventos de
transição e a duração do consumo.  Em seguida, a duração de consumo é
multiplicada pela demanda média do aparelho durante a fase de
treinamento para obter o consumo estimado. A fase de treinamento,
período em que há medição paralela dos equipamentos, e, por isso,
ocorrendo intrusão da propriedade do consumidor, é feita para um
período de uma semana. A reconstrução de energia diária para os
equipamentos é na margem de $-10,5\%$ a $15,9\%$.

O estudo em sequêcia aperfeiçoou o anterior com uma abordagem única
para determinar os aparelhos em operação. Ele compara, em ordem
decrescente em termos de demanda média operativa, se a magnetude
do evento é próxima à média do nível operativo de um dos aparelhos
almejados, empregando como limiar de corte dois desvios padrões. Ainda
assim, a referência emprega diversas regras de pré/pós-processamento
determinadas empiricamente para melhorar a resolução em energia, assim
como também necessita do período de treinamento através de medição
paralela de 1~semana, limitando a aplicabilidade do método para uma
gama maior de equipamentos. Por outro lado, o método serve para o seu
próposito, obtendo reconstruções na faixa de 10\% para a maioria das
análises realizadas.

\item \emph{Utilização da estatística de uso}

A abordagem empregada cerca de 20 anos depois por
\citet*{nilm_zeifman_statistical_approach_resumo_2013} (os dois primeiros são os
autores do levantamento bibliográfico \cite{nilm_zeifman_review_2011}), 
operou em cima de dados amostrados por um mostrador de energia
domiciliar na taxa de 1~\acrshort{hz}, obtendo apenas amostragens da
\acrlong{p} (o medidor usado foi o \gls{ted}). Baseou-se no
conhecimento prévio de utilização dos aparelhos para encontrar aquele
mais representativo estatisticamente.  Para isso, o estudo utilizou o
conceito de máxima entropia. Também se limitou a identificar os
aparelhos de maior uso-final, no caso, os aparelhos de interesse são:
\begin{enumerate*}[label=\itshape\alph*\upshape)]
\item ar condicionado;
\item aquecedores de ambiente;
\item aquecimento de água doméstica;
\item \label{itm:iluminacao} iluminação;
\item geladeira;
\item secadores de roupa elétricos;
\item \label{itm:aparelhoeletronico} aparelhos eletrônicos;
\end{enumerate*} que representam em média 80\% do consumo residencial,
no caso, para os \gls{eua}. 

Com base em outros estudos, a referência faz o levantamento da
distribuição conjunta Beta para o consumo e tempo de operação dos
aparelhos tratando alguns casos específicos. Por exemplo, para os
secadores de roupa elétricos, são utilizados duas distribuições Beta,
uma para o ciclo do elemento de aquecimento e outra componente entre
os ciclos de secagem; enquanto para as lampadas, utiliza-se
dependência estatística para o consumo e tempo de duração em relação
ao ambiente para os quais elas operam, resultando em uma função de
probabilidade conjunta que é a mistura das funções de probabilidades
conjuntas para cada ambiente. Ainda, para melhorar a performance,
adiciona-se como característica as assinaturas específicas dos
aparelhos. O exemplo é a televisão em comparação com uma lampada, onde
aquele varia o seu consumo conforme flutuações na imagem e som,
enquanto este tem o consumo bastante estável. Essas
características ``finas'' dos aparelhos podem ser modeladas
matematicamente e empregadas em conjunto com o conhecimento prévio de
utilização dos aparelhos.

Os autores utilizaram o classificador \emph{Naïve Bayes} com base para os
sete uso-finais indicados, adicionados de uma probabilidade conjunta
uniforme de potência e tempo para as outras cargas possíveis. Os
resultados empregaram a medida-F (\ref{eq:fmeasure}), obtendo valores
de acurária para os uso-finais mais desafiantes na ordem de 65\% e
70\%, sendo os mesmos os itens \ref{itm:iluminacao} e
\ref{itm:aparelhoeletronico}, respectivamente. Já quando utilizando a
informação de características finais, essas mesmas acurácias se
elevam para 92\% e 90\%.  

\end{itemize}

\item \textbf{\Acrlong{q} não disponível}

\begin{itemize}[wide=\parindent]
\item \emph{A abordagem de 
\citeauthor*{nilm_baranski_genetic_base_2003_19}}

Com o intuíto de possibilitar a saturação da aplicação de \glspl{nilm}
na Alemanha, \citet*{nilm_baranski_genetic_base_2003_19,
nilm_baranski_genetic_detalhado_2004_20,nilm_baranski_summary_2004_21}
recorreram a leitura ótica dos medidores eletromecânicos
(o trabalho \cite{nilm_baranski_genetic_base_2003_19}, realizado em
2003, indica que mais de 99\% dos medidores desse país possuem essa
configuração) para obter as medições com frequência de
1~\acrshort{hz}. Por isso, apenas \gls{p} estava disponível para esses
estudos. 

Apesar das limitações, essa abordagem é uma referência de destaque
devido às diversas contribuições feitas. Para melhorar a capacidade de
discriminação entre os equipamentos, além da potência ativa, o estudo
adicionou como característica o pico de consumo para o evento de
transição, bem como o período que o mesmo leva para estabilizar
(apresentado em \cite{nilm_baranski_genetic_detalhado_2004_20}),
diferente de \cite{nilm_cole_data_extraction_1998_14,
nilm_cole_extra_info_surge_1998_15} que observou essas propriedades,
mas empregou somente a última com o intuíto de melhorar a
capacidade de reconstrução de energia. Ainda, o autor contribuiu
transformando a estratégia aplicada por
\citeauthor*{nilm_hart_1992_8} em uma Abordagem~\ref{itm:abordagem2}.

Tratar-se-á de detalhes das técnicas aplicadas pelos autores por dois
motivos:

\begin{itemize}
\item a técnica teve bons resultados apesar de utilização de pouca
informação, podendo ter melhores resultados quando alimentada com mais
informação e, por isso, sendo um possível caminho a ser percorrido;
\item os artigos não são de compreensão trivial, em especial para a
elucidação da adaptação do algoritmo de Viterbi\footnote{Para os
leitores que desejarem se aprofundar, também se recomenda a leitura de
\cite{nilm_bergman_distribuido_2011} antes dos artigos de
\citeauthor*{nilm_baranski_genetic_detalhado_2004_20}.}. 
\end{itemize}

Os leitores não interessados em detalhes técnicos podem
seguir para a próxima abordagem mantendo em mente que as contribuições
das técnicas foram: 

\begin{enumerate}
\item criação das \gls{fsm} por algoritmos genéticos
para reduzir o tempo de otimização; 
\item em seguida essas são otimizadas por \gls{es} para obter os
parâmetros da \gls{fsm} (tempo e consumo em cada estado); 
\item e finalmente, os modelos utilizam lógica \emph{fuzzy}, permitindo que
dois ou mais modelos sejam criados para um mesmo distúrbio na rede,
sendo depois escolhido o modelo que melhor se aplica.
\end{enumerate}

A estratégia começa com o agrupamento dos dados em centróides,
limitando-se a degraus acima de um limiar mínimo de potência (valores
aplicados de 50 \acrshort{watt} em
\cite{nilm_baranski_genetic_base_2003_19} e 80 \acrshort{watt}
\cite{nilm_baranski_genetic_detalhado_2004_20}). As abordagens em
\cite{nilm_baranski_genetic_base_2003_19,
nilm_baranski_genetic_detalhado_2004_20} se basearam no agrupamento
utilizando lógica \emph{fuzzy}, mas na última referência, além desse
método, cita-se o emprego de \gls{som} para essa etapa. A fim de
reduzir a complexidade do problema (a estimativa de eventos é de
$16.000$ por dia), o autor desconsidera os centróides com poucas
ocorrências, limitando-se a identificar apenas aparelhos com padrões
recorrentes. 

Na primeira abordagem \cite{nilm_baranski_genetic_base_2003_19},
\citeauthor*{nilm_baranski_genetic_base_2003_19} segmentaram a etapa
de modelar os aparelhos. A primeira modela as \glspl{c3} simplesmente
encontrando pares de centróides no espaço. Para validar os modelos
\glspl{c3} encontrados, é gerado a matriz de correlação cruzada
utilizando o estado de operação para os aparelhos em cada instante de
tempo. Se o aparelho $i$ e o aparelho $j$ forem na realidade uma
\gls{fsm}, espera-se $r_{ij}\approx1$, onde $r_{ij}$ indica a
frequência de operação do aparelho $j$ quando $i$ está operando,
associando, assim, $j$ com a operação $i$ (o corte utilizado é de
0,8). Já para as \gls{c2}, são criados todos os modelos que
juntos somam aproximadamente zero e seus centróides tem frequência de
eventos também próximos. Esses modelos são então validados
temporalmente, e junto com as \glspl{c3} são comparados com um
possível catálogo antigo a fim de atualizá-lo. Em seguida, uma rede
neural é treinada com os padrões encontrados para os aparelhos (o
autor cita como exemplo: tempo médio de consumo, consumo médio, número
de estados) para encontrar esses padrões na residência.

Essa abordagem é aprimorada em
\cite{nilm_baranski_genetic_detalhado_2004_20,nilm_baranski_summary_2004_21},
que ao invés de encontrar todos possíveis modelos de \gls{fsm},
faz a otimização das possíveis máquinas através de algoritmo genético.
Nessa abordagem não há a discriminação para a criação de \gls{c2} ou
\gls{c3}, a abordagem única utiliza $N_{ap}$ (o número de aparelhos
deve ser maior que o número de centróides, no entanto, não é
especificado um bom valor a ser utilizado) \glspl{fsm} para os quais e o
algoritmo genético fica encarregado de alterar valores binários em uma
matriz $\underline{X}$ representando se um determinado centróide
pertence, ou não, à \gls{fsm}. É possível que um mesmo centróide
pertença a mais de uma \gls{fsm}. São utilizados três critérios para
otimização: 

\begin{itemize}
\item minimização do valor absoluto de potência da soma dos
centróides pertencentes a \gls{fsm}; 
\item o item anterior, mas levando em conta a frequência de
eventos em cada centróide; 
\item e minimização do número de centróides em
cada \gls{fsm} (priorizando aparelhos com menos estados).
\end{itemize}

Uma adaptação do algoritmo de Viterbi é utilizada para encontrar os
modelos de \gls{fsm}. Com os modelos resultantes, é criado as
sequências de estado para elas supondo que as mesmas são \gls{c2a}.
Mais precisamente, os autores consideram que as sequências de estados
devem ser recorrentes com seus parâmetros em uma área limitada dentro
de seu valor esperado (os autores citam dois exemplos de parâmetros:
tempo de duração no determinado estado e a capacidade de reconstrução
de energia para o consumo estimado nos estados da \gls{fsm} em relação
ao consumido nos caminhos percorridos; mas não dá detalhes de quais
empregou) e apenas visitados uma vez em cada ciclo. Para isso, é
realizada uma otimização em dois tempos. Primeiro, encontra-se o
melhor caminho para aqueles que obedecem as restrições (consumo de
potência positiva e permanencia em um estado por um tempo não muito
longo), juntando os estados da máquina em um caminho de operação com a
melhor qualidade em relação aos parâmetros escolhidos. Os parâmetros
podem ser iniciados com os valores da mediana para todos os eventos
acoplados a \gls{fsm}. A qualidade é avaliada pela entropia de
\emph{Shannon}, \ref{eq:shannon}. Isso é repetido iterativamente
utilizando um algoritmo de \gls{es} até a convergência da qualidade.

\begin{equation}\label{eq:shannon}
Q_{shannon} = - \Delta{e_{i}} \log{|\Delta{e_{i}}|}
\end{equation}

Os melhores caminhos operativos para as \glspl{fsm} ainda precisam ser
resolvidos quanto aos centróides que pertencem a mais de um aparelho.
Para isso, \cite{nilm_baranski_summary_2004_21} cita resumidamente um
algoritmo de força bruta que irá investigar para cada sobreposição
qual caminho tem a melhor qualidade.

Os autores revelam que o método necessita de 5 a 10 dias para
encontrar os modelos dos utensílios típicos, enquanto dados diários
são suficientes para atualizar o catálogo de utensílios detectados em
cada residência. Os resultados mostram que os aparelhos de maiores
consumo, como geladeira, aquecedor elétrico (de fluxo) e fogão podem
ser detectados com eficiência.

\item \emph{\gls{dnilm}}

A abordagem de \citeauthor*{nilm_baranski_summary_2004_21} é a base
empregada para o trabalho de \citet*{nilm_bergman_distribuido_2011},
contando com a mesma sequência de criação através de algoritmo
genético e otimização das \gls{fsm}. Diferenças podem ser notadas
apenas para a técnica de agrupamento, que não utilizam as informações
de tempo nem o pico atingido no transitório, contudo, podem utilizar a
\gls{q} se o medidor da residência realizar essa medida. Além disso, o
agrupamento é realizado em tempo real, atualizando a média e desvio
padrão de cada centróide conforme os eventos ocorrem. Se o evento não
for atribuído a nenhum centróide, um novo centróide é criado. 

Entretanto a maior contribuição do trabalho
em questão é uma nova arquitetura, distruibuída, para o \gls{nilm}.
Nesse caso, diferente dos medidores eletromecânicos disponíveis para
\citeauthor*{nilm_baranski_summary_2004_21}, o trabalho opera com
medidores inteligentes (os medidores das redes inteligentes descritos
na Subsessão~\ref{ssec:ret_tec}) para a colheita de
dados. Os medidores inteligentes também servem como pontos de
processamento local, mas devido às limitações de processamento, apenas
a deteção e identificação dos eventos é realizado no mesmo. Por isso,
a geração das \glspl{fsm} é realizada em uma central, com maior
capacidade de processamento, no qual este envia os eventos de
transição para que aquela os processe e retorne um catálogo com as
\glspl{fsm} e seus padrões a serem encontrados. O catálogo é chamado
de tabela estática. Assim, o medidor fica encarregado apenas de
comparar, localmente, os distúrbios encontrados com o catálogo,
identificando assim os estados operativos dos aparelhos. O estados
operativos de cada aparelho é chamado de tabela dinâmica, e é
preenchida por um algoritmo adaptado para otimização do problema da
mochila (do inglês \emph{Knapsack problem}). A etapa de criação da
tabela estática, chamada de aprendizagem, é realizada devido a
critérios do controlador (iniciação pró-ativa) ou do medidor
(reativamente). No primeiro caso, o controlador atualiza as tabelas do
medidor se as mesmas expirarem. Já o medidor inteligente pode
requisitar um novo treinamento de acordo com um dos critérios:

\begin{itemize}
\item a diferença absoluta entre a soma da demanda real e estimada
está superior a um patamar;
\item uma \gls{fsm} muda de estado frequentemente, onde os patamares
aplicados para determinar se a mudança de estado é frequente dependem
do aparelho (é mais aceitável observar mudanças frequentes no ar
condicionado ou aquecedor do que em um carregador de bateria
veicular);
\item mais de um determinado número de \glspl{fsm} alteram de estado
em um único evento.
\end{itemize}

Uma das dificuldades do projeto está em ajustar o fluxo de dados. A
referência considera armazenar os dados em períodos de maior atividade
nas residências, enviando as alterações de estado posteriormente
conforme a rede de comunicação não estiver congestionada. Diversas
outras considerações são feitas em relação ao processo de
aprendizagem.

Os resultados reportados são em comparação com um \gls{nilm}
centralizado. O trabalho reservou-se a detectar aparelhos com consumo
superior a 1000 \acrshort{watt}. O \gls{nilm} centralizado recebe uma
tabela estática otimizada para todo o período, enquanto o \gls{dnilm}
recebe uma tabela treinada para o primeiro dia, podendo atualizá-la de
acordo com o critérios anteriormente citados. As diferenças de
acurácia entre o \gls{dnilm} e o \gls{nilm} centralizado ficaram entre
60\% e 90\%.

\item \emph{Otimização da tabela dinâmica}

O problema para a otimização da tabela dinâmica --- construção
temporal dos estados operativos dos aparelhos --- foi abordado por
\citet*{nilm_genetic_2013} (aparentemente sem conhecimento do trabalho
de \citeauthor{nilm_bergman_distribuido_2011}). Ao invés de um
algoritmo de um força bruta adaptado, aplicou-se um algoritmo genético
para realizar a otimização do problema da mochila. O artigo limitou-se
a estudar a performance do algoritmo em resolver o problema da mochila
e, por isso, considerou-se que são conhecidos os momentos de transição
e os consumos de cada aparelho.

A referência utilizou simulações de 2 horas, gerando aleatoriamente o
tempo de operação dos aparelhos e suas potências. Foram simuladas
diversas condições, variando o número de aparelhos, transições, a
presença de ruído e de equipamentos desconhecidos. As observações
foram: 

\begin{itemize}
\item quanto maior o \gls{nt} e \gls{nap}, menor é a eficiência de
detecção. Casos com pequenos números de \gls{nt} e \gls{nap} obteram
100\% de eficiência de detecção;
\item o algoritmo se comportou bem na presença de ruído, não sendo
bastante afetado;
\item no entanto, a presença de aparelhos desconhecidos ou aparelhos
com sobreposição de potências deterioram a performance do algoritmo.
\end{itemize}

\end{itemize}

\end{enumerate}

\end{enumerate}

\subsubsection{2. Medição com Alta Amostragem}
\label{top:nilm_alta_am}

Taxas ainda maiores de amostragem podem ser utilizadas, permitindo a utilização
de outras técnicas. A estratégia em
\cite{nilm_laughman_continuous_variables_2003_9} envolvem encontrar
transientes e subtrair as ondas pré e pós transiente de corrente em
regimente permantente. Aplica-se a Transformada de Fourier e
compara-se com um banco de dados. 

Utilizou-se em \cite{nilm_coppe_nascimento} uma amostragem de 13,6
\acrshort{hz}. Nesse trabalho se utiliza uma estrutura de árvore de
decisão para a discriminação. Uma pré-classificação é realizada
utilizando o \emph{hardware} para equipamentos com valores específicos
de \gls{p} e \gls{q}. Caso não seja um dos equipamentos de simples
discriminação, modifica-se a informação em uma série de etapas, nesta
ordem: aplica-se Transformada de Hilbert, Transformada Wavelet e por
último o Método de Burg. As características utilizadas são os picos 
para os níveis de detalhes do Método de Burg adicionados da \gls{fp},
esse último adicionado para melhorar a capacidade de discriminação.

A informação é apresentada a um
classificador que utiliza como padrão o vizinho mais próximo no banco
de dados, que realiza a discriminação em duas etapas, primeiro
selecionando um grupo genérico ao qual o equipamento pertence, para
depois selecionar o grupo específico. A seleção é feita através do
erro médio quadrático aplicado entre o banco de dados e o resultado do
espectro de Burg para cada nível de detalhe utilizado da Transformada
Wavelet. O equipamento do banco de dados que tiver maior número de
detalhes compatível com o sendo testado é o resultado do processo de
discriminação.

% \item multiplas tecnicas

% Colocar o zeifman et all 2011 Nonintrusive appliance load monitoring
% (NIALM) for energy control in residential buildings e o liang aqui.

\subsection{Discussão}
\label{ssec:nilm_mundo_padroes}

% Falar das c1 e c4 que só podem ser detectadas utilizando

% Ver patel smart grid sensing

A combinação de múltiplas técnicas para o reconhecimento de padrões é
benéfico para a eficácia do discriminador. Essa abordagem requer mais
um subpasso para combinar as respostas dos diversos discriminadores
para produzir uma resposta única --- questão conhecida como fusão de
\emph{fusão de informação}. Esse passo é discutido em
\cite{nilm_liang_pt1_2010_34}, nomeando o processo de \gls{cdm}, no
qual foram testadas as seguintes abordagens:

\begin{description}
\item \gls{mco}: escolha do candidato mais comum entre os
membros da comissão. É o processo mais trivial de ser executada
computacionalmente, realizando apenas a contagem de votos. Essa
abordagem pode criar soluções não-únicas devido ao empate na votação;
\item \gls{lur}:
\end{description}

\cite{nilm_zeifman_review_2011} levanta a hipótese da utilização de
técnicas adaptando a teoria de \emph{Dempster-Shafer} para realizar
tal tarefa, e cita \cite{information_fusion_basir_2007_40} como
exemplo. 
\cite{nilm_zeifman_review_2011} 

\section[A metodologia no CEPEL]{A metodologia no \acrshort{cepel}}
\label{sec:nilm_cepel}

Os estudos anteriores realizados no \gls{cepel}
\cite{nilm_cepel_alvaro,nilm_cepel_bezerra,nilm_cepel_aguiar}
utilizaram uma amostragem de 60 \acrshort{hz}, podendo ser considerada
como uma taxa de amostragem intermediária. Ela permite a extração 
da informação da onda causada pelo distúrbio transitório causado pelos
utensílios na rede, assim como as características macroscópicas já
detalhadas em \ref{top:nilm_baixa_am}. 

As primeiras abordagens se concentraram em explorar a informação da
onda envoltória da corrente --- e apenas da corrente --- que é
propagada para uma \gls{rna} no intuíto de diferenciar os equipamentos
em grupos pré-determinados de consumo. Os aparelhos eram agrupados
conforme a similiridade de suas envoltórias. A última versão da
\gls{rna} utilizou estes grupos:

% TODO Colocar os grupos do alvaro

Nesses trabalhos o medidor foi desenvolvido pelo próprio \gls{cepel},
de forma que \cite{nilm_cepel_aguiar} explorou, além dos resultados
da \gls{rna}, os aspectos necessários para manter a
qualidade da informação e compactá-la utilizando técnicas como
\gls{pca}. Já \cite{nilm_cepel_bezerra} forneceu ao sistema a
possibilidade de uma arquitetura distribuída de processamento, similar
àquela explicada em \cite{nilm_bergman_distribuido_2011}. Nesse
trabalho também foi analisado o uso de \gls{pcd} em comparação ao
pré-processamento por \gls{pca}. Esses estudos apenas englobaram o
caso de colheita de dados em residências monofásicas.

Em \cite{nilm_cepel_alvaro}, além da continuação do desenvolvimento da
\gls{rna}, implementou-se um detector automático de eventos de
ligamentos de equipamentos, sem generalizar para o caso de
desligamentos ou estudar a performance para mudança de estados em
aparelhos do tipo \gls{c2}. Esse estudo também foi limitado para
redes monofásicas. Foram comparados três detectores de eventos
transitórios diferentes, sendo eles:

% TODO Colocar técnicas utilizadas pelo alvaro para detecção de
% ligamentos

Devido à boa capacidade de identificação dos equipamento nos grupos de
cargas, essas referências utilizaram apenas a informação de
transitório. No entanto, a utilização do medidor do \gls{cepel}
começou a ser questionada uma vez que o seu desenvolvimento é mais um
fator a ser considerado no projeto, em especial quando expandindo o
mesmo para o caso trifásico.



