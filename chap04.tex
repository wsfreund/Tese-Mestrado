\chapter[Monitoramento Não-Invasivo de Cargas Elétricas]{\acrfull{nilm}}
\label{cap:nilm}

Este capítulo trata em detalhe as particularidades envolvidas no
desenvolvimento da tecnologia conhecida como \gls{nilm}, em especial
quanto aos aspectos técnicos envolvidos. A
Sessão~\ref{sec:nilm_aspec_gerais} é a única que não leva em conta os
aspectos técnicos, onde são apresentados a tecnologia é apresentada
sob seus pontos gerais. Em seguida, as diversas metodologias utilizadas
no mundo são utilizadas para um levantamento das metodologias
utilizadas e dos aspectos envolvidos no desenvolvimento da tecnologia
na Sessão~\ref{sec:nilm_mundo}. Finalmente, será apresentado na
Sessão~\ref{sec:nilm_cepel} a evolução dessa tecnologia no escopo
deste trabalho, desenvolvido em conjunto com o \gls{cepel}.

\section{Aspectos Gerais}
\label{sec:nilm_aspec_gerais}

O \gls{nilm} é uma alternativa para fornecer a informação de consumo
de energia elétrica desagregado por utensílio. Ao invés das técnicas
normalmente utilizadas --- onde se faz mão de sensores dispostos em cada
utensílio, esses enviando informações para uma central encarregada de
decodificá-las e, assim, identificar os utensílios que estão demandando
consumo na rede ---, o \gls{nilm} é um método em que não ocorre a intrusão
da propriedade do usuário, contendo geralmente apenas um medidor
central no fornecimento de energia dessa propriedade. O peso da
identificação dos utensílios é transferido da utilização de uma grande
quantidade de sensores e \emph{hardware} complexo para um
\emph{software} e algoritmos de processamento de sinais no intuíto de
realizar uma análise profunda das medições e desagregar as
informações.

Ainda que o peso da identificação esteja no \emph{software}, a
aptidão dos algoritmos implementados dependem da capacidade do
medidor de extrair informações da rede e das distorções causadas pelos
utensílios na mesma, de forma que quanto mais avançado for o
\emph{hardware} de medição, enviando uma quantidade e/ou frequência
maior de informação aos algoritmos encarregados de realizar a
discriminação dos rastros deixados na rede elétrica pelos aparelhos,
também maior será a capacidade dos mesmos de desagregar o consumo
específico dos utensílios.

Possíveis aplicações para esse dispositivo são: 

\begin{itemize}
\item auxiliar ou substituir as \glspl{pph} sem que seja necessário
causar incômodo ao consumidor devido a presença de medidores na
residencia, fornecendo assim dados com maior fidelidade em caracter
desagregado por utensílio e maior frequência para estudos de \gls{ee}
no consumo de eletricidade (Sessão~\ref{sec:ee_dificuldades}), em
especial para o setor residencial;
\item disponibilizar a informação desagregada para o fornecimento do
Retorno em Tempo Real Desagregado em programas de \gls{ee}. Os
programas em \gls{ee} utilizando retorno de informação são possíveis
fontes de redução de consumo nos grandes centros urbanos, no entanto,
são necessários estudos no Brasil para determinar seu potencial 
(Sessão~\ref{sec:ee_res_exp}). O Retorno em Tempo Real Desagregado
também pode ser utilizado por empresas que oferecem serviços de
redução de consumo (Subsessão~\ref{ssec:ret_tec}), ou mesmo por
consumidores que precisam ter algum tipo de controle sobre seu consumo
de energia;
\item no interesse das concessionárias, a informação desagregada pode
auxiliar os clientes a identificarem consumo não essencial durante o
horário de ponta, auxiliando no deslocamento de carga (ver item
\emph{Resposta de Ponta e Demanda versus Economia Fora de Ponta} na
Subsessão~\ref{ssec:ret_outros}), assim como identificar clientes com
maior potencial de redução de consumo durante esses períodos para
oferecer incentivos nesse sentido (Subssessão~\ref{ssec:ret_tec}); 
\item identificação de aparelhos defeituosos ou com consumo excessivo
de energia;
\end{itemize}

As possibilidades de aplicação como um meio de melhoria da \gls{ee} e
redução da intensidade elétrica tem elevado o interesse nesse assunto,
em especial nos países desenvolvidos, como uma forma de aliviar
a pressão de consumo nos grandes centros urbanos e na redução de
emissões de \gls{co2} \cite{nilm_zeifman_review_2011}.
As possibilidades de aplicação dessa tecnologia tem levado a gigantes
no setor eletrônico, como \emph{Intel} e \emph{Belkin}, a investirem
fortemente no desenvolvimento dessa tecnologia. Essa
alternativa é mais simples quando comparando com a automação
residencial por não requerer mudanças na produção dos eletrodomésticos
--- a automação requer comunicação nos dois sentidos (entre a
interface e o aparelho), tal como a capacidade de controle do
aparelho, de forma que se faz necessário adaptar aparelhos antigos e a
produção dos novos equipamentos com essas capacidades --- juntamente
com o fato de haver um relutância social quanto às residências
automatizadas, apesar de esforços governamentais e da mídia local
\cite{Lipoff_Automation_2010} (a referência estudou a falta de
interesse na automação residencial nos \gls{eua}). Não bastasse,
também se pode citar o nascimento de \emph{start-ups}
procurando espaço na corrida por esse novo mercado, como
\emph{GetEmme} \cite{getemme_site} e \emph{Navetas}
\cite{navetas_site}.

\section{As diversas metodologias utilizadas no mundo}
\label{sec:nilm_mundo}

A ideia de desagregar a informação não é nova, sendo possível
encontrar pesquisas nesse sentido datando da década de 1980. Apesar
disso, um extenso levantamento bibliográfico
\cite{nilm_hart_1992_8,nilm_bouloutas_viterbi_ext_1991_11,
nilm_hart_fsm_viterbi_1993_12,nilm_norford_leeb_medianfilt_1996_13,
nilm_cole_data_extraction_1998_14,nilm_cole_extra_info_surge_1998_15,
nilm_powers_15minsamp_1991_16,nilm_farinaccio_16ssamp_1999_17,
nilm_marceau_16ssamp_improved_1999_18,nilm_baranski_genetic_base_2003_19,
nilm_baranski_genetic_detalhado_2004_20,nilm_baranski_summary_2004_21,
nilm_matthews_overview_2008_22,nilm_laughman_continuous_variables_2003_9,
nilm_leeb_spectral_envelope_1995_23,nilm_lee_variable_speed_estimation_2005_24,
nilm_wichakool_2009_25,nilm_shaw_2008_26,nilm_srinivasan_nn_2006_27,
nilm_akbar_2007_28,nilm_patel_2007_29,nilm_gupta_patel_2010_30,
nilm_sultanem_1991_10,nilm_chan_2000_31,nilm_lee_2004_32,nilm_lam_2007_33,
nilm_liang_pt1_2010_34,nilm_suzuki_2011_35,nilm_berges_2008_7,
nilm_berges_2009_36,2010_nilm_melhorando_pph_usa_37,
nilm_liang_pt2_2010_40} realizado em \cite{nilm_zeifman_review_2011}
expõe que as técnicas aplicadas em \glspl{nilm} ou não são robustas no
sentido de atenderem especificamente a um grupo limitado de utensílios
estudados, ou apresentam acurácia marginal, mostrando que o processo
de desagregação da informação não é trivial. Indo além das
dificuldades técnicas, ressalta-se os pontos enfatizados nas
Subsessões \ref{ssec:asp_psic} e \ref{ssec:asp_visuais}, nas quais
foram evidenciadas as características multidisciplinares desse método
quando aplicado como um meio de economia de energia e intensificação
da \gls{ee}.

Dividiu-se esta sessão da seguinte maneira. As primeiras subsessões,
\ref{ssec:modelos_carga} e \ref{ssec:metodologia_generica}, referem-se
aos modelos de carga utilizados para os aparelhos eletrodomésticos e às
etapas usualmente envolvidas na desagregação da informação de
consumo quando utilizando \glspl{nilm} respectivamente. Uma
proposta de padronização para o cálculo de eficiência é apresentado na
Subsessão~\ref{ssec:nilm_eff_calc}. Em seguida, a
Subsessão~\ref{ssec:nilm_tecnicas} irá apresentar as diversas
abordagens utilizadas, sendo guiada na proposta de divisão das
técnicas e características feita por \cite{nilm_zeifman_review_2011}.
Finalmente, a discussão das informações levantas é realizada em
\ref{ssec:nilm_mundo_padroes}.

\subsection{Modelos de Carga}
\label{ssec:modelos_carga}

Os utensílios podem ser modelados devido às características de
comportamento de suas carga. A seguir encontram-se possíveis
características de carga elétrica dos eletrodomésticos. Os quatro
primeiros itens são modelos de cargas mutuamente exclusivos, enquanto
os dois seguintes podem ser incluídos, dependendo das propriedades dos
aparelhos ligados à rede, para caracterizar aparelhos potencialmente
dificultadores do processo de desagregação \cite[com adaptações]{
nilm_hart_1992_8,nilm_zeifman_review_2011,nilm_zeifman_nonintrusive_2011,
nilm_apresentacao_review_2011,nilm_liang_pt2_2010_40}:

\begin{itemize}
\item \textbf{\gls{c1}}: aparelhos que permanecem
ligados 24~h/dia, 7~dias/semana, com consumo de energia praticamente
constante. Ex.: detectores de fumaça, fontes de alimentação
constantemente ligadas à rede;
\item \textbf{\gls{c2}}: essa categoria é utilizada para identificar
aparelhos que passam um conjunto de estados definidos no qual os
ciclos de mudança de estados são repetidos com frequência suficiente
nos eventos diários ou semanais. Mesmo equipamentos que são apenas
ligados/desligados pelo usuário podem acabar sendo modelados como
\glspl{c2} por mudar seus patamares de consumo enquanto estiver
ligado. Ex.: máquinas de lavar roupa, máquina de lavar louça,
ventiladores de múltiplas velocidades (como fraco, médio, forte);
\item \textbf{\gls{c3}}: um caso partícular das \glspl{c2} ocorre
quando o aparelho pode ser modelado como tendo apenas dois estados:
ligado/desligado. Ex.: lampadas, torradeiras, bombas de água;
\item \textbf{\gls{c4}}: uma generalização das \glspl{c2}, onde
há uma infinidade de estados para os quais o aparelho pode operar.
Essa categoria pressupõe que o aparelho irá estabilizar o seu consumo
em um patamar após um período de tempo. Ex.: lampadas com
\emph{dimmer}, ferramentas elétricas (furadeiras, serras etc.).
\item \textbf{\gls{c5}}\footnote{As referências optaram por não
criar essas categorias uma vez que essas categorias são apenas
características das cargas. Por sua vez, as mesmas são citadas, no
mínimo, em \cite{nilm_zeifman_review_2011,nilm_liang_pt2_2010_40}
 como dificultadores no processo de desagregação e
por esse motivo preferiu-se adicionar diretamente essas categorias para
enfatizar e facilitar a identificação de cargas com essas
características.\label{fn:categoria_add}}: aparelhos que causam
distúrbios na rede continuamente devido a flutuações no seu nível de
consumo. Ex.: televisores, onde o brilho e cores na tela alteram seu
consumo; computadores, que alteram sua potência conforme a demanda dos
processadores e \emph{coolers}, consumindo mais quando o
usuário está realizando tarefas como, por exemplo, executando
algoritmos do mestrado, escutando música etc.;
cargas de demanda dinâmica tais como o ar condicionado --- esse quando com
o compressor ligado --- alteram seu consumo para baleancear a demanda
através de resposta em frequência. Dependendo da grandeza dos disturbios,
os mesmos são potenciais dificultadores à identificação de rastros
deixados por outros aparelhos, particularmente os de menor consumo,
como foi notado em \cite{nilm_liang_pt2_2010_40}. A
Figura~\ref{fig:ar_cond_dinamica} demonstra a demanda dinâmica para
esse aparelho. Outros exemplos de utensílios com motores que também
geram oscilações --- mas em ordem inferior ao ar condicionado --- são:
microondas, geladeira, desumidificador \cite{nilm_liang_pt2_2010_40};
\item \textbf{\gls{c6}}\fnref{fn:categoria_add}: apesar de não ser
uma característica de um utensílio \emph{per se} e nem constituir um
modelo de carga elétrica, estudos de performance de \glspl{nilm} podem
considerar quais equipamentos serão potencialmente vistos como se
fossem um mesmo equipamento por possuirem os mesmos padrões. Essa categoria
varia conforme os aparelhos ligados à rede e quais são as
características sendo extraídas. Por exemplo, um computador e uma
lampada incandescente possuem consumos semelhantes quando procurando
padrões no plano \acrshort{dp}$\times$\acrshort{dq}
\cite{nilm_laughman_continuous_variables_2003_9}, enquanto
equipamentos com motores de potências distintas podem não ser
desagregados quando apenas olhando para seus transitórios --- em
especial quando normalizados, o caso para \acrfull{rna}.
\end{itemize}

\begin{figure}[h!t]
\centering
\includegraphics[width=\textwidth]
{imagens/ArCondicionado-CargaDemandaDinamica_ComTextoImpr.pdf}
\caption[Exemplo de carga com demanda dinâmica: Ar Condicionado]
{Exemplo de carga com demanda dinâmica: Ar Condicionado. No caso, as
pequenas oscilações vistas são causadas pela demanda dinâmica do ar
condicionado que, ao detectar oscilações na frequência, responde
alterando seu consumo para manter o balanço na rede.}
\label{fig:ar_cond_dinamica}
\end{figure}

\subsection{Metodologia genérica}
\label{ssec:metodologia_generica}

O \gls{nilm} pode ser resumido em quatro etapas para o tratamento da
informação na rede elétrica e em três abordagens quanto às técnicas
empregadas: 

\subsubsection[Etapas]{Etapas \cite{nilm_matthews_overview_2008_22}}
\label{top:etapas}

\begin{enumerate}
\item\label{itm:etapa1} \textbf{\gls{fex}}: são extraídas
informações das amostragens realizadas. A diversidade de
características que podem ser extraídas depende da capacidade do
medidor. As características serão utilizadas tanto para a detecção de
eventos de transitório quanto na identificação de padrões dos
equipamentos. Em alguns casos, para reduzir o esforço de processamento
ou reduzir a necessidade de armazenamento, a \gls{fex} para a
identificação de equipamentos pode ser realizada ou armazenada somente
quando identificados os eventos na Etapa~\ref{itm:etapa2};
\item\label{itm:etapa2}\textbf{Detecção de eventos de
transitório}: identificar alterações causadas por utensílios na rede.
Essa etapa é necessária para identificar alterações no consumo de
equipamentos. Pode-se empregar limiares estáticos ou dinâmicos. Os
limiares dinâmicos permitem o ajuste de operação, reduzindo ou
aumentando a sensibilidade do detector conforme a presença de
equipamentos \acrshort{c5}. Em algumas topologias, essa etapa é realizada
depois da Etapa~\ref{itm:etapa3} retornando o estado de operação dos
equipamentos continuamente, sendo necessário procurar nessa informação
os eventos de transição;
\item\label{itm:etapa3}\textbf{Reconhecimento de padrões}: utilizar as
características pertinentes para o reconhecimento de padrões,
deduzindo, assim, qual foi o utensílio que causou o disturbio na rede
e qual seu novo consumo. É desejável que o algoritmo seja capaz de identificar
a ocorrência de novos padrões e reconhecê-los em suas próximas 
aparições pois a construção de um catálogo com todos os possíveis
eletrodomésticos é impraticável, se não impossível. Tal tarefa só será
possível com a capacidade dos \glspl{nilm} de incluirem novos
equipamentos ao catálogo. Diversas técnicas podem ser utilizadas em
conjunto para esta etapa;
\item\label{itm:etapa4}\textbf{Refinamento dos resultados}: após as
Etapas~\ref{itm:etapa2} e/ou \ref{itm:etapa3}, pode-se adicionar uma
etapa opcional para procurar por possíveis erros ou melhorias na
informação desagregada. Por exemplo, um aparelho que remanesce
consumindo energia da rede por dias enquanto sua operação normalmente
ocorre em intervalos curtos, como uma torradeira. Isso pode ocorrer
por falhas na Etapa~\ref{itm:etapa2}, onde o desligamento do
equipamento não foi encontrado, ou na Etapa~\ref{itm:etapa3}, na qual
o desligamento foi identificado como causado por outro equipamento.
Outra possível melhoria seria encontrar possíveis novos ciclos de
operações para aparelhos~\gls{c2}. As estratégias corretivas podem ser
meramente remediativas, ou seja, simplesmente ignorar alterações de
estados que permanecem em um estado de consumo durante um grande
período de tempo para melhorar a resolução em energia do \gls{nilm}.
\end{enumerate}

\subsubsection[Abordagens]{Abordagens \cite{nilm_zeifman_review_2011}}
\label{top:metodos}

\begin{enumerate}
\item\label{itm:metodo1}\textbf{Abordagem por otimização}: procura por
uma combinação de aparelhos cujo o sinal agregado resultante é a
melhor aproximação possível do sinal observado. A concentração dos
dados em longos períodos de tempo é utilizada para identificar o
consumo desagregado, retornando a operação dos diversos equipamentos
nesse período. Por isso, nesse tipo de abordagem, não é possível
retornar a informação do consumo por equipamento em tempo real, mas
seria possível utilizá-la para fornecer o Retorno de Indireto Diário
ou Semanal. Ainda assim, espera-se que esse método tenha maior
eficiência quando comparado com a Abordagem~\ref{itm:metodo2}, uma vez
que ele processa grandes quantidades de informação de uma só vez;
\item\label{itm:metodo2}\textbf{Abordagem por reconhecimento de
padrões}: nesse caso, pertubações encontradas são tratadas
instantaneamente, tratando as características extraídas constantemente
em busca de alterações no consumo dos aparelhos. Desta forma, os
eventos são detectados e seus padrões reconhecidos um a um,
depositando grande peso na capacidade dos algoritmos de reconhecer
padrões. Geralmente as técnicas sofrem por processos de treinamento
ou ajustes antes de serem aplicadas em uma etapa anterior, preparando
os algoritmos para os padrões que devem ser buscados na rede;
\item\label{itm:metodo3}\textbf{Abordagem intermediária}: outra
possibilidade é utilizar abordagens que fornecem uma solução parcial
em tempo real que depois será refinada por uma etapa adicional
utilizando a informação em um maior período de tempo, a
Etapa~\ref{itm:etapa4}.
\end{enumerate}

\subsection{Cálculo da eficiência}
\label{ssec:nilm_eff_calc}

\subsubsection{Padronização}
\label{sssec:nilm_padrao}

O estudo bibliográfico realizado por \cite{nilm_zeifman_review_2011}
teve dificuldades ao tentar comparar as diferentes técnicas utilizadas
nos \glspl{nilm}. O empecilho foi o fato de cada autor
utilizar uma métrica própria para o cálculo das taxas de eficiência.
Em especial, normalmente os autores não reportaram as taxas de falsos
positivos na Etapa~\ref{itm:etapa2}, apenas a capacidade dos
algoritmos de detectarem os eventos. Em outros casos, apenas a
capacidade de discriminação dos equipamentos é reportada, sendo
meramente uma das etapas necessária para a operação do \gls{nilm}.

Por isso, \cite{nilm_zeifman_review_2011} recomenda a utilização da
métrica apresentada por \cite{nilm_liang_pt1_2010_34}, onde se
apresentou considerações metódicas para o tema. Foram apresentadas
três métricas. A primeira métrica, \gls{det_eff} considera a
capacidade do \gls{nilm} de desagregar a informação nos eventos que
foram detectados. Quando interessado apenas em estudar a capacidade do
classificador, a métrica \gls{class_eff} deve ser utilizada.
Finalmente, a \gls{total_eff} é dada por \ref{eq:total_eff}, levando em
conta apenas a capacidade do \gls{nilm} de corretamente classificar os
eventos reais, causados pelos equipamentos na rede.

\begin{subequations}\label{eq:eff}
\begin{equation}\label{eq:det_eff}
\eta_{total} = \frac{N_{class}}{N_{reais} + N_{fp} - N_{ni}}
\end{equation}
\begin{equation}\label{eq:class_eff}
\eta_{class} = \frac{N_{class}}{N_{reais} - N_{ni}}
\end{equation}
\begin{equation}\label{eq:total_eff}
\eta_{geral} = \frac{N_{class}}{N_{reais}}
\end{equation}
\end{subequations}

\noindent onde:  

\begin{description}
\item[$N_{class}$] são eventos corretamente classificados pelo
\gls{nilm}; 
\item[$N_{reais}$] são os disturbios causados pelos equipamentos
na rede;
\item[$N_{fp}$] são eventos devido a falsos positivos, ou seja,
evento erroneamente identificados;
\item[$N_{ni}$] são eventos não identificados, ou perdas de alvo.
\end{description}

Segmenta-se \ref{eq:total_eff} para obter a eficiência do \gls{nilm}
por aparelho conforme,

\begin{equation}\label{eq:app_eff}
\eta_{geral}^i\approx\frac{N_{class}^i}{N_{real}^i} ~~ \forall ~~ i =
1,2,...,N_{ap}
\end{equation}

\noindent onde $N_{class}^i$ e $N_{true}^i$ são os respectivos
$N_{class}$ e $N_{real}$ para o i-ésimo aparelho dos $N_{ap}$
disponíveis.

Apesar dessas métricas serem simples de serem calculadas, algumas
considerações podem ser feitas sobre elas. Primeiro, as mesmas
não levam em conta o consumo de energia dos aparelhos, dando a mesma
importância para aparelhos com parcelas pequenas ou grandes de
consumo. Assim como se dá a mesma importância para os diferentes
equipamentos, a $N_{class}$ não significa que a energia será
corretamente reconstruída, dependendo da capacidade do \gls{nilm} de
unir essas informações para gerar a informação do consumo desagregado.
Ainda, como apontado por \cite{nilm_zeifman_review_2011},
elas apenas representam a eficiência no ponto de operação, não sendo
possível observar como o \gls{nilm} se portaria para outros pontos. 

\subsubsection{Outras representações}
\label{sssec:outras_eff}

Por isso, além das métricas apontadas, outras maneiras de representar
a eficiência podem ser utilizadas quando possível. Uma técnica para
representar a sensibilidade entre a capacidade detectar eventos e a
quantidade de falsos positivos encontrados é a curva \gls{roc} também
recomendada por \cite{nilm_zeifman_review_2011}.

Outras métricas na literatura incluem a fração de energia corretamente
identificada no nível global, fração de energia
desagregada corretamente por aparelho, que foram escritas
matematicamente por \ref{eq:frac_en} e \ref{eq:frac_en_app}. 

\begin{subequations}\label{eq:en_eff}
\begin{equation}\label{eq:frac_en}
\eta_{Energia} = \frac{E_{class}}{E_{real}}
\end{equation}
\begin{equation}\label{eq:frac_en_app}
\eta_{Energia}^i = \frac{E_{class}^i}{E_{real}^i} ~~ \forall ~~ 
i = 1,2,...,N_{ap}
\end{equation}
\end{subequations}

\noindent onde $E_{class}$ e $E_{real}$ são o consumo de energia
estimado e verdadeiramente consumido respectivamente. Já em
\ref{eq:frac_en}, os índices $i$ indicam a fração para o i-ésimo
equipamento.

Ainda cabe discutir como considerar o $E_{class}$ para equipamentos
\gls{c2}. Aqui se sugere o uso de \ref{eq:p_class} para determinar o
consumo corretamente classificado,

\begin{equation}\label{eq:p_class}
E_{class}^i = \frac{(E_{det}^i-\varepsilon^i)}{E_{real}^i}
\end{equation}

\noindent onde se apresenta a ideia de \gls{en_res}, o consumo de
energia atribuído a algum aparelho que não a consumiu. $E_{det}^i$
representa o consumo de energia detectada para o i-ésimo aparelho.

\begin{equation}\label{eq:en_res}
\varepsilon^i = \left\{\begin{array}{rl}
 E_{det}^i - E_{real}^i &\mbox{ se $E_{det}^i>E_{real}^i$} \\
 0 &\mbox{o.c.}
\end{array} \right. ~~ \forall ~~ i = 1,2,...,N_{ap}
\end{equation}


\subsection{Técnicas aplicadas}
\label{ssec:nilm_tecnicas}

As abordagens aplicadas desde o início dos estudos ao tema utilizadas
como referências fizeram mão de ostensivos métodos para realizar a
\gls{fex}. A capacidade dos algoritmos de extrair características é
correlacionada com a frequência de amostragem. Em vista disso,
dividir-se-á os métodos aplicados de acordo com a taxa de amostragem
utilizada. A ideia de subdivisão aqui seguida foi de autoria da
referência \cite{nilm_zeifman_review_2011}.

\subsubsection{Medição com Baixa Amostragem}
\label{sssec:nilm_baixa_am}

A utilização de características mascroscópicas de consumo do aparelho,
como alterações no patamar de utilização de energia da rede, foi a
primeira abordagem encontrada ao tema. Esse tipo de abordagem
beneficia-se de medidores de baixo custo, amplamente disponíveis no
mercado. A taxa de amostragem mais frequentemente utilizada é 1 Hz.
Exemplos de medidores utilizados no exterior para as medições são
\gls{ted} \cite{ted_site} e \emph{Watts up? PRO} \cite{wattsup_site},
o último sendo capaz de informar o consumo de potência reativa.

\paragraph{\acrlong{p} e \acrlong{q}}

A referência inicial de grande destaque no tema,
\cite{nilm_hart_1992_8}, ocorreu em 1992. Nela, utiliza-se medições de
\gls{p} e \gls{q} com uma taxa de amostragem de 1 Hz. A abordagem
utiliza uma normalização para reduzir flutuações no consumo devido a
alterações na tensão de acordo com \ref{eq:norm_hart} com o intuíto de
reduzir disperções nos dados. O estudo limitou-se a identificar apenas
cargas com potência maior a 150 \acrshort{watt}. A essência da
metodologia ainda pode ser encontrada em \glspl{nilm} mais atuais, sendo
esta: 

\begin{equation} \label{eq:norm_hart}
P_{\text{norm}}(t) = 
\begin{bmatrix}
\frac{120}{V(t)}
\end{bmatrix}^2
P(t)
\end{equation}

\begin{enumerate}
\item Detecta-se transitórios de consumo na rede devido a mudança de
estado de uma carga através de alterações no consumo (apenas a
variável \gls{p} é utilizada para a detecção de eventos) que devem
superar um limiar específico. As amostragens entre um evento
transitório e outro são modificadas para a média do consumo no
intervalo entre eles; 
\item Os eventos de transitório são analizados por um algoritmo de
agrupamento que irá gerar os centróides das mudanças de estado
possíveis causadas pelas cargas no plano
\acrshort{dp}$\times$\acrshort{dq}.
Centróides com valores \emph{absolutos} próximos são tomados em
pares e são criados modelos de carga para \gls{fsm} de dois
estados utilizando o par de centróide. Para os centróides remanescentes, determina-se possíveis
ciclos cujos os mesmos quando somados resultam em um ciclo fechado. Os
ciclos obtidos serão aparelhos modelados como \glspl{fsm} de multiplo
estados;
\item Com os modelos das cargas realizados, coloca-se etiquetas de
tempo nos eventos, de forma a determinar a sequência correta de
operação das maquinas de estados;
\item Em seguida é feito um levantamento da estatística detalhando o
comportamento de consumo, como o tempo ligado e desligado de cada
equipamento;
\item As estatítiscas de comportamento, sua potência e sequência de
estados são utilizados para identificar o equipamento.
\end{enumerate}

O método pode facilmente detectar as cargas do Tipo 2

A exata abordagem é aprimorada em
\cite{nilm_cole_extra_info_surge_1998_15}, tratando problemas de eventos de
duas cargas distintas ocorrendo simultaneamente. Esses eventos irão se
parecer com uma terceira carga, dificultando a identificação correta
do ligamento ou da sequência de operação dos estados, de forma que um
aparelho irá aparecer como ligado ou em um determinado estado por
dias. Para tratar desse problema, ele considera estatisticamente a
probabilidade de ocorrer um evento simultâneo, dependendo do número de
aparelhos, quantidade de eventos e taxa de amostragem.

Outra otimização é realizada nos métodos fornecidos por 
\cite{nilm_baranski_genetic_base_2003_19,nilm_baranski_genetic_detalhado_2004_20,
nilm_baranski_summary_2004_21}.
A primeira contribuição está em utilizar ao invés de casar um único $+\Delta{P}$ 
com outro -\acrshort{dp}, se realiza o casamento
de um conjunto de possíveis $+\Delta{P}_i$ com um outro possível grupo de
$-\Delta{P}_i$, reduzindo assim as dificuldades de eventos simultâneos.
Para melhorar a eficiência, esse casamento foi otimizado através 
de um algoritmo de força bruta. Em seguida a abordagem foi aprimorada,
utilizando um algoritmo de agrupamento com lógica \emph{fuzzy} para a criação
dos grupos correspondentes aos eventos e um algoritmo genético para otimização
dos casamentos.

Essa abordagem é utilizada similarmente para a otimização das
possíveis maquinas de estado em \cite{nilm_bergman_distribuido_2011},
contando também com algoritmo genético para a inicialização e
otimização das possíveis máquinas de estado. Entretanto a
contribuição desse trabalho é uma nova arquitetura, distruibuída, para
o \gls{nilm}. Nesse caso, é utilizado um medidor inteligente para a
colheita de dados e devido a limitações de processamento, apenas a
deteção e identificação dos eventos é realizado no mesmo. A geração
das possíveis máquinas de estado e aparelhos é realizada em uma
central, com maior capacidade de processamento, no qual este envia os
eventos de transição para que aquela gere as sequências de estado de
transição dos aparelhos com sua identificação em uma tabela.
Posteriormente a essa etapa, chamada de aprendizagem, a tabela é
enviada ao medidor inteligente que poderá identificar os eventos sem
grandes custos compotacionais. Se necessário o processo pode ser
novamente realizado para minimizar erros introduzidos por novos
aparelhos e/ou aparelhos não identificados ou utilizados durante a
última aprendizagem.

Há ainda a possibilidade de se explorar em baixa amostragem, mesmo que 
limitadamente, outras as características da onda, como o tempo do transiente
e seu valor de pico \cite{nilm_cole_extra_info_surge_1998_15}. Essas informações podem
auxiliar e minimizar os erros dos casamentos de eventos anteriores. O uso de
outras grandezas como \gls{q}, \gls{d} etc. pode ser realizada com o mesmo
objetivo. Um exemplo é \cite{nilm_bergman_distribuido_2011}, onde também se utilizou a 
\gls{q} quando disponível no medidor inteligente.

Indo além, as possibilidades também aumentam conforme se eleva a taxa
de amostragem. 

\subsubsection{Medição com Alta Amostragem}
\label{sssec:nilm_alta_am}

Taxas ainda maiores de amostragem podem ser utilizadas, permitindo a utilização
de outras técnicas. A estratégia em
\cite{nilm_laughman_continuous_variables_2003_9} envolvem encontrar
transientes e subtrair as ondas pré e pós transiente de corrente em
regimente permantente. Aplica-se a Transformada de Fourier e
compara-se com um banco de dados. 

Utilizou-se em \cite{nilm_coppe_nascimento} uma amostragem de 13,6
\acrshort{hz}. Nesse trabalho se utiliza uma estrutura de árvore de
decisão para a discriminação. Uma pré-classificação é realizada
utilizando o \emph{hardware} para equipamentos com valores específicos
de \gls{p} e \gls{q}. Caso não seja um dos equipamentos de simples
discriminação, modifica-se a informação em uma série de etapas, nesta
ordem: aplica-se Transformada de Hilbert, Transformada Wavelet e por
último o Método de Burg. As características utilizadas são os picos 
para os níveis de detalhes do Método de Burg adicionados da \gls{fp},
esse último adicionado para melhorar a capacidade de discriminação.

A informação é apresentada a um
classificador que utiliza como padrão o vizinho mais próximo no banco
de dados, que realiza a discriminação em duas etapas, primeiro
selecionando um grupo genérico ao qual o equipamento pertence, para
depois selecionar o grupo específico. A seleção é feita através do
erro médio quadrático aplicado entre o banco de dados e o resultado do
espectro de Burg para cada nível de detalhe utilizado da Transformada
Wavelet. O equipamento do banco de dados que tiver maior número de
detalhes compatível com o sendo testado é o resultado do processo de
discriminação.

\subsection{Discussão}
\label{ssec:nilm_mundo_padroes}

A combinação de múltiplas técnicas para o reconhecimento de padrões é
benéfico para a eficácia do discriminador. Essa abordagem requer mais
um subpasso para combinar as respostas dos diversos discriminadores
para produzir uma resposta única --- questão conhecida como fusão de
\emph{fusão de informação}. Esse passo é discutido em
\cite{nilm_liang_pt1_2010_34}, nomeando o processo de \gls{cdm}, no
qual foram testadas as seguintes abordagens:

\begin{description}
\item \gls{mco}: escolha do candidato mais comum entre os
membros da comissão. É o processo mais trivial de ser executada
computacionalmente, realizando apenas a contagem de votos. Essa
abordagem pode criar soluções não-únicas devido ao empate na votação;
\item \gls{lur}:
\end{description}

\cite{nilm_zeifman_review_2011} levanta a hipótese da utilização de
técnicas adaptando a teoria de \emph{Dempster-Shafer} para realizar
tal tarefa, e cita \cite{information_fusion_basir_2007_40} como
exemplo. 
\cite{nilm_zeifman_review_2011} 

\section{A metodologia no \acrshort{cepel}}
\label{sec:nilm_cepel}

Os estudos anteriores realizados no \gls{cepel}
\cite{nilm_cepel_alvaro,nilm_cepel_bezerra,nilm_cepel_aguiar}
utilizaram uma amostragem de 60 \acrshort{hz}, podendo ser considerada
como uma taxa de amostragem intermediária. Ela permite a extração 
da informação da onda causada pelo distúrbio transitório causado pelos
utensílios na rede, assim como as características macroscópicas já
detalhadas em \ref{sssec:nilm_baixa_am}. 

As primeiras abordagens se concentraram em explorar a informação da
onda envoltória da corrente --- e apenas da corrente --- que é
propagada para uma \gls{rna} no intuíto de diferenciar os equipamentos
em grupos pré-determinados de consumo. Os aparelhos eram agrupados
conforme a similiridade de suas envoltórias. A última versão da
\gls{rna} utilizou estes grupos:

% TODO Colocar os grupos do alvaro

Nesses trabalhos o medidor foi desenvolvido pelo próprio \gls{cepel},
de forma que \cite{nilm_cepel_aguiar} explorou, além dos resultados
da \gls{rna}, os aspectos necessários para manter a
qualidade da informação e compactá-la utilizando técnicas como
\gls{pca}. Já \cite{nilm_cepel_bezerra} forneceu ao sistema a
possibilidade de uma arquitetura distribuída de processamento, similar
àquela explicada em \cite{nilm_bergman_distribuido_2011}. Nesse
trabalho também foi analisado o uso de \gls{pcd} em comparação ao
pré-processamento por \gls{pca}. Esses estudos apenas englobaram o
caso de colheita de dados em residências monofásicas.

Em \cite{nilm_cepel_alvaro}, além da continuação do desenvolvimento da
\gls{rna}, implementou-se um detector automático de eventos de
ligamentos de equipamentos, sem generalizar para o caso de
desligamentos ou estudar a performance para mudança de estados em
aparelhos do tipo \gls{c2}. Esse estudo também foi limitado para
redes monofásicas. Foram comparados três detectores de eventos
transitórios diferentes, sendo eles:

% TODO Colocar técnicas utilizadas pelo alvaro para detecção de
% ligamentos

Devido à boa capacidade de identificação dos equipamento nos grupos de
cargas, essas referências utilizaram apenas a informação de
transitório. No entanto, a utilização do medidor do \gls{cepel}
começou a ser questionada uma vez que o seu desenvolvimento é mais um
fator a ser considerado no projeto, em especial quando expandindo o
mesmo para o caso trifásico.



