\chapter{Metodologia}
\label{chap:metodologia}

Neste capítulo está presente outras informações relevantes antes de
entrar nos méritos de resultados. A Sessão~\ref{sec:base_de_dados}
contém a base de dados utilizados no trabalho. A sessão seguinte
(Sessão~\ref{sec:aplic_es}) revela como o algoritmo genético foi
adaptado para otimizar o problema em questão, enquanto o \acs{som}
será discutido na Sessão~\ref{ssec:som}.

% Multi-espécies
\section{Descrição da base de dados}
\label{sec:base_de_dados}

Foram utilizados três conjuntos de dados fornecidos pelo \acs{cepel},
todos amostrados em situações controladas. Serão descritos as
características de cada um deles, sendo seus códigos para
identificação \emph{Temporizado} (Subsessão~\ref{ssec:temp}),
\emph{Empilhado4} (Subsseão~\ref{ssec:emp4}) e \emph{Empilhado7}
(Subsessão~\ref{ssec:emp7}).

\subsection{Conjunto de dados \emph{Temporizado}}
\label{ssec:temp}

\FloatBarrier
O conjunto de dados \emph{Temporizado} contém apenas cinco
equipamentos:

\begin{itemize}
\item Televisão LCD;
\item Geladeira;
\item Lâmpada fluorescente 23W, 54W;
\item Ventilador.
\end{itemize}

Sua medição foi realizada com o medidor \emph{Yokogawa}, e o perfil de
consumo dos aparelhos pode ser visualizada em
\ref{fig:temporizado_overview}.
Este conjunto é o que tem a maior presença de ruído, causado pela
televisão LCD, em especial para os períodos das 00:00 às 02:00 e 04:30
às 08:00 do dia 21.

A informação no gabarito pode ser observada nas figuras
\ref{fig:temporizado_app_time}--\ref{fig:temporizado_televisao}. 
A Figura~\ref{fig:temporizado_app_time} contém a informação do consumo
temporal dos aparelhos, enquanto a
Figura~\ref{fig:temporizado_app_pie} contém o gráfico circular do
consumo estimado no gabarito para os aparelhos. As figuras
\ref{fig:temporizado_geladeira}--\ref{fig:temporizado_televisao}
contêm os transitórios dos aparelhos marcados pelo usuário durante a
criação do gabarito. Todos os eventos são movidos para obterem média
zero de forma que a figura fique uniforme e seja possível compará-los.
A informação contida nesse gráfico auxilia a identificar eventos no
gabarito que fogem do padrão, seja por erro do usuário no
preenchimento, ou por caracterizar um evento excêntrico, facilitando a
identificação desses casos. Um exemplo pode ser observado na
Figura~\ref{fig:temporizado_ventilador}, onde há a ocorrência de um
evento que foge do padrão dos outros coletados. Nessas figuras também
permitem observar a quantidade de eventos para cada alteração de
estado (indicado entre parênteses no título das subfiguras).

\begin{sidewaysfigure}[p]
\centering
\includegraphics[width=\textwidth]{imagens/Temporizado_Overview.pdf}
\caption{Perfil de consumo para o conjunto de dados \emph{Temporizado}.}
\label{fig:temporizado_overview}
\end{sidewaysfigure}

\begin{sidewaysfigure}[p]
\centering
\includegraphics[width=\textwidth]{imagens/Temporizado_AppTime.pdf}
\caption{Informação no gabarito para o conjunto de dados
\emph{Temporizado}: consumo temporal dos aparelhos.}
\label{fig:temporizado_app_time}
\end{sidewaysfigure}

\begin{sidewaysfigure}[p]
\centering
\includegraphics[width=.5\textwidth]{imagens/Temporizado_AppPie.pdf}
\caption{Informação no gabarito para o conjunto de dados
\emph{Temporizado}: gráfico circular do consumo dos aparelhos.}
\label{fig:temporizado_app_pie}
\end{sidewaysfigure}

\begin{sidewaysfigure}[p]
\centering
\includegraphics[width=\textwidth]{imagens/Temporizado_App_Geladeira.pdf}
\caption{Informação no gabarito para o conjunto de dados
\emph{Temporizado}: envoltória para as diversas variáveis para a
geladeira.}
\label{fig:temporizado_geladeira}
\end{sidewaysfigure}

\begin{sidewaysfigure}[p]
\centering
\includegraphics[width=\textwidth]{imagens/Temporizado_App_Ventilador.pdf}
\caption{Informação no gabarito para o conjunto de dados
\emph{Temporizado}: envoltória para as diversas variáveis para a
ventilador.}
\label{fig:temporizado_ventilador}
\end{sidewaysfigure}

\begin{sidewaysfigure}[p]
\centering
\includegraphics[width=\textwidth]{imagens/Temporizado_App_LF23W.pdf}
\caption{Informação no gabarito para o conjunto de dados
\emph{Temporizado}: envoltória para as diversas variáveis para a
lâmpada fluorescente 23W.}
\label{fig:temporizado_lf23}
\end{sidewaysfigure}

\begin{sidewaysfigure}[p]
\centering
\includegraphics[width=\textwidth]{imagens/Temporizado_App_LF54W.pdf}
\caption{Informação no gabarito para o conjunto de dados
\emph{Temporizado}: envoltória para as diversas variáveis para a
lâmpada fluoresecente 54W.}
\label{fig:temporizado_lf54}
\end{sidewaysfigure}

\begin{sidewaysfigure}[p]
\centering
\includegraphics[width=\textwidth]{imagens/Temporizado_App_Televisao.pdf}
\caption{Informação no gabarito para o conjunto de dados
\emph{Temporizado}: envoltória para as diversas variáveis para a
televisão.}
\label{fig:temporizado_televisao}
\end{sidewaysfigure}

\FloatBarrier

\subsection{Conjunto de dados \emph{Empilhado4}}
\label{ssec:emp4}

O conjunto de dados \emph{Empilhado4} contém os seguintes
equipamentos:

\begin{itemize}
\item Forno elétrico;
\item Lâmpada fluorescente (LF) 25 W (2 unidades), 22W (2 unidades), 15W (3
unidades), 9W;
\item Lâmpada incandescente (LI) 40W (2 unidades);
\item Televisão CRT.
\end{itemize}

Sua medição foi realizada com o medidor do \acs{cepel}. Contém a maior
quantidade de alterações de estados de aparelhos de baixo consumo (no
caso as lâmpadas fluorescentes). O
perfil de seu consumo pode ser visualizado na
Figura~\ref{fig:empilhado4_overview}.

A informação no gabarito pode ser observada nas figuras
\ref{fig:empilhado4_app_time}--\ref{fig:empilhado4_app_pie}. 
A Figura~\ref{fig:empilhado4_app_time} contém a informação do consumo
temporal dos aparelhos, enquanto a Figura~\ref{fig:empilhado4_app_pie}
contém o gráfico circular do consumo estimado no gabarito para os
aparelhos.

%Figure~\ref{fig:empilhado4_app_pie} contém o gráfico circular do
%consumo estimado no gabarito para os aparelhos. As figuras
%\ref{fig:empilhado4_geladeira}--\ref{fig:empilhado4_televisao}

\begin{sidewaysfigure}[p]
\centering
\includegraphics[width=\textwidth]{imagens/Empilhado4_Overview.pdf}
\caption{Perfil de consumo para o conjunto de dados \emph{Empilhado4}.}
\label{fig:empilhado4_overview}
\end{sidewaysfigure}

\begin{sidewaysfigure}[p]
\centering
\includegraphics[width=\textwidth]{imagens/Empilhado4_AppTime.pdf}
\caption{Informação no gabarito para o conjunto de dados
\emph{Empilhado4}: consumo temporal dos aparelhos.}
\label{fig:empilhado4_app_time}
\end{sidewaysfigure}

\begin{sidewaysfigure}[p]
\centering
\includegraphics[width=\textwidth]{imagens/Empilhado4_AppPie.png}
\caption{Informação no gabarito para o conjunto de dados
\emph{Empilhado4}: gráfico circular do consumo dos aparelhos.}
\label{fig:empilhado4_app_pie}
\end{sidewaysfigure}

%\begin{sidewaysfigure}[p]
%\centering
%\includegraphics[width=\textwidth]{imagens/Empilhado4_App_Geladeira.pdf}
%\caption{Informação no gabarito para o conjunto de dados
%\emph{Empilhado4}: envoltória para as diversas variáveis para a
%geladeira.}
%\label{fig:empilhado4_geladeira}
%\end{sidewaysfigure}
\FloatBarrier

\subsection{Conjunto de dados \emph{Empilhado7}}
\label{ssec:emp7}

O conjunto de dados \emph{Empilhado7} contém os seguintes
equipamentos:

\begin{itemize}
\item Lâmpada incadescente (LI) 60 W, 100 W;
\item Lâmpada fluorescente (LF) 20 W, 21 W, 24 W, 26 W, 28 W, 40 W;
\item Secador de cabelo;
\item Ar condicionado;
\item Sanduicheira;
\item Geladeira (obs: essa geladeira tem consumo bastante superior
àquele utilizada no arquivo \emph{Temporizado});
\item Televisão CRT.
\end{itemize}

Sua medição foi realizada com o medidor do \acs{cepel}. Uma
peculiaridade desse arquivo é a distorção dos transitórios durante o
momento que o ar condicionado está operando.

A informação no gabarito pode ser observada nas figuras
\ref{fig:empilhado7_app_time}--\ref{fig:empilhado7_app_pie}. 
A Figura~\ref{fig:empilhado7_app_time} contém a informação do consumo
temporal dos aparelhos, enquanto a Figura~\ref{fig:empilhado7_app_pie}
contém o gráfico circular do consumo estimado no gabarito para os
aparelhos.


\begin{sidewaysfigure}[p]
\centering
\includegraphics[width=\textwidth]{imagens/Empilhado7_Overview.pdf}
\caption{Perfil de consumo para o conjunto de dados \emph{Empilhado7}.}
\label{fig:empilhado7_overview}
\end{sidewaysfigure}

\begin{sidewaysfigure}[p]
\centering
\includegraphics[width=\textwidth]{imagens/Empilhado7_AppTime.pdf}
\caption{Informação no gabarito para o conjunto de dados
\emph{Empilhado7}: consumo temporal dos aparelhos.}
\label{fig:empilhado7_app_time}
\end{sidewaysfigure}

\begin{sidewaysfigure}[p]
\centering
\includegraphics[width=\textwidth]{imagens/Empilhado7_AppPie.png}
\caption{Informação no gabarito para o conjunto de dados
\emph{Empilhado7}: gráfico circular do consumo dos aparelhos.}
\label{fig:empilhado7_app_pie}
\end{sidewaysfigure}

\FloatBarrier









\section[Aplicação do ES para Otimização do Detector de Eventos]{
Aplicação do \acf{es} para Otimização do Detector de Eventos}
\label{sec:aplic_es}

Para o ajuste automático, utilizou-se o ambiente de análise detalhado
no Capítulo~\ref{chap:framework}. Entretanto, algumas questões ainda
estavam em aberto. Além dos caminhos mostrados na
Figura~\ref{fig:cepel_transitorio}, foi implementado uma nova versão
para a remoção de eventos próximos utilizando a média de seus centros
(ver p.~\pageref{text:media}). Outra maneira de remoção dos eventos
também foi adicionada utilizando o conceito de incosistência (ver
p.~\pageref{text:incosistentes}). Uma estratégia em força bruta ---
aqui se referindo a otimizar todas as possíveis configurações e
identificar a melhor convergência delas --- não parecia ser a melhor
maneira de abordar o problema. Percebeu-se a necessidade de realizar a
escolha de algumas configurações a serem testadas para reduzir a
quantidade de caminhos possíveis.

Durante a realização do \emph{Ajuste Manual}, percebeu-se que os
eventos removidos devido à incosistência eram sempre eventos de falso
alarme, mas não havia a ocorrência de eventos de detecção causados por
esse tipo de remoção. Por isso, determinou-se que a remoção de eventos de
inconsistentes seria sempre realizada. Já para o caso da remoção de
eventos devido à ruído era importante para remoção de pertubações
rápidas geradas na rede que não constituiam na mudança do patamar
operativo da rede, e com base nisso se determinou que a remoção de
eventos ruidosos sempre seria realizada. Assim, essa variável
($\Delta I_{min}$) sempre estará presente no material genético das
espécies, bem como o $\sigma$ da Gaussiana a ser utilizada no filtro
de derivada de Gaussiana e o seu valor de corte $\delta_{min}$. Já
para o caso da remoção de eventos próximos, não era possível
determinar se sua utilização era necessária, nem qual das versões de
remoção --- por média, ou sem deslocamento --- era o que mais se
adequada ao problema. No caso de não usar corte, a variável $n_{min}$
não seria utilizada, porém no caso oposto as espécies teriam gene a
mais, cujo fenótipo é inteiro. Por outro lado, sua representação é no
conjunto real, sendo necessário determinar como codificar a
informação. Simplesmente se escolheu arredondar o valor na
representação para obter o valor do fenótipo. Finalmente, também não
se sabia \emph{a priori} determinar se a ordem influenciaria na
importância, e apesar de não se esperar grandes diferenças devido a
mudança da ordem em que eles seriam removidos, decidiu-se testar ambas
configurações.  

Assim, para essas determinações, haviam cinco caminhos a serem
percorridos: 

\begin{enumerate}[label=(\Roman*)]
\item Com remoção de eventos próximos utilizando sem deslocamento
\emph{depois} de remover eventos ruidosos;
\item Com remoção de eventos próximos utilizando a média 
\emph{depois} de remover eventos ruidosos;
\item Com remoção de eventos próximos utilizando sem deslocamento
\emph{antes} de remover eventos ruidosos;
\item Com remoção de eventos próximos utilizando a média 
\emph{antes} de remover eventos ruidosos;
\item Sem remoção de eventos próximos (essa espécie tem um gene a
menos que as espécies anteriores); 
\end{enumerate}

Além da questão dos caminhos, era necessário determinar a
inicialização. Para garantir melhor convergência, empregou-se
fronteiras mínimas e máximas para cada um dos valores representados
pelos genes. A inicialização foi aleatória uniforme dentro dessas
fronteiras. Para o parâmetro da estratégia evolutiva, porém,
iniciou-se esse valor de acordo com \ref{eq:sigma_init}, garantindo
que inicialmente cada individuo tenha, em média, 95\% de chance de
explorar uma região no interior a 10\% do espaço de solução permitido. Os
limites inferiores e superiores foram determinados por
\ref{eq:sigma_min} e \ref{eq:sigma_max}, respectivamente. Os valores
de $sigma_{min,i}$ e $sigma_{max,i}$ garante que no mínimo o material
genético irá sofrer pertubações dentro da região de 0,01\% da região
de solução em 67\% dos casos, enquanto no máximo ele irá pertubar o
material genético em 30\% da região para a mesma probabilidade.

\begin{subequations}
\begin{equation}\label{eq:sigma_init}
2\sigma_{init,i}=0,1\Delta x_i
\end{equation}
\begin{equation}\label{eq:sigma_min}
\sigma_{min,i}=1\times10^{-4}\Delta x_i
\end{equation}
\begin{equation}\label{eq:sigma_max}
\sigma_{max,i}=1\times10^{+3}\sigma_{min,i}
\end{equation}
\end{subequations}

\noindent onde $\Delta x_i$ é a região do espaço dentro das fronteiras
para a i-ésima variável.

Finalmente, cabe ainda decidir qual dos métodos de competição para as
espécies será utilizado. Como a implementação do método de competição
intraespécies acaba favorecendo espécies com carga genética mais
simples de serem otimizadas ou que começaram em condições 
privilegiadas --- quando sem o mecanismo de adiamento da competição
---, irá optar-se pela competição intraespécie.

\section[Mapas Auto-Organizáveis]{\acl{som}}
\label{ssec:som}

Para a construção do modelo neural será utilizado o aprendizado não
supervisionado. O \acf{som}\footnote{Também conhecido como mapa de
\emph{Kohonen}.} \cite[cap 9]{haykin1999neural} é um modelo de rede que
agrupa eventos similares em regiões do mapa ao realizar um mapeamento
do $\Re_{n} \rightarrow \Re_{k} | k \in {1,2}$. Inicialmente, gera-se
uma grade bidimencional $m \times n$ neurônios.

Os neurônios são unidades que recebem as entradas da rede e se
movimentam no decorrer do treinamento. Cada neurônio possui uma
coordenada no espaço dada pelo vetor peso $w$. Cada neurônio é
posicionado aleatoriamente 
	 
Após a criação e a inicialização, aplica-se em todos os neurônios um
evento do banco. A Figura~\ref{fig:som_representacao} mostra esse
processo. 

\begin{figure}[h!tb]
\centering
\includegraphics[width=6cm]{imagens/kohonen.pdf}
\caption[Representação da grade bidimensional de um SOM]{
Representação da grade bidimensional de um \acl{som}. O evento $x$
aplicado em todos os neurônios (circulos) do mapa e a atuação da
função de vizinhança em torno do neurônio mais similar ao evento
(circulo em rosa claro).}
\label{fig:som_representacao}
\end{figure}

Em seguida, calcula-se a métrica Euclidiana (\ref{eq:d_i_som}) entre o
evento e cada peso (neurônio) da grade. 

\begin{equation}\label{eq:d_i_som}
d_{i} = \sqrt{\sum_{n=1}^{N}(w_{ni}-x_{n})^{2}}
\end{equation}

\noindent onde $w_{i}$ é o vetor peso do neurônio $i$ dado por $wi =
[w_{i1},w_{i2},...,w_{ni}]^{T}$ e $x$ é o vetor evento dado por $x =
[x_{1},x_{2},...,x_{n}]^{T}$ . 

O neurônio cuja métrica Euclidiana será o neurônio
vencedor, chamado de \gls{bmu}. Esse neurônio
$j$ será o centro do raio da função de vizinhança que será empregada
para atualizar os vetores $w$ ao longo do treinamento apartir de
\ref{eq:som_atualiza}.

\begin{equation}\label{eq:som_atualiza}
w_{i}(t+1)=w{i}(t)+\eta(t)h_{ij}(t)(x(t)-w_{i}(t))
\end{equation}

\noindent onde $\eta(t)$ é a taxa de aprendizado que decresce monotonicamente e
$h_{ij}(t)$ é a função de vizinhança que tem seu valor máximo no
neurônio $j$ (\acs{bmu}), diminui conforme se afasta deste e abrange os
neurônios que estão dentro do raio que diminui com relação ao tempo
(interação). 

A função de vizinhança é definida na equação \ref{eq:som_vizinhanca}.

\begin{subequations}
\begin{equation}\label{eq:som_vizinhanca}
h_{ij}=e^{\frac{-d_{ij}^{2}}{2\sigma^{2}(t)}}
\end{equation}
\begin{equation}\label{eq:sigma_t}
\sigma(t)=\sigma(0)e^{{-\frac{t}{\tau}}}
\end{equation}
\end{subequations}

\noindent onde $d_{ij}$ é a distância Euclidiana entre o peso $w_{i}$ e o
neurônio $j$ ($BMU$) e $\sigma(t)$ definido por \ref{eq:sigma_t}.

Para $\sigma(0)$ igual ao raio definido no inicio do treinamento e
$\tau$ uma constante de tempo de decaimento da função. Na
Figura~\ref{fig:vizinhanca_grade} mostra a atuação da função de
vizinhança no treinamento com o \acs{bmu} ao centro.

\begin{figure}[h!tb]
\centering
\includegraphics[width=.5\textwidth]{imagens/vizinhanca_func.png}
\caption{Atuação da função de vizinhança na grade.}
\label{fig:vizinhanca_grade}
\end{figure}

Ao invés de se atualizar os pesos da rede a cada evento (treinamento
sequencial), utiliza-se o conceito de batelada. O treinamento em
batelada atualiza os pesos após um conjunto de eventos, geralmente
toda a base, for apresentada a rede. A cada passo da interação, cada
evento da base de dados é associado ao seu \acs{bmu}. Ao final desse
processo calcula-se o somatório para cada neurônio do mapa.

\begin{equation}\label{eq:sum_centroide}
s_{i}=\sum_{k=1}^{\eta_{i}}x_{k}
\end{equation}

\noindent onde $\eta_{i}$ é o número total de eventos associados a
cada neurônio e $x_{k}$ é o vetor evento associado. 

Em seguida os pesos dos neurônios do mapa são atualizados pela
equação~\ref{eq:som_wi}, sendo $m$ o número de neurônios do mapa.

%EQ 7
\begin{equation}
\label{eq:som_wi}
w_{i} = \frac{\sum_{j=1}^{m}h_{ij}(t)s_{j}}{\sum_{j=1}^{m}\eta_{i}h_{ij}(t)} 
\end{equation}

Portanto, esse treinamento utiliza médias ponderadas para atualizar os
pesos. Esse efeito, torna o treinamento mais suave a medida que ocorre
a redução da função de vizinhança. 

\subsection{Definições de Treinamento}
\label{sec:som_treinamento}

O mapa foi definido com uma grade de tamanho de $20 \times 15$ (300
neurônios) com disposição hexagonal. Para o algoritmo de treinamento
iremos utilizar o conceito de batelada apresentado na sessão anterior.
O processo de treinamento dar-se-á em duas etapas. A primeira fase é
configurada para rodar um número longo de épocas de treinamento
definido como 2000 épocas, e utilizar um raio inicial da função de
vizinhança mais abrangente, definido como 8. A segunda etapa do
treinamento é chamada de ajuste fino. Essa etapa possui um raio
inicial de 2 e um número de épocas mais curto definido como 500
épocas. Essa segunda etapa tem como objetivo tornar os agrupamentos
formados na primeira fase mais bem definidos.

Devido a grande variância das variáveis utilizadas para compor o
treinamento da rede, foram analisados dois tipos de normalizações. A
primeira divide cada variação de potência do evento por 127V. A
segunda representa o mesmo processo, porém divide as potências pela
tensão coletada no momento da medição das variações de transição do
aparelho. Vale lembrar que um evento é composto pelas variações de
potência real, reativa e harmônica e a variação de corrente.
Totalizando quatro variáveis coletadas por evento.

\subsection{Agrupamento por Centróides dos \acl{som}}

Após o treinamento do mapa, avaliou-se o \gls{qe}, o
\gls{te} e a \gls{matrizu}. O \gls{qe} é a média
dos erros dos $N$ casos aplicados no \acs{som} dado pela diferença entre o
vetor evento $x_{k}$ e o $w_{BMU}$ do seu respectivo \gls{bmu}. O
\gls{qe} é definido pela equação \ref{eq:qe}.

\begin{equation}\label{eq:qe}
Q_{e} = \frac{1}{N}\sum_{k=1}^{N}\|x_{k}-w_{BMU}\|
\end{equation}

Esse erro avalia o quanto o mapeamento se aproxima dos padrões da
entrada. Por outro lado, o \acs{te} (\ref{eq:te}) pode ser
interpretado como o erro da copia de informação da alta dimensão para
o espaço bidimensional. Dado dois neurônios que se aproximam da
entrada $x_{k}$, $BMU_{1}$ e $BMU_{2}$ .

\begin{subequations}
\begin{equation}\label{eq:te}
T_{e} = \frac{1}{N}\sum_{k=1}^{N}u(x_{x})
\end{equation}
\begin{equation}
u(x_{k}) = \left\{\begin{array}{rl}
 1 &\text{ se } BMU_{1} \;\; \text{e} \;\; BMU_{2} \;\; \text{não são vizinhos} \\
 0 &\mbox{o.c.}\end{array}\right.
\end{equation}
\end{subequations}

Dado um neurônio $j$ e o seu respectivo peso $w_{j}$ , a \acs{matrizu}
calcula a distância dos pesos dos neurônios vizinhos com relação a
$w_{j}$. O resultado é uma imagem onde cada hexágono pode ser
interpretado como a distância calculada. 
%Cores quentes indicam que os
%pesos estão afastados e cores frias indicam que estão próximos nessas
%regiões.
	
O algorítmo de \emph{k-means} é utilizado para rotular os neurônios do
mapa, já treinado, em $n$ centróides. Embora estejam disponíveis 13
tipos de diferentes aparelhos, o algoritmo será executado até
encontrar um número de centróides que atenda a disperção dos
diferentes tipos de aparelhos e transientes no mapa. Um número pequeno
de agrupamentos pode acabar englobando vários aparelhos dentro de um
mesmo centróide.  Por outro lado, um número muito grande de centróides
pode acabar particionando em diversas partes um agrupamento que contém
um tipo de aparelho, que cria uma resolução descenessária e dificulta
o entendimento do mapa. Um bom exemplo é o conjunto de transientes de
lâmpadas fluorescentes de baixa potência que é formado por diversas
potências, menores a 25W, que na prática representam as mesmas
características.
% FIXME Avaliar a parte de cima
	
O índice de \emph{Davies-Bouldin} avalia a similaridade entre
agrupamentos e é definido pela equação \ref{eq:bouldin}

\begin{subequations}
\begin{equation}
\label{eq:bouldin}
I_{DB} = \frac{1}{C}\sum_{k=1}^{C}max_{l\neq
k}{\frac{S_{c}(Q_{k})+S_{c}(Q_{l})}{d_{ce}(Q_{k},Q_{l})}}
\end{equation}
\begin{equation} \label{eq:intracentroides}
S_{c} = \frac{1}{N_{k}}\sum_{i}^{N_{k}}\|x_{i}-c_{k}\|
\end{equation}
\begin{equation} \label{eq:distancia_centroides}
d_{ce} = \|c_{k}-c_{l}\|
\end{equation}
\end{subequations}

\noindent onde: 

\begin{description}
\item [$Q$] é um centróide; 
\item [$C$] é o número de centróides; 
\item [$S_{c}$] é a medida de similaridade intracentróides dada pela equação
\ref{eq:intracentroides};
\item [$N_{k}$] o número de eventos pertencentes a cada centróide de
centróide $c_{k}$;
\item [$d_{ce}$] é a distância entre os
centróides dada pela equação \ref{eq:distancia_centroides}. Quanto
menor for esse índice, mais separados e bem definidos se encontram os
centróides formados.
\end{description}

