\chapter{Metodologia}
\label{chap:metodologia}

Neste capítulo está presente a descrição da base de dados
(Seção~\ref{sec:base_de_dados}) e como será aplicado o algoritmo
genético para a otimização dos parâmetros para detecção de eventos de
transitório (Seção~\ref{sec:aplic_es}). 

% Dos parametros utilizados na figura esboço da metodologia empregada
% pelo cepel

% Falar da presença de dois gabaritos para o temporizado

\section{Descrição da base de dados}
\label{sec:base_de_dados}

%Foi proposto uma evolução gradual de dificuldade para a avaliação do
%método e uma sequência de etapas para evoluir no sentido de obter um
%\gls{nilm} capaz de desagregar o consumo, e não apenas identificar
%equipamentos. A configuração mais simples incluiu a coleta de dados de
%equipamentos operando individualmente, para depois adicionar
%configurações com empilhamentos dois a dois, seguindo para
%configurações mais complexas, adicionando \gls{c4}, \gls{c5} e
%\gls{c6} de maneira gradual até chegar em condições reais. A validação
%do filtro de derivada de Gaussiana foi realizada em cima de
%configurações mais simples, havendo acúmulo em configurações
%simples e dados limpos. 


Foram utilizados três conjuntos de dados fornecidos pelo \acs{cepel},
todos amostrados em situações controladas. Serão descritos as
características de cada um deles, sendo seus códigos para
identificação \emph{Temporizado} (Subseção~\ref{ssec:temp}),
\emph{Empilhado4} (Subsseão~\ref{ssec:emp4}) e \emph{Empilhado7}
(Subseção~\ref{ssec:emp7}).

\subsection{Conjunto de dados \emph{Temporizado}}
\label{ssec:temp}

\FloatBarrier
O conjunto de dados \emph{Temporizado} contém apenas cinco
equipamentos:

\begin{itemize}
\item Televisão LCD, constitui-se de uma \acs{c5};
\item Geladeira;
\item Lâmpada fluorescente 23W, 54W;
\item Ventilador.
\end{itemize}

Esse arquivo é diferente dos anteriores por estes motivos: 
conter apenas acionamentos e desacionamentos de equipamentos; 
Sua medição foi realizada com o medidor \emph{Yokogawa}, e o perfil de
consumo dos equipamentos pode ser visualizada em
\ref{fig:temporizado_overview}. Este conjunto é o que tem a maior
presença de ruído, causado pela televisão LCD, em especial para os
períodos das 00:00 às 02:00 e 04:30 às 08:00 do dia 21.

A informação no gabarito pode ser observada nas figuras
\ref{fig:temporizado_app_time}--\ref{fig:temporizado_televisao}.  A
Figura~\ref{fig:temporizado_app_time} contém a informação do consumo
temporal dos equipamentos, enquanto a
Figura~\ref{fig:temporizado_app_pie} contém o gráfico circular do
consumo estimado no gabarito para os equipamentos. As figuras
\ref{fig:temporizado_geladeira}--\ref{fig:temporizado_televisao}
contêm os transitórios dos equipamentos marcados pelo usuário durante a
criação do gabarito. Todos os eventos são movidos para obterem média
zero de forma que a figura fique uniforme e seja possível compará-los.
A informação contida nesse gráfico auxilia a identificar eventos no
gabarito que fogem do padrão, seja por erro do usuário no
preenchimento, ou por caracterizar um evento excêntrico, facilitando a
identificação desses casos. Um exemplo pode ser observado na
Figura~\ref{fig:temporizado_ventilador}, onde há a ocorrência de um
evento que foge do padrão dos outros coletados. Essas figuras também
permitem observar a quantidade de eventos para cada alteração de
estado (indicado entre parênteses no título das subfiguras).

\begin{sidewaysfigure}[p]
\centering
\includegraphics[width=\textwidth]{imagens/Temporizado_Overview.pdf}
\caption{Perfil de consumo para o conjunto de dados \emph{Temporizado}.}
\label{fig:temporizado_overview}
\end{sidewaysfigure}

\begin{sidewaysfigure}[p]
\centering
\includegraphics[width=\textwidth]{imagens/Temporizado_AppTime.pdf}
\caption{Informação no gabarito para o conjunto de dados
\emph{Temporizado}: consumo temporal dos equipamentos.}
\label{fig:temporizado_app_time}
\end{sidewaysfigure}

\begin{sidewaysfigure}[p]
\centering
\includegraphics[width=.5\textwidth]{imagens/Temporizado_AppPie.pdf}
\caption{Informação no gabarito para o conjunto de dados
\emph{Temporizado}: gráfico circular do consumo dos equipamentos.}
\label{fig:temporizado_app_pie}
\end{sidewaysfigure}

\begin{sidewaysfigure}[p]
\centering
\includegraphics[width=\textwidth]{imagens/Temporizado_App_Geladeira.pdf}
\caption{Informação no gabarito para o conjunto de dados
\emph{Temporizado}: envoltória para as diversas variáveis para a
geladeira.}
\label{fig:temporizado_geladeira}
\end{sidewaysfigure}

\begin{sidewaysfigure}[p]
\centering
\includegraphics[width=\textwidth]{imagens/Temporizado_App_Ventilador.pdf}
\caption{Informação no gabarito para o conjunto de dados
\emph{Temporizado}: envoltória para as diversas variáveis para a
ventilador.}
\label{fig:temporizado_ventilador}
\end{sidewaysfigure}

\begin{sidewaysfigure}[p]
\centering
\includegraphics[width=\textwidth]{imagens/Temporizado_App_LF23W.pdf}
\caption{Informação no gabarito para o conjunto de dados
\emph{Temporizado}: envoltória para as diversas variáveis para a
lâmpada fluorescente 23W.}
\label{fig:temporizado_lf23}
\end{sidewaysfigure}

\begin{sidewaysfigure}[p]
\centering
\includegraphics[width=\textwidth]{imagens/Temporizado_App_LF54W.pdf}
\caption{Informação no gabarito para o conjunto de dados
\emph{Temporizado}: envoltória para as diversas variáveis para a
lâmpada fluoresecente 54W.}
\label{fig:temporizado_lf54}
\end{sidewaysfigure}

\begin{sidewaysfigure}[p]
\centering
\includegraphics[width=\textwidth]{imagens/Temporizado_App_Televisao.pdf}
\caption{Informação no gabarito para o conjunto de dados
\emph{Temporizado}: envoltória para as diversas variáveis para a
televisão.}
\label{fig:temporizado_televisao}
\end{sidewaysfigure}

\FloatBarrier

\subsection{Conjunto de dados \emph{Empilhado4}}
\label{ssec:emp4}

O conjunto de dados \emph{Empilhado4} contém os seguintes
equipamentos:

\begin{itemize}
\item Forno elétrico;
\item Lâmpada fluorescente (LF) 25 W (2 unidades), 22W (2 unidades), 15W (3
unidades), 9W;
\item Lâmpada incandescente (LI) 40W (2 unidades);
\item Televisão CRT.
\end{itemize}

Sua medição foi realizada com o medidor do \acs{cepel}. Contém a maior
quantidade de alterações de estados de equipamentos de baixo consumo (no
caso as lâmpadas fluorescentes). O
perfil de seu consumo pode ser visualizado na
Figura~\ref{fig:empilhado4_overview}.

A informação no gabarito pode ser observada nas figuras
\ref{fig:empilhado4_app_time}--\ref{fig:empilhado4_app_pie}. 
A Figura~\ref{fig:empilhado4_app_time} contém a informação do consumo
temporal dos equipamentos, enquanto a Figura~\ref{fig:empilhado4_app_pie}
contém o gráfico circular do consumo estimado no gabarito para os
equipamentos.

%Figure~\ref{fig:empilhado4_app_pie} contém o gráfico circular do
%consumo estimado no gabarito para os equipamentos. As figuras
%\ref{fig:empilhado4_geladeira}--\ref{fig:empilhado4_televisao}

\begin{sidewaysfigure}[p]
\centering
\includegraphics[width=\textwidth]{imagens/Empilhado4_Overview.pdf}
\caption{Perfil de consumo para o conjunto de dados \emph{Empilhado4}.}
\label{fig:empilhado4_overview}
\end{sidewaysfigure}

\begin{sidewaysfigure}[p]
\centering
\includegraphics[width=\textwidth]{imagens/Empilhado4_AppTime.pdf}
\caption{Informação no gabarito para o conjunto de dados
\emph{Empilhado4}: consumo temporal dos equipamentos.}
\label{fig:empilhado4_app_time}
\end{sidewaysfigure}

\begin{sidewaysfigure}[p]
\centering
\includegraphics[width=\textwidth]{imagens/Empilhado4_AppPie.png}
\caption{Informação no gabarito para o conjunto de dados
\emph{Empilhado4}: gráfico circular do consumo dos equipamentos.}
\label{fig:empilhado4_app_pie}
\end{sidewaysfigure}

%\begin{sidewaysfigure}[p]
%\centering
%\includegraphics[width=\textwidth]{imagens/Empilhado4_App_Geladeira.pdf}
%\caption{Informação no gabarito para o conjunto de dados
%\emph{Empilhado4}: envoltória para as diversas variáveis para a
%geladeira.}
%\label{fig:empilhado4_geladeira}
%\end{sidewaysfigure}
\FloatBarrier

\subsection{Conjunto de dados \emph{Empilhado7}}
\label{ssec:emp7}

O conjunto de dados \emph{Empilhado7} contém os seguintes
equipamentos:

\begin{itemize}
\item Lâmpada incadescente (LI) 60 W, 100 W;
\item Lâmpada fluorescente (LF) 20 W, 21 W, 24 W, 26 W, 28 W, 40 W;
\item Secador de cabelo;
\item Ar condicionado;
\item Sanduicheira;
\item Geladeira (obs: essa geladeira tem consumo bastante superior
àquele utilizada no arquivo \emph{Temporizado});
\item Televisão CRT.
\end{itemize}

Sua medição foi realizada com o medidor do \acs{cepel}. Uma
peculiaridade desse arquivo é a distorção dos transitórios durante o
momento que o ar condicionado está operando.

A informação no gabarito pode ser observada nas figuras
\ref{fig:empilhado7_app_time}--\ref{fig:empilhado7_app_pie}. 
A Figura~\ref{fig:empilhado7_app_time} contém a informação do consumo
temporal dos equipamentos, enquanto a Figura~\ref{fig:empilhado7_app_pie}
contém o gráfico circular do consumo estimado no gabarito para os
equipamentos.


\begin{sidewaysfigure}[p]
\centering
\includegraphics[width=\textwidth]{imagens/Empilhado7_Overview.pdf}
\caption{Perfil de consumo para o conjunto de dados \emph{Empilhado7}.}
\label{fig:empilhado7_overview}
\end{sidewaysfigure}

\begin{sidewaysfigure}[p]
\centering
\includegraphics[width=\textwidth]{imagens/Empilhado7_AppTime.pdf}
\caption{Informação no gabarito para o conjunto de dados
\emph{Empilhado7}: consumo temporal dos equipamentos.}
\label{fig:empilhado7_app_time}
\end{sidewaysfigure}

\begin{sidewaysfigure}[p]
\centering
\includegraphics[width=\textwidth]{imagens/Empilhado7_AppPie.png}
\caption{Informação no gabarito para o conjunto de dados
\emph{Empilhado7}: gráfico circular do consumo dos equipamentos.}
\label{fig:empilhado7_app_pie}
\end{sidewaysfigure}

\FloatBarrier

\section[Aplicação do ES para Otimização do Detector de Eventos]{
Aplicação do \acf{es} para Otimização do Detector de Eventos}
\label{sec:aplic_es}

Para o ajuste automático, utilizou-se o ambiente de análise detalhado
no Capítulo~\ref{chap:framework}. Entretanto, algumas questões ainda
estavam em aberto. Além dos caminhos mostrados na
Figura~\ref{fig:cepel_transitorio}, foi implementado uma nova versão
para a remoção de eventos próximos utilizando a média de seus centros
(ver pp.~\pageref{text:media}). Outra maneira de remoção dos eventos
também foi adicionada utilizando o conceito de incosistência (ver
pp.~\pageref{text:incosistentes}). Uma estratégia em força bruta ---
aqui se referindo a otimizar todas as possíveis configurações e
identificar a melhor convergência delas --- não parecia ser a melhor
maneira de abordar o problema. Percebeu-se a necessidade de realizar a
escolha de algumas configurações a serem testadas para reduzir a
quantidade de caminhos possíveis.

Durante a realização do \emph{Ajuste Manual}, percebeu-se que os
eventos removidos devido à incosistência eram sempre eventos de falso
alarme, mas não havia a ocorrência de perda de eventos de detecção causados por
esse tipo de remoção. Por isso, determinou-se que a remoção de eventos de
inconsistentes seria sempre realizada. Por outro lado, a remoção de
eventos devido à ruído era importante para remoção de pertubações
rápidas geradas na rede que não constituiam na mudança do patamar
operativo da rede e, com base nisso, determinou-se que a remoção de
eventos ruidosos sempre seria realizada. Assim, essa variável
($\Delta I_{min}$) sempre estará presente no material genético das
espécies, bem como o $\sigma$ da Gaussiana a ser utilizada no filtro
de derivada de Gaussiana e o seu valor de corte $\delta_{min}$. Já
para o caso da remoção de eventos próximos, não era possível
determinar se sua utilização era necessária, nem qual das versões de
remoção --- por média, ou sem deslocamento --- era o que mais se
adequada ao problema. No caso de não usar corte, a variável $n_{min}$
não seria utilizada, porém no caso oposto as espécies teriam um gene a
mais, cujo fenótipo é inteiro. Ainda assim, sua representação é no
conjunto real, sendo necessário determinar como codificar a
informação. Simplesmente se escolheu arredondar o valor na
representação para obter o valor do fenótipo. Finalmente, também não
se sabia \emph{a priori} determinar se a ordem influenciaria na
capacidade de detecção e, apesar de não se esperar grandes diferenças
devido à mudança da ordem em que eles seriam removidos, decidiu-se
testar ambas configurações.  

Assim, para essas determinações, haviam cinco caminhos a serem
percorridos, que determinam as espécies utilizadas no \acs{es}: 

\begin{enumerate}[label={Espécie} (\Roman*) -,ref=(\Roman*),align=left]
\item\label{item:esp1} Com remoção de eventos próximos sem
deslocamento, essa \emph{depois} de remover eventos ruidosos;
\item\label{item:esp2} Com remoção de eventos próximos utilizando a
média, essa \emph{depois} de remover eventos ruidosos;
\item\label{item:esp3} Com remoção de eventos próximos sem
deslocamento, essa \emph{antes} de remover eventos ruidosos;
\item\label{item:esp4} Com remoção de eventos próximos utilizando a
média, essa \emph{antes} de remover eventos ruidosos;
\item\label{item:esp5} Sem remoção de eventos próximos (essa espécie tem um gene a
menos que as espécies anteriores: o tamanho da janela); 
\end{enumerate}

Além da questão dos caminhos, era necessário determinar a
inicialização. Para garantir melhor convergência, empregou-se
fronteiras mínimas e máximas para cada um dos valores representados
pelos genes. A inicialização foi aleatória uniforme dentro dessas
fronteiras. Para o parâmetro da estratégia evolutiva, porém,
iniciou-se esse valor de acordo com \ref{eq:sigma_init}, garantindo
que inicialmente cada individuo tenha, em média, 95\% de chance de
explorar uma região no interior a 10\% do espaço de solução permitido. Os
limites inferiores e superiores foram determinados por
\ref{eq:sigma_min} e \ref{eq:sigma_max}, respectivamente. Os valores
de $\sigma_{min,i}$ e $\sigma_{max,i}$ garantem que, no mínimo, o material
genético irá sofrer pertubações dentro da região de 0,01\% da região
de solução em 67\% dos casos, enquanto no máximo ele irá pertubar o
material genético em 30\% da região para a mesma probabilidade:

\begin{subequations}
\begin{equation}\label{eq:sigma_init}
2\sigma_{init,i}=0,1\Delta x_i
\end{equation}
\begin{equation}\label{eq:sigma_min}
\sigma_{min,i}=1\times10^{-4}\Delta x_i
\end{equation}
\begin{equation}\label{eq:sigma_max}
\sigma_{max,i}=3\times10^{+3}\sigma_{min,i}
\end{equation}
\end{subequations}

\noindent onde $\Delta x_i$ é a região do espaço dentro das fronteiras
para a i-ésima variável.

Finalmente, cabe ainda decidir qual dos métodos de competição para as
espécies será utilizado. Como a implementação do método de competição
intraespécies acaba favorecendo espécies com carga genética mais
simples de serem otimizadas ou que começaram em condições 
privilegiadas --- quando sem o mecanismo de adiamento da competição
---, irá optar-se pela competição intraespécie.

