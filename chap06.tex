\chapter{Metodologia}
\label{chap:metodologia}

Uma vez definido o ambiente de análise implementado e suas
capacidades, cabe detalhar como o mesmo será aplicado para
determinar os parâmetros da abordagem final para detecção de
eventos de transitório e a base de dados que será
trabalhada. Este capítulo começa realizando a descrição das
características presentes nos conjuntos de dados trabalhados, que
podem ser separados em dois grandes grupos: um contendo a operação de
poucos equipamentos atuando em conjunto e praticamente nenhuma
injeção de ruído devido à equipamentos \acs{c5} (ver
Subseção~\ref{ssec:modelos_carga}) --- os dados puros contidos nos
arquivos NI00***\footnote{O símbolo ``*'' é utilizado com o mesmo
significado do metacaractere para os sistemas operacionais
\emph{Linux} na computação, ou seja, podem significar qualquer outro
caractere --- o análogo à carta coringa nos baralhos de cartas.};
enquanto o outro conta com a operação de múltiplos equipamentos
atuando simultaneamente (cerca de 5 a 8, em geral) e da presença de
equipamentos \acs{c5}, constituindo, por isso, de dados com menor
relação sinal-ruído --- conjuntos \emph{Temporizado},
\emph{Empilhado4} e \emph{Empilhado7}. Todos os conjuntos de dados
possuem apenas um gabarito, exceto o \emph{Temporizado}, que possui
dois gabaritos. O motivo para isso é a presença de um distúrbio
durante a operação da televisão LCD nesse conjunto de dados, que
\emph{não} é considerado como alvo na configuração \emph{Temporizado
Gabarito 1}, enquanto a configuração \emph{Temporizado Gabarito 2}
contém essa informação como eventos de transitórios desejados como
alvo.


Em seguida, detém-se atenção para quais configurações e parâmetros
presentes na metodologia proposta serão otimizados pelo \acs{es}
descrito na Seção~\ref{ssec:es} do capítulo anterior, aonde detalhará
como isso foi realizado. Como referência rápida, estes foram os
parâmetros e as configurações otimizadas pela versão multiespécie do
\acs{es} implementado:

\begin{itemize}
\item \textlabel{$\sigma_{gauss}$}{item:parametros}: O valor do
$\sigma$ da Gaussiana empregado para retirar sua derivada e compor o
\acs{fir} que é realiza a sua convolução com a envoltória de corrente;
\item $\delta_{min}$: O valor mínimo que a resposta do \acs{fir} deve
ter, em módulo, para as respectivas amostras constituam de uma região
sensibilizada;
\item $\Delta{I}_{min}$: Mínimo valor para a característica \acs{di} na
qual o candidato a evento não será eliminado como ruído 
\item $n_{min}$ (sujeito à utilização na configuração otimizada da
remoção por eventos próximos): Quantidade mínima de amostras que um
candidato deve estar distante de outro para não ser removido por evento
próximo. A estratégia para a remoção depende da configuração
escolhida, conforme próximo item; 
\item Configurações otimizadas (em competição intraespécie), todas com
remoção devido a eventos inconsistentes:
\begin{enumerate}[label={Espécie} (\Roman*) -,ref=(\Roman*),align=left]
\item\label{item:esp1} Com remoção de eventos próximos sem
deslocamento, essa \emph{depois} de remover eventos ruidosos;
\item\label{item:esp2} Com remoção de eventos próximos utilizando a
média, essa \emph{depois} de remover eventos ruidosos;
\item\label{item:esp3} Com remoção de eventos próximos sem
deslocamento, essa \emph{antes} de remover eventos ruidosos;
\item\label{item:esp4} Com remoção de eventos próximos utilizando a
média, essa \emph{antes} de remover eventos ruidosos;
\item\label{item:esp5} Sem remoção de eventos próximos (não otimiza
$n_{min}$). 
\end{enumerate}
\end{itemize}


\section{Descrição da base de dados}
\label{sec:base_de_dados}

O \acs{cepel} propôs uma evolução gradual para a análise de sua
técnica. As configurações mais simples incluem a coleta de dados com
equipamentos sendo acionados e desacionados sequencialmente operando
individualmente em apenas um único estado. Em seguida, simulou-se
condições de operação em conjunto de equipamentos dois a dois, e
equipamentos \acs{c2} operando em seus diversos estados. Essas
condições mais simples e com dados bastante limpos constituíram a base
de dados dos arquivos NI00*** que serão detalhados adiante nesta
seção. Para todos os conjuntos utilizados neste trabalho, foi
construído o gabarito com a infraestrutura implementada descrita no
Capítulo~\ref{chap:framework} e de posse de registros das alterações
nos estados operativos dos equipamentos fornecidos pelo \acs{cepel}.

Já os conjuntos de dados \emph{Temporizado}, \emph{Empilhado4} e
\emph{Empilhado7} constituem de simulações em laboratório mais
complexas, contendo a operação de diversos aparelhos simultaneamente
que têm sua detecção dificultada pela presença da dinâmica de
aparelhos \acs{c5}. O conjunto \emph{Temporizado} foi criado pelo
\acs{cepel} mais recentemente com o intuito de explorar os resultados
de sua técnica em condições ruidosas, enquanto os conjuntos
\emph{Empilhado4} e \emph{Empilhado7} foram elaborados durante a época
do estudo de \citeauthor*{nilm_cepel_alvaro}, que não teve a
oportunidade de analisar mais a fundo tais conjuntos de dados. Os
registros das alterações nos estados operativos dos equipamentos
nesses conjuntos de dados foram fornecidos pelo mesmo, o que
possibilitou a recuperação desses dados com a informação necessária
para a construção do gabarito.

\subsection{Conjunto de dados \emph{NI00168}}

Este conjunto de dados contém apenas a operação de equipamentos
individualmente. Há a presença de dois equipamentos desconhecidos, que
não foram possíveis de ter sua informação recuperada ao correlacionar
com os registros de alterações. Como não era prioridade neste trabalho
ter essa informação --- esses equipamentos não-rotulados não
apresentam características especiais nos eventos de transitório
decorrentes de seus acionamentos e desacionamentos, e essa informação
é mais importante para a etapa de discriminação ---, trabalhou-se com
os dados sem a rotulação desses equipamentos.

O consumo agregado neste conjunto de dados pode ser observado na
Figura~\ref{fig:ni00168_overview}, enquanto a
Figura~\ref{fig:ni00168_app_time} contém a estimativa do consumo
desagregado no gabarito.

Os equipamentos presentes neste conjunto de dados são, todos com dois
eventos de transitório, exceto quando especificado:

\begin{itemize}
\item Lâmpada incandescente 100 W, 60 W (com \emph{dimmer},
o que constitui uma \acs{c4});
\item Dois equipamentos desconhecidos;
\item Lâmpada fluorescente 15 W, 16 W, 18 W, 20 W, 21 W (circular),
40 W (tubular), 32 W (tubular), 28 W (tubular);
\item Secador de cabelo (4 eventos: Desligado $\rightarrow$ Fraco
$\rightarrow$ Forte $\rightarrow$ Fraco $\rightarrow$ Desligado);
\item Televisão (3 eventos: Desligado $\rightarrow$ Ligado
$\rightarrow$ Ligado e exibindo imagem $\rightarrow$
Desligado)\footnote{A televisão pode ser ligada em um canal que não
apresenta imagem.};
\item Geladeira;
\item Ar condicionado (4 eventos: Desligado $\rightarrow$ Ventilação
$\rightarrow$ Ventilação e compressor $\rightarrow$ Ventilação
$\rightarrow$ Desligado, constitui uma \acs{c5}).
\end{itemize}

Todos os arquivos NI00*** foram coletados utilizando o medidor do
\acs{cepel}.

\begin{sidewaysfigure}[p]
\centering
\includegraphics[width=\textwidth]{imagens/NI00168_Overview.pdf}
\caption{Perfil de consumo agregado para o conjunto de dados
\emph{NI00168}.}
\label{fig:ni00168_overview}
\end{sidewaysfigure}

\begin{sidewaysfigure}[p]
\centering
\includegraphics[width=\textwidth]{imagens/NI00168_AppTime.pdf}
\caption{Informação no gabarito para o conjunto de dados
\emph{NI00168} - consumo temporal dos equipamentos.}
\label{fig:ni00168_app_time}
\end{sidewaysfigure}

\begin{sidewaysfigure}[p]
\centering
\includegraphics[width=\textwidth]{imagens/NI00168_AppPie.png}
\caption{Informação no gabarito para o conjunto de dados
\emph{NI00168} - gráfico circular do consumo dos equipamentos.}
\label{fig:ni00168_app_pie}
\end{sidewaysfigure}

\FloatBarrier
\subsection{Conjunto de dados \emph{NI00171}}

Este conjunto de dados é bastante similar ao conjunto \emph{NI00168},
porém há a distinção quanto a alguns equipamentos presentes nesses
conjuntos. A diferença mais notável dos dois conjuntos é a presença de
eletrodomésticos com outros uso-finais distintos de iluminação, sendo
eles: ferro de passar roupas, computador portátil, ventilador e
liquidificador.  Os equipamentos de destaque nesse conjunto de dados
são o ventilador e o computador portátil, onde este apresenta mudanças
de estados com pequenos degraus de potência, como pode ser observado
na Figura~\ref{fig:ni00171_app_time} e aquele se constitui de uma
\acs{c5} com múltiplas oscilações de consumo durante o seu
acionamento, que ocorre em dois estágios exibidos na
Figura~\ref{fig:ni00171_laptop}. Os eventos de transitório nessa
figura têm sua média retirada para permitir a comparação das múltiplas
mudanças de estado presentes do equipamento, como os casos dos
equipamentos no conjunto de dados \emph{Temporizado}, disponíveis nas 
Figuras~\ref{fig:temporizado_geladeira}-\ref{fig:temporizado_televisao}.

\begin{itemize}
\item Lâmpada incandescente 100 W;
\item Quatro equipamentos desconhecidos;
\item Lâmpada fluorescente 18 W, 26 W, 40 W (tubular), 32 W (tubular),
28 W (tubular);
\item Secador de cabelo (4 eventos: Desligado $\rightarrow$ Fraco
$\rightarrow$ Forte $\rightarrow$ Fraco $\rightarrow$ Desligado);
\item Televisão (3 eventos: Desligado $\rightarrow$ Ligado
$\rightarrow$ Ligado e exibindo imagem $\rightarrow$
Desligado);
\item Geladeira;
\item Ferro de passar roupas;
\item Computador portátil (constitui uma \acs{c5} durante mudanças do
uso do processamento, que neste conjunto de dados ocorre apenas
durante o seu acionamento);
\item Liquidificador;
\item Ventilador (6 eventos: Desligado $\rightarrow$ Forte $\rightarrow$
Médio $\rightarrow$ Fraco $\rightarrow$ Médio $\rightarrow$ Forte
$\rightarrow$ Desligado);
\item Ar condicionado (4 eventos: Desligado $\rightarrow$ Ventilação
$\rightarrow$ Ventilação e compressor $\rightarrow$ Ventilação
$\rightarrow$ Desligado, constitui uma \acs{c5}).
\end{itemize}
 
Todos os arquivos NI00*** foram coletados utilizando o medidor do
\acs{cepel}.

\begin{sidewaysfigure}[p]
\centering
\includegraphics[width=\textwidth]{imagens/NI00171_Overview.pdf}
\caption{Perfil de consumo agregado para o conjunto de dados
\emph{NI00171}.}
\label{fig:ni00171_overview}
\end{sidewaysfigure}

\begin{sidewaysfigure}[p]
\centering
\includegraphics[width=\textwidth]{imagens/NI00171_AppTime.pdf}
\caption{Informação no gabarito para o conjunto de dados
\emph{NI00171} - consumo temporal dos equipamentos.}
\label{fig:ni00171_app_time}
\end{sidewaysfigure}

\begin{sidewaysfigure}[p]
\centering
\includegraphics[width=\textwidth]{imagens/NI00171_AppPie.png}
\caption{Informação no gabarito para o conjunto de dados
\emph{NI00171} - gráfico circular do consumo dos equipamentos.}
\label{fig:ni00171_app_pie}
\end{sidewaysfigure}

\begin{sidewaysfigure}[p]
\centering
\includegraphics[width=\textwidth]{imagens/NI00171_App_Laptop.pdf}
\caption{Informação no gabarito para o conjunto de dados
\emph{NI00171} - envoltória para as diversas variáveis para o
computador portátil.}
\label{fig:ni00171_laptop}
\end{sidewaysfigure}


\FloatBarrier
\subsection{Conjunto de dados \emph{NI00173}, \emph{NI00174},
\emph{NI00175} e \emph{NI00177}}

Estes conjuntos de dados apresentam acionamentos de equipamentos
simultaneamente, porém isso é realizado de maneira simples, apenas com
a operação de dois equipamentos atuando ao mesmo tempo. Em ambos os
casos é utilizado uma lâmpada incandescente (100 W) em conjunto com um
outro aparelho para realizar a operação conjunta de equipamentos. No
caso do conjunto de dados \emph{NI00173}, o equipamento operando em
conjunto com a lâmpada incandescente é um ventilador acionado em um
único estado (pode ser modelado como uma \acs{c3}, apesar de ser uma
\acs{c2}), enquanto nos demais conjuntos o equipamento é um secador de
cabelo que opera em dois estados (Forte e Fraco, e, portando, atua
como uma \acs{c2}). A única diferença entre os arquivos
\emph{NI00174}, \emph{NI00175} e \emph{NI00177} é a ordem na qual o
secador de cabelo e a lâmpada têm seus estados de operação alterados.
Para o conjunto \emph{NI00174}, a lâmpada permanece ligada enquanto o
estado de operação do secador é alterado.  Os conjuntos \emph{NI00175}
e \emph{NI00177} alteram a operação dos estados desses equipamentos de
diversas modos diferentes, tentando simular diversas configurações em
que esses aparelhos poderiam ter seus estados alterados em uma
residencia. As
Figuras~\ref{fig:ni00173_overview}-\ref{fig:ni00177_app_time} permitem
observar esses comportamentos, contendo o consumo agregado nesses
conjuntos de dados e a informação estimada do consumo desagregado em
seus respectivos gabaritos.

Todos os arquivos NI00*** foram coletados utilizando o medidor do
\acs{cepel}.


\begin{sidewaysfigure}[p]
\centering
\includegraphics[width=\textwidth]{imagens/NI00173_Overview.pdf}
\caption{Perfil de consumo agregado para o conjunto de dados
\emph{NI00173}.}
\label{fig:ni00173_overview}
\end{sidewaysfigure}

\begin{sidewaysfigure}[p]
\centering
\includegraphics[width=\textwidth]{imagens/NI00173_AppTime.pdf}
\caption{Informação no gabarito para o conjunto de dados
\emph{NI00173} - consumo temporal dos equipamentos.}
\label{fig:ni00173_app_time}
\end{sidewaysfigure}

\begin{sidewaysfigure}[p]
\centering
\includegraphics[width=\textwidth]{imagens/NI00174_Overview.pdf}
\caption{Perfil de consumo agregado para o conjunto de dados
\emph{NI00174}.}
\label{fig:ni00174_overview}
\end{sidewaysfigure}

\begin{sidewaysfigure}[p]
\centering
\includegraphics[width=\textwidth]{imagens/NI00174_AppTime.pdf}
\caption{Informação no gabarito para o conjunto de dados
\emph{NI00174} - consumo temporal dos equipamentos.}
\label{fig:ni00174_app_time}
\end{sidewaysfigure}

\begin{sidewaysfigure}[p]
\centering
\includegraphics[width=\textwidth]{imagens/NI00175_Overview.pdf}
\caption{Perfil de consumo agregado para o conjunto de dados
\emph{NI00175}.}
\label{fig:ni00175_overview}
\end{sidewaysfigure}

\begin{sidewaysfigure}[p]
\centering
\includegraphics[width=\textwidth]{imagens/NI00175_AppTime.pdf}
\caption{Informação no gabarito para o conjunto de dados
\emph{NI00175} - consumo temporal dos equipamentos.}
\label{fig:ni00175_app_time}
\end{sidewaysfigure}

\begin{sidewaysfigure}[p]
\centering
\includegraphics[width=\textwidth]{imagens/NI00177_Overview.pdf}
\caption{Perfil de consumo agregado para o conjunto de dados
\emph{NI00177}.}
\label{fig:ni00177_overview}
\end{sidewaysfigure}

\begin{sidewaysfigure}[p]
\centering
\includegraphics[width=\textwidth]{imagens/NI00177_AppTime.pdf}
\caption{Informação no gabarito para o conjunto de dados
\emph{NI00177} - consumo temporal dos equipamentos.}
\label{fig:ni00177_app_time}
\end{sidewaysfigure}

\FloatBarrier
\subsection{Conjunto de dados \emph{Temporizado}}
\label{ssec:temp}

Este conjunto possui a operação de cinco aparelhos que podem ser
modelados como \acs{c3} por atuarem apenas em um único estado quando
operando. Uma característica deste conjunto é a presença de um alto
nível de ruído durante a operação de uma \acs{c5}, a televisão LCD.
Outra peculiaridade deste conjunto é a sua longa duração, cerca de 18
horas foram coletadas. Além disso, ele possui dois gabaritos, motivo
detalhado um pouco mais a adiante. A atuação dos equipamentos foi
feito através de chaveamento automático, onde tentou-se gerar eventos
próximos de acionamentos e desacionamentos dos equipamentos. Um total
de 149 eventos de transitório estão presentes no conjunto, causados
por estes equipamentos (sempre na sequência Desligado $\rightarrow$
Ligado $\rightarrow$ Desligado):

\begin{itemize}
\item Televisão LCD (6 eventos de transitório), constitui-se de uma \acs{c5};
\item Geladeira (123 eventos, o número impar de eventos é causado pelo
fato da geladeira já estar operando quando a medição do conjunto é
iniciada);
\item Lâmpada fluorescente 23 W (4 eventos), 54 W (4 eventos);
\item Ventilador (12 eventos).
\end{itemize}

O consumo agregado pode ser observado na
Figura~\ref{fig:temporizado_app_time}. A informação contida no
gabarito para este conjunto de dados pode ser observada nas figuras
\ref{fig:temporizado_app_time}--\ref{fig:temporizado_televisao}.
A Figura~\ref{fig:temporizado_app_time} contém a informação do consumo
temporal dos equipamentos, enquanto a
Figura~\ref{fig:temporizado_app_pie} contém o gráfico circular do
consumo estimado no gabarito para os equipamentos. As figuras
\ref{fig:temporizado_geladeira}--\ref{fig:temporizado_televisao}
contêm os transitórios dos equipamentos marcados pelo usuário durante
a criação do gabarito. Todos os eventos são movidos para obterem média
zero, de forma que os eventos de transitório estejam centrados no mesmo
patamar e seja possível compará-los. A informação contida nesse
gráfico auxilia a identificar eventos no gabarito que fogem do padrão,
seja por erro do usuário no preenchimento, ou por caracterizar um
evento excêntrico, facilitando a identificação desses casos. Um
exemplo pode ser observado na Figura~\ref{fig:temporizado_ventilador},
onde há a ocorrência de um evento, marcado por uma etiqueta (a mesma
torna possível a identificação do evento através de sua chave), que
foge do padrão dos outros coletados. Essas figuras também permitem
observar a quantidade de eventos para cada alteração de estado
(indicado entre parênteses no título das subfiguras). Sua medição foi
realizada com o medidor \emph{Yokogawa}.

Este conjunto é o que tem a maior presença de ruído, causado pela
televisão LCD, em especial para os períodos das 00:00 às 02:00 e 04:30
às 08:00 do dia 21. Durante esses períodos, há a ocorrência de uma
pertubação que se assemelha com a operação de um equipamento \acs{c3},
como exibido na Figura~\ref{fig:temporizado_disturbio}. Durante a
aplicação da metodologia descrita na Seção~\ref{sec:aplic_es} e ao
observar os resultados presentes no Capítulo~\ref{chap:resultados},
observou-se uma alta presença de falsos alarmes nesse conjunto para
todas os parâmetros examinados, o que levou a descoberta dessa
pertubação. Fica então a questão de como tratá-la, se essa pertubação
deve ser considerado como um estado interno de operação da televisão
LCD, ou se esse distúrbio deve ser ignorado e tratado como falso
alarme. Para permitir a análise de ambas situações, gerou-se o
gabarito para este conjunto de dados com os dois casos, o primeiro
gabarito contendo apenas os acionamentos e desacionamentos registrados
pelo \acs{cepel} (149 eventos de transitório), e o segundo contém,
também, os distúrbios observados na
Figura~\ref{fig:temporizado_disturbio} (211 eventos).

\begin{sidewaysfigure}[p]
\centering
\includegraphics[width=\textwidth]{imagens/Temporizado_Overview.pdf}
\caption{Perfil de consumo agregado para o conjunto de dados \emph{Temporizado}.}
\label{fig:temporizado_overview}
\end{sidewaysfigure}

\begin{sidewaysfigure}[p]
\centering
\includegraphics[width=\textwidth]{imagens/Temporizado_AppTime.pdf}
\caption{Informação no gabarito para o conjunto de dados
\emph{Temporizado} - consumo temporal dos equipamentos.}
\label{fig:temporizado_app_time}
\end{sidewaysfigure}

\begin{sidewaysfigure}[p]
\centering
\includegraphics[width=.5\textwidth]{imagens/Temporizado_AppPie.pdf}
\caption{Informação no gabarito para o conjunto de dados
\emph{Temporizado} - gráfico circular do consumo dos equipamentos.}
\label{fig:temporizado_app_pie}
\end{sidewaysfigure}

\begin{sidewaysfigure}[p]
\centering
\includegraphics[width=\textwidth]{imagens/Temporizado_App_Geladeira.pdf}
\caption{Informação no gabarito para o conjunto de dados
\emph{Temporizado} - envoltória para as diversas variáveis para a
geladeira.}
\label{fig:temporizado_geladeira}
\end{sidewaysfigure}

\begin{sidewaysfigure}[p]
\centering
\includegraphics[width=\textwidth]{imagens/Temporizado_App_Ventilador.pdf}
\caption{Informação no gabarito para o conjunto de dados
\emph{Temporizado} - envoltória para as diversas variáveis para a
ventilador.}
\label{fig:temporizado_ventilador}
\end{sidewaysfigure}

\begin{sidewaysfigure}[p]
\centering
\includegraphics[width=\textwidth]{imagens/Temporizado_App_LF23W.pdf}
\caption{Informação no gabarito para o conjunto de dados
\emph{Temporizado} - envoltória para as diversas variáveis para a
lâmpada fluorescente 23W.}
\label{fig:temporizado_lf23}
\end{sidewaysfigure}

\begin{sidewaysfigure}[p]
\centering
\includegraphics[width=\textwidth]{imagens/Temporizado_App_LF54W.pdf}
\caption{Informação no gabarito para o conjunto de dados
\emph{Temporizado} - envoltória para as diversas variáveis para a
lâmpada fluorescente 54W.}
\label{fig:temporizado_lf54}
\end{sidewaysfigure}

\begin{sidewaysfigure}[p]
\centering
\includegraphics[width=\textwidth]{imagens/Temporizado_App_Televisao.pdf}
\caption{Informação no gabarito para o conjunto de dados
\emph{Temporizado} - envoltória para as diversas variáveis para a
televisão.}
\label{fig:temporizado_televisao}
\end{sidewaysfigure}

\begin{sidewaysfigure}[p]
\centering
\includegraphics[width=\textwidth]{imagens/temporizado_disturbio.pdf}
\caption[Distúrbio recorrente presente no conjunto de dados \emph{Temporizado}.]
{Distúrbio recorrente presente no conjunto de dados
\emph{Temporizado}. O distúrbio apresenta características semelhantes
a um aparelho \acs{c3} com consumo de 15 W e opera durante cerca de
3 s.}
\label{fig:temporizado_disturbio}
\end{sidewaysfigure}

\FloatBarrier

\subsection{Conjunto de dados \emph{Empilhado4}}
\label{ssec:emp4}

A principal característica deste conjunto é a operação simultânea de
lâmpadas fluorescentes e incandescentes de diversos consumos em
conjunto com outros três equipamentos:
forno elétrico, chuveiro elétrico e televisão CRT. Esses aparelhos
operam um a um em conjunto com as lâmpadas. Este conjunto também é o
que contém a maior quantidade de eventos de transitório de baixo
consumo causado pelas lâmpadas fluorescentes de baixo consumo. A
seguir segue a descrição dos equipamentos presentes neste conjunto de
dados:

\begin{itemize}
\item Forno elétrico: dinâmica de consumo após acionamento (queda
lenta do consumo até estabilizar); 
\item Chuveiro elétrico: dinâmica de consumo após acionamento (flutuações
após acionamento, tendência a estabilizar, porém durante a operação
observada neste conjunto se trata de uma \acs{c5} durante toda sua
atuação);
\item Televisão CRT: dinâmica de consumo permanente;
\item Lâmpada fluorescente (LF) 25 W (2 unidades), 22 W (2 unidades), 15 W (3
unidades), 9 W;
\item Lâmpada incandescente (LI) 40 W (2 unidades).
\end{itemize}

Sua medição foi realizada com o medidor do \acs{cepel}. O perfil de seu
consumo pode ser visualizado na Figura~\ref{fig:empilhado4_overview}.
A informação no gabarito pode ser observada nas figuras
\ref{fig:empilhado4_app_time} e \ref{fig:empilhado4_app_pie}.  A
Figura~\ref{fig:empilhado4_app_time} contém a informação do consumo
temporal dos equipamentos, enquanto a
Figura~\ref{fig:empilhado4_app_pie} contém o gráfico circular do
consumo estimado no gabarito para os equipamentos.

\begin{sidewaysfigure}[p]
\centering
\includegraphics[width=\textwidth]{imagens/Empilhado4_Overview.pdf}
\caption{Perfil de consumo agregado para o conjunto de dados \emph{Empilhado4}.}
\label{fig:empilhado4_overview}
\end{sidewaysfigure}

\begin{sidewaysfigure}[p]
\centering
\includegraphics[width=\textwidth]{imagens/Empilhado4_AppTime.pdf}
\caption{Informação no gabarito para o conjunto de dados
\emph{Empilhado4}: consumo temporal dos equipamentos.}
\label{fig:empilhado4_app_time}
\end{sidewaysfigure}

\begin{sidewaysfigure}[p]
\centering
\includegraphics[width=\textwidth]{imagens/Empilhado4_AppPie.png}
\caption{Informação no gabarito para o conjunto de dados
\emph{Empilhado4}: gráfico circular do consumo dos equipamentos.}
\label{fig:empilhado4_app_pie}
\end{sidewaysfigure}

%\begin{sidewaysfigure}[p]
%\centering
%\includegraphics[width=\textwidth]{imagens/Empilhado4_App_Geladeira.pdf}
%\caption{Informação no gabarito para o conjunto de dados
%\emph{Empilhado4}: envoltória para as diversas variáveis para a
%geladeira.}
%\label{fig:empilhado4_geladeira}
%\end{sidewaysfigure}
\FloatBarrier

\subsection{Conjunto de dados \emph{Empilhado7}}
\label{ssec:emp7}

Este é o conjunto com a maior operação simultânea de equipamentos de
diferentes uso-finais. Há a ocorrência de um evento de transitório
aonde ocorre o desacionamento de quatro equipamentos em conjunto
(geladeira, uma lâmpada incandescente 100 W, e duas lâmpadas
fluorescentes 20 W e 24 W). Outra peculiaridade deste arquivo é a
distorção dos eventos de transitório durante o momento que o ar
condicionado está operando (um total de 13 eventos presentes). Os
equipamentos presentes neste conjunto são:

\begin{itemize}
\item Lâmpada incandescente (LI) 60 W, 100 W;
\item Lâmpada fluorescente (LF) 20 W, 21 W (circular), 24 W, 26 W, 28
W, 40 W;
\item Secador de cabelo;
\item Ar condicionado (constitui de uma \acs{c5});
\item Sanduicheira;
\item Geladeira (obs: essa geladeira tem consumo bastante superior
àquele utilizada no arquivo \emph{Temporizado});
\item Televisão CRT (constitui de uma \acs{c5}).
\end{itemize}

Sua medição foi realizada com o medidor do \acs{cepel}.  A informação
no gabarito pode ser observada nas figuras
\ref{fig:empilhado7_app_time}--\ref{fig:empilhado7_app_pie}.  A
Figura~\ref{fig:empilhado7_app_time} contém a informação do consumo
temporal dos equipamentos, enquanto a
Figura~\ref{fig:empilhado7_app_pie} contém o gráfico circular do
consumo estimado no gabarito para os equipamentos.


\begin{sidewaysfigure}[p]
\centering
\includegraphics[width=\textwidth]{imagens/Empilhado7_Overview.pdf}
\caption{Perfil de consumo agregado para o conjunto de dados \emph{Empilhado7}.}
\label{fig:empilhado7_overview}
\end{sidewaysfigure}

\begin{sidewaysfigure}[p]
\centering
\includegraphics[width=\textwidth]{imagens/Empilhado7_AppTime.pdf}
\caption{Informação no gabarito para o conjunto de dados
\emph{Empilhado7}: consumo temporal dos equipamentos.}
\label{fig:empilhado7_app_time}
\end{sidewaysfigure}

\begin{sidewaysfigure}[p]
\centering
\includegraphics[width=\textwidth]{imagens/Empilhado7_AppPie.png}
\caption{Informação no gabarito para o conjunto de dados
\emph{Empilhado7}: gráfico circular do consumo dos equipamentos.}
\label{fig:empilhado7_app_pie}
\end{sidewaysfigure}

\FloatBarrier

\section[Aplicação do ES para Otimização do Detector de Eventos]{
Aplicação do \acf{es} para Otimização do Detector de Eventos}
\label{sec:aplic_es}

Esta seção se dedica a descrição da metodologia final adotada para a
detecção de eventos por este trabalho que foi aplicada na base de
dados descrita na seção anterior. A metodologia base é aquela
explicada na Seção~\ref{ssec:met_cepel}, proposta pelo \acs{cepel}
para a detecção de eventos de transitório. Resumidamente, a
metodologia utiliza um filtro para convoluir a derivada de Gaussiana
com o sinal de corrente, que irá gerar regiões sensibilizadas caso a
resposta do filtro ultrapasse um limiar. O ponto de inflexão dessas
regiões representam candidatos a evento de transitório, que serão
aceitos caso não tenham sido gerados por ruído ou estejam
demasiadamente próximos a outros candidatos. Porém, a mesma foi
reimplementada de acordo com as considerações realizadas no
Capítulo~\ref{chap:framework} que, além de diversas alterações
técnicas, trouxeram três alterações:

\begin{itemize}
\item Remoção de eventos próximos utilizando sua média (ver
pp.~\pageref{text:media});
\item Remoção de eventos inconsistentes (ver
pp.~\pageref{text:incosistentes});
\item Ajuste automático dos parâmetros através de um otimizador, que
neste trabalho se trata da versão multiespécie do \acs{es}
implementado e detalhado na Seção~\ref{sec:otimizacao}.
\end{itemize}

Antes de entrar no mérito do ajuste automático, este trabalho também
realizou um ajuste empírico dos parâmetros como era feito pelo
\acs{cepel} na sua metodologia original, que será referido como
\emph{Ajuste Manual}. Isso foi realizado para permitir uma melhor
afinidade com o ambiente de análise, uma vez que a utilização
unicamente de parâmetros otimizados automaticamente não permite a
compreensão de nuances da operação e capacidade da metodologia. Apesar
de ser realizada manualmente, algumas diferenças podem ser citadas em
relação ao ajuste realizado pelo \acs{cepel}. A primeira, e mais
importante, é que o ajuste foi realizado para os conjuntos de dados
\emph{Temporizado}, que contém a presença mais forte de ruídos.  Além
disso, diferente do ajuste automático de parâmetros que será explicada
a seguir e opera ajustando vários parâmetros citados no início deste
capítulo (pp.~\pageref{item:parametros}), o \emph{Ajuste Manual} foi
realizado apenas para o $\delta_{min}$, enquanto os demais parâmetros
são os mesmos empregados pela metodologia do \acs{cepel}. Contudo, a
versão do algoritmo já permitia a utilização de novas maneiras de
remoção de evento, que foram analisadas e optou-se empregar a versão
com remoção de eventos inconsistentes e utilização da média para
remoção de eventos próximos.

A metodologia original do \acs{cepel} já apresentava cinco diferentes
maneiras para a remoção de candidatos, indicadas na
Figura~\ref{fig:cepel_transitorio}. A adição de duas novas maneiras de
remoção de candidatos aumentou a quantidade de possibilidades de
configurações a serem analisadas com relação a quais exames empregar
para a remoção de candidatos. Não se espera que a questão dos exames
realizados para a remoção seja um fator chave no desempenho da
metodologia, e sim o valor dos parâmetros empregados. Porém, não se sabe
\emph{a priori} qual é a configuração mais indicada para a operação,
sendo necessário realizar comparações entre elas para determinar qual
é a mais adequada. Ao invés de realizar a otimização de cada
configuração individualmente e então determinar qual teve a melhor
convergência, implementou-se a versão multiespécie do \acs{es} (ver
pp.~\pageref{sssec:multiespecie}) que
permite dinâmica na quantidade de esforço computacional reservado para
cada configuração. Assim, analisar-se-á diversas configurações, cada
uma representada como uma espécie competindo durante a evolução no
\acs{es}, que terão maior ou menor esforço computacional dedicado
conforme sua melhor ou pior evolução.

Ainda assim, a avaliação de uma estratégia em força bruta --- aqui se
referindo a otimizar todas as possíveis configurações e identificar a
melhor convergência delas --- não parece ser a melhor maneira de
abordar o problema, mesmo que houvesse capacidade computacional para
realizar tal tarefa em tempo hábil. Percebeu-se a necessidade de
realizar a escolha de algumas configurações a serem testadas para
reduzir a quantidade de caminhos possíveis.  
Durante a realização do \emph{Ajuste Manual}, percebeu-se que os
eventos removidos devido à inconsistência eram sempre eventos de falso
alarme, mas não havia a ocorrência de perda de eventos de detecção
causados por esse tipo de remoção. Por isso, determinou-se que a
remoção de eventos de inconsistentes seria sempre realizada durante o
ajuste de parâmetros. Por outro lado, a remoção de eventos devido à
ruído era importante para remoção de pertubações rápidas geradas na
rede que não constituíam na mudança do patamar operativo da rede e,
com base nisso, determinou-se que a remoção de eventos ruidosos também
sempre seria realizada. Assim, a variável $\Delta I_{min}$ sempre
estará presente no material genético das espécies, bem como o $\sigma$
da Gaussiana a ser utilizada no filtro de derivada de Gaussiana e o
seu valor de corte $\delta_{min}$. Já para o caso da remoção de
eventos próximos, não era possível determinar se sua utilização era
necessária, nem qual das versões de remoção --- por média, ou sem
deslocamento --- era o que mais se adequada ao problema. No caso de
não usar corte, a variável $n_{min}$ não estará presente no material
genético das espécies, porém no caso oposto as espécies há a
ocorrência de um gene a mais, cujo fenótipo é inteiro. Ainda assim,
sua representação no genótipo é realizada por números reais, sendo
preciso determinar como codificar a informação. Simplesmente se
escolheu arredondar o valor na representação para obter o valor do
fenótipo. Finalmente, também não se sabia \emph{a priori} determinar
se a ordem de aplicação das remoções pode causar alguma influência na
capacidade de detecção e, apesar de não se esperar grandes diferenças
devido à mudança da ordem em que eles seriam removidos, decidiu-se
testar ambas configurações.  

Assim, decidiu-se empregar cinco configurações a serem analisadas, que
determinam as espécies utilizadas no \acs{es}. As mesmas estão
descritas no início deste capítulo, constituindo as espécies
\ref{item:esp1}--\ref{item:esp5} na pp.~\pageref{item:esp1}.

Além da questão das configurações, era necessário determinar a
inicialização. Para garantir melhor convergência, empregou-se
fronteiras mínimas e máximas para cada um dos valores representados
pelos genes, sendo estas:

\begin{subequations}\label{eq:fronteiras}
\begin{equation}\label{eq:fronteira_sigma}
0,003 < \sigma_{gauss} < 0,5
\end{equation}
\begin{equation}\label{eq:fronteira_delta}
0,009 < \delta_{min} < 0,5
\end{equation}
\begin{equation}\label{eq:fronteira_dimin}
0,0 < \Delta{I}_{min} < 1,0
\end{equation}
\begin{equation}\label{eq:fronteira_nmin}
5,0 < n_{min} < 500,0\footnote{Apesar do $n_{min}$ ser inteiro na
representação no fenótipo, sua representação é por número reais no
genótipo por estar utilizando-se do algoritmo genético por \acs{es}.
Também se reitera, aqui, que esse gene só estará presente nas
configurações que realizam a remoção de eventos próximos.}
\end{equation}
\end{subequations}

A inicialização foi aleatória uniforme dentro das fronteiras descritas
em \ref{eq:fronteiras}. Para o parâmetro da estratégia evolutiva, porém,
iniciou-se esse valor de acordo com \ref{eq:sigma_init}, garantindo
que inicialmente cada individuo tenha, em média, 95\% de chance de
explorar uma região no interior a 10\% do espaço de solução permitido. Os
limites inferiores e superiores foram determinados por
\ref{eq:sigma_min} e \ref{eq:sigma_max}, respectivamente. Os valores
de $\sigma_{min,i}$ e $\sigma_{max,i}$ garantem que, no mínimo, o material
genético sofrerá pertubações dentro da região de 0,01\% da região
de solução em 67\% dos casos, enquanto o mesmo será perturbado no máximo 
em 30\% da região para a mesma probabilidade:

\begin{subequations}
\begin{equation}\label{eq:sigma_init}
2\sigma_{init,i}=0,1\Delta x_i
\end{equation}
\begin{equation}\label{eq:sigma_min}
\sigma_{min,i}=1\times10^{-4}\Delta x_i
\end{equation}
\begin{equation}\label{eq:sigma_max}
\sigma_{max,i}=3\times10^{+3}\sigma_{min,i}
\end{equation}
\end{subequations}

\noindent onde $\Delta x_i$ é a região do espaço dentro das fronteiras
para a i-ésima variável.

Além disso, há de escolher os parâmetros livres no cálculo da aptidão
explicitada em \ref{eq:regra_pontuacao}. Os parâmetros $\gamma_{det}$
e $\gamma_{fa}$ permitem priorizar uma melhor taxa de detecção ou
menor taxa de falso alarme, em detrimento de um acréscimo na taxa de
falso alarme ou decréscimo na taxa de detecção, respectivamente. Isso
pode ser feito dando valores maiores ou menores para a recompensa de
encontrar corretamente um evento de transitório ou a penalidade de
aceitar um evento que se constitui de falso alarme.
Os valores utilizados foram a unidade para $\gamma_{det}$ e -0,9 para
$\gamma_{fa}$. O objetivo dessa diferença é uma leve priorização para
a detecção de eventos em detrimento de uma maior taxa de falso alarme.
O valor para $\gamma_{rem}$ foi de -0,05. Esse parâmetro é importante
para garantir a obtenção de soluções mais simples, que exploram a
capacidade do filtro de Gaussiana em gerar o mínimo possível de
candidatos, o que reduz a carga da remoção de eventos através de
cortes em $\Delta{I}_{min}$ ou $n_{min}$. Ao mesmo tempo, 
a geração de menos candidatos significa que a solução exige menor
esforço computacional, e é mais adequada para a aplicação na operação
em tempo real. Ainda, o parâmetro \acs{jmax} foi definido como 50
amostras. 

Finalmente, cabe ainda decidir qual dos métodos de competição para as
espécies será utilizado e os tamanhos das populações. Como a
implementação do método de competição intraespécies, da maneira
atualmente disponível neste trabalho, acaba favorecendo espécies com
carga genética mais simples de serem otimizadas ou que começaram em
condições privilegiadas --- quando sem o mecanismo de adiamento da
competição ---, irá optar-se pela competição intraespécie. Para
\acs{mu}, utilizou-se o valor de inicial de 30 indivíduos para cada
uma das espécies. Como hão cinco espécies que serão otimizadas, a
população global é de 150 indivíduos. Cada indivíduo gerará 7 filhos,
de forma que o valor de \acs{lambda} global é de 1050 indivíduos.





