
\documentclass[msc,numbers]{coppe}
\let\printglossary\relax
\let\theglossary\relax
\let\endtheglossary\relax
\usepackage[utf8]{inputenc}
\usepackage{amsmath,amssymb}
\usepackage[unicode]{hyperref}
\usepackage{indentfirst}
\usepackage{graphicx}
\usepackage{subcaption} % Caption inside subfigures
\usepackage{rotating} % Para rodar as figuras e tabelas
\usepackage{pdflscape} % Para colocar paginas na horizontal no pdf
\usepackage{multirow} % Tabelas com celulas agrupadas
\usepackage{placeins} % mais floats (figuras tabelas etc)!
\usepackage{array}
\usepackage{import} % To access created latex images in different folders
\usepackage{enumerate} % To have access to special enumerations
\usepackage[inline]{enumitem}
\usepackage{steinmetz} % Para ter acesso ao angulo de fase como
% deveria ser.
\usepackage{graphicx}% for \rotatebox
\usepackage{everypage}
\usepackage{environ}

\newcounter{abspage}% \thepage not reliab

\makeatletter
\newcommand{\newSFPage}[1]% #1 = \theabspage
  {\global\expandafter\let\csname SFPage@#1\endcsname\null}

\NewEnviron{SidewaysFigure}{\begin{figure}[p]
\protected@write\@auxout{\let\theabspage=\relax}% delays expansion until shipout
  {\string\newSFPage{\theabspage}}%
\ifdim\textwidth=\textheight
  \rotatebox{90}{\parbox[c][\textwidth][c]{\linewidth}{\BODY}}%
\else
  \rotatebox{90}{\parbox[c][\textwidth][c]{\textheight}{\BODY}}%
\fi
\end{figure}}

\AddEverypageHook{% check if sideways figure on this page
  \ifdim\textwidth=\textheight
    \stepcounter{abspage}% already in landscape
  \else
    \@ifundefined{SFPage@\theabspage}{}{\global\pdfpageattr{/Rotate 0}}%
    \stepcounter{abspage}%
    \@ifundefined{SFPage@\theabspage}{}{\global\pdfpageattr{/Rotate 90}}%
  \fi}
\makeatother


\makeatletter
\newcommand*{\textlabel}[2]{%
\edef\@currentlabel{#1}% Set target label
\phantomsection% Correct hyper reference link
#1\label{#2}% Print and store label
}
\makeatother

\hypersetup{
unicode=true,          % non-Latin characters in bookmarks
pdftitle={},           % title
pdfauthor={Werner Spolidoro Freund},     % author
pdfsubject={Dissertação de Mestrado},   % subject of the document
colorlinks=true,       % false: boxed links; true: colored links
pdfdisplaydoctitle=true,
citecolor=black,%
filecolor=black,%
linkcolor=black,%
urlcolor=black%
}


\usepackage[style=long,nomain,nonumberlist,shortcuts]{glossaries} % Para as listas de simbolos e abreviaturas
\renewcommand*{\glsacrpluralsuffix}{}
\renewcommand*{\glsupacrpluralsuffix}{}
\renewcommand*{\acrpluralsuffix}{}
\newglossary[slg]{Simb}{sym}{sbl}{Lista de S{í}mbolos}
\newglossary[alg]{Abrev}{abr}{abv}{Lista de Abreviaturas}

\usepackage{etoolbox} % Para robustify
\robustify{\gls}
\robustify{\url}

% Mutliplas referências de pé de nota com o mesmo número
\newcommand*{\fnref}[1]{\textsuperscript{\ref{#1}}}

\makeglossaries


\begin{document}

  \newacronym[type=Simb]{kcal}{kcal}{quilocalorias}
\newacronym[type=Simb]{toe}{toe}{toneladas equivalentes de petróleo}
\newacronym[type=Simb]{watt}{W}{watt}
\newacronym[type=Simb]{va}{VA}{volt-ampère}
\newacronym[type=Simb]{var}{VAr}{volt-ampère reativo}
\newacronym[type=Simb]{wh}{Wh}{watt hora}
\newacronym[type=Simb]{co2}{\protect{$CO_2$}}{gás carbônico}
\newacronym[type=Simb]{p}{P}{potência ativa}
\newacronym[type=Simb]{q}{Q}{potência reativa}
\newacronym[type=Simb]{d}{D}{potência harmônica}
\newacronym[type=Simb]{s}{S}{potência aparente}
\newacronym[type=Simb]{dp}{\protect{$\Delta{P}$}}{variação de
\acrlong{p}}
\newacronym[type=Simb]{dq}{$\Delta{Q}$}{variação de \acrlong{q}}
\newacronym[type=Simb]{dd}{$\Delta{D}$}{variação de \acrlong{d}}
\newacronym[type=Simb]{ds}{$\Delta{S}$}{variação de \acrlong{s}}
\newacronym[type=Simb]{i}{I}{corrente Elétrica}
\newacronym[type=Simb]{hz}{Hz}{hertz}
\newacronym[type=Simb]{c1}{C1}{aparelhos de consumo permanente}
\newacronym[type=Simb]{c2}{C2}{\gls{fsm} ou aparelhos de estados múltiplos}
\newacronym[type=Simb]{c2a}{C2a}{\gls{fsm} ou aparelhos de estados
múltiplos com ciclos bem-definidos}
\newacronym[type=Simb]{c2b}{C2b}{\gls{fsm} ou aparelhos de estados
múltiplos com ciclos aleatórios}
\newacronym[type=Simb]{c3}{C3}{\gls{fsm} de dois estados ou aparelhos liga/desliga}
\newacronym[type=Simb]{c4}{C4}{aparelhos com níveis continuos de consumo}
\newacronym[type=Simb]{c5}{C5}{aparelhos com consumo continuamente variável}
\newacronym[type=Simb]{c6}{C6}{aparelhos com características similares}
\newacronym[type=Simb]{fp}{FP}{fator de potência}
\newacronym[type=Simb]{det_eff}{$\eta_{det}$}{eficiência de
detecção}
\newacronym[type=Simb]{class_eff}{$\eta_{class}$}{eficiência de
classificação}
\newacronym[type=Simb]{total_eff}{$\eta_{total}$}{eficiência total}
\newacronym[type=Simb]{nid}{$N_{id}$}{número de eventos
corretamente detectados e classificados}
\newacronym[type=Simb]{nreais}{$N_{reais}$}{número de eventos
causados pelos equipamentos na rede}
\newacronym[type=Simb]{nfp}{$N_{fp}$}{número de eventos
devido a falsos positivos}
\newacronym[type=Simb]{nni}{$N_{ni}$}{número de eventos
não identificados}
\newacronym[type=Simb]{nap}{$N_{ap}$}{número de aparelhos}
\newacronym[type=Simb]{nt}{$N_{ap}$}{número de eventos de transitórios}
\newacronym[type=Simb]{p_eff_i}{$\rho_{En}^i$}{taxa de
reconstrução em energia para o i-ésimo aparelho}
\newacronym[type=Simb]{p_eff}{$\rho_{En}$}{taxa de
reconstrução em energia do \acrshort{nilm}}
\newacronym[type=Simb]{en_res_i}{$\varepsilon^i$}{energia redundante
para o i-ésimo aparelho}
\newacronym[type=Simb]{en_res}{$\varepsilon$}{energia redundante}
\newacronym[type=Simb]{e_id_i}{$E_{id}$}{energia corretamente
identificada para o i-ésimo aparelho}
\newacronym[type=Simb]{en_eff_i}{$\eta_{En}^i$}{eficiência de
reconstrução em energia para o i-ésimo aparelho}
\newacronym[type=Simb]{en_eff}{$\eta_{En}$}{eficiência de
reconstrução em energia do \acrshort{nilm}}
\newacronym[type=Simb]{red_eff_i}{$\rho_{red}^i$}{taxa de
redundância de energia para o i-ésimo aparelho}
\newacronym[type=Simb]{red_eff}{$\rho_{red}$}{taxa de
redundância de energia do \acrshort{nilm}}
\newacronym[type=Simb]{en_pres}{$\eta_{En,prec}^i$}{fração de energia
corretamente identifica em relação ao total de energia detectado para
o i-ésimo aparelho}
\newacronym[type=Simb]{medidaf}{$F^i$}{medida-F para o i-ésimo aparelho}

  \newacronym[type=Abrev]{ocde}{OCDE}{Organização para a Cooperação e
Desenvolvimento Econômico}
\newacronym[type=Abrev,\glslongpluralkey={Países em
Desenvolvimento}]{ped}{PED}{País em Desenvolvimento}
\newacronym[type=Abrev]{onu}{ONU}{Organização das Nações Unidas}
\newacronym[type=Abrev]{pme}{PME}{Programa de Mobilização Energética}
\newacronym[type=Abrev]{inmetro}{INMETRO}{Instituto Nacional de
Metrologia, Normalização e Qualidade Industrial}
\newacronym[type=Abrev]{proesco}{PROESCO}{Programa de apoio a Projetos
de Eficiência Energética}
\newacronym[type=Abrev]{pbe}{PBE}{Programa Brasil de Etiquetagem}
\newacronym[type=Abrev]{eletrobras}{Eletrobras}{Centrais Elétricas
Brasileiras S.A.}
\newacronym[type=Abrev]{procel}{PROCEL/\acs{eletrobras}}{Programa
Nacional de Conservação de Energia Elétrica}
\newacronym[type=Abrev]{bndes}{BNDES}{Banco Nacional de Desenvolvimento
Econômico e Social}
\newacronym[type=Abrev]{conpet}{CONPET}{Programa Nacional de Racionalização do
Uso dos Derivados do Petróleo e do Gás Natural}
\newacronym[type=Abrev]{petrobras}{Petrobrás}{Petróleo Brasileiro S.A.}
\newacronym[type=Abrev]{mdic}{MDIC}{Ministério do Desenvolvimento, da Indústria
e do Comércio Exterior}
\newacronym[type=Abrev]{mme}{MME}{Ministério de Minas e Energia}
\newacronym[type=Abrev]{proalcool}{Proálcool}{Programa Nacional do
Álcool}
\newacronym[type=Abrev]{pnef}{PNEf}{Programa Nacional de Eficiência
Energética}
\newacronym[type=Abrev]{epe}{EPE}{Empresa de Pesquisa Energética}
\newacronym[type=Abrev]{pee}{PEE}{Programa de Eficiência Energética}
\newacronym[type=Abrev]{ee}{EE}{Eficiência Energética}
\newacronym[type=Abrev]{pne2030}{PNE2030}{Plano Nacional de Energia
2030}
\newacronym[type=Abrev,\glslongpluralkey={Planos Decenais de
Energia}]{pde}{PDE}{Plano Decenal de Energia}
\newacronym[type=Abrev]{aneel}{ANEEL}{Agência Nacional de Energia
Elétrica}
\newacronym[type=Abrev,\glslongpluralkey={Pesquisas de Posse e Hábito
de Eletrodomésticos}]{pph}{PPH}{Pesquisa de Posse e Hábito de
Eletrodomésticos}
\newacronym[type=Abrev]{beu}{BEU}{Balanço de Energia {\'U}til}
\newacronym[type=Abrev]{ti}{TI}{Tecnologia da Informação}
\newacronym[type=Abrev]{glp}{GLP}{Gás Liquefeito do Petróleo}
\newacronym[type=Abrev]{ibge}{IBGE}{Instituto Brasileiro de Geografia
e Estatistica}
\newacronym[type=Abrev]{cepel}{CEPEL/\acs{eletrobras}}{Centro de
Pesquisas de Energia Elétrica}
\newacronym[type=Abrev]{nilm}{NILM}{Monitoramento Não-Invasivo de Cargas
Elétricas}
\newacronym[type=Abrev]{dnilm}{dNILM}{\acs{nilm} de Arquitetura
Distribuida}
\newacronym[type=Abrev]{vast}{VAST}{Algoritmo de \emph{Viterbi} com
Transições Esparsas}
\newacronym[type=Abrev,\glslongpluralkey={Tecnologias de Comunição e
Informação}] {ict}{ICT}{Tecnologia de Comunicação e Informação}
\newacronym[type=Abrev]{eua}{EUA}{Estados Unidos da América}
\newacronym[type=Abrev]{pca}{PCA}{Análise de Componentes Principais}
\newacronym[type=Abrev]{pcd}{PCD}{Análise de Componentes Discriminantes}
\newacronym[type=Abrev]{fsm}{FSM}{Máquina de Estados Finitos}
\newacronym[type=Abrev]{fex}{FEX}{Extração de Características}
\newacronym[type=Abrev]{ted}{TED}{\emph{The Energy Detective}}
\newacronym[type=Abrev,\glslongpluralkey={Redes Neurais Artificiais}]{rna}
{RNA}{Rede Neural Artificial}
\newacronym[type=Abrev]{cdm}{CDM}{Mecânismo de Decisão por Comissão}
\newacronym[type=Abrev]{mco}{MCO}{Ocorrência Mais Comum}
\newacronym[type=Abrev]{lur}{LUR}{Menor Resíduo Unificado}
\newacronym[type=Abrev]{mle}{MLE}{Estimativa de Máxima-Verossimilhança}
\newacronym[type=Abrev]{roc}{ROC}{\emph{Receiver Operating
Chracteristic}}
\newacronym[type=Abrev]{avac}{AVAC}{Aquecimento, Ventilação e Ar
Condicionado}
\newacronym[type=Abrev]{som}{SOM}{Mapas Auto-Organizáveis}
\newacronym[type=Abrev]{es}{ES}{Algoritmo Genético de Estratégia Evolutiva}
\newacronym[type=Abrev]{cse}{CSE}{\emph{Fraunhofer Center for
Sustainable Energy Systems}}
\newacronym[type=Abrev]{isodata}{ISODATA}{\emph{Iterative
Self-Organizing Data Analysis Techniques Algorithm}}
\newacronym[type=Abrev]{pdf}{pdf}{Função de Distribuição de Probabilidade}
\newacronym[type=Abrev]{cdf}{cdf}{Função de Distribuição Acumulada}
\newacronym[type=Abrev]{hmm}{HMM}{Modelo Oculto de \emph{Markov}}
\newacronym[type=Abrev]{fft}{FFT}{Transformada Rápida de \emph{Fourier}}
\newacronym[type=Abrev]{redd}{REDD}{\emph{The Reference Energy
Disaggregation Data Set}}
\newacronym[type=Abrev]{blued}{BLUED}{\emph{Building-Level
fUlly-labeled dataset for Electricity Disaggregation}}
\newacronym[type=Abrev]{mlp}{MLP}{\emph{MultiLayer Perceptron}}
\newacronym[type=Abrev]{rbf}{RBF}{\emph{Radial Basis Function}}
\newacronym[type=Abrev]{svm}{SVM}{\emph{Support Vector Machine}}
\newacronym[type=Abrev]{mse}{MSE}{Mínimo Erro Quadrático}
\newacronym[type=Abrev]{emi}{EMI}{\emph{Interferência Eletromagnética}}
\newacronym[type=Abrev]{tw}{TW}{Transformada \emph{Wavelet}}
\newacronym[type=Abrev]{svd}{SVD}{Decomposição em Valores Singulares}
\newacronym[type=Abrev]{coppe}{COPPE/UFRJ}{Instituto Alberto Luiz Coimbra
de Pós-graduação e Pesquisa de Engenharia}
\newacronym[type=Abrev]{peecoppe}{PEE/COPPE}{Programa de Engenharia
Elétrica da \acs{coppe}}
\newacronym[type=Abrev]{lps}{LPS}{Laboratório de Processamento de
Sinais}
\newacronym[type=Abrev]{fir}{FIR}{Filtro de Resposta ao Impulso Finita}


  \title{Monitoração Não-Invasiva de Cargas Elétricas Residenciais}
  \foreigntitle{Non-Intrusive Residential Load Monitoring}
  \author{Werner Spolidoro}{Freund}
  \advisor{Prof.}{José Manoel de}{Seixas}{D.Sc.}

  \examiner{Prof.}{José Manoel de Seixas}{D.Sc.}
  \examiner{Prof.}{Luiz Pereira Calôba}{Dr.Ing.}
  \examiner{Prof.}{Carlos Augusto Duque}{D.Sc.}
  \examiner{Dr.}{Charles Bezerra do Prado}{D.Sc.}

  \department{PEE}
  \date{12}{2013}

  \keyword{Monitoração Não-Invasiva de Cargas Elétricas}
  \keyword{NILM}
  \keyword{Algoritmo Genético}

  \maketitle

  % TODO Colocar páginas das figuras e tabelas extraídas
  \frontmatter
  \dedication{Dedico à força de vontade e superação.}


  \chapter*{Agradecimentos}

Agradeço aos meus pais por tornar isso tudo possível. Vocês que
tiveram o maior peso para que isso se transformasse em realidade.
Agradeço de coração por todo esforço e trabalho que tiveram com o
intuíto de me verem chegar aqui. Aos meus avós por todo o carinho e
apoio que sempre me deram, sempre me incentivando para conseguir
atingir meus sonhos. Ao meu irmão mais novo que sacríficou seu tempo
livre fazendo algumas de minhas tarefas enquanto eu estava
trabalhando.

Ao meu orientador, Seixas, que me deu suporte e me guiou neste
trabalho, apesar de todas condições atípicas. Seus conselhos foram
muito valiosos, sua ajuda foi muito mais do que essêncial. Aos outros
orientadores que tive, Torres e Damazio, suas orientações ajudaram a
tornar-me o que sou hoje.

Aos engenheiros do CATE no \acrlong{cepel}, João, Aroldo, Guilherme e
Victor por toda a ajuda e compreensão durante o tempo de trabalho. A
convivência com vocês foi muito agradável! Também à Beth por sempre
ter sido prestativa.

Ao pessoal do LPS, Balabram, Junior, Moura, Grael, Rodrigo, João
Victor, Diego e Hellen. Vocês fazem o laboratório ser o que ele é, sem
vocês ele não tem graça alguma. À Ana e Talia, vocês fazem muita falta
lá. Agradeço em especial ao João Victor pelo suporte providencial no
momento em que precisei, espero ter retornado da maneira possível com
o máximo de conhecimento em troca. Desculpas ao Grael por todo
incomodo causado, muito obrigado por fazer o melhor que podia. Aos
meus orientados, Diego e Hellen, me desculpem se exagerei na cobrança:
corram atrás de seus futuros que o \emph{Walhalla} espera por vocês!

Um agradecimento a toda ajuda, paciência e instruções oferecidos pela
Dani do PEE.

Finalmente, agradeço a todos meus amigos que escutaram diversas vezes:
``não posso, tenho que terminar o mestrado'', mas nunca deixaram de
compreender o momento em que estava passando. Eu senti tanto a falta
de vocês quanto vocês sentiram a minha. Hoje chegou o dia!


  \begin{abstract}

Este trabalho foi realizado em colaboração com o \acs{cepel} no
desenvolvimento da tecnologia conhecida como \acf{nilm}.
A monitoração não-invasiva pode ser aplicada para garantir a qualidade
de energia, o diagnóstico de carga, identificação de aparelhos
defeituosos ou com consumo excessivo de energia e eficiência
energética. Este trabalho abordará a etapa de detecção de eventos de
transitórios causados por aparelhos elétricos, o qual é necessária
para a obtenção dos traços de informação deixados por equipamentos
durante a alteração de seu estado de operação.
A proposta do \acs{cepel} é a utilização de um filtro de
derivada de Gaussiana que gera candidatos a eventos de transitórios,
os mesmos ainda analisados por outros testes para eliminação de falsos
eventos em seguida.

A contribuição do trabalho está na sistematização da determinação dos
parâmetros necessários para o método proposto pelo \acs{cepel}, que
permite o ajuste automático dos mesmos para diferentes cenários. Para
isso, foi implementada uma adaptação de um algoritmo genético, que
permite dinâmica para alocação de maior esforço computacional para
configurações que melhor se adequam ao problema. Além disso, o
otimizador faz parte de um extenso ambiente de análise que trouxe
vantagens operacionais em termo de infraestrutura para o projeto.

Ao aplicar o algoritmo na base de dados, o mesmo mostrou capacidade de
generalização para as condições avaliadas. Os resultados foram
comparados com os trabalhos recentes desenvolvidos neste contexto.
Obteve-se taxa de detecção superior a 80$\;$\% para taxa de falso alarme
inferior a 1$\;$\% em cenários com operação simultânea de diversos
aparelhos e dinâmica de carga. Já para os conjuntos com operações
simples de equipamentos as taxas são de 98,1$\;$\% e 0,6$\;$\%,
respectivamente.

\end{abstract}


  \begin{foreignabstract}

Non-intrusive monitoring may be applied to power quality, load
diagnosis, faulty appliance detection or with excessive energy
consumption and energy efficiency. This work will discuss the
technology developed when addressing the issue of energy efficiency
for the residential sector. The transient event detection is its
focus, where it was used a Gaussian core filter to obtain the
responsive regions which indicates an appliance state change. The
approach parameters were optimize via genetic algorithm. In order to
improve the problem comprehension, it was used Self-Organizing Maps to
explore the disturbance chracteristics due to an appliance state
change. They were also evaluated as an method to aid increasing previous
methodology efficiency. Detection rates higher than 84\% for false
alarm rates lower than 16\% were achieved while in a pratical scenario
application.

\end{foreignabstract}


  \addtocounter{page}{1}
  \vfill
  \cleardoublepage{} \phantomsection{} \addcontentsline{toc}{chapter}{Sumário}
  \tableofcontents \vfill
  \cleardoublepage{} \phantomsection{}
  \listoffigures \vfill
  \cleardoublepage{} \phantomsection{}
  \listoftables \vfill
  \cleardoublepage{} \phantomsection{} \addcontentsline{toc}{chapter}{Lista de Símbolos}
  %\renewcommand{\glossarypreamble}{Texto Simbulos}
  \printglossary[type=Simb] \vfill
  \cleardoublepage{} \phantomsection{} \addcontentsline{toc}{chapter}{Lista de Abreviaturas}
  \renewcommand{\glossarypreamble}{No caso de algumas abreviaturas internacionalmente conhecidas, optou-se por mantê-las em sua lingua original.}
  \printglossary[type=Abrev] \vfill

  \mainmatter{}
  \chapter{Introdução}

\section{Objetivo}

O \gls{cepel} vem atuado no desenvolvimento de um \gls{nialm}. Este trabalho
tem como objetivo auxiliar no progresso desse dispositivo, para que o
mesmo se torne operativo e possa ser aplicado em auxílio as \gls{pph}.

O Seminário de Mestrado apresentado realizou o levantamento bibliográfico com o
objetivo de guiar o desenvolvimento da dissertação. O Capítulo 4 detalha alguma
das topologias possíveis para o \gls{nialm}. No Capítulo 2 e 3 estão
apresentadas detalhadamente as principais aplicações para esse dispositivo.

%\section{Motivação}


%\section{Estrutura Capitular}

%A ser preenchido.



  \chapter{A questão energética-ambiental}

% O ser humano e sua capacidade de ir além da energia endossomática
Estima-se que o ser humano consome entre 2500 e 3000 quilocalorias/dia sob a 
forma de alimentos \cite{hemery}. Apenas cerca de 20\% dessa energia poderá ser 
reinvestida em atividades, de forma que a capacidade humana de produção através
de energia endossomática - provida por seu metabolismo - é de apenas 500 a 600 
quilocalorias/dia. No entanto, é a especificidade do ser humano de utilizar 
fluxos de energia exossomáticos - fontes não provenientes de seu metabolismo 
- que o permite transformar o meio facilitando e melhorando sua qualidade de
vida \cite{rippel}.

% A utilização de energia exossomática e os danos ao planeta
Os fluxos exossomáticos estão presentes em todas sociedades e são dependentes do
nível de desenvolvimento das mesmas. Em sociedades primitivas, os resultados dos
atos da humanidade exerciam pouca influência no meio ambiente, pois estes eram
absorvidos por aquele. Entranto, com o surgimento de sociedades industriais,
ocorreu uma mudança radical no uso dos recursos naturais e nos seus efeitos
ambientais. Com a produção e o consumo de massa, baseados no uso intensivo do
petróleo e da eletricidade como fontes energéticas, a agressão humana ao meio 
ambiente tornou-se maior de maneira que o mesmo, em muitas situações, não 
consegue mais reequilibrar-se. Os principais problemas ambientais que surgiram 
no cenario mundial tem como fator de destaque vinculados a cadeia da energia, 
da produção ao uso final \cite{rippel,pen15_eff_energ,jatoba}. 

% A questão de sustentabilidade
% Ecologia radical
Em resposta à degradação do meio ambiente se dá a questão de
sustentabilidade. As primeiras objeções à poluição ambiental devido ao processo 
de industrialização, em especial sua aceleração no início do século XX 
foi da ecologia radical, dando atenção as questões de proteção e conservação da natureza.
Há uma separação de territórios especiais para uma proteção integral e, numa
visão em larga medida romantizada da natureza, muitas vezes sem permissão 
de nenhum uso antrópico \cite{jatoba}.

% Ambientalismo moderado
A política ambiental preponderante atualmente é o discurso do ambientalismo moderado, 
surgido em meio a Crise do Petróleo na década 1970, onde, percebendo-se a
inviabilidade de sustentação do modelo econômico levando em conta o esgotamento
progressivo dos recursos naturais do planeta, se fez necessário a discussão de como 
colocar em prática as propostas da ecologia radical, mas tendo cautela de não 
necessariamente frear o crescimento econômico ou alterar substancialmente o 
modelo de desenvolvimento vigente - trazendo o conceito de desenvolvimento
sustentável. Devido à Crise, houve uma busca mundial pela redução da 
dependência no petróleo e outras fontes fósseis em sua matriz energética, 
procurando fontes renováveis e menos poluentes que as fontes fósseis 
\cite{jatoba,eff_dec_energ_2012,rippel}. 

% Ressaltar aqui a componente social do ambientalismo 

% Eficiencia energetica
A maneira mais efetiva, eleita pela \gls{onu} \cite{onu}, de reduzir os impactos ambientais locais e globais sem
prejuizos ao desenvolvimento, refletindo a imagem de desenvolvimento
sustentável, é através da eficiência energética \cite{rippel,dissert_artur_cursino}. O uso eficiente de 
energia deve ser entendido como o menor consumo possível de energia para obter uma mesma 
quantidade de produto ou serviço \cite{pen15_eff_energ}. A melhoria pode ser obtida nos diversos
fluxos energéticos da sociedade, através da melhoria da geração de uma fonte 
energética, substituição por outra fonte mais eficiente, ou melhoria da
eficiência nos consumidores. Nas sociedades modernas a eletricidade 
como recurso energético vem adquirindo importância cada vez mais vital devido a
sua versabilidade e eficiência \cite{pen15_eff_energ,dissert_caires}, 
obtendo um lugar de destaque quando em debate a eficiência energética. 

Além dos beneficios ambientais, a eficiência energética \cite{jannuzzi,slides_eff_energetica}: 

\begin{itemize}
\item reduz ou posterga as necessidades de investimentos em geração, transmissão 
e distribuição de energia elétrica; 
\item reduz o custo de energia para o consumidor final; 
\item contribui para a confiabilidade do sistema elétrico; 
\item traz o aumento da intensividade econômica com a redução da intensividade
energética. 
\end{itemize}

% Eficiencia energetica no mundo desenvolvido
A tendência nos países desenvolvidos é a realização de esforços 
cada vez maiores no sentindo de aumentar a eficiência energética, 
estimando-se, por exemplo, um consumo 49\% maior nos paises da \gls{ocde} 
em 1996 caso não houvessem sido adotadas medidas de racionalização 
e eficiência energética após a Crise do Petróleo
\cite{goldemberg,slides_eff_energetica}.

% Eficiencia energética nos países em desenvolvimento 
Os \glspl{ped} não possuem uma capacidade de redução energética tão grande
quanto os países da \gls{ocde}, uma vez que seu consumo energético per capita é
reduzido, justamente por ser necessário o desenvolvimento para aumentar o
consumo. Por outro lado, é possível aliviar a pressão sobre a
oferta energética através da melhoria dos níveis de eficiência energética nos
diversos setores da sociedade, contrapondo ao pensamento de que para que
haja desenvolvimento é preciso que ocorram impactos ambientais e crescimento no
consumo total de energia - chamado de efeito \emph{leapfrogging}
\cite{goldemberg,dissert_maria_ines_matos}. 

% Eficiencia energética no Brasil
Assim, a preocupação com eficiência energética se justifica mesmo no Brasil, 
que apresenta quase metade de sua matriz energética proveniente de fontes 
renováveis e preços de produção de energia economicamentes competitivos.
Diversas iniciativas vêm sendo empreendidas há mais de 20 anos, dentre elas se
destacam \cite{projecao_demanda_2012,slides_eff_energetica}:

\begin{enumerate}
\item 
\end{enumerate}

% Consumo residencial bastante elevado no brasil

% Necessidade de obter o perfil de consumo para escolher equipamentos a serem
% melhorados

% Ainda, estudos no exterior mostram que o feedback de consumo poder mudar o
% habito de consumo dos consumidores. Problema, Brasil energia barata e
% dificilmente consumidores mudam habito por isso.

%
%
%
%
%
%\cite{techreport-energetico}


  \chapter{A Eficiência Energética no Setor Residencial}
\label{chap:ee_retorno}

O setor residencial representa um grande potencial para o alcance de uma melhor
\gls{ee}, em especial quando considerando apenas o uso da eletricidade. Isso se
tornará ainda mais relevante com o desenvolvimento do país. Este capítulo detalha
o seu perfil de consumo e potencial para explorar a \gls{ee}, dando 
especial enfase a eletricidade (Sessão~\ref{sec:ee_setor_residencial}). Faz-se
referência as novas tendências para explorar esse potencial no
mundo (Sessão~\ref{sec:ee_res_exp}), que podem se utilizar das possibilidades 
fornecidas pelas redes inteligentes (\emph{smart grid}) e seus aparelhos de 
medição inteligentes (\emph{advanced/smart meters}).

\section{Eletricidade e Potencial de Eficiência Energética 
no Setor Residêncial}
\label{sec:ee_setor_residencial}

A eletricidade é o segundo meio energético de maior participação na matriz 
energética brasileira, representando 16,7\% da demanda, 
Figura~\ref{fig:matriz_bra_2011}. 
Há uma tendência de crescimento dessa parcela na matriz indicado na 
Figura~\ref{fig:matriz_bra_evo}, que deverá progredir devido à fatores 
como \cite{iea_weo2010}:

\begin{figure}[h!t]
    \label{fig:eletricidade_brasil}
    \begin{center}
%
        \subfigure[]{%
            \label{fig:matriz_bra_2011}
            \includegraphics[width=0.8\textwidth]{imagens/matriz_energetica_brasileira_2011.pdf}
        } \\ %\hspace{0.05\textwidth}
        \subfigure[]{%
            \label{fig:matriz_bra_evo}
            \includegraphics[width=0.8\textwidth]{imagens/evolucao_matriz_energetica_brasil.pdf}
        }
%
    \end{center}
    \caption[Matriz energética brasileira.]{A matriz elétrica brasileira
em 2011 (a) e seu histório (b). Baseado (a) e extraído (b) de \cite{ben2012}.}% 
\end{figure}

\begin{itemize}
\item substituição da biomassa para eletricidade como meio de iluminação 
e aquecimento
no setor residêncial;
\item nesse setor também há um aumento do número de eletrodomésticos como
reflexo do desenvolvimento e melhor distribuição da riqueza;
\item a expansão do setor comercial e de serviços utilizando uma quantidade
maiores de aparelhos elétricos (como ar condicionado, iluminação e equipamentos
de \acrshort{ti}); 
\item transformação do setor industrial, gradualmente substituindo o carvão e 
aumentando a presença de dispositivos elétricos.
\end{itemize}

No que diz respeito ao consumo de eletricidade, 
Figura~\ref{fig:eletricidade_por_setor}, 
o setor residencial possui uma posição de destaque no consumo, com uma demanda
de 111,97 T\acrshort{wh} e uma parcela equivalente à 26\% do total no ano de 2011, 
exprimindo a importância desse setor para explorar a capacidade de economia de 
eletricidade através de \gls{ee}. 

\begin{figure}[h!t]
\centering
\includegraphics[width=.6\textwidth]{imagens/consumo_por_setor.pdf}
\caption[Consumo de eletricidade por setor em 2011.]
{O consumo de eletricidade por setor em 2011. Baseado em \cite{ben2012}.}
\label{fig:eletricidade_por_setor}
\end{figure}

De acordo com o último estudo do \gls{beu}\footnote{O \gls{beu} é realizado em
intervalos decenais desde 1985.} com o ano base de 2004 \cite{beu}, o potencial de 
economia no consumo de eletricidade nesse setor chega a 13,60 T\acrshort{wh}, 
ou 1,17 milhões de \acrshort{toe}. Apenas como figura de mérito, quando considerando 
melhorias de eficiência em todas as fontes energéticas, o setor residencial
possui o terceiro maior potencial de economia, após o setor industrial e de
transporte, com uma capacidade de economia de 2,97 milhões de 
\acrshort{toe}. No entanto, ressalta-se que esses valores de potenciais de economia 
apresentados são aproximados e reduzidos em relação ao valor real, onde se
considera apenas a perda de energia na primeira transformação do processo
produtivo. Também não são consideradas possíveis alterações no consumo de fontes
energéticas, como a mencionada alteração de biomassa para eletricidade, uma
fonte menos poluente e mais eficiente. Finalmente, deve-se acrescentar que 
esse potencial é calculado utilizando o rendimento de equipamentos no estado 
da arte entre aqueles normalmente comercializados, e não os possíveis de 
se alcançar quando considerando a literatura técnica. Os dados  
do \gls{beu} são normalmente utilizados para os estudos em \gls{ee} para o setor
industrial e comercial.

No entanto, os dados utilizados pela \gls{epe} para os estudos atuais de \gls{ee} no setor
residencial, como os \glspl{pde} e \gls{pne2030}, por sua vez, são das
informações obtidas na última \gls{pph} realizada no Brasil entre 2004--2006 que 
possibilitam o estudo baseado em uma abordagem desagregada. Essa abordagem 
depende do número de domícilios, a posse média e hábito de consumo específico 
dos equipamentos eletrodomésticos --- que estão implicitamente indicados na curva de carga
ilustrada na Figura~\ref{fig:curva_carga} para a região sudeste --- e rendimento médio desses 
equipamentos no país, informação baseada nas tabelas do \gls{pbe}, coordenado pelo \gls{inmetro}. 
Também são utilizadas variáveis agregadas para o ajuste do modelo em questão
, sendo elas a relação entre o número de consumidores residenciais e população 
(que permite a projeção do número de consumidores a partir da projeção da população), 
e consumo médio por consumidor residencial
\cite{epe_eficiencia_2012,pde_2020,pne30_eff_energ}. 

% TODO Inserir a figura aqui de posse média
\begin{figure}[h!t]
\centering
\includegraphics[width=.9\textwidth]{imagens/curva_demanda_sudeste.pdf}
\caption[Curva de carga média para a região sudeste, ano base 2005.]{Curva de
carga média para a região sudeste, ano base 2005. Extraído de
\cite{result_procel_2005}.}
\label{fig:curva_carga}
\end{figure}

Esses estudos exploram a conservação de energia no ganho de eficiência 
de equipamentos eletrodomésticos, mas não consideram melhorias possíveis
em mudanças nos hábitos de consumo. Isso se justifica uma vez que a abordagem 
representa o ganho de conservação de energia através do progresso autônomo
ou tendencial, progresso esse ``que se dá por iniciativa do mercado, sem a
interferência de políticas públicas de forma espontânea, ou seja, através da
reposição natural do parque de equipamentos por similares novos e mais
eficientes ou tecnologias novas que produzem o mesmo serviço de forma mais
eficiente'' \cite[p.1]{pnef}, assim como por ``efeitos de programas e ações de
conservação já em execução no país'' \cite[p.247]{pde_2020}. 

Por outro lado, o progresso induzido refere-se à ``instituição de programas e ações
adicionais orientados para determinados setores, refletindo políticas púbicas;
programas e mecanismos ainda não implantados no Brasil'' \cite[p.247]{pde_2020}.
As políticas de progresso induzido estão detalhadas no \gls{pnef}, e a
estratégia adotada no tocante a mudanças nos hábitos de consumo é através da
educação com os programas do \gls{procel} e \gls{conpet}, \textlabel{com as
linhas}{text:prog_cepel}: Eficiência Energética na Educação Básica; Eficiência 
Energética na Formação Profissional; e Rede de Laboratórios e Centros de Pesquisa em Eficiência
Energética \cite[cap.5]{pnef}. No caso do \gls{procel}, investiu-se um montante superior a
R\$~4,5 Mi em 2011 em projetos voltados para o desenvolvimento e
aperfeiçoamento dessas três linhas \cite{procel_resultados_2012}, trabalhando na
coincientização, sensibilização e informação para obter uma melhor \gls{ee},
atuando nos três níveis de educação.

Entretanto, estudos no exterior mostram um potencial ainda a ser explorado de
econômia de energia elétrica no setor residencial, através do retorno de
informação da utilização de energia para o consumidor, uma estratégia de
progresso induzido que pode ser adicionada aos programas de \gls{ee} no Brasil. 
Conforme será visto na próxima sessão, esse potencial depende de aspectos, como a quantidade 
de informação retornada ao consumidor, essa inerentemente ligada à quantidade de
investimento utilizada na tecnologia para permitir um maior retorno, mas ainda,
dependem do modo em que esse retorno é fornecido para o mesmo no intuíto de
motivar ações sustentaveis de energia, sendo um problema complexo dependente de
aspectos sociais, culturais e psicológicos. 

Por fim, outros setores também podem reduzir seu consumo com políticas de
\gls{ee} aplicadas para o setor residencial devido a sua natureza similar a esse
setor --- nesse caso, não se referindo somente à politica de \gls{ee} sugerida. 
Os setores público e comercial, por exemplo, possuem prédios com natureza
de consumo correspondente ao do setor residencial, de forma que estratégias 
desenvolvidas possam sinergeticamente apresentar um potencial maior de 
economia de energia na matriz brasileira, em especial no que concerne estudos
para melhor eficiência de equipamentos.

\section{Expandindo o Potencial de Eficiência Energética através do Retorno de
Informação de Consumo}
\label{sec:ee_res_exp}

% Invisibilidade da eletricidade para os consumidores residencias
As novas fontes energéticas, dentre elas a eletricidade e gás natural
que atendem a demanda dos consumidores residenciais para a vasta 
variedade de serviços nas quais são utilizadas 
--- desde cocção, condicionamento do ambiente, a lazer e 
entreterimento ---, fluem invisivel e silenciosamente para seus domicílios, sem
deixar qualquer traço notável de sua utilização além do efeito final desejado pelo
consumidor. Para eles, o único retorno de seu consumo é informado na conta
apresentada pela concessionária, fornecida em um longo período após o consumo
(mensalmente, por exemplo). As informações nas contas são precárias, não informando 
muito além do total de energia consumido e o preço de energia. 
Os usuários não tem como inferir quais são os meios de uso final que demandam 
maior energia, nem a que ponto possíveis mudanças podem afetar sua demanda, 
seja através da mudança de seus hábitos ou na escolha 
de aparelhos mais eficientes. Atualmente, os usuários estão cegos quanto a essas
mudanças, não é possível vizualisar a energia que consomem. Além disso, as
informações fornecidas não permitem o consumidor comparar seu consumo com o de
outros, de modo que ele não é capaz de criar uma referência social para seu consumo.
Sem uma referência, o consumidor tem dificuldades para determinar se o consumo é
excessivo ou moderado e se é necessário algum tipo de intervenção 
\cite{aceee_2010_estudos_feedback}.

% Modos de alterar o hábito de consumo
Estratégias para intervir no comportamento podem ser classificadas de dois modos
\cite{aceee_2010_estudos_feedback,2009_epri}:
\begin{enumerate}
\item \textbf{Antecedentes}, que envolvem esforços para influênciar o que define 
um comportamento antes de sua realização; 
\item \textbf{De consequência}, que buscam alterar o que determina o 
comportamento, após que ele tenha ocorrido. 
\end{enumerate}

Exemplos de estratégias antecedentes são campanhas de informação 
com o objetivo de aumentar o conhecimento público sobre o impacto de suas 
escolhas e das opções para econômia de energia disponíveis --- como 
as já citadas ações do \gls{procel} e \gls{conpet} (ver p.~\pageref{text:prog_cepel}) ---,
engajar o indivíduo com um compromisso de mudança, criar metas de mudança
comportamentais, ou modelar e demonstrar o comportamento desejado. Já para
estratégias de consequência se pode citar recompensas, punições ou o
retorno de informação \cite{aceee_2010_estudos_feedback,2009_epri}. 

Iniciativas utilizando o retorno de informação mostraram-se altamente eficientes
em mudanças comportamentais com relação ao consumo energético \cite{
aceee_2010_estudos_feedback,2009_epri,2012_schleich__austria,
2011_zhifeng_smart_energy_savings,2006_darby,2009_nber_studies_us,
ucla_studies_1975_2011_usa}. O uso do retorno
de informação basea-se em que tanto resultados positivos ou
negativos podem modelar o comportamento. Resultados atribuidos como positivos 
irão torná-los em comportamentos mais atraentes, enquanto a atribuição de
resultados negativos propiciam comportamentos ruins em menos desejáveis. 
Sempre que possível, a atribuição negativa 
deve ser evitada, pois ela tende a reduzir a motivação e não coloca nada no 
lugar do comportamento evitado \cite{2010_aspectos_psicologicos_usa}.

A questão, por outro lado, não é apenas fornecer retorno do consumo ao usuário
final --- a própria conta de energia pode ser encarada como um meio de retorno
---, mas como o retorno pode ser utilizado para efetivamente motivar pessoas
para reduzir o seu consumo. Algumas considerações devem ser tomadas: primeiro,
quais são os tipos de retornos disponíveis (Subsessão \ref{ssec:ret_tipos}) e,
dentre eles, quais tem mostrado resultados mais eficientes na redução do
consumo energético (Subsessão \ref{ssec:ret_eff})? O que mais deve ser levado
em consideração quando preparando tais programas e estudos de \gls{ee}
(Subsessão~\ref{ssec:ret_outros})? Quais são as técnologias disponíveis para
fornecer essa informação e suas tendências (Subsessão \ref{ssec:ret_tec})?
Ainda, pessoas possuem diferentes atitudes, crenças e valores, sendo motivadas
de modos distintos. Uma breve consideração sobre a perspectiva psicológica que
envolve mudança comportamental será realizada (Subsessão \ref{ssec:asp_psic})
uma vez que esse aspecto é de principal relevância para o sucesso dos
programas.  Do mesmo modo, a apresentação visual da informação também irá
influênciar no êxito, e por isso o tema também será colocado em pauta
(Subsessão~\ref{ssec:asp_visuais}). De nenhuma maneira as breves considerações
realizadas nas subsessões sobre os aspectos psicológicos e visuais devem
substituir a análise de profissionais dessas áreas, servindo apenas para chamar
atenção para a importância, bem como introduzir os leitores, aos temas. 

\subsection{Tipos de Retorno}
\label{ssec:ret_tipos}

A categorização dos tipos de retorno que será apresentada iniciou-se em
\cite{2000_darby} e depois foi aprimorada por \cite{2009_epri}. A primeira divisão
toma em conta o modo no qual o retorno é fornecido, sendo possíveis o retorno 
direto ou indireto. O termo \emph{indireto} é
utilizado quando há alguma espécie de processamento antes de atingir o
consumidor, enquanto \emph{direto} determina o retorno instantaneamente entregue
ao usuário. Em seguida, realiza-se uma divisão em termos de frequência,
diferenciando quatro tipos de retorno \emph{indiretos}, e dois retornos
\emph{diretos}. A divisão é apresentada resumidamente em ordem crescente em 
termos de custo e 
quantidade de informação disponível, onde os dois retornos \emph{diretos} estão no
final da lista \cite{aceee_2010_estudos_feedback,2009_epri}:

\begin{enumerate}
\item \textbf{Retorno por Faturamento Simples}: conta de energia contendo 
k\acrshort{wh} 
consumido, o preço da tarifa unitária ($\text{R\$}/$k\acrshort{wh}), o custo 
total e outros possíveis ônus. Nessa forma de retorno normalmente carece 
estatísticas comparativas ou qualquer informação detalhada sobre os aspectos 
temporais do consumo;
\item \textbf{Retorno por Faturamento Aprimorado}: fornece informações mais 
detalhadas
sobre o padrão de consumo de energia, incluindo em alguns casos estatísticas
comparativas, tanto comparando o maior consumo do mes atual e sua despesa
aliados ao consumo histórico e/ou a comparação com outros domicílios
pertencentes ao grupo do consumidor;
\item \textbf{Retorno Estimado}: essa abordagem utiliza geralmente de técnicas
estatísticas para desagregar o total de energia baseado no tipo do
domicílio do consumidor, informação de aparelhos e dados de faturamento. O
retorno resultante fornece um relato detalhado do uso de eletricidade pelos
utensilios e dispositivos de maior importância. A forma mais comum é através de
ferramentas de auditoria de energia residencial baseadas na internet, oferecida
por um fornecedor de serviços a seus consumidores;
\item \textbf{Retorno Diário/Semanal}: esses relatórios utilizam a média de
dados e frequentemente incluem estudos de leitura dos medidores pelos próprios 
consumidores, assim como estudos nos quais individuos são providos com
relatórios mensais ou semanais do fornecedor de serviços ou entidade de
pesquisa;
\item \textbf{Retorno em Tempo-Real}: fornecido por dispositivos que exibem o
consumo em (praticamente) tempo-real e informações de custo da energia em nível
agregado domiciliar;
\item \textbf{Retorno em Tempo-Real Desagregado}: nesse caso, as informações são
exibidas desagregadas ao nível dos utensílios.
\end{enumerate}

\subsection{Resultados por Tipo de Retorno}
\label{ssec:ret_eff}

No exterior, há uma farta quantidade de estudos envolvendo o tema, com estudos iniciando na
década de 1970 em resposta à Crise do Petróleo, que tiveram um declínio durante
a década posterior. O interesse retornou a pauta recentemente devido à crescente 
preocupação com o meio ambiente e mudanças climáticas, assim 
como o aparecimento de novas possibilidades tecnológicas 
associadas a \gls{ict} \cite{aceee_2010_estudos_feedback}. É
possível encontrar pesquisas nas quais se compilam diversos estudos
de retorno de informação para consumidores residenciais no intuíto de generalizar 
resultados \cite{aceee_2010_estudos_feedback,2011_zhifeng_smart_energy_savings,
2006_darby,2009_nber_studies_us,ucla_studies_1975_2011_usa}.

Infelizmente, o mesmo não pode ser dito para o Brasil, onde não se
encontrou pesquisas nesse sentido. Assim, este trabalho irá se guiar
na pesquisa que mais se destacou com resultado de estudos no exterior
\cite{aceee_2010_estudos_feedback}. Nela se revisou 57 estudos
primários de retorno, realizados em países desenvolvidos incluindo
\gls{eua} (58\% dos estudos), quatro países da Europa Ocidental
(Países Baixos, Finlândia, Dinamarca e Reino Unido, com um total de
22\%), Canada (15\%), Japão (5\%) e Australia (1 estudo). Em termos de
retorno, metade dos estudos envolvem retorno \emph{indireto}, dentre
os quais 11 envolvem Faturamento Aprimorado, três estudos envolvem o
uso de Retorno Estimado e 15 estudos consideram Retorno
Diário/Semanal. Os remanescentes envolvem retorno \emph{direto}, dos
quais 23 exploram retorno agregado e outros seis estudos onde são
fornecidos a informação em tempo real no nível dos utensílios.
Salienta-se que pesquisas nesse sentido no Brasil são necessárias para
validar os resultados, tomando em posse as diferenças culturais,
sociais e econômicas quanto a nossa realidade.

%Realiza-se nela uma divisão quanto à data em que os estudos foram realizados,
%utilizando o termo \gls{ece} para estudos realizados entre 1974 e 1994 
%e \gls{emc} nos anos conseguintes até 2009. 
%Essa divisão é importante pois os novos estudos provêm de
%tecnologias mais recentes, em especial dispõe de novas tecnologias de \gls{ict}.
%Essas oferecem meios inovativos de aumentar o efeito de
%mecanismos de retorno de informação, assim como reduziram os custos associados
%com fornecer retorno frequente e confiável para consumidores residenciais.
%Utilizou-se dois terços dos estudos para a era mais recente, \gls{emc}. 

%Com o objetivo de explorar melhor as diferenças de resultados, 
%foram realizadas mais divisões qualitativas: 
%quanto ao tamanho do estudo, referindo-se à quantidade de domicílios 
%envolvidos, no qual considera o estudo como pequeno
%(32\%) quando utilizando menos de 100 participantes, e grande caso 
%contrário; quanto a duração do estudo, considerada curta (40\%) para períodos
%inferiores à seis meses, e longa caso contrário; e elementos motivacionais
%utilizados além de fatores economicos ou apelo ao meio ambiente, como atribuir
%metas, competições e engajamento, e normas sociais.

Um resumo dos resultados obtidos nessa referência, para os estudos realizados entre 1995 e 
2010 (cerca de dois terços dos revisados), estão na Figura~\ref{fig:potencial_consumo_retorno}. 
A redução do consumo mostrada leva em conta os
resultados globais, ou seja, considerando a taxa de adesão da população aos
programas de \gls{ee}, supondo que os mesmos terão participação voluntária. 

Identifica-se nos resultados que os tipos de retorno são tanto incrementais
em custo e complexidade, quanto nos resultados de economia energética. Assim, 
é natural a implementação do sistema de retorno ser realizada de 
maneira continua, aplicando sistemas já disponíveis enquanto se realiza 
investimento em tecnologia para o desenvolvimento do
próximo nível de informação. Os desenvolvedores devem
manter o sistema o mais flexível possível, sendo desenvolvidos 
sempre preparados para a mudança e considerando o surgimento de novos 
mecanismos de retorno com o avanço da tecnologia 
\cite{aceee_2010_estudos_feedback}. Por exemplo, por ora é possível obter
econômia de energia utilizando um sistema de baixo custo, como o Faturamento
Aprimorado, que informa melhor o consumidor, ou até mesmo, com mais ambição, 
fornecer o Retorno Diário/Semanal.

\begin{figure}[h!t]
\centering
\includegraphics[width=\textwidth]{imagens/estudo_economia_aceee.pdf}
\caption[O potencial de consumo para cada tipo de retorno]
{O potencial de economia de consumo para cada tipo de retorno. 
Adaptado de \cite{aceee_2010_estudos_feedback}.}
\label{fig:potencial_consumo_retorno}
\end{figure}


\subsection{Indo Além dos Resultados}
\label{ssec:ret_outros}

Ademais, outras considerações devem ser levadas quando no desenvolvimento 
de programas ou estudos de \gls{ee} através do retorno do consumo de energia
\cite{aceee_2010_estudos_feedback,2006_darby,2009_epri}:

\begin{itemize}
\item \textbf{Retorno Indireto versus Direto}: 
Os retornos indiretos são mais adequados que diretos 
para demonstrar o efeito de mudanças no condicionamento do ambiente, 
composição domiciliar e o impacto de investimentos 
em medidas de eficiência ou utensílios de alto consumo. Já o retorno instantâneo 
se adequa, geralmente, no fornecimento do impacto do consumo de aparelhos com 
usos de energia menores;
%\item \textbf{Era do Programa/Estudo}: Estudos anteriores a 1995 (não utilizados
%para gerar os resultados apresentados na 
%Figura~\ref{fig:potencial_consumo_retorno}) 
%apresentam economia de energia maiores aos posteriores. 
%Assim, recomenda-se a sua não utilização com o objetivo de evitar espectativas 
%infladas sobre o potencial de economia atualmente;
\item \textbf{Participação Voluntária}: Programas nos quais os usuários tem de
optar por não participar (\emph{opt-out}) tiveram adesão significamente 
maior (75\%-85\%) do que aqueles nos quais os usuários escolhem em colaborar
(\emph{opt-in}, participações menores a 10\%), sendo assim recomendada essa 
abordagem para maximizar a participação dos consumidores;
\item \textbf{Elementos Motivacionais}: A utilização de outros elementos para
motivar a população aquém do financeiro e apelo ao meio ambiente mostram-se 
importantes para aumentar a eficiência dos programas de \gls{ee}. São citados
como exemplo criar metas, compromissos, competições e normais sociais 
(tanto descritivas quanto injutivas). A Subsessão~\ref{ssec:asp_psic} 
irá tratar do tema com mais detalhes;
\item \textbf{Contexto Regional}: Diferenças regionais e culturais afetam os 
resultados. Os resultados para a Europa Ocidental superam os obtidos nos
\gls{eua}, podendo possivelmente ser atribuidos as diferenças em como o 
discurso sobre as mudanças climáticas pelas lideranças políticas nas duas regiões 
é feito e assim a preocupação ao tema da população. Nesse caso, chama-se atenção
novamente aos antecedentes no intuíto de preparar a população para os programas e
maximizar os resultados. Outro aspecto importante é a necessidade de estudos
sobre o tema afim de especificar como o brasileiro irá reagir em tais programas;
\item \textbf{Duração do Estudo e Persitência dos Resultados}: Quando os 
estudos são de menor duração ($< 6$ meses) se obtém resultados mais 
eficientes (média de 10,1\% de economia) que estudos mais longos (7,7\%),
discrepância essa atribuida a inaptidão de estudos curtos em observar variações
sazionais na utilização de energia. Alguns estudos indicam que se faz necessário 
a presença do retorno em longo termo para que os resultados persistam, 
enquanto outros apontam a necessidade do retorno continuamente, enfatizando
assim a necessidade na extensão dos programas de \gls{ee};
\item \textbf{Tamanho do Estudo}: Estudos com grandes ($> 100$) amostragens
domiciliares tendem a ter resultados mais modestos. Como esses estudos tem uma
representatividade melhor das residenciais, isso indica que programas de
\gls{ee} em larga escala também devem apresentar resultados mais modestos que
aqueles apresentados na Subsessão~\ref{ssec:ret_eff}.
Ainda, esses estudos mostram-se menos suscetíveis às oscilações 
quanto a duração dos estudos;
\item \textbf{Resposta de Ponta e Demanda versus Economia Fora de Ponta}:
Reduções de pico e demanda são de particular interesse das concessionárias que
buscam atender essencialmente o mesmo nível de serviços mas com custos totais
menores. Há dois modos de obter tal efeito: com uma melhoria em \gls{ee} ou
através do deslocamento de parte do consumo no horário de ponta para fora da
ponta. O interesse em resposta de demanda, ou seja, em reduzir o
consumo durante os horários de ponta difere dos programas de \gls{ee} que focam
em ter reduções eficientes economicamente durante todo o ano. Ainda que não seja
deprezível, programas de resposta de demanda apresentam economia de energia
bastante baixos quando em comparação aos de \gls{ee}. Os consumidores
normalmente não percebem a diferença entre os dois programas, do mesmo modo 
que a integração dos programas é plausível e sinergética, 
onde estudos mostram que a junção causa melhores resultados tanto 
em econômia de energia quanto na 
redução de picos, por isso, sendo interessantes tanto do lado do consumidor 
quanto para a consessionária. Desta forma, a abordagem ótima ao tema deve
ser conseguir todos os meios economicamente atraentes de reduzir o desperdício 
e ineficiências antes de procurar oportunidades restantes de reduzir cargas 
durante os picos;
\item \textbf{Hábitos, Escolhas e Estilos de Vida}: Dentre os tipos de
comportamentos de \gls{ee} e conservação, os que aparecem mais frequentemente são 
investimentos em novos equipamentos e utensílios em populações mais ricas, sendo
geralmente empreendido em conjunto com mudança de residência ou melhoria no
estilo (referido em oposição a funcional) do domícilio. Para a maioria da
população, os domícilios obtém melhor \gls{ee} através da mudança de hábitos e 
rotina, ou pela avaliação dos comportamentos relacionados a energia. Esses
comportamentos de \gls{ee} são motivados assim por uma variedade de fatores,
incluindo interesse próprio (financeiro) e outros motivos altruístas e
preocupações cívicas. Desta forma, programas de \gls{ee} que procuram apenas 
a instalação de equipamentos mais novos e eficientes irão 
desperdiçar o potencial relacionado à mudança comportamental, assim como
programas que apelam apenas para o interesse financeiro não irão influenciar um
largo grupo de fatores que motivam as pessoas para agir;
\item \textbf{Segmentação Populacional}: Poucas pesquisas exploraram como o
potencial de redução de consumo é afetado pelas diferentes classes sociais.
Desses estudos, as descobertas sugerem grandes níves de economia tendem estar
associados a alto nível educacional e renda, grandes residenciais e dentre elas
as com maior número de pessoas, consumidores jovens e/ou com grande tendência a
valores ambientais. Essas considerações indicam que os potenciais apontados
provavelmente estão inflados para a aplicação no Brasil como um todo.
\end{itemize} 

\subsection{Tecnologias e Tendências}
\label{ssec:ret_tec}

Como constatado, os medidores atualmente utilizados para medição pelas 
concessionárias, os medidores analógicos e eletrônicos, permitem fornecer um 
retorno com baixo custo mas com um potencial melhor de economia de energia, 
o Faturamento Aprimorado. As contas de energia de companias como a Light,
Ampla, Cemig e Eletropaulo fornecem o histórico de consumo dos últimos 12 meses,
uma informação que pode auxiliar o consumidor, já podendo ser consideradas um 
Faturamento Aprimorado. Entretanto outras informações podem ser utilizadas, 
como referencias do consumo acumulado e, em especial, comparações do consumo 
com o de vizinhos ou grupo pertencente. Também é possível estimar o uso 
energético por uso-final utilizando os valores médios de consumo das residenciais
para cada uso no sentido de auxiliar o cliente. A ideia é
transformar a conta de energia em uma espécie de relatório do consumo energético,
com um visual mais atraente (ver Subsessão~\ref{ssec:asp_visuais}), 
contendo gráficos e informações no sentido de atrair o consumidor a se 
preocupar com o tema, sendo esse o primeiro passo \cite{2009_epri}.

No entanto, o sistema elétrico atual está se tornando obsoleto para atender aos problemas de 
aumento de carga nos centros urbanos devido ao crescimento do setor 
de serviços e do consumo das residencias. Há uma presença cada vez maior
de cargas eletrônicas injetando harmônicos e a geração centralizada exige 
excessivamente da capacidade de transmissão e distribuição, 
sobrecarregando as linhas nesses grandes centros que nem sempre podem 
corresponder à necessidade de novas linhas. A falta de informações sobre o 
estado do sistema dificultam a operação e planejamento de uma rede cada vez mais 
sobrecarregada. As \gls{ict} revolucionaram as redes de telecomunicações e 
serão a tendência para a criação das redes elétricas inteligentes (\emph{smart 
grids}), o novo sistema elétrico que tem como objetivo responder à essas
dificuldades. Ainda não foram definidas todas as características desse sistema,
por outro lado as principais características são o uso de comunicações em tempo real
para o controle e informação, o uso massivo de sensores e medidores para
monitoramento do sistema, faturamento com preços para o momento de uso, 
gestão pelo lado da demanda, a integração de 
componentes avançados como linhas de transmissão supercondutoras, armazenamento, 
eletrônica de potência, geração distribuida etc. 
\cite{dissert_caires,aceee_2010_estudos_feedback}

Os aparelhos eletrônicos de medição utilizados nas redes inteligentes, referidos
neste trabalho como medidores inteligentes (\emph{advanced/smart metering}), irão 
fornecer uma gama maior de informações em tempo-real para as concessionárias,
melhorando a operação e planejamento. Ao mesmo tempo, será possível a
concessionária se comunicar com o cliente, oferecendo incentivos (como
descontos) para reduções de carga durante os horários de ponta, 
outros planos de tarifação com preços dinâmicos de acordo com os horários,
aumentando a interação da concessionária com o cliente. 

Por outro lado, pelo ponto de vista da demanda (ou dos consumidores) essa 
informação também estará disponível, trazendo uma gama de novas oportunidades 
para os usuários participarem ativamente. Mais especificamente, na abordagem do
tema atual, os medidores inteligentes oferecem uma base
a ser explorada para fornecer o retorno em larga escala para os consumidores, 
tanto o direto quanto indireto. Nos medidores utilizados nos \gls{eua} foram 
apontadas algumas dificuldades
técnicas para esse fornecimento, sendo elas: a necessidade de grande quantidade
de energia para enviar um sinal frequente ao consumidor e o sinal ser enviado em
intervalos de 7 s. Com um custo adicional, um estudo na industria mostrou que
é possível dos medidores terem seu \emph{hardware} substituídos 
no futuro para que possam fornecer medições de pequena energia e \emph{chips} 
de comunicação para habilitar dados de utensílios específicos, assim como 
automação para grandes cargas, como unidades de condicionamento ambiental, 
bombas etc. Desta forma, sendo possível 
fornecer tanto a tecnologia para o Retorno em Tempo Real com a utilização de 
mostradores dentro do domicílio, quanto o Retorno em Tempo Real Desagregado
\cite{aceee_2010_estudos_feedback}.

Algumas empresas se estabeleceram no novo mercado para informar o consumidor
sobre o seu uso de energia e em auxiliá-lo nas atitudes para reduzir o seu
consumo, antes mesmo que estivessem disponíveis os medidores inteligentes. 
Elas fornecem o retorno indireto em alguns países
desenvolvidos, dentre eles o \gls{eua}, Australia, Nova Zelândia, Reino Unido.
Dentre essas empresas, faz-se referência a \emph{Positive Energy} 
\cite{opower_site} e \emph{C3 Energy} \cite{c3_site} que disponibilizam seus 
serviços, organizados na Tabela~\ref{tab:servicos_ret_ind}, 
utilizando os dados da concessionária, independente
quando presentes na residência os medidores convencionais ou inteligêntes. 
É importante notar que as abordagens utilizadas por essas empresas utilizarão 
análises mais complexas conforme a presença de dados mais detalhados, 
frequentes e desagregados estejam disponíveis.

\begin{table}[h!t]
\resizebox{\textwidth}{!}{
\begin{tabular}{m{2.5cm}m{5cm}m{8cm}}
\hline \hline 
\centering{\textbf{Empresa}} & \textbf{Tecnologia de Retorno} & 
\textbf{Principios Comportamentais} \\
\hline \hline
\centerline{\textbf{Positive}}\centerline{\textbf{Energy}}\centerline{\cite{opower_site}} & 
Dependendo da concessionária, envia correspondencias mensais ou
trimestrais e/ou fornecem um portal na internet com novas redes sociais &
\emph{Tipo de Retorno}: Retorno indireto incluindo informação sobre o domicílio
e conselhos, auditorias de energia através do uso da \emph{web}, análise de 
faturamento, consumo estimado por aparelho, \gls{co2}, k\acrshort{wh} e \$.

\emph{Principios Comportamentais}: Comparações sociais, metas, comparações
pessoais e plano de ações. \\
\hline
\centerline{\textbf{C3}}\centerline{\textbf{Energy}}\centerline{\cite{c3_site}} & 
Portal de comunidade social com retorno de consumo de energia e água & 
\emph{Tipo de Retorno}: Retorno indireto incluindo informação sobre o domicílio
e conselhos, auditorias de energia através do uso da \emph{web}, análise de 
faturamento, consumo estimado por aparelho, \gls{co2}, k\acrshort{wh}, \$ e
outras unidades.

\emph{Principios Comportamentais}: Comparações sociais, metas, competições
redes sociais, comparações pessoais e plano de ações. \\
\hline \hline
\end{tabular}
}
\caption[Empresas utilizando informação da concessionária e as 
oportunidades e insentivo de economia de energia oferecidas.]
{Empresas utilizando informação da concessionária e as 
oportunidades e insentivo de economia de energia oferecidas. Extraído e
atualizado de \cite[tradução própria]{aceee_2010_estudos_feedback}.}
\label{tab:servicos_ret_ind}
\end{table}

Já o retorno direto pode ser encontrado através de
mostradores de energia no domicílio. A Tabela~\ref{tab:servicos_ret_dir}
identifica alguns dos mostradores oferecidos atualmente e suas propriedades.
Muitas vezes as companias oferecem também análises e estimativas do consumo 
especifico de aparelhos, comparações sociais e outros principios para motivar os
consumidores a economizar energia. A informação de consumo de aparelhos
especifico ou é estimada, ou realizada através de sensores nos aparelhos.

\begin{table}[h!t]
\resizebox{\textwidth}{!}{
\begin{tabular}{p{4cm}p{7cm}p{7cm}}
\hline \hline 
&
\multicolumn{1}{c}{\textbf{The Energy Detector TED} \cite{ted_site} }& 
\multicolumn{1}{c}{\textbf{Wattson}                 \cite{wattson_site}}\\
\hline \hline
\textbf{Descrição da \newline Tecnologia} & 
\emph{Software} de suporte, aplicativos para celular &
\emph{Software} de suporte com acesso a comunidades \\
\hline 
\textbf{Mecanismos de \newline Retorno} & 
Mostradores em tempo real de k\acrshort{watt}, \$/hr, \gls{co2}, consumo e gastos
diários, conta estimada em k\acrshort{wh} e \$, pico de consumo, voltagem
min/max e custo/demanda projetada &
Mostradores em tempo real aproximado do consumo em \acrshort{watt},
k\acrshort{watt}, conta estimada. Leituras entre 3 a 20 s. Brilha conforma o
consumo: azul para consumo baixo; roxo para médio; vermelho para alto. \\
\hline
{\multirow{5}{4cm}{\textbf{Principios Comportamentais}}} &
\multicolumn{2}{c}{\emph{Retorno de Informação:}}
\\
& & \\
& 
\multicolumn{2}{p{14cm}}{
Retorno direto incluindo conselhos, auditorias de energia baseadas na \emph{web},
análise do consumo, estimativa de consumo por utensílios, \gls{co2} e \$.
}
\\
& & \\
&
\multicolumn{2}{p{14cm}}{\emph{Motivações Oferecidas:}
\centering 
Comparações sociais, metas, comparações pessoais e etapas de ações.
}
\\
\hline \hline
& 
\multicolumn{1}{c}{\textbf{PowerCost Monitor} \cite{powercost_site}}& 
\multicolumn{1}{c}{\textbf{Efergy Elite}      \cite{efergy_site}}\\
\hline \hline
\textbf{Descrição da\newline Tecnologia} & 
\emph{Software} de suporte, aplicativos para celular &
\emph{Software} de suporte, aplicativos para celular \\
\hline
\textbf{Mecanismos de\newline Retorno} & 
Mostradores em tempo real aproximado do consumo em
k\acrshort{watt} e \$/hr, pico de consumo nas últimas 24 horas, contagem de
k/\acrshort{wh} (reiniciável), recurso para medição de aparelhos específicos. &
Mostradores em tempo real aproximado do consumo em k\acrshort{watt} e \$/hora
(leituras em 6, 12 ou 18 s), informação de consumo média por hora, semanal,
mensal. Alarmes para consumo alto. \\
\hline
{\multirow{5}{4cm}{\textbf{Principios Comportamentais}}} &
\multicolumn{2}{c}{\emph{Retorno de Informação:}} \\
& & \\
& 
Retorno direto incluindo conselhos, auditorias de energia baseadas na \emph{web},
análise do consumo, estimativa de consumo por utensílios, \gls{co2} e \$.  &
Retorno direto, análise de consumo, estimativa de consumo em \$.  \\
& & \\
&
\multicolumn{2}{p{14cm}}{\emph{Motivações Oferecidas:} 
\centering Metas e comparações pessoais}
\\
\hline \hline 
\end{tabular}
}
\caption[Especificações de mostradores domiciliares disponíveis.]{
Especificações de mostradores domiciliares disponíveis. Tradução própria de
\cite{aceee_2010_estudos_feedback}.}
\label{tab:servicos_ret_dir}
\end{table}

% TODO Atualizar a tabela

Uma outra maneira mais eficiente economicamente para fornecer o
Retorno em Tempo Real Desagregado é através do uso de um \gls{nialm}
(Capítulo~\ref{cap:nialm}). Essa técnica coloca o peso da desagregação 
da informação no \emph{software}, reduzindo a necessidade
de investimento em sensores e \emph{hardware}, sendo assim um método
economicamente atraente para a implementação de programas de \gls{ee} fornecendo 
esse tipo de retorno. No Brasil, os esforços da \gls{aneel} em regulamentar as
bases para os novos medidores inteligentes dão poder ao consumidor de exigir
à concessionária acesso às medições de tensão e corrente de cada fase, como rege
no art.~3$^o$ da Resolução Normativa n$^o$~502 \cite{ren502}. Ainda não se especificou 
a taxa de amostragem na qual essas leituras serão disponiblizadas, no entanto,
pode-se aproveitar toda a infra-estrutura das medidas e comunição oferecida pelos
medidores inteligentes no intuíto de maximizar o custo-benefício.  
Caso a amostragem seja baixa, ou de interesse aumentá-la para obter uma maior 
capacidade de identificação dos aparelhos, o \gls{nialm} pode utilizar de um 
\emph{hardware} próprio de medição.

Percebe-se que ainda é incerto se os medidores inteligentes são a melhor 
alternativa de fornecer retorno de informação, contudo parece natural sua 
utilização. Diversas tecnologias podem ser utilizadas, envolvendo, ou não, as 
concessionárias. Apenas com o desenvolvimento dessas tecnologias será
possível determinar as limitações, custos e vantagens para definir o que é 
economicamente mais atraente.

Finalmente, uma outra tendência é o uso de automação da rede doméstica. A
automação, além de melhorar a qualidade de vida dos consumidores, pode aumentar
o potencial de redução de consumo. Com uma maior capacidade de administração de
sua demanda sem grande esforço, facilita-se aos consumidores de realizarem a
mudança de hábitos no sentido de um comportamento sustentável, simplificando
uma condução do sistema de modo mais econômico.

\subsection{Aspectos Psicológicos}
\label{ssec:asp_psic}

A tecnologia apenas concebe as possibilidades de informação a
serem repassadas ao consumidor, no entanto, a questão ainda está em como
apresentar essa informação e motivar o usuário para a mudança. 
O conselho de profissionais nos campos de psicologia, sociologia,
\emph{marketing}, mudança e economia comportamental serão criticos para
motivar, habilitar e continuamente empreender consumidores na gestão de sistemas 
de energia residenciais \cite{aceee_2010_estudos_feedback}. 
Um exemplo é a empresa \emph{Positive Energy}, que utiliza psicólogos para
auxiliar no desenvolvimento de suas tecnologias, disponibilizando ferramentas
sociais e estratégias persuativas para um engajamento maior de seus clientes.

Um levantamento das noções básicas da psicologia motivacional foi realizada em
\cite{2010_aspectos_psicologicos_usa}, assim como uma estrutura para os desenvolvedores 
da tecnologia aplicada nos programas de retorno de informação no sentido de motivar a 
mudança comportamental para uma melhor \gls{ee} e um mundo sustentável.
Será realizado um resumo desses tópicos a seguir guiado nessa referência, mas
vale enfatizar que os profissionais nessas áreas devem analisar o tema e
escolher a melhor abordagem a ser utilizada nas tecnologias desenvolvidas.

O objetivo é motivar o consumidor para a mudança através 
de recomendações levando em conta o processo de mudança comportamental 
do consumidor. Define-se a motivação como 
\cite[p.927-928, tradução própria]{2010_aspectos_psicologicos_usa}:

\begin{quote}
Motivação é um questionamento ao porquê do comportamento. Ela é um estado
interno ou condição (as vezes descrita como uma necessidade, desejo ou querer)
que serve para ativar ou energizar o comportamento. Motivação está fortemente
ligada a processos emocionais. Emoções podem estar envolvidas na iniciação
comportamental (como a emoção de solidão pode motivar a ação de procurar
compania). Em alternativa, o desejo para viver uma emoção em particular pode
também motivar para a ação (como a decisão de correr uma maratona pode ser
motivada pelo desejo de experimentar a sensação de realização de um feito).
\end{quote}

Ela é influênciada por ideáis psicológicos que foram aprendidos pelos
indivíduos. Nota-se que diferentes indivíduos tem ideáis psicológicos distintos,
estes estando apresentados em ordem decrescente quanto a possibilidade de 
sofrerem alteração:

\begin{itemize}
\item \emph{Atitudes} são pré-disposições aprendidas quanto a respostas
para uma pessoa, objeto ou ideia em um modo favorável ou desfavorável. Por
exemplo, o ato de tomar banho curtos devido a uma atitude favorável em respeito
ao meio ambiente;
\item \emph{Crenças} são os meios nos quais as pessoas estruturam seu
entendimento da realidade, reflitindo a ideia do que é certo e o que é errado. 
A maioria das crenças são baseadas em experiências passadas, como a reciclagem 
ajudar o meio ambiente.
\item \emph{Valores} são os fundamentos para o conceito de um indivíduo sobre si
mesmo. Podem ser conceituadas como ideáis comportamentais ou preferências por
vivências. No caso dos primeiros, valores funcionam como conceitos duradouros de
bem e mal, certo e errado, enquanto para preferências por vivências os valores
guiam individuos para vivenciarem situações nas quais as proporcionam certos
tipos de emoções. A Tabela~\ref{tab:valores} contém um subgrupo de valores
definidos por Rokeach e Maslow, onde se propõe que pessoas possuem
uma estrutura hierárquica ou prioritária de valores individuais. Rokeach
acredita que as diferenças no comportamento ocorrem devido a diferenças na
classificação de importância de valores, enquanto Maslow fornece uma ordem de
valores em níveis que serão priorizados pelos indivíduos pelos níveis mais
baixos antes dos níveis superiores.
\end{itemize}

\begin{table}[h!t]
\resizebox{\textwidth}{!}{
\begin{tabular}{ccc}
\hline 
\multicolumn{1}{|p{6cm}|}{\centering \textbf{Ideáis Comportamentais} (Rokeach)} & 
\multicolumn{1}{p{6cm}|}{\centering \textbf{Preferências por Experiências} (Rokeach)} &
\multicolumn{1}{p{6cm}|}{\centering \textbf{Preferências por Experiências} 
(Maslow, níveis em ordem crescente)} \\
\hline \hline 

\multicolumn{1}{|m{6cm}|}{
\textbf{Capaz}\newline Competente, eficiente. \newline 
\textbf{Prestativo}\newline Trabalhando para o bem-estar dos outros. \newline 
\textbf{Honestidade}\newline Sinceridade e confiável. \newline
\textbf{Imaginativo}\newline Ousado e criativo. \newline
\textbf{Independente}\newline Auto-confiante, auto-suficiente. \newline
\textbf{Intelectual}\newline Inteligente e pensativo. \newline
\textbf{Lógico}\newline Consistente e racional. \newline
\textbf{Obediente}\newline Atencioso e respeitoso. \newline
\textbf{Responsável}\newline Fidedigno e confiável.
} &
\multicolumn{1}{m{6cm}|}{
\textbf{Vida Confortável}\newline Uma vida próspera. \newline 
\textbf{Liberdade}\newline Independência e liberdade de escolha. \newline
\textbf{Saúde}\newline Bem-estar psicológico e físico. \newline
\textbf{Harmônia Interna}\newline Ausência de conflitos interiores. \newline
\textbf{Sentimento de Realização}\newline Uma contribuição duradoura. \newline
\textbf{Reconhecimento Social}\newline Respeito e admiração. \newline
\textbf{Sabedoria}\newline Um entendimento maduro da vida. \newline
\textbf{Um Mundo Belo}\newline Beleza da natureza e das artes.
} &
\multicolumn{1}{m{6cm}|}{
\textbf{Psicológico}\newline Homeostáse e apetites. \newline
\textbf{Segurança}\newline Segurança do corpo, emprego, recursos, familia, saúde,
propriedade. \newline
\textbf{Amor/Aceitação}\newline Afeição e ser aceito. \newline
\textbf{Estima}\newline Respeito próprio, auto-estima, estima dos outros \newline
\textbf{Realização Pessoal}\newline Encontrar satisfação pessoal e compreender seu
potencial.
} \\
\hline 
\end{tabular}
}
\caption[Valores propostos por Rokeach e Maslow.]
{Valores propostos por Rokeach e Maslow. Tradução própria de 
\cite[p.928]{2010_aspectos_psicologicos_usa}.}
\label{tab:valores}
\end{table}

Outra questão importante é a persistência ou durabilidade do
comportamento, ou seja, da capacidade do comportamento manter-se, sem
a necessidade de intervenções.  Para atingir esse objetivo é
aconselhável motivação intrínseca, que é a realização de uma atividade
pelas satisfações que a tarefa oferece, enquanto o seu oposto, a
motivação extrínseca, está ligada à realização para obter uma
consequência separável. Exemplos para o primeiro são: curiosidade,
competência e satisfação; no outro caso: incentivos materiais e
reconhecimento social.

Utilizou nessa referência o Modelo Transteórico\footnote{A referência
levanta outros modelos comumente utilizados e seus prós e contras.
Também há uma preocupação quanto a modelar a mudança através de
estados discretos, que podem não representar bem a realidade do
processo. A utilização do modelo é justificada por seu valor
heurístico, usado como um modelo simplificado de uma mudança ideal.},
onde se considera que o processo da mudança ocorre em uma série de
estados, sendo a motivação a força motriz para se deslocar entre os
estágios. O objetivo em cada etapa, assim como recomendações para o
atingir estão a seguir:

\begin{itemize}
\item Estrutura Exemplo --- \emph{Etapa}: explicação.
\begin{enumerate}
\item Objetivo nessa etapa.
\begin{enumerate}
\item Recomendação para atingir o objetivo.
\end{enumerate}
\end{enumerate}
\item \emph{Pré-contemplação}: o indivíduo está desencorajado,
relutante em manter, mal-informado ou desconhece o problema
comportamental. Não há previsão de ação no futuro, esse medido
normalmente como os próximos 6 meses.
\begin{enumerate}
\item Apresentar informação em moderação com o próposito de plantar a
semente no sentido dos indivíduos tomarem conhecimento de seus
comportamentos (de consumo de energia) atuais problemáticos. A
moderação é importante pois uma maior intensidade irá, geralmente,
produzir menores resultados nessa etapa.
\begin{enumerate}
\item Fornecer retorno de informação personalizado notando tanto os
prós e contras de um comportamento \emph{não-sustentável}. Apresentar
os beneficios e consequências em relação aos valores pessoais, de modo
não tendêncioso.  
\item Utilizar normas sociais a respeito de comportamentos de consumo
sustentáveis combinando o uso de normas injuntivas e descritivas. As
normas sociais são as regras ou espectativas por um comportamento
apropriedado em uma situação social em particular. Elas ``podem levar
pessoas a dizer coisas que sabem que não são verdade, utilizar drogas
ilicitas ou deixar de reagir a uma ameaça eminente '' \cite[p.51,
tradução própria]{aceee_2010_estudos_feedback}.  Normas descritivas
são percepções de comportamentos normalmente realizados (ex. 85\% da
sua vizinhança reciclam), apelando ao valor de Maslow de
\emph{Amor/Aceitação}. Normas injutivas são percepções e
comportamentos que são normalmente aceitos ou aprovados (ex. um sinal
de polegar positivo com o texto ``Gere menos resíduos''). Essas apelam
para o valor de Rokeach \emph{Obediente}.  Ao juntar ambas normas
descritivas e injutivas, há uma chance ainda maior de sucesso quando
em comparação da aplicação delas isoladamente.
\item Fornecer uma variedade de pequenas ações de consumo que, se realizadas, 
podem ter impactos positivos no meio ambiente. Isso irá trabalhar em duas
barreiras para a motivação, que são: não se sentir competente e não acreditar
que suas ações irão levar a um resultado positivo. Apresentar uma variedade de
ações apela ao valor de Rokeach de \emph{Liberdade} e aumenta o senso de
controle pessoal assim como a motivação intrínseca.
\end{enumerate}
\end{enumerate}
\item \emph{Contemplação}: há conhecimento de seu problema comportamental e
planeja-se uma mudança no futuro. Contempladores são abertos a informação sobre
o problema, ainda que estão distantes do compromisso de mudança devido ao
sentimento de ambivalência;
\begin{enumerate}
\item Pender a balança no sentido de mudança. Ambivalência\footnote{Presença de
sentimentos/pensamentos conflitantes perante uma coisa ou pessoa.} é o problema chave
que precisa ser resolvido, uma vez que a avaliação dos prós e contras tem mais
ou menos o mesmo peso.
\begin{enumerate}
\item Nessa etapa, fornecer enfatizar no retorno de informação os prós de um
comportamento sustentável e os contras do comportamento não-sustentável. É
importante auxiliar o consumidor em perceber os prós, uma vez que os mesmos irão
resistir às mudanças enquanto perceberem isso como um fator redutor de sua
qualidade de vida, em especial àquelas que salientam o sacrificio pessoal pelo
bem comum. Assim como os contras devem enfatizar nos custos de comportamentos
não-sustentáveis, numa perspectiva de perda em detrimento de ganho. O foco nos
valores pessoais podem ser extremamente persuasivos nesse estágio.
\item Lembrar o individuo de sua atitude em favor do meio-ambiente, informá-lo
da discrepância de suas atitudes e o comportamento correspondente, encorajar a
mudança. Essa técnica apela para a dissonância cognitiva\footnote{Um estado
desconfortável que ocorre quando a pessoa possui uma atitude e um comportamento
que são psicologicamente inconsistentes.}. Como as pessoas mudam de atitudes mais
fácil que de comportamento, é importante encorajar o comportamento sustentável.
\item Fornecer incentivo para pequenas mudanças de consumo (independente da
intenção do consumidor original era a utilização sustentável de energia) para
fomentar maiores mudanças no futuro.
\item Vincular a tecnologia de retorno a um portal de uma comunidade virtual e
incentivar o indivíduo para procurar e ler a informação de experiências de
outros usuários com consumos sustentáveis na comunidade. Isso apela para normas
sociais de um modo vivo e personalizado, explorando a abertura dos
contempladores ao tema, mas, ao mesmo tempo, sem forçar nenhum tipo de ação.
\end{enumerate}
\end{enumerate}
\item \emph{Preparação}: momento em que o indivíduo está pronto para ação no
futuro iminente (medido normalmente como 1 mes), e tem como objetivo desenvolver
e engajar-se a um plano. Pelo menos uma tentativa de mudança foi realizada no
último ano;
\begin{enumerate}
\item Ajudar os usuários a desenvolver um plano que seja aceitável, acessível e
efetivo. Esses planos podem se relacionar a ações extraordinárias (compra de um
refrigerador eficiente) ou diárias (tomar banhos mais curtos). Um objetivo é
definido como uma representação interna de um resultado desejado. Usuários na
fase de preparação podem ter objetivos abstratos mas não saber necessariamente
como os alcançar.
\begin{enumerate}
\item Ajudar na criação de metas pessoais especificas e quantitativas
(o nível de dificuldade deve se elevar conforme o sucesso, começando com metas 
simples e partindo para mais complicadas). Metas dificeis, pessoais e
especificas tem maior engajamento quando comparadas a tarefas ``faça o seu
melhor'', fáceis e que lhe foram atribuídas.
\item Desenvolver modos múltiplos para os consumidores atingirem suas metas e
incentivá-los para utilizar sua habilidade e experiência pessoal nesses planos.
\item Fornecer no portal da comunidade virtual aos usuários a opção de ter
um ``Conselheiro/Tutor'', esses composto por pessoas exemplares na fase de ação ou
manutenção. Essa conexão fornece um maior nível de engajamento. 
\end{enumerate}
\end{enumerate}
\item \emph{Ação}: ocorre a manifestação de modo evidente da mudança
comportamental, usualmente dentro dos últimos 6 meses;
\begin{enumerate}
\item Reforçar positivamente ações sustentáveis de energia. O reforço positivo é
a técnica mais efetiva para motivar a maior ocorrência de um comportamento
desejado, ela tende a aumentar a motivação intrínseca. Técnicas como a punição ou 
reforço negativo evitam um comportamento não desejado, mas não o substitui por
nada, além de reduzir a motivação intrínseca.
\begin{enumerate}
\item Fornecer o reforço positivo imediatamente após o comportamento desejado
ocorrer e em multiplos modos com o intuíto de aumentar sua eficácia.
\end{enumerate}
\item Desenvolver motivações intrínsecas para o comportamento sustentável.
\begin{enumerate}
\item Permitir uma exploração interativa, personalizável e anotações na
interface oferecida.
\end{enumerate}
\end{enumerate}
\item \emph{Manutenção, Recaída e Reciclagem}: trabalho no sentido de manter o
comportamento alterado e a luta para previnir recaídas. Se esse ocorrer, o
indivíduo regressa a um dos estágios anteriores e o processo recomeça.
\begin{enumerate}
\item Manter o comportamento sustentável permanente. Em algum momento as
mudanças irão se tornar sustentáveis por si próprias, sendo possível a saida 
do indivíduo do ciclo de mudança. Enquanto isso, o objetivo deve ser fazer o
indivíduo apenas um pouco mais consciente e informado.
\begin{enumerate}
\item Apoiar ações sustentáveis para que elas virem hábitos, relembrando os
usuários para realizar ações específicas.
\item Fornecer a opção para usuários nessa etapa de se tornarem
``Conselheiros/Tutores'' para indivíduos na etapa de preparação. Essa tecnica
aplica dissonância cognitiva, uma vez que indivíduos que tentaram persuadir
alguém irão racionalizar internamente o seu comportamento, e assim estarão
propensos a intensificar seu engajamento. Esse método apela para o valor de
Rokeach \emph{Reconhecimento Pessoal} e \emph{Sabedoria}, e, em retorno, pode
gerar satisfações intrínsecas de competência e satisfação.
\item Encorajar os usuários para reforço e reflexões pessoais em suas
experiências através de um diário. A reflexão de suas atitudes em relação a
energia e percepção de seu progresso podem trazer a tona satisfações intrínsecas
de interesse, competência e satisfação. O reforço pessoal (na forma de orgulho
ou senso de realização) também irá trazer satisfações intrínsecas, no caso de
competência, e ainda levar a percepções de auto-eficácia. Isso é
importante pois, para um indivíduo vivenciar sucesso de longo-termo, eles
precisam de auto-eficácia e atribuições intrínsecas do comportamento.
\item Manter um ciclo de motivação intrínseca de interesse, curiosidade, desafio
ótimo, competência e satisfação. A motivação intrínseca é um ciclo de dois
passos. Primeiro, estimulos como novidade, complexidade e mudança atraem a
atenção, curiosidade e interesse, o que convida para a investigação, exploração
e manipulação dos estímulos. Segundo, desempenhos de competência em tarefas são
desfrutados, enquanto o crecimento da satisfação aumenta tanto o desejo na
atividade quanto a capacidade de confrontar desafios parecidos no futuro. 
\end{enumerate}
\end{enumerate}
\end{itemize}





\subsection{Aspectos Visuais da Informação}
\label{ssec:asp_visuais}

A apresentação na qual a informação é realizada também irá afetar o entendimento
e engajamento do consumidor nos programas de \gls{ee}. Novamente, especialistas
e profissionais na área de \emph{design} da informação serão importantes nesse
intuíto, percebendo que o problema jaz além de apenas disponibilizar a
tecnologia para fornecer a informação: também será necessário tornar a
informação atrativa ao consumidor.

O processo do fluxo de informação ocorre entre o remetente e o receptor, onde
este analisa a informação necessaria e a apresenta na mensagem através de textos
e imagens, enquanto aquele realiza escolhas entre as informações disponíves nela
e opta se irá processá-las mentalmente. A visão é o sentimento mais 
importante para a compreensão e vivência humana do meio externo. 
Cerca de 70\% de nossas células sensoriais estão nos olhos, assim a visualizão 
é um meio bastante efetivo de realizar a comunicação da informação e de dados. 
Tornar a mensagem interessante visualmente depende de vários elementos 
\cite{2012_visualisation_sweden}. Alguns exemplos são: fornecer para a 
mensagem uma estrutura na qual serão guiados os princípios para
a inserção das representações; clareza para a simplicidade de compreensão e
legibilidade da informação; enfase para atrair, direcionar ou manter a atenção;
unidade da mensagem, com uma coerência e união global \cite{it_depends}.

Em \cite{2012_visualisation_sweden}, apresentou-se a jovens de um colégio sueco
uma visualização do consumo de energia em um mostrador em tempo real na sala de 
aula. Os jovens tiveram participação na criação da interface\footnote{A
participação dos usuários no desenvolvimento facilita à tecnologia atender às
especificações e necessidades dos mesmos. No caso, a participação dos usuários
utilizada está de acordo com a recomendação de
\cite{2009_extreme_user_filandia}, na qual há a descrição
de uma abordagem orientada ao usuário para o retorno do consumo de energia. Nela, o usuário apenas fornece
inspiração aos desenvolvedores profissionais, que desenvolvem as soluções.}, 
onde desenharam e escolheram uma imagem para informar se o consumo estava elevado, 
médio ou baixo, assim como informaram o que entederam das informações nos mostradores.
Esse estudo revelou cinco aspectos importantes no desenvolvimento de sua
visualização nos mostradores:

\begin{enumerate}
\item Deve chamar a atenção dos usuários, realizando o uso de cores brilhantes,
contrastes e quando possível uma exibição dinâmica; 
\item Mostrar comparações entre o consumo de modo a deixar evidente aos usuários
resultados positivos de um esforço;
\item Fornecer o consumo em tempo real para estimular a mudança direta no
comportamento;
\item Deve conter um tom positivo e encorajador, potencializando a positividade
de um comportamento correto;
\item Ser explicativa, com pequenos textos instrutivos que fazem aos usuários
simples de entender o que eles estão vendo.
\end{enumerate}


%O \gls{nialm} pode ser utilizado diretamente na obtenção de 
%uma melhor \gls{ee}, visto que estudos no exterior
%\cite{2010_advanced_metering,liikkanen2009extreme,schleich2012does,darby2006effectiveness}
%indicam uma possível econômia de até 12\% no consumo residencial através do fornecimento de um
%retorno detalhado em tempo real ao usuário de sua utilização da eletricidade, 
%com informação do consumo por equipamento e recomendações no intuíto de reduzir
%o consumo de energia. Esses estudos mostram uma grande variação conforme a
%cultura e perfil social do consumidor, sendo necessário um estudo para
%determinar esse potencial no Brasil, entretanto, percebe-se que o maior
%potencial está nos consumidores pertencentes a classe média, que são os mais aptos a reduzirem seu
%consumo através do retorno de informação em detrimento aos mais pobres, que 
%raramente possuem mudanças que possam ser exploradas uma vez que seu consumo
%é vital e menor, e os mais ricos, que tem pouca sensibilidade aos gastos com
%energia elétrica.



  \chapter[Monitoramento Não-Invasivo de Cargas Elétricas]{\acrfull{nilm}}
\label{cap:nilm}

Este capítulo trata em detalhe as particularidades envolvidas no
desenvolvimento da tecnologia conhecida como
\gls{nilm}\footnote{NIALM e NALM são outras abreviaturas utilizadas na
literatura.}, em especial quanto aos aspectos técnicos envolvidos. A
Sessão~\ref{sec:nilm_aspec_gerais} é a única que não leva em conta os
aspectos técnicos, onde são apresentados seus pontos gerais. Em
seguida, as diversas metodologias utilizadas no mundo são utilizadas
para um levantamento dos aspectos envolvidos no desenvolvimento da
tecnologia na Sessão~\ref{sec:nilm_mundo}. Finalmente, será
apresentado na Sessão~\ref{sec:nilm_cepel} a evolução dessa tecnologia
no escopo deste trabalho, desenvolvido em conjunto com o \gls{cepel}.

\section{Aspectos Gerais}
\label{sec:nilm_aspec_gerais}

O \gls{nilm} é uma alternativa para fornecer a informação de consumo
de energia elétrica desagregado por utensílio. Ao invés das técnicas
normalmente utilizadas --- onde se faz mão de sensores dispostos em cada
utensílio, esses enviando informações para uma central encarregada de
decodificá-las e, assim, identificar os utensílios que estão demandando
consumo na rede ---, o \gls{nilm} é um método em que não ocorre a intrusão
da propriedade do usuário (ou intrusão em escala mínima), contendo
geralmente apenas um medidor central no fornecimento de energia dessa
propriedade. O peso da identificação dos utensílios é transferido da
utilização de uma grande quantidade de sensores e \emph{hardware}
complexo para um \emph{software} e algoritmos de processamento de
sinais no intuíto de realizar uma análise profunda das medições e
desagregar as informações.

Ainda que o peso da identificação esteja no \emph{software}, a
aptidão dos algoritmos implementados dependem da capacidade do
medidor de extrair informações da rede e das distorções causadas pelos
utensílios na mesma, de forma que quanto mais avançado for o
\emph{hardware} de medição, enviando uma quantidade e/ou frequência
maior de informação aos algoritmos encarregados de realizar a
discriminação dos rastros deixados na rede elétrica pelos aparelhos,
também maior será a capacidade dos mesmos de desagregar o consumo
específico dos utensílios.

Possíveis aplicações para esse dispositivo são: 

\begin{itemize}
\item auxiliar ou substituir as \glspl{pph} sem que seja necessário
causar incômodo ao consumidor devido a presença de medidores na
residencia, fornecendo assim dados com maior fidelidade em caracter
desagregado por utensílio e maior frequência para estudos de \gls{ee}
no consumo de eletricidade (Sessão~\ref{sec:ee_dificuldades}), em
especial para o setor residencial;
\item disponibilizar a informação desagregada para o fornecimento do
Retorno Indireto Diário/Semanal e/ou Retorno em Tempo Real Desagregado
em programas de \gls{ee}. Os programas em \gls{ee} utilizando retorno
de informação são possíveis fontes de redução de consumo nos grandes
centros urbanos, no entanto, são necessários estudos no Brasil para
determinar seu potencial (Sessão~\ref{sec:ee_res_exp}). O retorno de
informação desagregada por utensílio pode ser utilizado por empresas
que oferecem serviços de redução de consumo
(Subsessão~\ref{ssec:ret_tec}), ou mesmo por consumidores que precisam
ter algum tipo de controle sobre seu consumo de energia; \item no
interesse das concessionárias, a informação desagregada pode auxiliar
os clientes a identificarem consumo não essencial durante o horário de
ponta, auxiliando no deslocamento de carga (ver item \emph{Resposta de
Ponta e Demanda versus Economia Fora de Ponta} na
Subsessão~\ref{ssec:ret_outros}), assim como identificar clientes com
maior potencial de redução de consumo durante esses períodos para
oferecer incentivos nesse sentido (Subssessão~\ref{ssec:ret_tec}); 
\item monitoramento da qualidade de energia, diagnóstico de carga,
identificação de aparelhos defeituosos ou com consumo excessivo de
energia;
\end{itemize}

As possibilidades de aplicação como um meio de melhoria da \gls{ee} e
redução da intensidade elétrica tem elevado o interesse nesse assunto,
em especial nos países desenvolvidos, como uma forma de aliviar
a pressão de consumo nos grandes centros urbanos e na redução de
emissões de \gls{co2} \cite{nilm_zeifman_review_2011}.
As possibilidades de aplicação dessa tecnologia tem levado a gigantes
no setor eletrônico, como \emph{Intel} e \emph{Belkin}, a investirem
fortemente no desenvolvimento dessa tecnologia. O crescente interesse
na evolução por parte da academia levou a organização do primeiro
\emph{workshop} especificamente para o tema em 2012
\cite{workshop_nilm}. Essa alternativa é mais simples quando
comparando com a automação residencial por não requerer mudanças na
produção dos eletrodomésticos --- a automação requer comunicação nos
dois sentidos (entre a interface e o aparelho), tal como a capacidade
de controle do aparelho, de forma que se faz necessário adaptar
aparelhos antigos e a produção dos novos equipamentos com essas
capacidades --- juntamente com o fato de haver um relutância social
quanto às residências automatizadas, apesar de esforços governamentais
e da mídia local \cite{Lipoff_Automation_2010} (a referência estudou a
falta de interesse na automação residencial nos \gls{eua}). Não
bastasse, também se pode citar o nascimento de \emph{start-ups}
procurando espaço na corrida por esse novo mercado, como
\emph{GetEmme} \cite{getemme_site} e \emph{Navetas}
\cite{navetas_site}.

\section{As diversas metodologias utilizadas no mundo}
\label{sec:nilm_mundo}

A ideia de desagregar a informação não é nova, sendo possível
encontrar pesquisas nesse sentido datando da década de 1980. Apesar
disso, um extenso levantamento bibliográfico
\cite{nilm_hart_1992_8,nilm_bouloutas_viterbi_ext_1991_11,
nilm_hart_fsm_viterbi_1993_12,nilm_norford_leeb_medianfilt_1996_13,
nilm_cole_data_extraction_1998_14,nilm_cole_extra_info_surge_1998_15,
nilm_powers_15minsamp_1991_16,nilm_farinaccio_16ssamp_1999_17,
nilm_marceau_16ssamp_improved_1999_18,nilm_baranski_genetic_base_2003_19,
nilm_baranski_genetic_detalhado_2004_20,nilm_baranski_summary_2004_21,
nilm_matthews_overview_2008_22,nilm_laughman_continuous_variables_2003_9,
nilm_leeb_spectral_envelope_1995_23,nilm_lee_variable_speed_estimation_2005_24,
nilm_wichakool_2009_25,nilm_shaw_2008_26,nilm_srinivasan_nn_2006_27,
nilm_akbar_2007_28,nilm_patel_2007_29,nilm_gupta_patel_2010_30,
nilm_sultanem_1991_10,nilm_chan_2000_31,nilm_lee_2004_32,nilm_lam_2007_33,
nilm_liang_pt1_2010_34,nilm_suzuki_2011_35,nilm_berges_2008_7,
nilm_berges_2009_36,2010_nilm_melhorando_pph_usa_37,
nilm_liang_pt2_2010_40} realizado em \citet*{nilm_zeifman_review_2011}
expõe que as técnicas aplicadas em \glspl{nilm} ou não são robustas no
sentido de atenderem especificamente a um grupo limitado de utensílios
estudados, ou apresentam acurácia marginal, mostrando que o processo
de desagregação da informação não é trivial. Indo além das
dificuldades técnicas, ressalta-se os pontos enfatizados nas
Subsessões \ref{ssec:asp_psic} e \ref{ssec:asp_visuais}, nas quais
foram evidenciadas as características multidisciplinares desse método
quando aplicado como um meio de economia de energia e intensificação
da \gls{ee}. Outras referências utilizadas que não constam em
\cite{nilm_zeifman_review_2011} são
\cite{nilm_zeifman_statistical_approach_resumo_2013,
nilm_bergman_distribuido_2011,
nilm_genetic_2013,nilm_zeifman_2011,nilm_patel_2011,
nilm_zeifman_nonintrusive_2011,nilm_ihome_tomek_2012,
wavelet_transients_2009,nilm_berges_multisensor_2010,
nilm_coppe_nascimento,nilm_itajuba_rodrigues}.

Dividiu-se esta sessão da seguinte maneira. As primeiras subsessões,
\ref{ssec:modelos_carga} e \ref{ssec:metodologia_generica}, referem-se
aos modelos de carga utilizados para os aparelhos eletrodomésticos e às
etapas usualmente envolvidas na desagregação da informação de
consumo quando utilizando \glspl{nilm} respectivamente. Uma
proposta de padronização para o cálculo de eficiência é apresentado na
Subsessão~\ref{ssec:nilm_eff_calc}. Em seguida, a
Subsessão~\ref{ssec:nilm_tecnicas} irá apresentar as diversas
abordagens utilizadas, sendo guiada na proposta de divisão das
técnicas e características feita por \cite{nilm_zeifman_review_2011}.
Finalmente, a discussão das informações levantas é realizada em
\ref{ssec:nilm_mundo_padroes}.

\subsection{Modelos de Carga}
\label{ssec:modelos_carga}

Os utensílios podem ser modelados devido às características de
comportamento de suas cargas. A seguir encontram-se possíveis
características de carga elétrica dos eletrodomésticos. Os quatro
primeiros itens são modelos de cargas mutuamente exclusivos, enquanto
os dois seguintes podem ser incluídos, dependendo das propriedades dos
aparelhos ligados à rede, para caracterizar aparelhos potencialmente
dificultadores do processo de desagregação \cite[com adaptações]{
nilm_hart_1992_8,nilm_baranski_genetic_base_2003_19,
nilm_zeifman_review_2011,nilm_zeifman_nonintrusive_2011,
nilm_apresentacao_review_2011,nilm_liang_pt2_2010_40}:

\begin{itemize}
\item \textbf{\Gls{c1}}: aparelhos que permanecem
ligados 24~h/dia, 7~dias/semana, com consumo de energia praticamente
constante. Ex.: detectores de fumaça, fontes de alimentação
constantemente ligadas à rede;
\item \textbf{\gls{c2}}\footnote{Nas referências, não há separação na
categoria \gls{c2}, que é utilizada para descrever exclusivamente
aparelhos modelados como \acrshort{c2a}. Nelas, a categoria
\acrshort{c2b} é incluída em \acrshort{c4}, uma vez que em ambos os
casos as \glspl{fsm} podem alterar o seu estado de operação para
qualquer outro estado independente tanto de qual estado anterior ele estava
operando como o tempo de operação nesse estado. A separação das
\glspl{fsm} com estados discretos em duas categorias parece mais
familiar do que incluir \glspl{fsm} com patamares discretos e
aleatórios em uma categoria que permite patamares
continuos.\label{fn:subdivisao}}: essa categoria é utilizada
para identificar aparelhos que passam um conjunto de estados definidos
no qual os ciclos de mudança de estados são repetidos com frequência
suficiente nos eventos diários ou semanais. Mesmo equipamentos que são
apenas ligados/desligados pelo usuário podem acabar sendo modelados
como \glspl{c2} por mudar seus patamares de consumo enquanto estiver
ligado. Essa categoria pode ser dividida em dois conjuntos:
\begin{itemize}
\item \textbf{\gls{c2a}}\fnref{fn:subdivisao}: os estados do aparelho repetem-se em
padrões definidos temporalmente, garantindo que os seus ciclos serão
observados frequentemente durante intervalos diários ou semanais. Ex.:
máquina de lavar roupas, máquinas de lavar louças;
\item \textbf{\gls{c2b}}\fnref{fn:subdivisao}: nesses aparelhos não há
um padrão para os seus ciclos de operação. A operação por uma fonte
externa, como o consumidor, altera o seu padrão de consumo sem ser
possível encontrar uma regra operativa para o ciclo através da busca
de repetições de suas trocas de estado na rede, os estados mudam
aleatoriamente depois de quantidades de tempo também aleatórios. Ex.:
ventilador de múltiplas velocidades, liquidificador --- ambos
dependendo de como operados pelo consumidor: se apenas ligados e
desligados, irão comportar-se como \acrshort{c3}s, enquanto se
operados de modo padronizado, irão se comportar como \acrshort{c2a}s;
\end{itemize}
\item \textbf{\gls{c3}}: um caso partícular das \glspl{c2} ocorre
quando o aparelho pode ser modelado como tendo apenas dois estados:
ligado/desligado. Ex.: lampadas, torradeiras, bombas de água;
\item \textbf{\Gls{c4}}: uma generalização das \glspl{c2}, onde
há uma infinidade de estados para os quais o aparelho pode operar.
Essa categoria pressupõe que o aparelho irá estabilizar o seu consumo
em um patamar após um período de tempo. Sua operação pode ser dividida
em dois grupos: operação manual do operador, aparelhos
auto-controlados. Estes são de mais simples detecção quando comparados
com aqueles, uma vez que seus ciclos de mudanças de estado são
distribuidos uniformemente no tempo. Ainda assim, essa categoria é o
maior desafio para as técnicas empregadas nos \glspl{nilm}, sendo
raramente tratada por elas. Ex.: lampadas com \emph{dimmer},
ferramentas elétricas (furadeiras, serras etc.), bomba de aquário.
\item \textbf{\Gls{c5}}\footnote{As referências optaram por não
criar essas categorias uma vez que essas são apenas
características das cargas. Por sua vez, as mesmas são citadas, no
mínimo, em \cite{nilm_zeifman_review_2011,nilm_liang_pt2_2010_40}
como dificultadores no processo de desagregação e, por esse motivo,
preferiu-se adicionar diretamente essas categorias para
enfatizar e facilitar a identificação de cargas com essas
características.\label{fn:categoria_add}}: aparelhos que causam
distúrbios na rede continuamente devido a flutuações no seu nível de
consumo. Ex. (observados pela equipe do \gls{cepel}): televisores,
onde a variabilidade de brilho, cores e som, alteram seu consumo
(também observado em 
\cite{nilm_zeifman_statistical_approach_resumo_2013});
computadores, que alteram sua potência conforme a demanda dos
processadores e \emph{coolers}, consumindo mais quando o usuário está
realizando tarefas como, por exemplo, executando algoritmos do
mestrado, escutando música etc.; cargas de demanda dinâmica tais como
o ar condicionado --- esse quando com o compressor ligado --- alteram
seu consumo para baleancear a demanda através de resposta em
frequência.  Dependendo da grandeza dos disturbios, os mesmos são
potenciais dificultadores à identificação de rastros deixados por
outros aparelhos, particularmente os de menor consumo, caso estudado 
em \cite{nilm_liang_pt2_2010_40}. A Figura~\ref{fig:ar_cond_dinamica}
demonstra a demanda dinâmica para esse aparelho. Outros exemplos de
utensílios com motores que também geram oscilações --- mas em ordem
inferior ao ar condicionado --- são: microondas, geladeira,
desumidificador \cite{nilm_liang_pt2_2010_40};
\item \textbf{\Gls{c6}}\fnref{fn:categoria_add}: apesar de não ser
uma característica de um utensílio \emph{per se} e nem constituir um
modelo de carga elétrica, estudos de performance de \glspl{nilm} podem
considerar quais equipamentos serão potencialmente vistos como se
fossem um mesmo equipamento por possuirem os mesmos padrões. Essa categoria
varia conforme os aparelhos ligados à rede e quais são as
características sendo extraídas. Por exemplo, um computador e uma
lampada incandescente possuem consumos semelhantes quando procurando
padrões no plano \acrshort{dp}$\times$\acrshort{dq}
\cite{nilm_laughman_continuous_variables_2003_9}, enquanto
equipamentos com motores de potências distintas podem não ser
desagregados quando apenas olhando para seus transitórios --- em
especial quando normalizados, abordagem que seria utilizada se
utilizando \acrfull{rna}.
\end{itemize}

\begin{figure}[h!t]
\centering
\includegraphics[width=\textwidth]
{imagens/ArCondicionado-CargaDemandaDinamica_ComTextoImpr.pdf}
\caption[Exemplo de carga com demanda dinâmica: Ar Condicionado]
{Exemplo de carga com demanda dinâmica: Ar Condicionado. No caso, as
pequenas oscilações vistas são causadas pela demanda dinâmica do ar
condicionado que, ao detectar oscilações na frequência, responde
alterando seu consumo para manter o balanço na rede.}
\label{fig:ar_cond_dinamica}
\end{figure}

\subsection{Metodologia genérica}
\label{ssec:metodologia_generica}

O \gls{nilm} pode ser resumido em quatro etapas para o
tratamento da informação na rede elétrica e em três abordagens quanto
às técnicas empregadas: 

\subsubsection[Etapas]{Etapas \cite{nilm_matthews_overview_2008_22}}
\label{top:etapas}

\begin{enumerate}[label={Etapa} \arabic* - ,ref=\arabic*,align=left]
\item\label{itm:etapa1} \textbf{\gls{fex}}: são extraídas
informações das amostragens realizadas. A diversidade de
características que podem ser extraídas depende da capacidade do
medidor. As características serão utilizadas tanto para a detecção de
eventos de transitório quanto na identificação de padrões dos
equipamentos. Em alguns casos, para reduzir o esforço de processamento
ou reduzir a necessidade de armazenamento, a \gls{fex} para a
identificação de equipamentos pode ser realizada ou armazenada somente
quando identificados os eventos na Etapa~\ref{itm:etapa2};
\item\label{itm:etapa2}\textbf{Detecção de eventos de
transitório}: identificar alterações causadas por utensílios na rede.
Essa etapa é necessária para identificar alterações no consumo de
equipamentos. Pode-se empregar limiares estáticos ou dinâmicos. Os
limiares dinâmicos permitem o ajuste de operação, reduzindo ou
aumentando a sensibilidade do detector conforme a presença de
equipamentos \acrshort{c5}. Em algumas topologias, essa etapa não é
realizada. Nesses casos, a Etapa~\ref{itm:etapa3} retorna para cada
intervalo de tempo estudado (normalmente um período completo na frequência
da rede, 50/60~\acrshort{hz}) o estado de operação dos equipamentos,
sendo necessário levantar nessa informação o estado de consumo de cada
aparelho para cada instante de tempo;
\item\label{itm:etapa3}\textbf{Reconhecimento de padrões}: utilizar as
características pertinentes para o reconhecimento de padrões,
deduzindo, assim, qual foi o utensílio que causou o disturbio na rede
e qual seu novo consumo. É desejável que o algoritmo seja capaz de identificar
a ocorrência de novos padrões e reconhecê-los em suas próximas 
aparições pois a construção de um catálogo com todos os possíveis
eletrodomésticos é impraticável, se não impossível. Tal tarefa só será
possível com a capacidade dos \glspl{nilm} de incluirem novos
equipamentos ao catálogo. Diversas técnicas podem ser utilizadas em
conjunto para esta etapa;
\item\label{itm:etapa4}\textbf{Refinamento dos resultados}: após as
Etapas~\ref{itm:etapa2} e/ou \ref{itm:etapa3}, pode-se adicionar uma
etapa opcional para procurar por possíveis erros ou melhorias na
informação desagregada. Por exemplo, corrigir a informação de um
aparelho que remanesce consumindo energia da rede por dias enquanto
sua operação normalmente ocorre em intervalos curtos. Isso pode
ocorrer por falhas na Etapa~\ref{itm:etapa2}, onde o desligamento do
equipamento não foi encontrado, ou na Etapa~\ref{itm:etapa3}, na qual
o desligamento foi identificado como causado por outro equipamento.
Outra possível melhoria seria encontrar possíveis novos ciclos de
operações para aparelhos~\gls{c2}. As estratégias corretivas podem ser
meramente remediativas, ou seja, simplesmente ignorar alterações de
estados que permanecem em no mesmo patamar de consumo durante um grande
período de tempo para melhorar a resolução em energia do \gls{nilm},
aplicarem técnicas complementares para reanalizar a informação
parcial, realizar uma otimização complementar utilizando a informação
obtida nas etapas anteriores, ou simplesmente realizarem uma nova
análise através da Abordagem~\ref{itm:abordagem2}.
\end{enumerate}

\subsubsection[Abordagens]{Abordagens \cite[com
adaptações]{nilm_zeifman_review_2011}}
\label{top:abordagens}

\begin{enumerate}[label={Abordagem} \arabic* - ,ref=\arabic*,align=left]
\item\label{itm:abordagem1}\textbf{Abordagem por reconhecimento de
padrões}\footnote{A referência considera
que a maior diferença entre as abordagens é o tempo de resposta, onde
a Abordagem~\ref{itm:abordagem1} responderia em tempo real e a
Abordagem~\ref{itm:abordagem2} para cada período otimizado, o que pode
não ser verdade. Geralmente ambos os casos passam por um período de
otimização antes de serem empregados e, depois de otimizados, são
utilizados para a detecção dos padrões dos aparelhos na rede. A
distinção está que o primeiro otimiza a capacidade de
discernir os padrões --- sendo a reconstrução consequência disso ---,
enquanto o segundo a capacidade de reconstruir com maior fidelidade
possível o sinal original --- obtendo os padrões como
resultado.\label{fn:diff_abordagens}}: as técnicas de reconhecimento
de padrões são treinadas (otimizadas) em conjuntos de dados similares
aos quais eles irão operar. O reconhecimento de padrões pode ocorrer
apenas para as respostas da Etapa~\ref{itm:etapa2} ou para cada ciclo
da rede. Algumas técnicas utilizadas nessa abordagem podem ser
robustas aos aparelhos desconhecidos, sendo capaz de destacar seu padrão
dos outros já conhecidos e adicioná-lo ao catálogo de padrões. Assim,
quando esses padrões ocorrerem novamente, eles serão identificados
como o mesmo aparelho --- chamado de aprendizado em tempo real. Em
\cite{nilm_matthews_overview_2008_22} é observado a importância dessa
estratégia para tornar possível o crescimento do catálogo, que, tendo
o novo aparelho nomeado pelo consumidor, torna possível a criação de
um catálogo universal de equipamentos. Essa tarefa é impraticável, se
não impossível, de ser realizada em laboratório;
\item\label{itm:abordagem2}\textbf{Abordagem
por otimização}\fnref{fn:diff_abordagens}: concentra a capacidade de
suas técnicas na otimização, onde é realizada a procura por uma
combinação de aparelhos cujo o sinal agregado resultante é a melhor
aproximação do possível do sinal observado. Em alguns casos,
utiliza-se a concentração dos dados em longos períodos de tempo para
identificar o consumo desagregado, retornando a operação dos diversos
equipamentos no final do processo. Nesses casos, a informação final
pode ser utilizada como padrões a serem identificados posteriomente.
Para manter os equipamentos atualizados, novos períodos (possivelmente
menores ao período inicial) podem ser utilizados para garantir a
resposta adequada a possíveis alterações na presença ou utilização de
aparelhos. Na outra possibilidade, a otimização é realizada no nível
de ciclo da rede, onde, sabendo o padrão dos possíveis aparelhos
presentes, se busca a melhor combinação operativa que reflitam o sinal
observado;
\end{enumerate}

\subsection{Cálculo da eficiência}
\label{ssec:nilm_eff_calc}

\subsubsection{Padronização}
\label{top:nilm_padrao}

O estudo bibliográfico realizado por \cite{nilm_zeifman_review_2011}
teve dificuldades ao tentar comparar as diferentes técnicas utilizadas
nos \glspl{nilm}. O primeiro empecilho está na variedade das base
de dados utilizadas, possuindo aparelhos e estados de operações
bastante distintos, criando condições que podem previlegiar a
eficiência de um determinado \gls{nilm}. Para a unificação dos dados
estudados e permitir a comparação de performance entre os algoritmos
empregados nos \glspl{nilm}, foi disponibilizado por
\cite{nilm_dataset_blued_2012} um conjunto de dados públicos para
a análise, sendo aqui sugerida a sua utilização. O conjunto de dados
foi construido para representar a realidade de residências nos
\gls{eua} e por isso podem não corresponder a realidade brasileira,
mas o conjunto de dados servem como base para comparação com a
perfomance dos \glspl{nilm} aqui desenvolvidos com os do exterior,
assim como nada impede da utilização em paralelo de dados próprios.

Outra dificuldade foi o fato de autores utilizarem uma
medida própria para o cálculo das taxas de eficiência.  Além disso,
normalmente os autores não reportaram as taxas de falsos positivos na
Etapa~\ref{itm:etapa2}, apenas a capacidade dos algoritmos de
detectarem os eventos (excessões observadas são
\cite{nilm_marceau_16ssamp_improved_1999_18,nilm_liang_pt2_2010_40}).
Em outros casos, os autores concentraram-se apenas na capacidade dos
algoritmos da Etapa~\ref{itm:etapa3} de discriminarem equipamentos,
reportando medidas representativas para essa eficiência.

Por isso, \cite{nilm_zeifman_review_2011} recomenda a utilização das
medidas apresentadas por \cite{nilm_liang_pt1_2010_34}, no qual se
apresentou considerações metódicas para o tema. Foram apresentadas
três medidas. A primeira medida, \gls{det_eff}, considera a
capacidade do \gls{nilm} de desagregar a informação nos eventos que
foram detectados\footnote{Empregada por \cite{nilm_hart_1992_8} quando não
disponível a medição paralela de energia dos utensílios e por
\cite{nilm_gupta_patel_2010_30}.}. Quando
interessado apenas em estudar a capacidade do classificador para os
eventos detectados, a medida \gls{class_eff} deve ser 
utilizada. Finalmente, a \gls{total_eff} é
dada por \ref{eq:total_eff}, levando em conta apenas a capacidade do
\gls{nilm} de corretamente classificar os eventos
reais, causados pelos equipamentos na rede\footnote{Empregada por
\cite{nilm_patel_2007_29,nilm_berges_2009_36} pois a
Etapa~\ref{itm:etapa2} foi realizada
manualmente \label{fn:patel_manual} e, geralmente, por demais estudos
que estudaram apenas a Etapa~\ref{itm:etapa3}.}.

\begin{subequations}\label{eq:eff}
\begin{equation}\label{eq:det_eff}
\eta_{det} = \frac{N_{id}}{N_{real} + N_{fp} - N_{ni}}
\end{equation}
\begin{equation}\label{eq:class_eff}
\eta_{class} = \frac{N_{id}}{N_{real} - N_{ni}}
\end{equation}
\begin{equation}\label{eq:total_eff}
\eta_{total} = \frac{N_{id}}{N_{real}}
\end{equation}
\end{subequations}

\noindent onde:  

\begin{description}
\item[$N_{id}$] são eventos identificados, ou seja, corretamente
detectados e classificados pelo \gls{nilm}; 
\item[$N_{real}$] são os disturbios causados pelos equipamentos
na rede;
\item[$N_{fp}$] são eventos devido a falsos positivos, ou seja,
evento erroneamente identificados;
\item[$N_{ni}$] são eventos não identificados, ou perdas de alvo.
\end{description}

Segmenta-se \ref{eq:total_eff} para obter a eficiência do \gls{nilm}
por aparelho conforme:

\begin{equation}\label{eq:app_eff}
\eta_{total}^i\approx\frac{N_{id}^i}{N_{real}^i} ~~ \forall ~~ i =
1,2,...,N_{ap}
\end{equation}

\noindent onde $N_{id}^i$ e $N_{real}^i$ são os respectivos
$N_{id}$ e $N_{real}$ para o i-ésimo aparelho dos $N_{ap}$
disponíveis.

O grande favorecimento para essas medidas é sua simplicidade de serem
obtidas, porém algumas considerações podem ser feitas sobre elas.
Primeiro, a medida com maior sensibilidade à capacidade do \gls{nilm}
é a \ref{eq:det_eff}, uma vez que os valores por ela representados
levam em conta as perdas de alvo e os falsos positivos.
Segundo, as mesmas não levam em conta o consumo de energia dos
aparelhos, dando importância análoga para aparelhos com parcelas
pequenas ou grandes de consumo. Além disso, a correta identificação
dos $N_{id}$ não significa que a energia será corretamente
reconstruída, dependendo da capacidade do \gls{nilm} de unir essas
informações para gerar a informação do consumo desagregado.  Ainda,
como apontado por \cite{nilm_zeifman_review_2011}, elas apenas
representam a eficiência no ponto de operação, não sendo possível
observar como o \gls{nilm} se portaria para outros pontos. Indo além,
elas não permitem comparações de técnicas utilizadas exclusivamente para
as Etapas~\ref{itm:etapa2} e \ref{itm:etapa3}, impedindo a
contraposição de técnicas onde os autores se limitaram a uma dessas
etapas.

\subsubsection{Outras representações}
\label{top:outras_eff}

Por isso, além das medidas apontadas, outras maneiras de representar
a eficiência podem ser utilizadas para complementar o estudo do
comportamendo da abordagem utilizada. Uma técnica para representar o
compromisso entre a capacidade de detectar eventos e a quantidade de
falsos positivos encontrados é a curva \gls{roc}, também recomendada
por \cite{nilm_zeifman_review_2011}. A \gls{roc} além de ser utilizada
para expressar de maneira geral a capacidade do algoritmo de detectar
e identificar em função dos falsos positivos, pode ser utilizada para
estudar a eficiência específica da Etapa~\ref{itm:etapa2}. Já para a
Etapa~\ref{itm:etapa3}, a matriz de confusão permite entender quais
aparelhos ou classes de aparelhos são confundidos em outras classes,
assim como a eficiência de classificação em uma única representação.

As outras medidas utilizadas na literatura levantada são: o
percentual de classificações corretas por ciclo da rede, ou seja, a
correta classificação do estado de operação para cada ciclo dividido
pelo número total de ciclos
\cite{nilm_srinivasan_nn_2006_27,nilm_suzuki_2011_35}; porcentagem de
detecção de transitórios \cite{nilm_patel_2007_29}\footnote{O estudo
reportou eficiência para as Etapas~\ref{itm:etapa2} e \ref{itm:etapa3}
separadamente. Como foi dito na nota \ref{fn:patel_manual}, a
Etapa~\ref{itm:etapa3} utilizou eventos recortados manualmente.};
\gls{p_eff_i}
\cite{nilm_hart_1992_8,nilm_cole_data_extraction_1998_14,
nilm_cole_extra_info_surge_1998_15,nilm_farinaccio_16ssamp_1999_17,
nilm_marceau_16ssamp_improved_1999_18}; \gls{p_eff} 
\cite{2010_nilm_melhorando_pph_usa_37}; desvio
do tempo em que o aparelho foi identificado operando em relação ao
tempo que ele realmente estava operando
\cite{nilm_farinaccio_16ssamp_1999_17}\footnote{O estudo focou na
identificação de grandes cargas elétricas como ar condicionado e
aquecedores de água, modelados por \gls{c3}\label{fn:valc3}. Essa
medida seria limitada para outros modelos.}; porcentagem de detecção
de eventos de transição para ligado perdidos
\cite{nilm_farinaccio_16ssamp_1999_17}\fnref{fn:valc3}; erro médio
absoluto de reconstrução de energia e outras estatísticas por aparelho
\cite{nilm_powers_15minsamp_1991_16}.

A medida mais utilizada pelas referências, a \gls{p_eff_i}, embora
por elas não definida matematicamente, concebe-se que seja dada por:

\begin{subequations}
\begin{equation}\label{eq:frac_en_app}
\rho_{En}^i = \frac{E_{det}^i}{E_{real}^i} ~~ \forall ~~ 
i = 1,2,...,N_{ap}
\end{equation}
\begin{equation}\label{eq:frac_en}
\rho_{En} = \frac{\sum_{i}^{N_{ap}}E_{id}^i}{\sum_{i}^{N_{ap}}E_{real}^i} 
\end{equation}
\end{subequations}

\noindent onde $E_{det}^i$, $E_{real}^i$ é o consumo detectado e
consumo real do i-ésimo aparelho, respectivamente, o último sendo
obtido por submedição ou por um estimador. A \gls{p_eff_i}
pode ser generalizada para calcular a \gls{p_eff} através de
\ref{eq:frac_en}.  Essas medidas levam em consideração o consumo
detectado pelo \gls{nilm}, mas perdem a capacidade das medidas 
\ref{eq:eff} de representar a informação que foi corretamente
identificada. Por exemplo, se um equipamento é considerado como ligado
em um espaço de tempo em que o mesmo está desligado, isso irá
contribuir para corrigir possíveis erros que seriam atribuidos quando
o estado estimado e a operação estiverem na lógica oposta.

Assim, fica evidente que essas medidas precisam ser refinadas para
identificar os momentos nos quais a energia foi corretamente
reconstruída. Para isso, aqui se sugere o uso de \ref{eq:e_id_i} com
o intuíto de determinar a \gls{e_id_i}. A ideia
é representar que identificações do aparelho em outros estados, mas
com pequena diferença de energia, não irão afetar tanto na resolução
de energia, assim como resguardar que identificações em estados de
consumo maiores para os quais os aparelhos realmente operam não
arremeterão na conta de energia corretamente identificada:

\begin{equation}\label{eq:e_id_i}
E_{id}^i = E_{det}^i-\varepsilon^i
\end{equation}

A \gls{en_res} representa a ideia, em energia, para tanto falsos
positivos ou quanto identificações errôneas para estados de maior
consumo, ou seja, a parcela de $E_{det}^i$ que excede àquela lida por
um medidor externo ou estimada $E_{real}^i$. Ela pode ser descrita
por:

\begin{equation}\label{eq:en_res}
\varepsilon^i = \left\{\begin{array}{rl}
 E_{det}^i - E_{real}^i &\mbox{ se $E_{det}^i>E_{real}^i$} \\
 0 &\mbox{o.c.}
\end{array} \right. ~~ \forall ~~ i = 1,2,...,N_{ap}
\end{equation}

Isto posto, para obter a \gls{en_eff_i} e sua generalização,
\gls{en_eff}, basta empregar:

\begin{subequations}
\begin{equation}\label{eq:en_eff_i}
\eta_{En}^i = \frac{E_{id}^i}{E_{real}^i} ~~ \forall ~~ i =
1,2,...,N_{ap}
\end{equation}
\begin{equation}\label{eq:en_eff}
\eta_{En} = \frac{\sum_{i}^{N_{ap}}E_{id}^i}{\sum_{i}^{N_{ap}}E_{real}^i}
\end{equation}
\end{subequations}

E para as taxas de redundância:

\begin{subequations}
\begin{equation}\label{eq:p_red_i}
\rho_{red}^i = \frac{\varepsilon^i}{E_{real}^i} ~~ \forall ~~ i =
1,2,...,N_{ap}
\end{equation}
\begin{equation}\label{eq:p_red}
\rho_{red} = \frac{\sum_{i}^{N_{ap}}\varepsilon^i}{\sum_{i}^{N_{ap}}E_{real}^i}
\end{equation}
\end{subequations}

Posteriormente, descobriu-se que o próprio autor de
\cite{nilm_zeifman_review_2011} criou uma medida que permite também
exprimir a questão de energia redudante e corretamente detectada, e
uniu-as através da chamada medida-F (tradução própria de
\emph{F-measure}) \cite{nilm_zeifman_statistical_approach_2012}. A
mesma, \ref{eq:fmeasure}, é o quadrado da média geométrica normalizado
pela média aritmética entre duas outras grandezas: 

\begin{itemize}
\item parcela de energia atribuida ao aparelho que foi realmente
consumida pelo mesmo, um parâmetro parecido com a ideia de energia
redundante descrita por \ref{eq:en_recon};
\item parcela de energia que foi corretamente identificada em relação
ao consumo total do aparelho, a própria \gls{en_eff_i} aqui descrita
em \ref{eq:en_eff_i}.
\end{itemize}

\begin{equation}\label{eq:en_recon}
\eta_{En,prec}^i = \frac{E_{id}^i}{E_{det}^i}
\end{equation}

\begin{equation}\label{eq:fmeasure}
F^i=\frac{2 \eta_{En,prec}^i \eta_{En}^i}{\eta_{En,prec}^i+\eta_{En}^i}
\end{equation}

\subsection{Técnicas aplicadas}
\label{ssec:nilm_tecnicas}

As abordagens aplicadas desde o início dos estudos ao tema e
utilizadas como referências fizeram mão de ostensivas técnicas para a
desagregação do consumo. Cada vertente buscou extrair características
ou inovar aplicando outras técnicas, de forma que é possível observar
uma grande diversidade de abordagens. As abordagens serão agrupadas em
relação à \gls{fex} realizada. A capacidade de extrair
características dos sinais é correlacionada com a frequência de
amostragem e, em vista disso, dividir-se-á os métodos aplicados de
acordo com a taxa de amostragem utilizada. A ideia de subdivisão aqui
seguida foi de autoria da referência \cite{nilm_zeifman_review_2011}.

\subsubsection{1. Medição com Baixa Amostragem}
\label{top:nilm_baixa_am}

A utilização de características mascroscópicas de consumo do aparelho,
como alterações no patamar de consumo da rede, foi a
primeira abordagem encontrada ao tema. As mesmas podem ser obtidas sem
grande granularidade na taxa de amostragem, por isso, esse tipo de
abordagem beneficia-se de medidores de baixo custo, amplamente
disponíveis no mercado. No entanto, \cite{nilm_berges_2008_7} alerta
para discrepâncias entre medidores na ordem de 10\%-20\%, bem maiores
que aquelas alegadas, de 3\%. Os medidores testados no caso foram
\emph{Brand Meter I}, \emph{Watts Up? PRO} e \emph{EnerSure}.

A taxa de amostragem mais frequentemente utilizada é 1 Hz, entretanto
alguns estudos fizeram mão taxa de amostragem ainda menores por
desejarem identificar aparelhos que se ressaltam dentre os outros
devido ao seu relativo alto consumo, como ar condicionado, aquecedores
de água e geladeira. Exemplos de medidores utilizados no exterior são
\gls{ted} \cite{ted_site} e \emph{Watts up? PRO} \cite{wattsup_site},
o último sendo capaz de informar o consumo de \acrlong{q}.

\begin{enumerate}[label=\textbf{1.\arabic*},wide=\parindent]
\item \textbf{\Acrlong{p} e \Acrlong{q}}
\label{nilm:pot_real_reat}

\indent A referência inicial de grande destaque no tema,
\citet*{nilm_hart_1992_8}, ocorreu em 1992. Nela, utiliza-se medições de
\gls{p} e \gls{q} com uma taxa de amostragem de 1 Hz. A abordagem
aplica uma normalização para reduzir flutuações no consumo devido a
alterações na tensão de acordo com \ref{eq:norm_hart} com o intuíto de
reduzir disperções nos dados. O estudo de \citeauthor*{nilm_hart_1992_8}
limitou-se a identificar apenas cargas com potência maior a 150
\acrshort{watt}. A essência da metodologia ainda pode ser encontrada
em \glspl{nilm} mais atuais, sendo esta: 

\begin{equation} \label{eq:norm_hart}
P_{\text{norm}}(t) = \left[ \frac{120}{V(t)} \right]^2 P(t)
\end{equation}

\begin{enumerate}[label=\arabic*]
\item Detecta-se transitórios de consumo na rede devido a mudança de
estado de um utensílio através de alterações no consumo que devem
superar um limiar específico (15 \acrshort{watt}/\acrshort{var}) para os sinais
normalizados como em \ref{eq:norm_hart} para \acrshort{p} e \acrshort{q}.
As amostragens dentro de um regime permanente são normalizadas para
sua média com o objetivo de tirar o ruído. Para a \gls{fex}, usa-se o
degrau entre o regime permanente posterior e anterior (já no valor de
suas médias) ao evento transitório para \gls{p} e \gls{q};
\item Os eventos de transitório são analizados por um algoritmo de
agrupamento que irá gerar os centróides das mudanças de estado
possíveis causadas pelos utensílios no plano
\acrshort{dp}$\times$\acrshort{dq};
\item Centróides com simétria em relação aos eixos são tomados em
pares e com eles são criados modelos \gls{c3}. Para os centróides
remanescentes, além de regras heurísticas como a junção de centróides
próximos que permitam o pareamento com um outro refletido nos eixos,
determina-se possíveis combinações de centróides que possam formar uma
\gls{c2} utilizando uma adaptação do algoritmo de Viterbi
\cite{nilm_bouloutas_viterbi_ext_1991_11,
nilm_hart_fsm_viterbi_1993_12}. Assim que é determinada uma combinação
que permite a criação de uma \gls{c2}, os centróides da mesma são
removidos, e o processo continua até que todas as \glspl{fsm} tenham
sido construídas. A adaptação utilizada permite várias operações para
consertar corrupções e retornar uma estimativa ótima da \gls{fsm}
original. Como a reconstrução depende da estatística do processo, é
necessário que as mudanças de estado das \glspl{fsm} observadas tenham
um comportamento para que a \gls{fsm} original seja reconstruída, e
por isso, restringe-se apenas às \glspl{c2a}. As \glspl{c2b} podem ser
reconstruídas se houver conhecimento prévio da presença das mesmas, de
modo que elas sejam medidas operando em cada um de seus estados e
então inseridas manualmente no catálogo do \gls{nilm};
\item Em seguida é levantado o comportamento dos equipamentos,
montando o estados de consumo para cada aparelho. É utilizado um
algoritmo de força-bruta para corrigir ocorrências de dois ligamentos
ou desligamentos em sequência. A causa desses erros é, geralmente, a
ocorrência de um evento simultâneo de dois equipamentos. Assim, o
algoritmo busca por eventos não-usuais cuja soma é o valor de dois
outros eventos perdidos;
\item Finalmente, é levantada a estatística detalhando o
comportamento de consumo, como o tempo ligado e desligado de cada
equipamento. Essa informação, junto com a potência do equipamento é
utilizada para auxiliar a identificar o equipamento. 
\end{enumerate}

O método é robusto para o desagregamento de cargas \glspl{c3}
($> 150~$\acrshort{watt}) e as adaptações \cite{nilm_bouloutas_viterbi_ext_1991_11,
nilm_hart_fsm_viterbi_1993_12} do algoritmo de Viterbi
parecem resolver o problema das \glspl{c2a}, no entanto, essas
referências para o tratamento das \glspl{c2a} fundamentam a
matemática envolvida, mas não dão detalhes mais práticos da
implementação\footnote{Outras referências nesses artigos
não foram consultadas, podendo possuir essas informações.}.
Outro problema das metodologias envolvendo
algoritmos de agrupamento é a lenta alteração da resitência conforme a
operação do aparelho. Geralmente, ao interromper a operação, o
aparelho tem alterações no consumo na margem de 5\%-10\% em relação ao
inicio de operação \cite{nilm_sultanem_1991_10}.
\citeauthor*{nilm_hart_1992_8} observa o degrau geralmente é
menor em valor absoluto para os desligamentos nos casos de
equipamentos com motores, que reduzem o consumo conforme seu
aquecimento.

Uma estratégia bastante parecida é realizada por
\citet*{nilm_cole_data_extraction_1998_14,nilm_cole_extra_info_surge_1998_15}, 
onde são feitas considerações em relação as características de bordas e
inclinações. Aqueles são definidos como o auge atingido de potência
durante o acionamento e estes variações lentas de mudança no consumo.
Apesar de definir as bordas como o pico de potência, as referências
empregam as bordas apenas como os eventos de transição de consumo, não
empregando essa informação para classificação. A abordagem aplicada,
ao invés de agrupar os dados para depois procurar por possíveis
aparelhos como feito por \citeauthor*{nilm_hart_1992_8}, primeiro
busca temporalmente por ciclos fechados nos eventos de transição (ou
bordas, como na nomenclatura da referência), que depois serão
adicionadas aos centróides no espaço
\acrshort{dp}$\times$\acrshort{dq}. Se o centróide não existir, será
criado um candidato a centróide. Conforme a quantidade de ciclos dos
centróides aumenta, o mesmo irá se tornar um candidato a uma carga.
Para as cargas \glspl{c3}, a carga será aceita apenas se a detecção
das bordas ocorrerem repetidamente. Já para as cargas \glspl{c2a}, foi
realizado um estudo da probabilidade dela ter sido originada pela
sobreposição de duas bordas geradas por equipamentos distintos. A
conclusão foi que se forem encontrados mais de um ciclo de três bordas
em um período de 6 horas é suficiente para aceitá-lo como uma
\gls{c2a}. Finalmente, o envelope só foi considerado para a melhoria
em resolução de energia e, no entanto, a referência indica que a
utilização da média de consumo entre as bordas apresenta melhores
resoluções.

\item \textbf{\Acrlong{p}, \Acrlong{q} e Transitório}
\label{nilm:pot_real_trans}

Um trabalho paralelo ao de \citeauthor*{nilm_hart_1992_8} foi
realizado pelo mesmo instituto para aplicar o \gls{nilm} no setor
comercial, sendo realizado por
\citet*{nilm_norford_leeb_medianfilt_1996_13}. No setor comercial
são encontrados aparelhos com características diferentes ao setor
residencial, geralmente com transitórios mais lentos (podendo chegar a
cerca de centenas de segundos
\cite{nilm_norford_leeb_medianfilt_1996_13}), menor consumo reativo
devido às preocupações com a qualidade de energia e consequentemente
correção do fator de potência, e a presença de \gls{c5}, como exemplo,
na referência foi observada uma bomba com picos periódicos de 20
k\acrshort{watt}. Tipicamente há também uma maior presença de
equipamentos com cargas variáveis, \glspl{c4}, como motores de velocidade
variável. Assim, foi adicionado a informação do transitório da
envoltória em amostragens de 1~\acrshort{hz}, que por serem maiores,
ainda podem ser observados em amostragens baixas, suprindo, ao mesmo
tempo, a menor capacidade de discriminação da variável \gls{q} nesse
setor. Para redução dos ruídos, utilizou-se um filtro de mediana
com 11 pontos, esse sendo mais indicado para a eliminação dos picos
quando comparado aos filtros lineares, que terão dificuldades de
distinguir os picos e os degraus, uma vez que eles tem espectros de
frequência parecidos. É aplicada uma medida de distância entre as
sessões dos transitórios observados e os transitórios característicos
que, ao estarem dentro de um limiar, serão identificados como um
determinado equipamento.  Para o tratamento das \glspl{c4}, a
referência indica o emprego de variáveis de controle, quando
disponíveis, correlacionadas com o seu consumo para estimá-las, como o
caso para os equipamentos de \gls{avac} em geral.

\item \textbf{Unicamente \acrlong{p}}
\label{nilm:pot_real}

A medição de \acrlong{q} adiciona custo ao \gls{nilm} --- ainda que
não tão oneroso quanto medições em altas frequências --- e, para
detectar certos aparelhos com assinaturas de destaque na rede,
essa variável pode ser desnecessária. Em outros casos, medidores que
disponibilizam essa informação podem não estar disponíveis, sendo
possível operar apenas com a \acrlong{p}.

\begin{enumerate}[label*=.\textbf{\arabic*},wide=\parindent]
\item \textbf{Separação dos principais equipamentos por uso-final}

Exemplos do primeiro caso são os estudos de
\citet*{nilm_powers_15minsamp_1991_16,nilm_farinaccio_16ssamp_1999_17,
nilm_marceau_16ssamp_improved_1999_18,
nilm_zeifman_statistical_approach_resumo_2013},
para os quais os autores se preocuparam em identificar apenas
equipamentos de maior uso-final.
\cite{nilm_powers_15minsamp_1991_16,nilm_farinaccio_16ssamp_1999_17}
utilizaram somente regras heurísticas, enquanto
\cite{nilm_marceau_16ssamp_improved_1999_18} também empregou os
degraus em potencia real e um filtro para a detecção dos
ligamentos/desligamentos dos aparelhos. Finalmente,
\cite{nilm_zeifman_statistical_approach_resumo_2013} abordou o problema
estatísticamente.

\begin{itemize}[wide=\parindent]
\item \emph{Regras empíricas}

Em \cite{nilm_powers_15minsamp_1991_16}, foram reportadas a capacidade
de reconstrução para ar condicionado e aquecedores de água. A
amostragem é realizada a cada 15 minutos e os arquivos são analisados
dia a dia. Por ser proprietário, as regras não são detalhadas (é
utilizada uma árvore de decisões, embora o estudo considere a aplicação
de redes neurais), mas o algoritmo procura por picos no consumo, assim
como sua duração, tempo e magnetude, que são utilizados pelas regras
para determinar se os mesmos foram utilizados para os usos-finais
cobiçados. Posteriomente, eles são ajustados conforme
verificações de consistência. Para o ar condicionado, é relatado que o
valor de pico estimado médio para as residências difere cerca de
apenas 5\% do valor original médio, enquanto o consumo fica na margem
de 10\% e observa-se boa capacidade de estimar os horários de consumo.
 
Já os estudos \cite{nilm_farinaccio_16ssamp_1999_17,
nilm_marceau_16ssamp_improved_1999_18}, realizados por outro
grupo, empregaram amostragem de \acrshort{p} a cada 16~segundos. Os
aparelhos estudados foram: geladeira, aquecedor de água e aquecedores
de ambiente (este somente em
\cite{nilm_marceau_16ssamp_improved_1999_18}\footnote{O algoritmo da
referência \cite{nilm_marceau_16ssamp_improved_1999_18}
também leva em consideração a máquina de lavar roupa, mas os
resultados focaram apenas nos outros três aparelhos.}).
\cite{nilm_zeifman_review_2011} expõe a arbitrariedade e não
intuitividade das regras utilizadas em
\cite{nilm_farinaccio_16ssamp_1999_17}, que precisam ser estudadas
para cada caso de aparelho. Foram determinadas 8 regras para cada
aparelho (algumas regras são reaproveitadas entre aparelhos),
divididas em duas etapas: determinar o conjunto de eventos de
transição e a duração do consumo.  Em seguida, a duração de consumo é
multiplicada pela demanda média do aparelho durante a fase de
treinamento para obter o consumo estimado. A fase de treinamento,
período em que há medição paralela dos equipamentos, e, por isso,
ocorrendo intrusão da propriedade do consumidor, é feita para um
período de uma semana. A reconstrução de energia diária para os
equipamentos é na margem de $-10,5\%$ a $15,9\%$.

O estudo em sequêcia aperfeiçoou o anterior com uma abordagem única
para determinar os aparelhos em operação. Ele compara, em ordem
decrescente em termos de demanda média operativa, se a magnetude
do evento é próxima à média do nível operativo de um dos aparelhos
almejados, empregando como limiar de corte dois desvios padrões. Ainda
assim, a referência emprega diversas regras de pré/pós-processamento
determinadas empiricamente para melhorar a resolução em energia, assim
como também necessita do período de treinamento através de medição
paralela de 1~semana, limitando a aplicabilidade do método para uma
gama maior de equipamentos. Por outro lado, o método serve para o seu
próposito, obtendo reconstruções na faixa de 10\% para a maioria das
análises realizadas.

\item \emph{Utilização da estatística de uso}

A abordagem empregada cerca de 20 anos depois por
\citet*{nilm_zeifman_statistical_approach_resumo_2013} (os dois primeiros são os
autores do levantamento bibliográfico \cite{nilm_zeifman_review_2011}), 
operou em cima de dados amostrados por um mostrador de energia
domiciliar na taxa de 1~\acrshort{hz}, obtendo apenas amostragens da
\acrlong{p} (o medidor usado foi o \gls{ted}). Baseou-se no
conhecimento prévio de utilização dos aparelhos para encontrar aquele
mais representativo estatisticamente.  Para isso, o estudo utilizou o
conceito de máxima entropia. Também se limitou a identificar os
aparelhos de maior uso-final, no caso, os aparelhos de interesse são:
\begin{enumerate*}[label=\itshape\alph*\upshape)]
\item ar condicionado;
\item aquecedores de ambiente;
\item aquecimento de água doméstica;
\item \label{itm:iluminacao} iluminação;
\item geladeira;
\item secadores de roupa elétricos;
\item \label{itm:aparelhoeletronico} aparelhos eletrônicos;
\end{enumerate*} que representam em média 80\% do consumo residencial,
no caso, para os \gls{eua}. 

Com base em outros estudos, a referência faz o levantamento da
distribuição conjunta Beta para o consumo e tempo de operação dos
aparelhos tratando alguns casos específicos. Por exemplo, para os
secadores de roupa elétricos, são utilizados duas distribuições Beta,
uma para o ciclo do elemento de aquecimento e outra componente entre
os ciclos de secagem; enquanto para as lampadas, utiliza-se
dependência estatística para o consumo e tempo de duração em relação
ao ambiente para os quais elas operam, resultando em uma função de
probabilidade conjunta que é a mistura das funções de probabilidades
conjuntas para cada ambiente. Ainda, para melhorar a performance,
adiciona-se como característica as assinaturas específicas dos
aparelhos. O exemplo é a televisão em comparação com uma lampada, onde
aquele varia o seu consumo conforme flutuações na imagem e som,
enquanto este tem o consumo bastante estável. Essas
características ``finas'' dos aparelhos podem ser modeladas
matematicamente e empregadas em conjunto com o conhecimento prévio de
utilização dos aparelhos.

Os autores utilizaram o classificador \emph{Naïve Bayes} com base para os
sete uso-finais indicados, adicionados de uma probabilidade conjunta
uniforme de potência e tempo para as outras cargas possíveis. Os
resultados empregaram a medida-F (\ref{eq:fmeasure}), obtendo valores
de acurária para os uso-finais mais desafiantes na ordem de 65\% e
70\%, sendo os mesmos os itens \ref{itm:iluminacao} e
\ref{itm:aparelhoeletronico}, respectivamente. Já quando utilizando a
informação de características finais, essas mesmas acurácias se
elevam para 92\% e 90\%.  

\end{itemize}

\item \textbf{\Acrlong{q} não disponível}

\begin{itemize}[wide=\parindent]
\item \emph{A abordagem de 
\citeauthor*{nilm_baranski_genetic_base_2003_19}}

Com o intuíto de possibilitar a saturação da aplicação de \glspl{nilm}
na Alemanha, \citet*{nilm_baranski_genetic_base_2003_19,
nilm_baranski_genetic_detalhado_2004_20,nilm_baranski_summary_2004_21}
recorreram a leitura ótica dos medidores eletromecânicos
(o trabalho \cite{nilm_baranski_genetic_base_2003_19}, realizado em
2003, indica que mais de 99\% dos medidores desse país possuem essa
configuração) para obter as medições com frequência de
1~\acrshort{hz}. Por isso, apenas \gls{p} estava disponível para esses
estudos. 

Apesar das limitações, essa abordagem é uma referência de destaque
devido às diversas contribuições feitas. Para melhorar a capacidade de
discriminação entre os equipamentos, além da potência ativa, o estudo
adicionou como característica o pico de consumo para o evento de
transição, bem como o período que o mesmo leva para estabilizar
(apresentado em \cite{nilm_baranski_genetic_detalhado_2004_20}),
diferente de \cite{nilm_cole_data_extraction_1998_14,
nilm_cole_extra_info_surge_1998_15} que observou essas propriedades,
mas empregou somente a última com o intuíto de melhorar a
capacidade de reconstrução de energia. Ainda, o autor contribuiu
transformando a estratégia aplicada por
\citeauthor*{nilm_hart_1992_8} em uma Abordagem~\ref{itm:abordagem2}.

Tratar-se-á de detalhes das técnicas aplicadas pelos autores por dois
motivos:

\begin{itemize}
\item a técnica teve bons resultados apesar de utilização de pouca
informação, podendo ter melhores resultados quando alimentada com mais
informação e, por isso, sendo um possível caminho a ser percorrido;
\item os artigos não são de compreensão trivial, em especial para a
elucidação da adaptação do algoritmo de Viterbi\footnote{Para os
leitores que desejarem se aprofundar, também se recomenda a leitura de
\cite{nilm_bergman_distribuido_2011} antes dos artigos de
\citeauthor*{nilm_baranski_genetic_detalhado_2004_20}.}. 
\end{itemize}

Os leitores não interessados em detalhes técnicos podem
seguir para a próxima abordagem mantendo em mente que as contribuições
das técnicas foram: 

\begin{enumerate}
\item criação das \gls{fsm} por algoritmos genéticos
para reduzir o tempo de otimização; 
\item em seguida essas são otimizadas por \gls{es} para obter os
parâmetros da \gls{fsm} (tempo e consumo em cada estado); 
\item e finalmente, os modelos utilizam lógica \emph{fuzzy}, permitindo que
dois ou mais modelos sejam criados para um mesmo distúrbio na rede,
sendo depois escolhido o modelo que melhor se aplica.
\end{enumerate}

A estratégia começa com o agrupamento dos dados em centróides,
limitando-se a degraus acima de um limiar mínimo de potência (valores
aplicados de 50 \acrshort{watt} em
\cite{nilm_baranski_genetic_base_2003_19} e 80 \acrshort{watt}
\cite{nilm_baranski_genetic_detalhado_2004_20}). As abordagens em
\cite{nilm_baranski_genetic_base_2003_19,
nilm_baranski_genetic_detalhado_2004_20} se basearam no agrupamento
utilizando lógica \emph{fuzzy}, mas na última referência, além desse
método, cita-se o emprego de \gls{som} para essa etapa. A fim de
reduzir a complexidade do problema (a estimativa de eventos é de
$16.000$ por dia), o autor desconsidera os centróides com poucas
ocorrências, limitando-se a identificar apenas aparelhos com padrões
recorrentes. 

Na primeira abordagem \cite{nilm_baranski_genetic_base_2003_19},
\citeauthor*{nilm_baranski_genetic_base_2003_19} segmentaram a etapa
de modelar os aparelhos. A primeira modela as \glspl{c3} simplesmente
encontrando pares de centróides no espaço. Para validar os modelos
\glspl{c3} encontrados, é gerado a matriz de correlação cruzada
utilizando o estado de operação para os aparelhos em cada instante de
tempo. Se o aparelho $i$ e o aparelho $j$ forem na realidade uma
\gls{fsm}, espera-se $r_{ij}\approx1$, onde $r_{ij}$ indica a
frequência de operação do aparelho $j$ quando $i$ está operando,
associando, assim, $j$ com a operação $i$ (o corte utilizado é de
0,8). Já para as \gls{c2}, são criados todos os modelos que
juntos somam aproximadamente zero e seus centróides tem frequência de
eventos também próximos. Esses modelos são então validados
temporalmente, e junto com as \glspl{c3} são comparados com um
possível catálogo antigo a fim de atualizá-lo. Em seguida, uma rede
neural é treinada com os padrões encontrados para os aparelhos (o
autor cita como exemplo: tempo médio de consumo, consumo médio, número
de estados) para encontrar esses padrões na residência.

Essa abordagem é aprimorada em
\cite{nilm_baranski_genetic_detalhado_2004_20,nilm_baranski_summary_2004_21},
que ao invés de encontrar todos possíveis modelos de \gls{fsm},
faz a otimização das possíveis máquinas através de algoritmo genético.
Nessa abordagem não há a discriminação para a criação de \gls{c2} ou
\gls{c3}, a abordagem única utiliza $N_{ap}$ (o número de aparelhos
deve ser maior que o número de centróides, no entanto, não é
especificado um bom valor a ser utilizado) \glspl{fsm} para os quais e o
algoritmo genético fica encarregado de alterar valores binários em uma
matriz $\underline{X}$ representando se um determinado centróide
pertence, ou não, à \gls{fsm}. É possível que um mesmo centróide
pertença a mais de uma \gls{fsm}. São utilizados três critérios para
otimização: 

\begin{itemize}
\item minimização do valor absoluto de potência da soma dos
centróides pertencentes a \gls{fsm}; 
\item o item anterior, mas levando em conta a frequência de
eventos em cada centróide; 
\item e minimização do número de centróides em
cada \gls{fsm} (priorizando aparelhos com menos estados).
\end{itemize}

Uma adaptação do algoritmo de Viterbi é utilizada para encontrar os
modelos de \gls{fsm}. Com os modelos resultantes, é criado as
sequências de estado para elas supondo que as mesmas são \gls{c2a}.
Mais precisamente, os autores consideram que as sequências de estados
devem ser recorrentes com seus parâmetros em uma área limitada dentro
de seu valor esperado (os autores citam dois exemplos de parâmetros:
tempo de duração no determinado estado e a capacidade de reconstrução
de energia para o consumo estimado nos estados da \gls{fsm} em relação
ao consumido nos caminhos percorridos; mas não dá detalhes de quais
empregou) e apenas visitados uma vez em cada ciclo. Para isso, é
realizada uma otimização em dois tempos. Primeiro, encontra-se o
melhor caminho para aqueles que obedecem as restrições (consumo de
potência positiva e permanencia em um estado por um tempo não muito
longo), juntando os estados da máquina em um caminho de operação com a
melhor qualidade em relação aos parâmetros escolhidos. Os parâmetros
podem ser iniciados com os valores da mediana para todos os eventos
acoplados a \gls{fsm}. A qualidade é avaliada pela entropia de
\emph{Shannon}, \ref{eq:shannon}. Isso é repetido iterativamente
utilizando um algoritmo de \gls{es} até a convergência da qualidade.

\begin{equation}\label{eq:shannon}
Q_{shannon} = - \Delta{e_{i}} \log{|\Delta{e_{i}}|}
\end{equation}

Os melhores caminhos operativos para as \glspl{fsm} ainda precisam ser
resolvidos quanto aos centróides que pertencem a mais de um aparelho.
Para isso, \cite{nilm_baranski_summary_2004_21} cita resumidamente um
algoritmo de força bruta que irá investigar para cada sobreposição
qual caminho tem a melhor qualidade.

Os autores revelam que o método necessita de 5 a 10 dias para
encontrar os modelos dos utensílios típicos, enquanto dados diários
são suficientes para atualizar o catálogo de utensílios detectados em
cada residência. Os resultados mostram que os aparelhos de maiores
consumo, como geladeira, aquecedor elétrico (de fluxo) e fogão podem
ser detectados com eficiência.

\item \emph{\gls{dnilm}}

A abordagem de \citeauthor*{nilm_baranski_summary_2004_21} é a base
empregada para o trabalho de \citet*{nilm_bergman_distribuido_2011},
contando com a mesma sequência de criação através de algoritmo
genético e otimização das \gls{fsm}. Diferenças podem ser notadas
apenas para a técnica de agrupamento, que não utilizam as informações
de tempo nem o pico atingido no transitório, contudo, podem utilizar a
\gls{q} se o medidor da residência realizar essa medida. Além disso, o
agrupamento é realizado em tempo real, atualizando a média e desvio
padrão de cada centróide conforme os eventos ocorrem. Se o evento não
for atribuído a nenhum centróide, um novo centróide é criado. 

Entretanto a maior contribuição do trabalho
em questão é uma nova arquitetura, distruibuída, para o \gls{nilm}.
Nesse caso, diferente dos medidores eletromecânicos disponíveis para
\citeauthor*{nilm_baranski_summary_2004_21}, o trabalho opera com
medidores inteligentes (os medidores das redes inteligentes descritos
na Subsessão~\ref{ssec:ret_tec}) para a colheita de
dados. Os medidores inteligentes também servem como pontos de
processamento local, mas devido às limitações de processamento, apenas
a deteção e identificação dos eventos é realizado no mesmo. Por isso,
a geração das \glspl{fsm} é realizada em uma central, com maior
capacidade de processamento, no qual este envia os eventos de
transição para que aquela os processe e retorne um catálogo com as
\glspl{fsm} e seus padrões a serem encontrados. O catálogo é chamado
de tabela estática. Assim, o medidor fica encarregado apenas de
comparar, localmente, os distúrbios encontrados com o catálogo,
identificando assim os estados operativos dos aparelhos. O estados
operativos de cada aparelho é chamado de tabela dinâmica, e é
preenchida por um algoritmo adaptado para otimização do problema da
mochila (do inglês \emph{Knapsack problem}). A etapa de criação da
tabela estática, chamada de aprendizagem, é realizada devido a
critérios do controlador (iniciação pró-ativa) ou do medidor
(reativamente). No primeiro caso, o controlador atualiza as tabelas do
medidor se as mesmas expirarem. Já o medidor inteligente pode
requisitar um novo treinamento de acordo com um dos critérios:

\begin{itemize}
\item a diferença absoluta entre a soma da demanda real e estimada
está superior a um patamar;
\item uma \gls{fsm} muda de estado frequentemente, onde os patamares
aplicados para determinar se a mudança de estado é frequente dependem
do aparelho (é mais aceitável observar mudanças frequentes no ar
condicionado ou aquecedor do que em um carregador de bateria
veicular);
\item mais de um determinado número de \glspl{fsm} alteram de estado
em um único evento.
\end{itemize}

Uma das dificuldades do projeto está em ajustar o fluxo de dados. A
referência considera armazenar os dados em períodos de maior atividade
nas residências, enviando as alterações de estado posteriormente
conforme a rede de comunicação não estiver congestionada. Diversas
outras considerações são feitas em relação ao processo de
aprendizagem.

Os resultados reportados são em comparação com um \gls{nilm}
centralizado. O trabalho reservou-se a detectar aparelhos com consumo
superior a 1000 \acrshort{watt}. O \gls{nilm} centralizado recebe uma
tabela estática otimizada para todo o período, enquanto o \gls{dnilm}
recebe uma tabela treinada para o primeiro dia, podendo atualizá-la de
acordo com o critérios anteriormente citados. As diferenças de
acurácia entre o \gls{dnilm} e o \gls{nilm} centralizado ficaram entre
60\% e 90\%.

\item \emph{Otimização da tabela dinâmica}

O problema para a otimização da tabela dinâmica --- construção
temporal dos estados operativos dos aparelhos --- foi abordado por
\citet*{nilm_genetic_2013} (aparentemente sem conhecimento do trabalho
de \citeauthor{nilm_bergman_distribuido_2011}). Ao invés de um
algoritmo de um força bruta adaptado, aplicou-se um algoritmo genético
para realizar a otimização do problema da mochila. O artigo limitou-se
a estudar a performance do algoritmo em resolver o problema da mochila
e, por isso, considerou-se que são conhecidos os momentos de transição
e os consumos de cada aparelho.

A referência utilizou simulações de 2 horas, gerando aleatoriamente o
tempo de operação dos aparelhos e suas potências. Foram simuladas
diversas condições, variando o número de aparelhos, transições, a
presença de ruído e de equipamentos desconhecidos. As observações
foram: 

\begin{itemize}
\item quanto maior o \gls{nt} e \gls{nap}, menor é a eficiência de
detecção. Casos com pequenos números de \gls{nt} e \gls{nap} obteram
100\% de eficiência de detecção;
\item o algoritmo se comportou bem na presença de ruído, não sendo
bastante afetado;
\item no entanto, a presença de aparelhos desconhecidos ou aparelhos
com sobreposição de potências deterioram a performance do algoritmo.
\end{itemize}

\end{itemize}

\end{enumerate}

\end{enumerate}

\subsubsection{2. Medição com Alta Amostragem}
\label{top:nilm_alta_am}

Taxas ainda maiores de amostragem podem ser utilizadas, permitindo a utilização
de outras técnicas. A estratégia em
\cite{nilm_laughman_continuous_variables_2003_9} envolvem encontrar
transientes e subtrair as ondas pré e pós transiente de corrente em
regimente permantente. Aplica-se a Transformada de Fourier e
compara-se com um banco de dados. 

Utilizou-se em \cite{nilm_coppe_nascimento} uma amostragem de 13,6
\acrshort{hz}. Nesse trabalho se utiliza uma estrutura de árvore de
decisão para a discriminação. Uma pré-classificação é realizada
utilizando o \emph{hardware} para equipamentos com valores específicos
de \gls{p} e \gls{q}. Caso não seja um dos equipamentos de simples
discriminação, modifica-se a informação em uma série de etapas, nesta
ordem: aplica-se Transformada de Hilbert, Transformada Wavelet e por
último o Método de Burg. As características utilizadas são os picos 
para os níveis de detalhes do Método de Burg adicionados da \gls{fp},
esse último adicionado para melhorar a capacidade de discriminação.

A informação é apresentada a um
classificador que utiliza como padrão o vizinho mais próximo no banco
de dados, que realiza a discriminação em duas etapas, primeiro
selecionando um grupo genérico ao qual o equipamento pertence, para
depois selecionar o grupo específico. A seleção é feita através do
erro médio quadrático aplicado entre o banco de dados e o resultado do
espectro de Burg para cada nível de detalhe utilizado da Transformada
Wavelet. O equipamento do banco de dados que tiver maior número de
detalhes compatível com o sendo testado é o resultado do processo de
discriminação.

% \item multiplas tecnicas

% Colocar o zeifman et all 2011 Nonintrusive appliance load monitoring
% (NIALM) for energy control in residential buildings e o liang aqui.

\subsection{Discussão}
\label{ssec:nilm_mundo_padroes}

% Falar das c1 e c4 que só podem ser detectadas utilizando

% Ver patel smart grid sensing

A combinação de múltiplas técnicas para o reconhecimento de padrões é
benéfico para a eficácia do discriminador. Essa abordagem requer mais
um subpasso para combinar as respostas dos diversos discriminadores
para produzir uma resposta única --- questão conhecida como fusão de
\emph{fusão de informação}. Esse passo é discutido em
\cite{nilm_liang_pt1_2010_34}, nomeando o processo de \gls{cdm}, no
qual foram testadas as seguintes abordagens:

\begin{description}
\item \gls{mco}: escolha do candidato mais comum entre os
membros da comissão. É o processo mais trivial de ser executada
computacionalmente, realizando apenas a contagem de votos. Essa
abordagem pode criar soluções não-únicas devido ao empate na votação;
\item \gls{lur}:
\end{description}

\cite{nilm_zeifman_review_2011} levanta a hipótese da utilização de
técnicas adaptando a teoria de \emph{Dempster-Shafer} para realizar
tal tarefa, e cita \cite{information_fusion_basir_2007_40} como
exemplo. 
\cite{nilm_zeifman_review_2011} 

\section[A metodologia no CEPEL]{A metodologia no \acrshort{cepel}}
\label{sec:nilm_cepel}

Os estudos anteriores realizados no \gls{cepel}
\cite{nilm_cepel_alvaro,nilm_cepel_bezerra,nilm_cepel_aguiar}
utilizaram uma amostragem de 60 \acrshort{hz}, podendo ser considerada
como uma taxa de amostragem intermediária. Ela permite a extração 
da informação da onda causada pelo distúrbio transitório causado pelos
utensílios na rede, assim como as características macroscópicas já
detalhadas em \ref{top:nilm_baixa_am}. 

As primeiras abordagens se concentraram em explorar a informação da
onda envoltória da corrente --- e apenas da corrente --- que é
propagada para uma \gls{rna} no intuíto de diferenciar os equipamentos
em grupos pré-determinados de consumo. Os aparelhos eram agrupados
conforme a similiridade de suas envoltórias. A última versão da
\gls{rna} utilizou estes grupos:

% TODO Colocar os grupos do alvaro

Nesses trabalhos o medidor foi desenvolvido pelo próprio \gls{cepel},
de forma que \cite{nilm_cepel_aguiar} explorou, além dos resultados
da \gls{rna}, os aspectos necessários para manter a
qualidade da informação e compactá-la utilizando técnicas como
\gls{pca}. Já \cite{nilm_cepel_bezerra} forneceu ao sistema a
possibilidade de uma arquitetura distribuída de processamento, similar
àquela explicada em \cite{nilm_bergman_distribuido_2011}. Nesse
trabalho também foi analisado o uso de \gls{pcd} em comparação ao
pré-processamento por \gls{pca}. Esses estudos apenas englobaram o
caso de colheita de dados em residências monofásicas.

Em \cite{nilm_cepel_alvaro}, além da continuação do desenvolvimento da
\gls{rna}, implementou-se um detector automático de eventos de
ligamentos de equipamentos, sem generalizar para o caso de
desligamentos ou estudar a performance para mudança de estados em
aparelhos do tipo \gls{c2}. Esse estudo também foi limitado para
redes monofásicas. Foram comparados três detectores de eventos
transitórios diferentes, sendo eles:

% TODO Colocar técnicas utilizadas pelo alvaro para detecção de
% ligamentos

Devido à boa capacidade de identificação dos equipamento nos grupos de
cargas, essas referências utilizaram apenas a informação de
transitório. No entanto, a utilização do medidor do \gls{cepel}
começou a ser questionada uma vez que o seu desenvolvimento é mais um
fator a ser considerado no projeto, em especial quando expandindo o
mesmo para o caso trifásico.




  \chapter{Ambiente de Análise}
\label{chap:framework}

Com base nas necessidades do projeto
(Sessão~\ref{sec:motivacao_framework}) foi realizado o desenvolvimento
de um ambiente (\emph{framework}) de análise. Seus módulos são:
Leitura, Representação e Interação com os Dados
(Sessão~\ref{sec:daq_info}); Interação Gráfica com o Usuário
(Sessão~\ref{sec:gui}); Análise dos Dados (Sessão~\ref{sec:analise});
e Otimização dos Parâmetros (Sessão~\ref{sec:otimizacao}).


\section{Da Necessidade}
\label{sec:motivacao_framework}

No capítulo anterior foram observados diversos aspectos envolvidos
para o densevolvimento da tecnologia do \gls{nilm} e a configuração do
projeto no \gls{cepel}. Com base nisso, as seguintes dificuldades
podem ser destacadas:

\begin{enumerate}[label={Item} \arabic* - ,ref=\arabic*,align=left]
\item\label{itm:dif1} havia uma necessidade de trazer para o
\gls{cepel} uma gama de possibilidades e caminhos a serem tomados,
pareando o projeto com a informação fornecida atualmente na
literatura;
\item\label{itm:dif2} mesmo para metodologia adotada não era possível
ter uma boa interpretação do comportamento do algoritmo e como
escolher os valores de corte, sendo necessário avaliar caso a caso os
valores testados em busca de uma configuração que traria melhores
resultados;
\item\label{itm:dif3} apesar de haver uma tendência para simplificar o
projeto para obter uma resposta em tempo de projeto menor, não se sabe
exatamente qual estratégia será seguida, incluindo a técnica empregada
para discriminação, frequência de amostragem (e nesse caso, havendo
modificação da mesma, seria necessário alterar/adaptar a técnica atual
para detecção de eventos) e o medidor;
\item\label{itm:dif4} a larga gama de técnicas encontradas no
levantamento bibliográfico, e em especial a chamada de atenção para o
fato de sua utilização em paralelo ser benéfica para a capacidade de
desagregação do \gls{nilm}, mostra que o projeto deve ter aptidão de
agregar em um único ambiente tudo aquilo que for desenvolvido, pois
mesmo que uma técnica não seja ótima, sua operação em paralelo pode
ser benéfica para o sistema de desagregação como um todo;
\item\label{itm:dif5} ainda que o item anterior não seja de interesse,
é importante manter todas as abordagens já desenvolvidas em um único
ambiente para garantir uma melhor evolução do projeto;
\item\label{itm:dif6} o levantamento bibliográfico mostrou que há uma
dificuldade dos autores em obter dados onde a informação desagregada
em energia também esteja disponível\footnote{Para fugir dessa
dificuldade, \cite{nilm_liang_pt2_2010_40} chegou a criar um simulador
de Monte-Carlo (ver p.~\pageref{nilm:multiplas_tecnicas}).}. Tem-se uma
necessidade de tanto obter um meio para armazenar os momentos de
transição dos estados operativos para treinar (se existentes) técnicas
supervisionados, quanto ter o consumo desagregado para avaliar a
eficiência do \gls{nilm}. No exterior disponibilizaram-se dois
conjuntos de dados (ver Subsessão~\ref{top:nilm_padrao}) justamente
com o intuíto de facilitar os autores de terem essa informação e
compararem suas técnicas.  Para obter a informação desagregada, pelo
menos uma das informações seguintes deve estar disponível, no entanto,
as duas se complementam, sendo desejável trabalhar com ambas: as
marcas caracterizando os momentos de alteração de estados;
e informação de consumo temporal dos aparelhos através de submedidores.
Porém, nem sempre é possível obter as duas informações juntamente,
devido a dificuldades como o fato não ser fácil monitorar os estados
operativos de alguns aparelhos por não ter acesso nem controle de seus
ciclos de maneira trivial, bem como, nem sempre ser possível realizar a
submedição de todos aparelhos, estando essa informação parcialmente ou
até mesmo não disponível.
\end{enumerate}

Exceto pelo Item~\ref{itm:dif1}, que foi resolvido pelo levante
realizado na Sessão~\ref{sec:nilm_mundo}, viu-se a necessidade do
desenvolvimento de um ambiente de análise que atendesse os seguintes
pontos:

\begin{enumerate}
\item Maleabilidade: em vista dos itens
\ref{itm:dif3}--\ref{itm:dif5}, o ambiente deve ser capaz de se
adequar às mudanças no projeto conforme elas ocorram sem que outras
partes do ambiente sejam afetadas. Devido a isso, optou-se por uma
implementação orientada a objeto, mas se limitando a escolha para uma
linguagem de amplo conhecimento no campo da engenharia. Por isso,
optou-se por desenvolver o ambiente no \emph{Matlab}, que
disponibiliza desde sua versão \emph{R2008a} essa capacidade.
Ainda que o \emph{Matlab}, e em especial sua linguagem orientada a
objeto, sofram em relação a sua performance, o ambiente de análise é
\emph{a posteriori} à coleta de dados, sendo aceitável essa
desvantagem. Está apenas interessado em saber como as técnicas irão se
portar antes de implementá-las para operação em tempo real. O ambiente
foi organizado procurando modularizar na medida do possível os
componentes, para que, se fosse necessário o desenvolvimento ou
adaptação de código, simplificasse o processo para que apenas o módulo
em questão seja ser atacado;

\item Capacidade de Interpretação dos Dados e Resultados: o tempo
investido neste ponto retorna em capacidade de interpretação, de forma
que o projeto irá ter um melhor andamento. Um dos aspectos para
atingir isso, é através de uma boa visualização \cite{it_depends} já
que a mesma é um meio bastante efetivo para a comunicação da
informação (no caso se referindo a informação presente nos dados,
análise etc.). No caso, uma visualização dinâmica permite ainda melhor
compreensão das nuances contidas na informação. Uma outra maneira é
através de automatizar as tarefas de modo a obter os resultados de
maneira agregada e relevante, facilitando a comparação. Este ponto
atende o Item~\ref{itm:dif2};

\item Otimização dos Parâmetros: também considerando o
Item~\ref{itm:dif2}, seria interessante obter configurações ótimas de
maneira automática, sem a necessidade do usuário ficar alterando
parametros empiricamente até obter um valor considerado bom.
Inclusive, que o algoritmo seja capaz de realizar isso encontrando uma
das melhores configurações possíveis para os parâmetros (não
necessariamente a melhor);

\item Estimação da Informação a ser Desagregada pelos Algoritmos: já
quanto ao Item~\ref{itm:dif6}, os dados disponíveis no \acs{cepel} não
foram amostrados com submedição, tendo apenas acesso às marcas de
mudanças de estado operativo e, por isso, é necessário uma maneira para
estimar a informação desagregada contida nos mesmos. Com esse intuíto,
aproveitou-se o ponto ``Capacidade de Interprezação dos Dados e
Resultados'', e adicionou-se essa capacidade na
visualização dinâmica oferecida ao usuário. Essa estimativa da
informação desagregada será referida neste trabalho como
\emph{gabarito}.

\end{enumerate}

O resultado foi um ambiente de análise com $\sim$29.000 linhas de
código distribuidas em $\sim$190 arquivos. Um esboço de sua
arquitetura pode ser observado na Figura~\ref{fig:ambiente_analise}.
As palavras em inglês representam as classes mais importantes dos
módulos como referidas no ambiente. Observa-se que o Módulo de
Leitura, Representação e Interação com os Dados
(Sessão~\ref{sec:daq_info}) é a base do ambiente, sendo utilizado
para análise e otimização. O Módulo de Análise dos Dados
(Sessão~\ref{sec:analise}) é executado pelo Módulo de Otimização dos
Parâmetros (Sessão~\ref{sec:otimizacao}), que realiza diversas
análises iterativamente em busca de uma configuração ótima para os
dados sendo alimentados. 

\begin{figure}[h!t]
\centering
\includegraphics[width=\textwidth]
{imagens/ambiente_de_analise.pdf}
\caption{Esboço do ambiente de análise implementado.}
\label{fig:ambiente_analise}
\end{figure}

É importante notar que apesar do esboço mostrar toda a cadeia para a
obter os parâmetros otimizados, essa não é sua única operação. O
usuário irá determinar sua operação, ex.: seja só para realizar uma
analise obter os resultados e explorar graficamente a sua resposta,
construir um gabarito para um novo conjunto de dados, explorar os
dados a procura de alguma informação etc.

Um detalhe, a implementação foi realizada em inglês por opção do autor
do trabalho com o intuíto de que o programa também seja compreensível
no exterior. Nem todas as informações puderam ser traduzidas para
colocá-las no trabalho, nesses casos será realizado a tradução e
referência aos elementos no texto do trabalho.

\section{Leitura, Representação e Interação com os Dados}
\label{sec:daq_info}

O Módulo de Leitura, Representação e Interação com os dados é o mais
complexo em termos de estrutura no ambiente implementado. Ele conta
com os seguintes segmentos:

\begin{itemize}
\item Dados do Medidor (Subsessão~\ref{ssec:dados_medidor}):
representação em memória transitória dos dados do medidor. Uma série
de aspectos tiveram de ser tratados neste segmento;
\item Evento de Transitório (Subsessão~\ref{ssec:evento}): contém a informação de um
eventos de transitório. Essa representação pode ser criada tanto pelo
usuário durante a criação de um gabarito, ou seja, informar um evento
de transitório e suas propriedades a serem encontradas para avaliar a
performance de análise ou otimizar os parâmetros baseando-se nessa
informação, ou quanto pelo Módulo de Análise, que irá gerar essa
informação através de sua metodologia;
\item Aparelhos (Subsessão~\ref{ssec:aparelho}): contém a informação
do estado de consumo dos aparelhos, seus consumo temporal estimado bem
como a estimativa de seu consumo total para o conjunto de dados.
Apesar deste trabalho ainda não ter tratado do problema de geração da
informação dos aparelhos (e por isso essa informação só ser gerada
pelo usuário para o gabarito), é interessante em termos de
continuidade do projeto que essa informação já fosse gerada nos
gabaritos, para que eles não precisem ser revisados no futuro,
contando com toda informação necessária para a otimização e avaliação
de performance. Como foi visto na Subsessão~\ref{ssec:nilm_eff_calc} e
frisado diversas vezes em tal capítulo, é necessário dar a eficiência
do \gls{nilm} em termos de energia. Outro aspecto importante para a
motivação da criação dessa informação foi da capacidade de compreensão
dos dados, a informação por aparelho é muito mais intuítiva que os
eventos de transitório, constituindo em um nível mais alto informativo
para a compreensão dos dados, bem como facilitando a geração do
gabarito;
\end{itemize}

A seguir, entrar-se-á em mais detalhes para cada um deles.

\subsection{Dados do Medidor}
\label{ssec:dados_medidor}

Para atender as necessidades do projeto, a implementação da interface
para leitura e representação dos dados do medidor abordou os seguintes
tópicos:

\begin{itemize}
\item Transformação dos dados em um formato único: atualmente o
\gls{nilm} utiliza dados de dois medidores diferentes, sendo
necessário representar essa informação de uma única maneira para
atender a questão de Maleabilidade. Um efeito colateral decorrente da
transformação foi a compressão dos dados, que estavam em formatos de
texto e ao serem armazenados em formato binário sofreram compressões de
30$\times$ a 40$\times$ dependendo do número de fases;
\item Robustez: a leitura e transformação dos dados para o formato
único deve ser robusta a possíveis erros durante a aquisição de dados,
sendo eles: descontinuidade da amostragem, seja por intervenção humana
ou algum problema no medidor; ou sobrecarga devido ao consumo
excessivo na rede, geralmente causado pelo acionamento de um aparelho
a motor de maior consumo, como o ar condicionado. Para o primeiro
caso, implementou-se um algoritmo capaz de identificar esses momentos,
e no caso da descontinuidade ser pequena (ex. 10~s, determinado pelo
usuário), a informação entre as bordas dos arquivos é completada com
amostras geradas através de um ajuste linear. Essas amostras são
marcadas para identificar que foram criadas e não medidas. Enquanto
para a sobrecarga, é grampeado o valor de consumo máximo para as
variáveis em que isso ocorre, bem como as amostras são marcas para
identificar os momentos em que isso ocorre;
\item Segmentação da memória persistente: alguns dados contém dias de
amostragens, sendo impossível analisar toda essa informação de uma vez
só em memória transitória. Por isso, segmentou-se os dados em diversos
arquivos com um tamanho pré-definido (ex. 1~hora). A abordagem do
\acs{cepel}, que teria de segmentar a informação manualmente. No
entanto, a leitura da base de dados deve ser transparente, sem que o
usuário precise se preocupar em como o conjunto de dados esteja
representado e compreendido pelo ambiente;
\item Redução da necessidade de leitura de disco: devido a segmentação
em memória, era necessário garantir que informações nas bordas dos
arquivos estivessem disponíveis para os algoritmos de análise sem que
eles tivessem de requisitar a troca da informação mantida em memória
transitória. Para isso, uma quantidade de amostras nas bordas dos
arquivos segmentados é mantida em memória transitória sempre
disponível, evitando que seja necessário uma navegação excessiva entre
a informação segmentada, reduzindo drasticamente a velocidade dos
algoritmos já que a leitura em disco é lenta;
\item Leitura de redes elétricas com até três fases: era necessário
compatibilidade de leitura de dados de redes monofásicas, bifásicas e
trifásicas, representando essa informação de uma maneira universal. Um
dos fatores que influênciou também na compressão dos dados foi armazenar
para as fases com pouca atividade somente os momentos em que havia
consumo;
\item Informação gráfica: representar graficamente a informação contida
nos dados. Essa funcionalidade é utilizada como base pela interface
gráfica para realizar a interação com os dados.
\end{itemize}

Um exemplo de dados trifásicos em uma residência
\emph{real}\footnote{A palavra real é empregada para identificar
amostragens não geradas em condições de laboratório.} está
disponível na Figura~\ref{fig:casa_real}. As três primeiras subfiguras
são a injeção de corrente (medido em valor eficaz) no sistema,
enquanto a última figura é o fluxo em potência trifásico para as
variáveis descritas na p.~\pageref{eq:ipqds}. O fluxo de potência é
informado para o consumo trifásico porque o medidor \emph{Yokogawa}
utilizado nessa coleta de dados só permite o acesso a essa informação,
sendo necessário operar com um nível mais agregado de consumo que a
corrente. As barras verticais cinzas indicam como está realizada a
segmentação dos dados em disco. Apesar de não se utilizar esse
conjunto de dados para análise --- nesses dados não há como construir
o gabarito, não sendo possível a otimização dos valores, nem calcular
sua eficiência ---, ele revela uma série de aspectos importantes para
a compreensão do problema envolvido na desagregação do \gls{nilm}, bem
como algumas necessidades durante a implementação da parte de
representação dos dados no ambiente de análise. 

%\begin{landscape}
%\begin{figure}[h!p]
\begin{sidewaysfigure}[p]
\centering
\includegraphics[width=\textwidth]{imagens/RealHouse.pdf}
\caption[Informação gráfica para o interação com dados do medidor]
{Informação gráfica para a interação com os dados do medidor. Gráfico
gerado através do ambiente de análise para um conjunto de dados com
amostragem em 60~\acs{hz} de uma rede trifásica em uma casa
\emph{real} durante aproximadamente um dia de coleta. A injeção de
corrente para cada uma das três fases encontra-se nas subfiguras
superiores, enquanto o fluxo trifásico de potência entrando na rede
elétrica é representado na subfigura inferior. São utilizados as cores
azul, vermelho, verde e preto para as potências ativa, reativa,
harmônica e aparente, respectivamente.}
\label{fig:casa_real}
\end{sidewaysfigure}
%\end{figure}
%\end{landscape}

Quanto a questão da descontinuidade, há uma falha na medição próximo
às 07:15~h do dia~31, mostrando que o algoritmo foi capaz de perceber
essa falha e montar a descontinuidade. Já próximo às 19:20~h, ocorreu
uma outra descontinuidade menor, de 30~s, onde o algoritmo identificou
e uniu as bordas através de um ajuste linear para simular a
continuidade e recuperar a informação perdida. Enquanto isso, o ajuste para a
descontinuidade às 07:15~h não pode ser feito porque houve alterações
de estados operativos dos aparelhos\footnote{Estimar essas
alterações sem nenhuma informação é mais complexo do que a tarefa do
próprio \acs{nilm}, que realiza isso tendo a informação agregada de
consumo.}. Por isso, esse conjunto de dados seria analisado em duas
partes, uma partindo do início até às 07:15~h, e outra do fim da
descontinuidade até o fim da medição.

Também há nesse período a ocorrência de sobrecarga do medidor próximo
às 21:50 do dia 30 (provavelmente causada por um ar condicionado, ver
Figura~\ref{fig:sobrecarga}), onde o algoritmo foi capaz de
identificá-la e alterar os valores dessa amostras para a capacidade
máxima de medição.

Outras condições que se referiu no Capítulo~\ref{cap:nilm} também
podem ser observadas e melhores compreendidas nesse conjunto de dados.
Por exemplo, é possível observar na fase~A
Figura~\ref{fig:c5_ruido}\footnote{Foi necessário reduzir a qualidade
da Figura~\ref{fig:c5_ruido} para permitir a navegação na versão
digital deste trabalho, a figura vetorizada exigia grande capacidade
computacional.}
há a ocorrência de uma \gls{c5} a partir das
20:00~h injetando ruído na rede elétrica devido a sua dinâmica
(provavelmente esse aparelho é uma televisão). Já um exemplo típico de dinâmica causada
pela máquina de lavar roupa se encontra na
Figura~\ref{fig:maquina_lavar}. Apenas como curiosidade, também é
possível observar no período de menor atividade da rede ---
08:00~h--18:00~h do dia 31 --- nitidamente a operação da geladeira (ou
outro aparelho similar) na fase~B.

\begin{figure*}[p!]
  \begin{center}
    \begin{subfigure}[c]{\textwidth}
      \includegraphics[width=\textwidth,height=0.26\textheight]{imagens/RealHouse_ZoomSobrecarga.pdf}
      \caption{Sobrecarga do medidor causado por um equipamento na
        fase A.}
      \label{fig:sobrecarga}
    \end{subfigure}
    \hfill
    \begin{subfigure}[c]{\textwidth}
      \includegraphics[width=\textwidth,height=0.26\textheight]{imagens/RealHouse_maquina_lavar.eps}
      \caption{Um dos estados operativos da máquina de lavar
        roupa.}
      \label{fig:maquina_lavar}
    \end{subfigure}
    \hfill
    \begin{subfigure}[c]{\textwidth}
      \includegraphics[width=\textwidth,height=0.26\textheight]{imagens/RealHouse_aparelho_c5.jpg}
      \caption{Dinâmica de uma carga C5 na fase~A. Observe a diferença
entre a relação de sinal ruído dessa fase em relação com a fase~B.}
      \label{fig:c5_ruido}
    \end{subfigure}
  \end{center}
\caption[Alguns exemplos de dificuldades encontrados nos dados reais]{
Alguns exemplos de dificuldades encontrados nos dados reais da
Figura~\ref{fig:casa_real}.}
\label{fig:dificuldades}
\end{figure*}

%\begin{figure*}[ht!]
%  \ContinuedFloat
%    \begin{subfigure}[c]{\textwidth}
%      \label{fig:maquina_lavar}
%      \caption{Um dos estados operativos da máquina de lavar
%        roupa.}
%      \includegraphics[width=\textwidth,height=0.26\textheight]{imagens/RealHouse_maquina_lavar.eps}
%    \end{subfigure}
%
%  \caption{Casos destacados no conjunto de dados da Figura~\ref{fig:casa_real}.}
%\end{figure*}
%\FloatBarrier


\subsection{Evento de Transitório}
\label{ssec:evento}

Eventos são gerados tanto pelo usuário quando criando o gabarito
ou quanto pelo algoritmo de análise. A informação nos eventos de
transitório são de mais baixo nível que àquelas contidas no objeto que
representa o aparelho. Suas capacidades são: 

\begin{itemize}
\item \acs{fex} para classificação: Realiza o cálculo das variáveis
\acs{di}, \acs{dp}, \acs{dq}, \acs{dd} e \acs{ds} e extrai uma janela
da envoltória dessas variáveis durante o transitório;
\item Remoção de eventos ruidosos: se \acs{di}, \acs{dp} ou \acs{ds}
forem menores a um limiar, o evento é considerado como ruidoso. O
corte pode ser realizado em apenas uma dessas variáveis, no momento o
corte é apenas em \acs{di}.
\item \textlabel{Remoção de eventos próximos}{text:media}: Para a remoção de eventos próximos
adicionou-se uma nova maneira de realizar a mesma substituindo os
eventos próximos pela média dentro da janela.
\item \textlabel{Remoção de eventos incosistentes}{text:incosistentes}: 
ao observar que grande parte
dos eventos que eram removidos por serem eventos próximos na verdade
eram causados por eventos criados após um pico de consumo devido ao
acionamento de um aparelho, que gerava um evento na descita após o
pico, decidiu-se adicionar um novo tipo de corte. Esse corte remove os
candidatos que tiverem o sinal da resposta do filtro de derivada de
Gaussiana invertido em relação ao degrau de consumo causado pelo
evento. Para esses eventos citados, a resposta do filtro é negativa,
enquanto o degrau é positivo, havendo assim incosistência entre os
dois;
\item \textlabel{Estado do eventos}{text:estados_eventos}: para
melhorar a capacidade de análise, eventos removidos não são excluidos,
tendo apenas seus estados alterados. Os possíves estados dos eventos
são:
\begin{itemize}
\item Em bom estado;
\item Removido devido a evento próximo, nesse caso indicando qual
evento que causou sua remoção;
\item Evento ruidoso;
\item Inconsistente;
\item Quantidade de amostras insuficientes, se não houver amostras
suficientes para construir o evento;
\item Ainda não preenchido, quando o evento é criado mas ainda é
necessário preencher a \gls{fex} e realizar os cortes para determinar
se ele é um evento.
\end{itemize}
\item Mudança de estado e aparelho: os eventos também armazenam a
informação de qual aparelho pertencem e qual foi a mudança de estado
por eles representadas. Por enquanto essa informação só é preenchida
pelo usuário durante a criação do gabarito;
\item Informação gráfica: os eventos são representados conforme o seu
estado (observe exemplos na Figura~\ref{fig:analise_eventos}).
São utilizadas linhas verdes verticais para indicar eventos de
transitório onde houve acréscimo no consumo, e linhas vermelhas para
decréscimo.  Eventos removidos possuem linhas cinza tracejadas.
Dependendo de como a geração gráfica é realizada, será criado uma área
cinza para a região onde será realizada a extração da envoltória e
duas faixas amarelas indicando as amostras para as quais será
calculado o valor pré/pós-transitório utilizados para calcular o
degrau, como indicado nas equações
\ref{eq:deltasMacro}.
\end{itemize}


\subsection{Aparelho}
\label{ssec:aparelho}

A informação contida no aparelho une toda aquela contida nos eventos.
Nela, se reconstrói todos os estados operativos do aparelho
temporalmente e seus consumos. No momento, essa informação só é gerada
pelo usuário quando preenchendo a informação do gabarito. Para ter uma
ideia de como o processo é realizado observe a
Figura~\ref{fig:gui_informacao}. As
capacidades desse elemento são:

\begin{itemize}
\item Detecção automática de estados: quando criando o gabarito,
faz-se uma análise de dendograma para agregar os eventos de
transitório com informações de \acs{di}, \acs{dp}, \acs{dq}, \acs{ds}
próximas. O usuário só necessita alterar o nome dos estados pré/pós
transitório dos eventos agrupados, simplificando o processo de geração
do gabarito;
\item Informação gráfica: a capacidade de geração de informação dos
aparelhos é bem mais extensa que a dos eventos. Para evitar
redundância de informação, refere-se diretamente as figuras onde são
mostradas as características dos dados que foram gerados através do
método de informação gráfica de cada aparelho. Estes são os possíveis 
gráficos de serem gerados são:
%\begin{itemize}
%\item
%\item gráfico do consumo temporal por aparelho para os dados,
%Figura~\ref{}. Esse gráfico representa a energia desagregada estimada
%a ser encontrada;
%\item gráfico circular de consumo dos aparelhos, Figura~\ref{}
%(p.~\pageref{});
%\item gráfico das envoltórias para todos os eventos de transitório,
%Figura~\ref{} (p.~\pageref{}). Essa informação auxilia a encontrar
%eventos de trasitório que anormais.
%\end{itemize}
\end{itemize}

\section{Interação Gráfica com o Usuário}
\label{sec:gui}

A interação gráfica com o usuário oferece uma melhor compreensão dos
dados. Além disso, este módulo também permite a capacidade de geração
do gabarito onde está toda informação considerada como alvo para o
\gls{nilm}, desde os momentos onde ocorreu os transitórios, até a
estimativa de informação de energia. Por enquanto o módulo só operava
com a informação dos dados do medidor, contudo sua expansão para
realizar a interação com a informação de análise não é complexa e
pretende-se realizar sua implementação no futuro. A seguir estão
suas capacidades:

\begin{figure*}[p]
  \begin{center}
    \begin{subfigure}[c]{\textwidth}
      \includegraphics[width=\textwidth]{imagens/Temporizado_gui_evento_sobreposto.png}
      \caption{Evento com sobreposição: a região com as amostras para o cálculo da
média de pós transitório está sobrepondo com outro evento de
transitório.}
      \label{fig:gui_evento_sobreposto}
    \end{subfigure}
    \hfill
    \begin{subfigure}[c]{\textwidth}
      \includegraphics[width=\textwidth]{imagens/Temporizado_gui_evento_sobreposto_consertado.png}
      \caption{Ao arrastar a região com o ponteiro a sobreposição foi
corrigida, resultando em uma estimativa de consumo mais fiel.}
      \label{fig:gui_evento_sobreposto_corrigido}
    \end{subfigure}
  \end{center}
\caption[Informação gráfica para o Módulo de Interação Gráfica com os
Dados: Eventos de Transitório.]{Informação gráfica para o Módulo de Interação Gráfica com os
Dados: Eventos de Transitório. A barra vertical preta indica a amostra
mais próxima ao ponteiro. }
\label{fig:gui_evento}
\end{figure*}

\begin{itemize}
\item Informação da amostra próxima ao ponteiro: funciona de maneira
muito similar à um medidor, mostrando os valores das amostras \acs{i}
(essa para cada fase), \acs{p}, \acs{q}, \acs{d}, \acs{s} para a
amostra mais próxima ao ponteiro (ver
Figura~\ref{fig:gui_evento_sobreposto}, na região superior à esquerda
denominada de \emph{Cursor Info}, ou informação do ponteiro em
português). Assim, o usuário pode comparar os
valores em cada amostra com facilidade;
\item Geração do gabarito: a dinâmica para a geração do gabarito
ocorre da seguinte maneira: o usuário
seleciona a opção ``Adicionar Evento'' (na figura estando em ingles:
\emph{Add Event}) e então seleciona a amostra que deseja ser o centro
do transitório. Nessa amostra é criado um evento, onde são calculados
os \acs{di}, \acs{dp}, \acs{dq}, \acs{dd}, \acs{ds}. Essa informação é
disponibilizada já calculada para o usuário no canto inferior da
esquerda, contendo tanto o valor do patamar operativo do 
pré/pós-transitório, quando essas variáveis. O algoritmo
detecta e agrupa os possíveis estados automaticamente por uma análise
em dendograma. Conforme o usuário vai preenchendo a informação do
gabarito, a interface gráfica mostra-lhe o consumo desagregado por
equipamento e mantém-na atualizada para cada alteração realizada,
possibilitando o usuário saber se o gabarito está sendo preenchido de
maneira correta ou não. Cabe ao usuário determinar o nome do aparelho
e o nome de seus estados. Para aparelhos \acs{c3} a interface gráfica
já os cria automaticamente com o estados \emph{on} (ligado) e
\emph{off} (desligado). É possível selecionar qualquer evento e
alterar suas características, bem como arrastar as regiões aonde são
calculadas as variáveis (ver
Figura~\ref{fig:gui_evento_sobreposto_corrigido}) para corrigir
possíveis sobreposições de eventos ou regiões inicialmente mal
selecionadas.
\item Escolha da informação disponível: há a opção de escolher a
informação disponível na tela (observar
figuras~\ref{fig:temporizado_gui_eventos} e
\ref{fig:temporizado_gui_legenda}),
podendo controlar a disponibilidade gráfica das seguintes informações:
informação de consumo temporal dos aparelhos; centros dos eventos de
transitório (no caso, dos eventos criados pelo usuário no arquivo em
questão); e consumo amostrado pelo medidor. Isso é realizado porque a
junção de toda essa informação de uma só vez acaba criando confusão e
dificuldade para a sua interpretação, assim, com essas opções o
usuário tem controle sobre elas, permitindo que haja comparação das
informações sem que haja sobreposição delas;
\item Armazenar e carregar arquivos: como o gabarito na verdade é uma
estimativa da informação desagregada, é possível gerar diversos
gabaritos para um mesmo conjunto de dados. Por isso, há um mecânismo de
proteção da memória persistente e transitória. Ou seja, enquanto o
usuário altera a informação do gabarito, o mesmo permanece intacto em
disco, assim como se o usuário tentar encerrar a interface gráfica ou
carregar um outro gabarito e houver dessincronia entre a informação
na memória persistente e transitória, ele será perguntado se deseja
descartar suas alterações ou sincronizar a informação;
\item Janela de legenda para o consumo dos aparelhos: a informação do
consumo temporal possui uma cor única para cada aparelho, porém
adicionar essa legenda na própria figura com os dados do medidor e
informação do gabarito não era possível sem gerar dificuldade para sua
interpretação, bem como não haveria espaço suficiente para colocar
todos os nomes dos aparelhos (espera-se cerca de 30 a 50 aparelhos nas
residências) na tabela sem comprometer a figura. Por isso, o usuário
tem a opção de abrir uma janela que irá conter essa legenda
(Figura~\ref{fig:temporizado_gui_legenda}), que será atualizada
automaticamente enquanto o usuário preenche o gabarito.
\end{itemize}

\begin{figure*}[p!]
  \begin{center}
    \begin{subfigure}[c]{\textwidth}
      \includegraphics[width=\textwidth]{imagens/Temporizado_gui_eventos.png}
      \caption{Informação disponível ao selecionar amostragem do
medidor e eventos de transitório para o conjunto de dados \emph{Temporizado}.}
      \label{fig:temporizado_gui_eventos}
    \end{subfigure}
    \hfill
    \begin{subfigure}[c]{\textwidth}
      \includegraphics[width=\textwidth]{imagens/Temporizado_gui_legenda.png}
      \caption{Informação disponível ao selecionar a informação
estimada de consumo desagregado por aparelho e a janela de legenda
para o conjunto de dados \emph{Temporizado}.}
      \label{fig:temporizado_gui_legenda}
    \end{subfigure}
  \end{center}
\caption{Informação gráfica para o Módulo de Interação Gráfica com os
Dados: Disposição da informação.}
\label{fig:gui_informacao} \end{figure*}



\section{Análise dos Dados}
\label{sec:analise}

O Módulo de Análise dos Dados foi implementado para refletir o ponto
inicial da metodologia empregada pelo \acs{cepel}, mas adaptando o
mesmo para o ambiente de análise, corrigindo possíveis pequenos
equívocos no código original e expandindo o mesmo. Suas aptidões
são:

\begin{itemize}
\item Continuidade da análise: a versão de ponto de partida gerava
um evento na leitura de cada novo arquivo. Isso ocorria porque o
consumo inicial em cada arquivo raramente é zero, e a inicialização
das condições do filtro pelo \emph{Matlab} é realizada considerando
que ele estava em repouso recebendo entradas nulas e respondendo
valores também nulos. Para dar a continuidade entre as segmentações
dos dados o próprio \emph{Matlab} fornece o vetor de condições finais
$\underline{z}_f$ no final da leitura de um arquivo para ser aplicado
na leitura do próximo arquivo como condições inicias
$\underline{z}_i$. Porém, no caso do início do arquivo e nas possíveis
reinicializações da análise devido a descontinuidade das amostras, era
necessário determinar como simular a condição onde o filtro estava em
repouso (com respostas nulas em estado permanente) para a entrada com
o valor da amostra inicial. Seja, assim, a determinação da resposta
$y(m)$ para um \acs{fir} dada por \ref{eq:y_m_fir}
\cite[p.~311-312]{oppenheim}\footnote{Referência utilizada pelo
\emph{Matlab} na implementação do \acs{fir}.}, onde $b(k)$ é o k-ésimo
coeficiente do filtro de ordem $n-1$. Deseja-se encontrar o
vetor $\underline{z}_i$ que, dado o vetor de amostras de entrada
$\underline{x}$, irá gerar uma resposta $\underline{y}$ constante e
nula para toda janela do filtro. Ao representar \ref{eq:y_m_fir} em
forma matricial e igualando $\underline{y}=0$, obtém-se
\ref{eq:matrix_fir}. No caso, quer-se simular que o filtro estava em
repouso para a primeira amostra do arquivo, bastando fazer
$\underline{x}=a_1$, onde $a_1$ é a primeira amostra do conjunto de
dados ou a primeira amostra após uma descontinuidade. Assim,
utiliza-se \ref{eq:matrix_fir} com esse valor para $\underline{x}$;

\begin{subequations}
\begin{eqnarray}\label{eq:y_m_fir}
y(m) = b(1)x(m)+z_1(m-1)  \nonumber \\
z_1(m) = b(2)x(m)+z_2(m-1)  \nonumber \\
\;\;\vdots\;\;\;\;\;\; =
\;\;\;\;\;\;\;\vdots\;\;\;\;\;+\;\;\;\;\;\;\vdots\;\;\;\;\;\;\;  \\
z_{n-2}(m) = b(n-1)x(m)+z_{n-1}(m-1)  \nonumber \\
z_{n-1}(m) = b(n)x(m) \nonumber
\end{eqnarray}
\begin{equation} \label{eq:matrix_fir}
\underline{z} = 
\underbrace{\begin{bmatrix}
b(n-1) & b(n-2)   & b(n-3) & \dots & b(1) \\
       & b(n-2)   & b(n-3) & \dots & b(1) \\
       &          & b(n-3) & \dots & b(1) \\
       &\mathbf{0}&        & \ddots & \vdots \\
       &          &        &        & b(1) \\
\end{bmatrix}}_{\mathbf{B}'}\underline{x}
\end{equation}
\end{subequations}

\item Pareamento da resposta do \acs{fir} com os dados: como a análise
é \emph{a posteriori}, para facilitar a interpretação, remove-se o
atraso na resposta do \acs{fir} para que a mesma fique alinhada com a
amostra sendo analisada, facilitando a compreensão do problema;

\item Janela do filtro com apenas com valores relevantes: o tamanho da
janela do filtro tem influência direta no tempo de execução do
algoritmo, uma vez que a convolução será realizada com uma janela
maior de pontos para cada uma das amostras analisadas. Diminuir a
janela em duas amostras significa que para cada amostra no arquivo
serão realizados pelo menos dois cálculos a menos. Assim, além de dar
um valor limite para o tamanho da janela do filtro, é necessário podar
o mesmo para conter apenas valores relevantes, desconsiderando pontos
com grandeza irrelevante. Para isso, se realiza um corte para pontos
com ordem de grandeza $10^{-3}$ menores que o ponto de maior
relevância do filtro. Apenas com esse corte já foi possível obter
grandes reduções no tamanho da janela e consequentemente no tempo de
execução do algoritmo;

\item Reconhecimento de múltiplas análises com mesma configuração:
suponha que se deseje realizar quatro análises, duas delas com um filtro com
$\sigma=2$ e valores de corte \acs{di}$_{min}=0,1$ e
\acs{di}$_{min}=0,2$, enquanto os outros dois filtros irão ter 
$\sigma=3$ com os mesmos cortes. Nesse caso, não é necessário gerar a
respostas dos filtros de derivada de Gaussiana quatro vezes, apenas
duas vezes e então aplicar os dois cortes em cada uma dessas
respostas. O ambiente de análise realiza a identificação dessas
ocorrências quando executando múltiplas análises e as executa apenas
uma vez, partilhando essa memória para os algoritmos seguintes; 

\item Informação gráfica: a Figura~\ref{fig:analise_eventos} contém um
exemplo de gráfico gerado para uma análise. É possível observar a
resposta do filtro de derivada de Gaussiana, as regiões
sensibilizadas, os eventos, suas janelas para cálculo das \acs{fex} e
os seus estados (ver p.~\pageref{text:estados_eventos}) informando se
os mesmos foram aceitos ou foram eliminados devido a algum dos cortes.

\end{itemize}


\begin{sidewaysfigure}[p]
\centering
\includegraphics[width=.8\textwidth]{imagens/Empilhado7_ex_incosistencia_e_media.pdf}
\caption[Exemplo de informação gráfica para o Módulo de Análise dos
Dados.]{Exemplo de informação gráfica para o Módulo de Análise dos
Dados. Na subfigura inferior, as regiões verdes e vermelhas indicam
regiões sensibilizadas por respostas positivas e negativas,
respectivamente. A resposta para o filtro de
derivada de Gaussiana é representado pela linha pontilhada, enquanto a
linha horizontal cinza é o limear de corte para a geração de uma região
sensibilizada. É possível observar um caso de evento incosistente e
outro removido devido a evento próximo. Para o caso do evento inconsistente, em azul, seu
degrau de potência é positivo enquanto sua resposta é negativa,
revelando sua incosistência. Já os eventos próximos representados
pelas caixas amarelas foram removidos por estarem próximos, sendo
substituídos pela sua média (a linha verde). Nessa figura também é possível observar
as regiões que serão utilizadas para a extração do transitório (região
cinza) e as regiões utilizadas para calcular o degrau de potência
(regiões amarelas pré/pós-transitório). }
\label{fig:analise_eventos}
\end{sidewaysfigure}


\section{Otimização dos Parâmetros}
\label{sec:otimizacao}

Retornando agora para o esboço do ambiente
(Figura~\ref{fig:ambiente_analise}), observa-se que o Módulo de
Otimização dos Parâmetros utiliza o módulo de análise iterativamente
para obter parâmetros ótimos (não necessariamente o ótimo global). 

\begin{itemize}

\item Redução de memória transitória: um detalhe operativo, as
análises geradas pelo usuário mantém a informação da resposta do
filtro de Gaussiana após o término da análise pois essa informação é
necessária para gerar a informação gráfica, porém, no caso do
otimizador diversas análises serão geradas, sendo interessante limitar
ao máximo o consumo de memória em cada uma delas. Por isso, as
análises geradas pelo algoritmo remove qualquer informação irrelevante
durante a execussão da análise como a resposta do filtro e qualquer
outros elementos não necessários, mantendo apenas os eventos de
transitório;

\item Regras de Pontuação: para realizar a otimização, é necessário
haver uma regra para o otimizador avaliar aquilo que é desejável do
que não é. É natural que essa regra tome como recompensa identificar
corretamente um evento de transitório e como punição gerar um evento
de transitório aonde essa informação não existe. O primeiro caso é
referido de detecção, enquanto o segundo consiste de um falso positivo
ou falso alarme. Também é possível adicionar outras punições, por
exemplo, não é desejável que sejam gerados muito candidatos para serem
removidos, isso irá reduzir a velocidade de processamento e se uma
solução consegue realizar a detecção com acurácia parecida mas gerando
menos candidatos, é preferível optar por essa opção, mesmo que ela
tenha uma performance ligeiramente reduzida para obter uma performance
de execução quando aplicado em um \gls{nilm} na residência melhor.
Porém, essa punição deve ser pequena, por não ser a grande questão a
ser otimizada. A regra de pontuação utilizada está representada em
\ref{eq:regra_pontuacao}. Ainda, os eventos detectados pelo Módulo de
Análise não estarão exatamente na mesma posição que os eventos gerados
pelo usuário no gabarito, sendo preciso definir uma janela de um
número máximo de amostras \acs{jmax} para os quais irá aceitar o
evento de detecção;
\begin{equation}\label{eq:regra_pontuacao}
\textbf{Aptidão}=\gamma_{det}N_{det}+\gamma_{fa}N_{fa}+\gamma_{rem}N_{rem}
\end{equation}
\noindent onde:
\begin{description}
\item[$\text{Aptidão}$] mede a capacidade de resposta da análise realizada,
sendo de interesse maximizar esse valor.
\item[$\gamma_{det}$] é a pontuação que a análise recebe para cada
evento de detecção;
\item[$N_{det}$] é a quantidade de eventos detectados;
\item[$\gamma_{fa}$] é a pontuação que a análise recebe para cada
ocorrência de falso alarme;
\item[$N_{fa}$] é a quantidade de ocorrências de falso alarme;
\item[$\gamma_{rem}$] é a pontuação que a análise recebe para cada
ocorrência de candidatos removidos;
\item[$N_{rem}$] é a quantidade de candidatos removidos;
\end{description}

\item Comparação da resposta da análise com o gabarito: em posse da
regra de pontuação \ref{eq:regra_pontuacao} e \acs{jmax}, realiza-se a
comparação entre as duas informações e retona-se a eficiência em termos
de Aptidão;

\item Escolha do algoritmo: a função a ser otimizada não é
diferenciável e portanto não é possível utilizar os métodos
convencionais de otimização. É necessário empregar algum método de
tentativa e erro para realizar essa tarefa. Neste trabalho, optou-se
pela utilização de um \acs{es}, porém uma outras estratégias de
otimização podem ser utilizadas, como Inteligência de Enxame. 
Irá se aprofundar nas características do algoritmo
implementado na Subsessão~\ref{ssec:es}.

\end{itemize}


\subsection[Algoritmo Genético de Estratégia Evolutiva]{\acf{es}}
\label{ssec:es}

Foi realizado a implementação de uma versão própria de um \acs{es} com
base em \cite[cap. 4]{eiben2003introduction}, mas antes de entrar em
detalhes sobre a versão implementada cabe introduzir o tema sobre
algoritmos de estratégia evolutiva.

\begin{figure}[h!t]
\centering
\includegraphics[width=.9\textwidth]
{imagens/ga.pdf}
\caption[Esboço de um algoritmo evolutivo genérico.]
{Esboço de um algoritmo evolutivo genérico. Baseado em
\cite[p. 17]{eiben2003introduction}.}
\label{fig:esboco_ga}
\end{figure}

Dispõe-se na Figura~\ref{fig:esboco_ga} a sequência de otimização
de um algoritmo evolutivo genérico. Ela conta com uma população
inicial, onde se recomenda que a mesma seja iniciada aleatóriamente.
Essa população inicial passará por um processo de seleção parental
aonde serão obtidos os espécimes ou individuos para a propagação de
sua informação genética para sua prole, a \gls{mu}. Porém, a
\gls{lambda} recebe o, material perturbado através de operações de
recombinação e mutação. A mutação é uma pertubação que ocorre somente
levando em conta o material de um pai, enquanto a recombinação ocorre
no mínimo com dois pais. A capacidade de encontrar novas regiões
promissoras (genes alelos) --- descoberta --- no espaço de busca de
soluções e ganhar informação no problema é dada pela mutação através
de diversificações aleatórias, enquanto a recombinação realiza a
otimização dentro de uma área promissora (no caso, dentro da
informação genética dos pais) --- exploração ---, dando grandes pulos
para uma região dentro de duas áreas. O resultado do material genético
pertubado constitui da base para a geração da prole. Irá ocorrer então
a seleção dos sobreviventes, que pode levar em consideração a prole e
os pais nesse processo ($\mu$+$\lambda$) ou apenas a prole
($\mu$,$\lambda$). Os individuos selecionados são a nova geração da
população, repetindo o processo até uma condição de parada. Geralmente
a parada é através de um número máximo de gerações.

Cabe ainda definir a diferença entre genótipo e fenótipo do ponto e
vista computacional. A representação em genótipo é aquela que sofre as
pertubações e codifica a representação do espécime, enquanto o
fenótipo representa a informação como é demonstrada pelo espécime para
o problema em questão, ou seja, o espaço de solução. Pode haver
diferença entre as duas representações ou não, por exemplo, a
representação \{Norte,Leste,Sul,Oeste\} seria as possíveis
representações do fenótipo, e sua codificação em genótipo poderia ser
feita como \{1,2,3,4\}, respectivamente. Nota-se a importância de não
só saber representar a informação em genótipo para realizar as
operações de pertubação no material, como possuir uma maneira de
decodificá-la novamente no espaço do fenótipo. Para isso, cada solução
do fenótipo deve ser mapeável, bem como cada genótipo tenha a
decodificação em apenas uma solução. Ainda, a escolha da representação
irá afetar o problema: no exemplo citado a representação escolhida
não parece ter o significado do fenótipo, uma vez que Norte e Oeste são
vizinhos entre si, enquanto na representação eles estão distantes de três
unidades. Uma escolha em variáveis cíclicas parece representar o
problema fidedignamente
$\{(\text{sen}(\frac{1\pi}{2}),\cos(\frac{1\pi}{2})),
(\text{sen}(\frac{2\pi}{2}),\cos(\frac{2\pi}{2})),
(\text{sen}(\frac{3\pi}{2}),\cos(\frac{3\pi}{2})),
(\text{sen}(\frac{4\pi}{2}),\cos(\frac{4\pi}{2}))\}$, porém o genótipo
seria representado em duas variáveis.

\subsubsection{Versão original}

O \acs{es} é um algoritmo genético cuja especialidade é a
auto-adaptação de sua estratégia evolutiva. Enquanto nos outros
algoritmos genéticos geralmente a taxa de mutação é pequena, e a taxa
de recombinação a de maior peso no processo de otimização, isso é o
oposto para o caso do \acs{es}. Todos os individuos passam por
pertubações Gaussianas em seu material genético, no entanto, a ordem
dessas pertubações são ajustadas conforme a evolução da espécie,
aumentando ou diminuindo sua ordem conforme as necessidades de
evolução. Assim, o \acs{es} irá aumentar a ordem de suas pertubações
quando distante de um valor ótimo --- sujeito a capacidade de perceber
uma tendência no espaço de solução apontando na direção de um ótimo
--- e reduzir as pertubações conforme se aproxima desse valor para
realizar o ajuste fino. Há também uma pequena taxa de recombinação
para aumentar a velocidade de convergência, normalmente na faixa de
10\%\footnote{As taxas de recombinação e mutação são dadas em termos
de probabilidade.}. A seleção parental não é influenciada pela aptidão
dos individuos, quando utilizando taxa de recombinação a escolha dos
pais para haver troca de material genético é realizado de maneira
aleatória uniforme. A melhoria gradual das gerações é realizado pela
seleção dos sobreviventes, que é realizada através de
($\mu$,$\lambda$) com pressão alta de seleção. Entende-se como pressão
de seleção dos sobreviventes a grandeza demonstrada por
\ref{eq:pressao_selecao}. É importante a seleção
através de ($\mu$,$\lambda$) em comparação com a ($\mu$+$\lambda$)
para evitar ótimos locais, bem como garantir que não haverá propagação
de individuos com estratégias mal-adaptadas através das gerações. O
mesmo se dá versões que utilizam elitismo --- manter ao menos uma
cópia do membro mais apto da população na próxima geração ---, não
sendo recomendado no \acs{es} pelo mesmo problema da seleção parental
através de ($\mu$+$\lambda$). É importante também que a seleção de
sobreviventes aplique uma alta pressão de seleção, garantindo a
capacidade adaptativa do \acs{es}, normalmente utilizando
$\frac{\lambda}{\mu}=7$.

\begin{equation}\label{eq:pressao_selecao}
\text{Pressão de Seleção} = \dfrac{\lambda}{\mu}
\end{equation}

Optou-se pela mutação descorrelacionada com $n$ tamanhos de passo
\cite[p. 76--78]{eiben2003introduction}. Nessa configuração, cada
variável representado no genótipo tem sua própria pertubação, sendo
descrita por \ref{eq:s_esbegin}, enquanto sua pertubação é adaptada
anteriormente de acordo com \ref{eq:sigma_esbegin}. A taxa de
aprendizado é divida em dois parâmetros, $\tau$ e $\tau'$, onde aquele
é a base de aprendizado, que garante uma mudança global na
mutabilidade para preservar os graus de liberdade do problema, e este
é uma mutação específica por coordenada, fornecendo flexibilidade para
empregar diversas estratégias de mutação em diferentes difereções. Os
valores indicados para ambos são $1/\sqrt{2n}$ e $1/\sqrt{2\sqrt{n}}$,
respectivamente. O alcance da pertubação no material genético dado uma
determinada probabilidade de ocorrência fixa formam elipsóides no
espaço de solução alinhadas com os eixos da representação escolhida. 

\begin{subequations}
\begin{equation}\label{eq:s_esbegin}
x_i' = x_i\sigma_i'N_i(0,1)
\end{equation}
\begin{equation}\label{eq:sigma_esbegin}
\sigma_i' = \sigma_ie^{\tau'N(0,1)+\tau N_i(0,1)}
\end{equation}
\end{subequations}

\noindent onde: 

\begin{description}
\item[$N(0,1)$] e $N_i(0,1)$ são uma pertubação Gaussiana com média
zero e $\sigma$ unitário, a primeira sendo um único valor para todos
as representações , enquanto a segunda uma para cada representação $i$; 
\item[$x_i$] e $x_i'$ é a i-ésimo representação e a mesmo após sofrer a
pertubação;
\item[$\sigma_i$] e $\sigma_i'$ é a i-ésima estratégia evolutiva e a
mesma após sofrer a pertubação.
\end{description}

Para a recombinação, implementou-se a versão local da mesma por ela
ser mais simples de elaborar, pretendendo alterar no futuro para
a recombinação global que é mais indicada para o caso do \acs{es}.
Porém, a recombinação utilizada não é o quesito principal do
algoritmo, servindo apenas para melhorar a velocidade de convergência.
Para o caso implementado, o material genético é mesclado entre apenas
dois individuos. A informação genética que representa o espaço de
solução é misturada através de \ref{eq:rec_discreta} --- recombinação discreta
---, enquanto a versão para a estratégia evolutiva é realizada através
de \ref{eq:rec_intermediaria} --- recombinação intermediária.

\begin{subequations}
\begin{equation}\label{eq:rec_discreta}
\left\{\begin{array}{l}
x_{i,1}' = x_{i,1} \;\; \textit{ou} \;\; x_{i,2} \;\;\;\;\;\; \text{escolhidos
aleatoriamente}\\
x_{i,2}' = x_{i,\text{o.c.}}
\end{array}\right.
\end{equation}
\begin{equation}\label{eq:rec_intermediaria}
\left\{\begin{array}{l}
\sigma_{i,1}' = \dfrac{(\sigma_{i,1}+\sigma_{i,2})}{2} \\
\sigma_{i,2}' = \dfrac{(\sigma_{i,1}+\sigma_{i,2})}{2}
\end{array}\right.
\end{equation}
\end{subequations}

\noindent onde:

\begin{description}
\item[$x_{i,1}$] e $x_{i,2}$ são a i-ésima representação para o
primeiro e segundo pai, respectivamente; 
\item[$x_{i,o.c.}$] é a i-ésima representação para o pai não
selecionado para $x_{i,1}'$;
\item[$\sigma_{i,1}$] e$\sigma_{i,2}$ são a i-ésima estratégia
evolutiva para o primeiro e segundo pai, respectivamente;
\item[$x_{i,1}'$] e $x_{i,2}'$ são a i-ésima representação para o
primeiro e segundo pai após a pertubação, respectivamente; 
\item[$\sigma_{i,1}'$] e $\sigma_{i,2}'$ são a i-ésima estratégia
evolutiva para o primeiro e segundo pai após a pertubação,
respectivamente.
\end{description}

\begin{figure}[h!t]
\centering
\includegraphics[width=.9\textwidth]{imagens/Ackley.png}
\caption{Função de \emph{Ackley} em duas dimensões para $-30,0< x_i <
30,0$.}
\label{fig:funcao_ackley}
\end{figure}

A referência \cite[p. 84]{eiben2003introduction} cita um exemplo de um
outro autor que aplicou o \acs{es} para a função de \emph{Ackley},
onde foi utilizado \acs{mu}$ = 30$, \acs{lambda}$ = 200$, e $x_i$
inicial entre $-30,0 < x_i < +30,0$ e um total de 200.000 avaliações
da função. Para um total de 10 execuções, o exemplo citado obteve a
melhor solução com valor da função de $7,48\times10^{-8}$. A função de
\emph{Ackley} é altamente multimodal, com um grande número de mínimos
locais, mas com apenas um máximo global $\overline{x}=0$ e seu valor
$f(\overline{x})=0,0$. Para validar o algoritmo implementado, executou-se o
\acs{es} para essas configurações, obtendo uma ocorrência de
convergência para mínimo local com o valor de $1,34$, e todas as
outras ocorrências dão uma aptidão de média de
$1,87\times10^{-7}\pm2.56\times10^{-7}$, onde a melhor solução é de
tem a aptidão $1,34\times10^{-8}$. A evolução da execução com melhor
convergência está na Figura~\ref{fig:es_standard}, mostrando o valor
médio de aptidão na geração da população na convergência. Apesar do exemplo
citado informar que todos os mínimos encontrados foram os globais, na
versão aqui implementada, ocorrem casos em que não há a convergência
para o mínimo global, ainda que em outras execuções seja possível
encontrar todas as 10 minimizações convergindo para o mínimo global. É
importante ter em mente que a convergência não ocorre necessariamente
para o mínimo global, sendo uma propriedade bem conhecida dos
algoritmos genéticos.  De qualquer forma, os valores obtidos estão
próximos da referência e existem parâmetros não informados como o
valor mínimo de pertubação $\sigma_{min}$ e o valor inicial para as
pertubações $\sigma_{inicial}$ que podem influênciar na resposta.
Apenas como referência, os valores utilizados para esses casos foram
$\sigma_{min}=1\times10^{-9}$ e $\sigma_{inicial}=1$.



\begin{figure}[h!t]
\centering
\includegraphics[width=\textwidth]{imagens/es_standard.pdf}
\caption[Evolução para o melhor individuo para a validação da versão
original do ES]{Evolução para o melhor individuo para a validação da versão
original do \acs{es}. O objetivo é minimizar a função de
\emph{Ackley}, que pela visão do \acs{es} funciona como maximizar a
função com seus valores opostos. Por isso, os valores mostrados são
negativos.}
\label{fig:es_standard}
\end{figure}

\subsubsection{Versão Multiespécie}

Poderia ser utilizado a versão original do \acs{es} para realizar a
otimização dos parâmetros necessários na abordagem do problema. No
entanto, a decorrência da dúvida quanto a qual caminho
percorrer para a análise (ordem de remoção de eventos e quais delas
empregar) haveriam de ser realizadas diversas execuções
do algoritmo para otimizar os valores de maneira individual. Ao invés
de executar cada uma delas, motivado pela ideia de otimização
multiobjetivo \cite[cap. 9]{eiben2003introduction}, decidiu-se utilizar
a ideia de subpopulações --- que serão referidas por espécies, por
poderem ter cromossomos diferentes dependendo da configuração
utilizada ---, mas aplicando a mesma para um outro conceito. No caso,
ao invés de utilizar espécies para otimização de múltiplas funções
objetivo, esse conceito irá ser utilizado para criar uma dinâmica
entre as várias otimizações sendo realizadas no problema.

A ideia da dinâmica é reservar o esforço computacional para aquelas
abordagens que estão mostrando capacidade de resolver o problema com
maior aptidão, revelando-se uma espécie mais adequada para o
\emph{habitat} em que os espécimes estão sendo avaliados. Porém, isso
deve ser realizado sem comprometer a evolução de espécies que, por
algum motivo, sofreram desvantagem durante o processo evolutivo ---
seja por uma inicialização em condições desprivilegiadas, ou por uma
demora maior para ajustar sua estratégia evolutiva. Assim, a proposta
é executar apenas uma otimização para as diferentes maneiras de
tratar o problema, aonde todas as configurações desejadas irão
competir entre si de modo que o algoritmo irá reservar maior
esforço computacional para otimizar mais profundamente aquela que se
melhor adequa ao espaço de solução, diferente da versão onde se
executaria para cada espécie, reservando esforço computacional igual
para espécies que não tem se mostrado adequadas para a solução do
problema.

Assim, propuseram-se duas configurações para as competições dos
espécimes:

\begin{itemize}
\item Interespécie: nesse caso há cooperação entre os individuos de
uma mesma espécie. É calculada a aptidão para cada espécie
(\ref{eq:aptidao_especie}) para ser utilizada como parâmetro na
competição das mesmas e determinar a parcela da população global que
elas tem direito. Uma vez determinado o tamanho da população de cada
espécie, seus individuos irão competir entre si para determinar os
sobreviventes. É necessário escolher um método para avaliar a aptidão
das espécies e como determinar suas populações através dele;
\item Intraespécie: os espécimes disputam entre si na população global
independente de qual espécie pertencem. Nessa configuração não há
cooperação entre os individuos de uma mesma espécie, apenas os
melhores da população global sobrevivem;
\end{itemize}

Para evitar que espécies não tenham a oportunidade de se desenvolverem
antes que sua população seja drasticamente reduzida ou até mesmo
extinta, tratou-se cada um dos casos individualmente. No caso da
seleção interespécie, é necessário escolher uma função que privilegie
espécies mais aptadas, mas que uma diferença de aptidão muito grande
--- que irá ocorrer em especial durante o inicio da evolução devido a
espécies condicionadas em ambientes mais propícios que outras --- não
elimine toda a diversidade das populações antes que elas adequem sua
estratégia evolutiva. Para isso, escolheu-se empiricamente a função
\ref{eq:funcao_interespecie}, para suavizar a pressão aplicada
em espécies menos aptas. A população reservada para uma espécie para a
próxima geração é dada por \ref{eq:mu}. Entretanto, como se utiliza a
função de corte \emph{floor} para transformar os valores em inteiros,
ao somar $\mu_i'$ para cada espécie pode acabar resultando em uma
população menor que $\mu$. Assim, distribui-se aleatoriamente os
individuos faltantes nas espécies de forma que a soma dos $\mu_i'$
seja o mesmo que $\mu$. Um exemplo de execução para 10 espécies para a
função de \emph{Ackley} pode ser visto na
Figura~\ref{fig:interespecies}.

\begin{subequations}\label{eq:inter_especie}
\begin{equation} \label{eq:aptidao_especie}
\text{Aptidão}_{(i)} = \sum^{\lambda_i}_{j=1} \text{Aptidão}_{(i,j)}
\end{equation} 
\begin{equation} \label{eq:funcao_interespecie}
f_{inter}(i)=log_2(\text{Aptidão}_{(i)}-min(\text{Aptidão}_{(i)}|\forall i\in \Gamma)+2)
\end{equation} 
\begin{equation} \label{eq:mu}
\mu_i' = floor\left(\dfrac{\text{Aptidão}_{(i)}}{f_{inter,norm}}\right)
\end{equation}
\begin{equation} \label{eq:fnorm}
f_{inter,norm}= \sum_{\forall i\in \Gamma} f_{inter}(i)
\end{equation}
\end{subequations}

\noindent onde:

\begin{description}
\item[$\lambda_i$] é o tamanho da população da prole da i-ésima
espécie;
\item[$\Gamma$] é o espaço contendo todas as espécies;
\item[$\mu_{i}'$] é o tamanho da população dos pais da i-ésima espécia
para a próxima geração;
\end{description}


\begin{figure}[h!t]
\centering
\includegraphics[width=\textwidth]{imagens/es_interspecies.pdf}
\caption[Competição interespécie.]{Competição interespécie. Na
subfigura superior, as linhas contínuas, tracejadas finas e tracejadas
grossas indicam respectivamente os individuos mais aptos de cada
espécie, a média de aptidão da população de cada espécie e os
individuos menos aptos de cada espécie. As subfiguras inferiores
indicam a população para cada espécie, sendo a mais inferior a
população para a prole, e o outro caso a população dos pais da
espécie.}
\label{fig:interespecies}
\end{figure}

Um outro mecânismo foi implementado para impedir a extinção de uma
espécie. Ele funciona como um órgão de proteção da diversidade de
espécies, limitando de espécies próximas de entrarem em extinção de
reduzirem a sua população, independente do quão mal esses individuos
se adequam ao \emph{habitat} para o qual estão sendo avaliados.

Isso foi especialmente importante para o caso de competição
intraespécie, onde uma espécie ao encontrar um material genético de
melhor qualidade em comparação com as outras, rapidamente tomava conta
da população global por espalhar esse material entre sua população com
grande velocidade. Na Figura~\ref{fig:intraspecies_nopressure}
observa-se que se não fosse esse mecânismo, a espécie azul ou amarela
iriam extinguir todas as outras tomando conta da população
global. Fica evidente também que apenas um órgão de proteção da
diversidade de espécies não é suficiente para garantir a evolução das
espécies, é necessário suavizar a competição, de modo que uma espécie
que por algum motivo se tornou mais apta não extermine outras espécies
rapidamente acabando com sua diversidade e não as dê a oportunidade
para evoluir, já que os individuos que sobraram possivelmente ainda não
ajustaram sua estratégia evolutiva. A 
Figura~\ref{fig:intraspecies_nopressurecontrol_info} mostra as
pressões de seleção em ordens muito além daquelas que deveriam
ocorrer, obtendo valores na ordem de 20 logo no ínicio da evolução
para as espécies que foram iniciadas em condições menos favoráveis,
eliminando toda sua diversidade em poucas gerações. Já as espécies que
conseguiram se desenvolver, observa-se que as mesmas ao conseguirem
uma aptidão melhor que a da outra espécie rapidamente tomam conta da
população global, pontos que são marcados pelos picos na pressão de
seleção das espécies antes dominantes.

\begin{figure}[h!t]
\centering
\includegraphics[width=\textwidth]{imagens/es_intraspcies_nopressurecontrol.pdf}
\caption[Competição intraespécies sem intervenção na
competição.]{Competição intraespécies sem intervenção na competição.}
\label{fig:intraspecies_nopressure}
\end{figure}

\begin{figure}[h!t]
\centering
\includegraphics[width=\textwidth]{imagens/es_intraspcies_nopressurecontrol_pressureInfo.pdf}
\caption[Pressão de seleção para competição intraespécie sem
intervenção na competição.]{Pressão de seleção para competição
intraespécie sem intervenção na competição.}
\label{fig:intraspecies_nopressurecontrol_info}
\end{figure}

Em vista disso, implementou-se um mecânismo de controle de pressão de
seleção por espécie. Esse mecânismo irá aceitar um valor máximo de
pressão para cada espécie, se o valor ultrapassar o limiar, então irá
reduzir sua pressão ao alocar espaço da população para essa espécie
retirando das espécies com maior alocação de população até que a
alocação de população dessas espécies que ultrapassaram o corte máximo
de pressão resultem em um valor aceitável da mesma. A escolha de
retirada da alocação de individuos é feita da seguinte maneira:

\begin{itemize}
\item Inicia-se reduzindo a alocação de população da espécie com maior
população;
\item Se o valor da espécie de maior população atingir o tamanho da
população de uma outra espécie, adiciona-se essa espécie para a
redução de população e continua o processo. Caso isso ocorra
novamente, a próxima espécie também será adicionada para redução de
população e o processo continua até que seja determinado quais
espécies irão ceder espaço para que a população das espécies com altas
pressões satisfaça o critério mínimo;
\item Quando há o agrupamento de espécies para redução e a
necessidade de reduzir um número não inteiro de população em cada
espécie agrupada, a escolha do residuo da divisão é feita
aleatoriamente. Ex. se houver de suprir 100 espécimes para garantir que
a pressão de seleção em níveis aceitáveis, e houver 3 espécies tendo
sua população reduzida, irá retirar 33 individuos de cada uma delas,
porém a escolha da espécie que perderá mais um individuo será
realizada aleatoriamente.
\end{itemize}

A execução para uma pressão máxima de 7,3 pode ser visualizada na
Figura~\ref{fig:intraspecies_pressurecontrol}, onde fica evidente a
convergência da população para a máxima aptidão (ou mínimo da função
da \emph{Ackley}). Também se observa uma mudança menos brusca quando
comparado à versão sem intervenção na competição, mostrando que o
controle é importante para garantir mudanças mais suaves na
configuração da população. Uma observação importante pode ser
realizada quanto ao órgão de proteção de diversidade de espécies: não
fosse sua operação, diversas espécies teriam sido extintas durante o
processo evolutivo. Na
Figura~\ref{fig:intraspecies_pressurecontrol_info} observa-se a
pressão de seleção requerida pela competição natural, e aquela
aplicada pelo sistema de intervenção. Observa-se que a competição
natural chega a exigir pressões de até 500, o que acabaria com a
diversidade de uma espécie em uma única geração. Assim, ao optar pela
versão de competição intraespécies faz-se necessário interferir na
competição para que todas as espécies tenham chances de evoluir. 


\begin{figure}[h!t]
\centering
\includegraphics[width=\textwidth]{imagens/es_intraspecies_pressurecontrol.pdf}
\caption[Competição intraespécies com intervenção na
competição.]{Competição intraespécies com intervenção na competição.}
\label{fig:intraspecies_pressurecontrol}
\end{figure}

\begin{figure}[h!t]
\centering
\includegraphics[width=\textwidth]{imagens/es_intraspcies_pressurecontrol_pressureInfo.pdf}
\caption[Pressão de seleção para competição intraespécie com 
intervenção na competição.]{Pressão de seleção para competição
intraespécie com intervenção na competição.}
\label{fig:intraspecies_pressurecontrol_info}
\end{figure}

Há uma nítida diferença entre como as duas seleções se comportam. Ao
comparar as figuras~\ref{fig:interespecies} e
\ref{fig:intraspecies_pressurecontrol}, percebe-se que o caso de
competição interespécie tem uma mudança bastante tênue na configuração
da população, privilegiando as espécies com melhor aptidão
proporcionalmente à sua aptidão como espécie. No caso, a escolha da
função torna a vantagem pequena entre elas, já que se utiliza uma
atenuação logaritmica. Enquanto isso, na competição intraespécie com
intervenção (a versão sem intervenção não é recomendada) observa-se
que o crescimento da população da espécie ocorre gradualmente conforme
seus individuos ocupam posições privilegiadas no espaço de
solução.

Finalmente, também se adicionou uma outra funcionalidade ao \acs{es}.
Como as otimizações podem levar dias para serem executadas, viu-se a
necessidade de armazenar o processo enquanto ele evoluia, para
garantir que se ocorresse algum problema na máquina em execução, não
houvesse de recomeçar o processo desde a primeira geração. Assim, a
versão do algoritmo é capaz de armazenar, se o usuário requirir, as
gerações e recuperar o processo caso ocorra a interrupção do processo
por algum motivo. 


  \chapter{Metodologia}
\label{chap:metodologia}

Uma vez definido o ambiente de análise implementado e suas
capacidades, cabe detalhar como o mesmo será aplicado para
determinar os parâmetros da abordagem final para detecção de
eventos de transitório e a base de dados que será
trabalhada. Este capítulo começa realizando a descrição das
características presentes nos conjuntos de dados trabalhados, que
podem ser separados em dois grandes grupos: um contendo a operação de
poucos equipamentos atuando em conjunto e praticamente nenhuma
injeção de ruído devido à equipamentos \acs{c5} (ver
Subseção~\ref{ssec:modelos_carga}) --- os dados puros contidos nos
arquivos NI00***\footnote{O símbolo ``*'' é utilizado com o mesmo
significado do metacaractere para os sistemas operacionais
\emph{Linux} na computação, ou seja, podem significar qualquer outro
caractere --- o análogo à carta coringa nos baralhos de cartas.};
enquanto o outro conta com a operação de múltiplos equipamentos
atuando simultaneamente (cerca de 5 a 8, em geral) e da presença de
equipamentos \acs{c5}, constituindo, por isso, de dados com menor
relação sinal-ruído --- conjuntos \emph{Temporizado},
\emph{Empilhado4} e \emph{Empilhado7}. Todos os conjuntos de dados
possuem apenas um gabarito, exceto o \emph{Temporizado}, que possui
dois gabaritos. O motivo para isso é a presença de um distúrbio
durante a operação da televisão LCD nesse conjunto de dados, que
\emph{não} é considerado como alvo na configuração \emph{Temporizado
Gabarito 1}, enquanto a configuração \emph{Temporizado Gabarito 2}
contém essa informação como eventos de transitórios desejados como
alvo.


Em seguida, detém-se atenção para quais configurações e parâmetros
presentes na metodologia proposta serão otimizados pelo \acs{es}
descrito na Seção~\ref{ssec:es} do capítulo anterior, aonde detalhará
como isso foi realizado. Como referência rápida, estes foram os
parâmetros e as configurações otimizadas pela versão multiespécie do
\acs{es} implementado:

\begin{itemize}
\item \textlabel{$\sigma_{gauss}$}{item:parametros}: O valor do
$\sigma$ da Gaussiana empregado para retirar sua derivada e compor o
\acs{fir} que é realiza a sua convolução com a envoltória de corrente;
\item $\delta_{min}$: O valor mínimo que a resposta do \acs{fir} deve
ter, em módulo, para as respectivas amostras constituam de uma região
sensibilizada;
\item $\Delta{I}_{min}$: Mínimo valor para a característica \acs{di} na
qual o candidato a evento não será eliminado como ruído 
\item $n_{min}$ (sujeito à utilização na configuração otimizada da
remoção por eventos próximos): Quantidade mínima de amostras que um
candidato deve estar distante de outro para não ser removido por evento
próximo. A estratégia para a remoção depende da configuração
escolhida, conforme próximo item; 
\item Configurações otimizadas (em competição intraespécie), todas com
remoção devido a eventos inconsistentes:
\begin{enumerate}[label={Espécie} (\Roman*) -,ref=(\Roman*),align=left]
\item\label{item:esp1} Com remoção de eventos próximos sem
deslocamento, essa \emph{depois} de remover eventos ruidosos;
\item\label{item:esp2} Com remoção de eventos próximos utilizando a
média, essa \emph{depois} de remover eventos ruidosos;
\item\label{item:esp3} Com remoção de eventos próximos sem
deslocamento, essa \emph{antes} de remover eventos ruidosos;
\item\label{item:esp4} Com remoção de eventos próximos utilizando a
média, essa \emph{antes} de remover eventos ruidosos;
\item\label{item:esp5} Sem remoção de eventos próximos (não otimiza
$n_{min}$). 
\end{enumerate}
\end{itemize}


\section{Descrição da base de dados}
\label{sec:base_de_dados}

O \acs{cepel} propôs uma evolução gradual para a análise de sua
técnica. As configurações mais simples incluem a coleta de dados com
equipamentos sendo acionados e desacionados sequencialmente operando
individualmente em apenas um único estado. Em seguida, simulou-se
condições de operação em conjunto de equipamentos dois a dois, e
equipamentos \acs{c2} operando em seus diversos estados. Essas
condições mais simples e com dados bastante limpos constituíram a base
de dados dos arquivos NI00*** que serão detalhados adiante nesta
seção. Para todos os conjuntos utilizados neste trabalho, foi
construído o gabarito com a infraestrutura implementada descrita no
Capítulo~\ref{chap:framework} e de posse de registros das alterações
nos estados operativos dos equipamentos fornecidos pelo \acs{cepel}.

Já os conjuntos de dados \emph{Temporizado}, \emph{Empilhado4} e
\emph{Empilhado7} constituem de simulações em laboratório mais
complexas, contendo a operação de diversos aparelhos simultaneamente
que têm sua detecção dificultada pela presença da dinâmica de
aparelhos \acs{c5}. O conjunto \emph{Temporizado} foi criado pelo
\acs{cepel} mais recentemente com o intuito de explorar os resultados
de sua técnica em condições ruidosas, enquanto os conjuntos
\emph{Empilhado4} e \emph{Empilhado7} foram elaborados durante a época
do estudo de \citeauthor*{nilm_cepel_alvaro}, que não teve a
oportunidade de analisar mais a fundo tais conjuntos de dados. Os
registros das alterações nos estados operativos dos equipamentos
nesses conjuntos de dados foram fornecidos pelo mesmo, o que
possibilitou a recuperação desses dados com a informação necessária
para a construção do gabarito.

\subsection{Conjunto de dados \emph{NI00168}}

Este conjunto de dados contém apenas a operação de equipamentos
individualmente. Há a presença de dois equipamentos desconhecidos, que
não foram possíveis de ter sua informação recuperada ao correlacionar
com os registros de alterações. Como não era prioridade neste trabalho
ter essa informação --- esses equipamentos não-rotulados não
apresentam características especiais nos eventos de transitório
decorrentes de seus acionamentos e desacionamentos, e essa informação
é mais importante para a etapa de discriminação ---, trabalhou-se com
os dados sem a rotulação desses equipamentos.

O consumo agregado neste conjunto de dados pode ser observado na
Figura~\ref{fig:ni00168_overview}, enquanto a
Figura~\ref{fig:ni00168_app_time} contém a estimativa do consumo
desagregado no gabarito.

Os equipamentos presentes neste conjunto de dados são, todos com dois
eventos de transitório, exceto quando especificado:

\begin{itemize}
\item Lâmpada incandescente 100 W, 60 W (com \emph{dimmer},
o que constitui uma \acs{c4});
\item Dois equipamentos desconhecidos;
\item Lâmpada fluorescente 15 W, 16 W, 18 W, 20 W, 21 W (circular),
40 W (tubular), 32 W (tubular), 28 W (tubular);
\item Secador de cabelo (4 eventos: Desligado $\rightarrow$ Fraco
$\rightarrow$ Forte $\rightarrow$ Fraco $\rightarrow$ Desligado);
\item Televisão (3 eventos: Desligado $\rightarrow$ Ligado
$\rightarrow$ Ligado e exibindo imagem $\rightarrow$
Desligado)\footnote{A televisão pode ser ligada em um canal que não
apresenta imagem.};
\item Geladeira;
\item Ar condicionado (4 eventos: Desligado $\rightarrow$ Ventilação
$\rightarrow$ Ventilação e compressor $\rightarrow$ Ventilação
$\rightarrow$ Desligado, constitui uma \acs{c5}).
\end{itemize}

Todos os arquivos NI00*** foram coletados utilizando o medidor do
\acs{cepel}.

\begin{sidewaysfigure}[p]
\centering
\includegraphics[width=\textwidth]{imagens/NI00168_Overview.pdf}
\caption{Perfil de consumo agregado para o conjunto de dados
\emph{NI00168}.}
\label{fig:ni00168_overview}
\end{sidewaysfigure}

\begin{sidewaysfigure}[p]
\centering
\includegraphics[width=\textwidth]{imagens/NI00168_AppTime.pdf}
\caption{Informação no gabarito para o conjunto de dados
\emph{NI00168} - consumo temporal dos equipamentos.}
\label{fig:ni00168_app_time}
\end{sidewaysfigure}

\begin{sidewaysfigure}[p]
\centering
\includegraphics[width=\textwidth]{imagens/NI00168_AppPie.png}
\caption{Informação no gabarito para o conjunto de dados
\emph{NI00168} - gráfico circular do consumo dos equipamentos.}
\label{fig:ni00168_app_pie}
\end{sidewaysfigure}

\FloatBarrier
\subsection{Conjunto de dados \emph{NI00171}}

Este conjunto de dados é bastante similar ao conjunto \emph{NI00168},
porém há a distinção quanto a alguns equipamentos presentes nesses
conjuntos. A diferença mais notável dos dois conjuntos é a presença de
eletrodomésticos com outros uso-finais distintos de iluminação, sendo
eles: ferro de passar roupas, computador portátil, ventilador e
liquidificador.  Os equipamentos de destaque nesse conjunto de dados
são o ventilador e o computador portátil, onde este apresenta mudanças
de estados com pequenos degraus de potência, como pode ser observado
na Figura~\ref{fig:ni00171_app_time} e aquele se constitui de uma
\acs{c5} com múltiplas oscilações de consumo durante o seu
acionamento, que ocorre em dois estágios exibidos na
Figura~\ref{fig:ni00171_laptop}. Os eventos de transitório nessa
figura têm sua média retirada para permitir a comparação das múltiplas
mudanças de estado presentes do equipamento, como os casos dos
equipamentos no conjunto de dados \emph{Temporizado}, disponíveis nas 
Figuras~\ref{fig:temporizado_geladeira}-\ref{fig:temporizado_televisao}.

\begin{itemize}
\item Lâmpada incandescente 100 W;
\item Quatro equipamentos desconhecidos;
\item Lâmpada fluorescente 18 W, 26 W, 40 W (tubular), 32 W (tubular),
28 W (tubular);
\item Secador de cabelo (4 eventos: Desligado $\rightarrow$ Fraco
$\rightarrow$ Forte $\rightarrow$ Fraco $\rightarrow$ Desligado);
\item Televisão (3 eventos: Desligado $\rightarrow$ Ligado
$\rightarrow$ Ligado e exibindo imagem $\rightarrow$
Desligado);
\item Geladeira;
\item Ferro de passar roupas;
\item Computador portátil (constitui uma \acs{c5} durante mudanças do
uso do processamento, que neste conjunto de dados ocorre apenas
durante o seu acionamento);
\item Liquidificador;
\item Ventilador (6 eventos: Desligado $\rightarrow$ Forte $\rightarrow$
Médio $\rightarrow$ Fraco $\rightarrow$ Médio $\rightarrow$ Forte
$\rightarrow$ Desligado);
\item Ar condicionado (4 eventos: Desligado $\rightarrow$ Ventilação
$\rightarrow$ Ventilação e compressor $\rightarrow$ Ventilação
$\rightarrow$ Desligado, constitui uma \acs{c5}).
\end{itemize}
 
Todos os arquivos NI00*** foram coletados utilizando o medidor do
\acs{cepel}.

\begin{sidewaysfigure}[p]
\centering
\includegraphics[width=\textwidth]{imagens/NI00171_Overview.pdf}
\caption{Perfil de consumo agregado para o conjunto de dados
\emph{NI00171}.}
\label{fig:ni00171_overview}
\end{sidewaysfigure}

\begin{sidewaysfigure}[p]
\centering
\includegraphics[width=\textwidth]{imagens/NI00171_AppTime.pdf}
\caption{Informação no gabarito para o conjunto de dados
\emph{NI00171} - consumo temporal dos equipamentos.}
\label{fig:ni00171_app_time}
\end{sidewaysfigure}

\begin{sidewaysfigure}[p]
\centering
\includegraphics[width=\textwidth]{imagens/NI00171_AppPie.png}
\caption{Informação no gabarito para o conjunto de dados
\emph{NI00171} - gráfico circular do consumo dos equipamentos.}
\label{fig:ni00171_app_pie}
\end{sidewaysfigure}

\begin{sidewaysfigure}[p]
\centering
\includegraphics[width=\textwidth]{imagens/NI00171_App_Laptop.pdf}
\caption{Informação no gabarito para o conjunto de dados
\emph{NI00171} - envoltória para as diversas variáveis para o
computador portátil.}
\label{fig:ni00171_laptop}
\end{sidewaysfigure}


\FloatBarrier
\subsection{Conjunto de dados \emph{NI00173}, \emph{NI00174},
\emph{NI00175} e \emph{NI00177}}

Estes conjuntos de dados apresentam acionamentos de equipamentos
simultaneamente, porém isso é realizado de maneira simples, apenas com
a operação de dois equipamentos atuando ao mesmo tempo. Em ambos os
casos é utilizado uma lâmpada incandescente (100 W) em conjunto com um
outro aparelho para realizar a operação conjunta de equipamentos. No
caso do conjunto de dados \emph{NI00173}, o equipamento operando em
conjunto com a lâmpada incandescente é um ventilador acionado em um
único estado (pode ser modelado como uma \acs{c3}, apesar de ser uma
\acs{c2}), enquanto nos demais conjuntos o equipamento é um secador de
cabelo que opera em dois estados (Forte e Fraco, e, portando, atua
como uma \acs{c2}). A única diferença entre os arquivos
\emph{NI00174}, \emph{NI00175} e \emph{NI00177} é a ordem na qual o
secador de cabelo e a lâmpada têm seus estados de operação alterados.
Para o conjunto \emph{NI00174}, a lâmpada permanece ligada enquanto o
estado de operação do secador é alterado.  Os conjuntos \emph{NI00175}
e \emph{NI00177} alteram a operação dos estados desses equipamentos de
diversas modos diferentes, tentando simular diversas configurações em
que esses aparelhos poderiam ter seus estados alterados em uma
residencia. As
Figuras~\ref{fig:ni00173_overview}-\ref{fig:ni00177_app_time} permitem
observar esses comportamentos, contendo o consumo agregado nesses
conjuntos de dados e a informação estimada do consumo desagregado em
seus respectivos gabaritos.

Todos os arquivos NI00*** foram coletados utilizando o medidor do
\acs{cepel}.


\begin{sidewaysfigure}[p]
\centering
\includegraphics[width=\textwidth]{imagens/NI00173_Overview.pdf}
\caption{Perfil de consumo agregado para o conjunto de dados
\emph{NI00173}.}
\label{fig:ni00173_overview}
\end{sidewaysfigure}

\begin{sidewaysfigure}[p]
\centering
\includegraphics[width=\textwidth]{imagens/NI00173_AppTime.pdf}
\caption{Informação no gabarito para o conjunto de dados
\emph{NI00173} - consumo temporal dos equipamentos.}
\label{fig:ni00173_app_time}
\end{sidewaysfigure}

\begin{sidewaysfigure}[p]
\centering
\includegraphics[width=\textwidth]{imagens/NI00174_Overview.pdf}
\caption{Perfil de consumo agregado para o conjunto de dados
\emph{NI00174}.}
\label{fig:ni00174_overview}
\end{sidewaysfigure}

\begin{sidewaysfigure}[p]
\centering
\includegraphics[width=\textwidth]{imagens/NI00174_AppTime.pdf}
\caption{Informação no gabarito para o conjunto de dados
\emph{NI00174} - consumo temporal dos equipamentos.}
\label{fig:ni00174_app_time}
\end{sidewaysfigure}

\begin{sidewaysfigure}[p]
\centering
\includegraphics[width=\textwidth]{imagens/NI00175_Overview.pdf}
\caption{Perfil de consumo agregado para o conjunto de dados
\emph{NI00175}.}
\label{fig:ni00175_overview}
\end{sidewaysfigure}

\begin{sidewaysfigure}[p]
\centering
\includegraphics[width=\textwidth]{imagens/NI00175_AppTime.pdf}
\caption{Informação no gabarito para o conjunto de dados
\emph{NI00175} - consumo temporal dos equipamentos.}
\label{fig:ni00175_app_time}
\end{sidewaysfigure}

\begin{sidewaysfigure}[p]
\centering
\includegraphics[width=\textwidth]{imagens/NI00177_Overview.pdf}
\caption{Perfil de consumo agregado para o conjunto de dados
\emph{NI00177}.}
\label{fig:ni00177_overview}
\end{sidewaysfigure}

\begin{sidewaysfigure}[p]
\centering
\includegraphics[width=\textwidth]{imagens/NI00177_AppTime.pdf}
\caption{Informação no gabarito para o conjunto de dados
\emph{NI00177} - consumo temporal dos equipamentos.}
\label{fig:ni00177_app_time}
\end{sidewaysfigure}

\FloatBarrier
\subsection{Conjunto de dados \emph{Temporizado}}
\label{ssec:temp}

Este conjunto possui a operação de cinco aparelhos que podem ser
modelados como \acs{c3} por atuarem apenas em um único estado quando
operando. Uma característica deste conjunto é a presença de um alto
nível de ruído durante a operação de uma \acs{c5}, a televisão LCD.
Outra peculiaridade deste conjunto é a sua longa duração, cerca de 18
horas foram coletadas. Além disso, ele possui dois gabaritos, motivo
detalhado um pouco mais a adiante. A atuação dos equipamentos foi
feito através de chaveamento automático, onde tentou-se gerar eventos
próximos de acionamentos e desacionamentos dos equipamentos. Um total
de 149 eventos de transitório estão presentes no conjunto, causados
por estes equipamentos (sempre na sequência Desligado $\rightarrow$
Ligado $\rightarrow$ Desligado):

\begin{itemize}
\item Televisão LCD (6 eventos de transitório), constitui-se de uma \acs{c5};
\item Geladeira (123 eventos, o número impar de eventos é causado pelo
fato da geladeira já estar operando quando a medição do conjunto é
iniciada);
\item Lâmpada fluorescente 23 W (4 eventos), 54 W (4 eventos);
\item Ventilador (12 eventos).
\end{itemize}

O consumo agregado pode ser observado na
Figura~\ref{fig:temporizado_app_time}. A informação contida no
gabarito para este conjunto de dados pode ser observada nas figuras
\ref{fig:temporizado_app_time}--\ref{fig:temporizado_televisao}.
A Figura~\ref{fig:temporizado_app_time} contém a informação do consumo
temporal dos equipamentos, enquanto a
Figura~\ref{fig:temporizado_app_pie} contém o gráfico circular do
consumo estimado no gabarito para os equipamentos. As figuras
\ref{fig:temporizado_geladeira}--\ref{fig:temporizado_televisao}
contêm os transitórios dos equipamentos marcados pelo usuário durante
a criação do gabarito. Todos os eventos são movidos para obterem média
zero, de forma que os eventos de transitório estejam centrados no mesmo
patamar e seja possível compará-los. A informação contida nesse
gráfico auxilia a identificar eventos no gabarito que fogem do padrão,
seja por erro do usuário no preenchimento, ou por caracterizar um
evento excêntrico, facilitando a identificação desses casos. Um
exemplo pode ser observado na Figura~\ref{fig:temporizado_ventilador},
onde há a ocorrência de um evento, marcado por uma etiqueta (a mesma
torna possível a identificação do evento através de sua chave), que
foge do padrão dos outros coletados. Essas figuras também permitem
observar a quantidade de eventos para cada alteração de estado
(indicado entre parênteses no título das subfiguras). Sua medição foi
realizada com o medidor \emph{Yokogawa}.

Este conjunto é o que tem a maior presença de ruído, causado pela
televisão LCD, em especial para os períodos das 00:00 às 02:00 e 04:30
às 08:00 do dia 21. Durante esses períodos, há a ocorrência de uma
pertubação que se assemelha com a operação de um equipamento \acs{c3},
como exibido na Figura~\ref{fig:temporizado_disturbio}. Durante a
aplicação da metodologia descrita na Seção~\ref{sec:aplic_es} e ao
observar os resultados presentes no Capítulo~\ref{chap:resultados},
observou-se uma alta presença de falsos alarmes nesse conjunto para
todas os parâmetros examinados, o que levou a descoberta dessa
pertubação. Fica então a questão de como tratá-la, se essa pertubação
deve ser considerado como um estado interno de operação da televisão
LCD, ou se esse distúrbio deve ser ignorado e tratado como falso
alarme. Para permitir a análise de ambas situações, gerou-se o
gabarito para este conjunto de dados com os dois casos, o primeiro
gabarito contendo apenas os acionamentos e desacionamentos registrados
pelo \acs{cepel} (149 eventos de transitório), e o segundo contém,
também, os distúrbios observados na
Figura~\ref{fig:temporizado_disturbio} (211 eventos).

\begin{sidewaysfigure}[p]
\centering
\includegraphics[width=\textwidth]{imagens/Temporizado_Overview.pdf}
\caption{Perfil de consumo agregado para o conjunto de dados \emph{Temporizado}.}
\label{fig:temporizado_overview}
\end{sidewaysfigure}

\begin{sidewaysfigure}[p]
\centering
\includegraphics[width=\textwidth]{imagens/Temporizado_AppTime.pdf}
\caption{Informação no gabarito para o conjunto de dados
\emph{Temporizado} - consumo temporal dos equipamentos.}
\label{fig:temporizado_app_time}
\end{sidewaysfigure}

\begin{sidewaysfigure}[p]
\centering
\includegraphics[width=.5\textwidth]{imagens/Temporizado_AppPie.pdf}
\caption{Informação no gabarito para o conjunto de dados
\emph{Temporizado} - gráfico circular do consumo dos equipamentos.}
\label{fig:temporizado_app_pie}
\end{sidewaysfigure}

\begin{sidewaysfigure}[p]
\centering
\includegraphics[width=\textwidth]{imagens/Temporizado_App_Geladeira.pdf}
\caption{Informação no gabarito para o conjunto de dados
\emph{Temporizado} - envoltória para as diversas variáveis para a
geladeira.}
\label{fig:temporizado_geladeira}
\end{sidewaysfigure}

\begin{sidewaysfigure}[p]
\centering
\includegraphics[width=\textwidth]{imagens/Temporizado_App_Ventilador.pdf}
\caption{Informação no gabarito para o conjunto de dados
\emph{Temporizado} - envoltória para as diversas variáveis para a
ventilador.}
\label{fig:temporizado_ventilador}
\end{sidewaysfigure}

\begin{sidewaysfigure}[p]
\centering
\includegraphics[width=\textwidth]{imagens/Temporizado_App_LF23W.pdf}
\caption{Informação no gabarito para o conjunto de dados
\emph{Temporizado} - envoltória para as diversas variáveis para a
lâmpada fluorescente 23W.}
\label{fig:temporizado_lf23}
\end{sidewaysfigure}

\begin{sidewaysfigure}[p]
\centering
\includegraphics[width=\textwidth]{imagens/Temporizado_App_LF54W.pdf}
\caption{Informação no gabarito para o conjunto de dados
\emph{Temporizado} - envoltória para as diversas variáveis para a
lâmpada fluorescente 54W.}
\label{fig:temporizado_lf54}
\end{sidewaysfigure}

\begin{sidewaysfigure}[p]
\centering
\includegraphics[width=\textwidth]{imagens/Temporizado_App_Televisao.pdf}
\caption{Informação no gabarito para o conjunto de dados
\emph{Temporizado} - envoltória para as diversas variáveis para a
televisão.}
\label{fig:temporizado_televisao}
\end{sidewaysfigure}

\begin{sidewaysfigure}[p]
\centering
\includegraphics[width=\textwidth]{imagens/temporizado_disturbio.pdf}
\caption[Distúrbio recorrente presente no conjunto de dados \emph{Temporizado}.]
{Distúrbio recorrente presente no conjunto de dados
\emph{Temporizado}. O distúrbio apresenta características semelhantes
a um aparelho \acs{c3} com consumo de 15 W e opera durante cerca de
3 s.}
\label{fig:temporizado_disturbio}
\end{sidewaysfigure}

\FloatBarrier

\subsection{Conjunto de dados \emph{Empilhado4}}
\label{ssec:emp4}

A principal característica deste conjunto é a operação simultânea de
lâmpadas fluorescentes e incandescentes de diversos consumos em
conjunto com outros três equipamentos:
forno elétrico, chuveiro elétrico e televisão CRT. Esses aparelhos
operam um a um em conjunto com as lâmpadas. Este conjunto também é o
que contém a maior quantidade de eventos de transitório de baixo
consumo causado pelas lâmpadas fluorescentes de baixo consumo. A
seguir segue a descrição dos equipamentos presentes neste conjunto de
dados:

\begin{itemize}
\item Forno elétrico: dinâmica de consumo após acionamento (queda
lenta do consumo até estabilizar); 
\item Chuveiro elétrico: dinâmica de consumo após acionamento (flutuações
após acionamento, tendência a estabilizar, porém durante a operação
observada neste conjunto se trata de uma \acs{c5} durante toda sua
atuação);
\item Televisão CRT: dinâmica de consumo permanente;
\item Lâmpada fluorescente (LF) 25 W (2 unidades), 22 W (2 unidades), 15 W (3
unidades), 9 W;
\item Lâmpada incandescente (LI) 40 W (2 unidades).
\end{itemize}

Sua medição foi realizada com o medidor do \acs{cepel}. O perfil de seu
consumo pode ser visualizado na Figura~\ref{fig:empilhado4_overview}.
A informação no gabarito pode ser observada nas figuras
\ref{fig:empilhado4_app_time} e \ref{fig:empilhado4_app_pie}.  A
Figura~\ref{fig:empilhado4_app_time} contém a informação do consumo
temporal dos equipamentos, enquanto a
Figura~\ref{fig:empilhado4_app_pie} contém o gráfico circular do
consumo estimado no gabarito para os equipamentos.

\begin{sidewaysfigure}[p]
\centering
\includegraphics[width=\textwidth]{imagens/Empilhado4_Overview.pdf}
\caption{Perfil de consumo agregado para o conjunto de dados \emph{Empilhado4}.}
\label{fig:empilhado4_overview}
\end{sidewaysfigure}

\begin{sidewaysfigure}[p]
\centering
\includegraphics[width=\textwidth]{imagens/Empilhado4_AppTime.pdf}
\caption{Informação no gabarito para o conjunto de dados
\emph{Empilhado4}: consumo temporal dos equipamentos.}
\label{fig:empilhado4_app_time}
\end{sidewaysfigure}

\begin{sidewaysfigure}[p]
\centering
\includegraphics[width=\textwidth]{imagens/Empilhado4_AppPie.png}
\caption{Informação no gabarito para o conjunto de dados
\emph{Empilhado4}: gráfico circular do consumo dos equipamentos.}
\label{fig:empilhado4_app_pie}
\end{sidewaysfigure}

%\begin{sidewaysfigure}[p]
%\centering
%\includegraphics[width=\textwidth]{imagens/Empilhado4_App_Geladeira.pdf}
%\caption{Informação no gabarito para o conjunto de dados
%\emph{Empilhado4}: envoltória para as diversas variáveis para a
%geladeira.}
%\label{fig:empilhado4_geladeira}
%\end{sidewaysfigure}
\FloatBarrier

\subsection{Conjunto de dados \emph{Empilhado7}}
\label{ssec:emp7}

Este é o conjunto com a maior operação simultânea de equipamentos de
diferentes uso-finais. Há a ocorrência de um evento de transitório
aonde ocorre o desacionamento de quatro equipamentos em conjunto
(geladeira, uma lâmpada incandescente 100 W, e duas lâmpadas
fluorescentes 20 W e 24 W). Outra peculiaridade deste arquivo é a
distorção dos eventos de transitório durante o momento que o ar
condicionado está operando (um total de 13 eventos presentes). Os
equipamentos presentes neste conjunto são:

\begin{itemize}
\item Lâmpada incandescente (LI) 60 W, 100 W;
\item Lâmpada fluorescente (LF) 20 W, 21 W (circular), 24 W, 26 W, 28
W, 40 W;
\item Secador de cabelo;
\item Ar condicionado (constitui de uma \acs{c5});
\item Sanduicheira;
\item Geladeira (obs: essa geladeira tem consumo bastante superior
àquele utilizada no arquivo \emph{Temporizado});
\item Televisão CRT (constitui de uma \acs{c5}).
\end{itemize}

Sua medição foi realizada com o medidor do \acs{cepel}.  A informação
no gabarito pode ser observada nas figuras
\ref{fig:empilhado7_app_time}--\ref{fig:empilhado7_app_pie}.  A
Figura~\ref{fig:empilhado7_app_time} contém a informação do consumo
temporal dos equipamentos, enquanto a
Figura~\ref{fig:empilhado7_app_pie} contém o gráfico circular do
consumo estimado no gabarito para os equipamentos.


\begin{sidewaysfigure}[p]
\centering
\includegraphics[width=\textwidth]{imagens/Empilhado7_Overview.pdf}
\caption{Perfil de consumo agregado para o conjunto de dados \emph{Empilhado7}.}
\label{fig:empilhado7_overview}
\end{sidewaysfigure}

\begin{sidewaysfigure}[p]
\centering
\includegraphics[width=\textwidth]{imagens/Empilhado7_AppTime.pdf}
\caption{Informação no gabarito para o conjunto de dados
\emph{Empilhado7}: consumo temporal dos equipamentos.}
\label{fig:empilhado7_app_time}
\end{sidewaysfigure}

\begin{sidewaysfigure}[p]
\centering
\includegraphics[width=\textwidth]{imagens/Empilhado7_AppPie.png}
\caption{Informação no gabarito para o conjunto de dados
\emph{Empilhado7}: gráfico circular do consumo dos equipamentos.}
\label{fig:empilhado7_app_pie}
\end{sidewaysfigure}

\FloatBarrier

\section[Aplicação do ES para Otimização do Detector de Eventos]{
Aplicação do \acf{es} para Otimização do Detector de Eventos}
\label{sec:aplic_es}

Esta seção se dedica a descrição da metodologia final adotada para a
detecção de eventos por este trabalho que foi aplicada na base de
dados descrita na seção anterior. A metodologia base é aquela
explicada na Seção~\ref{ssec:met_cepel}, proposta pelo \acs{cepel}
para a detecção de eventos de transitório. Resumidamente, a
metodologia utiliza um filtro para convoluir a derivada de Gaussiana
com o sinal de corrente, que irá gerar regiões sensibilizadas caso a
resposta do filtro ultrapasse um limiar. O ponto de inflexão dessas
regiões representam candidatos a evento de transitório, que serão
aceitos caso não tenham sido gerados por ruído ou estejam
demasiadamente próximos a outros candidatos. Porém, a mesma foi
reimplementada de acordo com as considerações realizadas no
Capítulo~\ref{chap:framework} que, além de diversas alterações
técnicas, trouxeram três alterações:

\begin{itemize}
\item Remoção de eventos próximos utilizando sua média (ver
pp.~\pageref{text:media});
\item Remoção de eventos inconsistentes (ver
pp.~\pageref{text:incosistentes});
\item Ajuste automático dos parâmetros através de um otimizador, que
neste trabalho se trata da versão multiespécie do \acs{es}
implementado e detalhado na Seção~\ref{sec:otimizacao}.
\end{itemize}

Antes de entrar no mérito do ajuste automático, este trabalho também
realizou um ajuste empírico dos parâmetros como era feito pelo
\acs{cepel} na sua metodologia original, que será referido como
\emph{Ajuste Manual}. Isso foi realizado para permitir uma melhor
afinidade com o ambiente de análise, uma vez que a utilização
unicamente de parâmetros otimizados automaticamente não permite a
compreensão de nuances da operação e capacidade da metodologia. Apesar
de ser realizada manualmente, algumas diferenças podem ser citadas em
relação ao ajuste realizado pelo \acs{cepel}. A primeira, e mais
importante, é que o ajuste foi realizado para os conjuntos de dados
\emph{Temporizado}, que contém a presença mais forte de ruídos.  Além
disso, diferente do ajuste automático de parâmetros que será explicada
a seguir e opera ajustando vários parâmetros citados no início deste
capítulo (pp.~\pageref{item:parametros}), o \emph{Ajuste Manual} foi
realizado apenas para o $\delta_{min}$, enquanto os demais parâmetros
são os mesmos empregados pela metodologia do \acs{cepel}. Contudo, a
versão do algoritmo já permitia a utilização de novas maneiras de
remoção de evento, que foram analisadas e optou-se empregar a versão
com remoção de eventos inconsistentes e utilização da média para
remoção de eventos próximos.

A metodologia original do \acs{cepel} já apresentava cinco diferentes
maneiras para a remoção de candidatos, indicadas na
Figura~\ref{fig:cepel_transitorio}. A adição de duas novas maneiras de
remoção de candidatos aumentou a quantidade de possibilidades de
configurações a serem analisadas com relação a quais exames empregar
para a remoção de candidatos. Não se espera que a questão dos exames
realizados para a remoção seja um fator chave no desempenho da
metodologia, e sim o valor dos parâmetros empregados. Porém, não se sabe
\emph{a priori} qual é a configuração mais indicada para a operação,
sendo necessário realizar comparações entre elas para determinar qual
é a mais adequada. Ao invés de realizar a otimização de cada
configuração individualmente e então determinar qual teve a melhor
convergência, implementou-se a versão multiespécie do \acs{es} (ver
pp.~\pageref{sssec:multiespecie}) que
permite dinâmica na quantidade de esforço computacional reservado para
cada configuração. Assim, analisar-se-á diversas configurações, cada
uma representada como uma espécie competindo durante a evolução no
\acs{es}, que terão maior ou menor esforço computacional dedicado
conforme sua melhor ou pior evolução.

Ainda assim, a avaliação de uma estratégia em força bruta --- aqui se
referindo a otimizar todas as possíveis configurações e identificar a
melhor convergência delas --- não parece ser a melhor maneira de
abordar o problema, mesmo que houvesse capacidade computacional para
realizar tal tarefa em tempo hábil. Percebeu-se a necessidade de
realizar a escolha de algumas configurações a serem testadas para
reduzir a quantidade de caminhos possíveis.  
Durante a realização do \emph{Ajuste Manual}, percebeu-se que os
eventos removidos devido à inconsistência eram sempre eventos de falso
alarme, mas não havia a ocorrência de perda de eventos de detecção
causados por esse tipo de remoção. Por isso, determinou-se que a
remoção de eventos de inconsistentes seria sempre realizada durante o
ajuste de parâmetros. Por outro lado, a remoção de eventos devido à
ruído era importante para remoção de pertubações rápidas geradas na
rede que não constituíam na mudança do patamar operativo da rede e,
com base nisso, determinou-se que a remoção de eventos ruidosos também
sempre seria realizada. Assim, a variável $\Delta I_{min}$ sempre
estará presente no material genético das espécies, bem como o $\sigma$
da Gaussiana a ser utilizada no filtro de derivada de Gaussiana e o
seu valor de corte $\delta_{min}$. Já para o caso da remoção de
eventos próximos, não era possível determinar se sua utilização era
necessária, nem qual das versões de remoção --- por média, ou sem
deslocamento --- era o que mais se adequada ao problema. No caso de
não usar corte, a variável $n_{min}$ não estará presente no material
genético das espécies, porém no caso oposto as espécies há a
ocorrência de um gene a mais, cujo fenótipo é inteiro. Ainda assim,
sua representação no genótipo é realizada por números reais, sendo
preciso determinar como codificar a informação. Simplesmente se
escolheu arredondar o valor na representação para obter o valor do
fenótipo. Finalmente, também não se sabia \emph{a priori} determinar
se a ordem de aplicação das remoções pode causar alguma influência na
capacidade de detecção e, apesar de não se esperar grandes diferenças
devido à mudança da ordem em que eles seriam removidos, decidiu-se
testar ambas configurações.  

Assim, decidiu-se empregar cinco configurações a serem analisadas, que
determinam as espécies utilizadas no \acs{es}. As mesmas estão
descritas no início deste capítulo, constituindo as espécies
\ref{item:esp1}--\ref{item:esp5} na pp.~\pageref{item:esp1}.

Além da questão das configurações, era necessário determinar a
inicialização. Para garantir melhor convergência, empregou-se
fronteiras mínimas e máximas para cada um dos valores representados
pelos genes, sendo estas:

\begin{subequations}\label{eq:fronteiras}
\begin{equation}\label{eq:fronteira_sigma}
0,003 < \sigma_{gauss} < 0,5
\end{equation}
\begin{equation}\label{eq:fronteira_delta}
0,009 < \delta_{min} < 0,5
\end{equation}
\begin{equation}\label{eq:fronteira_dimin}
0,0 < \Delta{I}_{min} < 1,0
\end{equation}
\begin{equation}\label{eq:fronteira_nmin}
5,0 < n_{min} < 500,0\footnote{Apesar do $n_{min}$ ser inteiro na
representação no fenótipo, sua representação é por número reais no
genótipo por estar utilizando-se do algoritmo genético por \acs{es}.
Também se reitera, aqui, que esse gene só estará presente nas
configurações que realizam a remoção de eventos próximos.}
\end{equation}
\end{subequations}

A inicialização foi aleatória uniforme dentro das fronteiras descritas
em \ref{eq:fronteiras}. Para o parâmetro da estratégia evolutiva, porém,
iniciou-se esse valor de acordo com \ref{eq:sigma_init}, garantindo
que inicialmente cada individuo tenha, em média, 95\% de chance de
explorar uma região no interior a 10\% do espaço de solução permitido. Os
limites inferiores e superiores foram determinados por
\ref{eq:sigma_min} e \ref{eq:sigma_max}, respectivamente. Os valores
de $\sigma_{min,i}$ e $\sigma_{max,i}$ garantem que, no mínimo, o material
genético sofrerá pertubações dentro da região de 0,01\% da região
de solução em 67\% dos casos, enquanto o mesmo será perturbado no máximo 
em 30\% da região para a mesma probabilidade:

\begin{subequations}
\begin{equation}\label{eq:sigma_init}
2\sigma_{init,i}=0,1\Delta x_i
\end{equation}
\begin{equation}\label{eq:sigma_min}
\sigma_{min,i}=1\times10^{-4}\Delta x_i
\end{equation}
\begin{equation}\label{eq:sigma_max}
\sigma_{max,i}=3\times10^{+3}\sigma_{min,i}
\end{equation}
\end{subequations}

\noindent onde $\Delta x_i$ é a região do espaço dentro das fronteiras
para a i-ésima variável.

Além disso, há de escolher os parâmetros livres no cálculo da aptidão
explicitada em \ref{eq:regra_pontuacao}. Os parâmetros $\gamma_{det}$
e $\gamma_{fa}$ permitem priorizar uma melhor taxa de detecção ou
menor taxa de falso alarme, em detrimento de um acréscimo na taxa de
falso alarme ou decréscimo na taxa de detecção, respectivamente. Isso
pode ser feito dando valores maiores ou menores para a recompensa de
encontrar corretamente um evento de transitório ou a penalidade de
aceitar um evento que se constitui de falso alarme.
Os valores utilizados foram a unidade para $\gamma_{det}$ e -0,9 para
$\gamma_{fa}$. O objetivo dessa diferença é uma leve priorização para
a detecção de eventos em detrimento de uma maior taxa de falso alarme.
O valor para $\gamma_{rem}$ foi de -0,05. Esse parâmetro é importante
para garantir a obtenção de soluções mais simples, que exploram a
capacidade do filtro de Gaussiana em gerar o mínimo possível de
candidatos, o que reduz a carga da remoção de eventos através de
cortes em $\Delta{I}_{min}$ ou $n_{min}$. Ao mesmo tempo, 
a geração de menos candidatos significa que a solução exige menor
esforço computacional, e é mais adequada para a aplicação na operação
em tempo real. Ainda, o parâmetro \acs{jmax} foi definido como 50
amostras. 

Finalmente, cabe ainda decidir qual dos métodos de competição para as
espécies será utilizado e os tamanhos das populações. Como a
implementação do método de competição intraespécies, da maneira
atualmente disponível neste trabalho, acaba favorecendo espécies com
carga genética mais simples de serem otimizadas ou que começaram em
condições privilegiadas --- quando sem o mecanismo de adiamento da
competição ---, irá optar-se pela competição intraespécie. Para
\acs{mu}, utilizou-se o valor de inicial de 30 indivíduos para cada
uma das espécies. Como hão cinco espécies que serão otimizadas, a
população global é de 150 indivíduos. Cada indivíduo gerará 7 filhos,
de forma que o valor de \acs{lambda} global é de 1050 indivíduos.






  \chapter{Resultados}%
\label{chap:resultados}

A metodologia (Seção~\ref{sec:aplic_es}) foi aplicada na base de dados
(Seção~\ref{sec:base_de_dados}), ambos descritos no capítulo anterior,
para verificar a capacidade da versão multiespécie implementada do
\acs{es} (Subseção~\ref{sssec:multiespecie}) de realizar o ajuste
automático de parâmetros para os diferentes cenários simulados nos
conjuntos de dados. Como dito durante o detalhamento da metodologia, antes do
ajuste automático dos
parâmetros, também se realizou um ajuste empírico para
os novos cenários presentes no conjunto de dados \emph{Temporizado
Gabarito 1}, que representa um cenário com atuação simultânea de
cinco equipamentos e operação de um equipamento com dinâmica de carga
--- os equipamentos \acs{c5} (ver definições na
Subseção~\ref{ssec:modelos_carga}). Esse ajuste será referido como
\emph{Ajuste Manual}, enquanto o ajuste empírico realizado pelo
\acs{cepel}, que foi feito levando em consideração os conjuntos de
dados \emph{NI00***}, será referido como \emph{Ajuste CEPEL} (mais
detalhes na discussão do método, na Subseção.~\ref{ssec:met_cepel}). Por sua
vez, foram realizados três ajustes automáticos, todos seguindo o método
descrito na Seção~\ref{sec:aplic_es}. Estas são as características e objetivos
de cada um desses ajustes:

\begin{itemize}
\item \emph{ES 1}: a otimização foi realizada alimentando o \acs{es} com
os conjuntos de dados \emph{NI00***}. O objetivo é permitir uma
comparação entre o otimizador e o \emph{Ajuste CEPEL};
\item \emph{ES 2}: neste caso os conjuntos de dados alimentados para o
otimizador foram o \emph{Temporizado Gabarito 1}, \emph{Empilhado4} e
\emph{Empilhado7}. Assim, é possível observar o comportamento dos
métodos quando otimizada para os dados que representam uma
simulação mais real das características presentes em uma rede
residencial;
\item \emph{ES 3}: já a última configuração alimentou o \acs{es}
apenas com o conjunto \emph{Temporizado Gabarito 1} e
\emph{Empilhado4} para analisar a capacidade de generalização do
algoritmo.
\end{itemize}

Não obstante, aplicou-se o Detector de Patamar Elaborado desenvolvido
pelo trabalho de \citet*{nilm_cepel_alvaro} para verificar o
comportamento do mesmo nos novos dados. Porém, não foi realizado um
novo ajuste de seus parâmetros, de forma que os resultados mostrados
para esse detector não representam sua real capacidade.

Um detalhe que aqui deve ser recapitulado é a presença de dois
gabaritos para o conjunto \emph{Temporizado}. Isso se deve à presença
de um distúrbio que tem características similares a uma \acs{c3}
exibido na Figura~\ref{fig:temporizado_disturbio}, que está
presente apenas na informação do gabarito \emph{Temporizado Gabarito
2} (ver Subseção~\ref{ssec:temp} para mais detalhes).

\section{Otimização Automática dos Parâmetros}
\label{sec:otim_es}

Na Tabela~\ref{tab:resultados}, encontram-se os
resultados os ajustes comentados anteriormente no início deste
capítulo, exceto o Detector de Patamar Elaborado que será apresentado
mais adiante.

Fica evidente a maior sensibilidade da escolha dos parâmetros para o
caso \emph{Ajuste CEPEL}, para o qual ocorre uma maior capacidade de detecção,
em especial para o \emph{Empilhado4} que contém eventos de
equipamentos de menor consumo (lâmpadas fluorescentes). Porém, isso
também reflete em uma taxa de falso alarme excessiva para o conjunto
\emph{Temporizado} aonde, mesmo na sua versão com o gabarito contendo o
distúrbio descrito, ocorrem 234 eventos de falso alarme. A alta
sensibilidade dos valores também é refletida no conjunto
\emph{Empilhado7}, que, assim como o \emph{Temporizado}, contém a
presença de duas \acs{c5}. Porém, o arquivo é mais curto e conta com a
televisão operando durante apenas 30 min e ar condicionado por cerca
de 20 min, enquanto a televisão no conjunto de dados
\emph{Temporizado} opera por $\sim12$~h. Já os conjuntos
\emph{NI00***}, para os quais foi realizado o ajuste dessa
configuração, apresentam reconstrução praticamente perfeita, exceto
pela presença de dois falsos alarmes no conjunto \emph{NI00171}. A
Figura~\ref{fig:ni171_cepel} apresenta a ocorrência desses dois falsos
alarmes, que ocorrem justamente após o acionamento do computador
portátil aonde há diversas oscilações até a estabilização do consumo.
As taxa global de detecção e falso alarme para os conjuntos
\emph{NI00***} foram respectivamente de 100~\% e 1,2~\%, enquanto
esses valores de 90,9~\% e 111,7~\% para os conjuntos
\emph{Temporizado Gabarito 1}, \emph{Empilhado4} e \emph{Empilhado7}.
No caso do \emph{Temporizado} ser utilizado com a segunda versão de
seu gabarito, as taxas são respectivamente de 90,2~\% e 82,9~\%.


\begin{table}[ht!]
\resizebox{\textwidth}{!}{
\begin{tabular}{>{\centering}m{3cm}>{\centering}m{1.3cm}cccccccccc}
\hline \hline \hline
\multicolumn{2}{c}{\parbox[t]{4.3cm}{\centering Conjunto de Dados}} &
\multicolumn{2}{c}{\textbf{ES 1}} &
\multicolumn{2}{c}{\textbf{Manual}} &
\multicolumn{2}{c}{\textbf{CEPEL}} &
\multicolumn{2}{c}{\textbf{ES 2}} &
\multicolumn{2}{c}{\textbf{ES 3}}
\tabularnewline \hline
& &
DET & FA &
DET & FA &
DET & FA &
DET & FA &
DET & FA \\
\hline\hline
\multirow{2}{3cm}{\centering\emph{NI00168}
\footnotesize{(39~eventos)}} & \scriptsize{Ocorr.} &
38 & 0 &
33 & 0 &
39 & 0 &
36 & 0 &
23 & 2 \\
 & \scriptsize{Taxa (\%)} &
97,4  & 0,0 &
86,8  & 0,0 &
100,0 & 0,0 &
92,3  & 0,0 &
59,0  & 5,1 \\ \hline
\multirow{2}{3cm}{\centering\emph{NI00171}
\footnotesize{(48~eventos)}} & \scriptsize{Ocorr.} &
47 & 1 &
42 & 0 &
48 & 2 &
43 & 0 &
39 & 2 \\
 & \scriptsize{Taxa (\%)} &
97,9  & 2,1 &
89,4  & 0,0 &
100,0 & 4,2 &
89,6  & 0,0 &
83,0  & 4,2 \\ \hline
\multirow{2}{3cm}{\centering\emph{NI00173}
\footnotesize{(20~eventos)}} & \scriptsize{Ocorr.} &
20 & 1 &
19 & 1 &
20 & 0 &
20 & 1 &
20 & 0 \\
 & \scriptsize{Taxa (\%)} &
100,0 & 5,0 &
95,0  & 5,0 &
100,0 & 0,0 &
100,0 & 5,0 &
100,0 & 0,0 \\ \hline
\multirow{2}{3cm}{\centering\emph{NI00174}
\footnotesize{(8~eventos)}} & \scriptsize{Ocorr.} &
8 & 0 &
7 & 1 &
8 & 0 &
8 & 0 &
7 & 0 \\
 & \scriptsize{Taxa (\%)} &
100,0 & 0,0  &
87,5  & 12,5 &
100,0 & 0,0  &
100,0 & 0,0  &
87,5  & 0,0  \\ \hline
\multirow{2}{3cm}{\centering\emph{NI00175}
\footnotesize{(23~eventos)}} & \scriptsize{Ocorr.} &
23 & 0 &
23 & 0 &
23 & 0 &
23 & 0 &
23 & 0 \\
 & \scriptsize{Taxa (\%)} &
100,0 & 0,0 &
100,0 & 0,0 &
100,0 & 0,0 &
100,0 & 0,0 &
100,0 & 0,0 \\ \hline
\multirow{2}{3cm}{\centering\emph{NI00177}
\footnotesize{(24~eventos)}} & \scriptsize{Ocorr.} &
24 & 0 &
24 & 0 &
24 & 0 &
24 & 0 &
24 & 0 \\
 & \scriptsize{Taxa (\%)} &
100,0 & 0,0 &
100,0 & 0,0 &
100,0 & 0,0 &
100,0 & 0,0 &
100,0 & 0,0 \\ \hline
\multirow{2}{3cm}{\centering\emph{Temp. Gab. 1}
\footnotesize{(149~eventos)}} & \scriptsize{Ocorr.} &
147 & 52  &
147 & 28  &
147 & 259 &
148 & 43  &
148 & 13  \\
 & \scriptsize{Taxa (\%)} &
98,7 & 34,9  &
98,7 & 18,8  &
98,7 & 173,8 &
99,3 & 28,9  &
99,3 & 8,7 \\ \hline
\multirow{2}{3cm}{\centering\emph{Temp. Gab. 2}
\footnotesize{(211~eventos)}} & \scriptsize{Ocorr.} &
191 &  8  &
174 &  1  &
201 & 234 &
188 &  3  &
161 &  1 \\
 & \scriptsize{Taxa (\%)} &
90,5 & 3,8   &
82,5 & 0,0   &
95,3 & 110,9 &
89,1 & 1,4   &
76,3 & 0,5 \\ \hline
\multirow{2}{3cm}{\centering\emph{Empilhado4}
\footnotesize{(74~eventos)}} & \scriptsize{Ocorr.} &
42 & 1 &
24 & 0 &
56 & 10 &
37 & 0 &
34 & 0 \\
 & \scriptsize{Taxa (\%)} &
56,8 & 1,4  &
32,4 & 0,0  &
78,0 & 12,3 &
50,0 & 0,0  &
45,9 & 0,0 \\ \hline
\multirow{2}{3cm}{\centering\emph{Empilhado7}
\footnotesize{(42~eventos)}} & \scriptsize{Ocorr.} &
39 & 1  &
35 & 1  &
38 & 27 &
37 & 0  &
34 & 1 \\
 & \scriptsize{Taxa (\%)} &
92,9 & 2,4  &
83,3 & 2,4  &
90,5 & 64,3 &
88,1 & 0,0  &
80,9 & 2,4 \\
\hline \hline
\end{tabular}}
\caption[Taxa de detecção de eventos de transitório e falso
alarme para os três ajustes automáticos e os dois ajustes manuais.]{
Taxa de detecção de eventos de transitório e falso alarme
para os três ajustes automáticos e os dois ajustes manuais.  As
configurações Manual e CEPEL referem-se respectivamente aos casos
determinados empiricamente pelo autor do trabalho e pelo grupo do
\gls{cepel}, sendo o primeiro ajustado para os dados
\emph{Temporizado Gabarito 1} e
o segundo para os dados \emph{NI00***}. As configurações ES 1, ES 2 e ES
3 referem-se ao ajuste automático do \acs{es} alimentado
respectivamente pelos conjuntos: \emph{NI00***}; \emph{Temporizado
Gabarito 1}, \emph{Empilhado4} e \emph{Empilhado7}; \emph{Temporizado
Gabarito 1} e \emph{Empilhado7}. As abreviaturas DET e FA referem-se a
taxa de detecção e falso alarme, respectivamente.}
\label{tab:resultados}
\end{table}

\begin{SidewaysFigure}
\centering
\includegraphics[width=\textheight]{imagens/ni171_CepelStandard_FA.pdf}
\caption[Falsos alarmes para a configuração \emph{Ajuste CEPEL} no
conjunto de dados \emph{NI00171}.]
{Falsos alarmes para a configuração \emph{Ajuste CEPEL} no
conjunto de dados \emph{NI00171}. Os eventos marcados por etiquetas
roxas indicam as ocorrências dos falsos alarmes durante o acionamento
do computador portátil antes da estabilização de seu consumo. Observe
que há a remoção de diversos outros candidatos devido à eventos
próximos ou ruído que causariam ocorrência ainda maior de falsos
alarmes caso não houvessem suas remoções.}
\label{fig:ni171_cepel}
\end{SidewaysFigure}

A versão de \emph{Ajuste Manual} foi a menos sensível, na qual se
obteve baixa taxa de detecção para o conjunto de dados
\emph{Empilhado4} que contém as lâmpadas de baixo consumo, mas ao
mesmo tempo garantindo um falso alarme total de apenas 29 ocorrências
(28 ocorrências no \emph{Temporizado Gabarito 1} e 1 no
\emph{Empilhado7}). Levando em conta que o seu ajuste foi realizado
levando em consideração apenas o conjunto de dados \emph{Temporizado
Gabarito 1}, isso mostra que a capacidade de visualização permite
encontrar um valor que reduziria a sensibilidade do detector, sem que
fosse analisado vários casos até a obtenção dos parâmetros.  Porém,
para os conjuntos sem a presença de ruídos, essa menor sensibilidade
se reflete em perdas de detecção nos conjuntos \emph{NI00168} e
\emph{NI00171} que possuem equipamentos de menor consumo.

Antes de discutir a questão das taxas de detecção e falso alarme para
os ajustes automáticos, detém-se a convergência de suas espécies em
cada um dos casos. No ES 1, a espécie que obteve maior aptidão foi a
\ref{item:esp3}, enquanto nas configurações ES 2 e 3, foi a
Espécie~\ref{item:esp2}. As
Figuras~\ref{fig:convergencia_es_1}--\ref{fig:convergencia_es_3}
mostram a evolução das espécies em termo de aptidão para os ES 1, 2 e
3, respectivamente. Observa-se que a Espécie~\ref{item:esp3} nas
otimizações ES 2 e 3, não consegue acompanhar o processo evolutivo das
demais espécies, porém isso não ocorre na configuração ES 1, onde
todas as espécies convergem. Há indícios de que a espécie
\ref{item:esp3} não é adequada para os conjuntos de dados ruidosos,
porém é necessário a realização de mais execuções do algoritmo nessas
condições para obter mais estatística e confirmar isso, uma vez que
esses casos podem ter ocorrido devido à convergência de um máximo
local de aptidão. Porém, não foi possível aumentar essa estatística
devido à demora no tempo de execução das configurações ES 2 e 3, que
levam cerca de uma semana para convergir.

Indo além da questão da Espécie~\ref{item:esp3}, observa-se que não há
grande diferença na aptidão das demais espécies na convergência. Isso
indica que quaisquer outras configurações além da
espécie~\ref{item:esp3} é capaz de resolver igualmente bem o problema.
Um motivo para isso ocorrer é a presença de eventos próximos apenas
nos acionamentos com um pico de consumo, que causa a sensibilização e
geração de um candidato na descida de consumo após o pico. Esses
candidatos são devidamente removidos por inconsistência, método de
remoção adicionado justamente para o tratamento desses casos. Ao
observar o material genético de convergência dos ajustes automáticos,
observa-se que o valor de $n_{min}$ foi de 6, 5 e 495, para as
melhores espécies na convergência dos ES 1, 2
e 3, respectivamente. Exceto para o ES 3,
ocorrência \emph{sui generis} que será comentado mais adiante quando
se referindo a sua taxa de detecção e falso alarme, os valores de
convergência para $n_{min}$ são próximos ao limite inferior, o que
mostra que esse parâmetro não é relevante para o problema. Assim, para
os conjuntos de dados analisados, a remoção de eventos inconsistentes
é suficiente para a remoção de eventos de falso alarme que antes eram
removidos por constituírem de eventos próximos, porém, a remoção de
eventos inconsistentes é mais recomendada uma vez que ela não adiciona
tempo morto de resposta do filtro, que pode causar perda de alvo.

%A convergência para um máximo local pode
%ter ocorrido devido à função avaliada não ser continua no espaço, mas
%por diversos patamares discretos que representam valores maiores ou
%menores conforme a incidência maior ou menor de detecção, falso alarme
%e, em menor escala, eventos removidos. Como a função não é contínua,
%não há uma deriva, uma tendência que possibilite uma exploração do
%espaço de soluções gradual. Enquanto os individuos estão em cima de um
%patamar, eles não conseguem perceber qual direção localmente eles
%devem seguir, simplesmente explorando o espaço aleatóriamente em busca
%de um outro patamar na função, patamar esse que gere maiores
%ocorrências de detecção ou menores falso alarme. Quando um indivíduo
%encontrar uma região com um patamar maior, seu material genético irá
%se espalhar em sua espécie, de forma que a espécie irá subir para o
%novo patamar e voltar a explorá-lo aleatoriamente. Porém, se não
%houverem patamares no alcance da estratégia evolutiva dessa espécie,
%ela não irá conseguir deixar esse patamar, uma vez que todos os
%indivíduos que sairem do patamar serão eliminados. Pelo mesmo motivo,
%a estratégia evolutiva dessa espécie tenderá a reduzir seus valores
%de pertubação no material genético, causando a convergência para um
%máximo de aptidão local.
%
%Com o objetivo de tratar esse problema, propõe-se uma
%otimização em dois níveis, com um ciclo interno ajustando
%especificamente o limiar de corte da derivada de Gaussiana. A
%otimização no nível externo irá conter como material genético todas as
%variáveis exceto o limiar de corte, e irá otimizar com a mesma
%metodologia atual os individuos. Porém, o corte não será fixo, após
%determinar a resposta do filtro, um otimizador local irá procurar pela
%configuração ótima de regiões sensibilizadas em relação ao gabarito.
%Para isso, a mesma função pode ser considerada, com os mesmos valores
%de $\gamma_{det}$ e $\gamma_{fa}$, ou a escolha pode ser de um
%parâmetro para cada caso, permitindo encontrar mais regiões
%sensibilizadas na resposta do filtro quando em comparação com a
%resposta final do detector, justamente para explorar os cortes por
%ruído, distância de amostras e incosistência. Ao invés de
%$\gamma_{det}$ e $\gamma_{fa}$, é possível utilizar outras medidas,
%como aquelas descritas na Subseção~\ref{ssec:nilm_eff_calc}. Ao
%realizar a otimização dessa maneira, retira-se do algoritmo genético
%um grande peso para a descoberta de novas regiões, provavelmente
%garantindo um potencial de melhor convergência.

% FIXME Editar a legenda dessas figuras
\begin{figure}[!htb]
\centering
\includegraphics[width=\textwidth]{imagens/convergence_ES.pdf}
\caption[Convergência das espécies para o ajuste automático da
configuração ES 1.]{Convergência das espécies para o ajuste automático da
configuração ES 1. As espécies e suas cores são: Espécie \ref{item:esp1}:
amarelo; Espécie \ref{item:esp2}: azul; Espécie \ref{item:esp3}: rosa;
Espécie \ref{item:esp4}: verde; Espécie \ref{item:esp5}: marrom.
Observa-se que praticamente não há diferença na aptidão das espécies
na convergência.}
\label{fig:convergencia_es_1}
\end{figure}

\begin{figure}[!htb]
\centering
\includegraphics[width=\textwidth]{imagens/convergence_ES2.pdf}
\caption[Convergência das espécies para o ajuste automático da
configuração ES 2.]{Convergência das espécies para o ajuste automático da
configuração ES 2. As espécies e suas cores são: Espécie \ref{item:esp1}:
amarelo; Espécie \ref{item:esp2}: azul; Espécie \ref{item:esp3}: rosa;
Espécie \ref{item:esp4}: verde; Espécie \ref{item:esp5}: marrom. A
Espécie \ref{item:esp3} não obteve convergência como as demais
espécies.}
\label{fig:convergencia_es_2}
\end{figure}

\begin{figure}[!htb]
\centering
\includegraphics[width=\textwidth]{imagens/convergence_ES3.pdf}
\caption[Convergência das espécies para o ajuste automático da
configuração ES 3.]{Convergência das espécies para o ajuste automático da
configuração ES 3. As espécies e suas cores são: Espécie \ref{item:esp1}:
amarelo; Espécie \ref{item:esp2}: azul; Espécie \ref{item:esp3}: rosa;
Espécie \ref{item:esp4}: verde; Espécie \ref{item:esp5}: marrom. A
Espécie \ref{item:esp3} novamente não obteve convergência como as
demais espécies.}
\label{fig:convergencia_es_3}
\end{figure}

Retornando à questão das taxas de detecção e falso alarme, a
configuração ES 1 obteve eficiência compatível ao \emph{Ajuste CEPEL}
nos conjuntos \emph{NI00***}, com umas poucas diferenças. Há a perda
de alvo de um evento nos conjuntos \emph{NI00168} e \emph{NI00171} que
é causada durante o desacionamento de uma lâmpada com \emph{dimmer},
que ocorre em um processo lento e, portanto, que necessita de uma
maior sensibilidade do filtro. A ocorrência de falso alarme no
conjunto de dados é devida a um ruído próximo ao acionamento do
ventilador, causado por uma má manipulação da chave que seleciona a
velocidade do ventilador e indicado na Figura~\ref{fig:ni173_fa}.
Porém, é ao observar as taxas para os conjuntos \emph{Temporizado
Gabarito 1}, \emph{Empilhado4} e \emph{Empilhado7} que se percebe a
real vantagem do ajuste automático dos parâmetros. Apesar do mesmo não
ter tido acesso à esses conjuntos de dados, sua capacidade de
generalização foi bastante superior àquela apresentada pelo
\emph{Ajuste CEPEL}. Isso é causado pela presença do parâmetro
$\gamma_{rem}$ (ver equação~\ref{eq:regra_pontuacao}, para o cálculo
da aptidão na pp.~\pageref{eq:regra_pontuacao}), que causa uma
tendência maior para a exploração dos limites do limiar de
sensibilização. Ao invés da excessiva ocorrência de falso alarme do
\emph{Ajuste CEPEL} com 259, 10 e 27 casos para esses conjuntos,
obteve-se apenas 52, 1 e 1 casos de falso alarme.  Isso, por sua vez,
trouxe uma menor detecção no conjunto \emph{Empilhado4} que contém os
aparelhos de menor consumo. A taxa de detecção e falso alarme do ES 1
para os conjuntos \emph{NI00***} é de 98,8~\% e 1,2~\%,
respectivamente. Para os conjuntos \emph{Temporizado Gabarito 1},
\emph{Empilhado4} e \emph{Empilhado7} essas taxas são de 86,03~\% e
16,51~\%, enquanto esses valores são de 83,2 \% e 3,0 \% quando
empregando o Gabarito 2.

\begin{SidewaysFigure}
\centering
\includegraphics[width=\textheight]{imagens/173_All_fa.pdf}
\caption[Falso alarme para a configuração \emph{ES 1} e \emph{ES 2} no
conjunto de dados \emph{NI00173}.]{Falso alarme para a configuração
\emph{ES 1} e \emph{ES 2} no conjunto de dados \emph{NI00173}. O mesmo
é causado por uma má manipulação da chave que seleciona a
velocidade do ventilador.}
\label{fig:ni173_fa}
\end{SidewaysFigure}

Em seguida, alimentou-se o otimizador para realizar a otimização dos
parâmetros recebendo os conjuntos de dados ruidosos, na esperança de
que isso trouxesse a redução na taxa de falso alarme ainda elevada no
conjunto \emph{Temporizado Gabarito 1}. Na época, ainda não havia sido
observada a pertubação presente nesse conjunto de dados. A resposta
para essa configuração está na coluna ES 2 da
Tabela~\ref{tab:resultados}. Ao analisar sua taxa de eficiência,
percebeu-se que a taxa de falso alarme, apesar de sofrer redução, não
foi na ordem esperada: o valor se manteve alto, 43 ocorrências de
falso alarme nesse conjunto de dados, o que necessitou de uma
investigação mais profunda.  A descoberta foi a pertubação presente na
Figura~\ref{fig:ruido_temporizado}, e observada em mais detalhes na
Figura~\ref{fig:temporizado_disturbio}. Como foi dito durante a
descrição desse conjunto de dados, essa pertubação tem uma
característica muito marcante, contendo cerca de 2-3s e consumo de 15
W. Em vista disso, gerou-se o segundo gabarito para esse conjunto de
dados, que para o ES 2 apresentou apenas 3 falsos alarmes. Os mesmos
são causados pelo acionamento da televisão, que, como pode ser
observado na Figura~\ref{fig:temporizado_televisao}, possuem dois
estágios durante o seu acionamento, o falso alarme sendo causado
justamente durante a primeira crista de consumo. Por outro lado, ao
considerar essas pertubações como alvo, observa-se que nem todas são
detectadas, o que causa uma queda na taxa de detecção razoável. A
Figura~\ref{fig:temporizado_newTargetLoss} indica algumas perdas de
alvo quando considerando esse distúrbio como de detecção desejável.
Assim, esse é um problema ainda a ser resolvido para a aplicação do
\acs{nilm}, uma vez que da maneira que o mesmo está configurado, não é
possível obter a detecção de uma taxa razoável de eventos, nem de
evitá-los. Porém, o caminho parece ser tratar a ocorrência desses
eventos através de sua detecção. A taxa de detecção e falso alarme do
ES 2 para os conjuntos \emph{Temporizado Gabarito 1},
\emph{Empilhado4}, \emph{Empilhado7} é de 83,8 \% e 16,2 \%,
respectivamente, enquanto as mesmas taxas são de 80,12 \% e 0,9 \%
quando considerando o \emph{Temporizado Gabarito 2}. Já suas taxa
nos conjuntos \emph{NI00***} são de 98,1 \% e 0,6 \%.

\begin{SidewaysFigure}
\centering
\includegraphics[width=\textheight]{imagens/temporizadoFA.pdf}
\caption[Exemplos de falsos alarmes no conjunto de dados
\emph{Temporizado Gabarito 1}.]
{Exemplos de falsos alarmes no conjunto de dados \emph{Temporizado
Gabarito 1}. Os falsos alarmes são as caixas verdes e vermelhas
exibidas. Na subfigura inferior, são mostrados alguns pontos de
inflexão e seus valores em unidades da resposta do filtro, bem como
valores da resposta do filtro para ruídos.}
\label{fig:ruido_temporizado}
\end{SidewaysFigure}

\begin{SidewaysFigure}
\centering
\includegraphics[width=\textheight]{imagens/temp_newTargetLoss.pdf}
\caption{Exemplos de perdas de alvos do distúrbio no conjunto de dados
\emph{Temporizado Gabarito 2}.}
\label{fig:temporizado_newTargetLoss}
\end{SidewaysFigure}

Também se deteve atenção às perdas de alvo no conjunto de dados
\emph{Empilhado4}, o conjunto em que há maior dificuldade de
detectar eventos. Na Figura~\ref{fig:lampadas_emp4},
observa-se que os valores absolutos dos pontos de inflexão,
por volta de $0,006$, estão na ordem do ruído do arquivo
\emph{Temporizado} (ver Figura~\ref{fig:ruido_temporizado}), ou seja,
para detectar esses eventos é necessário descer o patamar do filtro a
um valor que causará uma excessiva ocorrência de falso alarme
nesse conjunto de dados --- caso do \emph{Ajuste CEPEL}, cujo patamar
é de $0,003$. A ocorrência de um grande número de falsos alarmes
acarretará numa perda de aptidão, tornando essas soluções
como indivíduos não desejáveis e, portanto, os mesmos serão eliminados
durante a evolução do \acs{es}.

\begin{SidewaysFigure}
\centering
\includegraphics[width=\textheight]{imagens/emp4_targetloss.pdf}
\caption[Exemplos de perdas de alvo no conjunto de dados
\emph{Empilhado4} para a configuração \emph{ES 2}] {Exemplos de perdas
de alvo no conjunto de dados \emph{Empilhado4} para a configuração
\emph{ES 2}. Os eventos não detectados tem seus pontos de inflexão e
seus valores em unidades da resposta do filtro exibidos. As etiquetas
em preto indicam os valores de suas características, aonde é possível
observar que o evento de maior potencia exibido que constitui perda de
alvo apresenta um degrau em valor absoluto de cerca de 24 W.}
\label{fig:lampadas_emp4}
\end{SidewaysFigure}

Outra otimização foi realizada, desta vez alimentando o algoritmo
genético com os conjuntos \emph{Temporizado Gabarito 1} e
\emph{Empilhado4} para verificar a capacidade de sua generalização
para as condições ruidosas. Os resultados estão representados na
coluna ES 3. Ao observar os valores obtidos para os conjuntos para os
quais essa configuração foi alimentada, percebe-se que houve uma
drástica redução do falso alarme para o conjunto \emph{Temporizado
Gabarito 1}. Esse fato se deve ao parâmetro utilizado pelo \acs{es}
para a remoção de eventos próximos, de 495 amostras, como dito
anteriormente. O valor desse parâmetro indica que há uma taxa de tempo
morto de resposta de aceitação de outro candidato de no mínimo 8,3 s.
Como geralmente não há a ocorrência de eventos causados por
acionamentos de equipamentos próximos nesses dois conjuntos, isso não
se mostrou relevante para o algoritmo genético, que percebeu uma
brecha na fronteira desse parâmetro e descobriu uma configuração que
reduz em quase 30 ocorrências a taxa de falso alarme para o
\emph{Temporizado Gabarito 1} no qual a aptidão de seus indivíduos
estava sendo analisada.

Porém, esse alto tempo morto pode causar a
perda de alvo, como observado nos conjuntos \emph{NI00168} e
\emph{NI00171}, aonde o desacionamento é seguido de um acionamento de
outro equipamento em curtos espaços de tempo, como pode ser observado
nas figuras~\ref{fig:ni00168_overview} e \ref{fig:ni00171_overview}, e
está refletido na perda de eficiência nesses conjuntos, especialmente
no \emph{NI00168}. Assim, fica evidente a necessidade de análise
pós-convergência das configurações resultantes do \acs{es}, aonde o
valor obtido de convergência representa o melhor valor obtido pelo
\acs{es} para o conjunto de dado analisado, porém, pode não
representar uma configuração aplicável à realidade, como o caso do ES
3 que explorou uma brecha na fronteira do parâmetro $n_{min}$ que
permitia a utilização de valores muito elevados.

Além disso, também se empregou a utilização do Detector de Patamar
Elaborado desenvolvido por \citet*{nilm_cepel_alvaro}. Os valores dos
parâmetros dessa configuração não foram alterados, de forma que seus
valores estão exatamente como ajustado em seu trabalho. Os conjuntos de
dados utilizados para o ajuste da técnica por esse autor não continham
a atuação de equipamentos simultaneamente e, portanto, os resultados
para os conjuntos \emph{Temporizado}, \emph{Empilhado4} e
\emph{Empilhado7} não apresentam a real capacidade de operação dessa
técnica. Ainda, como foi dito na
Seção~\ref{ssec:cepel_anteriores}, essa técnica é limitada apenas à
eventos de transitório com acréscimo de consumo. Por isso, foi gerada
uma nova tabela (Tabela~\ref{tab:det_elab_res}) contendo os resultados
das demais configurações anteriores apenas para esses eventos e
adicionada uma nova coluna com os resultados para tal detector.

\begin{table}[ht!]
\resizebox{\textwidth}{!}{
\begin{tabular}{>{\centering}m{3cm}>{\centering}m{1.3cm}cccccccccccc}
\hline \hline \hline
\multicolumn{2}{c}{\parbox[t]{4.3cm}{\centering Conjunto de Dados}} &
\multicolumn{2}{c}{\textbf{ES 1}} &
\multicolumn{2}{c}{\textbf{Manual}} &
\multicolumn{2}{c}{\textbf{CEPEL}} &
\multicolumn{2}{c}{\textbf{ES 2}} &
\multicolumn{2}{c}{\textbf{ES 3}} &
\multicolumn{2}{c}{\textbf{Det.Elab.}}
\tabularnewline \hline
\multicolumn{2}{c}{\emph{Eventos acrésc. consum.}}&
DET & FA &
DET & FA &
DET & FA &
DET & FA &
DET & FA &
DET & FA \\
\hline\hline
\multirow{2}{3cm}{\centering\emph{NI00168}
\footnotesize{(20~eventos)}} & \scriptsize{Ocorr.} &
20 & 0 &
19 & 0 &
20 & 0 &
20 & 0 &
11 & 2 &
20 & 1 \\
 & \scriptsize{Taxa (\%)} &
100,0 & 0,0 &
95,0  & 0,0 &
100,0 & 0,0 &
100,0 & 0,0 &
55,0  & 10,0 &
100,0 & 5,0 \\ \hline
\multirow{2}{3cm}{\centering\emph{NI00171}
\footnotesize{(26~eventos)}} & \scriptsize{Ocorr.} &
26 & 1 &
24 & 0 &
26 & 1 &
24 & 0 &
19 & 2 &
25 & 1 \\
 & \scriptsize{Taxa (\%)} &
100,0 & 3,8 &
92,3  & 0,0 &
100,0 & 3,8 &
92,3  & 0,0 &
73,1  & 7,7 &
96,2  & 3,4\\ \hline
\multirow{2}{3cm}{\centering\emph{NI00173}
\footnotesize{(10~eventos)}} & \scriptsize{Ocorr.} &
10 & 1 &
10 & 0 &
10 & 0 &
10 & 1 &
10 & 0 &
10 & 0 \\
 & \scriptsize{Taxa (\%)} &
100,0 & 10,0 &
100,0 & 0,0  &
100,0 & 0,0  &
100,0 & 10,0 &
100,0 & 0,0  &
100,0 & 0,0 \\ \hline
\multirow{2}{3cm}{\centering\emph{NI00174}
\footnotesize{(4~eventos)}} & \scriptsize{Ocorr.} &
4 & 0 &
4 & 0 &
4 & 0 &
4 & 0 &
3 & 0 &
4 & 1 \\
 & \scriptsize{Taxa (\%)} &
100,0 & 0,0 &
100,0 & 0,0 &
100,0 & 0,0 &
100,0 & 0,0 &
75,0  & 0,0 &
100,0 & 25,0 \\ \hline
\multirow{2}{3cm}{\centering\emph{NI00175}
\footnotesize{(12~eventos)}} & \scriptsize{Ocorr.} &
12 & 0 &
12 & 0 &
12 & 0 &
12 & 0 &
12 & 0 &
12 & 1 \\
 & \scriptsize{Taxa (\%)} &
100,0 & 0,0 &
100,0 & 0,0 &
100,0 & 0,0 &
100,0 & 0,0 &
100,0 & 0,0 &
100,0 & 8,3 \\ \hline
\multirow{2}{3cm}{\centering\emph{NI00177}
\footnotesize{(12~eventos)}} & \scriptsize{Ocorr.} &
12 & 0 &
12 & 0 &
12 & 0 &
12 & 0 &
12 & 0 &
12 & 1 \\
 & \scriptsize{Taxa (\%)} &
100,0 & 0,0 &
100,0 & 0,0 &
100,0 & 0,0 &
100,0 & 0,0 &
100,0 & 0,0 &
100,0 & 8,3 \\ \hline
\multirow{2}{3cm}{\centering\emph{Temp. Gab. 1}
\footnotesize{(74~eventos)}} & \scriptsize{Ocorr.} &
72 & 29 &
73 & 12 &
71 & 144 &
73 & 21 &
71 & 7  &
71 & 62 \\
 & \scriptsize{Taxa (\%)} &
97,3 & 39,0 &
98,6 & 16,2 &
95,9 & 194,6 &
98,6 & 28,4 &
95,9 & 9,5 &
95,9 & 83,8 \\ \hline
\multirow{2}{3cm}{\centering\emph{Temp. Gab. 2}
\footnotesize{(105~eventos)}} & \scriptsize{Ocorr.} &
72  & 29  &
85  & 0   &
102 & 116 &
91  & 3   &
77  & 1   &
99 & 34 \\
 & \scriptsize{Taxa (\%)} &
97,3 & 39,2  &
81,0 & 0,0   &
97,1 & 110,5 &
86,7 & 2,9 &
73,3 & 0,9 &
94,3 & 32,4 \\ \hline
\multirow{2}{3cm}{\centering\emph{Empilhado4}
\footnotesize{(40~eventos)}} & \scriptsize{Ocorr.} &
23 & 1 &
18 & 0 &
28 & 8 &
24 & 0 &
22 & 0 &
31 & 1 \\
 & \scriptsize{Taxa (\%)} &
57,5 & 2,5  &
45,0 & 0,0  &
70,0 & 20,0 &
60,0 & 0,0  &
55,0 & 0,0  &
77,5 & 2,5 \\ \hline
\multirow{2}{3cm}{\centering\emph{Empilhado7}
\footnotesize{(24~eventos)}} & \scriptsize{Ocorr.} &
22 & 0  &
22 & 0  &
21 & 16 &
22 & 0  &
18 & 2  &
23 & 8 \\
 & \scriptsize{Taxa (\%)} &
91,7 & 0,0  &
91,7 & 0,0  &
87,5 & 66,7 &
91,7 & 0,0  &
75,0 & 8,3  &
95,8 & 33,3 \\
\hline \hline
\end{tabular}}
\caption[Taxa de detecção de eventos de transitório e falso
alarme para os três ajustes automáticos, os dois ajustes manuais e o
Detector de Patamar Elaborado.]{Taxa de detecção de eventos
de transitório e falso alarme para os três ajustes automáticos, os
dois ajustes manuais e o Detector de Patamar Elaborado. Apenas os
eventos com acréscimo de consumo foram considerados devido à
limitação do Detector de Patamar Elaborado em detectar apenas esse
tipo de eventos. As abreviaturas DET e FA referem-se a
taxa de detecção e falso alarme, respectivamente.}
\label{tab:det_elab_res}
\end{table}

Observa-se que o Detector de Patamar Elaborado, mesmo não ajustado
para os conjuntos de dados analisados, comportou-se razoavelmente bem
nos mesmos. Essa técnica obteve boa taxa de detecção, inclusive
obtendo a taxa mais alta no conjunto \emph{Empilhado4} e não
detectando apenas 6 eventos no conjunto \emph{Temporizado Gabarito 2}.
A taxa de falso alarme no último conjunto citado foi de 34 eventos,
próxima àquela obtida pelo ES 1, indicando o potencial dessa técnica.
Porém, o conjunto \emph{Empilhado7} contém uma maior taxa de falsos
alarmes para essa técnica, que ocorrem durante a operação da televisão
CRT. Além disso, há a presença de alguns falsos alarmes nos conjuntos
\emph{NI00***} que não ocorrem na método com o filtro de Gaussiana.
Esses falsos alarmes foram levantados e identificados pela operação
das chaves do ventilador e secador de cabelo, que são do tipo
``interrompa e então faça'' (do inglês \emph{break then make}), aonde
há a interrupção do consumo do equipamento quando alterando seu estado
de operação. Um exemplo de distúrbio que causa falso alarme de um
acionamento na operação dessa chave pode ser observado na
Figura~\ref{fig:ni00174_overview}, onde o segundo desacionamento nesse
conjunto causa um evento de acionamento pelo Detector de Patamar
Elaborado. Talvez seja possível trabalhar este método de forma a
eliminar esse problema, que aliado com o ajuste de seus parâmetros,
pode se tornar uma técnica bastante adequada para a detecção de
eventos. O ajuste de parâmetros pode ser feito de maneira automática,
utilizando um método similar àquela empregada neste trabalho.

Finalmente, como foi dito no Capítulo~\ref{cap:nilm}, é necessário
indicar a capacidade de reconstrução de energia do \acs{nilm}, porém, o
mesmo não faria sentido nas simulações em laboratório, que não
apresentam a operação dos aparelhos nos tempos que normalmente ocorrem
nas residencias. Indo além, os dados de residência disponíveis (ver
Figura~\ref{fig:casa_real}) não tinham a informação necessária para a
construção do gabarito. Assim, as taxas aqui disponibilizadas não
revelam a real capacidade de reconstrução de energia nas residências
brasileiras, mas servem como um estudo inicial nesse sentido.
Espera-se nos estudos posteriores empregar a análise em dados
representativos, de forma a trazer as taxas já calculadas em termos de
energia e capacidade de desagregação de consumo do \acs{nilm}.



  \chapter{Conclusão}
\label{chap:conclusao}



%\section{Trabalhos Futuros}
%\label{sec:trabfut}



  \glsaddall{}

  \backmatter{}
  \bibliographystyle{coppe-unsrt}
  \bibliography{thesis}

  \appendix
  %\include{appenA}
\end{document}
