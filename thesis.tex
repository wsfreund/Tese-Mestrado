
\documentclass[msc,numbers]{coppe}
\let\printglossary\relax
\let\theglossary\relax
\let\endtheglossary\relax
\usepackage[utf8]{inputenc}
\usepackage{amsmath,amssymb}
\usepackage[unicode]{hyperref}
\usepackage{indentfirst}
\usepackage{graphicx}
\usepackage{subcaption} % Caption inside subfigures
\usepackage{rotating} % Para rodar as figuras e tabelas
\usepackage{pdflscape} % Para colocar paginas na horizontal no pdf
\usepackage{multirow} % Tabelas com celulas agrupadas
\usepackage{placeins} % mais floats (figuras tabelas etc)!
\usepackage{array}
\usepackage{import} % To access created latex images in different folders
\usepackage{enumerate} % To have access to special enumerations
\usepackage[inline]{enumitem}
\usepackage{steinmetz} % Para ter acesso ao angulo de fase como
% deveria ser.
\usepackage{graphicx}% for \rotatebox
\usepackage{everypage}
\usepackage{environ}

\newcounter{abspage}% \thepage not reliab

\makeatletter
\newcommand{\newSFPage}[1]% #1 = \theabspage
  {\global\expandafter\let\csname SFPage@#1\endcsname\null}

\NewEnviron{SidewaysFigure}{\begin{figure}[p]
\protected@write\@auxout{\let\theabspage=\relax}% delays expansion until shipout
  {\string\newSFPage{\theabspage}}%
\ifdim\textwidth=\textheight
  \rotatebox{90}{\parbox[c][\textwidth][c]{\linewidth}{\BODY}}%
\else
  \rotatebox{90}{\parbox[c][\textwidth][c]{\textheight}{\BODY}}%
\fi
\end{figure}}

\AddEverypageHook{% check if sideways figure on this page
  \ifdim\textwidth=\textheight
    \stepcounter{abspage}% already in landscape
  \else
    \@ifundefined{SFPage@\theabspage}{}{\global\pdfpageattr{/Rotate 0}}%
    \stepcounter{abspage}%
    \@ifundefined{SFPage@\theabspage}{}{\global\pdfpageattr{/Rotate 90}}%
  \fi}
\makeatother


\makeatletter
\newcommand*{\textlabel}[2]{%
\edef\@currentlabel{#1}% Set target label
\phantomsection% Correct hyper reference link
#1\label{#2}% Print and store label
}
\makeatother

\hypersetup{
unicode=true,          % non-Latin characters in bookmarks
pdftitle={},           % title
pdfauthor={Werner Spolidoro Freund},     % author
pdfsubject={Dissertação de Mestrado},   % subject of the document
colorlinks=true,       % false: boxed links; true: colored links
pdfdisplaydoctitle=true,
citecolor=black,%
filecolor=black,%
linkcolor=black,%
urlcolor=black%
}


\usepackage[style=long,nomain,nonumberlist,shortcuts]{glossaries} % Para as listas de simbolos e abreviaturas
\renewcommand*{\glsacrpluralsuffix}{}
\renewcommand*{\glsupacrpluralsuffix}{}
\renewcommand*{\acrpluralsuffix}{}
\newglossary[slg]{Simb}{sym}{sbl}{Lista de S{í}mbolos}
\newglossary[alg]{Abrev}{abr}{abv}{Lista de Abreviaturas}

\usepackage{etoolbox} % Para robustify
\robustify{\gls}
\robustify{\url}

% Mutliplas referências de pé de nota com o mesmo número
\newcommand*{\fnref}[1]{\textsuperscript{\ref{#1}}}

\makeglossaries


\begin{document}

  \newacronym[type=Simb]{kcal}{kcal}{unidade de energia quilocalorias}
\newacronym[type=Simb]{toe}{toe}{unidade de energia toneladas equivalente de petróleo}
\newacronym[type=Simb]{watt}{W}{unidade de potência \emph{watt}}
\newacronym[type=Simb]{a}{A}{unidade de corrente {ampère}}
\newacronym[type=Simb]{v}{V}{unidade de tensão \emph{volt}}
\newacronym[type=Simb]{va}{VA}{unidade de potência \emph{volt-ampère}}
\newacronym[type=Simb]{var}{VAr}{unidade de potência reativa
\emph{volt-ampère} reativo}
\newacronym[type=Simb]{btu}{BTU}{unidade de energia}
\newacronym[type=Simb]{wh}{Wh}{unidade de energia \emph{watt} hora}
\newacronym[type=Simb]{co2}{\protect{$CO_2$}}{gás carbônico}
\newacronym[type=Simb]{theta}{$\theta$}{ângulo de fase}
\newacronym[type=Simb]{i}{I}{corrente elétrica}
\newacronym[type=Simb]{p}{P}{potência ativa}
\newacronym[type=Simb]{q}{Q}{potência reativa}
\newacronym[type=Simb]{d}{D}{potência harmônica}
\newacronym[type=Simb]{s}{S}{potência aparente}
\newacronym[type=Simb]{di}{\protect{$\Delta{I}$}}{variação de
\acl{i}}
\newacronym[type=Simb]{dp}{\protect{$\Delta{P}$}}{variação de
\acl{p}}
\newacronym[type=Simb]{dq}{$\Delta{Q}$}{variação de \acl{q}}
\newacronym[type=Simb]{dd}{$\Delta{D}$}{variação de \acl{d}}
\newacronym[type=Simb]{ds}{$\Delta{S}$}{variação de \acl{s}}
\newacronym[type=Simb]{hz}{Hz}{unidade de frequência \emph{hertz}}
\newacronym[type=Simb]{c1}{C1}{equipamentos de consumo permanente}
\newacronym[type=Simb]{c2}{C2}{\acs{fsm} ou equipamentos de estados múltiplos}
\newacronym[type=Simb]{c2a}{C2a}{\acs{fsm} ou equipamentos de estados
múltiplos com ciclos bem-definidos}
\newacronym[type=Simb]{c2b}{C2b}{\acs{fsm} ou equipamentos de estados
múltiplos com ciclos aleatórios}
\newacronym[type=Simb]{c3}{C3}{\acs{fsm} de dois estados ou equipamentos liga/desliga}
\newacronym[type=Simb]{c4}{C4}{equipamentos com níveis continuos de consumo}
\newacronym[type=Simb]{c5}{C5}{equipamentos com dinâmica de carga}
\newacronym[type=Simb]{c6}{C6}{equipamentos com características similares}
\newacronym[type=Simb]{fp}{FP}{fator de potência}
\newacronym[type=Simb]{det_eff}{$\eta_{det}$}{eficiência de
detecção}
\newacronym[type=Simb]{class_eff}{$\eta_{class}$}{eficiência de
classificação}
\newacronym[type=Simb]{total_eff}{$\eta_{total}$}{eficiência total}
\newacronym[type=Simb]{nid}{$N_{id}$}{número de eventos
corretamente detectados e classificados}
\newacronym[type=Simb]{nreais}{$N_{reais}$}{número de eventos
causados pelos equipamentos na rede}
\newacronym[type=Simb]{nfp}{$N_{fp}$}{número de eventos
devido a falsos positivos}
\newacronym[type=Simb]{nni}{$N_{ni}$}{número de eventos
não identificados}
\newacronym[type=Simb]{nap}{$N_{ap}$}{número de equipamentos}
\newacronym[type=Simb]{nt}{$N_{ap}$}{número de eventos de transitórios}
\newacronym[type=Simb]{p_eff_i}{$\rho_{En}^i$}{taxa de
reconstrução em energia para o i-ésimo equipamento}
\newacronym[type=Simb]{p_eff}{$\rho_{En}$}{taxa de
reconstrução em energia do \acs{nilm}}
\newacronym[type=Simb]{en_res_i}{$\varepsilon^i$}{energia redundante
para o i-ésimo equipamento}
\newacronym[type=Simb]{en_res}{$\varepsilon$}{energia redundante}
\newacronym[type=Simb]{e_id_i}{$E_{id}$}{energia corretamente
identificada para o i-ésimo equipamento}
\newacronym[type=Simb]{en_eff_i}{$\eta_{En}^i$}{eficiência de
reconstrução em energia para o i-ésimo equipamento}
\newacronym[type=Simb]{en_eff}{$\eta_{En}$}{eficiência de
reconstrução em energia do \acs{nilm}}
\newacronym[type=Simb]{red_eff_i}{$\rho_{red}^i$}{taxa de
redundância de energia para o i-ésimo equipamento}
\newacronym[type=Simb]{red_eff}{$\rho_{red}$}{taxa de
redundância de energia do \acs{nilm}}
\newacronym[type=Simb]{en_pres}{$\eta_{En,prec}^i$}{fração de energia
corretamente identifica em relação ao total de energia detectado para
o i-ésimo equipamento}
\newacronym[type=Simb]{medidafenergia}{$F_{en}^i$}{medida-eF para o i-ésimo equipamento}
\newacronym[type=Simb]{medidaf}{$F^i$}{medida-F para o i-ésimo equipamento}
\newacronym[type=Simb]{cr}{$CR$}{razão de complementação}
\newacronym[type=Simb]{ir}{$IR$}{resíduo individual}
\newacronym[type=Simb]{ur}{$UR$}{resíduo unificado}
\newacronym[type=Simb]{jmax}{$J_{max}$}{janela máxima de amostras para
correlacionar a informação de análise com a do gabarito}
\newacronym[type=Simb]{mu}{$\mu$}{população dos pais}
\newacronym[type=Simb]{lambda}{$\lambda$}{população da prole}
\newacronym[type=Simb]{te}{$T_e$}{erro de quantização}
\newacronym[type=Simb]{qe}{$Q_e$}{erro topográfico}
\newacronym[type=Simb]{bmu}{$BMU$}{\emph{Best Machine Unit}}

  \newacronym[type=Abrev]{ocde}{OCDE}{Organização para a Cooperação e
Desenvolvimento Econômico}
\newacronym[type=Abrev,\glslongpluralkey={Países em
Desenvolvimento}]{ped}{PED}{País em Desenvolvimento}
\newacronym[type=Abrev]{onu}{ONU}{Organização das Nações Unidas}
\newacronym[type=Abrev]{pme}{PME}{Programa de Mobilização Energética}
\newacronym[type=Abrev]{inmetro}{INMETRO}{Instituto Nacional de
Metrologia, Normalização e Qualidade Industrial}
\newacronym[type=Abrev]{proesco}{PROESCO}{Programa de apoio a Projetos
de Eficiência Energética}
\newacronym[type=Abrev]{pbe}{PBE}{Programa Brasil de Etiquetagem}
\newacronym[type=Abrev]{eletrobras}{Eletrobras}{Centrais Elétricas
Brasileiras S.A.}
\newacronym[type=Abrev]{procel}{PROCEL/\acrshort{eletrobras}}{Programa
Nacional de Conservação de Energia Elétrica}
\newacronym[type=Abrev]{bndes}{BNDES}{Banco Nacional de Desenvolvimento
Econômico e Social}
\newacronym[type=Abrev]{conpet}{CONPET}{Programa Nacional de Racionalização do
Uso dos Derivados do Petróleo e do Gás Natural}
\newacronym[type=Abrev]{petrobras}{Petrobrás}{Petróleo Brasileiro S.A.}
\newacronym[type=Abrev]{mdic}{MDIC}{Ministério do Desenvolvimento, da Indústria
e do Comércio Exterior}
\newacronym[type=Abrev]{mme}{MME}{Ministério de Minas e Energia}
\newacronym[type=Abrev]{proalcool}{Proálcool}{Programa Nacional do
Álcool}
\newacronym[type=Abrev]{pnef}{PNEf}{Programa Nacional de Eficiência
Energética}
\newacronym[type=Abrev]{epe}{EPE}{Empresa de Pesquisa Energética}
\newacronym[type=Abrev]{pee}{PEE}{Programa de Eficiência Energética}
\newacronym[type=Abrev]{ee}{EE}{Eficiência Energética}
\newacronym[type=Abrev]{pne2030}{PNE2030}{Plano Nacional de Energia
2030}
\newacronym[type=Abrev,\glslongpluralkey={Planos Decenais de
Energia}]{pde}{PDE}{Plano Decenal de Energia}
\newacronym[type=Abrev]{aneel}{ANEEL}{Agência Nacional de Energia
Elétrica}
\newacronym[type=Abrev,\glslongpluralkey={Pesquisas de Posse e Hábito
de Eletrodomésticos}]{pph}{PPH}{Pesquisa de Posse e Hábito de
Eletrodomésticos}
\newacronym[type=Abrev]{beu}{BEU}{Balanço de Energia {\'U}til}
\newacronym[type=Abrev]{ti}{TI}{Tecnologia da Informação}
\newacronym[type=Abrev]{glp}{GLP}{Gás Liquefeito do Petróleo}
\newacronym[type=Abrev]{ibge}{IBGE}{Instituto Brasileiro de Geografia
e Estatistica}
\newacronym[type=Abrev]{cepel}{CEPEL/\acrshort{eletrobras}}{Centro de
Pesquisas de Energia Elétrica}
\newacronym[type=Abrev]{nilm}{NILM}{Monitoramento Não-Invasivo de Cargas
Elétricas}
\newacronym[type=Abrev]{dnilm}{dNILM}{\acrshort{nilm} de Arquitetura
Distribuida}
\newacronym[type=Abrev,\glslongpluralkey={Tecnologias de Comunição e
Informação}] {ict}{ICT}{Tecnologia de Comunicação e Informação}
\newacronym[type=Abrev]{eua}{EUA}{Estados Unidos da América}
\newacronym[type=Abrev]{pca}{PCA}{Análise de Componentes Principais}
\newacronym[type=Abrev]{pcd}{PCD}{Análise de Componentes Discriminantes}
\newacronym[type=Abrev]{fsm}{FSM}{Máquina de Estados Finitos}
\newacronym[type=Abrev]{fex}{FEX}{Extração de Características}
\newacronym[type=Abrev]{ted}{TED}{\emph{The Energy Detective}}
\newacronym[type=Abrev,\glslongpluralkey={Redes Neurais Artificiais}]{rna}
{RNA}{Rede Neural Artificial}
\newacronym[type=Abrev]{cdm}{CDM}{Mecânismo de Decisão por Comissão}
\newacronym[type=Abrev]{mco}{MCO}{Ocorrência Mais Comum}
\newacronym[type=Abrev]{lur}{LUR}{Menor Resíduo Unificado}
\newacronym[type=Abrev]{mle}{MLE}{Estimativa de Máxima-Verossimilhança}
\newacronym[type=Abrev]{roc}{ROC}{\emph{Receiver Operating
Chracteristic}}
\newacronym[type=Abrev]{avac}{AVAC}{Aquecimento, Ventilação e Ar
Condicionado}
\newacronym[type=Abrev]{som}{SOM}{Mapas Auto-Organizáveis}
\newacronym[type=Abrev]{es}{ES}{Estratégia Evolutiva}


  \title{Monitoração Não-Invasiva de Cargas Elétricas Residenciais}
  \foreigntitle{Non-Intrusive Residential Load Monitoring}
  \author{Werner Spolidoro}{Freund}
  \advisor{Prof.}{José Manoel de}{Seixas}{D.Sc.}

  \examiner{Prof.}{José Manoel de Seixas}{D.Sc.}
  \examiner{Prof.}{Luiz Pereira Calôba}{Dr.Ing.}
  \examiner{Prof.}{Carlos Augusto Duque}{D.Sc.}
  \examiner{Dr.}{Charles Bezerra do Prado}{D.Sc.}

  \department{PEE}
  \date{12}{2013}

  \keyword{Monitoração Não-Invasiva de Cargas Elétricas}
  \keyword{NILM}
  \keyword{Algoritmo Genético}

  \maketitle

  % TODO Colocar páginas das figuras e tabelas extraídas
  \frontmatter
  \dedication{Yeahhh.}


  \chapter*{Agradecimentos}

Agradeço aos meus pais por tornar isso tudo possível. Vocês que
tiveram o maior peso para que isso se transformasse em realidade.
Agradeço de coração por todo esforço e trabalho que tiveram com o
intuíto de me verem chegar aqui. Aos meus avós por todo o carinho e
apoio que sempre me deram, sempre me incentivando para conseguir
atingir meus sonhos. Ao meu irmão mais novo que sacríficou seu tempo
livre fazendo algumas de minhas tarefas enquanto eu estava
trabalhando. Ao meu outro irmão pelas risadas.

Ao meu orientador, Seixas, que me deu suporte e me guiou neste
trabalho, apesar de todas condições atípicas. Seus conselhos foram
muito valiosos, sua ajuda foi muito mais do que essêncial. Aos outros
orientadores que tive, Torres e Damazio, suas orientações ajudaram a
tornar-me o que sou hoje.

Aos engenheiros do CATE no \acl{cepel}, João, Aroldo, Guilherme e
Victor por toda a ajuda e compreensão durante o tempo de trabalho. A
convivência com vocês foi muito agradável! Também à Beth por sempre
ter sido prestativa. Ao Alvaro pela informação necessária
para a construção dos gabaritos dos conjuntos de dados
\emph{Empilhado4} e \emph{Empilhado7}.

Ao pessoal do \acs{lps}: Balabram, Junior, Moura, Grael, Rodrigo, João
Victor, Diego e Hellen. Vocês fazem o laboratório ser o que ele é, sem
vocês ele não tem graça alguma. À Ana e Talia, vocês fazem muita falta
lá. Agradeço em especial ao João Victor pelo suporte providencial no
momento em que precisei, espero ter retornado da maneira possível com
o máximo de conhecimento em troca. Desculpas ao Grael por todo
incomodo causado, muito obrigado por fazer o melhor que podia. Ao
Pedro que ajudou com parte do material utilizado para o
levantamento bibliográfico das técnicas empregadas no \acs{nilm}.
Aos meus orientados, Diego e Hellen, me desculpem se exagerei na
cobrança: corram atrás de seus futuros que o \emph{Walhalla} espera
por vocês!

À toda ajuda, paciência e instruções oferecidas pela Dani do
\acs{peecoppe}. Aos professores da \acs{coppe} por todo seu
empenho, ética e dedicação.

Finalmente, agradeço a todos meus amigos que escutaram diversas vezes:
``não posso, tenho que terminar o mestrado'', mas nunca deixaram de
compreender o momento em que estava passando. Hoje chegou o dia!


  \begin{abstract}

Este trabalho foi realizado em colaboração com o \acs{cepel} no
desenvolvimento da tecnologia conhecida como \acf{nilm}. 
A monitoração não-invasiva pode ser aplicada para garantir a qualidade
de energia, o diagnóstico de carga, identificação de aparelhos
defeituosos ou com consumo excessivo de energia e eficiência
energética. Este trabalho abordará a etapa de detecção de eventos de
transitórios causados por aparelhos elétricos, o qual é necessária
para a obtenção dos traços de informação deixados por equipamentos
durante a alteração de seu estado de operação.
A proposta do \acs{cepel} é a utilização de um filtro de
derivada de Gaussiana que gera candidatos a eventos de transitórios,
os mesmos ainda analisados por outros testes para eliminação de falsos
eventos em seguida. 

A contribuição do trabalho está na sistematização da determinação dos
parâmetros necessários para o método proposto pelo \acs{cepel}, que
permite o ajuste automático dos mesmos para diferentes cenários. Para
isso, foi implementada uma adaptação de um algoritmo genético, que
permite dinâmica para alocação de maior esforço computacional para
configurações que melhor se adequam ao problema. Além disso, o
otimizador faz parte de um extenso ambiente de análise que trouxe
vantagens operacionais em termo de infraestrutura para o projeto.

Ao aplicar o algoritmo na base de dados, o mesmo mostrou capacidade de
generalização para as condições avaliadas. Os resultados foram
comparados com os trabalhos recentes desenvolvidos neste contexto.
Obteve-se taxa de detecção superior a 80\% para taxa de falso alarme
inferior a 1 \% em cenários com operação simultânea de diversos
aparelhos e dinâmica de carga. Já para os conjuntos com operações
simples de equipamentos as taxas são de 98,1~\% e 0,6~\%,
respectivamente.

\end{abstract}


  \begin{foreignabstract}

Non-intrusive monitoring may be applied to power quality, load
diagnosis, faulty appliance detection or with excessive energy
consumption and energy efficiency. This work will discuss the
technology developed when addressing the issue of energy efficiency
for the residential sector. The transient event detection is its
focus, where it was used a Gaussian core filter to obtain the
responsive regions which indicates an appliance state change. The
approach parameters were optimize via genetic algorithm. %In order to
%improve the problem comprehension, it was used Self-Organizing Maps to
%explore the disturbance chracteristics due to an appliance state
%change. They were also evaluated as an method to aid increasing
%previous methodology efficiency. 
Detection rates higher than 83\% were achieved for
false alarm rates lower than 17\% while in a pratical
scenario application. False alarm rates may be lower depending on 
the problem representation.

\end{foreignabstract}


  \addtocounter{page}{1}
  \vfill
  \cleardoublepage{} \phantomsection{} \addcontentsline{toc}{chapter}{Sumário}
  \tableofcontents \vfill
  \cleardoublepage{} \phantomsection{}
  \listoffigures \vfill
  \cleardoublepage{} \phantomsection{}
  \listoftables \vfill
  \cleardoublepage{} \phantomsection{} \addcontentsline{toc}{chapter}{Lista de Símbolos}
  %\renewcommand{\glossarypreamble}{Texto Simbulos}
  \printglossary[type=Simb] \vfill
  \cleardoublepage{} \phantomsection{} \addcontentsline{toc}{chapter}{Lista de Abreviaturas}
  \renewcommand{\glossarypreamble}{No caso de algumas abreviaturas internacionalmente conhecidas, optou-se por mantê-las em sua lingua original.}
  \printglossary[type=Abrev] \vfill

  \mainmatter{}
  \chapter{Introdução}

A proposta de sustentabilidade, surgida como resposta à degradação do
meio ambiente devido ao processo de industrialização, leva, em seu
discurso, a questão das limitações tanto em termos de recursos
presentes na natureza, ou quanto da sua capacidade de absorver resíduos
de processos em ordens cada vez mais intensas. Sua vertente
predominante traz a ideia do desenvolvimento sustentável, que busca
regular a demanda de recursos do meio ambiente, de modo que seja
possível o desenvolvimento sem que haja sua deterioração para as
gerações futuras.

Adicionalmente, há uma correlação entre os impactos ambientais e a
necessidade energética, onde os motivos prevalecentes para suas
causas têm como fator de destaque a sua vinculação à cadeia da
energia, desde a produção ao uso final. Entretanto, essa dependência
não é decorrente apenas da intensidade, mas da eficiência energética
relacionada com o seu consumo. Entende-se como eficiência energética a
capacidade de produzir os mesmos resultados finais realizando o menor
consumo possível de energia.

A eletricidade tem uma participação cada vez maior na matriz
energética mundial devido à sua maior versatilidade, eficiência,
limpeza, segurança e conveniência, quando comparada com as outras
fontes energéticas. A eficiência energética do ponto de vista da
eletricidade traz uma série de vantagens, além de benefícios
ambientais:

\begin{itemize}
\item reduz ou posterga as necessidades de investimentos em geração, transmissão
e distribuição de energia elétrica;
\item reduz o custo de energia para o consumidor final;
\item contribui para a confiabilidade do sistema elétrico;
\item traz o aumento da atividade econômica com a redução da
intensidade energética.
\end{itemize}

Além disso, o crescimento do consumo nos grandes centros urbanos tem
levado ao surgimento de uma nova rede elétrica --- as redes elétricas
inteligentes ---, que procura descentralizar a geração devido à
exigência excessiva da capacidade de transmissão e distribuição da
versão usual centralizada. Dependendo da regulamentação dos medidores
inteligentes --- os medidores das novas redes elétricas ---, uma série
de novas possibilidades podem ser exploradas, como:

\begin{itemize}
\item maior versatilidade de operação e planejamento das redes
elétricas devido à maior informação presente; 
\item tarifação variável conforme a demanda na rede, incentivando os
consumidores a deslocarem cargas não-essenciais para operarem em
horários fora de ponta de forma a reduzir ou postergar a necessidade
de investimentos na expansão da rede --- deslocamento de carga;
\item o retorno da informação do consumo em tempo real nas residências pode ser
explorado para obter eficiência energética, assunto melhor
debatido a seguir.
\end{itemize}

\section{Motivação}

Uma tecnologia que tem adquirido maior interesse como forma de aliviar
a pressão de consumo nos grandes centros urbanos e de atingir maior
eficiência energética é a \gls{nilm}, seja no mundo corporativo --- atraindo
empresas como \emph{Intel} e \emph{Belkin} --- como no meio acadêmico,
em especial nos países desenvolvidos.

O monitoramento não-invasivo utiliza-se de um único medidor central no
fornecimento de energia da residência para identificar o consumo dos
equipamentos através dos distúrbios causados na rede elétrica pelos
mesmos. Para isso, empregam-se técnicas de processamento
de sinais, inteligência computacional e estatística para identificar
os padrões dos distúrbios e correlacioná-los com o equipamento de origem.
Dependendo da metodologia aplicada no \acs{nilm}, isso pode ser
realizado de maneira cega, ou seja, encontrando padrões recorrentes
na rede elétrica e os identificando quando eles se repetem.
A topologia mais comum pode ter sua operação resumida através destes
passos:

\begin{itemize}
\item Aquisição de dados: a eficácia do \acs{nilm} depende diretamente
da capacidade do medidor de extrair informação da rede elétrica, sendo
desejável amostragens com frequências elevadas ou com uma maior
quantidade de representações, independentes entre si;
\item Extração de características: com base na informação obtida pelo
sistema de aquisição de dados, é possível transformá-la em
características que serão utilizadas pelas etapas seguintes para obter
o consumo desagregado por equipamento;
\item Detecção de eventos de transitório na rede devido à mudança de
operação de um equipamento: quando o \acs{nilm} utiliza a informação no
transitório na operação dos equipamentos para identificar a operação do
equipamento e estimar o seu consumo --- configuração mais comum ---, é
necessário detectar esses momentos e diferenciá-los de
ruídos causados por demais oscilações no consumo da rede;
\item Identificação da operação do equipamento e seu consumo:
nessa etapa, processam-se as características do consumo de forma a
reconhecer os padrões para estimar o consumo dos
equipamentos. Geralmente essa tarefa ocorre somente quando é
identificado um distúrbio na rede, reduzindo o processamento dos
dados.
\end{itemize}

As aplicações do \acs{nilm} são diversas, como monitoramento da
qualidade de energia, diagnóstico de carga, identificação de equipamentos
defeituosos ou com consumo excessivo de energia. No que se refere à
eficiência energética, o mesmo pode auxiliar destas maneiras:

\begin{itemize}
\item Auxiliar na obtenção de dados para estudos de eficiência
energética de eletricidade: no caso do setor residencial, os estudos
de eficiência energética precisam de dados com informação desagregada
por equipamento para obter melhor precisão e direcionar os esforços para
sua obtenção. Atualmente, o levantamento para o setor residencial é
realizado através de pesquisas de posse e hábito de consumo, havendo
tanto uma demora para a sua obtenção, quanto sofre de certa degeneração
devido à sua imprecisão. Essa tecnologia pode auxiliar no processo de
obtenção dessa informação, melhorando a precisão e agilizando a
obtenção de dados recentes;
\item Fornecer o retorno de informação de consumo desagregado por
equipamento para o consumidor: diversas pesquisas nos países
desenvolvidos mostram que retornar uma informação mais detalhada para
o consumidor --- além daquela contida na conta de energia --- é favorável
no sentido de incentivá-lo a tomar ações para redução do consumo e,
consequentemente, melhor eficiência energética. Esses
estudos indicam que quanto maior for a quantidade de informação
disponível melhor será essa capacidade, sendo o melhor caso a
informação de consumo desagregado por equipamento em tempo-real,
justamente a capacidade dessa tecnologia. Porém, nesse caso, não apenas se
faz somente necessário o retorno da informação de consumo da mesma,
mas também o envolvimento de outras áreas do conhecimento como
psicologia e \emph{design} da informação para que seu emprego seja
eficiente e motive o consumidor para tomar ações sustentáveis no uso
de energia.
\end{itemize}

O \acs{nilm} pode aproveitar-se das redes elétricas inteligentes para
obter a informação desagregada através da utilização dos medidores
inteligentes --- subordinado à capacidade de fornecer informação dos
mesmos, onde uma taxa mínima de 1~\acs{hz} é indicada nesse sentido.
Isso facilitaria a impregnação do método e, com isso, maximizaria seu
potencial de eficiência energética ao atender ambos itens anteriores
em maior escala. Porém, isso está sujeito a limitação
da escolha da configuração dos medidores inteligentes no Brasil, de
forma que o projeto do \acs{nilm}, apesar de ser desejável sua
impregnação, deve ser realizado sem a certeza de poder utilizar a
futura estrutura provida pelas redes elétricas inteligentes no país.

A pluralidade de técnicas aplicadas na tecnologia envolvendo o tema
revelam que o seu projeto, apesar de parecer simples, na verdade
engloba diversos desafios. O maior desafio geralmente encontrado pelos
autores é expandir a técnica para a aplicação em condições reais de
operação das redes elétricas residenciais, aonde estarão presentes
diversos equipamentos operando simultaneamente, alguns deles com dinâmica
no seu consumo, o que torna complexa a identificação dos padrões
deixados na rede por demais equipamentos.

\section{Objetivo}

O \acs{cepel} vem atuando no desenvolvimento de um \acs{nilm}. Este
trabalho tem como objetivo auxiliar no progresso desse dispositivo,
sendo desenvolvido em parceria com o \acs{cepel}.

A principal intenção de aplicação da tecnologia pelo \acs{cepel} é
para o auxilio nas \glspl{pph} no setor residencial. Outro motivo para
focar apenas nesse setor é a maior dificuldade de aplicação no setor
comercial e industrial, que apresentam redes elétricas com um nível de
desafio mais elevado quando comparado ao setor residencial. Não
obstante, a rede residencial já apresenta obstáculos suficientes a
serem superados, em especial nas condições de operação mais ativas da
mesma. Nesses momentos, há uma maior quantidade de equipamentos
operando, o que adiciona, potencialmente, dinâmica no consumo da rede
e, com isso, torna a tarefa a detecção dos eventos de transitório
dos estados de operação dos equipamentos não trivial pois há uma menor
relação sinal-ruído. Essa dinâmica também tornará a identificação de
padrões dos eletrodomésticos mais complexa, aonde o discriminador terá
de lidar com distorções em seus padrões. 

O trabalho atual irá expandir a metodologia proposta pelo \acs{cepel}
para operação nessas condições, propondo uma abordagem sistemática que
permita o ajuste da técnica aplicada de acordo com as condições
presentes nas redes elétricas residenciais, limitando-se ao estudo da
capacidade da metodologia proposta em detecção de eventos de
transitório. Porém, como será visto a seguir, o trabalho trouxe
diversas outras contribuições.

\section{Contribuições do Trabalho}

Durante o levantamento bibliográfico, percebeu-se um forte apelo no
exterior às questões de eficiência energética que vão além do intuito
de aplicação do \acs{nilm} nas \glspl{pph}. Diversos estudos nos
Estados Unidos e Europa Ocidental citam a capacidade do consumidor de
economizar energia ao retornar sua informação de consumo de energia
elétrica. Com o objetivo de compreender melhor como isso pode ser
realizado, este trabalho procurou explorar em maiores detalhes essas
questões, trazendo uma compilação dos estudos que parecem ser de maior
relevância envolvendo esse tema.

Como a própria questão das \glspl{pph} estão
relacionadas com a eficiência energética, ficou evidente a necessidade
de trazer no corpo do trabalho um levantamento da origem do tema ---
discurso ambiental e ecologia --- e o que tem sido feito no mundo e no
Brasil nesse sentido para tornar o assunto de mais fácil acesso para
os leitores, que podem não estar familiarizados com o tema e
importância das questões ambientais envolvidas.

Além disso, o \acs{nilm} vem sido desenvolvido desde 1992, sendo
possível encontrar uma vasta quantidade de metodologias propostas por
diversos autores. O trabalho centralizou e uniformizou, na medida do
possível, a informação relevante para o projeto dessa tecnologia para
facilitar seu desenvolvimento pelo \acs{cepel} e por demais autores
quando levando em conta o tema de eficiência energética.

Já quanto à metodologia aplicada, como foi dito, o \acs{nilm}
constitui-se de diversas etapas, sendo necessário tratar de todas elas
para que o projeto seja aplicável. Cada uma delas precisa ser estudada
e compreendida, de tal modo que nem sempre é possível tratar de
todos os pontos em um único trabalho. O \acs{cepel} propôs uma nova
abordagem para a detecção de eventos de transitório na operação dos
equipamentos, que utiliza como núcleo um filtro de derivada de Gaussiana.
Essa abordagem será explorada e estendida pelo trabalho, porém,
limitar-se-á à questão de detecção de eventos de transitório --- apenas
uma das etapas necessárias para a operação do \acs{nilm}. Indo além, é
importante notar que o objetivo do trabalho é estudar o comportamento
da metodologia proposta pelo \acs{cepel} e sua extensão realizada no
trabalho em cenários de aplicação prática. Os conjuntos de dados com
essas condições foram fornecidos pelo próprio \acs{cepel}.

A evolução da metodologia proposta pelo \acs{cepel} foi realizada em
termos de estruturação e sistematização. Também no levantamento
bibliográfico se percebeu a capacidade de complementação das técnicas
aplicadas, de forma que um ambiente único de análise não apenas
permite uma melhor compreensão das técnicas e rapidez no
desenvolvimento do projeto, mas também de explorar a capacidade de
suplementar outras técnicas, o que permite ao \gls{nilm} explorar uma
maior quantidade de equipamentos e/ou eficiência de desagregação do
consumo por equipamento.

O ajuste dos parâmetros realizados pela metodologia do \acs{cepel} era
feito empiricamente, aonde se viu a necessidade de sistematizar o
processo. Foi implementado um sistema de otimização através de um
algoritmo genético de estratégia evolutiva para auxiliar no ajuste dos
parâmetros. 

\section{Estrutura Capitular}

Os Capítulos \ref{chap:ee} e \ref{chap:ee_retorno} compilam a
informação sobre as aplicações do \acs{nilm} para eficiência
energética. O Capítulo~\ref{chap:ee} tratará a origem do tema de
eficiência energética (Seção~\ref{sec:ee_origem}) e prosseguir até as
suas necessidades para sua melhor obtenção no Brasil
(Seção~\ref{sec:ee_dificuldades}), em especial ao que concerne
energia elétrica para o setor residencial.

Por outro lado, o Capítulo~\ref{chap:ee_retorno} trata da expansão do
potencial de eficiência energética através de um novo programa
abrangendo o tema. Antes de tratá-lo, é realizado considerações de
como são obtidos os potenciais de eficiência energética para o setor
residencial na Seção~\ref{sec:ee_setor_residencial}. Somente em
seguida, na Seção~\ref{sec:ee_res_exp}, é tratado como expandir esse
potencial através do retorno de informação para o consumidor. É
feito uma compilação das informações relevantes obtidas ao
observar estudos no exterior envolvendo o tema.

O Capítulo~\ref{cap:nilm} realizará uma introdução sobre a tecnologia
na Seção~\ref{sec:nilm_aspec_gerais}, para então trazer um extenso
levantamento da informação envolvida do ponto de vista de
desenvolvimento, identificando os aspectos já resolvidos, as
dificuldades e as tendências para o seu futuro na
Seção~\ref{sec:nilm_mundo}. Em sequência, as técnicas aplicadas pelo
\acs{cepel} e os trabalhos anteriores em colaboração com a \acs{coppe}
serão tratados na Seção~\ref{sec:nilm_cepel}, aonde será realizado o
levante das dificuldades e necessidades do projeto. Nessa seção,
também está disponível a metodologia original do \acs{cepel} na
Subseção~\ref{ssec:met_cepel}.

As informações levantadas na Seção~\ref{sec:nilm_cepel} serão
resumidas e discutidas na Seção~\ref{sec:motivacao_framework} do
Capítulo~\ref{chap:framework}. Esse capítulo se dedica à descrição do
ambiente desenvolvido com o objetivo de melhorar a capacidade e
sistematizar a análise, bem como unificar o projeto para facilitar sua
continuidade. As alterações da metodologia original podem ser
encontradas ao longo desse capítulo, no entanto, na
Seção~\ref{sec:otimizacao} é descrita a principal ampliação realizada
na metodologia original, o Módulo de Otimização dos Parâmetros. 

O Capítulo~\ref{chap:metodologia} irá descrever as condições simuladas
na base de dados fornecida pelo \acs{cepel}
(Seção~\ref{sec:base_de_dados}), contendo tanto condições mais simples
com apenas acionamentos e desacionamentos, quanto condições mais
próximas às reais nas redes elétricas residenciais, com operação
simultânea de diversos equipamentos e dinâmica no consumo de alguns
equipamentos. Também é apresentado a metodologia para a aplicação do
algoritmo genético para ajuste dos parâmetros
(Seção~\ref{sec:aplic_es}). Os resultados para a metodologia empregada
por este trabalho estão no Capítulo~\ref{chap:resultados}. A conclusão
do trabalho encontra-se no Capítulo~\ref{chap:conclusao}.

\glsunsetall

  \chapter{A questão energética-ambiental}

% O ser humano e sua capacidade de ir além da energia endossomática
Estima-se que o ser humano consome entre 2500 e 3000 quilocalorias/dia sob a 
forma de alimentos \cite{hemery}. Apenas cerca de 20\% dessa energia poderá ser 
reinvestida em atividades, de forma que a capacidade humana de produção através
de energia endossomática - provida por seu metabolismo - é de apenas 500 a 600 
quilocalorias/dia. No entanto, é a especificidade do ser humano de utilizar 
fluxos de energia exossomáticos - fontes não provenientes de seu metabolismo 
- que o permite transformar o meio facilitando e melhorando sua qualidade de
vida \cite{rippel}.

% A utilização de energia exossomática e os danos ao planeta
Os fluxos exossomáticos estão presentes em todas sociedades e são dependentes do
nível de desenvolvimento das mesmas. Em sociedades primitivas, os resultados dos
atos da humanidade exerciam pouca influência no meio ambiente, pois estes eram
absorvidos por aquele. Entranto, com o surgimento de sociedades industriais,
ocorreu uma mudança radical no uso dos recursos naturais e nos seus efeitos
ambientais. Com a produção e o consumo de massa, baseados no uso intensivo do
petróleo e da eletricidade como fontes energéticas, a agressão humana ao meio 
ambiente tornou-se maior de maneira que o mesmo, em muitas situações, não 
consegue mais reequilibrar-se. Os principais problemas ambientais que surgiram 
no cenario mundial tem como fator de destaque vinculados a cadeia da energia, 
da produção ao uso final \cite{rippel,pen15_eff_energ,jatoba}. 

% A questão de sustentabilidade
% Ecologia radical
Em resposta à degradação do meio ambiente se dá a questão de
sustentabilidade. As primeiras objeções à poluição ambiental devido ao processo 
de industrialização, em especial sua aceleração no início do século XX 
foi da ecologia radical, dando atenção as questões de proteção e conservação da natureza.
Há uma separação de territórios especiais para uma proteção integral e, numa
visão em larga medida romantizada da natureza, muitas vezes sem permissão 
de nenhum uso antrópico \cite{jatoba}.

% Ambientalismo moderado
A política ambiental preponderante atualmente é o discurso do ambientalismo moderado, 
surgido em meio a Crise do Petróleo na década 1970, onde, percebendo-se a
inviabilidade de sustentação do modelo econômico levando em conta o esgotamento
progressivo dos recursos naturais do planeta, se fez necessário a discussão de como 
colocar em prática as propostas da ecologia radical, mas tendo cautela de não 
necessariamente frear o crescimento econômico ou alterar substancialmente o 
modelo de desenvolvimento vigente - trazendo o conceito de desenvolvimento
sustentável. Devido à Crise, houve uma busca mundial pela redução da 
dependência no petróleo e outras fontes fósseis em sua matriz energética, 
procurando fontes renováveis e menos poluentes que as fontes fósseis 
\cite{jatoba,eff_dec_energ_2012,rippel}. 

% Ressaltar aqui a componente social do ambientalismo 

% Eficiencia energetica
A maneira mais efetiva, eleita pela \gls{onu} \cite{onu}, de reduzir os impactos ambientais locais e globais sem
prejuizos ao desenvolvimento, refletindo a imagem de desenvolvimento
sustentável, é através da eficiência energética \cite{rippel,dissert_artur_cursino}. O uso eficiente de 
energia deve ser entendido como o menor consumo possível de energia para obter uma mesma 
quantidade de produto ou serviço \cite{pen15_eff_energ}. A melhoria pode ser obtida nos diversos
fluxos energéticos da sociedade, através da melhoria da geração de uma fonte 
energética, substituição por outra fonte mais eficiente, ou melhoria da
eficiência nos consumidores. Nas sociedades modernas a eletricidade 
como recurso energético vem adquirindo importância cada vez mais vital devido a
sua versabilidade e eficiência \cite{pen15_eff_energ,dissert_caires}, 
obtendo um lugar de destaque quando em debate a eficiência energética. 

Além dos beneficios ambientais, a eficiência energética \cite{jannuzzi,slides_eff_energetica}: 

\begin{itemize}
\item reduz ou posterga as necessidades de investimentos em geração, transmissão 
e distribuição de energia elétrica; 
\item reduz o custo de energia para o consumidor final; 
\item contribui para a confiabilidade do sistema elétrico; 
\item traz o aumento da intensividade econômica com a redução da intensividade
energética. 
\end{itemize}

% Eficiencia energetica no mundo desenvolvido
A tendência nos países desenvolvidos é a realização de esforços 
cada vez maiores no sentindo de aumentar a eficiência energética, 
estimando-se, por exemplo, um consumo 49\% maior nos paises da \gls{ocde} 
em 1996 caso não houvessem sido adotadas medidas de racionalização 
e eficiência energética após a Crise do Petróleo
\cite{goldemberg,slides_eff_energetica}.

% Eficiencia energética nos países em desenvolvimento 
Os \glspl{ped} não possuem uma capacidade de redução energética tão grande
quanto os países da \gls{ocde}, uma vez que seu consumo energético per capita é
reduzido, justamente por ser necessário o desenvolvimento para aumentar o
consumo. Por outro lado, é possível aliviar a pressão sobre a
oferta energética através da melhoria dos níveis de eficiência energética nos
diversos setores da sociedade, contrapondo ao pensamento de que para que
haja desenvolvimento é preciso que ocorram impactos ambientais e crescimento no
consumo total de energia - chamado de efeito \emph{leapfrogging}
\cite{goldemberg,dissert_maria_ines_matos}. 

% Eficiencia energética no Brasil
Assim, a preocupação com eficiência energética se justifica mesmo no Brasil, 
que apresenta quase metade de sua matriz energética proveniente de fontes 
renováveis e preços de produção de energia economicamentes competitivos.
Diversas iniciativas vêm sendo empreendidas há mais de 20 anos, dentre elas se
destacam \cite{projecao_demanda_2012,slides_eff_energetica}:

\begin{enumerate}
\item 
\end{enumerate}

% Consumo residencial bastante elevado no brasil

% Necessidade de obter o perfil de consumo para escolher equipamentos a serem
% melhorados

% Ainda, estudos no exterior mostram que o feedback de consumo poder mudar o
% habito de consumo dos consumidores. Problema, Brasil energia barata e
% dificilmente consumidores mudam habito por isso.

%
%
%
%
%
%\cite{techreport-energetico}


  \chapter{A Eficiência Energética no Setor Residencial}
\label{chap:ee_retorno}

O setor residencial representa um grande potencial para o alcance de uma melhor
\gls{ee}, em especial quando considerando apenas o uso da eletricidade. Isso se
tornará ainda mais relevante com o desenvolvimento do país. Este capítulo detalha
o seu perfil de consumo e potencial para explorar a \gls{ee}, dando 
especial enfase a eletricidade (Sessão~\ref{sec:ee_setor_residencial}). Faz-se
referência as novas tendências para explorar esse potencial no
mundo (Sessão~\ref{sec:ee_res_exp}), que podem se utilizar das possibilidades 
fornecidas pelas redes inteligentes (\emph{smart grid}) e seus aparelhos de 
medição inteligentes (\emph{advanced/smart meters}).

\section{Eletricidade e Potencial de Eficiência Energética 
no Setor Residêncial}
\label{sec:ee_setor_residencial}

A eletricidade é o segundo meio energético de maior participação na matriz 
energética brasileira, representando 16,7\% da demanda, 
Figura~\ref{fig:matriz_bra_2011}. 
Há uma tendência de crescimento dessa parcela na matriz indicado na 
Figura~\ref{fig:matriz_bra_evo}, que deverá progredir devido à fatores 
como \cite{iea_weo2010}:

\begin{figure}[h!t]
    \label{fig:eletricidade_brasil}
    \begin{center}
%
        \subfigure[]{%
            \label{fig:matriz_bra_2011}
            \includegraphics[width=0.8\textwidth]{imagens/matriz_energetica_brasileira_2011.pdf}
        } \\ %\hspace{0.05\textwidth}
        \subfigure[]{%
            \label{fig:matriz_bra_evo}
            \includegraphics[width=0.8\textwidth]{imagens/evolucao_matriz_energetica_brasil.pdf}
        }
%
    \end{center}
    \caption[Matriz energética brasileira.]{A matriz elétrica brasileira
em 2011 (a) e seu histório (b). Baseado (a) e extraído (b) de \cite{ben2012}.}% 
\end{figure}

\begin{itemize}
\item substituição da biomassa para eletricidade como meio de iluminação 
e aquecimento
no setor residêncial;
\item nesse setor também há um aumento do número de eletrodomésticos como
reflexo do desenvolvimento e melhor distribuição da riqueza;
\item a expansão do setor comercial e de serviços utilizando uma quantidade
maiores de aparelhos elétricos (como ar condicionado, iluminação e equipamentos
de \acrshort{ti}); 
\item transformação do setor industrial, gradualmente substituindo o carvão e 
aumentando a presença de dispositivos elétricos.
\end{itemize}

No que diz respeito ao consumo de eletricidade, 
Figura~\ref{fig:eletricidade_por_setor}, 
o setor residencial possui uma posição de destaque no consumo, com uma demanda
de 111,97 T\acrshort{wh} e uma parcela equivalente à 26\% do total no ano de 2011, 
exprimindo a importância desse setor para explorar a capacidade de economia de 
eletricidade através de \gls{ee}. 

\begin{figure}[h!t]
\centering
\includegraphics[width=.6\textwidth]{imagens/consumo_por_setor.pdf}
\caption[Consumo de eletricidade por setor em 2011.]
{O consumo de eletricidade por setor em 2011. Baseado em \cite{ben2012}.}
\label{fig:eletricidade_por_setor}
\end{figure}

De acordo com o último estudo do \gls{beu}\footnote{O \gls{beu} é realizado em
intervalos decenais desde 1985.} com o ano base de 2004 \cite{beu}, o potencial de 
economia no consumo de eletricidade nesse setor chega a 13,60 T\acrshort{wh}, 
ou 1,17 milhões de \acrshort{toe}. Apenas como figura de mérito, quando considerando 
melhorias de eficiência em todas as fontes energéticas, o setor residencial
possui o terceiro maior potencial de economia, após o setor industrial e de
transporte, com uma capacidade de economia de 2,97 milhões de 
\acrshort{toe}. No entanto, ressalta-se que esses valores de potenciais de economia 
apresentados são aproximados e reduzidos em relação ao valor real, onde se
considera apenas a perda de energia na primeira transformação do processo
produtivo. Também não são consideradas possíveis alterações no consumo de fontes
energéticas, como a mencionada alteração de biomassa para eletricidade, uma
fonte menos poluente e mais eficiente. Finalmente, deve-se acrescentar que 
esse potencial é calculado utilizando o rendimento de equipamentos no estado 
da arte entre aqueles normalmente comercializados, e não os possíveis de 
se alcançar quando considerando a literatura técnica. Os dados  
do \gls{beu} são normalmente utilizados para os estudos em \gls{ee} para o setor
industrial e comercial.

No entanto, os dados utilizados pela \gls{epe} para os estudos atuais de \gls{ee} no setor
residencial, como os \glspl{pde} e \gls{pne2030}, por sua vez, são das
informações obtidas na última \gls{pph} realizada no Brasil entre 2004--2006 que 
possibilitam o estudo baseado em uma abordagem desagregada. Essa abordagem 
depende do número de domícilios, a posse média e hábito de consumo específico 
dos equipamentos eletrodomésticos --- que estão implicitamente indicados na curva de carga
ilustrada na Figura~\ref{fig:curva_carga} para a região sudeste --- e rendimento médio desses 
equipamentos no país, informação baseada nas tabelas do \gls{pbe}, coordenado pelo \gls{inmetro}. 
Também são utilizadas variáveis agregadas para o ajuste do modelo em questão, 
sendo elas a relação entre o número de consumidores residenciais e população 
(que permite a projeção do número de consumidores a partir da projeção da população), 
e consumo médio por consumidor residencial
\cite{epe_eficiencia_2012,pde_2020,pne30_eff_energ}. 

% TODO Inserir a figura aqui de posse média
\begin{figure}[h!t]
\centering
\includegraphics[width=.9\textwidth]{imagens/curva_demanda_sudeste.pdf}
\caption[Curva de carga média para a região sudeste, ano base 2005.]{Curva de
carga média para a região sudeste, ano base 2005. Extraído de
\cite{result_procel_2005}.}
\label{fig:curva_carga}
\end{figure}

Esses estudos exploram a conservação de energia no ganho de eficiência 
de equipamentos eletrodomésticos, mas não consideram melhorias possíveis
em mudanças nos hábitos de consumo. Isso se justifica uma vez que a abordagem 
representa o ganho de conservação de energia através do progresso autônomo
ou tendencial, progresso esse ``que se dá por iniciativa do mercado, sem a
interferência de políticas públicas de forma espontânea, ou seja, através da
reposição natural do parque de equipamentos por similares novos e mais
eficientes ou tecnologias novas que produzem o mesmo serviço de forma mais
eficiente'' \cite[p.1]{pnef}, assim como por ``efeitos de programas e ações de
conservação já em execução no país'' \cite[p.247]{pde_2020}. 

Por outro lado, o progresso induzido refere-se à ``instituição de programas e ações
adicionais orientados para determinados setores, refletindo políticas púbicas;
programas e mecanismos ainda não implantados no Brasil'' \cite[p.247]{pde_2020}.
As políticas de progresso induzido estão detalhadas no \gls{pnef}, e a
estratégia adotada no tocante a mudanças nos hábitos de consumo é através da
educação com os programas do \gls{procel} e \gls{conpet}, \textlabel{com as
linhas}{text:prog_cepel}: Eficiência Energética na Educação Básica; Eficiência 
Energética na Formação Profissional; e Rede de Laboratórios e Centros de Pesquisa em Eficiência
Energética \cite[cap.5]{pnef}. No caso do \gls{procel}, investiu-se um montante superior a
R\$~4,5 Mi em 2011 em projetos voltados para o desenvolvimento e
aperfeiçoamento dessas três linhas \cite{procel_resultados_2012}, trabalhando na
coincientização, sensibilização e informação para obter uma melhor \gls{ee},
atuando nos três níveis de educação.

Entretanto, estudos no exterior mostram um potencial ainda a ser explorado de
econômia de energia elétrica no setor residencial, através do retorno de
informação da utilização de energia para o consumidor, uma estratégia de
progresso induzido que pode ser adicionada aos programas de \gls{ee} no Brasil. 
Conforme será visto na próxima sessão, esse potencial depende de aspectos, como a quantidade 
de informação retornada ao consumidor, essa inerentemente ligada à quantidade de
investimento utilizada na tecnologia para permitir um maior retorno, mas ainda,
dependem do modo em que esse retorno é fornecido para o mesmo no intuíto de
motivar ações sustentaveis de energia, sendo um problema complexo dependente de
aspectos sociais, culturais e psicológicos. 

Por fim, outros setores também podem reduzir seu consumo com políticas de
\gls{ee} aplicadas para o setor residencial devido a sua natureza similar a esse
setor --- nesse caso, não se referindo somente à politica de \gls{ee} sugerida. 
Os setores público e comercial, por exemplo, possuem prédios com natureza
de consumo correspondente ao do setor residencial, de forma que estratégias 
desenvolvidas possam sinergeticamente apresentar um potencial maior de 
economia de energia na matriz brasileira, em especial no que concerne estudos
para melhor eficiência de equipamentos.

\section{Expandindo o Potencial de Eficiência Energética através do Retorno de
Informação de Consumo}
\label{sec:ee_res_exp}

% Invisibilidade da eletricidade para os consumidores residencias
As novas fontes energéticas, dentre elas a eletricidade e gás natural
que atendem a demanda dos consumidores residenciais para a vasta 
variedade de serviços nas quais são utilizadas 
--- desde cocção, condicionamento do ambiente, a lazer e 
entreterimento ---, fluem invisivel e silenciosamente para seus domicílios, sem
deixar qualquer traço notável de sua utilização além do efeito final desejado pelo
consumidor. Para eles, o único retorno de seu consumo é informado na conta
apresentada pela concessionária, fornecida em um longo período após o consumo
(mensalmente, por exemplo). As informações nas contas são precárias, não informando 
muito além do total de energia consumido e o preço de energia. 
Os usuários não tem como inferir quais são os meios de uso final que demandam 
maior energia, nem a que ponto possíveis mudanças podem afetar sua demanda, 
seja através da mudança de seus hábitos ou na escolha 
de aparelhos mais eficientes. Atualmente, os usuários estão cegos quanto a essas
mudanças, não é possível vizualisar a energia que consomem. Além disso, as
informações fornecidas não permitem o consumidor comparar seu consumo com o de
outros, de modo que ele não é capaz de criar uma referência social para seu consumo.
Sem uma referência, o consumidor tem dificuldades para determinar se o consumo é
excessivo ou moderado e se é necessário algum tipo de intervenção 
\cite{aceee_2010_estudos_feedback}.

% Modos de alterar o hábito de consumo
Estratégias para intervir no comportamento podem ser classificadas de dois modos
\cite{aceee_2010_estudos_feedback,2009_epri}:
\begin{enumerate}
\item \textbf{Antecedentes}, que envolvem esforços para influênciar o que define 
um comportamento antes de sua realização; 
\item \textbf{De consequência}, que buscam alterar o que determina o 
comportamento após a sua ocorrência. 
\end{enumerate}

Exemplos de estratégias antecedentes são campanhas de informação 
com o objetivo de aumentar o conhecimento público sobre o impacto de suas 
escolhas e das opções para econômia de energia disponíveis --- como 
as já citadas ações do \gls{procel} e \gls{conpet} (ver p.~\pageref{text:prog_cepel}) ---,
engajar o indivíduo com um compromisso de mudança, criar metas de mudança
comportamentais, ou modelar e demonstrar o comportamento desejado. Já para
estratégias de consequência se pode citar recompensas, punições ou o
retorno de informação \cite{aceee_2010_estudos_feedback,2009_epri}. 

Iniciativas utilizando o retorno de informação mostraram-se altamente eficientes
em mudanças comportamentais com relação ao consumo energético \cite{
aceee_2010_estudos_feedback,2009_epri,2012_schleich__austria,
2011_zhifeng_smart_energy_savings,2006_darby,2009_nber_studies_us,
ucla_studies_1975_2011_usa}. O uso do retorno
de informação basea-se em que tanto resultados positivos ou
negativos podem modelar o comportamento. Resultados atribuidos como positivos 
irão torná-los em comportamentos mais atraentes, enquanto a atribuição de
resultados negativos propiciam comportamentos ruins em menos desejáveis. 
Sempre que possível, a atribuição negativa 
deve ser evitada, pois ela tende a reduzir a motivação e não coloca nada no 
lugar do comportamento evitado \cite{2010_aspectos_psicologicos_usa}.

A questão, por outro lado, não é apenas fornecer retorno do consumo ao usuário
final --- a própria conta de energia pode ser encarada como um meio de retorno
---, mas como o retorno pode ser utilizado para efetivamente motivar pessoas
para reduzir o seu consumo. Algumas considerações devem ser tomadas: primeiro,
quais são os tipos de retornos disponíveis (Subsessão \ref{ssec:ret_tipos}) e,
dentre eles, quais tem mostrado resultados mais eficientes na redução do
consumo energético (Subsessão \ref{ssec:ret_eff})? O que mais deve ser levado
em consideração quando preparando tais programas e estudos de \gls{ee}
(Subsessão~\ref{ssec:ret_outros})? Quais são as técnologias disponíveis para
fornecer essa informação e suas tendências (Subsessão \ref{ssec:ret_tec})?
Ainda, pessoas possuem diferentes atitudes, crenças e valores, sendo motivadas
de modos distintos. Uma breve consideração sobre a perspectiva psicológica que
envolve mudança comportamental será realizada (Subsessão \ref{ssec:asp_psic})
uma vez que esse aspecto é de principal relevância para o sucesso dos
programas.  Do mesmo modo, a apresentação visual da informação também irá
influênciar no êxito, e por isso o tema também será colocado em pauta
(Subsessão~\ref{ssec:asp_visuais}). De nenhuma maneira as breves considerações
realizadas nas subsessões sobre os aspectos psicológicos e visuais devem
substituir a análise de profissionais dessas áreas, servindo apenas para chamar
atenção para a importância, bem como introduzir os leitores, aos temas. 

\subsection{Tipos de Retorno}
\label{ssec:ret_tipos}

A categorização dos tipos de retorno que será apresentada se iniciou em
\cite{2000_darby} e depois foi aprimorada por \cite{2009_epri}. A primeira divisão
toma em conta o modo no qual o retorno é fornecido, sendo possíveis o retorno 
direto ou indireto. O termo \emph{indireto} é
utilizado quando há alguma espécie de processamento antes de atingir o
consumidor, enquanto \emph{direto} determina o retorno
instantaneamente\footnote{A palavra instantâneo é empregada apesar da
existência de atraso entre a informação e o consumo pois esse
retardo, normalmente na ordem de segundos, é desprezível para os fins.} entregue
ao usuário. Em seguida, realiza-se uma divisão em termos de frequência,
diferenciando quatro tipos de retorno \emph{indiretos}, e dois retornos
\emph{diretos}. A divisão é apresentada resumidamente em ordem crescente em 
termos de custo e 
quantidade de informação disponível, onde os dois retornos \emph{diretos} estão no
final da lista \cite{aceee_2010_estudos_feedback,2009_epri}:

\begin{enumerate}
\item \textbf{Retorno por Faturamento Simples}: conta de energia contendo 
k\acrshort{wh} 
consumido, o preço da tarifa unitária ($\text{R\$}/$k\acrshort{wh}), o custo 
total e outros possíveis ônus. Nessa forma de retorno, normalmente se carece 
estatísticas comparativas ou qualquer informação detalhada sobre os aspectos 
temporais do consumo;
\item \textbf{Retorno por Faturamento Aprimorado}: fornece informações mais 
detalhadas
sobre o padrão de consumo de energia, incluindo em alguns casos estatísticas
comparativas, tanto comparando o maior consumo do mes atual e sua despesa
aliados ao consumo histórico e/ou a comparação com outros domicílios
pertencentes ao grupo do consumidor;
\item \textbf{Retorno Estimado}: essa abordagem utiliza geralmente de técnicas
estatísticas para desagregar o total de energia baseado no tipo do
domicílio do consumidor, informação de aparelhos e dados de faturamento. O
retorno resultante fornece um relato detalhado do uso de eletricidade pelos
utensilios e dispositivos de maior importância. A forma mais comum é através de
ferramentas de auditoria de energia residencial baseadas na internet, oferecida
por um fornecedor de serviços a seus consumidores;
\item \textbf{Retorno Diário/Semanal}: esses relatórios utilizam a média de
dados e frequentemente incluem estudos de leitura dos medidores pelos próprios 
consumidores, assim como estudos nos quais individuos são providos com
relatórios mensais ou semanais do fornecedor de serviços ou entidade de
pesquisa;
\item \textbf{Retorno em Tempo-Real}: fornecido por dispositivos que exibem o
consumo em (praticamente) tempo-real e informações de custo da energia em nível
agregado domiciliar;
\item \textbf{Retorno em Tempo-Real Desagregado}: nesse caso, as informações são
exibidas desagregadas ao nível dos utensílios.
\end{enumerate}

\subsection{Resultados por Tipo de Retorno}
\label{ssec:ret_eff}

No exterior, há uma farta quantidade de estudos envolvendo o tema, com estudos iniciando na
década de 1970 em resposta à Crise do Petróleo, que tiveram um declínio durante
a década posterior. O interesse retornou a pauta recentemente devido à crescente 
preocupação com o meio ambiente e mudanças climáticas, assim 
como o aparecimento de novas possibilidades tecnológicas 
associadas a \gls{ict} \cite{aceee_2010_estudos_feedback}. É
possível encontrar pesquisas nas quais se compilam diversos estudos
de retorno de informação para consumidores residenciais no intuíto de generalizar 
resultados \cite{aceee_2010_estudos_feedback,2011_zhifeng_smart_energy_savings,
2006_darby,2009_nber_studies_us,ucla_studies_1975_2011_usa}.

Infelizmente, o mesmo não pode ser dito para o Brasil, onde não se
encontrou pesquisas nesse sentido. Assim, este trabalho irá se guiar
na pesquisa que mais se destacou com resultado de estudos no exterior
\cite{aceee_2010_estudos_feedback}. Nela se revisou 57 estudos
primários de retorno, realizados em países desenvolvidos incluindo
\gls{eua} (58\% dos estudos), quatro países da Europa Ocidental
(Países Baixos, Finlândia, Dinamarca e Reino Unido, com um total de
22\%), Canada (15\%), Japão (5\%) e Australia (1 estudo). Em termos de
retorno, metade dos estudos envolvem retorno \emph{indireto}, dentre
os quais 11 envolvem Faturamento Aprimorado, três estudos envolvem o
uso de Retorno Estimado e 15 estudos consideram Retorno
Diário/Semanal. Os remanescentes envolvem retorno \emph{direto}, dos
quais 23 exploram retorno agregado e outros seis estudos onde são
fornecidos a informação em tempo real no nível dos utensílios.
Salienta-se que pesquisas nesse sentido no Brasil são necessárias para
validar os resultados, tomando em posse as diferenças culturais,
sociais e econômicas quanto a nossa realidade.

%Realiza-se nela uma divisão quanto à data em que os estudos foram realizados,
%utilizando o termo \gls{ece} para estudos realizados entre 1974 e 1994 
%e \gls{emc} nos anos conseguintes até 2009. 
%Essa divisão é importante pois os novos estudos provêm de
%tecnologias mais recentes, em especial dispõe de novas tecnologias de \gls{ict}.
%Essas oferecem meios inovativos de aumentar o efeito de
%mecanismos de retorno de informação, assim como reduziram os custos associados
%com fornecer retorno frequente e confiável para consumidores residenciais.
%Utilizou-se dois terços dos estudos para a era mais recente, \gls{emc}. 

%Com o objetivo de explorar melhor as diferenças de resultados, 
%foram realizadas mais divisões qualitativas: 
%quanto ao tamanho do estudo, referindo-se à quantidade de domicílios 
%envolvidos, no qual considera o estudo como pequeno
%(32\%) quando utilizando menos de 100 participantes, e grande caso 
%contrário; quanto a duração do estudo, considerada curta (40\%) para períodos
%inferiores à seis meses, e longa caso contrário; e elementos motivacionais
%utilizados além de fatores economicos ou apelo ao meio ambiente, como atribuir
%metas, competições e engajamento, e normas sociais.

Um resumo dos resultados obtidos nessa referência, para os estudos realizados entre 1995 e 
2010 (cerca de dois terços dos revisados), estão na Figura~\ref{fig:potencial_consumo_retorno}. 
A redução do consumo mostrada leva em conta os
resultados globais, ou seja, considerando a taxa esperada de adesão da população aos
programas de \gls{ee}, supondo que os mesmos terão participação voluntária. 

Identifica-se nos resultados que os tipos de retorno são tanto incrementais
em custo e complexidade, quanto nos resultados de economia energética. Assim, 
é natural a implementação do sistema de retorno ser realizada de 
maneira continua, aplicando sistemas já disponíveis enquanto se realiza 
investimento em tecnologia para o desenvolvimento do
próximo nível de informação. Os desenvolvedores devem
manter o sistema o mais flexível possível, sendo desenvolvidos 
sempre preparados para a mudança e considerando o surgimento de novos 
mecanismos de retorno com o avanço da tecnologia 
\cite{aceee_2010_estudos_feedback}. Por exemplo, por ora é possível obter
econômia de energia utilizando um sistema de baixo custo, como o Faturamento
Aprimora, que informa melhor o consumidor, ou até mesmo, com mais ambição, 
fornecer o Retorno Diário/Semanal.

\begin{figure}[h!t]
\centering
\includegraphics[width=\textwidth]{imagens/estudo_economia_aceee.pdf}
\caption[O potencial de consumo para cada tipo de retorno]
{O potencial de economia de consumo para cada tipo de retorno. Estudos
no exterior e em países desenvolvidos. Adaptado de
\cite{aceee_2010_estudos_feedback}.}
\label{fig:potencial_consumo_retorno}
\end{figure}


\subsection{Indo Além dos Resultados}
\label{ssec:ret_outros}

Ademais, outras considerações devem ser levadas quando no desenvolvimento 
de programas ou estudos de \gls{ee} através do retorno do consumo de energia
\cite{aceee_2010_estudos_feedback,2006_darby,2009_epri}:

\begin{itemize}
\item \textbf{Retorno Indireto versus Direto}: 
Os retornos indiretos são mais adequados que diretos 
para demonstrar o efeito de mudanças no condicionamento do ambiente, 
composição domiciliar e o impacto de investimentos 
em medidas de eficiência ou utensílios de alto consumo. Já o retorno instantâneo 
se adequa, geralmente, no fornecimento do impacto do consumo de aparelhos com 
usos de energia menores;
%\item \textbf{Era do Programa/Estudo}: Estudos anteriores a 1995 (não utilizados
%para gerar os resultados apresentados na 
%Figura~\ref{fig:potencial_consumo_retorno}) 
%apresentam economia de energia maiores aos posteriores. 
%Assim, recomenda-se a sua não utilização com o objetivo de evitar espectativas 
%infladas sobre o potencial de economia atualmente;
\item \textbf{Participação Voluntária}: Programas nos quais os usuários tem de
optar por não participar (\emph{opt-out}) tiveram adesão significamente 
maior (75\%-85\%) do que aqueles nos quais os usuários escolhem em colaborar
(\emph{opt-in}, participações menores a 10\%), sendo assim recomendada
a primeira abordagem para maximizar a participação dos consumidores;
\item \textbf{Elementos Motivacionais}: A utilização de outros elementos para
motivar a população aquém do financeiro e apelo ao meio ambiente mostram-se 
importantes para aumentar a eficiência dos programas de \gls{ee}. São citados
como exemplo criar metas, compromissos, competições e normais sociais 
(tanto descritivas quanto injutivas). A Subsessão~\ref{ssec:asp_psic} 
irá tratar do tema com mais detalhes;
\item \textbf{Contexto Regional}: Diferenças regionais e culturais afetam os 
resultados. Os resultados para a Europa Ocidental superam os obtidos nos
\gls{eua}, podendo ser atribuidos as diferenças em como o 
discurso sobre as mudanças climáticas pelas lideranças políticas nas duas regiões 
é feito e assim a preocupação com o tema pela população. Nesse caso, chama-se atenção
novamente às estratégias antecedentes no intuíto de preparar a população para os programas e
maximizar os resultados. Outro aspecto importante é a necessidade de estudos
sobre o tema a fim de especificar como o brasileiro irá reagir em tais programas;
\item \textbf{Duração do Estudo e Persitência dos Resultados}: Quando os 
estudos são de menor duração ($< 6$ meses) se obtém resultados mais 
eficientes (média de 10,1\% de economia) que estudos mais longos (7,7\%),
discrepância essa atribuida a inaptidão de estudos curtos em observar variações
sazionais na utilização de energia. Alguns estudos indicam que se faz necessário 
a presença do retorno em longo termo para que os resultados persistam, 
enquanto outros apontam a necessidade do retorno continuamente, enfatizando
assim a necessidade na extensão dos programas de \gls{ee};
\item \textbf{Tamanho do Estudo}: Estudos com grandes ($> 100$) amostragens
domiciliares tendem a ter resultados mais modestos. Como esses estudos tem uma
representatividade melhor das residencias, isso indica que programas de
\gls{ee} em larga escala também devem apresentar resultados mais modestos que
aqueles apresentados na Subsessão~\ref{ssec:ret_eff}.
Ainda, esses estudos mostram-se menos suscetíveis às oscilações 
quanto a duração dos estudos;
\item \textbf{Resposta de Ponta e Demanda versus Economia Fora de Ponta}:
Reduções de pico e demanda são de particular interesse das concessionárias que
buscam atender essencialmente o mesmo nível de serviços mas com custos totais
menores. Há dois modos de obter tal efeito: com uma melhoria em \gls{ee} ou
através do deslocamento de parte do consumo no horário de ponta para fora da
ponta. O interesse em resposta de demanda, ou seja, em reduzir o
consumo durante os horários de ponta difere dos programas de \gls{ee} que focam
em ter reduções eficientes economicamente durante todo o ano. Ainda que não seja
deprezível, programas de resposta de demanda apresentam economia de energia
bastante baixos quando em comparação aos de \gls{ee}. Os consumidores
normalmente não percebem a diferença entre os dois programas, do mesmo modo 
que a integração dos programas é plausível e sinergética, 
onde estudos mostram que a junção causa melhores resultados tanto 
em econômia de energia quanto na 
redução de picos, por isso, sendo interessantes tanto do lado do consumidor 
quanto para a consessionária. Desta forma, a abordagem ótima ao tema deve
ser conseguir todos os meios economicamente atraentes de reduzir o desperdício 
e ineficiências antes de procurar oportunidades restantes de reduzir cargas 
durante os picos;
\item \textbf{Hábitos, Escolhas e Estilos de Vida}: Dentre os tipos de
comportamentos de \gls{ee} e conservação, os que aparecem mais frequentemente são 
investimentos em novos equipamentos e utensílios em populações mais ricas, sendo
geralmente empreendido em conjunto com mudança de residência ou melhoria no
estilo (referido em oposição a funcional) do domícilio. Para a maioria da
população, os domícilios obtém melhor \gls{ee} através da mudança de hábitos e 
rotina, ou pela avaliação dos comportamentos relacionados a energia. Esses
comportamentos de \gls{ee} são motivados assim por uma variedade de fatores,
incluindo interesse próprio (financeiro) e outros motivos altruístas e
preocupações cívicas. Desta forma, programas de \gls{ee} que procuram apenas 
a instalação de equipamentos mais novos e eficientes irão 
desperdiçar o potencial relacionado à mudança comportamental, assim como
programas que apelam apenas para o interesse financeiro não irão influenciar um
largo grupo de fatores que motivam as pessoas para agir;
\item \textbf{Segmentação Populacional}: Poucas pesquisas exploraram como o
potencial de redução de consumo é afetado pelas diferentes classes sociais.
Desses estudos, as descobertas sugerem grandes níves de economia tendem estar
associados a alto nível educacional e renda, grandes residenciais e dentre elas
as com maior número de pessoas, consumidores jovens e/ou com grande tendência a
valores ambientais. Essas considerações indicam que os potenciais apontados
provavelmente estão inflados para a aplicação no Brasil como um todo.
\end{itemize} 

\subsection{Tecnologias e Tendências}
\label{ssec:ret_tec}

Como constatado, os medidores atualmente utilizados pelas 
concessionárias, os medidores analógicos e eletrônicos, permitem fornecer um 
retorno de baixo custo mas com um potencial melhor de economia de energia, 
o Faturamento Aprimorado. As contas de energia de companias como a Light,
Ampla, Cemig e Eletropaulo fornecem o histórico de consumo dos últimos 12 meses,
uma informação que pode auxiliar o consumidor, já podendo ser consideradas um 
Faturamento Aprimorado. Entretanto outras informações podem ser utilizadas, 
como referencias do consumo acumulado e, em especial, comparações do consumo 
com o de vizinhos ou grupo pertencente. Também é possível estimar o uso 
energético por uso-final utilizando os valores médios de consumo das residenciais
para cada uso no sentido de auxiliar o cliente. A ideia é
transformar a conta de energia em uma espécie de relatório do consumo energético,
com um visual mais atraente (ver Subsessão~\ref{ssec:asp_visuais}), 
contendo gráficos e informações no sentido de atrair o consumidor a se 
preocupar com o tema, sendo esse o primeiro passo \cite{2009_epri}.

No entanto, o sistema elétrico atual está se tornando obsoleto para atender aos problemas de 
aumento de carga nos centros urbanos devido ao crescimento do setor 
de serviços e do consumo das residencias. Há uma presença cada vez maior
de cargas eletrônicas injetando harmônicos e a geração centralizada exige 
excessivamente da capacidade de transmissão e distribuição, 
sobrecarregando as linhas nesses grandes centros que nem sempre podem 
corresponder à necessidade de novas linhas. A falta de informações sobre o 
estado do sistema dificultam a operação e planejamento de uma rede cada vez mais 
sobrecarregada. As \gls{ict} revolucionaram as redes de telecomunicações e 
serão a tendência para a criação das redes elétricas inteligentes (\emph{smart 
grids}), o novo sistema elétrico que tem como objetivo responder a essas
dificuldades. Ainda não foram definidas todas as características desse sistema,
contudo, as principais características são o uso de comunicações em tempo real
para o controle e informação, o uso massivo de sensores e medidores para
monitoramento do sistema, faturamento com preços para o momento de uso, 
gestão pelo lado da demanda, a integração de 
componentes avançados como linhas de transmissão supercondutoras,
armazenamento de energia, eletrônica de potência, geração distribuida
etc.  \cite{dissert_caires,aceee_2010_estudos_feedback}

Os aparelhos eletrônicos de medição utilizados nas redes inteligentes, referidos
neste trabalho como medidores inteligentes (\emph{advanced/smart metering}), irão 
fornecer uma gama maior de informações em tempo-real para as concessionárias,
melhorando a operação e planejamento. Ao mesmo tempo, será possível a
concessionária comunicar-se com o cliente, oferecendo incentivos (como
descontos) para reduções de carga durante os horários de ponta, 
outros planos de tarifação com preços dinâmicos de acordo com os horários,
aumentando a interação da concessionária com o cliente. 

Por outro lado, pelo ponto de vista da demanda (ou dos consumidores) essa 
informação também estará disponível, trazendo uma gama de novas oportunidades 
para os usuários participarem ativamente. Mais especificamente, na abordagem do
tema atual, os medidores inteligentes oferecem uma base
a ser explorada para fornecer o retorno em larga escala para os consumidores, 
tanto o direto quanto indireto. Nos medidores utilizados nos \gls{eua}
foram apontadas algumas dificuldades técnicas para esse fornecimento,
sendo elas: a necessidade de grande quantidade de energia para enviar
um sinal frequente ao consumidor e a
atender a necessidade do sinal ser enviado em intervalos frequentes,
atendendo a taxa escolhida de 7 s. Com um custo adicional, um estudo
na industria estadunidense mostrou que
é possível dos medidores terem seu \emph{hardware} substituídos 
no futuro para que possam fornecer medições de pequena energia e \emph{chips} 
de comunicação para habilitar dados de utensílios específicos, assim como 
automação para grandes cargas, como unidades de condicionamento ambiental, 
bombas etc. Desta forma, sendo possível 
fornecer tanto a tecnologia para o Retorno em Tempo Real com a utilização de 
mostradores dentro do domicílio, quanto o Retorno em Tempo Real Desagregado
\cite{aceee_2010_estudos_feedback}.

Algumas empresas se estabeleceram no novo mercado para informar o consumidor
sobre o seu uso de energia e em auxiliá-lo nas atitudes para reduzir o seu
consumo, antes mesmo que estivessem disponíveis os medidores inteligentes. 
Elas fornecem o retorno indireto em alguns países
desenvolvidos, dentre eles o \gls{eua}, Australia, Nova Zelândia, Reino Unido.
Dentre essas empresas, faz-se referência a \emph{Positive Energy} 
\cite{opower_site} e \emph{C3 Energy} \cite{c3_site} que disponibilizam seus 
serviços, organizados na Tabela~\ref{tab:servicos_ret_ind}, 
utilizando os dados da concessionária, independente
quando presentes na residência os medidores convencionais ou inteligêntes. 
É importante notar que as abordagens utilizadas por empresas nesse
ramo utilizarão análises mais complexas conforme a presença de dados
mais detalhados, frequentes e desagregados estejam disponíveis.

\begin{table}[h!t]
\resizebox{\textwidth}{!}{
\begin{tabular}{m{2.5cm}m{5cm}m{8cm}}
\hline \hline 
\centering{\textbf{Empresa}} & \textbf{Tecnologia de Retorno} & 
\textbf{Principios Comportamentais} \\
\hline \hline
\centerline{\textbf{Positive}}\centerline{\textbf{Energy}}\centerline{\cite{opower_site}} & 
Dependendo da concessionária, envia correspondencias mensais ou
trimestrais e/ou fornecem um portal na internet com novas redes sociais &
\emph{Tipo de Retorno}: Retorno indireto incluindo informação sobre o domicílio
e conselhos, auditorias de energia através do uso da \emph{web}, análise de 
faturamento, consumo estimado por aparelho, \gls{co2}, k\acrshort{wh} e \$.

\emph{Principios Comportamentais}: Comparações sociais, metas, comparações
pessoais e plano de ações. \\
\hline
\centerline{\textbf{C3}}\centerline{\textbf{Energy}}\centerline{\cite{c3_site}} & 
Portal de comunidade social com retorno de consumo de energia e água & 
\emph{Tipo de Retorno}: Retorno indireto incluindo informação sobre o domicílio
e conselhos, auditorias de energia através do uso da \emph{web}, análise de 
faturamento, consumo estimado por aparelho, \gls{co2}, k\acrshort{wh}, \$ e
outras unidades.

\emph{Principios Comportamentais}: Comparações sociais, metas, competições
redes sociais, comparações pessoais e plano de ações. \\
\hline \hline
\end{tabular}
}
\caption[Empresas utilizando informação da concessionária e as 
oportunidades e insentivo de economia de energia oferecidas.]
{Empresas utilizando informação da concessionária e as 
oportunidades e insentivo de economia de energia oferecidas. Extraído e
atualizado de \cite[tradução própria]{aceee_2010_estudos_feedback}.}
\label{tab:servicos_ret_ind}
\end{table}

Já o retorno direto pode ser encontrado através de
mostradores de energia no domicílio. A Tabela~\ref{tab:servicos_ret_dir}
identifica alguns dos mostradores oferecidos atualmente e suas propriedades.
Muitas vezes as companias oferecem também análises e estimativas do consumo 
especifico de aparelhos, comparações sociais e outros principios para motivar os
consumidores a economizar energia. A informação de consumo de aparelhos
especifico ou é estimada, ou realizada através de sensores nos aparelhos.

\begin{table}[h!t]
\resizebox{\textwidth}{!}{
\begin{tabular}{p{4cm}p{7cm}p{7cm}}
\hline \hline 
&
\multicolumn{1}{c}{\textbf{The Energy Detector TED} \cite{ted_site} }& 
\multicolumn{1}{c}{\textbf{Wattson}                 \cite{wattson_site}}\\
\hline \hline
\textbf{Descrição da \newline Tecnologia} & 
\emph{Software} de suporte, aplicativos para celular &
\emph{Software} de suporte com acesso a comunidades \\
\hline 
\textbf{Mecanismos de \newline Retorno} & 
Mostradores em tempo real de k\acrshort{watt}, \$/hr, \gls{co2}, consumo e gastos
diários, conta estimada em k\acrshort{wh} e \$, pico de consumo, voltagem
min/max e custo/demanda projetada &
Mostradores em tempo real aproximado do consumo em \acrshort{watt},
k\acrshort{watt}, conta estimada. Leituras entre 3 a 20 s. Brilha conforma o
consumo: azul para consumo baixo; roxo para médio; vermelho para alto. \\
\hline
{\multirow{5}{4cm}{\textbf{Principios Comportamentais}}} &
\multicolumn{2}{c}{\emph{Retorno de Informação:}}
\\
& & \\
& 
\multicolumn{2}{p{14cm}}{
Retorno direto incluindo conselhos, auditorias de energia baseadas na \emph{web},
análise do consumo, estimativa de consumo por utensílios, \gls{co2} e \$.
}
\\
& & \\
&
\multicolumn{2}{p{14cm}}{\emph{Motivações Oferecidas:}
\centering 
Comparações sociais, metas, comparações pessoais e etapas de ações.
}
\\
\hline \hline
& 
\multicolumn{1}{c}{\textbf{PowerCost Monitor} \cite{powercost_site}}& 
\multicolumn{1}{c}{\textbf{Efergy Elite}      \cite{efergy_site}}\\
\hline \hline
\textbf{Descrição da\newline Tecnologia} & 
\emph{Software} de suporte, aplicativos para celular &
\emph{Software} de suporte, aplicativos para celular \\
\hline
\textbf{Mecanismos de\newline Retorno} & 
Mostradores em tempo real aproximado do consumo em
k\acrshort{watt} e \$/hr, pico de consumo nas últimas 24 horas, contagem de
k/\acrshort{wh} (reiniciável), recurso para medição de aparelhos específicos. &
Mostradores em tempo real aproximado do consumo em k\acrshort{watt} e \$/hora
(leituras em 6, 12 ou 18 s), informação de consumo média por hora, semanal,
mensal. Alarmes para consumo alto. \\
\hline
{\multirow{5}{4cm}{\textbf{Principios Comportamentais}}} &
\multicolumn{2}{c}{\emph{Retorno de Informação:}} \\
& & \\
& 
Retorno direto incluindo conselhos, auditorias de energia baseadas na \emph{web},
análise do consumo, estimativa de consumo por utensílios, \gls{co2} e \$.  &
Retorno direto, análise de consumo, estimativa de consumo em \$.  \\
& & \\
&
\multicolumn{2}{p{14cm}}{\emph{Motivações Oferecidas:} 
\centering Metas e comparações pessoais}
\\
\hline \hline 
\end{tabular}
}
\caption[Especificações de mostradores domiciliares disponíveis.]{
Especificações de mostradores domiciliares disponíveis. Tradução própria de
\cite{aceee_2010_estudos_feedback}.}
\label{tab:servicos_ret_dir}
\end{table}

% TODO Atualizar a tabela

Uma outra maneira para fornecer o
Retorno em Tempo Real Desagregado é através do uso de um \gls{nialm}
(Capítulo~\ref{cap:nialm}). Essa técnica, ainda em desenvolvimento,
coloca o peso da desagregação da informação no \emph{software},
reduzindo a necessidade de investimento em sensores e \emph{hardware},
sendo assim um método potencialmente mais favorável economicamente
para a implementação de programas de \gls{ee} fornecendo esse tipo de
retorno, no entanto isso irá depender de suas limitações e eficácia.
No Brasil, os esforços da \gls{aneel} em regulamentar as
bases para os novos medidores inteligentes dão poder ao consumidor de
exigir à concessionária acesso às medições de tensão e corrente de
cada fase, como rege no art.~3$^o$ da Resolução Normativa n$^o$~502
\cite{ren502}. Ainda não se especificou a taxa de amostragem na qual
essas leituras serão disponiblizadas, por sua vez, pode-se
utilizar toda a infra-estrutura das colheitas de medidas e comunição
oferecida pelos medidores inteligentes no intuíto de maximizar o
custo-benefício. Caso a amostragem deles seja baixa, ou de interesse
aumentá-la para obter uma maior capacidade de identificação dos
aparelhos, o \gls{nialm} pode utilizar de um \emph{hardware} próprio
de medição.

Percebe-se que ainda é incerto se os medidores inteligentes são a melhor 
alternativa para fornecer retorno de informação, contudo parece natural sua 
utilização. Diversas tecnologias podem ser utilizadas, envolvendo, ou não, as 
concessionárias. Apenas com o desenvolvimento dessas tecnologias será
possível determinar as limitações, custos e vantagens para definir o que é 
economicamente mais atraente.

Finalmente, uma outra tendência é o uso de automação da rede doméstica. A
automação, além de melhorar a qualidade de vida dos consumidores, pode aumentar
o potencial de redução de consumo. Com uma maior capacidade de administração de
sua demanda sem grande esforço, facilita-se aos consumidores de realizarem a
mudança de hábitos no sentido de um comportamento sustentável, simplificando
a conduta do sistema elétrico de um modo mais econômico.

\subsection{Aspectos Psicológicos}
\label{ssec:asp_psic}

A tecnologia apenas concebe as possibilidades de informação a
serem repassadas ao consumidor, no entanto, a questão ainda está em como
apresentar essa informação e motivar o usuário para a mudança. 
O conselho de profissionais nos campos de psicologia, sociologia,
\emph{marketing}, mudança e economia comportamental serão criticos para
motivar, habilitar e continuamente empreender consumidores na gestão de sistemas 
de energia residenciais \cite{aceee_2010_estudos_feedback}. 
Um exemplo é a empresa \emph{Positive Energy}, que utiliza psicólogos para
auxiliar no desenvolvimento de suas tecnologias, disponibilizando ferramentas
sociais e estratégias persuativas para um engajamento maior de seus clientes.

Um levantamento das noções básicas da psicologia motivacional foi realizada em
\cite{2010_aspectos_psicologicos_usa}, assim como uma estrutura para os desenvolvedores 
da tecnologia aplicada nos programas de retorno de informação no sentido de motivar a 
mudança comportamental para uma melhor \gls{ee} e um mundo sustentável.
Será realizado um resumo desses tópicos a seguir guiado nessa referência, mas
vale enfatizar que os profissionais nessas áreas devem analisar o tema e
escolher a melhor abordagem a ser utilizada nas tecnologias desenvolvidas.

O objetivo é motivar o consumidor para a mudança através 
de recomendações levando em conta o processo de mudança comportamental 
do consumidor. Define-se a motivação como 
\cite[p.927-928, tradução própria]{2010_aspectos_psicologicos_usa}:

\begin{quote}
Motivação é um questionamento ao porquê do comportamento. Ela é um estado
interno ou condição (as vezes descrita como uma necessidade, desejo ou querer)
que serve para ativar ou energizar o comportamento. Motivação está fortemente
ligada a processos emocionais. Emoções podem estar envolvidas na iniciação
comportamental (como a emoção de solidão pode motivar a ação de procurar
compania). Em alternativa, o desejo para viver uma emoção em particular pode
também motivar para a ação (como a decisão de correr uma maratona pode ser
motivada pelo desejo de experimentar a sensação de realização de um feito).
\end{quote}

Ela é influênciada por ideáis psicológicos que foram aprendidos pelos
indivíduos. Nota-se que diferentes indivíduos tem ideáis psicológicos distintos,
estes estando apresentados em ordem decrescente quanto a possibilidade de 
sofrerem alteração:

\begin{itemize}
\item \emph{Atitudes} são pré-disposições aprendidas quanto a respostas
para uma pessoa, objeto ou ideia em um modo favorável ou desfavorável. Por
exemplo, o ato de tomar banho curtos devido a uma atitude favorável em respeito
ao meio ambiente;
\item \emph{Crenças} são os meios nos quais as pessoas estruturam seu
entendimento da realidade, refletindo a ideia do que é certo e o que é errado. 
A maioria das crenças são baseadas em experiências passadas, como a reciclagem 
faz bem ao meio ambiente.
\item \emph{Valores} são os fundamentos para o conceito de um indivíduo sobre si
mesmo. Podem ser conceituadas como ideáis comportamentais ou preferências por
vivências. No caso dos primeiros, valores funcionam como conceitos duradouros de
bem e mal, certo e errado, enquanto para preferências por vivências os valores
guiam individuos para vivenciarem situações nas quais as proporcionam certos
tipos de emoções. A Tabela~\ref{tab:valores} contém um subgrupo de valores
definidos por Rokeach e Maslow, onde se propõe que pessoas possuem
uma estrutura hierárquica ou prioritária de valores individuais. Rokeach
acredita que as diferenças no comportamento ocorrem devido a diferenças na
classificação de importância de valores, enquanto Maslow fornece uma ordem de
valores em níveis que serão priorizados pelos indivíduos pelos níveis mais
baixos antes dos níveis superiores.
\end{itemize}

\begin{table}[h!t]
\resizebox{\textwidth}{!}{
\begin{tabular}{ccc}
\hline 
\multicolumn{1}{|p{6cm}|}{\centering \textbf{Ideáis Comportamentais} (Rokeach)} & 
\multicolumn{1}{p{6cm}|}{\centering \textbf{Preferências por Experiências} (Rokeach)} &
\multicolumn{1}{p{6cm}|}{\centering \textbf{Preferências por Experiências} 
(Maslow, níveis em ordem crescente)} \\
\hline \hline 

\multicolumn{1}{|m{6cm}|}{
\textbf{Capaz}\newline Competente, eficiente. \newline 
\textbf{Prestativo}\newline Trabalhando para o bem-estar dos outros. \newline 
\textbf{Honestidade}\newline Sinceridade e confiável. \newline
\textbf{Imaginativo}\newline Ousado e criativo. \newline
\textbf{Independente}\newline Auto-confiante, auto-suficiente. \newline
\textbf{Intelectual}\newline Inteligente e pensativo. \newline
\textbf{Lógico}\newline Consistente e racional. \newline
\textbf{Obediente}\newline Atencioso e respeitoso. \newline
\textbf{Responsável}\newline Fidedigno e confiável.
} &
\multicolumn{1}{m{6cm}|}{
\textbf{Vida Confortável}\newline Uma vida próspera. \newline 
\textbf{Liberdade}\newline Independência e liberdade de escolha. \newline
\textbf{Saúde}\newline Bem-estar psicológico e físico. \newline
\textbf{Harmônia Interna}\newline Ausência de conflitos interiores. \newline
\textbf{Sentimento de Realização}\newline Uma contribuição duradoura. \newline
\textbf{Reconhecimento Social}\newline Respeito e admiração. \newline
\textbf{Sabedoria}\newline Um entendimento maduro da vida. \newline
\textbf{Um Mundo Belo}\newline Beleza da natureza e das artes.
} &
\multicolumn{1}{m{6cm}|}{
\textbf{Psicológico}\newline Homeostáse e apetites. \newline
\textbf{Segurança}\newline Segurança do corpo, emprego, recursos, familia, saúde,
propriedade. \newline
\textbf{Amor/Aceitação}\newline Afeição e ser aceito. \newline
\textbf{Estima}\newline Respeito próprio, auto-estima, estima dos outros \newline
\textbf{Realização Pessoal}\newline Encontrar satisfação pessoal e compreender seu
potencial.
} \\
\hline 
\end{tabular}
}
\caption[Valores propostos por Rokeach e Maslow.]
{Valores propostos por Rokeach e Maslow. Tradução própria de 
\cite[p.928]{2010_aspectos_psicologicos_usa}.}
\label{tab:valores}
\end{table}

Outra questão importante é a persistência ou durabilidade do
comportamento, ou seja, da capacidade do comportamento manter-se, sem
a necessidade de intervenções.  Para atingir esse objetivo é
aconselhável motivação intrínseca, que é a realização de uma atividade
pelas satisfações que a tarefa oferece, enquanto o seu oposto, a
motivação extrínseca, está ligada à realização para obter uma
consequência separável. Exemplos para o primeiro são: curiosidade,
competência e satisfação; no outro caso: incentivos materiais e
reconhecimento social.

Utilizou nessa referência o Modelo Transteórico\footnote{A referência
levanta outros modelos comumente utilizados e seus prós e contras.
Também há uma preocupação quanto a modelar a mudança através de
estados discretos, que podem não representar bem a realidade do
processo. A utilização do modelo é justificada por seu valor
heurístico, usado como um modelo simplificado de uma mudança ideal.},
onde se considera que o processo da mudança ocorre em uma série de
estados, sendo a motivação a força motriz para se deslocar entre os
estágios. O objetivo em cada etapa, assim como recomendações para o
atingir estão a seguir:

\begin{itemize}
\item Estrutura Exemplo --- \emph{Etapa}: explicação.
\begin{enumerate}
\item Objetivo nessa etapa.
\begin{enumerate}
\item Recomendação para atingir o objetivo.
\end{enumerate}
\end{enumerate}
\item \emph{Pré-contemplação}: o indivíduo está desencorajado,
relutante em manter atitudes em prol da mudança, mal-informado ou
desconhece o problema comportamental. Não há previsão de ação no
futuro, esse medido normalmente como os próximos 6 meses.
\begin{enumerate}
\item Apresentar informação em moderação com o próposito de plantar a
semente no sentido dos indivíduos tomarem conhecimento de seus
comportamentos (de consumo de energia) atuais problemáticos. A
moderação é importante pois uma maior intensidade irá, geralmente,
produzir menores resultados nessa etapa.
\begin{enumerate}
\item Fornecer retorno de informação personalizado notando tanto os
prós e contras de um comportamento \emph{não-sustentável}. Apresentar
os beneficios e consequências em relação aos valores pessoais, de modo
não tendêncioso.  
\item Utilizar normas sociais a respeito de comportamentos de consumo
sustentáveis combinando o uso de normas injuntivas e descritivas. As
normas sociais são as regras ou espectativas por um comportamento
apropriedado em uma situação social em particular. Elas ``podem levar
pessoas a dizer coisas que sabem que não são verdade, utilizar drogas
ilicitas ou deixar de reagir a uma ameaça eminente '' \cite[p.51,
tradução própria]{aceee_2010_estudos_feedback}.  Normas descritivas
são percepções de comportamentos normalmente realizados (ex. 85\% da
sua vizinhança reciclam), apelando ao valor de Maslow de
\emph{Amor/Aceitação}. Normas injutivas são percepções e
comportamentos que são normalmente aceitos ou aprovados (ex. um sinal
de polegar positivo com o texto ``Gere menos resíduos''). Essas apelam
para o valor de Rokeach \emph{Obediente}.  Ao juntar ambas normas
descritivas e injutivas, há uma chance ainda maior de sucesso quando
em comparação da aplicação delas isoladamente.
\item Fornecer uma variedade de pequenas ações de consumo que, se realizadas, 
podem ter impactos positivos no meio ambiente. Isso irá trabalhar em duas
barreiras para a motivação, que são: não se sentir competente e não acreditar
que suas ações irão levar a um resultado positivo. Apresentar uma variedade de
ações apela ao valor de Rokeach de \emph{Liberdade} e aumenta o senso de
controle pessoal assim como a motivação intrínseca.
\end{enumerate}
\end{enumerate}
\item \emph{Contemplação}: há conhecimento de seu problema comportamental e
planeja-se uma mudança no futuro. Contempladores são abertos a informação sobre
o problema, ainda que estão distantes do compromisso de mudança devido ao
sentimento de ambivalência;
\begin{enumerate}
\item Pender a balança no sentido de mudança. Ambivalência\footnote{Presença de
sentimentos/pensamentos conflitantes perante uma coisa ou pessoa.} é o problema chave
que precisa ser resolvido, uma vez que a avaliação dos prós e contras tem mais
ou menos o mesmo peso.
\begin{enumerate}
\item Nessa etapa, deve-se fornecer o retorno de informação enfatizando os prós de um
comportamento sustentável e os contras do comportamento não-sustentável. É
importante auxiliar o consumidor em perceber os prós, uma vez que os mesmos irão
resistir às mudanças enquanto perceberem isso como um fator redutor de sua
qualidade de vida, em especial àquelas que salientam o sacrificio pessoal pelo
bem comum. Os contras devem enfatizar nos custos de comportamentos
não-sustentáveis, numa perspectiva de perda em detrimento de ganho. O foco nos
valores pessoais podem ser extremamente persuasivos nesse estágio.
\item Lembrar o individuo de sua atitude em favor do meio-ambiente, informá-lo
da discrepância de suas atitudes e o comportamento correspondente, encorajar a
mudança. Essa técnica apela para a dissonância cognitiva\footnote{Um estado
desconfortável que ocorre quando a pessoa possui uma atitude e um comportamento
que são psicologicamente inconsistentes.}. Como as pessoas mudam de atitudes mais
fácil que de comportamento, é importante encorajar o comportamento sustentável.
\item Fornecer incentivo para pequenas mudanças de consumo (independente da
intenção do consumidor original era a utilização sustentável de energia) para
fomentar maiores mudanças no futuro.
\item Vincular a tecnologia de retorno a um portal de uma comunidade virtual e
incentivar o indivíduo para procurar e ler a informação de experiências de
outros usuários com consumos sustentáveis na comunidade. Isso apela para normas
sociais de um modo vivo e personalizado, explorando a abertura dos
contempladores ao tema, mas, ao mesmo tempo, sem forçar nenhum tipo de ação.
\end{enumerate}
\end{enumerate}
\item \emph{Preparação}: momento em que o indivíduo está pronto para ação no
futuro iminente (medido normalmente como 1 mes), e tem como objetivo desenvolver
e engajar-se a um plano. Pelo menos uma tentativa de mudança foi realizada no
último ano;
\begin{enumerate}
\item Ajudar os usuários a desenvolver um plano que seja aceitável, acessível e
efetivo. Esses planos podem se relacionar a ações extraordinárias (compra de um
refrigerador eficiente) ou diárias (tomar banhos mais curtos). Um objetivo é
definido como uma representação interna de um resultado desejado. Usuários na
fase de preparação podem ter objetivos abstratos mas não saber necessariamente
como os alcançar.
\begin{enumerate}
\item Ajudar na criação de metas pessoais especificas e quantitativas
(o nível de dificuldade deve se elevar conforme o sucesso, começando com metas 
simples e partindo para mais complicadas). Metas difíceis, pessoais e
especificas tem maior engajamento quando comparadas a tarefas ``faça o seu
melhor'', fáceis ou que lhe foram atribuídas.
\item Desenvolver modos múltiplos para os consumidores atingirem suas metas e
incentivá-los para utilizar sua habilidade e experiência pessoal nesses planos.
\item Fornecer no portal da comunidade virtual aos usuários a opção de ter
um ``Conselheiro/Tutor''. Os conselheiros seriam pessoas exemplares na
fase de ação ou manutenção. Essa conexão fornece um maior nível de
engajamento. 
\end{enumerate}
\end{enumerate}
\item \emph{Ação}: ocorre a manifestação de modo evidente da mudança
comportamental, usualmente dentro dos últimos 6 meses;
\begin{enumerate}
\item Reforçar positivamente ações sustentáveis de energia. O reforço positivo é
a técnica mais efetiva para motivar a maior ocorrência de um comportamento
desejado, ela tende a aumentar a motivação intrínseca. Técnicas como a punição ou 
reforço negativo evitam um comportamento não desejado, mas não o substitui por
nada, além de reduzir a motivação intrínseca.
\begin{enumerate}
\item Fornecer o reforço positivo imediatamente após o comportamento desejado
ocorrer e em multiplos modos com o intuíto de aumentar sua eficácia.
\end{enumerate}
\item Desenvolver motivações intrínsecas para o comportamento sustentável.
\begin{enumerate}
\item Permitir uma exploração interativa, personalizável e anotações na
interface oferecida.
\end{enumerate}
\end{enumerate}
\item \emph{Manutenção, Recaída e Reciclagem}: trabalho no sentido de manter o
comportamento alterado e a luta para previnir recaídas. Se aquela ocorrer, o
indivíduo regressa a um dos estágios anteriores e o processo recomeça.
\begin{enumerate}
\item Manter o comportamento sustentável permanente. Em algum momento as
mudanças irão se tornar sustentáveis por si próprias, sendo possível a saida 
do indivíduo do ciclo de mudança. Enquanto isso, o objetivo deve ser fazer o
indivíduo apenas um pouco mais consciente e informado.
\begin{enumerate}
\item Apoiar ações sustentáveis para que elas virem hábitos, relembrando os
usuários para realizar ações específicas.
\item Fornecer a opção para usuários nessa etapa de se tornarem
``Conselheiros/Tutores'' para indivíduos na etapa de preparação. Essa tecnica
aplica dissonância cognitiva, uma vez que indivíduos que tentaram persuadir
alguém irão racionalizar internamente o seu comportamento, e assim estarão
propensos a intensificar seu engajamento. Esse método apela para o valor de
Rokeach \emph{Reconhecimento Pessoal} e \emph{Sabedoria}, e, em retorno, pode
gerar satisfações intrínsecas de competência e satisfação.
\item Encorajar os usuários para reforço e reflexões pessoais em suas
experiências através de um diário. A reflexão de suas atitudes em relação a
energia e percepção de seu progresso podem trazer a tona satisfações intrínsecas
de interesse, competência e satisfação. O reforço pessoal (na forma de orgulho
ou senso de realização) também irá trazer satisfações intrínsecas, no caso de
competência, e ainda levar a percepções de auto-eficácia. Isso é
importante pois, para um indivíduo vivenciar sucesso de longo-termo, eles
precisam de auto-eficácia e atribuições intrínsecas do comportamento.
\item Manter um ciclo de motivação intrínseca de interesse, curiosidade, desafio
ótimo, competência e satisfação. A motivação intrínseca é um ciclo de dois
passos. Primeiro, estimulos como novidade, complexidade e mudança atraem a
atenção, curiosidade e interesse, o que convida para a investigação, exploração
e manipulação dos estímulos. Segundo, desempenhos de competência em tarefas são
desfrutados, enquanto o crecimento da satisfação aumenta tanto o desejo na
atividade quanto a capacidade de confrontar desafios parecidos no futuro. 
\end{enumerate}
\end{enumerate}
\end{itemize}





\subsection{Aspectos Visuais da Informação}
\label{ssec:asp_visuais}

A apresentação na qual a informação é realizada também irá afetar o entendimento
e engajamento do consumidor nos programas de \gls{ee}. Novamente, os especialistas
e profissionais da área, no caso de \emph{design} e da informação,
serão importantes nesse intuíto, frisando que o problema jaz além de
apenas disponibilizar a tecnologia para fornecer a informação: também
será necessário tornar a informação atrativa ao consumidor.

O processo do fluxo de informação ocorre entre o remetente e o receptor, onde
este analisa a informação necessaria e a apresenta na mensagem através de textos
e imagens, enquanto aquele realiza escolhas entre as informações disponíves nela
e opta se irá processá-las mentalmente. A visão é o sentimento mais 
importante para a compreensão e vivência humana do meio externo. 
Cerca de 70\% de nossas células sensoriais estão nos olhos, assim, a visualizão 
é um meio bastante efetivo de realizar a comunicação da informação e de dados. 
Tornar a mensagem interessante visualmente depende de vários elementos 
\cite{2012_visualisation_sweden}. Alguns exemplos são: fornecer para a 
mensagem uma estrutura na qual serão guiados os princípios para
a inserção das representações; clareza para a simplicidade de compreensão e
legibilidade da informação; enfase para atrair, direcionar ou manter a atenção;
unidade da mensagem, com uma coerência e união global \cite{it_depends}.

Em \cite{2012_visualisation_sweden}, apresentou-se a jovens de um colégio sueco
uma visualização do consumo de energia em um mostrador em tempo real na sala de 
aula. Os jovens tiveram participação na criação da interface\footnote{A
participação dos usuários no desenvolvimento facilita à tecnologia atender às
especificações e necessidades dos mesmos. No caso, a participação dos usuários
utilizada está de acordo com a recomendação de
\cite{2009_extreme_user_filandia}, na qual há a descrição
de uma abordagem orientada ao usuário para o retorno do consumo de energia. Nela, o usuário apenas fornece
inspiração aos desenvolvedores profissionais, que desenvolvem as soluções.}, 
onde desenharam e escolheram uma imagem para informar se o consumo estava elevado, 
médio ou baixo, assim como informaram o que entederam das informações nos mostradores.
Esse estudo revelou cinco aspectos importantes no desenvolvimento de sua
visualização nos mostradores:

\begin{enumerate}
\item Deve chamar a atenção dos usuários, realizando o uso de cores brilhantes,
contrastes e quando possível uma exibição dinâmica; 
\item Mostrar comparações entre o consumo de modo a deixar evidente aos usuários
resultados positivos de um esforço;
\item Fornecer o consumo em tempo real para estimular a mudança direta no
comportamento;
\item Deve conter um tom positivo e encorajador, potencializando a positividade
de um comportamento correto;
\item Ser explicativa, com pequenos textos instrutivos que fazem aos usuários
simples de entender o que eles estão vendo.
\end{enumerate}


%O \gls{nialm} pode ser utilizado diretamente na obtenção de 
%uma melhor \gls{ee}, visto que estudos no exterior
%\cite{2010_advanced_metering,liikkanen2009extreme,schleich2012does,darby2006effectiveness}
%indicam uma possível econômia de até 12\% no consumo residencial através do fornecimento de um
%retorno detalhado em tempo real ao usuário de sua utilização da eletricidade, 
%com informação do consumo por equipamento e recomendações no intuíto de reduzir
%o consumo de energia. Esses estudos mostram uma grande variação conforme a
%cultura e perfil social do consumidor, sendo necessário um estudo para
%determinar esse potencial no Brasil, entretanto, percebe-se que o maior
%potencial está nos consumidores pertencentes a classe média, que são os mais aptos a reduzirem seu
%consumo através do retorno de informação em detrimento aos mais pobres, que 
%raramente possuem mudanças que possam ser exploradas uma vez que seu consumo
%é vital e menor, e os mais ricos, que tem pouca sensibilidade aos gastos com
%energia elétrica.



  \chapter[Monitoração Não-Invasiva de Cargas Elétricas]{\acrfull{nilm}}
\label{cap:nilm}

Este capítulo trata as particularidades envolvidas no
desenvolvimento da tecnologia conhecida como \gls{nilm}\footnote{NIALM
e NALM são outras abreviaturas utilizadas na literatura com o mesmo
significado àquela aqui utilizada, Monitoração
Não-Invasiva de Cargas Elétricas, tradução própria do
termo em inglês \emph{Non-Intrusive Appliance Load Monitoring}.}.
Diferente dos capítulos anteriores, que
se dedicaram à contextualizar as aplicações do \acs{nilm} em torno do
tema de \acs{ee}, este capítulo se dedica à descrição do projeto do
\acs{nilm} em termos técnicos e das abordagens utilizadas observadas
durante o levantamento bibliográfico realizado para este trabalho, bem
como o estado do projeto e a metodologia proposta pelo \acs{cepel} que
foi expandida e sistematizada pelo trabalho atual, que será descrita
nos dois próximos capítulos.

Antes, porém, o \acs{nilm} é abordado trazendo seus aspectos gerais na
Seção~\ref{sec:nilm_aspec_gerais}, aonde é realizado uma sinopse da
informação dos capítulos anteriores e citadas outras possíveis
aplicações para o \acs{nilm} --- suas aplicações não se limitam à
\acs{ee}. Em seguida, a Seção~\ref{sec:nilm_mundo} apresenta as
convenções adotadas por este trabalho, estando disponíveis os
conceitos e nomenclaturas que serão utilizados em diante no trabalho.
Essa seção apresenta, em seguida, um extenso levante de técnicas
empregadas por demais autores para realizar a desagregação do consumo
que será resumida e debatida em sua subseção final.  Por sua vez, a
metodologia proposta pelo \acs{cepel} está disponível na
Seção~\ref{sec:nilm_cepel}, aonde também serão tratados os estudos
anteriores realizados pela colaboração \acs{coppe}--\acs{cepel} e
considerações sobre a escolha das características discriminantes, que
utilizará como base os demais estudos de \acs{nilm} abordados por este
trabalho. 

Apesar deste capítulo trazer as técnicas para o \acs{nilm} e tratá-lo
de uma maneira geral, vale ressaltar que a metodologia aplicada pelo
trabalho se restringiu apenas para a detecção de eventos de
transitório. O tratamento do \acs{nilm} como um todo foi realizado
para elucidar e facilitar, da melhor maneira possível, o projeto dessa
técnica, não se limitando a compreensão apenas do ponto atacado pela
metodologia do trabalho, mas permitindo uma visão do projeto de uma
maneira global.


\section{Aspectos Gerais}
\label{sec:nilm_aspec_gerais}

O \gls{nilm} é uma alternativa para fornecer a informação de consumo
de energia elétrica desagregada por equipamento ou dispositivo. Ao
invés das técnicas normalmente utilizadas --- onde se lança mão de
sensores dispostos em cada equipamento, esses enviando informações
para uma central encarregada de decodificá-las e, assim, identificar
os equipamentos que estão demandando consumo na rede ---, o \gls{nilm}
é um método em que não ocorre a intrusão na propriedade do usuário (ou
intrusão em escala mínima), contendo geralmente apenas um medidor
central no fornecimento de energia dessa propriedade. O peso da
identificação dos equipamentos é transferido da utilização de uma
grande quantidade de sensores e \emph{hardware} complexo para um
\emph{software} e algoritmos de processamento de sinais no intuito de
realizar uma análise profunda das medições e desagregar as
informações.

Ainda que o peso da identificação esteja no \emph{software}, a
aptidão dos algoritmos implementados dependem da capacidade do
medidor de extrair informações da rede e das distorções causadas pelos
equipamentos na mesma, de forma que quanto mais avançado for o
\emph{hardware} de medição, com capacidade de enviar uma quantidade e/ou frequência
maior de informação aos algoritmos encarregados de realizar a
discriminação dos rastros deixados na rede elétrica pelos equipamentos,
também maior será a capacidade dos mesmos de desagregar o consumo
específico dos equipamentos.

Possíveis aplicações para esse dispositivo são: 

\begin{itemize}
\item auxiliar ou substituir as \glspl{pph} sem que seja necessário
causar incômodo ao consumidor devido a presença de medidores na
residencia, fornecendo assim dados com maior fidelidade em carácter
desagregado por equipamento e maior frequência para estudos de \gls{ee}
no consumo de eletricidade (Seção~\ref{sec:ee_dificuldades}), em
especial para o setor residencial;
\item disponibilizar a informação desagregada para o fornecimento do
Retorno Indireto Diário/Semanal e/ou Retorno em Tempo Real Desagregado
em programa de \gls{ee}. Os programas de \gls{ee} utilizando retorno
de informação são possíveis fontes de redução de consumo nos grandes
centros urbanos. No entanto, são necessários estudos, no Brasil, para
determinar seu potencial (Seção~\ref{sec:ee_res_exp}). O retorno de
informação desagregada por equipamento pode ser utilizado por empresas
que oferecem serviços de redução de consumo
(Subseção~\ref{ssec:ret_tec}), ou mesmo por consumidores que precisam
ter algum tipo de controle sobre seu consumo de energia; \item no
interesse das concessionárias, a informação desagregada pode auxiliar
os clientes a identificarem consumo não essencial durante o horário de
ponta, auxiliando no deslocamento de carga (ver item \emph{Resposta de
Ponta e Demanda versus Economia Fora de Ponta} na
Subseção~\ref{ssec:ret_outros}), assim como identificar clientes com
maior potencial de redução de consumo durante esses períodos, para
oferecer incentivos nesse sentido (Subseção~\ref{ssec:ret_tec}); 
\item identificar equipamentos defeituosos ou com consumo excessivo de
energia;
\item informar aos fabricantes de eletrodomésticos o
perfil de seus consumidores (caso seja possível obter essa
informação);
\item permitir a identificação de perdas não-técnicas através de
discrepâncias entre a informação de consumo agregado obtido pelo
\acs{nilm} e aquele obtido pela concessionária, o que indicaria
incoerências na tarifação;
\end{itemize}

As possibilidades de aplicação como um meio de melhoria da \gls{ee} e
redução da intensidade elétrica tem elevado o interesse nesse assunto,
em especial nos países desenvolvidos, como uma forma de aliviar
a pressão de consumo nos grandes centros urbanos e na redução de
emissões de \gls{co2} \cite{nilm_zeifman_review_2011}.
Isso tem levado a gigantes no setor eletrônico, como \emph{Intel} e
\emph{Belkin}, a investirem fortemente no desenvolvimento dessa
tecnologia. O crescente interesse na evolução por parte da academia
levou a organização do primeiro \emph{workshop} especificamente para o
tema em 2012 \cite{workshop_nilm}. Essa alternativa é mais simples
quando comparando com a automação residencial por não requerer
mudanças na produção dos eletrodomésticos --- a automação requer
comunicação nos dois sentidos (entre a interface e o equipamento), tal
como a capacidade de controle do equipamento, de forma que se faz
necessário adaptar equipamentos antigos e a produção dos novos
equipamentos com essas capacidades --- juntamente com o fato de haver
um relutância social quanto às residências automatizadas, apesar de
esforços governamentais e da mídia local \cite{Lipoff_Automation_2010}
(a referência estudou a falta de interesse na automação residencial
nos \gls{eua}). Não bastasse, também se pode citar o nascimento de
\emph{start-ups} procurando espaço na corrida por esse novo mercado,
como \emph{GetEmme} \cite{getemme_site} e \emph{Navetas}
\cite{navetas_site}.

Porém, é importante aqueles envolvidos no projeto terem em mente a
questão da ética: a informação do consumo não deve sair da residência
sem o consentimento do consumidor. Além disso, o projeto deve ser
realizado em um sistema autocondido, ou seja, deve ser garantido ao
consumidor a segurança da informação e sua privacidade, de forma que
esses itens também devem ser incluídos para o projeto e proliferação
da técnica.

\section{As Diversas Metodologias Utilizadas no Mundo}
\label{sec:nilm_mundo}

A ideia de desagregar a informação não é nova, sendo possível
encontrar pesquisas nesse sentido datando da década de 1980. Apesar
disso, um extenso levantamento bibliográfico
\cite{nilm_hart_1992_8,nilm_bouloutas_viterbi_ext_1991_11,
nilm_hart_fsm_viterbi_1993_12,nilm_norford_leeb_medianfilt_1996_13,
nilm_leeb_spectral_envelope_1995_23,
nilm_cole_data_extraction_1998_14,nilm_cole_extra_info_surge_1998_15,
nilm_powers_15minsamp_1991_16,nilm_farinaccio_16ssamp_1999_17,
nilm_marceau_16ssamp_improved_1999_18,nilm_baranski_genetic_base_2003_19,
nilm_baranski_genetic_detalhado_2004_20,nilm_baranski_summary_2004_21,
nilm_matthews_overview_2008_22,nilm_laughman_continuous_variables_2003_9,
nilm_lee_variable_speed_estimation_2005_24,
nilm_wichakool_2009_25,nilm_shaw_2008_26,nilm_srinivasan_nn_2006_27,
nilm_patel_2007_29,nilm_gupta_patel_2010_30,
nilm_sultanem_1991_10,nilm_chan_2000_31,nilm_lee_2004_32,nilm_lam_2007_33,
nilm_liang_pt1_2010_34,nilm_suzuki_2011_35,nilm_berges_2008_7,
nilm_berges_2009_36,2010_nilm_melhorando_pph_usa_37,
nilm_liang_pt2_2010_40} realizado em \citet*{nilm_zeifman_review_2011}
expõe que as técnicas aplicadas em \glspl{nilm} ou não são robustas no
sentido de atenderem especificamente a um grupo limitado de equipamentos
estudados, ou apresentam acurácia marginal, mostrando que o processo
de desagregação da informação não é trivial. Demais referências que não constam em
\cite{nilm_zeifman_review_2011} citadas nesta seção contendo
abordagens de \glspl{nilm} são
\cite{nilm_apresentacao_review_2011,
nilm_bergman_distribuido_2011,nilm_zeifman_vast_2011,
nilm_zeifman_vast_hisample_pdfmerge_2011,
nilm_zeifman_vastext_approach_2012,
nilm_zeifman_statistical_vastext_1stws_2012,
nilm_zeifman_statistical_naive_enduses_2013,
nilm_genetic_2013,nilm_patel_review_2011,
nilm_coppe_nascimento,nilm_itajuba_rodrigues,
wavelet_transients_2009,
seminilm_fhmm_empiricalnmeter_2013,
seminilm_ihome_tomek_2012,
seminilm_berges_multisensor_2010} (ver
Subseção~\ref{ssec:nilm_tecnicas}).


\begin{figure}[h!t]
\centering
\includegraphics[width=\textwidth]{imagens/projeto_do_nilm.pdf}
\caption{Esboço dos passos envolvidos no projeto do \acs{nilm} e da
sua operação.}
\label{fig:esboco_do_projeto_nilm}
\end{figure}

Na Figura~\ref{fig:esboco_do_projeto_nilm} há um esboço dos passos
envolvidos no projeto do \acs{nilm} e de sua operação, baseando-se nas
diversas técnicas observadas nas referências citadas, apenas quando
levando em conta o processo para a obtenção da informação desagregada
do consumo. Vale ressaltar que a informação do consumo desagregado
ainda precisa ser tratada levando em conta os aspectos psicológicos
(Subseção~\ref{ssec:asp_psic}) e visuais
(Subseção~\ref{ssec:asp_visuais}) para motivar os consumidores a
reduzirem seu consumo quando aplicando o \acs{nilm} como um meio de
economia de energia e intensificação da \gls{ee}. Outro ponto do
projeto para o projeto do \acs{nilm} é a questão da segurança da
informação e privacidade do consumidor, sendo necessário garantir que
a mesma não deixará a residencia sem seu consentimento. Dito isso, o
projeto começa com a questão de como modelar os equipamentos e
dispositivos presentes nas redes elétricas, sendo divididos de acordo
com o comportamento de suas cargas, descritos na
Subseção~\ref{ssec:modelos_carga}. Em seguida,
Subseção~\ref{ssec:metodologia_generica}, será tratado a escolha
do sistema de aquisição de dados e da metodologia a ser adotada. Em um
projeto sem limitações, a metodologia escolhida implica na escolha do
sistema de aquisição de dados a ser utilizado, porém, podem haver
limitações no mesmo, nesse caso a metodologia tem de ser ajustada
conforme as capacidades dele em fornecer informação. Com o crescente
interesse pelo desenvolvimento do \acs{nilm}, os autores perceberam a
dificuldade de comparar as diferentes técnicas já presentes devido ao
emprego de diferentes medidas para o cálculo de eficiência ou
utilização de conjuntos com características diferentes entre si, de
forma que o assunto merece atenção e será tratado na
Subseção~\ref{ssec:nilm_eff_calc}. Em seguida, a
Subseção~\ref{ssec:nilm_tecnicas} irá apresentar as diversas
abordagens utilizadas, sendo guiada na proposta de divisão das
técnicas e características feita por \cite{nilm_zeifman_review_2011}.
Essa subseção contém levantamento bastante técnico, sendo de
interesse somente daqueles que desejam se aprofundar no tema.  A
discussão das informações levantas é realizada em
\ref{ssec:nilm_discussao}, sendo sua leitura suficiente para
aqueles que desejam as informações mais relevantes.

\subsection{Modelos de Carga}
\label{ssec:modelos_carga}

Os equipamentos podem ser modelados devido às características de
comportamento de suas cargas. A seguir, encontram-se possíveis
características de carga elétrica dos eletrodomésticos. Os quatro
primeiros itens são modelos de cargas mutuamente exclusivos, enquanto
os dois seguintes podem ser incluídos, dependendo das propriedades dos
equipamentos ligados à rede, para caracterizar equipamentos potencialmente
dificultadores do processo de desagregação (baseado em \cite{
nilm_hart_1992_8,nilm_baranski_genetic_base_2003_19,
nilm_zeifman_review_2011,nilm_zeifman_vast_hisample_pdfmerge_2011,
nilm_apresentacao_review_2011,nilm_liang_pt2_2010_40,
nilm_liang_pt1_2010_34}):

\begin{itemize}
\item \textbf{\Gls{c1}}: equipamentos que permanecem
ligados 24~h/dia, 7~dias/semana, com consumo de energia praticamente
constante. Ex.: detectores de fumaça, fontes de alimentação
constantemente ligadas à rede;
\item \textbf{\gls{c2}}\footnote{Nas referências, não há separação na
categoria \gls{c2}, que é utilizada para descrever exclusivamente
equipamentos modelados como \acs{c2a}. Nelas, a categoria
\acs{c2b} é incluída em \acs{c4}, uma vez que em ambos os
casos as \glspl{fsm} podem alterar o seu estado de operação para
qualquer outro estado independente tanto de qual estado anterior ele estava
operando como o tempo de operação nesse estado. A separação da
\gls{fsm} com estados discretos em duas categorias parece mais
familiar do que incluir \gls{fsm} com patamares discretos e
aleatórios em uma categoria que permite patamares
contínuos.\label{fn:subdivisao}}: essa categoria é utilizada
para identificar equipamentos que contém um conjunto definido de
patamares discretos de consumo. Pode-se dividir esses equipamentos 
em dois conjuntos:
\begin{itemize}
\item \textbf{\gls{c2a}}\fnref{fn:subdivisao}: os estados do equipamento repetem-se em
padrões definidos temporalmente, garantindo que os seus ciclos serão
observados frequentemente durante intervalos diários ou semanais. Ex.:
máquina de lavar roupas, máquinas de lavar louças;
\item \textbf{\gls{c2b}}\fnref{fn:subdivisao}: nesses equipamentos não há
um padrão para os seus ciclos de operação. A operação por uma fonte
externa, como o consumidor, altera o seu padrão de consumo sem ser
possível encontrar uma regra operativa para o ciclo através da busca
de repetições de suas trocas de estado na rede, os estados mudam
aleatoriamente depois de quantidades de tempo também aleatórios. Ex.:
ventilador de múltiplas velocidades, liquidificador --- ambos
dependendo de como operados pelo consumidor: se apenas ligados e
desligados, irão comportar-se como \acs{c3}s, enquanto se
operados de modo padronizado, irão se comportar como \acs{c2a}s;
\end{itemize}
\item \textbf{\gls{c3}}: um caso particular das \glspl{c2} ocorre
quando o equipamento pode ser modelado como tendo apenas dois estados:
ligado/desligado. Ex.: lâmpadas, bombas de água;
\item \textbf{\Gls{c4}}: uma generalização das \glspl{c2}, onde
há uma infinidade de estados para os quais o equipamento pode operar.
Essa categoria pressupõe que o equipamento irá estabilizar o seu consumo
em um patamar após um período de tempo. Sua operação pode ser dividida
em dois grupos: operação manual do operador, equipamentos
autocontrolados. Estes são de mais simples detecção quando comparados
com aqueles, uma vez que seus ciclos de mudanças de estado são
distribuídos uniformemente no tempo. Ainda assim, essa categoria é o
maior desafio para as técnicas empregadas nos \glspl{nilm}, sendo
raramente tratada por elas. Ex.: lâmpadas com \emph{dimmer},
ferramentas elétricas (furadeiras, serras etc.), bomba de aquário.
\item \textbf{\Gls{c5}}\footnote{As referências optaram por não
criar essas categorias uma vez que essas são apenas
características das cargas. Por sua vez, as mesmas são citadas, no
mínimo, em \cite{nilm_zeifman_review_2011,nilm_liang_pt2_2010_40}
como dificultadores no processo de desagregação e, por esse motivo,
preferiu-se adicionar diretamente essas categorias para
enfatizar e facilitar a identificação de cargas com essas
características.\label{fn:categoria_add}}: equipamentos que causam
distúrbios na rede continuamente devido à dinâmica durante sua
operação, se referindo a possíveis oscilações inerentes à 
característica operativa do mesmo. Alguns exemplos observados pela
equipe do \gls{cepel}: televisores, onde a variabilidade de brilho,
cores e som, alteram seu consumo (também observado em
\cite{nilm_zeifman_statistical_naive_enduses_2013}); computadores, que
alteram sua potência conforme a demanda dos processadores e
\emph{coolers}, consumindo mais quando o usuário está realizando
tarefas como, por exemplo, executando algoritmos do mestrado,
escutando música etc.; alguns outros equipamentos como o ar condicionado
--- esse quando com o compressor ligado --- apresentam uma dinâmica
com oscilações em frequências sub-harmônicas, os mesmos são potenciais
dificultadores à identificação de rastros deixados por outros
equipamentos, particularmente os de menor consumo, caso estudado em
\cite{nilm_liang_pt2_2010_40}. A Figura~\ref{fig:ar_cond_dinamica}
demonstra a dinâmica sub-harmônica causada na envoltória para esse
equipamento. Outros exemplos de equipamentos com motores que também geram
oscilações --- mas em ordem inferior ao ar condicionado --- são:
microondas, geladeira, desumidificador
\cite{nilm_liang_pt2_2010_40};
\begin{figure}[h!t]
\centering
\includegraphics[width=\textwidth]
{imagens/ArCondicionado-CargaDemandaDinamica_ComTextoImpr.pdf}
\caption[\acf{c5}: ar condicionado]
{Exemplo de \acf{c5}: ar condicionado. Também é possível
observar uma alteração gradual no nível de consumo, que irá dificultar
o processo de reconstrução de energia.}
\label{fig:ar_cond_dinamica}
\end{figure}
\item \textbf{\Gls{c6}}\fnref{fn:categoria_add}: apesar de não ser
uma característica de um equipamento \emph{per se} e nem constituir um
modelo de carga elétrica, estudos de performance de \glspl{nilm} podem
considerar quais equipamentos serão potencialmente vistos como se
fossem um mesmo equipamento por possuírem os mesmos padrões. Essa categoria
varia conforme os equipamentos presentes e quais são as
características sendo extraídas. Por exemplo, um computador e uma
lâmpada incandescente possuem consumos semelhantes quando procurando
padrões no plano \acs{dp}$\times$\acs{dq}
\cite{nilm_laughman_continuous_variables_2003_9}, enquanto
equipamentos com motores de potências distintas podem não ser
desagregados quando apenas olhando para seus transitórios --- em
especial quando normalizados, caso que possivelmente ocorreria se
utilizando \acrfull{rna}. Um exemplo de ocorrência de \acs{c6} pode
ser observado na Figura~\ref{fig:exemplo_c6} para a característica
\acs{dp}. \cite{nilm_liang_pt1_2010_34} propõe a
equação \ref{eq:similaridade}\footnote{Nota-se, aqui, que qualquer 
outra divergência representativa pode ser empregada com esse 
intuito. Divergências são generalizações das métricas e medem a
quase-distância entre duas medidas não-negativas
\cite[cap. 2]{cichocki2009nonnegative}. No caso de algumas
divergências, como a métrica Euclidiana, é possível aplicar valores
negativos.} como uma maneira de medir a similaridade entre dois
equipamentos, onde $s^j_{ia,ib}$ é a relação entre a j-ésima
característica entre os equipamentos $ia$ e $ib$, $N$ é o tamanho da
dimensão dessa característica e $y_{k|(ia,j)}$ é o ponto k para o
equipamento $ia$ e j-ésima característica. Quanto mais próximo for
$s^j_{ia,ib}$ da unidade, mais dificil será desagregar $ia$ e $ib$
utilizando a característica $s^j$. Uma última consideração, a presença
de similaridade entre dois equipamentos não necessariamente é um
problema. Deseja-se para certos equipamentos, como por exemplo, lâmpadas
do mesmo tipo, ou equipamentos de marcas diferentes, que eles sejam
vistos como um mesmo equipamento pelo \gls{nilm}. Há certos casos em
que as características são tão discriminantes que dificultam a
aplicação do \gls{nilm} em larga escala (ver metodologia na
pp.~\pageref{nilm:emi}), sendo necessário encontrar um equilíbrio entre
a capacidade discriminante das características. No entanto, utiliza-se
o termo \glspl{c6} exclusivamente para identificar os casos em que a
similaridade dos equipamentos não é desejada causando erros de
classificação.
\end{itemize}

\begin{figure}[h!t]
\centering
\includegraphics[width=.7\textwidth]{imagens/exemplo_c6.pdf}
\caption[Exemplo de \acrfull{c6}]
{Exemplo de \acrfull{c6}. As cores amarelo e vermelho constituem
respectivamente acionamentos e desacionamentos de uma geladeira; rosa
e azul análogo para um ventilador; e verde e preto análogo para
lâmpadas fluorescentes no plano \acs{dp}$\times$\acs{dq}. Caso apenas 
observando \acs{dp}, percebe-se que os equipamentos possuem uma região
de confusão o que prejudicará o processo de discriminação se apenas
empregando essa característica.}
\label{fig:exemplo_c6}
\end{figure}

\begin{equation}\label{eq:similaridade}
s^j_{ia,ib} = \dfrac{
\sum^N_{k=1}y^2_{k|(ia,j)}}{
\left[\sum^N_{k=1}y^2_{k|(ia,j)} + \left(
\sum^N_{k=1}y^2_{k|(ia,j)} - \sum^N_{k=1}y^2_{k|(ib,j)}
\right)^2
\right]
}
\end{equation}

\subsection{Escolha da Metodologia}
\label{ssec:metodologia_generica}

A escolha da metodologia do \acs{nilm} pode estar limitada à
capacidade do sistema de aquisição de dados de extrair informação do
consumo da rede elétrica, caso que ocorre quando desejando aplicar o
\acs{nilm} a uma estrutura já existente, ou livre, aonde o custo do
sistema de aquisição de dados também torna a escolha de uma
metodologia com o menor custo e capaz de atender as necessidades do
projeto como a ideal a ser aplicada. Além disso, em ambos os casos há
uma limitação da capacidade de processamento dos dados em tempo real,
já que a informação desagregada deve ser informada com pequena
defasagem após a ocorrência do consumo. Como no esboço apresentado na
Figura~\ref{fig:esboco_do_projeto_nilm}, normalmente realiza-se a
transformação da informação em características que apresentam uma
melhor representação para a realização da discriminação e desagregação
do consumo. Há ainda a questão de quando extrair a informação, se em
regime permanente ou apenas quando ocorre a transição do estado
operativo de um equipamento. É necessário optar pelas técnicas para
realizar a discriminação ou se serão empregadas múltiplas técnicas e a
estratégia para detecção de transitórios quando extraindo informação
nesses momentos. As técnicas precisam ser ajustadas para a operação,
seja o treinamento de técnicas supervisionadas, a obtenção de dados
para o emprego de discriminadores estatísticos ou quais informações
devem ser trabalhadas para a otimização, assim cabe a questão de
quando e com que informação realizar o ajuste. Essas questões são
esquematizadas de maneira genérica a seguir, levando em conta as
diversas metodologias observadas e que serão detalhadas
na Subseção~\ref{ssec:nilm_tecnicas} e debatidas em seguida na
Subseção~\ref{ssec:nilm_discussao}.

\subsubsection{A procura por padrões de operação dos
equipamentos}
\label{top:procura_padroes}

Quando extrair informação para a procura por padrões de operação dos
equipamentos?

\begin{itemize}
\item \textbf{Em regime transitório}: identifica o novo consumo dos
equipamentos apenas quando ocorre alteração do estado de um ou mais
deles. Para isso, se faz necessário a detecção de eventos de
transitório aonde são diferenciados os distúrbios na rede devido a
ruído daqueles causados por transições no consumo dos equipamentos.
Isso possibilita o tratamento de menos informação pelas técnicas de
discriminação e a independência da informação tratada em relação à
presença de outros equipamentos (exceto quando há a presença da
operação de \glspl{c5}, que irão interferir na informação obtida); 
\item \textbf{Em regime permanente}: identifica o consumo durante a
operação contínua dos equipamentos em um mesmo estado de operação.
Isso traz como necessidade altas frequências de amostragem (superiores
a frequência da rede), porém permite a identificação de \acs{c1}.
Da maneira que essa informação foi extraída pelos autores nas
referências observadas, a exploração dessa informação é limitada para
a presença de poucos equipamentos, uma vez que há um crescimento
exponencial da complexidade problema de acordo com o número de
estados operativos dos equipamentos presentes na rede.
\end{itemize}

\subsubsection{Definição das representações da informação}
\label{top:caracteristicas}

Qual representação da informação vai ser utilizada como característica
para as técnicas empregadas para discriminação do consumo dos
equipamentos?

\begin{itemize}
\item Macroscópicas - informação simplificada dos ciclos da rede, ex.:
\begin{itemize}
\item Corrente eficaz/pico;
\item Tensão eficaz/pico;
\item Potência Ativa, Reativa, Harmônica, Aparente;
\item \gls{thd}.
\end{itemize}
\item Intermediária - informação contida nas envoltória das ondas
(corrente, tensão, potência etc.) --- geralmente relevante apenas para
acionamentos de equipamentos ou eventos de transitório com acréscimo
de consumo;
\item Microscópicas - informação detalhada de um ciclo da rede, ex.:
\begin{itemize}
\item Forma de onda (sem tratamento);
\item Decomposição harmônica resultante da Transformada de Fourier;
\item Informação nos níveis de detalhes da Transformada Wavelet;
\item Autovetores mais relevantes da \gls{svd} das formas de onda (de
corrente, por exemplo);
\item Curvas I-V.
\end{itemize}
\item Outros - demais informações que podem ser exploradas não
diretamente extraídas do consumo de energia dos equipamentos na rede,
ex.:
\begin{itemize} 
\item Informação de estatística de uso - tempo e horários que os
equipamentos costumam operar em cada um de seus possíveis estados
operativos;
\item \gls{emi} - assinaturas deixadas pelos equipamentos devido ao
chaveamento, seja por sua operação (ex.: fontes chaveadas, escova de
motores) ou durante seu acionamento e desacionamento (ex.: conectar o
equipamento na tomada, alteração do estado de um interruptor);
\item Informação do ambiente - correlacionar informação proveniente no
ambiente (ex.: movimento, som) com o uso dos equipamentos. Necessário a
utilização de outros sensores, constituindo uma abordagem
semi-invasiva (ver pp.~\pageref{top:seminilm}).
\end{itemize}
\end{itemize}

\subsubsection[Abordagens]{Abordagens (baseado em 
\cite{nilm_liang_pt1_2010_34,nilm_zeifman_review_2011})}
\label{top:abordagens}

Quais técnicas empregar para a discriminação das características?

\begin{enumerate}[label={Abordagem} \arabic* - ,ref=\arabic*,align=left]
\item\label{itm:abordagem1}\textbf{Abordagem por reconhecimento de
padrões}\footnote{A referência \cite{nilm_zeifman_review_2011}
trabalha a diferença das abordagens em termo da quantidade de dados
abordados, onde a Abordagem~\ref{itm:abordagem1} trata a informação
uma a uma, ou seja, para cada evento de transitório, enquanto a
Abordagem~\ref{itm:abordagem2} opera com todo o período sendo
otimizado, o que não é necessariamente verdade. Geralmente ambos os casos passam
por um período de otimização antes de serem empregados e, depois de
otimizados, são utilizados para a detecção dos padrões dos equipamentos
na rede. A distinção está que o primeiro otimiza a capacidade de
discernir os padrões --- sendo a reconstrução consequência disso ---,
enquanto o segundo a capacidade de reconstruir com maior fidelidade
possível o sinal original --- obtendo os padrões como
resultado.\label{fn:diff_abordagens}}: as técnicas de reconhecimento
de padrões podem ser tanto técnicas de aprendizado de máquina quanto
técnicas estatísticas ajustadas (ou treinadas) em conjuntos de dados similares
aos quais eles irão operar. O reconhecimento de padrões pode ocorrer
apenas para os eventos de transitórios detectados pela
Etapa~\ref{itm:etapa2} ou para cada ciclo da rede, dependendo de qual
informação está sendo tratado para a
identificação de padrões. Algumas técnicas utilizadas nessa abordagem
podem ser robustas aos equipamentos desconhecidos, sendo capaz de
destacar seu padrão dos outros já conhecidos e adicioná-lo ao catálogo
de padrões. Assim, quando esses padrões ocorrerem novamente, eles
serão identificados como o mesmo equipamento --- chamado de aprendizado
em tempo real. Em \cite{nilm_matthews_overview_2008_22} é observado a
importância dessa estratégia para tornar possível o crescimento do
catálogo, que, tendo o novo equipamento nomeado pelo consumidor, torna
possível a criação de um catálogo universal de equipamentos. Essa
tarefa é impraticável, se não impossível, de ser realizada em
laboratório. Exemplos de técnicas:
\begin{itemize}
\item Técnicas de Aprendizado de Máquina:
\begin{itemize}
\item Redes Neurais (\gls{mlp} e \gls{rbf});
\item \gls{svm}.
\end{itemize}
\item Mapeamento em grupos:
\begin{itemize}
\item \gls{som};
\item \gls{isodata};
\item \gls{art};
\item \emph{K-means}.
\end{itemize}
\item Discriminadores estatísticos:
\begin{itemize}
\item Vizinho próximo;
\item \emph{Naïve Bayes}.
\end{itemize}
\end{itemize}
\item\label{itm:abordagem2}\textbf{Abordagem
por otimização}\fnref{fn:diff_abordagens}: concentra a capacidade de
suas técnicas na otimização, onde é realizada a procura por uma
combinação de equipamentos cujo o sinal agregado resultante é a melhor
aproximação do possível do sinal observado. Em alguns casos,
utiliza-se a concentração dos dados em longos períodos de tempo para
identificar o consumo desagregado, retornando a operação dos diversos
equipamentos no final do processo. Nesses casos, a informação final
pode ser utilizada como padrões a serem identificados posteriormente.
Para manter os equipamentos atualizados, novos períodos (possivelmente
menores ao período inicial) podem ser utilizados para garantir a
resposta adequada a possíveis alterações na presença ou utilização de
equipamentos. Na outra possibilidade, a otimização é realizada no nível
de ciclo da rede, onde, sabendo o padrão dos possíveis equipamentos
presentes, busca-se a melhor combinação operativa que reflitam o sinal
observado. Exemplos de técnicas:
\begin{itemize}
\item Programação por Inteiros;
\item Algoritmo Genético;
\item Força Bruta.
\end{itemize}
\item\label{itm:abordagem3}\textbf{Abordagem empregando múltiplas
técnicas}: ao perseguir melhor eficiência, é possível aplicar
múltiplas técnicas, tanto de otimização quanto de reconhecimento de
padrões, que irão em paralelo para complementarem as deficiências
entre si. Por outro lado, isso torna necessário mais uma etapa no
processo de discriminação, aonde será unida a informação das diversas
técnicas aplicadas no \acs{nilm} (o assunto é tratado em mais detalhes
na pp.~\pageref{nilm:multiplas_tecnicas}).
\end{enumerate}

\subsubsection{Ajuste das técnicas}
\label{top:ajuste_das_tecnicas}

\begin{itemize}
\item \textbf{Pré-ajuste}: os padrões são obtidos via laboratório ou
em testes realizados anteriormente. As técnicas podem vir
pré-ajustadas antes de serem aplicadas na residência, estando prontas
para operação assim que o \acs{nilm} for instalado. Porém, isso pode
ter algumas limitações uma vez que o pré-ajuste dificilmente irá
representar a realidade exata da rede elétrica residencial. Mesmo que
o catálogo seja extenso, é necessário ajustá-lo posteriormente para
evitar a presença de \acs{c6}, limitando o catálogo apenas aos
equipamentos existentes na residencia;
\item \textbf{Ajuste na residência}: obtenção dos padrões através de
submedição ou operação individual dos equipamentos (realizado pelo
próprio consumidor ou com auxilio). Esse seria o ideal, se não fosse
uma tarefa árdua. A exigência da participação do consumidor pode
reduzir o interesse dos mesmos no produto, bem como eles podem não
conseguir realizar o ajuste de maneira adequada do NILM.
Mesmo que o processo seja realizado por técnicos (o que tornaria o
produto mais caro), é interessante evitar a intrusão
da residência em longos períodos de tempo para tornar o \acs{nilm}
em um produto mais atraente;
\item \textbf{Auto-ajuste}: a técnica aplicada é capaz de descobrir os
padrões a serem encontrados. Esforços foram feitos para a
identificação e modelagem automática dos equipamentos de maneira cega
durante a operação. Geralmente o \acs{nilm} passa por uma fase maior de
coleta de dados (ex. 1 semana), aonde é realizado a identificação dos
modelos a serem encontrados, para então operar e identificar esses
padrões. Porém, apenas equipamentos que tem padrões recorrentes podem
ser encontrados (são utilizadas adaptações do algoritmo de
\emph{Viterbi} que encontra a sequência mais provável para as
alterações de estado). Além disso, esse ajuste pode ser utilizado em
conjunto com o pré-ajuste, para permitir a operação do \acs{nilm}
assim que instalado na residencia, para então o mesmo se autoajustar
conforme a coleta de dados for sendo realizada.
\end{itemize}

\subsubsection[Etapas durante operação]{Etapas durante operação 
(baseado em \cite{nilm_matthews_overview_2008_22})}
\label{top:etapas}

\begin{enumerate}[label={Etapa} \arabic* - ,ref=\arabic*,align=left]
\item\label{itm:etapa1} \textbf{\gls{fex}}: são extraídas
informações das amostragens realizadas. A diversidade de
características que podem ser extraídas depende da capacidade do
sistema de aquisição de dados, em especial no medidor, mas podendo
sofrer devido a outras limitações (como capacidade de transferência de
dados etc.). As características são utilizadas nas etapas seguintes, podendo
haver reaproveitamento. Em alguns casos, a Etapa~\ref{itm:etapa2}
utiliza o sinal sem processamento e a \gls{fex} é realizada somente
para a Etapa~\ref{itm:etapa3}, o que permite reduzir o esforço de
processamento;
\item\label{itm:etapa2}\textbf{Detecção de eventos de transitório}:
identificar alterações causadas por equipamentos na rede. Essa etapa é
necessária para extrair a informação em regime transitório.
Pode-se empregar limiares estáticos ou dinâmicos para a detecção dos
eventos. Os limiares dinâmicos permitem o ajuste de operação,
reduzindo ou aumentando a sensibilidade do detector conforme a
presença de equipamentos \acs{c5};
\item\label{itm:etapa3}\textbf{Discriminação da informação}: utilizar as
características pertinentes para o reconhecimento de padrões,
deduzindo, assim, qual foi o equipamento que causou o distúrbio na rede
e qual seu novo estado de consumo. A discriminação da informação pode ser
realizada por técnicas de aprendizado de máquinas, estatística ou via
otimização. Possivelmente, múltiplas técnicas podem ser empregadas
nesta etapa a fim de maximizar a eficiência. É desejável que a técnica
seja capaz de identificar a ocorrência de novos padrões e
reconhecê-los em suas próximas aparições --- ajuste automático das
técnicas --- pois a construção de um catálogo com todos os possíveis
eletrodomésticos é impraticável, se não impossível. Tal tarefa só será
possível com a capacidade dos \glspl{nilm} de incluírem novos
equipamentos ao catálogo;
\item\label{itm:etapa4}\textbf{União da informação e refinamento dos
resultados}: quando empregando múltiplas técnicas, faz-se necessário a
união da informação discriminante fornecida pelas técnicas na
Etapa~\ref{itm:etapa3}. Além disso, durante a união da informação,
pode-se procurar por possíveis erros ou melhorias que possam ser
feitas na informação desagregada. Por exemplo, ao correlacionar a nova
informação obtida com a passada, pode ser possível identificar
um equipamento que remanesce consumindo energia da rede por
tempo demasiado, enquanto sua operação normalmente ocorre em
intervalos curtos, o que possivelmente foi causado por falhas na
Etapa~\ref{itm:etapa2}, onde o desacionamento do equipamento não foi
encontrado, ou na Etapa~\ref{itm:etapa3}, na qual o desacionamento foi
identificado como causado por outro equipamento. 
As estratégias corretivas podem ser meramente remediativas, ou seja,
simplesmente remover incoerências sem realizar um tratamento
procurando sua causa, ou aplicar técnicas complementares
para reanalisar a informação disponível.
\end{enumerate}

\subsection{Cálculo da Eficiência}
\label{ssec:nilm_eff_calc}

O assunto aqui tratado é de grande importância para permitir uma
comparação justa entre os \glspl{nilm} de diferentes autores e, por
isso, levado em consideração nesta subseção.
Porém, este trabalho não teve a oportunidade de empregar as medidas de
eficiência debatidas nesta subseção pois trabalhou a detecção de
eventos de transitório --- foram utilizados a taxa de detecção e falso
alarme dos eventos nos resultados divulgados no
Capítulo~\ref{chap:resultados} ---, não estando disponíveis, portanto,
a capacidade discriminante e a estimativa do consumo para empregar as
medidas aqui debatidas.

\subsubsection{Padronização}
\label{top:nilm_padrao}

O estudo bibliográfico realizado por \cite{nilm_zeifman_review_2011}
teve dificuldades ao tentar comparar as diferentes técnicas aplicadas
nos \glspl{nilm}. O primeiro empecilho está na variedade das base
de dados utilizadas, possuindo equipamentos e estados de operações
bastante distintos, criando condições que podem privilegiar a
eficiência de um determinado \gls{nilm}. Para a unificação dos dados
estudados e permitir a comparação de performance entre os algoritmos
empregados nos \glspl{nilm}, foi disponibilizado por
\cite{nilm_dataset_blued_2012} um conjunto de dados públicos para
a análise, o \acs{blued}. Um outro conjunto de dados público
utilizado nas referências é o \acs{redd}
\cite{nilm_dataset_redd_2011}, onde há dados gravados tanto em alta
frequência (15~k\acs{hz}) e baixa frequência (0,5~\acs{hz} e
1~\acs{hz}). Os conjuntos de dados foram construídos para
representar a realidade de residências nos \gls{eua} e por isso podem
não corresponder a realidade brasileira. Apesar disso, o conjunto
de dados podem servir como base para comparação da performance dos
\glspl{nilm} aqui desenvolvidos com os do exterior, assim como nada
impede o emprego em paralelo de dados próprios mais representativos
para demonstrar a realidade de aplicação e capacidade da abordagem
utilizada.

Outra dificuldade foi o fato de autores utilizarem uma
medida própria para o cálculo das taxas de eficiência.  Além disso,
normalmente os autores não reportaram as taxas de falsos positivos na
Etapa~\ref{itm:etapa2}, apenas a capacidade dos algoritmos de
detectarem os eventos (exceções observadas são
\cite{nilm_marceau_16ssamp_improved_1999_18,nilm_liang_pt2_2010_40}).
Em outros casos, os autores concentraram-se apenas na capacidade dos
algoritmos da Etapa~\ref{itm:etapa3} de discriminarem equipamentos,
reportando medidas representativas para essa eficiência.

Por isso, \cite{nilm_zeifman_review_2011} recomenda a utilização das
medidas apresentadas por \cite{nilm_liang_pt1_2010_34}, no qual se
apresentou considerações metódicas para o tema. Foram apresentadas
três medidas. A primeira medida, \gls{det_eff}, considera a capacidade
do \gls{nilm} de desagregar a informação nos eventos que foram
detectados\footnote{Empregada por \cite{nilm_hart_1992_8} quando não
disponível a medição paralela de energia dos equipamentos e por
\cite{nilm_gupta_patel_2010_30}.}. Quando interessado apenas em
estudar a capacidade do classificador para os eventos detectados, a
medida \gls{class_eff} deve ser utilizada. Finalmente, a
\gls{total_eff} é dada por \ref{eq:total_eff}, levando em conta apenas
a capacidade do \gls{nilm} de corretamente classificar os eventos
reais, causados pelos equipamentos na rede\footnote{Empregada por
\cite{nilm_patel_2007_29,nilm_berges_2009_36} pois a
Etapa~\ref{itm:etapa2} foi realizada manualmente
\label{fn:patel_manual} e, geralmente, por demais estudos que
estudaram apenas a Etapa~\ref{itm:etapa3}.}.

\begin{subequations}\label{eq:eff}
\begin{equation}\label{eq:det_eff}
\eta_{det} = \frac{N_{id}}{N_{real} + N_{fp} - N_{ni}}
\end{equation}
\begin{equation}\label{eq:class_eff}
\eta_{class} = \frac{N_{id}}{N_{real} - N_{ni}}
\end{equation}
\begin{equation}\label{eq:total_eff}
\eta_{total} = \frac{N_{id}}{N_{real}}
\end{equation}
\end{subequations}

\noindent onde:  

\begin{description}
\item[$N_{id}$] são eventos identificados, ou seja, corretamente
detectados e classificados pelo \gls{nilm}; 
\item[$N_{real}$] é o número de eventos realmente causados pelos
equipamentos na rede;
\item[$N_{fp}$] são eventos devido a falsos positivos, ou seja,
evento erroneamente identificados;
\item[$N_{ni}$] são eventos não identificados, ou perdas de alvo.
\end{description}

Segmenta-se \ref{eq:total_eff} para obter a eficiência do \gls{nilm}
por equipamento conforme:

\begin{equation}\label{eq:app_eff}
\eta_{total}^i\approx\frac{N_{id}^i}{N_{real}^i} ~~ \forall ~~ i =
1,2,...,N_{ap}
\end{equation}

\noindent onde $N_{id}^i$ e $N_{real}^i$ são os respectivos
$N_{id}$ e $N_{real}$ para o i-ésimo equipamento dos $N_{ap}$
disponíveis.

O grande favorecimento para essas medidas é sua simplicidade de serem
obtidas, porém algumas considerações podem ser feitas sobre elas.
Primeiro, a medida com maior sensibilidade à capacidade do \gls{nilm}
é a \ref{eq:det_eff}, uma vez que os valores por ela representados
levam em conta as perdas de alvo e os falsos positivos.
Segundo, as mesmas não levam em conta o consumo de energia dos
equipamentos, dando importância análoga para equipamentos com parcelas
pequenas ou grandes de consumo. Além disso, a correta identificação
dos $N_{id}$ não significa que a energia será corretamente
reconstruída, dependendo da capacidade do \gls{nilm} de unir essas
informações para gerar a informação do consumo desagregado.  Ainda,
como apontado por \cite{nilm_zeifman_review_2011}, elas apenas
representam a eficiência no ponto de operação, não sendo possível
observar como o \gls{nilm} se portaria para outros pontos. Indo além,
elas não permitem comparações de técnicas utilizadas exclusivamente para
as Etapas~\ref{itm:etapa2} e \ref{itm:etapa3}, impedindo a
contraposição de técnicas onde os autores se limitaram a uma dessas
etapas.

\subsubsection{Outras representações}
\label{top:outras_eff}

Por isso, além das medidas apontadas, outras maneiras de representar
a eficiência podem ser utilizadas para complementar o estudo do
comportamento da abordagem utilizada. Uma técnica para representar o
compromisso entre a capacidade de detectar eventos e a quantidade de
falsos positivos encontrados é a curva \gls{roc}, também recomendada
por \cite{nilm_zeifman_review_2011}. A \gls{roc} além de ser utilizada
para expressar de maneira geral a capacidade do algoritmo de detectar
e identificar em função dos falsos positivos, pode ser utilizada para
estudar a eficiência específica da Etapa~\ref{itm:etapa2}. Já para a
Etapa~\ref{itm:etapa3}, a matriz de confusão permite entender quais
equipamentos ou classes de equipamentos são confundidos em outras classes,
assim como a eficiência de classificação em uma única representação.

As outras medidas utilizadas na literatura levantada são: o
percentual de classificações corretas por ciclo da rede, ou seja, a
correta classificação do estado de operação para cada ciclo dividido
pelo número total de ciclos
\cite{nilm_srinivasan_nn_2006_27,nilm_suzuki_2011_35}; porcentagem de
detecção de transitórios \cite{nilm_patel_2007_29}\footnote{O estudo
reportou eficiência para as Etapas~\ref{itm:etapa2} e \ref{itm:etapa3}
separadamente. Como foi dito na nota \ref{fn:patel_manual}, a
Etapa~\ref{itm:etapa3} utilizou eventos recortados manualmente.};
\gls{p_eff_i}
\cite{nilm_hart_1992_8,nilm_cole_data_extraction_1998_14,
nilm_cole_extra_info_surge_1998_15,nilm_farinaccio_16ssamp_1999_17,
nilm_marceau_16ssamp_improved_1999_18}; \gls{p_eff} 
\cite{2010_nilm_melhorando_pph_usa_37}; desvio
do tempo em que o equipamento foi identificado operando em relação ao
tempo que ele realmente estava operando
\cite{nilm_farinaccio_16ssamp_1999_17}\footnote{O estudo focou na
identificação de grandes cargas elétricas como ar condicionado e
aquecedores de água, modelados por \gls{c3}\label{fn:valc3}. Essa
medida seria limitada para outros modelos.}; porcentagem de detecção
de eventos de transitório para ligado perdidos
\cite{nilm_farinaccio_16ssamp_1999_17}\fnref{fn:valc3}; erro médio
absoluto de reconstrução de energia e outras estatísticas por equipamento
\cite{nilm_powers_15minsamp_1991_16}.

A medida mais utilizada pelas referências, a \gls{p_eff_i}, embora
por elas não definida matematicamente, concebe-se que seja dada por:

\begin{subequations}
\begin{equation}\label{eq:frac_en_app}
\rho_{En}^i = \frac{E_{det}^i}{E_{real}^i} ~~ \forall ~~ 
i = 1,2,...,N_{ap}
\end{equation}
\begin{equation}\label{eq:frac_en}
\rho_{En} = \frac{\sum_{i}^{N_{ap}}E_{id}^i}{\sum_{i}^{N_{ap}}E_{real}^i} 
\end{equation}
\end{subequations}

\noindent onde $E_{det}^i$, $E_{real}^i$ é o consumo detectado e
consumo real do i-ésimo equipamento, respectivamente, o último sendo
obtido por submedição ou por um estimador. A \gls{p_eff_i}
pode ser generalizada para calcular a \gls{p_eff} através de
\ref{eq:frac_en}.  Essas medidas levam em consideração o consumo
detectado pelo \gls{nilm}, mas perdem a capacidade das medidas 
\ref{eq:eff} de representar a informação que foi corretamente
identificada. Por exemplo, se um equipamento é considerado como ligado
em um espaço de tempo em que o mesmo está desligado, isso irá
contribuir para corrigir possíveis erros que seriam atribuídos quando
o estado estimado e a operação estiverem na lógica oposta.

Assim, fica evidente que essas medidas precisam ser refinadas para
identificar os momentos nos quais a energia foi corretamente
reconstruída. Para isso, aqui se sugere o uso de \ref{eq:e_id_i} com
o intuito de determinar a \gls{e_id_i}. A ideia
é representar que identificações do equipamento em outros estados, mas
com pequena diferença de energia, não irão afetar tanto na resolução
de energia, assim como resguardar que identificações em estados de
consumo maiores para os quais os equipamentos realmente operam não
arremeterão na conta de energia corretamente identificada.

\begin{equation}\label{eq:e_id_i}
E_{id}^i = E_{det}^i-\varepsilon^i
\end{equation}

A \gls{en_res} representa a ideia, em energia, para tanto falsos
positivos ou quanto identificações errôneas para estados de maior
consumo, ou seja, a parcela de $E_{det}^i$ que excede àquela obtida
através de um submedidor --- medição diretamente da energia do
equipamento --- ou estimada $E_{real}^i$. Ela pode ser descrita
por:

\begin{equation}\label{eq:en_res}
\varepsilon^i = \left\{\begin{array}{rl}
 E_{det}^i - E_{real}^i &\mbox{ se $E_{det}^i>E_{real}^i$} \\
 0 &\mbox{o.c.}
\end{array} \right. ~~ \forall ~~ i = 1,2,...,N_{ap}
\end{equation}

Isto posto, para obter a \gls{en_eff_i} e sua generalização,
\gls{en_eff}, basta empregar:

\begin{subequations}
\begin{equation}\label{eq:en_eff_i}
\eta_{En}^i = \frac{E_{id}^i}{E_{real}^i} ~~ \forall ~~ i =
1,2,...,N_{ap}
\end{equation}
\begin{equation}\label{eq:en_eff}
\eta_{En} = \frac{\sum_{i}^{N_{ap}}E_{id}^i}{\sum_{i}^{N_{ap}}E_{real}^i}
\end{equation}
\end{subequations}

E para as taxas de redundância:

\begin{subequations}
\begin{equation}\label{eq:p_red_i}
\rho_{red}^i = \frac{\varepsilon^i}{E_{real}^i} ~~ \forall ~~ i =
1,2,...,N_{ap}
\end{equation}
\begin{equation}\label{eq:p_red}
\rho_{red} = \frac{\sum_{i}^{N_{ap}}\varepsilon^i}{\sum_{i}^{N_{ap}}E_{real}^i}
\end{equation}
\end{subequations}

Posteriormente, descobriu-se que o próprio autor de
\cite{nilm_zeifman_review_2011} criou uma medida que, depois de
modificada por \cite{seminilm_fhmm_empiricalnmeter_2013}, permite
também exprimir a questão de precisão de reconstrução de energia, bem
como reflete, de outra maneira, a quantidade de energia redundante. 
A alteração de \cite{seminilm_fhmm_empiricalnmeter_2013} será referida
como \glslink{medidafenergia}{medida-eF (\acs{medidafenergia})}.
Ela pode ser descrita pelo quadrado da média geométrica normalizado
pela média aritmética (\ref{eq:fmeasure_en}) de duas grandezas: 

\begin{itemize}
\item parcela de energia atribuída ao equipamento que foi realmente
consumida pelo mesmo em relação ao seu consumo pelo \gls{nilm}
atribuído, um parâmetro que também reflete a ideia de energia
redundante, descrita por \ref{eq:en_recon};
\item parcela de energia que foi corretamente identificada em relação
ao consumo total do equipamento, a própria \gls{en_eff_i} aqui descrita
em \ref{eq:en_eff_i}.
\end{itemize}

\begin{equation}\label{eq:en_recon}
\eta_{En,prec}^i = \frac{E_{id}^i}{E_{det}^i}
\end{equation}

\begin{equation}\label{eq:fmeasure_en}
F_{en}^i=\frac{2 \;\; \eta_{En,prec}^i \;\; \eta_{En}^i}{\eta_{En,prec}^i+\eta_{En}^i}
\end{equation}

Entretanto, a \glslink{medidaf}{medida-F} (\ref{eq:fmeasure}) original
apresentada por \citet*{nilm_zeifman_vastext_approach_2012}, como
observado pelo próprio autor, não é ótima como medida de desagregação,
por não levar em consideração erros na detecção de energia. Ela fica
aqui referenciada por ter sido utilizada em diversos estudos desse
autor.

\begin{equation}\label{eq:fmeasure}
F^i=2\dfrac{\left(\dfrac{N_{id}}{N_{id}+N_{fp}}\right)\eta_{total}}
{\left(\dfrac{N_{id}}{N_{id}+N_{fp}}\right)+\eta_{total}}
\end{equation}


\subsection{Técnicas Aplicadas por demais autores}
\label{ssec:nilm_tecnicas}

As abordagens aplicadas desde o início dos estudos ao tema e
utilizadas como referências fizeram mão de ostensivas técnicas para a
desagregação do consumo. Cada vertente buscou extrair características
ou inovar aplicando outras técnicas, de forma que é possível observar
uma grande diversidade de abordagens. As abordagens serão agrupadas em
relação à \gls{fex} realizada. A capacidade de extrair
características dos sinais é correlacionada com a frequência de
amostragem e, em vista disso, dividir-se-ão os métodos aplicados de
acordo com a taxa de amostragem utilizada. A ideia de subdivisão aqui
seguida foi de autoria da referência \cite{nilm_zeifman_review_2011}.

Em seguida, um outro tópico especifico para as topologias onde há
semi-intrusão da residência, utilizando submedição em pequenas
escalas, tanto energeticamente, quanto correlacionando a utilização de
energia com outros medidores, como temperatura, movimento, som etc.

\subsubsection{1. Medição com Baixa Amostragem}
\label{top:nilm_baixa_am}

A utilização de características macroscópicas de consumo do equipamento,
como alterações no patamar de consumo da rede, foi a
primeira abordagem encontrada ao tema. As mesmas podem ser obtidas sem
grande granularidade na taxa de amostragem, por isso, esse tipo de
abordagem beneficia-se de medidores de baixo custo, amplamente
disponíveis no mercado. No entanto,
\cite{nilm_berges_2008_7,nilm_matthews_overview_2008_22} alertam
para discrepâncias entre medidores na ordem de 10\%-20\%, bem maiores
que aquelas alegadas, de 3\%. Os medidores testados no caso foram
\emph{Brand Meter I}, \emph{Watts Up? PRO} e \emph{EnerSure}.

A taxa de amostragem mais frequentemente utilizada é 1 Hz, entretanto
alguns estudos fizeram mão taxa de amostragem ainda menores por
desejarem identificar equipamentos que se ressaltam dentre os outros
devido ao seu relativo alto consumo, como ar condicionado, aquecedores
de água e geladeira. Exemplos de medidores utilizados no exterior são
\gls{ted} \cite{ted_site} e \emph{Watts up? PRO} \cite{wattsup_site},
o último sendo capaz de informar o consumo de \acl{q}.

\begin{enumerate}[label=\textbf{1.\arabic*},wide=\parindent]
\item \textbf{\Acrlong{p} e \Acrlong{q}}
\label{nilm:pot_real_reat}

\indent A referência inicial de grande destaque no tema,
\citet*{nilm_hart_1992_8}\footnote{\citet*{nilm_sultanem_1991_10}
publicou seu estudo em \citeyear{nilm_sultanem_1991_10}, antes de 
\citeauthor*{nilm_hart_1992_8} que o publicou em
\citeyear{nilm_hart_1992_8}. Aparentemente, nenhum dos autores teve
conhecimento do trabalho do outro. Apesar do trabalho de
\citeauthor*{nilm_sultanem_1991_10} ter sido publicado primeiro,
\citeauthor*{nilm_hart_1992_8} retrata em seu trabalho que sua descoberta
ocorreu em 1982, tendo diversos trabalhos realizados pelo seu grupo em
relação ao tema desde essa data, resultando em um material bem mais
extenso que o de \citeauthor*{nilm_sultanem_1991_10}.}, ocorreu em
1992. Nela, utilizam-se medições de
\gls{p} e \gls{q} com uma taxa de amostragem de 1 Hz. A abordagem
aplica uma normalização para reduzir flutuações no consumo devido a
alterações na tensão de acordo com \ref{eq:norm_hart} com o intuito de
reduzir dispersões nos dados. O estudo de \citeauthor*{nilm_hart_1992_8}
limitou-se a identificar apenas cargas com potência maior a 150
\acs{watt}. A essência da metodologia ainda pode ser encontrada
em \glspl{nilm} mais atuais, sendo esta: 

\begin{equation} \label{eq:norm_hart}
P_{\text{norm}}(t) = \left[ \frac{120}{V(t)} \right]^2 P(t)
\end{equation}

\begin{enumerate}[label=\arabic*]
\item Detectam-se transitórios de consumo na rede devido a mudança de
estado de um equipamento através de alterações no consumo que devem
superar um limiar específico (15 \acs{watt}/\acs{var}) para os sinais
normalizados como em \ref{eq:norm_hart} para \acs{p} e \acs{q}.
As amostragens dentro de um regime permanente são normalizadas para
sua média com o objetivo de tirar o ruído. Para a \gls{fex}, utiliza-se o
degrau entre o regime permanente posterior e anterior (já no valor de
suas médias) ao evento transitório para \gls{p} e \gls{q};
\item Os eventos de transitório são analisados por um algoritmo de
agrupamento que irá gerar os centroides das mudanças de estado
possíveis causadas pelos equipamentos no plano
\acs{dp}$\times$\acs{dq};
% FIXME Tem que ver se ele cria C2, ou na verdade procura combinações
% para gerar C3.
\item \textlabel{Centroides com simetria em relação aos eixos são tomados em
pares e com eles são criados modelos \gls{c3}}{text:passo3}. Para os
centroides remanescentes, além de regras heurísticas como a junção de
centroides próximos que permitam o pareamento com um outro refletido
nos eixos --- etapa conhecida como resolução de conflitos ---,
determinam-se possíveis combinações de centroides que possam formar
uma \gls{c2} utilizando uma adaptação do algoritmo de \emph{Viterbi}
\cite{nilm_bouloutas_viterbi_ext_1991_11,
nilm_hart_fsm_viterbi_1993_12}\footnote{A versão original pode ser
encontrada em \cite{viterbi_alg}.}. Assim que é determinada uma combinação
que permite a criação de uma \gls{c2}, os centroides da mesma são
removidos, e o processo continua até que todas as \glspl{fsm} tenham
sido construídas. A adaptação utilizada
\cite{nilm_bouloutas_viterbi_ext_1991_11,
nilm_hart_fsm_viterbi_1993_12} permite várias operações para consertar
corrupções e retornar uma estimativa ótima das alterações de estados
operativos da \gls{fsm} original. Como a reconstrução depende da
estatística do processo, é necessário que as mudanças de estado das
\glspl{fsm} observadas tenham um comportamento recorrente para que a \gls{fsm}
original seja reconstruída, e por isso, restringe-se apenas às
\glspl{c2a}. As \glspl{c2b} podem ser reconstruídas se houver
conhecimento prévio da presença das mesmas, de modo que elas sejam
medidas operando em cada um de seus estados e então inseridas
manualmente no catálogo do \gls{nilm};
\item Em seguida é levantado o comportamento dos equipamentos,
montando o estados de consumo para cada equipamento. É utilizado um
algoritmo de força-bruta para corrigir ocorrências de dois acionamentos
ou desacionamentos de um mesmo equipamento encontrados em seguida. A causa
desses erros é, geralmente, a ocorrência de um evento simultâneo de
dois equipamentos. Assim, o algoritmo busca por eventos não-usuais
cuja soma é o valor de dois outros eventos perdidos;
\item Por fim, é levantada a estatística detalhando o
comportamento de consumo, como o tempo ligado e desligado de cada
equipamento. Essa informação, junto com a potência do equipamento é
utilizada para auxiliar a identificar o equipamento. 
\end{enumerate}

O método é robusto para o desagregação de cargas \glspl{c3}
($> 150~$\acs{watt}) e as adaptações \cite{nilm_bouloutas_viterbi_ext_1991_11,
nilm_hart_fsm_viterbi_1993_12} do algoritmo de \emph{Viterbi}
parecem resolver o problema das \glspl{c2a}. Outro problema das
metodologias envolvendo algoritmos de agrupamento é a lenta alteração
da resistência conforme a operação do equipamento. Geralmente, ao
interromper a operação, o equipamento tem alterações no consumo na margem
de 5\%-10\% em relação ao inicio de operação
\cite{nilm_sultanem_1991_10}. \citeauthor*{nilm_hart_1992_8} observa
o degrau geralmente é menor em valor absoluto para os desacionamentos
nos casos de equipamentos com motores, que reduzem o consumo conforme
seu aquecimento.

\textlabel{Uma estratégia bastante parecida é realizada por}{text:cole}
\citet*{nilm_cole_data_extraction_1998_14,
nilm_cole_extra_info_surge_1998_15}, 
onde são feitas considerações em relação as características de bordas e
inclinações. Aqueles são definidos como o auge atingido de potência
durante o acionamento e estes variações lentas de mudança no consumo.
Apesar de definir as bordas como o pico de potência, as referências
empregam as bordas apenas como os eventos de transitório de consumo, não
empregando essa informação para classificação. A abordagem aplicada,
ao invés de agrupar os dados para depois procurar por possíveis
equipamentos como feito por \citeauthor*{nilm_hart_1992_8}, primeiro
busca temporalmente por ciclos fechados nos eventos de transitório (ou
bordas, como na nomenclatura da referência), que depois serão
adicionadas aos centroides no espaço
\acs{dp}$\times$\acs{dq}. Se o centroide não existir, será
criado um candidato a centroide. Conforme a quantidade de ciclos dos
centroides aumenta, o mesmo irá se tornar um candidato a uma carga.
Para as cargas \glspl{c3}, a carga será aceita apenas se a detecção
das bordas ocorrerem repetidamente. Já para as cargas \glspl{c2a}, foi
realizado um estudo da probabilidade dela ter sido originada pela
sobreposição de duas bordas geradas por equipamentos distintos. A
conclusão foi que se forem encontrados mais de um ciclo de três bordas
em um período de 6 horas é suficiente para aceitá-lo como uma
\gls{c2a}. Finalmente, o envelope só foi considerado para a melhoria
em resolução de energia e, no entanto, a referência indica que a
utilização da média de consumo entre as bordas apresenta melhores
resoluções.

\item \textbf{\Acrlong{p}, \Acrlong{q} e Transitório}
\label{nilm:pot_real_trans}

Um trabalho paralelo ao de \citeauthor*{nilm_hart_1992_8} foi
realizado pelo mesmo instituto para aplicar o \gls{nilm} no setor
comercial e industrial, sendo realizado por
\citet*{nilm_norford_leeb_medianfilt_1996_13,
nilm_leeb_spectral_envelope_1995_23}.
No setor comercial são encontrados equipamentos com características
diferentes ao setor residencial, geralmente com transitórios mais
lentos (podendo chegar a cerca de centenas de segundos
\cite{nilm_norford_leeb_medianfilt_1996_13}), menor consumo reativo
devido às preocupações com a qualidade de energia e consequentemente
correção do fator de potência, e a presença de \gls{c5}, como exemplo,
na referência foi observada uma bomba com picos periódicos de 20
k\acs{watt}. Tipicamente há também uma maior presença de
equipamentos com cargas variáveis, \glspl{c4}, como motores de
velocidade variável. 

Assim, foi adicionado a informação do transitório
da envoltória em amostragens de 1~\acs{hz}, que por serem
maiores, ainda podem ser observados em amostragens baixas, suprindo,
ao mesmo tempo, a menor capacidade de discriminação da variável
\gls{q} nesse setor. Para redução dos ruídos, utilizou-se um filtro de
mediana com 11 pontos, esse sendo mais indicado para a eliminação dos
picos quando comparado aos filtros lineares, que terão dificuldades de
distinguir os picos e os degraus, uma vez que eles tem espectros de
frequência parecidos. Os trabalhos utilizam o conceito de seções-v
(tradução própria de \emph{v-sections}), que são subdivisões do
transitório em segmentos de variação significante. Para isso, o grupo
utilizou uma janela para identificar mudanças na média e criar as
seções-v, onde os valores ótimos são determinados por um processo de
treinamento pré-instalação ou em laboratório. É aplicada uma medida de
distância entre as seções-v observadas e os transitórios
característicos que, ao estarem dentro de um limiar, serão
identificados como um determinado equipamento. Para o tratamento das
\glspl{c4}, a referência indica o emprego de variáveis de controle,
quando disponíveis, correlacionadas com o seu consumo para estimá-las,
como o caso para os equipamentos de \gls{avac} em geral.

\item \textbf{Unicamente \acl{p}}
\label{nilm:pot_real}

A medição de \acl{q} adiciona custo ao \gls{nilm} --- ainda que
não tão oneroso quanto medições em altas frequências --- e, para
detectar certos equipamentos com assinaturas de destaque na rede,
essa variável pode ser desnecessária. Em outros casos, medidores que
disponibilizam essa informação podem não estar disponíveis, sendo
possível operar apenas com a \acl{p}.

\begin{enumerate}[label*=.\textbf{\arabic*},wide=\parindent]
\item \textbf{Separação dos principais equipamentos por uso-final}

Exemplos do primeiro caso são os estudos de
\citet*{nilm_powers_15minsamp_1991_16,nilm_farinaccio_16ssamp_1999_17,
nilm_marceau_16ssamp_improved_1999_18},
para os quais os autores se preocuparam em identificar apenas
equipamentos de maior uso-final.
\cite{nilm_powers_15minsamp_1991_16,nilm_farinaccio_16ssamp_1999_17}
utilizaram somente regras heurísticas, enquanto
\cite{nilm_marceau_16ssamp_improved_1999_18} também empregou os
degraus em potencia real e um filtro para a detecção dos
acionamentos/desacionamentos dos equipamentos.

Em \cite{nilm_powers_15minsamp_1991_16}, foram reportadas a capacidade
de reconstrução para ar condicionado e aquecedores de água. A
amostragem é realizada a cada 15 minutos e os arquivos são analisados
dia a dia. Por ser proprietário, as regras não são detalhadas (é
utilizada uma árvore de decisões, embora o estudo considere a aplicação
de redes neurais), mas o algoritmo procura por picos no consumo, assim
como sua duração, tempo e magnitude, que são utilizados pelas regras
para determinar se os mesmos foram utilizados para os usos-finais
cobiçados. Posteriormente, eles são ajustados conforme
verificações de consistência. Para o ar condicionado, é relatado que o
valor de pico estimado médio para as residências difere cerca de
apenas 5\% do valor original médio, enquanto o consumo fica na margem
de 10\% e observa-se boa capacidade de estimar os horários de consumo.
 
Já os estudos \cite{nilm_farinaccio_16ssamp_1999_17,
nilm_marceau_16ssamp_improved_1999_18}, realizados por outro
grupo, empregaram amostragem de \acs{p} a cada 16~segundos. Os
equipamentos estudados foram: geladeira, aquecedor de água e aquecedores
de ambiente (este somente em
\cite{nilm_marceau_16ssamp_improved_1999_18}\footnote{O algoritmo da
referência \cite{nilm_marceau_16ssamp_improved_1999_18}
também leva em consideração a máquina de lavar roupa, mas os
resultados focaram apenas nos outros três equipamentos.}).
\cite{nilm_zeifman_review_2011} expõe a arbitrariedade e não
intuitividade das regras utilizadas em
\cite{nilm_farinaccio_16ssamp_1999_17}, que precisam ser estudadas
para cada caso de equipamento. Foram determinadas 8 regras para cada
equipamento (algumas regras são reaproveitadas entre equipamentos),
divididas em duas etapas: determinar o conjunto de eventos de
transitório e a duração do consumo.  Em seguida, a duração de consumo é
multiplicada pela demanda média do equipamento durante a fase de
treinamento para obter o consumo estimado. A fase de treinamento,
período em que há medição paralela dos equipamentos, e, por isso,
ocorrendo intrusão da propriedade do consumidor, é feita para um
período de uma semana. A reconstrução de energia diária para os
equipamentos é na margem de $-10,5\%$ a $15,9\%$.

O estudo em sequência aperfeiçoou o anterior com uma abordagem única
para determinar os equipamentos em operação. Ele compara, em ordem
decrescente em termos de demanda média operativa, se a magnitude
do evento é próxima à média do nível operativo de um dos equipamentos
almejados, empregando como limiar de corte dois desvios padrão. Ainda
assim, a referência emprega diversas regras de pré/pós-processamento
determinadas empiricamente para melhorar a resolução em energia, assim
como também necessita do período de treinamento através de medição
paralela de 1~semana, limitando a aplicabilidade do método para uma
gama maior de equipamentos. Por outro lado, o método serve para o seu
proposito, obtendo reconstruções na faixa de 10\% para a maioria das
análises realizadas.

\item \textbf{\Acrlong{q} não disponível}

\begin{itemize}[wide=\parindent]
\item \emph{A abordagem de 
\citeauthor*{nilm_baranski_genetic_base_2003_19}}

Com o intuito de possibilitar a impregnação da aplicação de \glspl{nilm}
na Alemanha, \citet*{nilm_baranski_genetic_base_2003_19,
nilm_baranski_genetic_detalhado_2004_20,nilm_baranski_summary_2004_21}
recorreram a leitura ótica dos medidores eletromecânicos
(o trabalho \cite{nilm_baranski_genetic_base_2003_19}, realizado em
2003, indica que mais de 99\% dos medidores desse país possuem essa
configuração) para obter as medições com frequência de
1~\acs{hz}. Por isso, apenas \gls{p} estava disponível para esses
estudos. 

Apesar das limitações, essa abordagem é uma referência de destaque
devido às diversas contribuições feitas. Para melhorar a capacidade de
discriminação entre os equipamentos, além da potência ativa, o estudo
adicionou como característica o pico de consumo para o evento de
transitório, bem como o período que o mesmo leva para estabilizar
(apresentado em \cite{nilm_baranski_genetic_detalhado_2004_20}),
diferente de \cite{nilm_cole_data_extraction_1998_14,
nilm_cole_extra_info_surge_1998_15} que observou essas propriedades,
mas empregou somente a última com o intuito de melhorar a
capacidade de reconstrução de energia. A grande contribuição foi uma
abordagem por otimização para a identificação dos modelos de carga a
serem encontrados que trabalham como uma adaptação do algoritmo de
\emph{Viterbi}, que podem ser resumidos da seguinte maneira:

\begin{enumerate}
\item criação das \gls{fsm} por algoritmos genéticos
para reduzir o tempo de otimização; 
\item em seguida essas são otimizadas por \gls{es} para obter os
parâmetros da \gls{fsm} (tempo e consumo em cada estado); 
\item e finalmente, os modelos utilizam lógica \emph{fuzzy}, permitindo que
dois ou mais modelos sejam criados para um mesmo distúrbio na rede,
sendo depois escolhido o modelo que melhor se aplica.
\end{enumerate}

Tratar-se-ão de detalhes das técnicas aplicadas pelos autores por três 
motivos:

\begin{itemize}
\item a técnica teve bons resultados apesar de utilização de pouca
informação, podendo ter melhores resultados quando alimentada com mais
informação e, por isso, sendo um possível caminho a ser percorrido;
\item outros autores \cite{nilm_bergman_distribuido_2011,
nilm_zeifman_vast_2011,nilm_zeifman_vast_hisample_pdfmerge_2011,
nilm_zeifman_vastext_approach_2012,
nilm_zeifman_statistical_vastext_1stws_2012,
nilm_zeifman_statistical_naive_enduses_2013} se basearam nessas
ideias;
\item os artigos não são de compreensão trivial, em especial para a
elucidação da adaptação do algoritmo de \emph{Viterbi}\footnote{Para os
leitores que desejarem se aprofundar, recomenda-se a leitura de
\cite{nilm_bergman_distribuido_2011} antes dos artigos de
\citeauthor*{nilm_baranski_genetic_detalhado_2004_20}.}. 
\end{itemize}


A estratégia começa com o agrupamento dos dados em centroides,
limitando-se a degraus acima de um limiar mínimo de potência (valores
aplicados de 50 \acs{watt} em
\cite{nilm_baranski_genetic_base_2003_19} e 80 \acs{watt}
\cite{nilm_baranski_genetic_detalhado_2004_20}). As abordagens em
\cite{nilm_baranski_genetic_base_2003_19,
nilm_baranski_genetic_detalhado_2004_20} se basearam no agrupamento
utilizando lógica \emph{fuzzy}, mas na última referência
\cite{nilm_baranski_summary_2004_21}, além desse método, cita-se o
emprego de \gls{som} para essa etapa. A fim de reduzir a complexidade
do problema (a estimativa de eventos é de $16.000$ por dia), o autor
desconsidera os centroides com poucas ocorrências, limitando-se a
identificar apenas equipamentos com padrões recorrentes. 

Na primeira abordagem \cite{nilm_baranski_genetic_base_2003_19},
\citeauthor*{nilm_baranski_genetic_base_2003_19} segmentaram a etapa
de modelar os equipamentos. A primeira modela as \glspl{c3} simplesmente
encontrando pares de centroides no espaço. Para validar os modelos
\glspl{c3} encontrados, é gerado a matriz de correlação cruzada
utilizando o estado de operação para os equipamentos em cada instante de
tempo. Se o equipamento $i$ e o equipamento $j$ forem na realidade uma
\gls{fsm}, espera-se $r_{ij}\approx1$, onde $r_{ij}$ indica a
frequência de operação do equipamento $j$ quando $i$ está operando,
associando, assim, $j$ com a operação $i$ (o corte utilizado é de
0,8). Já para as \gls{c2}, são criados todos os modelos que
juntos somam aproximadamente zero e seus centroides tem frequência de
eventos também próximos. Esses modelos são então validados
temporalmente, e junto com as \glspl{c3} são comparados com um
possível catálogo antigo a fim de atualizá-lo. Em seguida, uma rede
neural é treinada com os padrões encontrados para os equipamentos (o
autor cita como exemplo: tempo médio de consumo, consumo médio, número
de estados) para encontrar esses padrões na residência.

Essa abordagem é aprimorada em
\cite{nilm_baranski_genetic_detalhado_2004_20,nilm_baranski_summary_2004_21},
que ao invés de encontrar todos possíveis modelos de \gls{fsm},
faz a otimização das possíveis máquinas através de algoritmo genético.
Nessa abordagem não há a discriminação para a criação de \gls{c2} ou
\gls{c3}, a abordagem única utiliza $N_{ap}$ (o número de equipamentos
deve ser maior que o número de centroides, no entanto, não é
especificado um bom valor a ser utilizado) \glspl{fsm} para os quais e o
algoritmo genético fica encarregado de alterar valores binários em uma
matriz $\mathbf{X}$ representando se um determinado centroide
pertence, ou não, à \gls{fsm}. É possível que um mesmo centroide
pertença a mais de uma \gls{fsm}. São utilizados três critérios para
otimização: 

\begin{itemize}
\item minimização do valor absoluto de potência da soma dos
centroides pertencentes a \gls{fsm}; 
\item o item anterior, mas levando em conta a frequência de
eventos em cada centroide; 
\item e minimização do número de centroides em
cada \gls{fsm} (priorizando equipamentos com menos estados).
\end{itemize}

Uma adaptação do algoritmo de \emph{Viterbi} é utilizada para encontrar os
modelos de \gls{fsm}. Com os modelos resultantes, é criado as
sequências de estado para elas supondo que as mesmas são \gls{c2a}.
Mais precisamente, os autores consideram que as sequências de estados
devem ser recorrentes com seus parâmetros em uma área limitada dentro
de seu valor esperado (os autores citam dois exemplos de parâmetros:
tempo de duração no determinado estado e a capacidade de reconstrução
de energia para o consumo estimado nos estados da \gls{fsm} em relação
ao consumido nos caminhos percorridos; mas não dá detalhes de quais
empregou) e apenas visitados uma vez em cada ciclo. Para isso, é
realizada uma otimização em dois tempos. Primeiro, encontra-se o
melhor caminho para aqueles que obedecem as restrições (consumo de
potência positiva e permanência em um estado por um tempo não muito
longo), juntando os estados da máquina em um caminho de operação com a
melhor qualidade em relação aos parâmetros escolhidos. Os parâmetros
podem ser iniciados com os valores da mediana para todos os eventos
acoplados a \gls{fsm}. A qualidade é avaliada pela entropia de
\emph{Shannon}, \ref{eq:shannon}. Isso é repetido iterativamente
utilizando \gls{es} que irá alterar os parâmetros até a convergência
da qualidade.

\begin{equation}\label{eq:shannon}
Q_{shannon} = - \Delta{e_{i}} \log{|\Delta{e_{i}}|}
\end{equation}

Os melhores caminhos operativos para as \glspl{fsm} ainda precisam ser
resolvidos quanto aos centroides que pertencem a mais de um equipamento.
Para isso, \cite{nilm_baranski_summary_2004_21} cita resumidamente um
algoritmo de força bruta que irá investigar para cada sobreposição
qual caminho tem a melhor qualidade.

Os autores revelam que o método necessita de 5 a 10 dias para
encontrar os modelos dos equipamentos típicos, enquanto dados diários
são suficientes para atualizar o catálogo de equipamentos detectados em
cada residência. Os resultados mostram que os equipamentos de maiores
consumo, como geladeira, aquecedor elétrico (de fluxo) e fogão podem
ser detectados com eficiência.

\item \emph{\gls{dnilm}}

A abordagem de \citeauthor*{nilm_baranski_summary_2004_21} é a base
empregada para o trabalho de \citet*{nilm_bergman_distribuido_2011},
contando com a mesma sequência de criação através de algoritmo
genético e otimização das \gls{fsm}. Diferenças podem ser notadas
apenas para a técnica de agrupamento, que não utilizam as informações
de tempo nem o pico atingido no transitório, contudo, podem utilizar a
\gls{q} se o medidor da residência realizar essa medida. Além disso, o
agrupamento é realizado em tempo real, atualizando a média e desvio
padrão de cada centroide conforme os eventos ocorrem. Se o evento não
for atribuído a nenhum centroide, um novo centroide é criado. Ainda, o
método utiliza outra otimização para fazer a identificação do equipamento
que alterou o estado operativo, um algoritmo de otimização de mochila
(do inglês \emph{Knapsack problem}), enquanto a abordagem anterior
citou a utilização de redes neurais na referência
\cite{nilm_baranski_genetic_base_2003_19}\footnote{
\citeauthor*{nilm_baranski_genetic_base_2003_19}
praticamente não levaram o assunto em consideração, a descrição dos
artigos desses autores gira em torno do tema de modelagem dos
equipamentos, mas tirando a informação aqui citada, não há outra
informação sobre como se tratou o reconhecimento dos estados
operativos através dos modelos.}.

Entretanto a maior contribuição do trabalho
em questão é uma nova arquitetura, distribuída, para o \gls{nilm}.
Nesse caso, diferente dos medidores eletromecânicos disponíveis para
\citeauthor*{nilm_baranski_summary_2004_21}, o trabalho opera com
medidores inteligentes (os medidores das redes inteligentes descritos
na Subseção~\ref{ssec:ret_tec}) para a coleta de
dados. Os medidores inteligentes também servem como pontos de
processamento local, mas devido às limitações de processamento, apenas
a detecção e identificação dos eventos é realizado no mesmo. Por isso,
a geração das \glspl{fsm} é realizada em uma central, com maior
capacidade de processamento, no qual este envia os eventos de
transitório para que aquela os processe e retorne um catálogo com as
\glspl{fsm} e seus padrões a serem encontrados. O catálogo é chamado
de tabela estática. Assim, o medidor fica encarregado apenas de
comparar, localmente, os distúrbios encontrados com o catálogo,
identificando assim os estados operativos dos equipamentos. O estados
operativos de cada equipamento é chamado de tabela dinâmica, e é
preenchida por um algoritmo adaptado para otimização do problema da
mochila. A etapa de criação da
tabela estática, chamada de aprendizagem, é realizada devido a
critérios do controlador (iniciação pró-ativa) ou do medidor
(reativamente). No primeiro caso, o controlador atualiza as tabelas do
medidor se as mesmas expirarem. Já o medidor inteligente pode
requisitar um novo treinamento de acordo com um dos critérios:

\begin{itemize}
\item a diferença absoluta entre a soma da demanda real e estimada
está superior a um patamar;
\item uma \gls{fsm} muda de estado frequentemente, onde os patamares
aplicados para determinar se a mudança de estado é frequente dependem
do equipamento (é mais aceitável observar mudanças frequentes no ar
condicionado ou aquecedor do que em um carregador de bateria
veicular);
\item mais de um determinado número de \glspl{fsm} alteram de estado
em um único evento.
\end{itemize}

Uma das dificuldades do projeto está em ajustar o fluxo de dados. A
referência considera armazenar os dados em períodos de maior atividade
nas residências, enviando as alterações de estado posteriormente
conforme a rede de comunicação não estiver congestionada. Diversas
outras considerações são feitas em relação ao processo de
aprendizagem.

Os resultados reportados são em comparação com um \gls{nilm}
centralizado. O trabalho reservou-se a detectar equipamentos com consumo
superior a 1000 \acs{watt}. O \gls{nilm} centralizado recebe uma
tabela estática otimizada para todo o período, enquanto o \gls{dnilm}
recebe uma tabela treinada para o primeiro dia, podendo atualizá-la de
acordo com o critérios anteriormente citados. As diferenças de
acurácia entre o \gls{dnilm} e o \gls{nilm} centralizado ficaram entre
60\% e 90\%.

\item \emph{Otimização dos estados operativos}

O problema para a otimização da tabela dinâmica --- construção
temporal dos estados operativos dos equipamentos --- foi abordado por
\citet*{nilm_genetic_2013} (aparentemente sem conhecimento do trabalho
de \citeauthor{nilm_bergman_distribuido_2011}). Ao invés de um
algoritmo de força bruta adaptado, aplicou-se um algoritmo genético
para realizar a otimização do problema da mochila. O artigo limitou-se
a estudar a performance do algoritmo em resolver o problema da mochila
e, por isso, considerou-se que são conhecidos os momentos de transição
e os consumos de cada equipamento.

A referência utilizou simulações de 2 horas, gerando aleatoriamente o
tempo de operação dos equipamentos e suas potências. Foram simuladas
diversas condições, variando o número de equipamentos, transições, a
presença de ruído e de equipamentos desconhecidos. As observações
foram: 

\begin{itemize}
\item quanto maior o \gls{nt} e \gls{nap}, menor é a eficiência de
detecção. Casos com pequenos números de \gls{nt} e \gls{nap} obtiveram
100\% de eficiência de detecção;
\item o algoritmo se comportou bem na presença de ruído, onde houve
pouca deterioração na eficiência de detecção;
\item no entanto, a presença de equipamentos desconhecidos ou equipamentos
com sobreposição de potências deterioram bruscamente a performance do
algoritmo.
\end{itemize}

\end{itemize}

\end{enumerate}

\item \textbf{\Acrlong{p} e estatística de uso}
\label{top:nilm_p_estatistica}

Também com o objetivo de possibilitar a impregnação do método, mas no
caso para os \gls{eua},
\citeauthor*{nilm_zeifman_vastext_approach_2012} propôs um novo método
\cite{nilm_zeifman_vast_2011,
nilm_zeifman_vast_hisample_pdfmerge_2011,
nilm_zeifman_vastext_approach_2012,
nilm_zeifman_statistical_vastext_1stws_2012,
nilm_zeifman_statistical_naive_enduses_2013} para abordar o tema em
cima de mostradores de energia domiciliares. Esses mostradores
normalmente disponibilizam apenas a \acl{p}, com uma taxa de
amostragem na faixa de 1~\acs{hz}. A fim de melhorar a capacidade
de desagregação do consumo que é limitada pela pouca informação
oferecida pela sistema de aquisição de dados, o estudo empregou o
conhecimento prévio de uso dos equipamentos através de uma abordagem
estatística para melhorar a capacidade de seu \gls{nilm}. O autor
observa que os estudos \cite{nilm_farinaccio_16ssamp_1999_17,
nilm_marceau_16ssamp_improved_1999_18} também utilizaram a estatística
de uso, servindo, provavelmente, de inspiração para o seu trabalho.
Será tratado da abordagem desse autor com detalhes pois os trabalhos
fornecidos pelo mesmo parecem serem a mais próximas de possibilitar o
\gls{nilm} em um programa de \gls{ee} em larga escala.

Devido à ausência da \acl{q}, a sobreposição dos centroides ocorrerá
com frequência maior, aumentando as ocorrências de erros decorrentes
de \gls{c6}. O autor criou o método \gls{vast} para empregar a informação
estatística no processo de discriminação, Etapa~\ref{itm:etapa3}.
Diferente das outras adaptações do algoritmo de \emph{Viterbi} que se
encontram no levantamento bibliográfico
\cite{nilm_bouloutas_viterbi_ext_1991_11,
nilm_hart_fsm_viterbi_1993_12,nilm_baranski_genetic_base_2003_19,
nilm_baranski_genetic_detalhado_2004_20,nilm_baranski_summary_2004_21},
onde o objetivo é modelar as \glspl{fsm} presentes na rede, a ideia,
aqui, é modelar os estados operativos dos equipamentos como sendo
dependentes, além de seu próprio estado operativo, no estado operativo
de outros equipamentos. Indo além, criar essa dependência para
auxiliar aonde a informação é precária, ou seja, quando há
sobreposição de potência.  O estudo observa que, normalmente, isso só
ocorre para trios de equipamentos, ou seja, supondo que os equipamentos
da residência estejam distribuídos em ordem crescente de consumo, o
equipamento $i$ só deverá ter sobreposição de consumo com o equipamento
$i-1$ e $i+1$. Assim, ao invés do algoritmo de \emph{Viterbi} de força
bruta que tornaria a aplicação inviável conforme o número de equipamentos
(ou estados operativos) aumentasse, já que todos os estados dos outros
equipamentos podem influenciar no equipamento $i$, somente será utilizado os
estados dos equipamentos $i-1$ e $i+1$ para determinar o seu estado
operativo.  Com isso, reduz-se a complexidade da versão original de
\emph{Viterbi}, que é exponencial conforme o número de estados
possíveis, para uma complexidade linear.

%Um exemplo dado pelo autor, supondo que para o trio de equipamentos com
%sobreposição, os equipamentos $i-1$ e $i+1$ tipicamente permanecem
%desligados durante 1 hora, enquanto essa duração para $i$ é de 10
%horas. Quando o equipamento $i$ está desligado, a probabilidade de um
%evento de transição positiva corresponder ao equipamento $i$ sendo ligado
%irá depender, quando apenas levando em consideração o tempo, se os
%outros dois equipamentos estão ligados ou desligados, e neste caso, por
%quanto tempo cada equipamento está desligado.

Para tornar mais simples a tarefa, considera-se que os equipamentos $i-1$
e $i+1$ também serão independente entre si, sendo necessário, assim,
determinar para cada equipamento apenas dois pares de cadeias de
\emph{Markov} (exclusive o equipamento de menor e maior consumo, que
terão apenas uma cadeia): $\{i-1,i\}$ e $\{i,i+1\}$. O preenchimento das
matrizes com as probabilidades de transição é feita com a informação
obtida \emph{a priori}, aplicando as estatísticas de tempo ligado e
desligado, distribuições dos momentos em que os equipamentos trocam de
estado operativo no dia e as distribuições dos eventos de transições
para cada equipamento. Outras características podem ser utilizadas, tais
como: as distribuições para as informações de tempo e amplitude para o
pico de acionamentos (influenciado por
\cite{nilm_baranski_genetic_base_2003_19,
nilm_baranski_genetic_detalhado_2004_20,nilm_baranski_summary_2004_21})
e características ``finas'' específicas de cada equipamento (o autor deu
exemplos em \cite{nilm_zeifman_statistical_naive_enduses_2013} e serão
citados mais adiante neste tópico). Essas probabilidades podem ainda
ser melhoradas considerando distribuições para períodos diurnos ou
noturnos e sazonalidade.

Como o ciclo de observações sofre de alterações como falta de
transições (no caso de perda de alvo para um evento de transitório) ou a
presença de intrusos (tanto falso positivo, ou quanto a presença de
transitório é causado na cadeia $\{i-1,i\}$ por um equipamento
$i+1$, por exemplo), é necessário que a matriz de
probabilidade considere mudanças não esperadas para garantir a solução
do problema. Por exemplo, no caso de um intruso, é necessário dar a
possibilidade dos equipamentos continuarem no mesmo estado reconhecendo
sua presença, enquanto também é necessário dar a possibilidade de um equipamento
$i$, que estava desligado, modificar seu estado para ligado, enquanto
o evento de transitório foi de desacionamento, caso que ocorreria
excepcionalmente para um evento de transitório perdido de acionamento para
o equipamento $i$ e o evento sendo julgado ser intruso. Obviamente essas
probabilidades serão pequenas, sendo utilizadas constantes para
preencher essas possibilidades. Essas constantes podem utilizar taxas
esperadas de perdas de alvo, falso positivos ou intrusão por outro
equipamento, entretanto a determinação dos valores parece estar deixada
em aberto pelo autor.

O algoritmo calcula, então, a probabilidade máxima de observar um
estado após cada evento de transitório para todas observações das
cadeias. Ao chegar no final delas, é escolhido a configuração com
maior probabilidade e então se rastreia a sequência de transições no
sentido oposto até chegar no estado inicial dos sistemas (algoritmo de
\emph{Viterbi}\footnote{A última versão do código utilizou um modelo
adaptado para a cadeia de \emph{Markov} de segunda ordem.}). A fusão
para as duas cadeias de \emph{Markov} de um mesmo equipamento $i$ é feita
utilizando o conceito de máxima verossimilhança, empregando a
configuração de maior probabilidade final para esse equipamento.

Para descobrir os equipamentos e suas estatísticas, o algoritmo realiza a
coleta de dados por um longo período de tempo (ex. duas semanas) e,
em seguida, realiza o agrupamento dos dados por centroides (foi
utilizado \acs{isodata}, mas o próprio autor pretende empregar
outras técnicas). A fim de se corrigir possíveis problemas com os centroides,
são feitas operações de separação e agrupamento dos mesmos de acordo
com regras que indicam a necessidade das mesmas. Os centroides finais
positivos e negativos são casados com a mesma abordagem utilizada por
\citeauthor*{nilm_hart_1992_8} (ver o passo~3 desse autor na
pp.~\pageref{text:passo3}, limitando-se a parte para \gls{c3}). A
montagem das \glslink{cdf}{cdfs} com as características de tempo é
realizada através do modelo empírico --- constrói a \acs{cdf}
através dos dados amostrados, sem ajuste ---, enquanto para as
\glslink{pdf}{pdfs} de tempo e potência, ajusta-se a distribuição
através da mistura de duas componentes de distribuições de
Laplace\footnote{A primeira versão realizou o ajuste com duas
componentes Gaussianas.}.

Por fim, após o processo de levantamento de estatística, o
algoritmo é aplicado em tempo real, limitando a janela de dados para um
tamanho aceitável, citando como exemplo 1~dia. O tamanho de janela
pode ser ajustado experimentalmente. A base de dados utilizadas para a
estatística também é atualizada conforme mais dados são recolhidos.
Assim, o método não necessita de treinamento, identificando os modelos
de carga e suas propriedades. O autor considera depois comparar a
informação encontrada com as estatísticas de uso e consumo típicas
para identificar o equipamento, mas havendo comunicação com o usuário, é
possível que o próprio realize a identificação, dando a comparação
apenas como sugestão. Outra consideração é trabalhar com trios de
cargas ao invés de duplas, já que alguns casos essa configuração
parece ser mais realista devido à sobreposição de vizinhos aquém
daqueles imediatamente subsequentes, reduzindo a presença de intrusos
nas cadeias de \emph{Markov}.

O autor empregou dados simulados e reais para provar a eficácia de seu
algoritmo, onde foi comparada a capacidade do mesmo em relação a um
classificador \emph{Naïve Bayes}, obtendo resultados em termos de
\acs{medidaf} (\ref{eq:fmeasure}) na ordem de 0,83 a 0,97 em relação à
medição paralela, em geral superiores ao classificador de
\emph{Bayes}. Nos dados simulados, o autor demonstra que as outras
abordagens normalmente empregadas para 1~\acs{hz} são deficientes para
as configurações de sobreposição de consumo, enquanto sua versão
consegue reconstrução praticamente perfeita. Entretanto, atualmente o
algoritmo só atende equipamentos modelados por \gls{c3}.

Já no estudo mais recente
\cite{nilm_zeifman_statistical_naive_enduses_2013}, o autor não
empregou o \gls{vast}. Como nas abordagens \cite{nilm_farinaccio_16ssamp_1999_17,
nilm_marceau_16ssamp_improved_1999_18}, limitou-se a identificar os
equipamentos de maior uso-final. No caso, os equipamentos de interesse foram:
\begin{enumerate*}[label=\itshape\alph*\upshape)]
\item ar condicionado;
\item aquecedores de ambiente;
\item aquecimento de água doméstica;
\item \label{itm:iluminacao} iluminação;
\item geladeira;
\item secadores de roupa elétricos;
\item \label{itm:equipamentoeletronico} equipamentos eletrônicos;
\end{enumerate*} que representam em média 80\% do consumo residencial,
no caso, para os \gls{eua}. Nesse configuração, utilizou-se o conceito
de máxima entropia para selecionar modelos estatísticos mais adequados
para os dados. Uma vez que a média e alcance das variáveis são conhecidos,
as distribuições Beta são as mais indicadas por esse conceito. Foram
feitas considerações em relação a iluminação, que tem dependência em
relação aos ambientes de uso. Ainda, para melhorar a performance,
adiciona-se como característica as assinaturas específicas dos
equipamentos. O exemplo é a televisão em comparação com uma lâmpada, onde
aquele varia o seu consumo conforme flutuações na imagem e som,
enquanto este tem o consumo bastante estável. Essas
características ``finas'' dos equipamentos podem ser modeladas
matematicamente e empregadas em conjunto com o conhecimento prévio de
utilização dos equipamentos. Já para o caso dos secadores de roupa
elétricos (uma \gls{fsm}), foi utilizado duas distribuições Beta, uma
para cada estado.

Os autores utilizaram o classificador \emph{Naïve Bayes} com base nas
distribuições conjuntas resultantes obtidas para os sete uso-finais
indicados, adicionados de uma classe ``outros''.  Os resultados
empregaram a \acs{medidaf}, obtendo valores de acurácia para os
uso-finais mais desafiantes na ordem de 0,65 e 0,70, sendo os mesmos
os itens \ref{itm:iluminacao} e \ref{itm:equipamentoeletronico},
respectivamente. Já quando utilizando a informação de características
finais, essas mesmas acurácias se elevam para 0,92 e 0,90. Para os
outros equipamentos, os resultados ficaram na margem de 0,9, atingindo a
melhor marca para os aquecedores de água doméstica: 0,99.

\end{enumerate}

\subsubsection{2. Medição com Alta Amostragem}
\label{top:nilm_alta_am}

A abordagem para os \glspl{nilm} operando com aquisição de dados
obtidos com alta amostragem possibilita a obtenção de características
possivelmente mais discriminantes devido à alta resolução. A maior
granularidade na amostragem permite o acesso aos harmônicos e formas
de onda do sinal, chamadas de características microscópicas. Vale
ressaltar aqui que a utilização de taxas de amostragem alta no
sistema de aquisição não implica na extração de características
microscópicas. Por exemplo, ao extrair a diferença de consumo de
potência ativa entre o pré e pós transitório com 100 M\acs{hz},
isso não define essas características como microscópicas caso elas
possam ser obtidas com frequência.
 
Devido à maior amostragem, é inerente que os autores, em muitos casos,
buscaram empregar técnicas de processamento de sinais para o
tratamento dos dados, onde a bem conhecida \gls{fft} foi a primeira
abordagem. Outras abordagens vão além da \gls{fft}, procurando outras
maneiras de representar a informação, em alguns casos trabalhando até
mesmo com a informação crua, sem tratamento. A criatividade dos
autores leva a tona uma gama de características para descrever o sinal
e buscar através de diversas técnicas as assinaturas dos equipamentos. As
características e técnicas empregadas pelos diversos autores para essa
amostragem é tratada a seguir.

Antes disso, considerações quanto aos requerimentos mínimos de
amostragem deve ser realizada \cite{nilm_matthews_overview_2008_22}.
Quando explorando as características no domínio da frequência, o
teorema de \emph{Nyquist-Shannon} formula que a frequência de
amostragem deve ser no mínimo duas vezes superior à frequência do
maior harmônico a ser estudado. Já no domínio do tempo, uma regra de
boa-prática \cite{nilm_matthews_overview_2008_22} formula que a
amostragem deve ser no mínimo de cinco vezes a frequência de interesse
(no caso, o mínimo necessário para representar a fundamental no Brasil é
uma amostragem em 300~\acs{hz}). Entretanto, é preciso ter em
mente que certos distúrbios só poderão ser observados em frequências
maiores, em especial a caracterização dos regimes transitórios, sendo
necessários nesses casos frequências bem maiores. As frequências
utilizadas pelos autores variaram nas ordens de $\sim100$ \acs{hz} até
$\sim1$ M\acs{hz}.

\begin{enumerate}[label=\textbf{2.\arabic*},wide=\parindent]
\item \textbf{Harmônicos e \gls{fft}}
\label{nilm:harmonic_fft}

A decomposição através de \gls{fft} fornece informação além daquelas
obtidas por \gls{p} e \gls{q}, cuja única informação presente é das
componentes fundamentais e sua defasagem. A informação presente na
decomposição auxilia na discriminação de cargas não-lineares,
separando as mesmas entre si e das cargas lineares. Por outro lado, é
importante notar que a decomposição não fornece mais informação
discriminante além da informação já contida em \gls{p} e \gls{q} para
desagregar cargas lineares.

\begin{itemize}[wide=\parindent]
\item \emph{Harmônicos para identificação de \acs{c4}}

O grupo que trabalhou com o intuito de levar o \gls{nilm} para o setor
comercial e industrial \cite{nilm_norford_leeb_medianfilt_1996_13,
nilm_leeb_spectral_envelope_1995_23} utilizava amostragem em
frequências superiores a 8~k\acs{hz} para construir o envelope
espectral do sinal \cite{nilm_laughman_continuous_variables_2003_9}. O
grupo se referia dessa maneira ao envelope pois ele era construído
através da resposta da \gls{fft}. Como foi dito durante o levante
desses trabalhos, o grupo subdividia os transitórios em seções-v que
eram utilizadas para classificar os equipamentos. 

Uma vez que a decomposição harmônica já estava sendo extraída, os
trabalhos posteriores procuraram empregar essa informação para tratar
um problema presente em maior escala nesses setores, as \glspl{c4}.
Os trabalhos de \citet{nilm_lee_variable_speed_estimation_2005_24,
nilm_wichakool_2009_25,nilm_shaw_2008_26} avaliaram como identificar
possíveis maneiras de reconstruir o consumo de cargas variáveis
em cenários previamente conhecidos. Um dos cenários incluiu inclusive
a aplicação do \gls{nilm} em um carro \cite{nilm_shaw_2008_26}.
Foi observado nas \glspl{c4} estudadas a presença de correlação entre
sua demanda e os harmônicos de potência aparente. Assim, é ajustado
uma função polinomial utilizando \gls{mse} para determinar a sua
mudança de consumo. A função polinomial é utilizada e comparada com as
envelope espectral das seções-v para determinar se a mudança foi
causada por alterações no estado operativo da \gls{c4}.

Apesar do método se comportar bem para as aplicações propostas,
há a necessidade de fazer o levantamento caso a caso, sendo
interessante apenas para a aplicação no setor comercial e industrial
de grande porte, onde dificilmente será possível aplicar um \gls{nilm}
genérico devido as diversas peculiaridades das cargas presentes. Além
disso, não se sabe como a presença de outros equipamentos irão afetar a
correlação, podendo limitar sua aplicação.

\item \emph{Extração harmônica ciclo a ciclo}
\label{nilm:harmonico_ciclo_ciclo}

A abordagem de \citet{nilm_srinivasan_nn_2006_27} alterou a estratégia
do \gls{nilm} ao não realizar a extração dos eventos de transitório.
Ao invés disso, ele realiza diretamente para cada ciclo a desagregação
da informação de consumo por equipamento. 

Para isso, utilizou-se técnicas de aprendizado
de máquina (\glspl{rna} com as arquiteturas \acs{mlp} e
\acs{rbf}, além de \acs{svm}) treinadas com a informação das
componentes harmônicas impares até o 15$^o$ harmônico. O caso
analisado continha oito equipamentos. Em cada possível combinação
operativa desses equipamentos ($2^{N_{ap}}=256$), obtiveram-se dados para
treinar e testar as técnicas aplicadas. Nessa configuração, o método
obteve reconstrução praticamente perfeita para a \gls{mlp} e
\gls{rbf}. Em seguida, para melhorar a aplicabilidade dos métodos,
utilizou-se as amostras coletadas com apenas um dos equipamentos ligados
por vez que foram, então, somadas para reproduzir os outros estados
operativos. Nessas condições, houve deterioração para quatro
equipamentos para taxas em torno de 70\%, enquanto os outros equipamentos
continuaram com taxas em próximas a 100\%, tanto para a \gls{mlp} e
\gls{rbf} (com algumas discrepâncias entre as duas). Também se testou
condições com 10 equipamentos. Para esse caso, como seriam necessárias
medições em 1024 situações, utilizou-se o método de obter apenas as
operações de cada equipamento sozinho, e somá-las para obter as outras
configurações. Outras configurações testadas incluíram adição de
ruído e teste para rede com topologia trifásica, onde se apresentou
que o método é robusto para essas configurações. O \gls{svm} não teve
respostas compatíveis com as da \gls{mlp} e \gls{rbf} para as diversas
configurações testadas.

Todavia, conforme o número de equipamentos cresce, a quantidade de
combinações necessárias para treinar a rede cresce exponencialmente.
Uma residência com cerca de 30 a 50 equipamentos (valores típicos
citados por \cite{nilm_zeifman_review_2011}) necessitariam do
treinamento para números na ordem de $10^7$ a $10^{15}$ configurações
--- isso sem considerar possíveis estados que serão adicionados por
equipamentos \glspl{c2}  ---, onde possivelmente diversas dessas configurações
seriam similares e, por isso, o método não conseguiria reproduzir as
capacidades para os exemplos citados. Além disso, é necessário saber
\emph{a priori} todos equipamentos pertencentes na residencia, assim como
o autor não levantou como as respostas seriam afetadas devido à
presença de equipamentos não-conhecidos. \textlabel{Mais adiante, também
é necessário estimar o consumo por 
equipamento}{text:transf_info_discr_energia}, assunto deixado em aberto
pelo autor.  Para isso, é possível realizar a identificação de
transitórios em cima da informação de discriminação, como considerando
uma janela para a qual se o equipamento mudar de estado operativo por um
tempo maior àquele da janela, então se confirma a mudança operativa e
calcula o consumo desse equipamento no novo estágio através da diferença
pós e pré-transitório. Possíveis adaptações podem ser feitas para
obter melhor resolução em energia. Uma abordagem mais elegante seria
realizar as duas etapas em paralelo e comparar as informações obtidas
em cada uma para validar se houve realmente a mudança de estado. Já
com o intuito de tornar essa tarefa trivial, pode-se evitar a
utilização da Etapa~\ref{itm:etapa2} e simplesmente empregar um valor
de consumo estimado para cada equipamento.

\item \emph{Ruído espectral de tensão}
\label{nilm:emi}

\citet*{nilm_patel_2007_29,nilm_gupta_patel_2010_30} encontraram
características com capacidade discriminantes \emph{sui generis}. A
metodologia utilizada permite diferenciar equipamentos iguais mas em
diferentes cômodos. De acordo com os autores, em alguns casos é
possível identificar qual interruptor de uma lâmpada \emph{three-way}
teve seu estado alterado. Para isso, ao invés de utilizar o consumo
agregado, os autores mudaram o conceito de como realizar a aquisição
de dados. No caso, seu sistema pode ser conectado em qualquer tomada
da residência, aonde serão analisados a \gls{emi} causado por
chaveamentos, sendo necessário medir somente a \emph{tensão}. 

As características internas do equipamento, bem como a configuração da
rede elétrica entre o equipamento e o medidor, alteram em como o ruído
será observado. Os ruídos podem ser gerados continuamente ou apenas
durante os eventos de transitório. Exemplos de equipamentos que geram
ruídos contínuos são: equipamentos com funcionamentos a base de motores
(como secadores de cabelo, ventiladores) que irão gerar ruídos de
tensão em frequências síncronas àquelas da rede causados pelo
chaveamento das escovas dos motores (os equipamentos desse tipo que não
tiverem escova, não vão produzir \emph{emi} continuamente); e os
equipamentos eletrônicos com fontes baseadas a chaveamento, nesse caso
gerando ruído síncrono ao oscilador de sua fonte. Já eventos de ruído
\gls{emi} podem ser observados ao colocar um equipamento na tomada,
alterar o estado de um interruptor ou ligar uma televisão.

Na primeira abordagem \cite{nilm_patel_2007_29}, o objetivo foi
encontrar os ruídos transitórios de chaveamentos na rede. São gerados
2048 pontos igualmente espaçados com os resultados da decomposição da
\gls{fft} (apenas magnitude) no alcance de 0 a 50 k\acs{hz}. Essa
informação é repassada para uma janela deslizante que calcula a média
para 1 $\mu$s de dados. Com os resultados das janelas, é tirada a
distância Euclidiana em relação a janela anterior, e se esse valor
superar um limiar considera o início de um transitório. O transitório
será registrado até a distância Euclidiana ter outra mudança drástica.
A resolução de 1 $\mu$s é importante pois alguns transitórios tem
duração de apenas alguns poucos $\mu$s. Tendo o transitório isolado,
obtém-se a média para os 2048 pontos em relação ao número de amostras
temporais. Essa informação alimenta uma \acs{svm}. A capacidade
de isolamento de transitórios (taxa de detecção apenas para a
Etapa~\ref{itm:etapa2}) obtida foi entre 88\% a 98\%, embora a taxa de
falsos positivos não tenha sido informada. Os acertos para a
quando considerando a Etapa~\ref{itm:etapa3} ficaram entre 85-90\%
quando a \acs{svm} é treinada com 5 instâncias para cada evento.
Caso seja coletado apenas 2 instâncias por evento, há uma deterioração
para cerca de 80\%.

Já a segunda abordagem tratou da abordagem para ruídos gerados
continuamente na rede \cite{nilm_gupta_patel_2010_30}. Os autores
perceberam que os transitórios contínuos não são dependentes da
configuração da rede, onde os equipamentos terão as mesmas
características independente das residenciais e tomadas em questão.
Nesse caso, o espectro analisado de \emph{Fourier} é de
36-500k\acs{hz} para a mesma quantidade de pontos igualmente
espaçados. A mesma ideia de janela móvel e corte através de um patamar
é realizada para essa abordagem, no entanto, a frequência de geração
de vetores é de 244 \acs{hz}, sendo utilizados 25 deles para
calcular a média da janela deslizante (a frequência resultante é bem
menor que aquela utilizada para estudar eventos de transitório). 
A extração de características é realizada em cima da diferença entre o
espectro do ruído atual e anterior, aonde são ajustados uma Gaussiana.
As características são a média, amplitude e variância, que são
comparadas utilizando a técnica de vizinho mais próximo. A acurácia 
reportada é de 91,75\%.

Apesar das boas capacidades, ainda há uma série de dúvidas quanto a
capacidade de aplicar essa metodologia nos \glspl{nilm}. Em
\cite{nilm_gupta_patel_2010_30}, os autores levantam uma série de
limitações quanto à abordagem por detecção de transitórios, devido ao
custo computacional necessário para avaliar ruído transitório, a
necessidade de treinar para cada eletrodoméstico, a dependência das
características na configuração da rede elétrica do consumidor e os
eventos tem deposição de energia pequena devido a seu longo espectro
de frequência. Percebe-se que a grande sensibilidade das
características presentes nos transitórios traz um problema inverso
àqueles carregados pelas \glspl{c6}: ao simplesmente trocar
o equipamento de tomada altera como o mesmo é visto pelo \gls{nilm}.
Como levantado por \cite{nilm_zeifman_review_2011}, a segunda
abordagem é mais robusta, no entanto, diversas questões ficam em
aberto. Ela é limitada para equipamentos que tenham chaveamento, por
exemplo, fornos e secadores de roupas elétricos não puderam ser
identificados. Os autores não consideraram como a presença de
\gls{emi} nos vizinhos, em especial para o caso de apartamentos, irá
interferir no processo de detecção. Na construção do catálogo, é
necessário considerar todas as mudanças de fontes chaveadas que forem
realizadas pelos fabricantes, assim como as diferenças entre as fontes
utilizadas por eles. A sobreposição de assinaturas \gls{emi} também
irá ocorrer, sendo necessário tratar esse problema. Enfim, os
autores apenas se preocuparam em identificar os equipamentos, não sendo
obtido o consumo desagregado dos mesmos.

\item \emph{Fourier como função de ajuste para Vizinho mais Próximo}

Os trabalhos de \cite{nilm_berges_2008_7,nilm_berges_2009_36,
2010_nilm_melhorando_pph_usa_37} buscaram apenas reproduzir as
eficácias dos \glspl{nilm} presentes na literatura, não tendo como
objetivo apresentar melhorarias para as técnicas presentes nos
\gls{nilm}, mas apenas fornecer essa tecnologia para o consumidor
ter acesso em baixo custo para uma ferramenta que o permita
compreender suas despesas com energia. O projeto começou desde o
sistema de aquisição de dados, onde se observou problemas com
medidores disponíveis no mercado estadunidense que apresentavam
discrepâncias de 10-20\% entre si, enquanto a taxa reportada era de
3\%. O seu sistema de aquisição de dados atual coleta amostras na
frequência de 100 k\acs{hz} utilizados para gerar características
em 20 \acs{hz}. Devido às dificuldades para rotular os momentos
exatos para o treinamento das técnicas, eles criaram um detector
\emph{wireless} de eventos de transitório, diretamente colocado no
cabo do equipamento.

Ele utiliza um modelo probabilístico para a detecção de eventos
alimentando um discriminador de vizinho mais próximo utilizando função
de ajuste como a \gls{fft} na sua melhor configuração. As eficiências
de detecção obtida estão em torno de 67\% e 100\%. A discrepância em
energia do \gls{nilm} para o valor total de energia foi de 14,8\%.

\end{itemize}

\item \textbf{\gls{tw}}
\label{nilm:wavelet}

Uma das diferenças mais marcantes entre a \gls{fft} e a \gls{tw} é a
possibilidade que a última oferece de resolução tanto de frequência
como temporal. Para isso, a \gls{tw} reduz o tamanho da janela
conforme há o aumento das frequências para permitir resolução
temporal. A decomposição é realizada em funções chamadas
\emph{wavelets} ao invés de senoides. A \gls{tw} tem diversas
aplicações em Sistemas de Potências, o levantamento realizado por
\cite{wavelet_overview} em 2002 mostra que as aplicações mais comuns
eram proteção, qualidade de energia e detecção de transitório. A
larga aplicação da \gls{tw} em detecção de transitório em sistema de
potência mostra que elas podem ser utilizadas para, no mínimo,
realizar a Etapa~\ref{itm:etapa2} do \gls{nilm}, sendo um substituto
potencial para as métricas simples adicionados de um corte linear
comumente aplicadas pelos \glspl{nilm} nessa etapa.

Indo além, o levantamento de \cite{nilm_zeifman_review_2011} encontrou
um estudo (\citet*{nilm_chan_2000_31}) realizado em
\citeyear{nilm_chan_2000_31} que mostrou a capacidade
de aplicação da \gls{tw} para a identificação de cargas não-lineares.
Entretanto, esse estudo não se aprofundou no tema, apenas levantando a
capacidade da \gls{tw} de diferenciar três equipamentos, seguindo uma
ideia similar àquela explicada para
\citeauthor{nilm_srinivasan_nn_2006_27}\footnote{No caso, o estudo de
\citeauthor{nilm_chan_2000_31} foi realizado depois.}. Para isso, 
explorou-se os níveis de detalhes da \gls{tw} discreta para todas as
combinações operativas dos três equipamentos, salientando que essa
informação era discriminante e seria possível aplicar \acs{rna}
para discriminar essa informação, ainda que isso não tenha sido
realizado.

Um trabalho no Brasil, realizado por \citet*{nilm_itajuba_rodrigues},
seguiu exatamente essa metodologia. A taxa de amostragem empregada foi
de 15.360 k\acs{hz}, onde se testou duas configurações, uma aonde
os equipamentos elétricos operavam isolados na rede e outra com as
diversas combinações operativas. Os testes foram feitos com três e
quatro equipamentos e obtiveram capacidade de identificação de 100\%. Em
seguida, o trabalho mostrou a importância da \gls{tw} quando em
comparação com a \gls{fft}, onde a \gls{fft} obteve 90\% mesmo
utilizando 16 componentes do espectro (linearmente espaçado) de
frequência, enquanto a \gls{tw} foi decomposta para 7 níveis de
detalhes. Entretanto, essa abordagem sofre das mesmas desvantagens
referidas para o trabalho de \citeauthor*{nilm_srinivasan_nn_2006_27}
(ver pp.~\pageref{nilm:harmonico_ciclo_ciclo}).

Apesar de no exterior ter sido encontrado apenas um estudo preliminar
mostrando a possibilidade de utilizar \gls{tw}
\cite{nilm_chan_2000_31}, um outro trabalho brasileiro também seguiu o
caminho do emprego de \gls{tw} para a identificação de equipamentos. O
trabalho de \citet{nilm_coppe_nascimento} utilizou uma amostragem em
256~\acs{hz}. É empregado um detector de eventos onde se calcula
o incremento de corrente em relação a amostra passada, no caso de o
valor ultrapassar o patamar de 32 m\acs{a}, é criado um evento de
transitório com 8 ciclos e um total de 2048 pontos a serem analisados.

Uma pré-classificação é realizada utilizando o \emph{hardware} para
equipamentos com valores específicos de \acs{dp} e \acs{dq}. Caso não
seja um dos equipamentos de simples discriminação, modifica-se a
informação em uma série de etapas, nesta ordem: aplica-se Transformada
de \emph{Hilbert} no módulo da corrente, \gls{tw} com cinco níveis e
por último o Método de Burg, esse adicionado para suavizar e
sintetizar a informação contida nos níveis de detalhes obtidos na
decomposição do sinal pela \gls{tw}. As características utilizadas são
os picos para os níveis de detalhes do Método de Burg adicionados do
\gls{fp}.

A discriminação é feita em duas etapas, primeiro se seleciona um grupo
genérico ao qual o equipamento pertence, para então selecionar o grupo
específico. A seleção é feita através do vizinho mais próximo com
métrica Euclidiana entre o valor no banco de dados e o resultado do
espectro de Burg para os níveis de detalhe da \gls{tw}. É realizada
uma votação (em cada etapa) para cada nível de detalhe, o equipamento
que tiver maior votação é o resultado da etapa do processo de
discriminação.

Os resultados para acionamentos individuais reportados pelo autor
foram de 99,7\% para a seleção na primeira etapa e 88,8\% na segunda.
Foi realizado um ensaio para verificar o comportamento do algoritmo em
casos empilhados, porém o mesmo foi um teste inicial,
adicionando uma lâmpada incandescente --- sem dinâmica de consumo --- 
antes de acionar os outros equipamentos.  Assim, não é possível
deduzir a capacidade do método para a aplicação em condições reais,
onde é necessário a robustez das técnicas aplicadas para ao
acúmulo de diversos equipamentos, injetando ruídos e dificultando
a desagregação da informação. O autor identificou, também, que é
necessário trabalhar no detector de eventos transitórios, que talvez
tenham sido o motivo para a deterioração da eficácia dos motores no
teste empregado. Ainda, as condições operativas normais em uma
residência podem revelar a presença de falsos positivos, sendo também
um levantamento necessário pelo trabalho. Por fim, os autores
preocuparam-se apenas com a capacidade de identificação dos equipamentos,
sendo necessário, pela perspectiva de aplicação do \gls{nilm}, ainda o
passo de transformar essa informação qualitativamente para termos de
consumo.

\item \textbf{Curvas I-V}
\label{nilm:curvas_iv}

As características geométricas das formas de onda para diversos
equipamentos foi explorado pelos estudos de
\citet{nilm_lee_2004_32,nilm_lam_2007_33}. Conforme
\cite{nilm_zeifman_review_2011}, a contribuição foi a investigação de
uma nova características, as curvas I-V\footnote{Provavelmente os
leitores já tiveram a oportunidade de observar essas curvas no
osciloscópio, caso oposto, diversas curvas podem ser observadas 
na referência \cite{nilm_lam_2007_33}.}. As curvas são formadas ao
distribuir as amostragens temporais de corrente contra tensão no plano
cartesiano. No caso de cargas lineares, elas formam elipses, podendo
transformar-se em uma reta caso o equipamento seja puramente resistivo.
Nesse caso, o sentido de rotação da curva dependente se a carga é
indutiva ou capacitiva, mais precisamente estando relacionada ao
ângulo de carga. De acordo com a presença de harmônicos, pode ocorrer
distorções nas elipses de modo que as mesmas enlacem a si mesmas em n
pontos de interseção.  Assim, o estudo propôs diversas características
a serem extraídas dessas curvas, cada uma representando
informações que permitam discernir as características dos equipamentos
entre si.

São utilizados os valores normalizados para que todos os equipamentos
tenham a mesma escala, removendo as discrepâncias causadas por
equipamentos de diferentes marcas. A fim de estudar a capacidade
discriminativa, a metodologia empregada foi a geração de dendrograma
para compreender a taxonomia resultante e compará-la com aquelas
normalmente obtidas quando utilizando variáveis padrão, no caso o
autor considerou: valor eficaz da corrente em um ciclo; potência
reativa; \gls{fp}; número total de harmônicos impares; número total de
harmônicos pares.  O estudo também comparou com a taxonomia obtida ao
realizar \gls{svd} da corrente em um período. Os
primeiros 24 autovetores foram utilizados como representação da
informação, contendo 99\% da energia, para serem formados os
agrupamentos.

Os resultados da separação taxonômica está na
Tabela~\ref{tab:taxonomias_lam}, (ver pp.~\pageref{tab:taxonomias_lam},
estando mais adiante simplesmente para agrupar as taxonomias com as do
\gls{cepel} em um único lugar neste trabalho).  Constatou-se a
presença de grupos mais representativos para as curvas I-V em relação
aos outros dois casos, obtendo grupos bem definidos para os tipos
resistivos, motores e eletrônicos. A representação com variáveis
padrão teve dificuldade em distinguir entre os equipamentos eletrônicos.
Ocorreu também a união de equipamentos resistivos, eletrônicos e
equipamentos de operação através de motores. Já no caso da decomposição,
os aglomerados formados permitiram a classificação dos equipamentos em
relação ao formato de suas ondas, no entanto, não é possível inferir
como a afinidade dos equipamentos foi gerada.

\item \textbf{Ondas sem processamento}

A metodologia empregada por \citet{nilm_suzuki_2011_35} também opera
ciclo a ciclo na rede, como o caso das abordagens de
\cite{nilm_srinivasan_nn_2006_27,nilm_itajuba_rodrigues}, mas
alterando a lógica quando em comparação com esses outros métodos. Ao
tratar o problema como sendo de otimização, foi resolvido o caso 
do paradoxo da quantidade de configurações para o treinamento das
técnicas supervisionada. O próprio autor observa que outros autores
japoneses tentaram resolver o problema da mesma maneira --- através de
redes neurais --- ressaltando a quantidade de dados necessários por
essa abordagem. A técnica consiste-se em ajustar a melhor configuração
das ondas cruas de corrente --- sem preprocessamento --- que se
justaponha àquela observada pela medição centralizada. Isso é
realizado através de Programação por Inteiros (tradução própria de
\emph{Integer Programming}), modelando tanto \gls{c1}, \gls{c2} quanto
\gls{c3}, desde que se saiba da existência dos mesmos \emph{a priori}.
São obtidos 666 pontos por ciclo, resultando em uma frequência de
amostragem de 40 k\acs{hz}.

A abordagem foi avaliada em uma residência, onde os residentes
realizaram suas tarefas normalmente, mas anotando os momentos em que
eles utilizavam os equipamentos durante 6 dias. Os resultados mostraram
que a fase~2 com 7 equipamentos obteve taxas de detecção na ordem de
96,8\%, enquanto a fase~1 com 15 equipamentos, sendo eles diversas
lâmpadas, obteve para essa acurácia, valor de 79,0\%. É esperado para
esse método a mesma dificuldade operativa das abordagens
\cite{nilm_srinivasan_nn_2006_27,nilm_itajuba_rodrigues}, já que a
quantidade de configurações possíveis cresce exponencialmente com o
número de estados possíveis nos equipamentos da residência. Como o
próprio autor observa, os erros na fase~2 foram causados por
configurações onde os estados de consumo agregado são semelhantes, no
caso ``Torradeira + Secador (Apenas Ventilação)'' $\approx$ ``Secador
de Cabelo (Com aquecimento no fraco)''. Segundo o crescimento de
equipamentos, é esperado que a ocorrência desses casos se torne maior. Os
autores planejam trabalhar em uma contramedida para possibilitar a
modelagem de \gls{c4}, bem como conseguir uma maneira automática para
unir equipamentos similares. Para isso, aqui se nota que utilizar um
patamar de corte em \ref{eq:similaridade} parece ser suficiente. Outra
questão ressaltada por \cite{nilm_zeifman_review_2011} é que a
utilização das ondas cruas pode ter reduzido a eficácia da abordagem,
considerando que a extração através de outras representações da mesma
são geralmente mais robustas. Por outro lado, nada impede o emprego de
Programação por Inteiros em cima de outras representações.

\item \textbf{Aplicação de múltiplas técnicas}
\label{nilm:multiplas_tecnicas}

Os estudos \citet*{nilm_liang_pt1_2010_34,nilm_liang_pt2_2010_40}
trataram o tema de desagregação do consumo de maneira metódica,
definindo matematicamente diversas grandezas que muitos autores
tentaram exprimir por palavras. No entanto, a maior contribuição foi
mostrar como combinar as técnicas presentes na vasta extensão de
abordagens já realizadas ao tema. A primeira questão já foi
explicitada na Subseção~\ref{ssec:modelos_carga} quando se referindo
às \glspl{c6}, sendo o ponto de similaridade entre as características.
A seleção de características que não apresentem similaridade para os
mesmos equipamentos possibilita que as técnicas de discriminação possam
obter padrões relevantes para a solução do problema. O mesmo pode ser
dito quanto a capacidade de complementação das técnicas empregadas na
Etapa~\ref{itm:etapa3}, os autores propuseram a \gls{cr} em
\ref{eq:cr}, sendo $Z_a$ e $Z_b$ dois discriminadores quaisquer
empregados.

\begin{equation}\label{eq:cr}
CR_{z_a,z_b} = \dfrac{P\{Z_a=false|Z_b=true\}}{P\{Z_a=true\}}
\end{equation}

Naturalmente, essa abordagem requer mais um subpasso para combinar as
respostas dos diversos discriminadores para produzir uma resposta
única. Essa técnica é um problema conhecido tratado
através pela ideia de um \gls{cdm}, que dita as regras de como será
tratado o valor de cada decisão para realizar a fusão de informação.
\cite{nilm_liang_pt1_2010_34}, levanta algumas possibilidades:

\begin{description}
\item \textlabel{\gls{mco}}{text:uniao_tecnicas}: escolha do candidato
mais comum entre os membros da comissão. É o processo mais trivial de
ser executada computacionalmente, realizando apenas a contagem de
votos. Essa abordagem pode criar soluções não-únicas devido ao empate
na votação;
\item \gls{lur}: seleciona o melhor candidato através da escolha
daquele que obtém o menor valor \gls{lur}. Para isso, são definidos o
\gls{ir}, onde $y_{(k|j)}$ representa a característica desconhecida,
que é comparada com a assinatura conhecida
$\hat{y}_{(k|(i,j))}$. A grandeza \gls{ur} é representada por
\ref{eq:ur}, onde $M$ é o número de candidatos sendo consideradas.
E, finalmente, o \gls{lur} sendo definido por \ref{eq:lur}, onde
$\Theta$ é a população de candidados.

\begin{subequations}
\begin{equation}\label{eq:ir}
IR_{(i,j)} = \dfrac{
\left(\sum^N_{k=1}y_{(k|j)}-\hat{y}_{(k|(i,j))}\right)^2}{
\sum^N_{k=1}y^2_{(k|j)}}
\end{equation}
\begin{equation}\label{eq:ur}
UR_i = \prod_{j=1}^M IR_{(i,j)}
\end{equation}
\begin{equation}\label{eq:lur}
LUR = min \left(UR_i |\forall i \in \Theta\right)
\end{equation}
\end{subequations}

O \gls{lur} considera assim, qual candidato tem o menor resíduo em
relação às decisões tomadas pelas diversas técnicas. Esse \gls{cdm} é
mais pesado computacionalmente quando comparado com \gls{mco}, por
outro lado, sempre permite soluções, e frequentemente únicas.

\item \gls{mle}: outra maneira é utilizar o levantamento de
estatística. Ao simular diversas condições com casos conhecidos, é
possível determinar a resposta mais provável para a k-ésima
característica utilizando a j-ésima técnica, sendo ela
determinada pela função $K(l,j)$. Assim, calcula-se a probabilidade
marginal de ter $ib$ como resposta verdadeira enquanto a j-ésima
técnica com a k-ésima característica dão como candidato a resposta
$i_a$ ($\theta_{ia}$). Um conjunto de verossimilhanças é dado por
\begin{equation}
\rho(ib) = \prod_{j=1}^M\prod_{l=1}^L
p(\theta_{ia},ib,j,l|ia=K(l,j),\forall i_b)
\end{equation}
onde M e L são o número de características e técnicas,
respectivamente. A máxima verossimilhança pode ser determinada através de 
\begin{equation}
\Lambda(\rho)=max\{\rho(ib)|\forall ib\in\Theta\}
\end{equation}
A \gls{mle} é o método mais intensivo computacionalmente dos citados
pelo autor.
\end{description}

Já \cite{nilm_zeifman_review_2011} propõe a utilização de
técnicas adaptando a teoria de \emph{Dempster-Shafer} para realizar
tal tarefa, e cita \cite{information_fusion_basir_2007_40} como
exemplo. Outra abordagem possível é o emprego de uma \gls{rna}.

Em \cite{nilm_liang_pt2_2010_40} os autores criaram um simulador de
Monte-Carlo, capaz de gerar situações de operação simultânea de equipamentos,
ruído devido a dinâmica de carga (\gls{c5}) ou fontes externas de
ruído, dando uma capacidade maior para as possibilidades dos
testes das técnicas a serem empregadas pelos \gls{nilm}. Com base nos
dados simuladores por esse gerador, eles mostraram que o uso de
múltiplas características e múltiplas técnicas é benéfico para o
desagregador, aumentando a eficiência em cerca de 10\% quando
comparado com a melhor resultado de qualquer outra técnica quando
aplicada em isolado, obtendo valores na ordem de 90\% para a
\acs{class_eff}. A facilidade do simulador também permitiu uma
série de outros levantamentos:

\begin{enumerate}[label=\itshape\alph*\upshape)]
\item as \gls{cdm} são tanto incrementais em peso computacional quanto
em performance, onde a \gls{mle} obteve a melhor performance para
todos os casos simulados. A \gls{lur}, por outro lado, apresenta
praticamente o mesmo resultado, sendo talvez uma abordagem com melhor
custo benefício computacional;
\item a opção com múltiplas técnicas se apresentou superior a
ruído que as opções singulares;
\item as simulações para o ar condicionado mostraram que sua presença
deteriora a eficácia do desagregador;
\item há deterioração conforme a maior demanda na rede, bem como uma
presença maior de equipamentos operando;
\item quanto maior for o incremento de potência em relação à demanda
da rede, maior será a facilidade do \gls{nilm} desagregar essa
informação; e
\item a análise por similaridade permitiu identificar os equipamentos com
maior confusão.
\end{enumerate}

O autor de \cite{nilm_zeifman_review_2011}, motivado pelas
considerações de \cite{nilm_liang_pt2_2010_40}, realizou o estudo
\cite{nilm_zeifman_vast_hisample_pdfmerge_2011} mostrando a intensão
de unir o seu método para baixas amostragens 
\cite{nilm_zeifman_vast_2011} com a combinação de múltiplos métodos
para altas amostragens. Baseado nos estudos de seu levantamento
bibliográfico, o autor utilizou amostragens de 500~k\acs{hz} para
trabalhar com análise de transitórios. O detector de transitório
observa por mudanças estatisticamente significantes na forma de onda,
que ao serem encontradas causam a coleta de diversas formas de onda
antes e depois da mudança. Realiza-se a subtração entre o valor pós e
pré transitório para a geração da diferença no formato da onda
decorrente do transitório. Essa informação alimenta duas técnicas para
discriminação: uma utilizando os componentes harmônicos e outra os
dados crus para alimentar uma técnica multivariável. Calcula-se o
valor de \acs{pdf} para os dados de entrada em relações às
\acs{pdf}s obtidas experimentalmente para todos os equipamentos. A
classificação final de cada algoritmo é o valor máximo de
\acs{pdf}.

Um teste simples foi realizado para comparar os dois métodos. Foram
alterados estados operativos de quatro equipamentos, obtendo desagregação
perfeita para o caso das ondas cruas, e um erro para a decomposição
harmônica. O método combinado também obteve reconstrução perfeita.
Em seguida, coletaram-se dados para 8 horas com os quatro equipamentos,
obtendo novamente 1 erro para a decomposição harmônica e 100\% para os
outros dois casos. O autor retrata que seria ainda possível combinar
essa informação com o \gls{vast}, que também apresenta saída
probabilística. No entanto, parece que o autor descontinuou o projeto
em virtude da melhor capacidade prática do método desenvolvido para
baixas frequências que tem se mostrado capaz de cumprir com as
necessidades de projeto para o \gls{nilm} ser aplicado em um programa
de \gls{ee}.

\end{enumerate}

\subsubsection{3. Utilização de outros sensores}
\label{top:seminilm}

Diferente das propostas anteriores, alguns autores consideraram uma
abordagem não centralizada. A ideia é empregar outros sensores aquém
do medidor central na alimentação da residência para permitir maior
capacidade de desagregação, seja por eles já estarem disponíveis, ou,
em alguns casos, através de sua instalação. Mesmo que esses sensores
já estejam disponíveis, dificilmente se terá acesso a sua informação
sem intrusão da propriedade, constituindo assim uma abordagem
intrusiva na maioria dos casos.

Já se referiu a um caso onde os autores consideraram a utilização de
outros sensores. A abordagem de
\cite{nilm_norford_leeb_medianfilt_1996_13} para o setor comercial foi
bem sucedida ao empregar variáveis de controle para possibilitar a
obtenção do consumo de equipamentos \gls{c4}. 

Em \cite{seminilm_berges_multisensor_2010}, os autores mostram a
intensão de seguir uma nova abordagem. A ideia é utilizar a informação
de múltiplos sensores no ambiente, como iluminação, som etc. e
correlacionar a informação por eles obtidas automaticamente com o
consumo de equipamentos, permitindo, por exemplo, desagregar o consumo de
uma lâmpada na cozinha com a de um banheiro devido à um sensor de
iluminação nesse ambiente.

Um projeto na Suíça por \cite{seminilm_ihome_tomek_2012}, utiliza
diversos submedidores de energia conectados via \emph{wireless} a uma
central. A ideia é reduzir o acumulo de equipamentos no sinal a ser
desagregado ao colocar os medidores mais próximos ao consumo,
facilitando assim a identificação dos equipamentos. Cada módulo de
submedição realiza medições em 3,2~k\acs{hz} de tensão e corrente, que
são utilizados para gerar a \acl{p}, \acl{q} e os harmônicos impares
de corrente com índices de 1-11. Quando é detectado um evento de
transitório, a informação é repassada para uma central, que procura no
catálogo o equipamento que melhor se corresponde com o transitório.
Reporta-se eficiência de detecção de 95\% para submedidores com um
único equipamento. 

Uma outra abordagem bastante interessante realizada por
\citet*{seminilm_fhmm_empiricalnmeter_2013}, altera o conceito
para uma perspectiva semi-intrusiva, onde a intenção é
reduzir a necessidade monetária associada com custos em
sensores fazendo a seleção ótima do número de sensores ao mesmo tempo
que se maximiza a capacidade em desagregar a informação. Para isso,
eles criaram uma regra heurística para encontrar o conjunto ótimo de
equipamentos a serem monitorados por submedição. Os dados foram obtidos
na frequência de amostragem de 1~\acs{hz} para o medidor central,
e $\frac{1}{3}$~\acs{hz} para os medidores por equipamento. Essa
informação foi reduzida para uma taxa de 1 amostra a cada 20 segundos
através de um filtro de mediana. Também se utiliza o
algoritmo de \emph{Viterbi} para modelar as \gls{fsm}, no caso
utilizando \gls{hmm}. Conseguiu-se experimentalmente eficiências na
ordem de 95\% por equipamento em termos de \acs{medidafenergia}
(\ref{eq:fmeasure_en}). Em um trabalho futuro, os autores também
desejam utilizar outros sensores no ambiente, como em
\cite{seminilm_berges_multisensor_2010}.

\subsection{Discussão}
\label{ssec:nilm_discussao}

A primeira questão que chama atenção no levantamento bibliográfico é o
quesito de escala. O \gls{nilm} deve ser robusto para a aplicação
independente da quantidade de equipamentos disponíveis na residência do
consumidor. Naturalmente irá ocorrer a presença de equipamentos
desconhecidos, bem como de \gls{c5} que, conforme o acúmulo de
diversos equipamentos, irá tornar a rede um ambiente muito mais complexo
para operação das técnicas. É justamente nesse ambiente que as
abordagens devem ser analisadas, demonstrando como será o seu
comportamento nas condições adversas presentes nas redes elétricas
residenciais.

Geralmente os \glspl{nilm} utilizam eventos de transitório para buscar
por alterações no estado operativo dos equipamentos na rede ---
Etapa~\ref{itm:etapa2}. Essa estratégia reduz a análise de informação
pelo discriminador, uma vez que a busca por padrões só é realizada
quando distúrbios são encontrados na rede. Apesar disso, as técnicas
empregadas para encontrar os eventos devem ser capazes de separar
distúrbios causados por ruídos daqueles causados pelos equipamentos, uma
vez que a eficiência do \gls{nilm} será diretamente afetada pela sua
capacidade. Isso também evidência que os discriminadores precisam ser
robustos à presença de intrusos, tanto causado por ruídos ou por
equipamentos desconhecidos. Poucos estudos trataram a capacidade do
detector de transitório ou como um transitório falso é tratado pelo
discriminador.

Outra capacidade desejada nos \glspl{nilm} é a criação dos modelos
de consumo dos equipamentos de maneira automática. Diversos
autores fizeram mão de modificações do algoritmo de \emph{Viterbi}
\cite{nilm_bouloutas_viterbi_ext_1991_11,
nilm_hart_fsm_viterbi_1993_12,nilm_baranski_genetic_base_2003_19,
nilm_baranski_genetic_detalhado_2004_20,nilm_baranski_summary_2004_21,
nilm_bergman_distribuido_2011,nilm_zeifman_vast_2011,
nilm_zeifman_vastext_approach_2012,
nilm_zeifman_statistical_vastext_1stws_2012,
seminilm_fhmm_empiricalnmeter_2013}
operando em cima de técnicas de agrupamento de dados para a modelagem
de \glspl{c2} e \glspl{c3}, no entanto, por modelar utilizando o
comportamento mais provável, está técnica não permite observar
\gls{c2b} e \gls{c4}. É compreensível que tal tarefa não seja possível
de ser realizada cegamente, justamente por não existir um padrão a ser
identificado. Possivelmente esses equipamentos terão seus estados
parcialmente modelados por \glspl{c3} e/ou \gls{c2}, sendo necessário
intervenção humana para corrigir sua modelagem. As \glspl{c4}
representam um desafio ainda maior para os \glspl{nilm}, podendo ser
mais uma fonte de ruídos para aquelas configurações que realizam a
Etapa~\ref{itm:etapa2}, bem como um empecilho para a reconstrução de
energia por não alterarem seu consumo de maneira discreta. Deste modo,
até o momento as técnicas para criação de modelos trabalharam apenas
com as características macroscópicas e estatística de uso.
Normalmente, os autores utilizaram em conjunto com os modelos
construídos cegamente técnicas de otimização
\cite{nilm_bergman_distribuido_2011} ou discriminadores estatísticos
\cite{nilm_zeifman_vast_2011, nilm_zeifman_vastext_approach_2012,
nilm_zeifman_statistical_vastext_1stws_2012,
seminilm_fhmm_empiricalnmeter_2013}, o que seria a continuação natural
ao problema. Ainda assim, a extração dos modelos necessita de uma
grande quantidade de estatística --- 5 dias à 2 semanas ---,
possibilitando a utilização dessa estatística obtida na construção dos
modelos para alimentar diretamente o treinamento de técnicas
supervisionadas.

Além do mais, há uma discussão em relação a como transformar a
informação dos modelos construídos cegamente em equipamentos. Os casos
observados \cite{nilm_hart_1992_8,nilm_bergman_distribuido_2011,
nilm_zeifman_vastext_approach_2012} tendem a comparar as informações
obtidas com um catálogo, mas como observado por
\cite{nilm_matthews_overview_2008_22}, a melhor estratégia ao tema
seria ter uma plataforma de comunicação com o consumidor, deixando a
tarefa de identificação para o mesmo. O catálogo poderia ser utilizado
como valor inicial ou sugerido para o equipamento, e se nenhum valor
fosse encontrado no catálogo, um valor temporário como
``desconhecido01'' poderia ser utilizado até o usuário modificar seu
valor. Com todas as facilidades oferecidas pelas \glspl{ict}, a
comunicação com o usuário pode ser realizada por uma interface
\emph{web}, um aplicativo para \emph{smart phones} ou alterar os
mostradores domiciliares para permitir interação com o consumidor.

Nesse caso, também seria interessante a identificação de novos
equipamentos conforme a utilização da rede. A identificação de equipamentos
não mais em uso pode ser benéfica para técnicas que sofrem com 
\gls{c6}, reduzindo a possibilidade de confusão se um equipamento não
estiver mais sendo utilizado. Aproveitando a possibilidade de
comunicação com o consumidor, o projeto pode permitir do mesmo
alterar possíveis erros na modelagem dos equipamentos, unindo ou
separando estados de \gls{fsm} erroneamente identificados. Isso também
pode ser expandido para o caso de erros de discriminação dos
equipamentos. Já para o caso das abordagens onde o \gls{nilm} não tem a
aptidão para modelar automaticamente os equipamentos, fica evidente a
necessidade de priorizar os equipamentos de maior uso-final para, assim,
maximizar sua reconstrução de energia. Ainda assim, o sucesso da
técnica depende de alta eficiência, onde a intervenção do consumidor
não pode ser um ponto necessário no projeto. Em especial para o caso
da aplicação da técnica como um programa de \gls{ee}, onde a tarefa do
\gls{nilm}, além de desagregar energia, é informar o consumidor sobre
como utilizar a energia de maneira sustentável através da redução do
consumo e resposta em demanda.

Do ponto de vista técnico, o emprego das abordagens devem ser
realizadas tendo em mente que a tarefa principal do \gls{nilm}
é a desagregação da informação de energia, independente de como a
técnica empregada entenda o problema. Por exemplo, técnicas de separação
cega de fontes irão ver o problema como um problema de desmistura de
fontes, técnicas de aprendizado de máquina como a capacidade de
entender as diferenças das características apresentadas e técnicas de
otimização como obter a melhor qualidade/aptidão ou menor desvio com
aquilo que é oferecido.  As técnicas têm seus próprios parâmetros para
informar o quão bem realizaram suas tarefas, no entanto, pela
perspectiva do \gls{nilm} e do consumidor, a eficiência deve ser
informada em termos da capacidade de reconstruir a energia por
equipamento. Isso se torna ainda mais evidente quando o \gls{nilm} opera
em tempo real, de tal modo que sua eficiência deve analisar
\emph{temporalmente} o acerto de energia identificando erros devido à
redundância ou subestimação do consumo por equipamento como foi tratado
na Subseção~\ref{ssec:nilm_eff_calc}.

Quanto à extração de características e técnicas empregadas, observam-se
três vertentes. Buscando a impregnação do método, uma vertente
limita-se ao emprego de técnicas em cima de características
macroscópicas obtidas com baixa amostragem 
\cite{nilm_hart_1992_8,nilm_cole_data_extraction_1998_14,
nilm_cole_extra_info_surge_1998_15,
nilm_norford_leeb_medianfilt_1996_13,
nilm_leeb_spectral_envelope_1995_23,
nilm_powers_15minsamp_1991_16,
nilm_farinaccio_16ssamp_1999_17,
nilm_marceau_16ssamp_improved_1999_18,
nilm_baranski_genetic_base_2003_19,
nilm_baranski_genetic_detalhado_2004_20,nilm_baranski_summary_2004_21,
nilm_bergman_distribuido_2011,nilm_genetic_2013,nilm_zeifman_vast_2011,
nilm_zeifman_vastext_approach_2012,
nilm_zeifman_statistical_vastext_1stws_2012,
nilm_zeifman_statistical_naive_enduses_2013}. Essa vertente apresenta
a configuração de menor embargo financeiro e pode aproveitar de uma
infraestrutura que está sendo disponibilizada pela rede elétrica
inteligente, os medidores inteligentes (no Brasil, sujeito a possíveis
restrições descritas na Subseção~\ref{ssec:ret_tec}).  Os autores de
\cite{nilm_bergman_distribuido_2011} realizaram uma arquitetura
distribuída para permitir desagregar informação de consumo sem
qualquer alteração da estrutura fornecida nos \gls{eua}.
\cite{nilm_baranski_genetic_base_2003_19,
nilm_baranski_genetic_detalhado_2004_20,nilm_baranski_summary_2004_21}
mostraram que é possível realizar essa tarefa em medidores
eletromecânicos. No entanto, a presença apenas das informações
macroscópicas irão deteriorar a eficiência do \gls{nilm}, onde
\gls{c6} irão afetar a capacidade de discriminação dessa vertente. 

Por isso, as tendências nessa vertente são o emprego de estatística de
uso tanto de maneira cega \cite{nilm_zeifman_vast_2011,
nilm_zeifman_vastext_approach_2012,
nilm_zeifman_statistical_vastext_1stws_2012}, como através de
levantamentos de dados por institutos
\cite{nilm_zeifman_statistical_naive_enduses_2013}. No último caso, a
técnica irá se concentrar nos equipamentos de maior uso-final para
maximizar a capacidade de reconstrução de energia total. Por outro
lado, não se sabe, no momento, se é possível seguir esse caminho no
Brasil uma vez que são necessários dados recentes e representativos
das diversas regiões do país. Como foi articulado na
Seção~\ref{sec:ee_dificuldades}, a concepção inicial para o
\gls{nilm} no Brasil foi justamente o intuito de facilitar a coleta e
precisão de dados, reduzindo o peso das \glspl{pph}. Nada obstante, o
levantamento de estatística na própria residência, como mostrado por
\cite{nilm_zeifman_vastext_approach_2012,nilm_zeifman_statistical_vastext_1stws_2012},
parece ser suficiente para atender os pontos necessários para a
aplicação dessa tecnologia em um programa de \gls{ee}. A única questão
dessa abordagem seria como rotular os modelos encontrados com os
nomes dos equipamentos, assunto debatido previamente nesta subseção.

Outra vertente, ao procurar por melhores capacidades de discriminação,
seguiu o caminho de elevar a taxa de amostragem \cite{
nilm_zeifman_vast_hisample_pdfmerge_2011,nilm_liang_pt1_2010_34,
nilm_liang_pt2_2010_40,nilm_patel_2007_29,nilm_gupta_patel_2010_30,
nilm_lee_variable_speed_estimation_2005_24,
nilm_wichakool_2009_25,nilm_shaw_2008_26,nilm_berges_2008_7,nilm_berges_2009_36,
2010_nilm_melhorando_pph_usa_37,nilm_chan_2000_31,nilm_coppe_nascimento,
nilm_lee_2004_32,nilm_lam_2007_33, nilm_srinivasan_nn_2006_27,
nilm_itajuba_rodrigues,nilm_suzuki_2011_35} para tornar possível a
avaliação de características microscópicas. A melhor capacidade
discriminativa das características a serem exploradas nesse universo
levaram aos autores geralmente a se contentarem em utilizar apenas uma
técnica para a desagregação do consumo. Em alguns casos \cite{
nilm_patel_2007_29,nilm_gupta_patel_2010_30}, a
resolução das características é tão fina que as próprias propriedades
da rede interferem nos padrões dos equipamentos. Logo, os discriminadores 
teriam a inconveniência de ser alimentados com a informação para
cada residência, bem como a utilização de um catálogo global seria
impossível, limitando a aplicação dessa abordagem se não houver
uma maneira automática de realizar esse processo. 

Os estudos realizados com características microscópicas geralmente
utilizaram do conhecimento prévio dos equipamentos presentes na rede e
analisaram a capacidade de desagregar a informação somente com esses
equipamentos presentes, ignorando a necessidade de escalabilidade do
projeto. Ainda, por analisar características com maior resolução,
essas técnicas são possivelmente mais sensíveis a ruído, já que os
mesmos distorcem com mais facilidade as características microscópicas
tornando as mesmas um padrão não similar àquele procurado. Por isso,
ainda é necessário descobrir se a maior capacidade de discriminação
também se aplica para essas condições.

A presença de informações discriminativas no nível de ciclo da rede
possibilitou abordagens onde a Etapa~\ref{itm:etapa2} não é realizada
antes da Etapa~\ref{itm:etapa3}. Uma das capacidades dos métodos que
exploram o regime operativo permanentemente é a identificação de
\glspl{c1}. Porém, eles sofrem com a necessidade de que todos os
padrões possíveis sejam conhecidos. No caso, só foram observadas três
abordagens que operam desse modo \cite{nilm_srinivasan_nn_2006_27,
nilm_itajuba_rodrigues,nilm_suzuki_2011_35}. As abordagens de
\cite{nilm_srinivasan_nn_2006_27,nilm_itajuba_rodrigues} trabalham com
todos os possíveis estados operativos da rede (combinação de todos os
estados operativos possíveis para os equipamentos), tornando essas
abordagens frágeis para a aplicação quando a quantidade de estados na
rede é grande, uma vez que é necessário treinar as técnicas
supervisionadas (no caso foram aplicadas \glspl{rna}) com um número de
padrões que cresce exponencialmente proporcional ao número total de
estados dos equipamentos. Já a abordagem de \cite{nilm_suzuki_2011_35},
por trabalhar o problema através de otimização, conseguiu uma maior
possibilidade prática para a aplicação de sua técnica, ainda estando
sujeito à possíveis problemas de patamares de consumo parecidos que
possam ser determinados por diversas combinações de equipamentos
diferentes. Além disso, não se sabe como a presença de equipamentos
desconhecidos afetarão essas abordagens. Ainda ficou em aberto como
determinar o consumo dos equipamentos através da informação retornada
pela etapa de discriminação, sugestões foram realizadas na
pp.~\pageref{text:transf_info_discr_energia}.  Finalmente, os
resultados mostram que para configurações onde há poucos equipamentos
conhecidos essas técnicas solucionam o problema bem-sucedido.

Uma possibilidade --- e \emph{quiçá} tendência --- para essa vertente é a
combinação de múltiplas características e técnicas para o
reconhecimento de padrões \cite{
nilm_zeifman_vast_hisample_pdfmerge_2011,nilm_liang_pt1_2010_34,
nilm_liang_pt2_2010_40}. A pluralidade de características
independentes é benéfica para a capacidade de discriminação, sendo
ainda um fato pouco explorado. Alguns autores chegaram a
utilizar/considerar as medidas macroscópicas para auxiliar no processo
discriminativo em conjunto com as medidas microscópicas. Por exemplo, 
\cite{nilm_coppe_nascimento} empregou uma etapa de pré-seleção para os
equipamentos que mais se destacavam nas características macroscópicas
facilitando no processo discriminativo. Nessa mesma abordagem, como se
utilizou apenas da magnitude da corrente como característica
microscópica, observou-se que a adição da variação do \gls{fp} era
benéfico para a discriminação. Porém, geralmente as características
microscópicas conseguem representar as características macroscópicas,
sendo necessário buscar por outras características microscópicas
representativas. Além disso, o emprego de múltiplas técnicas
independentes em paralelo (em cima, ou não, das mesmas características)
permite uma melhor discriminação, já que poderá acontecer
complementação das lacunas individuais de cada técnica. Há uma
variedade de possibilidades para a fusão dos resultados apresentados em
paralelo pelos discriminadores, que além de \glspl{rna}, podem ser
realizadas como proposto por \cite{nilm_liang_pt1_2010_34} ou
\cite{nilm_zeifman_review_2011} (ver
pp.~\pageref{text:uniao_tecnicas}). No entanto, é importante frisar que
isso também trará uma maior carga computacional ao \gls{nilm}, de
forma que a quantidade de características e técnicas aplicadas será
limitada pela capacidade de processamento disponível, particularmente
quando operando em tempo real. Ressalta-se aqui que, além da discussão
para aplicação de múltiplas características microscópicas e técnicas,
o emprego de estatística do uso, característica até o momento
\emph{sui generis} na vertente explorando características
macroscópicas, também pode ser realizada em quaisquer topologias do
\gls{nilm}.

A última vertente quebra a ideia de um medidor central para desagregar
toda a informação de consumo, percebendo que a questão do projeto é
reduzir o custo com sensores de forma a torná-lo viável em larga
escala, o que não implica, necessariamente, na utilização de apenas um
medidor. Para essa versão, o \gls{nilm} deixa de ser não-intrusivo,
para ter uma configuração \emph{semi-intrusiva}
\cite{seminilm_ihome_tomek_2012,seminilm_fhmm_empiricalnmeter_2013,
seminilm_berges_multisensor_2010}. O objetivo para essa configuração
pode ser descrito como minimizar o custo com sensores sem deteriorar a
capacidade de desagregação. Essas abordagens podem optar por
submedidores mais próximos ao nível de consumo, reduzindo o
acúmulo de operação de equipamentos que, intrinsecamente, torna a tarefa mais
simples; utilizar outros sensores que deem sensibilidade para a
discriminação, como correlacionar o nível sonoro, de iluminação e
movimento com certos equipamentos; ou submedição diretamente nos
equipamentos que as técnicas empregadas tiverem maior dificuldade. Porém,
deve-se considerar a estética quando na utilização de submedidores,
onde os consumidores tenderão a rejeitar configurações que reduzam a
harmônia do ambiente em seus domicílios. Isso também é verdade para
qualquer configuração que reduza a privacidade ou façam-nos ter a
sensação de que isso possivelmente irá ocorrer. Uma outra percepção em
vista das dificuldades das abordagens que operam ciclo a ciclo 
\cite{nilm_srinivasan_nn_2006_27,
nilm_itajuba_rodrigues,nilm_suzuki_2011_35}, uma maneira de superar a
dificuldade do crescimento exponencial seria através da aplicação de
submedidores atuando em cima de uma informação agregada até um número
máximo de equipamentos.

Portanto, as técnicas devem ser robustas para aplicação em residências
com a presença de equipamentos desconhecidos e para os momentos onde
haverá maior operação, causando a operação simultânea de diversos equipamentos
que, possivelmente, injetarão ruídos na rede através de sua dinâmica.
Devido a presença de equipamentos desconhecidos, um aspecto desejável para
o \gls{nilm} é conseguir detectá-los de maneira automática e de algum
modo possibilitar a correlação dos modelos identificados com seus
nomes. Outra questão é como obter informação discriminante garantindo
que o projeto seja viável. A utilização de amostragem baixa deteriora
a capacidade de discriminação, por outro lado é possível que a
estrutura para essas abordagens já esteja disponível. Essa abordagem
parece ser a mais próxima de tornar-se realidade. Enquanto a elevação
da amostragem aumenta a eficiência, dificulta a aplicação do projeto
por necessitar de \emph{hardware} específico. Além disso, as técnicas
presentes atualmente para alta amostragem ainda precisam ser testadas
quanto a sua robustez. Todavia, um outro caminho para obter maiores
eficiências pode ser através da escolha ótima de submedidores que, ao
mesmo tempo, maximizam a eficiência e minimizam o custo do projeto.

\section[A Metodologia no CEPEL]{A Metodologia no \acs{cepel}}
\label{sec:nilm_cepel}

Esta seção se dedica à descrição do projeto no \gls{cepel} e as
colaborações realizadas pela \acs{coppe} no mesmo. Os trabalhos de
autores precedentes ao presente trabalho é realizado na
Subseção~\ref{ssec:cepel_anteriores}. A contribuição realizada pelo
grupo de engenheiros do \gls{cepel} posterior ao último trabalho em
colaboração com a \acs{coppe} encontra-se na
Subseção~\ref{ssec:met_cepel}, sendo essa metodologia a base
para o trabalho atual. Na Subseção~\ref{ssec:caracteristicas} é
realizada considerações quanto a escolha da \gls{fex} em relação a
discriminação, levantamento deixado para contribuir na escolha das
mesmas em trabalhos futuros. Ver-se-á nesta seção que há uma série de
dificuldades a serem abordadas antes de retornar a questão de
discriminação dos equipamentos.

\subsection{Estudos Anteriores}
\label{ssec:cepel_anteriores}

Os estudos envolvendo a aplicação de técnicas não-invasivas para a
desagregação do consumo no \gls{cepel} tiveram seu início com o estudo de
\citet*{nilm_cepel_aguiar}, em \citeyear{nilm_cepel_aguiar}. Esse
estudo, bem como os seus sucessores:
\citet{nilm_cepel_bezerra} (\citeyear{nilm_cepel_bezerra}) e 
\citet*{nilm_cepel_alvaro} (\citeyear{nilm_cepel_alvaro}), utilizaram a
informação da envoltória da onda de corrente no regime transitório
para identificar o acionamento dos equipamentos domiciliares. Diferente
dos trabalhos realizados pelo mesmo grupo de
\citeauthor*{nilm_hart_1992_8} para o setor comercial e industria
\cite{nilm_norford_leeb_medianfilt_1996_13,
nilm_leeb_spectral_envelope_1995_23}\footnote{Esses são apenas alguns
dos trabalhos realizados, a literatura disponível pelo grupo é muito
mais extensa.} onde a extração da envoltória obtida através da decomposição
harmônica é realizada em amostras espaçadas em 1~\acs{hz}, os
trabalhos utilizaram a envoltória em 60~\acs{hz}. Essa diferença
é justificável pela diferença dos dois ambientes, onde os transitórios
de muitos equipamentos disponíveis para os trabalhos
\cite{nilm_norford_leeb_medianfilt_1996_13,
nilm_leeb_spectral_envelope_1995_23} tem características lentas (ordem
de segundos a poucos minutos), tanto que os autores tiveram que
explorar o conceito de seções-v (ver
pp.~\pageref{nilm:pot_real_trans}), enquanto os explorados no setor
residencial apresentam toda a informação relevante de
transitório em até 1~segundo (\cite{nilm_cepel_aguiar} observou para
seus equipamentos analisados 417~ms). Como foi notado por diversos autores
\cite{nilm_hart_1992_8,nilm_sultanem_1991_10,
nilm_cole_data_extraction_1998_14,nilm_cole_extra_info_surge_1998_15},
há um lento transitório em alguns equipamentos que tenuemente alteram
seu consumo --- por exemplo, devido ao aquecimento de motores ---, mas
aqui se refere essencialmente a informação discriminante do
transitório.

Todos os trabalhos utilizaram \gls{rna} para discriminar a envoltória
de corrente, entretanto algumas peculiaridades podem ser observadas
para cada estudo. Devido a maior limitação, \citet*{nilm_cepel_aguiar}
utilizou 16 pontos da envoltória (normalizados por 100~\acs{a})
espaçados não-linearmente para conseguir uma melhor representação da
informação do transitório --- distribuiu-se os pontos para obter uma
maior granularidade próxima ao transitório, que gradualmente vão se
esparsando conforme eles ficam distantes do centro ---, o valor de
corrente em regime permanente (normalizados por 50~\acs{a}) e
ângulo de carga. Apesar de todas as limitações técnicas, o método
conseguiu eficiência de classificação superiores à 90\%, mas sofrendo
deterioração para o grupo de eletrônicos. 

Já \citet{nilm_cepel_bezerra}, trabalhou em cima de um sistema
de aquisição mais adequado, onde sua melhor configuração empregou 60 amostras
da envoltória que são transformadas utilizando \gls{pcd}, resultando
em 98\% de eficiência de classificação.  Esse autor também realizou
outros estudos, sendo eles:
\begin{enumerate*}[label=\itshape\alph*\upshape)] 
\item consideração entre a utilização ou não da informação de regime
permanente para classificação, onde ela se mostrou desnecessária e
inclusive obtendo resultados piores que a sua melhor configuração; 
\item análise sobre como a sobreposição de transitórios afetaria a
classificação, mostrando que esses eventos são críticos para a
capacidade discriminativa; e 
\item implementação de diversos cenários possíveis para o projeto,
considerando inclusive a implementação de um \gls{dnilm}.  
\end{enumerate*}

Finalmente, o sistema de aquisição disponível para
\citet*{nilm_cepel_alvaro} possibilitou a análise de uma gama
maior de equipamentos, sendo capaz observar equipamentos com
consumo superior à 150~m\acs{a}, enquanto
\citet*{nilm_cepel_aguiar} operava somente para equipamentos consumindo
mais de 1~\acs{a}. Houve também uma melhoria no
levantamento do banco de dados, havendo uma amostragem bastante
superior aos trabalhos anteriores, com um total de 1324 acionamentos
observados. O trabalho implementou dois detectores especialistas para
a identificação dos momentos de acionamentos (a
Etapa~\ref{itm:etapa2}, porém apenas para acionamentos), e um baseado
em correlação. Foram utilizadas 150 amostras da envoltória de
corrente, obtendo taxa de detecção de 86~\%.  Grande parte da
inexatidão ocorreu devido aos erros no grupo de eletrônicos, obtendo
taxas de classificação bastante inadequadas para esse grupo, na ordem
de 30~\%.

Apesar da evolução dos trabalhos, diversos pontos podem ser
levantados. Primeiro, a principal característica utilizada pelos
estudos só é discriminante para os acionamentos de eventos, não
podendo utilizar a mesma na identificação de padrões de desacionamentos
ou mudanças para estados de menor consumo, que na maioria dos casos
são apenas uma queda abrupta no consumo.  Além disso, não é possível
inferir se essa informação também seria discriminante para mudanças de
estado positivas, já que os estudos não trataram o caso de \gls{c2}
--- sendo uma outra dificuldade a ser tratada. Não bastasse, os
estudos trataram apenas de casos de equipamentos sendo acionados
isoladamente. Como foi notado na Subseção~\ref{ssec:nilm_discussao},
a maior dificuldade para os \glspl{nilm} ocorrem devido à
operação simultânea de equipamentos na rede, em especial quando eles
se constituem de \gls{c5}. Os esforços para realizar a
Etapa~\ref{itm:etapa2} envolveram apenas a detecção de acionamentos,
bem como o trabalho de \cite{nilm_cepel_alvaro} não considera a
questão de classificação do desacionamento. Indo adiante, a taxonomia
levantada pelos autores realiza a discriminação em grupos bastante
genéricos de equipamentos.  Não se sabe até que ponto essa informação
pode contribuir para a desagregação do consumo de energia residencial,
em especial para a aplicação em programa de \gls{ee}, onde os
consumidores desejam saber o consumo por equipamento. Continuando, a
técnica se mostrou ineficiente para um grupo de equipamentos cada vez
mais presente nas residências, os eletrônicos, sem ser possível saber
o porquê isso ocorre. Por fim, não se tratou o tema de equipamentos
desconhecidos, sendo ainda necessário considerar como adaptar o método
para identificar novos padrões recorrentes.

Em vista das dificuldades de tornar a informação da classificação em
informação relevante para os \glspl{nilm}, o grupo do \gls{cepel}
empenhou-se no sentido de melhorar a capacidade de detecção de eventos
de transitório, como será visto a seguir. Quanto à questão do
problema da envoltória como característica para discriminação, serão
realizadas considerações mais adiante, na
Subseção~\ref{ssec:caracteristicas}.

\subsection{Metodologia de Partida do Trabalho}
\label{ssec:met_cepel}

A equipe do \gls{cepel} comprometeu-se desde o trabalho de
\citeauthor*{nilm_cepel_alvaro} para tornar o projeto mais próximo das
necessidades do \gls{nilm}. Reparou-se na necessidade de um
método para a detecção de eventos de transitório que fosse capaz de
identificar tanto mudanças para estados através do acréscimo de
energia, quanto no decréscimo. Os detectores fornecidos por esse autor
são especialistas para detecção de acionamentos, sendo dois deles
determinados empiricamente e um outro utilizando correlação entre o
sinal e formas de onda pré-determinadas. Exceto o último, que não teve
bons resultados, eles não são aplicáveis para detecção de decréscimo
de demanda causado pelos equipamentos, sendo necessário adaptar os
detectores empíricos para os casos de desacionamento, ou criar um novo
filtro. 

\begin{figure}[h!t]
\centering
\includegraphics[width=.8\textwidth]
{imagens/cepel_transitorio.pdf}
\caption[Esboço da metodologia empregada pelo CEPEL]{Esboço da
metodologia empregada pelo \acs{cepel}, ponto de partida para este
trabalho.}
\label{fig:cepel_transitorio}
\end{figure}


A opção tomada pelo grupo do \acs{cepel} foi a segunda, a equipe
formada por \citet*{rel_cepel_detevt} decidiu utilizar a derivada de
Gaussiana, baseando-se na literatura de processamento de imagens para a
detecção de bordas. Um esboço da metodologia empregada pelo
\acs{cepel} está disposto na Figura~\ref{fig:cepel_transitorio}. No
caso, a diferença na aplicação na derivada de Gaussiana é que, ao
invés de operar em cima de \emph{pixels}, os valores são amostras de
corrente e, por isso, a mesma opera em apenas uma dimensão. Para realizar a
convolução, aplicou-se um \acs{fir} com uma janela inicialmente com o
mesmo número de amostras que havia sendo utilizado, de 150 amostras a
uma taxa de 60~\acs{hz}, o que implica em 2,5~s. Vale ressaltar que
na verdade é o desvio padrão da Gaussiana que determina o tamanho
relevante da janela, já que ao se distanciar do centro os valores irão
tender a zero --- o mesmo sendo verdade para sua derivada ---, sendo
assim, o número de 150 amostras é apenas um corte para a quantidade
máxima de informação contida no filtro e utilizadas para a convolução.
O valor de resposta alimenta um corte linear no qual, se a resposta
for superior em absoluto, irá gerar uma região sensibilizada para ser
analisada até a resposta do filtro normalizar, ou continuar
sensibilizando mas com sinal oposto, nesse caso já criando uma nova
região sensibilizada. Para determinar o centro do transitório,
simplesmente é realizado o cálculo do ponto de inflexão da resposta do
filtro, e se houver mais de um, o ponto de inflexão com maior valor
absoluto é utilizado. Esse processo está ilustrado na
Figura~\ref{fig:resp_fir}.

\FloatBarrier

\begin{figure*}[p!]
  \begin{center}
    \begin{subfigure}[c]{\textwidth}
      \includegraphics[width=\textwidth]{imagens/exemplo_regioes_candidatos.pdf}
      \caption{Exemplo de resposta do \acs{fir}, de regiões sensibilizadas e dos candidatos
resultantes proposta pelo \acs{cepel}. Na subfigura superior está a
envoltória de corrente, que é utilizada como entrada do \acs{fir} e
tem sua resposta representada na subfigura inferior. Observe que a
resposta do \acs{fir}, ao ultrapassar o limiar de 0,003, gera uma região
sensibilizada, representada em verde e vermelho, respectivamente para
um candidato a acréscimo e decréscimo de consumo. Os candidatos são
gerados no centro do transitório, o ponto de inflexão da resposta do
filtro com maior valor absoluto, que estão circulados em amarelo na
imagem.}
      \label{fig:resp_fir}
    \end{subfigure}
    \hfill
    \begin{subfigure}[c]{\textwidth}
      \includegraphics[width=\textwidth]{imagens/exemplo_fex.pdf}
      \caption{A realização da \acs{fex}. O cálculo é feito para todas as
variáveis \acs{di}, \acs{dp}, \acs{dq}, \acs{dd} e \acs{ds}.}
      \label{fig:cepel_fex}
    \end{subfigure}
    \hfill
  \end{center}
\end{figure*}

\begin{figure*}[p!]
  \begin{center}
    \ContinuedFloat
    \begin{subfigure}[c]{\textwidth}
      \includegraphics[width=\textwidth]{imagens/evento_ruidoso.pdf}
      \caption{Exemplo de remoção de eventos ruidosos. A eliminação
desses candidatos é importante para a redução do falso alarme. No
caso, os candidatos possuem valores de \acs{di} menores ao limiar de
0,1, não sendo aceitos por esse motivo.}
      \label{fig:ex_ruidoso}
    \end{subfigure}
  \end{center}
\caption[Gráficos descrevendo a operação da metodologia proposta pelo
\acs{cepel}.]{Gráficos descrevendo a operação da metodologia proposta
pelo \acs{cepel}. Os gráficos foram gerados utilizando o ambiente de
análise implementado por este trabalho descrito no
Capítulo~\ref{chap:framework}.}
\label{fig:cepel_metodologia_operacao}
\end{figure*}

\FloatBarrier

A fim de avaliar a eficácia, o relatório utilizou de novos dados
(descritos na Seção~\ref{sec:base_de_dados}), onde
algumas configurações já apresentavam operação justaposta de equipamentos
operando em diversos estados, no entanto sem ruído. A ideia aplicada
foi de gradualmente aumentar a dificuldade a ser estudada e adaptar a
abordagem até chegar em condições similares àquelas encontradas nas
residências brasileiras. A análise para as condições com pouca
presença de ruído (os conjuntos de dados NI00***) revelou
que a derivada de Gaussiana obteve melhor eficiência que os filtros
apresentados no trabalho anterior, sendo a técnica escolhida desde
então para a detecção de eventos de transitório.  A versão do
algoritmo que foi o ponto de partida deste trabalho ainda considerou
mais alguns detalhes. Depois de ser determinado o centro, é calculado
o pré e o pós-transitório para as variáveis macroscópicas disponíveis.
No caso, elas são \acf{di}, \acf{dp}, \acf{dq}, \acf{dd} e \acf{ds},
obtidos através de \ref{eq:deltasMacro}.

Ao acompanhar a Figura~\ref{fig:cepel_fex}, dedicada para a ilustração
desse processo, observa-se que os valores determinados inicialmente
pelo \gls{cepel} para obter os valores de pré e pós transitório são a
média de 10 amostras afastados de 140 pontos em relação ao centro do
transitório. A variável \acs{di} é empregada para realizar um corte
linear para um valor mínimo ($\delta_{min} = 0,003$) em que o
candidato a evento --- o centro selecionado da região sensibilizada
pelo filtro --- com o intuito de remover possíveis regiões
sensibilizadas por ruído, Figura~\ref{fig:ex_ruidoso} (\acs{di}$_{min}
= 0,1$ A). Também se realiza um outro corte levando em conta a
proximidade de candidatos a evento de transitório. Caso os mesmos
estejam próximos entre si abaixo de um valor mínimo de amostras, os
eventos temporalmente posteriores ao primeiro evento são rejeitados
($n_{min} = 150$ amostras). Um exemplo pode ser observado na
Figura~\ref{fig:resp_fir}, aonde o candidato proveniente da região
sensibilizada negativa foi eliminado por estar próximo ao candidato da
região positiva. Ainda foi considerado se o corte por ruído deveria
ser realizado antes do corte por evento próximo, caminho (I) na
Figura~\ref{fig:cepel_transitorio}; vice-versa, caminho (II); ou
apenas um dos cortes, caminhos (III) e (IV). O \gls{cepel} conseguiu
excelentes resultados para a configuração sem ruído (os resultados
estão disponíveis no capítulo de resultados na
Tabela~\ref{tab:resultados}), entretanto, não havia uma maneira
sistemática para a determinação os valores de corte, bem como
visualizar o comportamento do algoritmo para entender suas nuances. Ao
analisar a metodologia em novas configurações, o processo para análise
sofria com a falta de infraestrutura.

\begin{subequations} \label{eq:deltasMacro}
\begin{equation} \label{eq:dI}
\Delta I = I_{\text{pós}} - I_{\text{pré}}
\end{equation}
\begin{equation} \label{eq:dP}
\Delta P = P_{\text{pós}} - P_{\text{pré}}
\end{equation}
\begin{equation} \label{eq:dQ}
\Delta Q = Q_{\text{pós}} - Q_{\text{pré}}
\end{equation}
\begin{equation} \label{eq:dD}
\Delta D = D_{\text{pós}} - D_{\text{pré}}
\end{equation}
\begin{equation} \label{eq:dS}
\Delta S = S_{\text{pós}} - S_{\text{pré}}
\end{equation}
\end{subequations}

Cabe definir o que são as características representadas em
\ref{eq:deltasMacro} com mais detalhes. As mesmas são bastante
conhecidas e amplamente utilizadas em análise de sistemas elétricos
operando em condições não-senoidais \cite{akagi2007instantaneous},
sendo definidas para sistemas \emph{monofásicos} no domínio da
frequência por \ref{eq:ipqds}. Também se define uma grandeza já
mencionada anteriormente no texto, o \gls{fp}, através de \ref{eq:fp}.
Uma observação importante quanto ao \acs{d}, o mesmo é obtido
exatamente como a equação \ref{eq:d}, ou seja, ele não é uma
variável amostrada, e sim \emph{estimada}. É importante notar que de
acordo com \ref{eq:d} essa variável só pode assumir valores positivos,
diferente de \acs{dq} que assume valores em ambas direções, sendo, na
realidade, a única característica que apresenta essa propriedade.
Obviamente, erros de medição irão ocorrer, de modo que o valor de
\acs{d} pode acabar sendo pertencente ao conjunto de números complexos.
Para evitar isso, nesses casos o valor de \acs{d} é determinado como
zero.

Apesar de se ter conhecimento das deficiências para as variáveis em
\ref{eq:q}--\ref{eq:d} de representar grandezas físicas \cite[cap.
2]{akagi2007instantaneous}, não se levará o assunto em consideração
pois: 

\begin{enumerate}
\item mesmo que a representação física dessas variáveis não seja clara, 
não se está interessado na mesma, apenas na capacidade delas de
servirem como característica para a detecção de eventos e
discriminação, este trabalho focando na primeira parte;
\item essas variáveis são representações possíveis de serem obtidas com
medidores de baixo custo. 
\end{enumerate}

\begin{subequations} \label{eq:ipqds}
\begin{eqnarray}\label{eq:v}
v(t) = \sqrt{2}V_1 \text{sen}(\omega_1 t + \phi_{1v}) +
\sqrt{2}V_2 \text{sen}(\omega_2 t + \phi_{2v}) +  \nonumber \\
\sqrt{2}V_3 \text{sen}(\omega_3 t + \phi_{3v}) + \dots
\end{eqnarray}
\begin{eqnarray}\label{eq:i}
i(t)=\sqrt{2}I_1 \text{sen}(\omega_1 t + \phi_{1v} - \theta_{1}) +
\sqrt{2}I_2 \text{sen}(\omega_2 t + \phi_{2v} - \theta_{2}) +
\nonumber \\
\sqrt{2}I_3 \text{sen}(\omega_3 t + \phi_{3v} - \theta_{3}) + \dots
\end{eqnarray}
\begin{equation}\label{eq:pn}
P_n=V_nI_n\cos{\theta_n} ~~,~~ P=\sum_{n=1}^{\infty}P_n
\end{equation}
\begin{equation}\label{eq:q}
Q_n=V_nI_n\text{sen}{\theta_n} ~~,~~ Q=\sum_{n=1}^{\infty}Q_n
\end{equation}
\begin{equation}\label{eq:s}
S=VI
\end{equation}
\begin{equation}\label{eq:d}
D^2=S^2-P^2-Q^2
\end{equation}
\end{subequations}

\noindent onde:

\begin{description}
\item[$V_n$] é a n-ésima componente da decomposição harmônica em valor
eficaz de tensão;
\item[$I_n$] é a n-ésima componente da decomposição harmônica em valor
eficaz de corrente; 
\item[$\omega_n$] é velocidade angular da n-ésima componente
harmônica, podendo ser escrita como $2n\pi f$ ($f$ frequência da
fundamental);
\item[$\phi_n$] é o ângulo de defasagem para a n-ésima componente de
tensão;
\item[$\theta_n$] é o ângulo de defasagem entre tensão e corrente para 
n-ésima componente da decomposição harmônica;
\item[$P_n$] é a potência ativa para o n-ésimo harmônico;
\item[$Q_n$] é a potência reativa para o n-ésimo harmônico;
\end{description}

\begin{equation} \label{eq:fp}
FP = \frac{P}{S}
\end{equation}

Em especial quanto ao segundo item, há um interesse por parte do
\gls{cepel} de procurar uma solução menos complexa, mesmo
que isso deteriore --- até níveis aceitáveis --- a capacidade do
\gls{nilm}, mas que resulte em uma resposta mais rápida em tempo de
projeto. Por esse mesmo motivo, o \gls{cepel} considera a utilização
de medidores presentes no mercado, dando preferência para os de baixo
custo para facilitar o projeto, já que o desenvolvimento do
\emph{hardware} adiciona peso e reduz a velocidade do projeto. Por
enquanto, dois medidores estão sendo utilizados, um proprietário do
\acs{cepel} e um medidor da empresa \emph{Yokogawa}
\cite{yokogawa_medidor}, estando alguns dados disponíveis em uma
configuração, e outros na outra.

As variáveis anteriormente citadas --- adicionadas de suas envoltórias
--- são as que estão disponíveis atualmente para serem utilizadas (em
amostragem de 60~\acs{hz}).  Este trabalho foca na detecção de
eventos, buscando nessa informação para identificá-los, e não entrando
em detalhes sobre a classificação.  Na subseção a seguir serão feitas
algumas considerações com o intuito de auxiliar trabalhos futuros na
escolha de características. Como é possível perceber, há uma série de
dúvidas a serem tratadas antes de chegar no mérito da discriminação.
Assim, focou-se em fornecer uma plataforma para dar maleabilidade ao
projeto e laborar a generalização do detector de eventos para as
condições de ruído que irão estar presentes nas redes elétricas
domiciliares. Em especial, pretende-se que isso seja realizado de
maneira automática, encontrando uma boa configuração sem ser
necessário realizar exaustivos testes em busca da mesma, independente
da estrutura dos dados sendo testados. A plataforma consiste-se de um
ambiente de análise \emph{a posteriori}, estando descrito no
Capítulo~\ref{chap:framework}. O conjunto de dados disponibilizados
pelo \acs{cepel} e o modo como o ambiente será aplicado para a
detecção de eventos de transitório estão disponíveis no
Capítulo~\ref{chap:metodologia}. Os seus resultados podem ser
encontrados no Capítulo~\ref{chap:resultados}. Porém, antes de chegar
nessas questões, a seguir serão realizadas considerações quanto à
escolha das características discriminantes, uma vez que as
características anteriormente utilizadas pelos estudos em conjunto com
o \acs{cepel} possuem informação discriminante apenas para mudanças de
estados com acréscimo no consumo.

\subsection{Quanto à Escolha das Características Discriminantes}
\label{ssec:caracteristicas}

Como foi observado anteriormente, há dúvidas quanto a utilização da
envoltória de corrente como característica discriminante. A primeira
consideração tratada leva em conta a taxonomia empregada para
discriminação, sendo essa muito abrangente. Esta subseção começa
tratando exatamente esse assunto.

O levantamento da taxonomia permite ter uma concepção do comportamento
da distribuição dos dados nas características utilizadas. Elas
oferecem um ponto de partida para evitar erros causados por \gls{c6},
ou seja, é possível estimar quais equipamentos terão padrões parecidos
utilizando tais características, começando assim, com uma configuração
onde esses equipamentos representam apenas uma única informação a ser
discriminada. Obviamente, a aglutinação de partículas não resolve o
problema quanto à discriminação específica para os elementos
aglutinados, mas possibilita obter uma informação com qualidade, ainda
que mais genérica. Se essa generalização for um problema, será
necessário procurar por características mais finas ou utilizar um
discriminador especializado para realizar a segregação posteriormente.
Em outros casos, a aglutinação pode ser realizada apenas para reduzir
a dimensão do problema, procurando apenas encontrar assinaturas que
permitam generalizar os equipamentos elétricos em grupos mais
abrangentes que são suficientes para a abordagem em questão. No ponto
de vista do \gls{nilm} quando aplicado com o objetivo de explorar a
\gls{ee} e/ou melhorar a precisão das informações das \glspl{pph}, a
generalização não é desejada já que é necessário a informação por
equipamento.

A taxonomia encontrada pelos estudos do \gls{cepel}
foi determinada empiricamente observando a forma das envoltórias e
estão na Tabela~\ref{tab:taxonomias_cepel}. Apenas como referência
para outros autores, foi adicionado também um outro levantamento
empírico realizado por \citet*{nilm_cepel_aguiar}, que se baseou nas
considerações de \cite{nilm_sultanem_1991_10}, para o plano
$I\times\theta$ (sendo $\theta$ o ângulo de fase). As principais
diferenças nas taxonomias entre os estudos no \gls{cepel} foram devido
às mudanças no sistema de aquisição de dados\footnote{A aglutinação
realizada por \citeauthor{nilm_cepel_bezerra} para equipamentos
resistivos e de ventilação foi realizada com o intuito de comparar o
seu método com outro utilizado pelo \gls{cepel}.}. Percebe-se que,
apesar de haver avanços na capacidade discriminante dos grupos, eles
ainda permanecem bastante genéricos, em especial para a aplicação do
\gls{nilm} em um programa de \gls{ee}. Avanços poderiam ser realizados
ao empregar discriminadores especializados, no entanto, não é possível
inferir qual seria a capacidade de discriminação para os diversos
grupos, nem até que ponto a envoltória pode ser discriminante para
disponibilizar informação desagregada de consumo.

Outro fato que aqui deve ser levado em conta é o aspecto levantado na
Subseção~\ref{ssec:cepel_anteriores}, onde se percebeu a dificuldade
da abordagem para discriminação do grupo de equipamentos eletrônicos.  As
considerações feitas por \citeauthor*{nilm_cepel_alvaro} revelam que
há dissimilaridade entre os padrões encontrados nos equipamentos.
Infelizmente o autor não detalhou a causa para a dissimilaridade, não
sendo possível saber se isso foi devido à aglutinação dos equipamentos
eletrônicos ser indevida por eles terem padrões diferentes,
e assim causando uma maior dificuldade ao discriminador para encontrar
um padrão nesse grupo, ou se os acionamentos desses equipamentos ocorrem
de maneira não bem-definida, por isso não sendo possível encontrar um
padrão. O único fato revelado é que há dissimilaridade entre os
acionamentos nesse grupo.

Independente da resposta para a questão anterior, as envoltórias podem
ser consideradas como características macroscópicas quando levando em
questão a \emph{informação contida em um período de onda}, sendo um nível
intermediário entre a informação de alteração no nível de consumo e
aquela contida dentro dos ciclos de onda. Percebe-se que os equipamentos
eletrônicos, por serem não-lineares, tem grande parte de sua
informação discriminante dentro do ciclo de onda. Essas
características --- na literatura foram utilizados harmônicos obtidos da
decomposição por \gls{fft}, níveis de detalhes das \glspl{tw}, curvas
I-V e, até mesmo, as ondas cruas --- poderiam ser utilizadas para
explorar a capacidade de discriminação dos equipamentos eletrônicos e
outras cargas não-lineares. Diferente da limitação das envoltórias,
onde a informação discriminante está disponível apenas para
acionamentos (ver
Figuras~\ref{fig:temporizado_geladeira}--\ref{fig:temporizado_televisao}
para exemplos), 
a informação discriminante para essas características
está disponível em todos os casos, aonde é
necessário apenas extrair a informação da diferença causada pela
mudança de estado dos equipamentos em pós e pré-transitório. Um exemplo
seria a própria taxonomia levantada por \citet{nilm_lam_2007_33} (para
mais detalhes ver pp.~\pageref{nilm:curvas_iv}), onde se percebe a
distribuição dos equipamentos eletrônicos em um grupo bem definido
(Grupo~7) e alguns outros equipamentos com grupos únicos para si. Vale
ter em mente que os autores não estavam interessados em generalizar
por equipamento, somente mostrar que é possível encontrar grupos de
equipamentos com características elétricas que façam sentido fisicamente,
e mesmo assim, alguns equipamentos tiveram assinaturas \emph{sui
generis}.

\begin{table}[p]
\resizebox{\textwidth}{!}{
\begin{tabular}{p{1cm}p{15cm}}
\hline \hline \hline
\multicolumn{1}{c}{\textbf{Grupo}} &
\multicolumn{1}{c}{\textbf{Equipamento}} \\

\hline \hline
\parbox[c]{16cm}{\centering\emph{Envoltória de corrente 
por \citet*{nilm_cepel_aguiar}} } \\ 
\hline \hline

\multicolumn{1}{c}{1} & Equipamentos de refrigeração em geral - refrigeradores,
freezers e condicionadores de ar \\
\multicolumn{1}{c}{2} & Equipamentos para aquecimento (resistivos) - chuveiros,
\emph{boilers}, ferros de passar roupas, fornos em geral \\
\multicolumn{1}{c}{3} & Equipamentos de agitação/movimento/mistura -
liquidificadores, batedeiras, máquinas de lavar, furadeiras \\
\multicolumn{1}{c}{4} & Equipamentos de ventilação - ventiladores e
circuladores \\
\multicolumn{1}{c}{5} & Equipamentos eletrônicos - televisores, microcomputadores,
equipamentos de som, videocassetes \\
\multicolumn{1}{c}{6} & Lâmpadas incandescentes \\
\multicolumn{1}{c}{7} & Lâmpadas fluorescentes \\

\hline \hline 
\parbox[c]{16cm}{\centering \emph{Envoltória de corrente 
por \citet{nilm_cepel_bezerra}}} \\
\hline \hline 
\multicolumn{1}{c}{1} & Equipamentos de refrigeração em geral - refrigeradores,
freezers e condicionadores de ar \\
\multicolumn{1}{c}{2} & Equipamentos resistivos (aquecimento) e
ventilação \\
\multicolumn{1}{c}{3} & Equipamentos de agitação/movimento/mistura -
liquidificadores, batedeiras, máquinas de lavar, furadeiras \\
\multicolumn{1}{c}{4} & Equipamentos eletrônicos - televisores, microcomputadores,
equipamentos de som, videocassetes \\
\multicolumn{1}{c}{5} & Lâmpadas incandescentes \\
\multicolumn{1}{c}{6} & Lâmpadas fluorescentes \\
\multicolumn{1}{c}{7} & Microondas \\

\hline \hline 
\parbox[c]{16cm}{\centering\emph{Envoltória de corrente 
por \citet*{nilm_cepel_alvaro}}} \\
\hline \hline 
\multicolumn{1}{c}{1} & Equipamentos de refrigeração em geral - refrigeradores,
freezers e condicionadores de ar \\
\multicolumn{1}{c}{2} & Equipamentos para aquecimento (resistivos) - chuveiros,
\emph{boilers}, ferros de passar roupas, fornos em geral \\
\multicolumn{1}{c}{3} & Equipamentos de agitação/movimento/mistura -
liquidificadores, batedeiras, máquinas de lavar, furadeiras \\
\multicolumn{1}{c}{4} & Equipamentos de ventilação - ventiladores e
circuladores \\
\multicolumn{1}{c}{5} & Equipamentos eletrônicos - televisores, microcomputadores,
equipamentos de som, videocassetes \\
\multicolumn{1}{c}{6} & Lâmpadas incandescentes \\
\multicolumn{1}{c}{7} & Lâmpadas fluorescentes de reator eletrônico \\
\multicolumn{1}{c}{8} & Lâmpadas fluorescentes de reator
eletromagnético \\
\multicolumn{1}{c}{9} & Máquinas de lavar \\
\multicolumn{1}{c}{10} & Monitores CRT \\
\multicolumn{1}{c}{11} & Monitores LCD \\


\hline \hline 
\parbox[c]{16cm}{\centering \emph{Plano I-$\theta$ por
\citet*{nilm_cepel_aguiar} baseado em 
\cite{nilm_sultanem_1991_10}}} \\
\hline \hline 
\multicolumn{1}{c}{1} & Refrigeradores e \emph{freezers} \\
\multicolumn{1}{c}{2} & Equipamentos resistivos (aquecimento) \\
\multicolumn{1}{c}{3} & Lâmpadas incandescentes \\
\multicolumn{1}{c}{4} & Motores universais (liquidificadores, batedeiras, aspiradores de
pó etc.) \\
\multicolumn{1}{c}{5} & Ventiladores e circuladores \\
\multicolumn{1}{c}{6} & Equipamentos eletrônicos e lâmpadas
fluorescentes \\
\multicolumn{1}{c}{7} & Ar condicionado \\
\hline \hline \hline

\end{tabular} }
\caption[Taxonomias utilizadas por autores de estudos anteriores
no CEPEL]{Taxonomias encontradas em estudos anteriores no \gls{cepel}.
Todas foram determinadas empiricamente.}
\label{tab:taxonomias_cepel}
\end{table}

\begin{table}[p]
\resizebox{\textwidth}{!}{
\begin{tabular}{p{1.3cm}p{17.7cm}}
\hline \hline \hline
\multicolumn{1}{c}{\textbf{Grupo}} &
\multicolumn{1}{c}{\textbf{Equipamento}} \\
\hline \hline
\parbox[c]{19cm}{\centering\emph{Variáveis padrão}}\\ 
\hline \hline 
\multicolumn{1}{c}{1} & Aparelhos eletrônicos em modo \emph{stand by}
- computadores \emph{desktop}, monitores \emph{LCD} \\
\multicolumn{1}{c}{2} & \emph{CD players}, geladeiras, carregadores de
bateria, desumidificadores \\
\multicolumn{1}{c}{3} & Mistura de equipamentos resistivos, eletrônicos e
com motores - jarros elétrico, lâmpadas fluorescentes de reator
eletrônico, lâmpadas incandescentes, ventiladores, computadores
portáteis, projetores, televisão de plasma, liquidificadores,
exaustores \\
\multicolumn{1}{c}{4} & Lâmpadas fluorescentes de reator
eletromecânicos, desumidificador \\
\multicolumn{1}{c}{5} & Cargas resistivas de potência média - secadores
de cabelo (potência média), aquecedores, ar condicionado, fogão de
indução (potência média) \\
\multicolumn{1}{c}{6} & Cargas resistivas de potência elevada e outros
equipamentos ligeiramente não-lineares - ferro de passar, aquecedor de
cabelo (potência alta), fogão de indução (potência alta) \\
\multicolumn{1}{c}{7} & Aspiradores de pó e forno microondas \\
\multicolumn{1}{c}{8} & Aspiradores de pó \\
\multicolumn{1}{c}{9} & Termo-ventilador (alta potência) \\
\multicolumn{1}{c}{10} & Computadores \emph{desktop} e impressora a jato
de tinta em \emph{stand by} \\
\multicolumn{1}{c}{11} & Computadores \emph{desktop} e projetores \\
\multicolumn{1}{c}{12} & Lâmpadas de econômia de energia, computadores
\emph{desktop}, impressora de tinta a jato em \emph{stand by},
televisores, \emph{video cassete}, \emph{scanner}, impressora a
\emph{laser} em \emph{stand by}, \emph{DVD player} ativo, televisores
\emph{stand by} \\
\multicolumn{1}{c}{Outros} & Secadores de cabelo operando na metade de
consumo e televisores em \emph{stand by} \\ 

\hline \hline 
\parbox[c]{19cm}{\centering\emph{Curvas I-V}} \\ 
\hline \hline 

\multicolumn{1}{c}{1} & Lâmpadas incandescentes, ventiladores,
aspiradores de pó, aquecedores, ar condicionado \\
\multicolumn{1}{c|}{1.1} & Lâmpadas incandescentes, ar condicionado \\
\multicolumn{1}{c|}{1.2} & Ventiladores \\
\multicolumn{1}{c|}{1.3} & Aspiradores de pó \\
\multicolumn{1}{c|}{1.4} & Televisores LCD \\
\multicolumn{1}{c}{2} & Lâmpadas fluorescentes com reator
eletromagnético, \emph{CD player}, carregadores de bateria,
geladeiras, desumidificadores \\
\multicolumn{1}{c|}{2.1} & Lâmpadas fluorescentes com reator
eletromagnético, geladeiras, desumidificadores \\
\multicolumn{1}{c|}{2.2} & \emph{CD player}, carregadores de bateria \\
\multicolumn{1}{c}{3} & \emph{CD player} e televisores LCD em
\emph{stand by} \\
\multicolumn{1}{c}{4} & Secadores de cabelo operando em baixa potência
\\
\multicolumn{1}{c}{5} & Forno microondas\\
\multicolumn{1}{c}{6} & Lâmpadas fluorescentes com reator eletrônico,
computadores portáteis, forno de indução \\
\multicolumn{1}{c|}{6.1} & Lâmpadas fluorescentes com reator eletrônico \\
\multicolumn{1}{c|}{6.2} & Computadores portáteis \\
\multicolumn{1}{c|}{6.3} & Fornos de indução \\
\multicolumn{1}{c}{7} & Computadores \emph{desktop}, televisores,
\emph{video cassete}, \emph{scanner}, impressoras a \emph{laser},
carregadores de bateria de telefones celulares \\
\multicolumn{1}{c}{8} & Aparelhos em modo \emph{stand by} -
computadores \emph{desktop}, monitores LCD, forno de indução \\
\multicolumn{1}{c}{Outros} & projetores, máquinas de lavar \\

\hline \hline 
\parbox[c]{19cm}{\centering\gls{svd}} \\ 
\hline \hline 

\multicolumn{1}{c}{1} & Mistura de equipamentos resistivos, indutivos e
outras ligeiramente não-lineares \\
\multicolumn{1}{c|}{1.1} & Lâmpadas incandescente, aquecedores, ar
condicionado \\
\multicolumn{1}{c|}{1.2} & Fornos microondas \\
\multicolumn{1}{c|}{1.3} & Secadores de cabelo (potência baixa) \\
\multicolumn{1}{c|}{1.4} & Lâmpadas fluorescente com reator
eletromagnético, geladeiras, desumidificadores \\
\multicolumn{1}{c|}{1.5} & Ventiladores, exaustor \\
\multicolumn{1}{c|}{1.6} & \emph{CD player} em \emph{stand by},
carregadores de bateria \\
\multicolumn{1}{c|}{1.7} & \emph{CD player} \\
\multicolumn{1}{c}{2} & Carregadores de bateria para telefones
celulares \\
\multicolumn{1}{c}{3} & Computadores, monitores LCD, televisão,
\emph{scanner}, impressora de tinta a jato \\
\multicolumn{1}{c}{4} & Aparelhos eletrônicos em modo \emph{stand by}
- computadores, monitores LCD, impressora de tinta a jato \\
\hline \hline \hline
\end{tabular} } 
\caption[Taxonomias determinadas por LAM et al.: curvas I-V,
decomposição SVD e variáveis padrão 
]{Taxonomias determinadas através de dendrograma por
\citet{nilm_lam_2007_33} (ver pp.~\pageref{nilm:curvas_iv}), tradução
própria. O grupo \emph{Outros} determina equipamentos com um grupo
único para si. Em alguns casos, o autor realizou a separação em
subgrupos, aqui indicados por uma barra vertical para facilitar a
identificação visual dos mesmos.}
\label{tab:taxonomias_lam}
\end{table}
 
Pode ser do interesse do \acs{cepel} alterar a amostragem dos
medidores dependendo, por exemplo, se for possível utilizar os
medidores inteligentes para garantir a impregnação do método, ou de
ser necessário melhores eficácias, nesse caso, sendo necessário
aumentar a frequência de amostragem. No primeiro caso,
ter-se-á acesso apenas a variáveis macroscópicas, como \acs{di}, \acs{dp}
e \acs{ds}, possivelmente outras variáveis podem ser empregadas, como
\acs{dq} e \acs{dd}, entretanto isso dependerá da capacidade do
medidor utilizado. Já para alta amostragem, 
frequências maiores à 60~\acs{hz} permitem a obtenção da informação
contida dentro do ciclo, possibilitando o acesso às informações
descritas no parágrafo anterior e, assim, uma maior capacidade de
discriminação. Nota-se que, como levantado na
pp.~\pageref{nilm:multiplas_tecnicas}, a escolha não precisa se limitar
a uma única característica ou técnica, podendo abranger-se para um
conjunto delas para obter melhor capacidade discriminativa. Nessa
alternativa, pode-se também explorar a envoltória das ondas uma vez que
elas mostram uma capacidade de, no mínimo, auxiliar na discriminação
de cargas lineares para os eventos de transitório de equipamentos durante
o acionamento. Essa capacidade pode estender-se para as
cargas não-lineares, porém não será tratado neste trabalho se isso
é verdadeiro, limitando-se aqui em apenas mostrar que a questão
está aberta. Além disso, a informação de estatística de uso, como foi
visto na discussão realizada para o levantamento bibliográfico (ver
Subseção~\ref{ssec:nilm_discussao}) é bastante discriminante e pode
auxiliar tanto na vertente de baixas amostragens, onde seu uso chega a
ser praticamente uma necessidade, ou quanto ao optar por uso de
diversas características, que mesmo não sendo necessária, a sua
utilização é fortemente indicada tomando como argumento os resultados
mostrados em
\cite{nilm_zeifman_vast_2011,nilm_zeifman_vastext_approach_2012,
nilm_zeifman_statistical_vastext_1stws_2012,
nilm_zeifman_vast_hisample_pdfmerge_2011,
nilm_zeifman_statistical_naive_enduses_2013}. Finalmente, existe a
possibilidade de utilizar submedidores, seja para reduzir a operação
justaposta dos equipamentos a uma quantidade aceitável, ou para obter
diretamente a informação desagregada de equipamentos complexos e
importantes em termos de uso-final
\cite{seminilm_ihome_tomek_2012,seminilm_fhmm_empiricalnmeter_2013,
seminilm_berges_multisensor_2010}.




  \chapter{Ambiente de Análise}
\label{chap:framework}

Inicialmente, no capítulo anterior, foi feito um levantamento dos
aspectos envolvidos no projeto do \acs{nilm}, levando em consideração
diversos os pontos técnicos para a aplicação do \acs{nilm} em um programa
de \acs{ee}. Em seguida, foi apresentado a evolução do projeto dessa
tecnologia no \acs{cepel}.  Observou-se a necessidade de detecção de
eventos de transitório para a operação do \acs{nilm} quando o mesmo
explora a informação presente na alteração de um estado operativo de
um equipamento. Essa etapa recebeu atenção do \acs{cepel} atualmente,
onde se decidiu aplicar um filtro de derivada de Gaussiana,
metodologia explicada em maiores detalhes na
Seção~\ref{ssec:met_cepel}. Aqui apenas será recapitulado que além da
aplicação de um limiar na resposta do filtro, há a remoção de
candidatos ruidosos e de candidatos próximos.

O trabalho atual firmou-se com o intuito de sistematizar o processo de
ajuste de parâmetros da metodologia proposta pelo \acs{cepel} e
estudar o comportamento da metodologia em uma nova base de dados que
contém a presença de ruídos causados pela dinâmica de carga
de alguns equipamentos operando na rede --- os equipamentos
\acs{c5} (ver definições na Subseção~\ref{ssec:modelos_carga}). 
Além disso, ao observar nos capítulos anteriores a complexidade do
projeto, percebeu-se a necessidade da implementação de uma
infraestrutura que permitisse unificar o projeto e facilitar o seu
desenvolvimento. Este capítulo se dedica à descrição dessa
infraestrutura --- o ambiente de análise. As necessidades para sua
implementação serão debatidas mais profundamente na
Seção~\ref{sec:motivacao_framework}. As seções seguintes
(\ref{sec:daq_info}--\ref{sec:otimizacao}) irão detalhar os módulos do
ambiente implementado. Houve diversas alterações na versão original,
trazendo benefícios operacionais e melhor capacidade de compreensão e
interação com os dados e análises, especialmente no aspecto gráfico.
Além disso, foram adicionados duas maneiras de remoção de eventos
(Subseção~\ref{ssec:evento}) e a construção do gabarito
(Seção~\ref{sec:gui}) --- estimativa da informação desagregada a ser
obtida --- que aqui são resumidos:

\begin{itemize}
\item Remoção de eventos próximos utilizando a média dos candidatos:
os candidatos dentro de uma janela deslizante são substituídos pelo valor
de sua média;
\item Remoção de eventos inconsistentes: remoção do evento caso os
sinais de entre a resposta do filtro e da variável \acs{di} sejam
opostos;
\item Construção do gabarito: a informação que se deseja obter é o
consumo desagregado de energia por equipamento, porém conseguir essa 
informação para o ajuste das técnicas não é simples, sendo necessário
realizar submedição em cada equipamento e anotar os instantes das
mudanças de estados de operação de cada equipamento. Em alguns casos
--- como os dados coletados pelo \acs{cepel} (descritos no próximo
capítulo) ---, apenas a segunda informação está disponível, de forma
que é necessário estimar o consumo desagregado dos equipamentos a ser
obtido nos arquivos. Para melhorar a estimativa de consumo a ser
obtida, contida no arquivo chamado de gabarito, possibilitou-se a
interação gráfica para a construção dessa estimativa.
\end{itemize}

A Seção~\ref{sec:otimizacao} apresenta a descrição da versão do
algoritmo genético implementado, uma adaptação de um \gls{es} para
possibilitar maior capacidade computacional às configurações que
melhor se adequam ao problema sendo otimizado. Os detalhes da
aplicação dessa versão adaptada do \gls{es} para o ajuste de
parâmetros será debatido mais adiante, no
Capítulo~\ref{chap:metodologia}.

\section{Da Necessidade}
\label{sec:motivacao_framework}

No capítulo anterior, foram observados diversos aspectos envolvidos
para o desenvolvimento da tecnologia do \gls{nilm} e a configuração do
projeto no \gls{cepel}. Com base nisso, as seguintes dificuldades
podem ser destacadas:

\begin{enumerate}[label={Item} \arabic* - ,ref=\arabic*,align=left]
\item\label{itm:dif1} era necessário melhorar a capacidade de
interpretação dos dados e da análise;
\item\label{itm:dif2} o \acs{cepel} pode julgar necessário alterar a
sua abordagem conforme as necessidades do projeto, como as técnicas
aplicadas, frequência de amostragem e o medidor;
\item\label{itm:dif3} a larga gama de técnicas encontradas no
levantamento bibliográfico, e em especial a chamada de atenção para o
fato de sua utilização em paralelo ser benéfica para a capacidade de
desagregação do \gls{nilm}, mostra que o projeto deve ter aptidão de
agregar em um único ambiente tudo aquilo que for desenvolvido, pois
mesmo que uma técnica não seja ótima, sua operação em paralelo pode
ser benéfica para o sistema de desagregação como um todo;
\item\label{itm:dif4} ainda que o item anterior não seja de interesse,
é importante manter todas as abordagens já desenvolvidas em um único
ambiente para garantir uma melhor evolução do projeto;
\item\label{itm:dif5} o levantamento bibliográfico mostrou que há uma
dificuldade dos autores em obter dados onde a informação desagregada
em energia também esteja disponível\footnote{Para fugir dessa
dificuldade, \cite{nilm_liang_pt2_2010_40} chegou a criar um simulador
de Monte-Carlo (ver pp.~\pageref{nilm:multiplas_tecnicas}).}. Tem-se uma
necessidade de tanto obter um meio para armazenar os momentos de
transição dos estados operativos para treinar (se existentes) técnicas
supervisionados, quanto ter o consumo desagregado para avaliar a
eficiência do \gls{nilm}. No exterior disponibilizaram-se dois
conjuntos de dados (ver Subseção~\ref{top:nilm_padrao}) justamente
com o intuito de possibilitar dados para autores e permití-los
compararem suas técnicas. Para obter a informação desagregada, pelo
menos uma das informações seguintes deve estar disponível, no entanto,
as duas se complementam, sendo desejável trabalhar com ambas: as
marcas caracterizando os momentos de alteração de estados;
e informação de consumo temporal dos equipamentos através de submedidores.
Porém, nem sempre é possível obter as duas informações juntamente.
Devido às dificuldades como o fato de não ser fácil monitorar os estados
operativos de alguns equipamentos, por não ter acesso nem controle de seus
ciclos de maneira trivial, bem como, nem sempre ser possível realizar a
submedição de todos equipamentos, estando essa informação parcialmente ou
até mesmo não disponível.
\end{enumerate}

Viu-se a necessidade do
desenvolvimento de um ambiente de análise que atendesse os seguintes
pontos:

\begin{enumerate}
\item Versatilidade: em vista dos itens
\ref{itm:dif2}--\ref{itm:dif4}, o ambiente deve ser capaz de se
adequar às mudanças no projeto conforme elas ocorram, sem que outras
partes do ambiente sejam afetadas. Devido a isso, optou-se por uma
implementação orientada a objeto, mas se limitando a escolha para uma
linguagem de amplo conhecimento no campo da engenharia. Por isso,
optou-se por desenvolver o ambiente no \emph{Matlab}, que
disponibiliza desde sua versão \emph{R2008a} essa capacidade.
Ainda que o \emph{Matlab} e, em especial, sua linguagem orientada a
objeto sofra em relação a sua performance, o ambiente de análise é
\emph{a posteriori} à coleta de dados, sendo aceitável essa
desvantagem. O interesse no ambiente é de explorar a capacidade das
técnicas, antes de implementá-las para operação em tempo real. O ambiente
foi organizado procurando modularizar os componentes, para que, se
fosse necessário o desenvolvimento ou adaptação de código,
simplificasse o processo para que apenas o módulo em questão seja 
atacado;

\item Capacidade de Interpretação dos Dados e Resultados: o tempo
investido neste ponto retorna em capacidade de interpretação, de forma
que o projeto irá ter um melhor andamento. Um dos aspectos para
atingir isso, é através de uma boa visualização \cite{it_depends}, já
que a mesma é um meio bastante efetivo para a comunicação da
informação (aqui se referindo a informação presente nos dados,
análise etc.). No caso, uma visualização dinâmica permite ainda melhor
compreensão das nuances contidas na informação;

\item Otimização dos Parâmetros: também considerando o
Item~\ref{itm:dif1}, seria interessante obter configurações ótimas de
maneira automática, sem a necessidade do usuário ficar alterando
parâmetros empiricamente até obter um valor considerado bom.
Ademais, que o algoritmo seja capaz de realizar isso encontrando uma
das melhores configurações possíveis para os parâmetros (não
necessariamente a melhor);

\item Estimação da Informação a ser Desagregada pelos Algoritmos
(Construção dos Gabaritos): já
quanto ao Item~\ref{itm:dif5}, os dados disponíveis no \acs{cepel} não
foram amostrados com submedição, tendo apenas acesso às marcas de
mudanças de estado operativo e, por isso, é necessário uma maneira para
estimar a informação desagregada contida nos mesmos. Com esse intuito,
aproveitou-se o ponto ``Capacidade de Interpretação dos Dados e
Resultados'', e adicionou-se essa capacidade na
visualização dinâmica oferecida ao usuário. Essa estimativa da
informação desagregada será referida neste trabalho como
\emph{gabarito}.

\end{enumerate}

O resultado foi um ambiente de análise com $\sim$29.000 linhas de
código distribuídas em $\sim$190 arquivos. Um esboço de sua
arquitetura pode ser observado na Figura~\ref{fig:ambiente_analise}.
As palavras em inglês representam as classes mais importantes dos
módulos como referidas no ambiente. Observa-se que o Módulo de
Leitura, Representação e Interação com os Dados
(Seção~\ref{sec:daq_info}) é a base do ambiente, sendo utilizado
para análise e otimização. O Módulo de Análise dos Dados
(Seção~\ref{sec:analise}) é executado pelo Módulo de Otimização dos
Parâmetros (Seção~\ref{sec:otimizacao}), que realiza diversas
análises iterativamente em busca de uma configuração ótima para os
dados alimentados. 

\begin{figure}[h!t]
\centering
\includegraphics[width=\textwidth]
{imagens/ambiente_de_analise.pdf}
\caption{Esboço do ambiente de análise implementado.}
\label{fig:ambiente_analise}
\end{figure}

É importante notar que apesar do esboço mostrar toda a cadeia para a
obter os parâmetros otimizados, essa não é sua única operação. O
usuário irá determinar sua operação, ex.: seja só para realizar uma
análise, obter os resultados e explorar graficamente a sua resposta,
construir um gabarito para um novo conjunto de dados, explorar os
dados a procura de alguma informação etc.

Um detalhe, a implementação foi realizada em inglês por opção do autor
do trabalho com o intuito de que o programa também seja compreensível
no exterior. Nem todas as informações puderam ser traduzidas para
colocá-las no trabalho, nesses casos será realizado a tradução e
referência aos elementos no texto do trabalho.

\section{Leitura, Representação e Interação com os Dados}
\label{sec:daq_info}

O Módulo de Leitura, Representação e Interação com os dados é o mais
complexo em termos de estrutura no ambiente implementado. Ele conta
com os seguintes segmentos:

\begin{itemize}
\item Dados do Medidor (Subseção~\ref{ssec:dados_medidor}):
representação em memória transitória dos dados do medidor. Uma série
de aspectos tiveram de ser tratados neste segmento;
\item Evento de Transitório (Subseção~\ref{ssec:evento}): contém a informação de
eventos de transitório. Essa representação pode ser criada tanto pelo
usuário durante a criação de um gabarito, ou seja, informar um evento
de transitório e suas propriedades a serem encontradas para avaliar a
performance de análise ou otimizar os parâmetros baseando-se nessa
informação, ou quanto pelo Módulo de Análise, que irá gerar essa
informação através de sua metodologia;
\item Equipamentos (Subseção~\ref{ssec:equipamento}): contém a informação
do estado de consumo dos equipamentos, seus consumo temporal estimado bem
como a estimativa de seu consumo total para o conjunto de dados.
Apesar deste trabalho ainda não ter tratado do problema de geração da
informação dos equipamentos (e por isso essa informação só ser gerada
pelo usuário para o gabarito), é interessante em termos de
continuidade do projeto que essa informação já fosse gerada nos
gabaritos, para que eles não precisem ser revisados no futuro,
contando com toda informação necessária para a otimização e avaliação
de performance. Como foi visto na Subseção~\ref{ssec:nilm_eff_calc} e
frisado diversas vezes em tal capítulo, é necessário dar a eficiência
do \gls{nilm} em termos de energia. Outro aspecto importante para a
motivação da criação dessa informação foi da capacidade de compreensão
dos dados, a informação por equipamento é muito mais intuitiva que os
eventos de transitório, constituindo em um nível mais alto informativo
para a compreensão dos dados, bem como facilitando a geração do
gabarito;
\end{itemize}

A seguir, entrar-se-á em mais detalhes para cada um deles.

\subsection{Dados do Medidor}
\label{ssec:dados_medidor}

Para atender as necessidades do projeto, a implementação da interface
para leitura e representação dos dados do medidor abordou os seguintes
tópicos:

\begin{itemize}
\item Transformação dos dados em um formato único: atualmente o
\gls{nilm} utiliza dados de dois medidores diferentes, sendo
necessário representar essa informação de uma única maneira para
atender a questão de Versatilidade. Um efeito colateral decorrente da
transformação foi a compressão dos dados, que estavam em formatos de
texto e ao serem armazenados em formato binário sofreram compressões de
30$\times$ a 40$\times$ dependendo do número de fases;
\item Robustez: a leitura e transformação dos dados para o formato
único deve ser robusta a possíveis erros durante a aquisição de dados,
sendo eles: descontinuidade da amostragem, seja por intervenção humana
ou algum problema no medidor; ou sobrecarga devido ao consumo
excessivo na rede, geralmente causado pelo acionamento de um equipamento
a motor de maior consumo, como o ar condicionado. Para o primeiro
caso, implementou-se um algoritmo capaz de identificar esses momentos,
e no caso da descontinuidade ser pequena (ex. 10~s, determinado pelo
usuário), a informação entre as bordas dos arquivos é completada com
amostras geradas através de um ajuste linear. Essas amostras são
marcadas para identificar que foram criadas e não medidas. Enquanto
para a sobrecarga, é grampeado o valor de consumo máximo para as
variáveis em que isso ocorre, bem como as amostras são marcas para
identificar os momentos em que isso ocorre;
\item Segmentação da memória persistente: alguns dados contém dias de
amostragens, sendo impossível analisar toda essa informação de uma vez
só em memória transitória. Por isso, segmentou-se os dados em diversos
arquivos com um tamanho pré-definido (ex. 1~hora). Anteriormente era
necessário segmentar a informação manualmente. A segmentação
automática foi realizada de maneira transparente durante a leitura da
base de dados, ou seja, a leitura ocorre sem que o usuário precise se
preocupar em como o conjunto de dados está representado e compreendido
no ambiente;
\item Redução da necessidade de leitura de disco: devido a segmentação
em memória, era necessário garantir que informações nas bordas dos
arquivos estivessem disponíveis para os algoritmos de análise sem que
eles tivessem de requisitar a troca da informação mantida em memória
transitória. Para isso, uma quantidade de amostras nas bordas dos
arquivos segmentados é mantida em memória transitória sempre
disponível, evitando que seja necessário uma navegação excessiva entre
a informação segmentada, reduzindo drasticamente a velocidade dos
algoritmos já que a leitura em disco é lenta;
\item Leitura de redes elétricas com até três fases: era necessário
compatibilidade de leitura de dados de redes monofásicas, bifásicas e
trifásicas, representando essa informação de uma maneira universal. Um
dos fatores que influenciou também na compressão dos dados foi armazenar
para as fases com pouca atividade somente os momentos em que havia
consumo;
\item Informação gráfica: representar graficamente a informação contida
nos dados. Essa funcionalidade é utilizada como base pela interface
gráfica para realizar a interação com os dados.
\end{itemize}

Um exemplo de dados trifásicos em uma residência
\emph{real}\footnote{A palavra real é empregada para identificar
amostragens não geradas em condições de laboratório.} está
disponível na Figura~\ref{fig:casa_real}. As três primeiras subfiguras
são a injeção de corrente (medido em valor eficaz) no sistema,
enquanto a última figura é o fluxo em potência trifásico para as
variáveis descritas na pp.~\pageref{eq:ipqds}. O fluxo de potência é
informado para o consumo trifásico porque o medidor \emph{Yokogawa}
utilizado nessa coleta de dados só permite o acesso a essa informação,
sendo necessário operar com um nível mais agregado de consumo que a
corrente. As barras verticais cinzas indicam como está realizada a
segmentação dos dados em disco. Apesar de não se utilizar esse
conjunto de dados para análise --- nesses dados não há como construir
o gabarito, não sendo possível a otimização dos valores, nem calcular
sua eficiência ---, ele revela uma série de aspectos importantes para
a compreensão do problema envolvido na desagregação do \gls{nilm}, bem
como algumas necessidades durante a implementação da parte de
representação dos dados no ambiente de análise. 

%\begin{landscape}
%\begin{figure}[h!p]
\begin{sidewaysfigure}[p]
\centering
\includegraphics[width=\textwidth]{imagens/RealHouse.pdf}
\caption[Informação gráfica para o interação com dados do medidor]
{Informação gráfica para a interação com os dados do medidor. Gráfico
gerado através do ambiente de análise para um conjunto de dados com
amostragem em 60~\acs{hz} de uma rede trifásica em uma casa
\emph{real} durante aproximadamente um dia de coleta. A injeção de
corrente para cada uma das três fases encontra-se nas subfiguras
superiores, enquanto o fluxo trifásico de potência entrando na rede
elétrica é representado na subfigura inferior. São utilizados as cores
azul, vermelho, verde e preto para as potências ativa, reativa,
harmônica e aparente, respectivamente.}
\label{fig:casa_real}
\end{sidewaysfigure}
%\end{figure}
%\end{landscape}

Quanto a questão da descontinuidade, há uma falha na medição próximo
às 07:15~h do dia~31, mostrando que o algoritmo foi capaz de perceber
essa falha e montar a descontinuidade. Já próximo às 19:20~h, ocorreu
uma outra descontinuidade menor, de 30~s, onde o algoritmo identificou
e uniu as bordas através de um ajuste linear para simular a
continuidade e recuperar a informação perdida. Enquanto isso, o ajuste para a
descontinuidade às 07:15~h não pode ser feito porque houve alterações
de estados operativos dos equipamentos\footnote{Estimar essas
alterações sem nenhuma informação é mais complexo do que a tarefa do
próprio \acs{nilm}, que realiza isso tendo a informação agregada de
consumo.}. Por isso, esse conjunto de dados seria analisado em duas
partes, uma partindo do início até às 07:15~h, e outra do fim da
descontinuidade até o fim da medição.

Também há nesse período a ocorrência de sobrecarga do medidor próximo
às 21:50 do dia 30 (provavelmente causada por um ar condicionado, ver
Figura~\ref{fig:sobrecarga}), onde o algoritmo foi capaz de
identificá-la e alterar os valores dessa amostras para a capacidade
máxima de medição.

Outras condições que se referiu no Capítulo~\ref{cap:nilm} também
podem ser observadas e melhores compreendidas nesse conjunto de dados.
Por exemplo, é possível observar na fase~A
Figura~\ref{fig:c5_ruido}\footnote{Foi necessário reduzir a qualidade
da Figura~\ref{fig:c5_ruido} para permitir a navegação na versão
digital deste trabalho, a figura vetorizada exigia grande capacidade
computacional.}
há a ocorrência de uma \gls{c5} a partir das
20:00~h injetando ruído na rede elétrica devido a sua dinâmica
(provavelmente esse equipamento é uma televisão). Já um exemplo típico de dinâmica causada
pela máquina de lavar roupa se encontra na
Figura~\ref{fig:maquina_lavar}. Apenas como curiosidade, também é
possível observar no período de menor atividade da rede ---
08:00~h--18:00~h do dia 31 --- nitidamente a operação da geladeira (ou
outro equipamento similar) na fase~B.

\begin{figure*}[p!]
  \begin{center}
    \begin{subfigure}[c]{\textwidth}
      \includegraphics[width=\textwidth,height=0.26\textheight]{imagens/RealHouse_ZoomSobrecarga.pdf}
      \caption{Sobrecarga do medidor causado por um equipamento na
        fase A.}
      \label{fig:sobrecarga}
    \end{subfigure}
    \hfill
    \begin{subfigure}[c]{\textwidth}
      \includegraphics[width=\textwidth,height=0.26\textheight]{imagens/RealHouse_maquina_lavar.eps}
      \caption{Um dos estados operativos da máquina de lavar
        roupa.}
      \label{fig:maquina_lavar}
    \end{subfigure}
    \hfill
    \begin{subfigure}[c]{\textwidth}
      \includegraphics[width=\textwidth,height=0.26\textheight]{imagens/RealHouse_aparelho_c5.jpg}
      \caption{Dinâmica de uma carga C5 na fase~A. Observe a diferença
entre a relação de sinal ruído dessa fase em relação com a fase~B.}
      \label{fig:c5_ruido}
    \end{subfigure}
  \end{center}
\caption[Alguns exemplos de dificuldades encontrados nos dados reais]{
Alguns exemplos de dificuldades encontrados nos dados reais da
Figura~\ref{fig:casa_real}.}
\label{fig:dificuldades}
\end{figure*}

%\begin{figure*}[ht!]
%  \ContinuedFloat
%    \begin{subfigure}[c]{\textwidth}
%      \label{fig:maquina_lavar}
%      \caption{Um dos estados operativos da máquina de lavar
%        roupa.}
%      \includegraphics[width=\textwidth,height=0.26\textheight]{imagens/RealHouse_maquina_lavar.eps}
%    \end{subfigure}
%
%  \caption{Casos destacados no conjunto de dados da Figura~\ref{fig:casa_real}.}
%\end{figure*}
%\FloatBarrier


\subsection{Evento de Transitório}
\label{ssec:evento}

Eventos são gerados tanto pelo usuário quando criando o gabarito ou
quanto pelo algoritmo de análise. A informação nos eventos de
transitório são de mais baixo nível àquelas contidas no objeto que
representa o equipamento. Suas capacidades são: 

\begin{itemize}
\item \acs{fex} para classificação: Realiza o cálculo das variáveis
\acs{di}, \acs{dp}, \acs{dq}, \acs{dd} e \acs{ds} e extrai uma janela
da envoltória dessas variáveis durante o transitório;
\item Remoção de eventos ruidosos: se \acs{di}, \acs{dp} ou \acs{ds}
forem menores a um limiar, o evento é considerado como ruidoso. O
corte pode ser realizado em apenas uma dessas variáveis, no momento o
corte é apenas em \acs{di}.
\item \textlabel{Remoção de eventos próximos pelo valor da média
dentro de uma janela deslizante}{text:media}: Para a
remoção de eventos próximos, adicionou-se uma outra maneira de realizar
a mesma. Ao invés de simplesmente ignorar a informação de outros
candidatos dentro de uma janela após um número determinado de
amostras, como realizado na metodologia proposta pelo \acs{cepel},
decidiu-se avaliar uma nova maneira de realizar essa tarefa,
empregando a média dos valores das amostras desses
candidatos. A inspiração para isso se deu pelo fato da maioria dos
eventos próximos removidos serem causados por um acionamento de
equipamento com um pico de consumo, assim, ao substituir o evento
causada pelo acréscimo de consumo e seu decréscimo logo em seguida por
sua média, obtém-se um evento mais próximo ao centro do distúrbio
causado pelo equipamento na rede. Um exemplo pode ser observado na
Figura~\ref{fig:analise_eventos}; 
\item \textlabel{Remoção de eventos
inconsistentes}{text:incosistentes}: ao observar que grande parte dos
eventos que eram removidos por serem eventos próximos na verdade eram
causados por eventos criados após um pico de consumo devido ao
acionamento de um equipamento, que gerava um evento na decréscimo de
consumo após o pico, decidiu-se adicionar um novo tipo de corte. Esse
corte remove os candidatos que tiverem o sinal da resposta do filtro
de derivada de Gaussiana invertido em relação ao degrau de consumo
causado pelo evento. Para esses eventos citados, a resposta do filtro
é negativa, enquanto o degrau é positivo, havendo assim inconsistência
entre os dois;
\item \textlabel{Estado dos eventos}{text:estados_eventos}: para
melhorar a capacidade de análise, eventos removidos não são excluídos,
tendo apenas seus estados alterados. Os possíveis estados dos eventos
são:
\begin{itemize}
\item Em bom estado;
\item Removido devido a evento próximo, nesse caso indicando qual
evento que causou sua remoção;
\item Evento ruidoso;
\item Inconsistente (ver item anterior);
\item Quantidade de amostras insuficientes, se não houver amostras
suficientes para construir o evento;
\item Ainda não preenchido, quando o evento é criado mas ainda é
necessário preencher a \gls{fex} e realizar os cortes para determinar
o seu estado.
\end{itemize}
\item Mudança de estado e equipamento: os eventos também armazenam a
informação de qual equipamento pertencem e qual foi a mudança de estado
por eles representadas. Por enquanto essa informação só é preenchida
pelo usuário durante a criação do gabarito;
\item Informação gráfica: os eventos são representados conforme o seu
estado (observar exemplos na Figura~\ref{fig:analise_eventos}).
São utilizadas linhas verdes verticais para indicar eventos de
transitório onde houve acréscimo no consumo, e linhas vermelhas para
decréscimo.  Eventos removidos possuem linhas cinza tracejadas.
Dependendo de como a geração gráfica é realizada, será criado uma área
cinza para a região onde será realizada a extração da envoltória e
duas faixas amarelas indicando as amostras para as quais será
calculado o valor pré/pós-transitório utilizados para calcular o
degrau, como indicado nas equações
\ref{eq:deltasMacro}.
\end{itemize}


\subsection{Equipamentos}
\label{ssec:equipamento}

A informação contida no equipamento une toda aquela contida nos eventos.
Nela, reconstrói-se todos os estados operativos do equipamento
temporalmente e seus consumos. No momento, essa informação só é gerada
pelo usuário quando preenchendo a informação do gabarito. Para ter uma
ideia de como o processo é realizado observe a
Figura~\ref{fig:gui_informacao}. As capacidades desse elemento são:

\begin{itemize}
\item Detecção automática de estados: quando gerando o gabarito,
faz-se uma análise de dendrograma para agregar os eventos de
transitório com informações de \acs{di}, \acs{dp}, \acs{dq}, \acs{ds}
próximas. O usuário só necessita alterar o nome dos estados pré/pós
transitório dos eventos agrupados, simplificando o processo de geração
do gabarito;
\item Informação gráfica: a capacidade de geração de informação dos
equipamentos é bem mais extensa que a dos eventos. Para evitar
redundância de informação, refere-se diretamente as figuras onde são
mostradas as características dos dados que foram gerados através do
método de informação gráfica de cada equipamento. Estes são os possíveis 
gráficos de serem gerados são:
\begin{itemize}
\item gráfico do consumo temporal por equipamento para os dados,
Figura~\ref{fig:temporizado_app_time}. Esse gráfico representa a
energia desagregada estimada a ser encontrada;
\item gráfico circular de consumo dos equipamentos,
Figura~\ref{fig:temporizado_app_pie};
\item gráfico das envoltórias para todos os eventos de transitório,
Figura~\ref{fig:temporizado_geladeira}. Essa informação auxilia a
encontrar eventos anômalos, como o evento destacado na
Figura~\ref{fig:temporizado_ventilador}.
\end{itemize}
\end{itemize}

\section{Interação Gráfica com o Usuário}
\label{sec:gui}

A interação gráfica com o usuário oferece uma melhor compreensão dos
dados. Além disso, este módulo também permite a capacidade de geração
do gabarito onde está toda informação considerada como alvo para o
\gls{nilm}, desde os momentos onde ocorreu os transitórios, até a
estimativa de informação de energia. Por enquanto o módulo só opera
com a informação dos dados do medidor, contudo sua expansão para
realizar a interação com a informação de análise não é complexa e
se pretende realizar sua implementação no futuro. A seguir estão
suas capacidades:

\begin{sidewaysfigure}[p]
\centering
\includegraphics[width=\textwidth]{imagens/Temporizado_gui_evento_sobreposto.png}
\caption[Informação gráfica para o Módulo de Interação Gráfica com os
Dados: Evento de Transitório com Sobreposição.]{Informação gráfica
para o Módulo de Interação Gráfica com os Dados: Evento de Transitório com
Sobreposição. A região com as amostras para o cálculo da média de pós
transitório está sobrepondo com outro evento.}
\label{fig:gui_evento_sobreposto}
\end{sidewaysfigure}

\begin{sidewaysfigure}[p]
\includegraphics[width=\textwidth]{imagens/Temporizado_gui_evento_sobreposto_consertado.png}
\caption[Informação gráfica para o Módulo de Interação Gráfica com os
Dados: Evento de Transitório Corrigido.]{
Informação gráfica para o Módulo de Interação Gráfica com os
Dados: Evento de Transitório Corrigido. A sobreposição foi corrigida
ao arrastar a região com o ponteiro, o que resulta em uma
estimativa de consumo mais fiel.}
\label{fig:gui_evento_sobreposto_corrigido}
\end{sidewaysfigure}

\begin{itemize}
\item Informação da amostra próxima ao ponteiro: funciona de maneira
muito similar à um medidor, mostrando os valores das amostras \acs{i}
(essa para cada fase), \acs{p}, \acs{q}, \acs{d}, \acs{s} para a
amostra mais próxima ao ponteiro (ver
Figura~\ref{fig:gui_evento_sobreposto}, na região superior à esquerda
denominada de \emph{Cursor Info}, ou informação do ponteiro em
português). Assim, o usuário pode comparar os
valores em cada amostra com facilidade;
\item Geração do gabarito: a dinâmica para a geração do gabarito
ocorre da seguinte maneira: o usuário
seleciona a opção ``Adicionar Evento'' (na figura estando em inglês:
\emph{Add Event}) e então seleciona a amostra que deseja ser o centro
do transitório. Nessa amostra é criado um evento, onde são calculados
os \acs{di}, \acs{dp}, \acs{dq}, \acs{dd}, \acs{ds}. Essa informação é
disponibilizada já calculada para o usuário no canto inferior da
esquerda, contendo tanto o valor do patamar operativo do 
pré/pós-transitório dessas variáveis. O algoritmo
detecta e agrupa os possíveis estados automaticamente por uma análise
em dendrograma. Conforme o usuário vai preenchendo a informação do
gabarito, a interface gráfica mostra-lhe o consumo desagregado por
equipamento e mantém-na atualizada para cada alteração realizada,
possibilitando o usuário saber se o gabarito está sendo preenchido de
maneira correta ou não. Cabe ao usuário determinar o nome do equipamento
e o nome de seus estados. Para equipamentos \acs{c3} a interface gráfica
já os cria automaticamente com o estados \emph{on} (ligado) e
\emph{off} (desligado). É possível selecionar qualquer evento e
alterar suas características, bem como arrastar as regiões aonde são
calculadas as variáveis (ver
Figura~\ref{fig:gui_evento_sobreposto_corrigido}) para corrigir
possíveis sobreposições de eventos ou regiões inicialmente mal
selecionadas.
\item Escolha da informação disponível: há a opção de escolher a
informação disponível na tela (observar
figuras~\ref{fig:temporizado_gui_eventos} e
\ref{fig:temporizado_gui_legenda}),
podendo controlar a disponibilidade gráfica das seguintes informações:
informação de consumo temporal dos equipamentos; centros dos eventos de
transitório (no caso, dos eventos criados pelo usuário no arquivo em
questão); e consumo amostrado pelo medidor. Isso é realizado porque a
junção de toda essa informação de uma só vez acaba criando confusão e
dificuldade para a sua interpretação, assim, com essas opções o
usuário tem controle sobre elas, permitindo que haja comparação das
informações sem que haja sobreposição delas;
\item Armazenar e carregar arquivos: como o gabarito na verdade é uma
estimativa da informação desagregada, é possível gerar diversos
gabaritos para um mesmo conjunto de dados. Por isso, há um mecanismo de
proteção da memória persistente e transitória. Ou seja, enquanto o
usuário altera a informação do gabarito, o mesmo permanece intacto em
disco, assim como se o usuário tentar encerrar a interface gráfica ou
carregar um outro gabarito e houver assincronia entre a informação
na memória persistente e transitória, ele será perguntado se deseja
descartar suas alterações ou sincronizar a informação;
\item Janela de legenda para o consumo dos equipamentos: a informação do
consumo temporal possui uma cor única para cada equipamento, porém
adicionar essa legenda na própria figura com os dados do medidor e
informação do gabarito não era possível sem gerar dificuldade para sua
interpretação, bem como não haveria espaço suficiente para colocar
todos os nomes dos equipamentos (espera-se cerca de 30 a 50 equipamentos nas
residências) na tabela sem comprometer a figura. Por isso, o usuário
tem a opção de abrir uma janela que irá conter essa legenda
(Figura~\ref{fig:temporizado_gui_legenda}), que será atualizada
automaticamente enquanto o usuário preenche o gabarito.
\end{itemize}

\begin{figure*}[p!]
  \begin{center}
    \begin{subfigure}[c]{\textwidth}
      \includegraphics[width=\textwidth]{imagens/Temporizado_gui_eventos.png}
      \caption{Informação disponível ao selecionar amostragem do
medidor e eventos de transitório para o conjunto de dados \emph{Temporizado}.}
      \label{fig:temporizado_gui_eventos}
    \end{subfigure}
    \hfill
    \begin{subfigure}[c]{\textwidth}
      \includegraphics[width=\textwidth]{imagens/Temporizado_gui_legenda.png}
      \caption{Informação disponível ao selecionar a informação
estimada de consumo desagregado por equipamento e a janela de legenda
para o conjunto de dados \emph{Temporizado}.}
      \label{fig:temporizado_gui_legenda}
    \end{subfigure}
  \end{center}
\caption{Informação gráfica para o Módulo de Interação Gráfica com os
Dados: Disposição da informação.}
\label{fig:gui_informacao} \end{figure*}



\section{Análise dos Dados}
\label{sec:analise}

O Módulo de Análise dos Dados foi implementado para refletir o ponto
inicial da metodologia empregada pelo \acs{cepel}, mas adaptando o
mesmo para o ambiente de análise, corrigindo possíveis pequenos
equívocos no código original e expandindo o mesmo. Suas aptidões
são:

\begin{itemize}
\item Continuidade da análise: a versão de ponto de partida gerava
um evento na leitura de cada novo arquivo. Isso ocorria porque o
consumo inicial em cada arquivo raramente é zero, e a inicialização
das condições do filtro pelo \emph{Matlab} é realizada considerando
que ele estava em repouso recebendo entradas nulas e respondendo
valores também nulos. Para dar a continuidade entre as segmentações
dos dados o próprio \emph{Matlab} fornece o vetor de condições finais
$\underline{z}_f$ no final da leitura de um arquivo para ser aplicado
na leitura do próximo arquivo como condições inicias
$\underline{z}_i$. Porém, no caso do conjunto de dados e nas possíveis
reinicializações da análise devido à descontinuidade das amostras, era
necessário determinar como simular a condição onde o filtro estava em
repouso (com respostas nulas em estado permanente) para a entrada com
o valor da amostra inicial. Por exemplo, no caso da
Figura~\ref{fig:casa_real} há uma descontinuidade às 07:15 do dia 31
causada por alguma falha na medição, que, ao executar a convolução do
filtro no período posterior a essa falha, irá gerar um evento de
transitório falso uma vez que haverá sensibilização na resposta do
\acs{fir} pelo motivo referido. 

Seja, assim, a determinação da resposta
$y(m)$ para um \acs{fir} dada por \ref{eq:y_m_fir}
\cite[pp.~311-312]{oppenheim}\footnote{Referência utilizada pelo
\emph{Matlab} na implementação do \acs{fir}.}, onde $b(k)$ é o k-ésimo
coeficiente do filtro de ordem $n-1$. Deseja-se encontrar o
vetor $\underline{z}_i$ que, dado o vetor de amostras de entrada
$\underline{x}$, gerará uma resposta $\underline{y}$ constante e
nula para toda janela do filtro. Ao representar \ref{eq:y_m_fir} em
forma matricial e igualando $\underline{y}=0$, obtém-se
\ref{eq:matrix_fir}. No caso, quer-se simular que o filtro estava em
repouso para a primeira amostra do arquivo, bastando fazer
$\underline{x}=a_1$, onde $a_1$ é a primeira amostra do conjunto de
dados ou a primeira amostra após uma descontinuidade. Assim,
utiliza-se \ref{eq:matrix_fir} com o valor de $a_1$ para todo o vetor 
$\underline{x}$;
\begin{subequations}
\begin{eqnarray}\label{eq:y_m_fir}
y(m) = b(1)x(m)+z_1(m-1)  \nonumber \\
z_1(m) = b(2)x(m)+z_2(m-1)  \nonumber \\
\;\;\vdots\;\;\;\;\;\; =
\;\;\;\;\;\;\;\vdots\;\;\;\;\;+\;\;\;\;\;\;\vdots\;\;\;\;\;\;\;  \\
z_{n-2}(m) = b(n-1)x(m)+z_{n-1}(m-1)  \nonumber \\
z_{n-1}(m) = b(n)x(m) \nonumber
\end{eqnarray}
\begin{equation} \label{eq:matrix_fir}
\underline{z} = 
\underbrace{\begin{bmatrix}
b(n-1) & b(n-2)   & b(n-3) & \dots & b(1) \\
       & b(n-2)   & b(n-3) & \dots & b(1) \\
       &          & b(n-3) & \dots & b(1) \\
       &\mathbf{0}&        & \ddots & \vdots \\
       &          &        &        & b(1) \\
\end{bmatrix}}_{\mathbf{B}'}\underline{x}
\end{equation}
\end{subequations}

\item Pareamento da resposta do \acs{fir} com os dados: como a análise
é \emph{a posteriori}, para facilitar a interpretação, remove-se o
atraso na resposta do \acs{fir} para que a mesma fique alinhada com a
amostra sendo analisada, facilitando a compreensão do problema;

\item Janela do filtro com apenas com valores relevantes: o tamanho da
janela do filtro tem influência direta no tempo de execução do
algoritmo, uma vez que a convolução será realizada com uma janela
maior de pontos para cada uma das amostras analisadas. Diminuir a
janela em duas amostras significa que, para cada amostra no arquivo,
serão realizados pelo menos dois cálculos a menos. Assim, além de dar
um valor limite para o tamanho da janela do filtro, é necessário podar
o mesmo para conter apenas valores relevantes, desconsiderando pontos
com grandeza irrelevante. Para isso, realiza-se um corte para pontos
com ordem de grandeza $10^{-3}$ menores que o ponto de maior
relevância do filtro. Apenas com esse corte já foi possível obter
grandes reduções no tamanho da janela e consequentemente no tempo de
execução do algoritmo;

\item Reconhecimento de múltiplas análises com mesma configuração:
suponha que se deseje realizar quatro análises, duas delas com um filtro com
$\sigma=2$ e valores de corte \acs{di}$_{min}=0,1$ e
\acs{di}$_{min}=0,2$, enquanto os outros dois filtros irão ter 
$\sigma=3$ com os mesmos cortes. Nesse caso, não é necessário gerar a
respostas dos filtros de derivada de Gaussiana quatro vezes, apenas
duas vezes e então aplicar os dois cortes em cada uma dessas
respostas. O ambiente de análise realiza a identificação dessas
ocorrências quando executando múltiplas análises e as executa apenas
uma vez, partilhando essa memória para os algoritmos seguintes; 

\item Informação gráfica: a Figura~\ref{fig:analise_eventos} contém um
exemplo de gráfico gerado para uma análise. É possível observar a
resposta do filtro de derivada de Gaussiana, as regiões
sensibilizadas, os eventos, suas janelas para cálculo das \acs{fex} e
os seus estados (ver pp.~\pageref{text:estados_eventos}) informando se
os mesmos foram aceitos ou foram eliminados devido a algum dos cortes.

\end{itemize}


\begin{sidewaysfigure}[p]
\centering
\includegraphics[width=.8\textwidth]{imagens/Empilhado7_ex_incosistencia_e_media.pdf}
\caption[Exemplo de informação gráfica para o Módulo de Análise dos
Dados.]{Exemplo de informação gráfica para o Módulo de Análise dos
Dados. Na subfigura inferior, as regiões verdes e vermelhas indicam
regiões sensibilizadas por respostas positivas e negativas,
respectivamente. A resposta para o filtro de
derivada de Gaussiana é representado pela linha pontilhada, enquanto a
linha horizontal cinza é o limiar de corte para a geração de uma região
sensibilizada. É possível observar um caso de evento inconsistente e
outro removido devido a evento próximo. Para o caso do evento inconsistente, em azul, seu
degrau de potência é positivo enquanto sua resposta é negativa,
revelando sua inconsistência. Já os eventos próximos representados
pelas caixas amarelas foram removidos por estarem próximos, sendo
substituídos pela sua média (a linha verde). Caso esses eventos não
fossem removidos, eles seriam constituídos de falsos alarmes. Na
subfigura superior é possível observar as regiões que serão utilizadas
para a extração do transitório (região cinza) e as regiões utilizadas
para calcular o degrau de potência (regiões amarelas
pré/pós-transitório). } 
\label{fig:analise_eventos}
\end{sidewaysfigure}


\section{Otimização dos Parâmetros}
\label{sec:otimizacao}

Retornando agora para o esboço do ambiente
(Figura~\ref{fig:ambiente_analise}), observa-se que o Módulo de
Otimização dos Parâmetros utiliza o Módulo de Análise iterativamente
para obter parâmetros ótimos (não necessariamente o ótimo global).
A otimização é feita em posse do gabarito, onde os parâmetros da
análise são alterados pelo otimizador em busca de valores que melhor
adequem a resposta da análise em relação ao gabarito. Os seguintes
tópicos resumem como é realizada a otimização dos parâmetros:

\begin{itemize}

\item Redução de memória transitória: um detalhe operativo, as
análises geradas pelo usuário mantém a informação da resposta do
filtro de Gaussiana após o término da análise pois essa informação é
necessária para gerar a informação gráfica, porém, no caso do
otimizador diversas análises serão geradas, sendo interessante limitar
ao máximo o consumo de memória em cada uma delas. Por isso, as
análises geradas pelo algoritmo remove qualquer informação irrelevante
durante a execução da análise como a resposta do filtro e qualquer
outros elementos não necessários, mantendo apenas os eventos de
transitório;

\item Regras de Pontuação: para realizar a otimização, é necessário
haver uma regra para o otimizador avaliar aquilo que é desejável do
que não é. É natural que essa regra tome como recompensa identificar
corretamente um evento de transitório e como punição gerar um evento
de transitório aonde essa informação não existe. O primeiro caso é
referido de detecção, enquanto o segundo consiste de um falso positivo
ou falso alarme. Também é possível adicionar outras punições, por
exemplo, não é desejável que sejam gerados muito candidatos para serem
removidos, isso irá reduzir a velocidade de processamento e se uma
solução consegue realizar a detecção com acurácia parecida mas gerando
menos candidatos, é preferível optar por essa opção, mesmo que ela
tenha uma performance ligeiramente reduzida para obter uma performance
de execução quando aplicando o algoritmo em um \gls{nilm} operando em
tempo real. Porém, essa punição deve ser pequena, por não ser a
grande questão a ser otimizada. A regra de pontuação utilizada está
representada em \ref{eq:regra_pontuacao}. Ainda, os eventos detectados
pelo Módulo de Análise não estarão exatamente na mesma posição que os
eventos gerados pelo usuário no gabarito, já que o usuário não
necessariamente irá marcar a amostra que constitui o ponto de inflexão da
resposta do filtro e, mesmo que isso ocorra, os eventos podem sofrer
deslocamento quando empregando a remoção de eventos próximos por sua
média. Assim, é preciso definir uma janela de um número máximo de
amostras \acs{jmax} para os quais se aceitará o evento de detecção. O
casamento do evento da resposta da análise com o do gabarito é feito
quando os dois se encontram no máximo à \acs{jmax} amostras de
distância. Caso existam mais de um evento dentro dessa janela, o
casamento é realizado com o evento mais próximo;
\begin{equation}\label{eq:regra_pontuacao}
\textbf{Aptidão}=\gamma_{det}N_{det}+\gamma_{fa}N_{fa}+\gamma_{rem}N_{rem}
\end{equation}
\noindent onde:
\begin{description}
\item[$\text{Aptidão}$] mede a capacidade de resposta da análise realizada,
sendo de interesse maximizar esse valor.
\item[$\gamma_{det}$] é a pontuação que a análise recebe para cada
evento de detecção;
\item[$N_{det}$] é a quantidade de eventos detectados;
\item[$\gamma_{fa}$] é a pontuação que a análise recebe para cada
ocorrência de falso alarme;
\item[$N_{fa}$] é a quantidade de ocorrências de falso alarme;
\item[$\gamma_{rem}$] é a pontuação que a análise recebe para cada
ocorrência de candidatos removidos;
\item[$N_{rem}$] é a quantidade de candidatos removidos.
\end{description}

\item Comparação da resposta da análise com o gabarito: em posse da
regra de pontuação \ref{eq:regra_pontuacao} e a \gls{jmax}, realiza-se a
comparação entre as duas informações e retorna-se a eficiência em termos
de Aptidão. Como dito no item anterior, \gls{jmax} é necessário para
casar os eventos de transitório da resposta do filtro com aqueles
presentes no gabarito;

\item Escolha do algoritmo: a função a ser otimizada não é
diferenciável e portanto não é possível utilizar os métodos
convencionais de otimização. É necessário empregar algum método de
tentativa e erro para realizar essa tarefa. Neste trabalho, optou-se
pela utilização de um \acs{es}, porém outras estratégias de
otimização podem ser utilizadas, como Inteligência de Enxame. 
Aprofundar-se-á nas características do algoritmo
implementado na Subseção~\ref{ssec:es};

\item Capacidade de recuperação do processo caso ocorra alguma falha: 
como as otimizações podem levar dias para serem executadas, viu-se a
necessidade de armazenar informações relevantes do processo enquanto
ele evolui, para garantir que se ocorresse algum problema na máquina
em execução, não houvesse de recomeçar o processo desde a primeira
geração. Assim, a versão do algoritmo é capaz de armazenar, se o
usuário requirir, a evolução do processo de otimização e retornar caso
o processo seja interrompido --- por exemplo, devido à falta de
energia. 

\end{itemize}


\subsection[Algoritmo Genético de Estratégia Evolutiva]{\acf{es}}
\label{ssec:es}

Foi realizada a implementação de uma versão própria de um \acs{es} com
base em \cite[cap. 4]{eiben2003introduction}, mas antes de entrar em
detalhes sobre a versão implementada cabe introduzir o tema sobre
algoritmos de estratégia evolutiva.

\begin{figure}[h!t]
\centering
\includegraphics[width=.9\textwidth]
{imagens/ga.pdf}
\caption[Esboço de um algoritmo evolutivo genérico.]
{Esboço de um algoritmo evolutivo genérico. Baseado em
\cite[pp. 17]{eiben2003introduction}.}
\label{fig:esboco_ga}
\end{figure}

Dispõe-se na Figura~\ref{fig:esboco_ga} a sequência de otimização de
um algoritmo evolutivo genérico. Ela conta com uma população inicial,
onde se recomenda que a mesma seja iniciada aleatoriamente.  Essa
população inicial passará por um processo de seleção parental aonde
serão obtidos os espécimes ou indivíduos para compor a \gls{mu},
população que irá realizar a propagação de sua informação genética
para a \gls{lambda}. Porém, a \acl{lambda} recebe o material
perturbado através de operações de recombinação e mutação. A mutação é
uma pertubação que ocorre somente levando em conta o material de um
único pai, enquanto a recombinação ocorre no mínimo com dois pais. A
capacidade de encontrar novas regiões promissoras (genes alelos) ---
descoberta --- no espaço de busca de soluções e, com isso, adquirir
informação no problema é dada pela mutação através de pertubações
aleatórias. Já a recombinação realiza a otimização dentro de uma área
promissora (no caso, dentro da informação genética dos pais) ---
exploração ---, dando grandes pulos para uma região dentro de duas
áreas. O resultado do material genético perturbado constitui da base
para a geração da prole. Irá ocorrer, então, a seleção dos
sobreviventes, que pode levar em consideração a prole e os pais nesse
processo ($\mu$+$\lambda$) ou apenas a prole ($\mu$,$\lambda$). Os
indivíduos selecionados são a nova geração da população, e o ciclo de
será repetido gerando novas populações até o processo atender uma
condição de parada. Geralmente essa é através de um número máximo de
gerações.

Cabe ainda definir a diferença entre genótipo e fenótipo do ponto de
vista computacional. A representação em genótipo é aquela que sofre as
pertubações e codifica a representação do espécime, enquanto o
fenótipo representa a informação como é demonstrada pelo espécime para
o problema em questão, ou seja, o espaço de solução. Pode haver
diferença entre as duas representações ou não, por exemplo, a
representação \{Norte,Leste,Sul,Oeste\} seria as possíveis
representações do fenótipo, e sua codificação em genótipo poderia ser
feita em \{1,2,3,4\}. Nota-se a importância de não
só saber representar a informação em genótipo para realizar as
operações de pertubação no material, como possuir uma maneira de
decodificá-la novamente no espaço do fenótipo. Para isso, cada solução
do fenótipo deve ser mapeável, bem como cada genótipo tenha a sua
decodificação em apenas uma solução. Ainda, a escolha da representação
irá afetar o problema: no exemplo citado a representação escolhida
não parece ter o significado do fenótipo, uma vez que Norte e Oeste são
vizinhos entre si, enquanto na representação eles estão distantes de três
unidades. Uma escolha em variáveis cíclicas parece representar o
problema fidedignamente
$\{(\text{sen}(\frac{1\pi}{2}),\cos(\frac{1\pi}{2})),
(\text{sen}(\frac{2\pi}{2}),\cos(\frac{2\pi}{2})),
(\text{sen}(\frac{3\pi}{2}),\cos(\frac{3\pi}{2})),
(\text{sen}(\frac{4\pi}{2}),\cos(\frac{4\pi}{2}))\}$, porém, nesse
caso, o genótipo seria representado em duas variáveis.

\subsubsection{Versão original}

O \acs{es} é um algoritmo genético cuja especialidade é a
autoadaptação de sua estratégia evolutiva. O \acs{es} concentra sua
capacidade evolutiva na mutação, enquanto outros algoritmos genéticos
costumam focar na recombinação. Todos os indivíduos passam por
pertubações Gaussianas em seu material genético, no entanto, a ordem
dessas pertubações são ajustadas conforme a evolução da espécie,
aumentando ou diminuindo sua ordem de acordo com as necessidades de
evolução. Assim, o \acs{es} irá aumentar a ordem de suas pertubações
quando distante de um valor ótimo --- sujeito a capacidade de perceber
uma tendência no espaço de solução apontando na direção de um ótimo
--- e reduzir as pertubações conforme se aproxima desse valor para
realizar o ajuste fino. Há também uma pequena taxa de recombinação
para aumentar a velocidade de convergência, normalmente na faixa de
10\%\footnote{As taxas de recombinação e mutação são dadas em termos
de probabilidade de ocorrência.}. A seleção parental não é influenciada pela aptidão
dos indivíduos, quando utilizando taxa de recombinação a escolha dos
pais para haver troca de material genético é realizado de maneira
aleatória uniforme. A melhoria gradual das gerações é realizado pela
seleção dos sobreviventes, que é realizada através de
($\mu$,$\lambda$) com pressão alta de seleção, aonde apenas os
indivíduos mais aptos permanecem para a próxima geração. Entende-se como pressão
de seleção dos sobreviventes a grandeza descrita por
\ref{eq:pressao_selecao}. É importante a seleção
através de ($\mu$,$\lambda$) em comparação com a ($\mu$+$\lambda$)
para evitar ótimos locais, bem como garantir que não haverá propagação
de indivíduos com estratégias mal-adaptadas através das gerações. O
mesmo se dá para versões que utilizam elitismo --- manter ao menos uma
cópia do membro mais apto da população na próxima geração ---, não
sendo recomendado no \acs{es} pelo mesmo problema da seleção parental
através de ($\mu$+$\lambda$). É importante também que a seleção de
sobreviventes aplique uma alta pressão de seleção, garantindo a
capacidade adaptativa do \acs{es}, normalmente utilizando
$\frac{\lambda}{\mu}=7$.

\begin{equation}\label{eq:pressao_selecao}
\text{Pressão de Seleção} = \dfrac{\lambda}{\mu}
\end{equation}

Optou-se pela mutação descorrelacionada com $n$ tamanhos de passo
\cite[pp. 76--78]{eiben2003introduction}. Nessa configuração, cada
variável representado no genótipo tem sua própria pertubação, sendo
descrita por \ref{eq:s_esbegin}, enquanto sua pertubação é adaptada
anteriormente de acordo com \ref{eq:sigma_esbegin}. A taxa de
aprendizado é divida em dois parâmetros, $\tau$ e $\tau'$, onde aquele
é a base de aprendizado, que garante uma mudança global na
mutabilidade para preservar os graus de liberdade do problema, e este
é uma mutação específica por coordenada, fornecendo flexibilidade para
empregar diversas estratégias de mutação em diferentes direções. Os
valores indicados para ambos são $1/\sqrt{2n}$ e $1/\sqrt{2\sqrt{n}}$,
respectivamente. O alcance da pertubação no material genético, dado uma
determinada probabilidade de ocorrência fixa, formam elipsoides no
espaço de solução alinhadas com os eixos da representação escolhida. 

\begin{subequations}
\begin{equation}\label{eq:s_esbegin}
x_i' = x_i\sigma_i'N_i(0,1)
\end{equation}
\begin{equation}\label{eq:sigma_esbegin}
\sigma_i' = \sigma_ie^{\tau'N(0,1)+\tau N_i(0,1)}
\end{equation}
\end{subequations}

\noindent onde: 

\begin{description}
\item[$N(0,1)$] e $N_i(0,1)$ são uma pertubação Gaussiana com média
zero e $\sigma$ unitário, a primeira sendo um único valor para todas
as representações, enquanto a segunda uma para cada representação $i$; 
\item[$x_i$] e $x_i'$ é a i-ésima representação e a mesma após sofrer a
pertubação;
\item[$\sigma_i$] e $\sigma_i'$ é a i-ésima estratégia evolutiva e a
mesma após sofrer a pertubação.
\end{description}

Para a recombinação, implementou-se a versão local da mesma por ela
ser mais simples de elaborar, pretendendo alterar no futuro para
a recombinação global que é mais indicada para o caso do \acs{es}.
Porém, a recombinação utilizada não é o quesito principal do
algoritmo, servindo apenas para melhorar a velocidade de convergência.
No caso implementado, o material genético é mesclado entre apenas
dois indivíduos. A informação genética que representa o espaço de
solução é misturada através de \ref{eq:rec_discreta} --- recombinação discreta
---, enquanto a versão para a estratégia evolutiva é realizada através
de \ref{eq:rec_intermediaria} --- recombinação intermediária.

\begin{subequations}
\begin{equation}\label{eq:rec_discreta}
\left\{\begin{array}{l}
x_{i,1}' = x_{i,1} \;\; \textit{ou} \;\; x_{i,2} \;\;\;\;\;\; \text{escolhidos
aleatoriamente}\\
x_{i,2}' = x_{i,\text{o.c.}}
\end{array}\right.
\end{equation}
\begin{equation}\label{eq:rec_intermediaria}
\left\{\begin{array}{l}
\sigma_{i,1}' = \dfrac{(\sigma_{i,1}+\sigma_{i,2})}{2} \\
\sigma_{i,2}' = \dfrac{(\sigma_{i,1}+\sigma_{i,2})}{2}
\end{array}\right.
\end{equation}
\end{subequations}

\noindent onde:

\begin{description}
\item[$x_{i,1}$] e $x_{i,2}$ são a i-ésima representação para o
primeiro e segundo pai, respectivamente; 
\item[$x_{i,o.c.}$] é a i-ésima representação para o pai não
selecionado para $x_{i,1}'$;
\item[$\sigma_{i,1}$] e $\sigma_{i,2}$ são a i-ésima estratégia
evolutiva para o primeiro e segundo pai, respectivamente;
\item[$x_{i,1}'$] e $x_{i,2}'$ são a i-ésima representação para o
primeiro e segundo pai após a pertubação, respectivamente; 
\item[$\sigma_{i,1}'$] e $\sigma_{i,2}'$ são a i-ésima estratégia
evolutiva para o primeiro e segundo pai após a pertubação,
respectivamente.
\end{description}

\begin{figure}[h!t]
\centering
\includegraphics[width=.9\textwidth]{imagens/Ackley.png}
\caption{Função de \emph{Ackley} em duas dimensões para $-30,0< x_i <
30,0$.}
\label{fig:funcao_ackley}
\end{figure}

A referência \cite[pp. 84]{eiben2003introduction} cita um exemplo de um
outro autor que aplicou o \acs{es} para a função de \emph{Ackley},
onde foi utilizado \acs{mu} $= 30$, \acs{lambda} $= 200$, $x_i$
inicial entre $-30,0 < x_i < +30,0$ e um total de 200.000 avaliações
da função de aptidão. Para um total de 10 execuções, o exemplo citado obteve a
melhor solução com valor da função de $7,48\times10^{-8}$. A função de
\emph{Ackley} (Figura~\ref{fig:funcao_ackley}) é altamente multimodal,
com um grande número de mínimos locais, mas com apenas um máximo
global $\overline{x}=0$ e seu valor $f(\overline{x})=0,0$. Para
validar o algoritmo implementado, executou-se o \acs{es} para essas
configurações, obtendo uma ocorrência de convergência para mínimo
local com o valor de $1,34$, e todas as outras ocorrências dão uma
aptidão de média de $1,87\times10^{-7}\pm2.56\times10^{-7}$, onde a
melhor solução obtém a aptidão $1,34\times10^{-8}$. A evolução da
execução com melhor convergência está na Figura~\ref{fig:es_standard},
mostrando o valor médio de aptidão na geração da população na
convergência. Apesar do exemplo citado informar que todos os mínimos
encontrados foram os globais, na versão aqui implementada, ocorrem
casos em que não há a convergência para o mínimo global, ainda que em
outras execuções seja possível encontrar todas as 10 minimizações
convergindo para o mínimo global. É importante ter em mente que a
convergência não ocorre necessariamente para o mínimo global, sendo
uma propriedade bem conhecida dos algoritmos genéticos.  De qualquer
forma, os valores obtidos estão próximos da referência e existem
parâmetros não informados como o valor mínimo de pertubação
$\sigma_{min}$ e o valor inicial para as pertubações
$\sigma_{inicial}$ que podem influenciar na resposta.  Apenas como
referência, os valores utilizados para esses casos foram
$\sigma_{min}=1\times10^{-9}$ e $\sigma_{inicial}=1$.



\begin{figure}[h!t]
\centering
\includegraphics[width=\textwidth]{imagens/es_standard.pdf}
\caption[Evolução para o melhor individuo para a validação da versão
original do ES]{Evolução para o melhor individuo para a validação da versão
original do \acs{es}. O objetivo é minimizar a função de
\emph{Ackley}, que pela visão do \acs{es} funciona como maximizar a
função com seus valores opostos. Por isso, os valores mostrados são
negativos.}
\label{fig:es_standard}
\end{figure}

\subsubsection{Versão Multiespécie}
\label{sssec:multiespecie}

Poderia ser utilizada a versão original do \acs{es} para realizar a
otimização dos parâmetros necessários na abordagem do problema. No
entanto, devido à decorrência da dúvida quanto a qual caminho
percorrer para a análise (ordem de remoção de eventos e quais delas
empregar), são necessárias diversas execuções
do algoritmo para otimizar os valores de maneira individual. Ao invés
de executar cada uma delas individualmente, motivado pela ideia de otimização
multiobjetivo \cite[cap. 9]{eiben2003introduction}, decidiu-se utilizar
a ideia de subpopulações --- que serão referidas por espécies, por
poderem ter cromossomos diferentes dependendo da configuração
utilizada ---, mas aplicando a mesma para um outro conceito. No caso,
ao invés de utilizar espécies para otimização de múltiplas funções
objetivo, esse conceito irá ser utilizado para criar uma dinâmica
entre as várias otimizações sendo realizadas no problema.

A ideia da dinâmica é reservar o esforço computacional para aquelas
abordagens que estão mostrando capacidade de resolver o problema com
maior aptidão, revelando-se uma espécie mais adequada para o
\emph{habitat} em que os espécimes estão sendo avaliados. Porém, isso
deve ser realizado sem comprometer a evolução de espécies que, por
algum motivo, sofreram desvantagem durante o processo evolutivo ---
seja por uma inicialização em condições desprivilegiadas, ou por uma
demora maior para ajustar sua estratégia evolutiva. Assim, a proposta
é executar apenas uma otimização para as diferentes maneiras de
tratar o problema, aonde todas as configurações desejadas irão
competir entre si de modo que o algoritmo irá reservar maior
esforço computacional para otimizar mais profundamente aquela que se
melhor adéqua ao espaço de solução, diferente da versão aonde se
executaria para cada espécie, reservando esforço computacional igual
para espécies que não têm se mostrado adequadas para a solução do
problema.

Assim, propuseram-se duas configurações para as competições dos
espécimens:

\begin{itemize}
\item Interespécie: nesse caso há cooperação entre os individuos de
uma mesma espécie. É calculada a aptidão para cada espécie
(\ref{eq:aptidao_especie}) a ser utilizada como parâmetro na
competição das mesmas e determinar a parcela da população global que
elas tem direito. Uma vez determinado o tamanho da população de cada
espécie, seus indivíduos irão competir entre si para determinar os
sobreviventes. É necessário escolher um método para avaliar a aptidão
das espécies e como determinar suas populações através dele;
\item Intraespécie: os espécimes disputam entre si na população global
independente de qual espécie pertencem. Nessa configuração não há
cooperação entre os indivíduos de uma mesma espécie, apenas os
melhores da população global sobrevivem;
\end{itemize}

Para evitar que espécies não tenham a oportunidade de se desenvolverem
antes que sua população seja drasticamente reduzida ou até mesmo
extinta, tratou-se cada um dos casos individualmente. No caso da
seleção interespécie, é necessário escolher uma função que privilegie
espécies mais aptadas, mas que uma diferença de aptidão muito grande
--- que irá ocorrer em especial durante o inicio da evolução devido a
espécies condicionadas em ambientes mais propícios que outras --- não
elimine toda a diversidade das populações antes que elas adequem sua
estratégia evolutiva. Para isso, escolheu-se empiricamente a função
\ref{eq:funcao_interespecie}, para suavizar a pressão aplicada
em espécies menos aptas. A população reservada para uma espécie para a
próxima geração é dada por \ref{eq:mu}. Entretanto, como se utiliza a
função de transformação em inteiro \emph{floor},
ao somar $\mu_i'$ para cada espécie pode acabar resultando em uma
população menor que $\mu$. Assim, distribui-se aleatoriamente os
indivíduos faltantes nas espécies de forma que a soma dos $\mu_i'$
seja o mesmo que $\mu$. 

Para a avaliação da configuração intraespécie, bem como interespécie,
será utilizada a função de \emph{Ackley} contendo 10 espécies. As
espécies têm de 20 a 29 dimensões, sendo rotuladas sequencialmente de
01 a 10. As espécies de menores índices tem uma carga genética mais
simples de ser adaptada através da estratégia evolutiva por possuir
menos parâmetros livres que aquelas com índices maiores, porém todas
espécies tem a mesma capacidade de resolver o problema --- todas podem
obter o mínimo em zero. Ao observar a Figura~\ref{fig:interespecies},
percebe-se que as espécies inferiores tendem a tomar conta da
população nas primeiras gerações confirmando essa hipótese. Por outro
lado, conforme a evolução ocorre, há uma tendência para o equilíbrio,
uma vez que, nesse caso, todas as espécies são capazes de resolver
igualmente bem o problema. Percebe-se que a espécie 05 (marrom) é
aquela que mais sofre durante as gerações iniciais, porém, por volta
da 30$^a$ geração ela alcança o nível de desenvolvimento das outras
espécies, crescendo gradualmente sua população.

\begin{subequations}\label{eq:inter_especie}
\begin{equation} \label{eq:aptidao_especie}
\text{Aptidão}_{(i)} = \sum^{\lambda_i}_{j=1} \text{Aptidão}_{(i,j)}
\end{equation} 
\begin{equation} \label{eq:funcao_interespecie}
f_{inter}(i)=log_2(\text{Aptidão}_{(i)}-min(\text{Aptidão}_{(i)}|\forall i\in \Gamma)+2)
\end{equation} 
\begin{equation} \label{eq:mu}
\mu_i' = floor\left(\dfrac{f_{inter}(i)}{f_{inter,norm}}\right)
\end{equation}
\begin{equation} \label{eq:fnorm}
f_{inter,norm}= \sum_{\forall i\in \Gamma} f_{inter}(i)
\end{equation}
\end{subequations}

\noindent onde:

\begin{description}
\item[$\lambda_i$] é o tamanho da população da prole da i-ésima
espécie;
\item[$\Gamma$] é o espaço contendo todas as espécies;
\item[$\mu_{i}'$] é o tamanho da população dos pais da i-ésima espécia
para a próxima geração.
\end{description}


\begin{figure}[h!t]
\centering
\includegraphics[width=\textwidth]{imagens/es_interspecies.pdf}
\caption[Competição interespécie.]{Competição interespécie. As linhas
contínuas na subfigura superior indicam os indivíduos mais aptos de
cada espécie. As subfiguras inferiores indicam a população para cada
espécie, sendo a mais inferior a população para a prole, e o outro
caso a população dos pais da espécie.% Na
%subfigura superior
%, as linhas contínuas, tracejadas finas e tracejadas
%grossas indicam respectivamente os individuos mais aptos de cada
%espécie, a média de aptidão da população de cada espécie e os
%individuos menos aptos de cada espécie. As subfiguras inferiores
%indicam a população para cada espécie, sendo a mais inferior a
%população para a prole, e o outro caso a população dos pais da
%espécie.}
}
\label{fig:interespecies}
\end{figure}

Um outro mecanismo foi implementado para impedir a extinção de uma
espécie. Ele funciona como um órgão de proteção da diversidade de
espécies, limitando que espécies próximas de entrarem em extinção
tenham sua população reduzida, independente do quão mal esses
indivíduos se adequam ao \emph{habitat} para o qual estão sendo
avaliados.

Isso foi especialmente importante para o caso de competição
intraespécie, onde uma espécie ao encontrar um material genético de
melhor qualidade em comparação com as outras, rapidamente tomava conta
da população global por espalhar esse material entre sua população com
grande velocidade. Na Figura~\ref{fig:intraspecies_nopressure},
observa-se que, se não fosse esse mecanismo, a espécie azul ou amarela
iriam extinguir todas as outras tomando conta da população
global. Fica evidente também que apenas um órgão de proteção da
diversidade de espécies não é suficiente para garantir a evolução das
espécies, é necessário suavizar a competição, de modo que uma espécie
que por algum motivo se tornou mais apta não extermine outras espécies
rapidamente acabando com sua diversidade e não as dê a oportunidade
para evoluir, já que os indivíduos que sobraram possivelmente ainda não
ajustaram sua estratégia evolutiva. A 
Figura~\ref{fig:intraspecies_nopressurecontrol_info} mostra as
pressões de seleção em ordens muito além daquelas que deveriam
ocorrer, obtendo valores na ordem de 20 logo no inicio da evolução
para as espécies que foram iniciadas em condições menos favoráveis,
eliminando toda sua diversidade em poucas gerações. Já as espécies que
conseguiram se desenvolver, observa-se que as mesmas ao conseguirem
uma aptidão melhor que a da outra espécie rapidamente tomam conta da
população global, acontecimentos marcados pelos picos na pressão de
seleção das espécies antes dominantes. Observa-se também que apenas as
espécies mais simples conseguiram evoluir o suficiente para convergir
para o mínimo global.

\begin{figure}[h!t]
\centering
\includegraphics[width=\textwidth]{imagens/es_intraspcies_nopressurecontrol.pdf}
\caption[Competição intraespécies sem intervenção na
competição.]{Competição intraespécies sem intervenção na competição.}
\label{fig:intraspecies_nopressure}
\end{figure}

\begin{figure}[h!t]
\centering
\includegraphics[width=\textwidth]{imagens/es_intraspcies_nopressurecontrol_pressureInfo.pdf}
\caption[Pressão de seleção para competição intraespécie sem
intervenção na competição.]{Pressão de seleção para competição
intraespécie sem intervenção na competição.}
\label{fig:intraspecies_nopressurecontrol_info}
\end{figure}

Em vista disso, implementou-se um mecanismo de controle de pressão de
seleção por espécie. Esse mecanismo aceita um valor máximo de
pressão para cada espécie, se o valor ultrapassar o limiar, então se 
reduz sua pressão ao alocar espaço da população para essa espécie
retirando das espécies com maior alocação de população até que a
alocação de população dessas espécies que ultrapassaram o corte máximo
de pressão resultem em um valor aceitável da mesma. A escolha de
retirada da alocação de indivíduos é feita da seguinte maneira:

\begin{itemize}
\item Inicia-se reduzindo a alocação de população da espécie com maior
população;
\item Caso o valor da espécie de maior população atingir o tamanho da
população de uma outra espécie, adiciona-se essa espécie para a
redução de população e continua o processo. Caso isso ocorra
novamente, a próxima espécie também será adicionada para redução de
população e o processo continua até que seja determinado quais
espécies irão ceder espaço para que a população das espécies com altas
pressões satisfaça o critério mínimo;
\item Quando há o agrupamento de espécies para redução e a
necessidade de reduzir um número não inteiro de população em cada
espécie agrupada, a escolha do resíduo da divisão é feita
aleatoriamente. Ex. se houver de suprir 100 espécimes para garantir que
a pressão de seleção seja feita em níveis aceitáveis, e houver 3
espécies tendo sua população reduzida, retira-se 33 indivíduos de
cada uma delas, porém a escolha da espécie que perderá mais um
individuo será realizada aleatoriamente uniforme.
\end{itemize}

A execução para uma pressão máxima de 7,3 pode ser visualizada na
Figura~\ref{fig:intraspecies_pressurecontrol}, onde fica evidente a
convergência da população para a máxima aptidão (ou mínimo da função
da \emph{Ackley}). Também se observa uma mudança menos brusca quando
comparado à versão sem intervenção na competição, mostrando que o
controle é importante para garantir mudanças mais suaves na
configuração da população. Uma observação importante pode ser
realizada quanto ao órgão de proteção de diversidade de espécies: não
fosse sua operação, diversas espécies teriam sido extintas durante o
processo evolutivo. Na
Figura~\ref{fig:intraspecies_pressurecontrol_info}, observa-se a
pressão de seleção requerida pela competição natural, e aquela
aplicada pelo sistema de intervenção. Em certos casos, a competição
natural chega a exigir pressões de até 500 unidades, o que acabaria
com a diversidade de uma espécie em uma única geração. Assim, ao optar
pela versão de competição intraespécies faz-se necessário interferir
na competição para que todas as espécies tenham chances de evoluir.
Ainda, as espécies de índices mais altos precisam de mais tempo para
chegarem em seu auge. Isso ocorre pela maior dificuldade para o ajuste
da estratégia de evolução devido a maior presença de parâmetros livres
nessas espécies, bem como a menor alocação de população para sua
evolução uma vez que essas espécies irão tender a se inicializar
probabilisticamente com menor aptidão --- a dispersão nas dimensões a
mais tenderão a aumentar o erro e, consequentemente, reduzir a aptidão
dessas espécies.

\begin{figure}[h!t]
\centering
\includegraphics[width=\textwidth]{imagens/es_intraspecies_pressurecontrol.pdf}
\caption[Competição intraespécies com intervenção na
competição.]{Competição intraespécies com intervenção na competição.}
\label{fig:intraspecies_pressurecontrol}
\end{figure}

\begin{figure}[h!t]
\centering
\includegraphics[width=\textwidth]{imagens/es_intraspcies_pressurecontrol_pressureInfo.pdf}
\caption[Pressão de seleção para competição intraespécie com 
intervenção na competição.]{Pressão de seleção para competição
intraespécie com intervenção na competição.}
\label{fig:intraspecies_pressurecontrol_info}
\end{figure}

Há uma nítida diferença entre como as duas seleções se comportam. Ao
comparar as figuras~\ref{fig:interespecies} e
\ref{fig:intraspecies_pressurecontrol}, percebe-se que o caso de
competição interespécie tem uma mudança bastante tênue na configuração
da população, privilegiando as espécies com melhor aptidão
proporcionalmente à sua aptidão como espécie. No caso, a escolha da
função torna a vantagem pequena entre elas, já que se utiliza uma
atenuação logarítmica. Enquanto isso, na competição intraespécie com
intervenção (a versão sem intervenção não é recomendada) observa-se
que o crescimento da população da espécie ocorre gradualmente conforme
seus indivíduos ocupam posições privilegiadas no espaço de
solução. Seria interessante adiar a competição, dando um número de
gerações iniciais para os quais as espécies evoluiriam sem competir,
deixando as mesmas ajustarem sua estratégia evolutiva e estarem
em um estágio pré-evoluido para então começarem a competir entre si.
Comparando essas mesmas figuras novamente, observa-se que o método
de competição intraespécies, com a implementação desse novo mecanismo,
favorece espécies com carga genética mais simples de ser otimizada ou
que começaram em condições mais favoráveis.


  \chapter{Metodologia}
\label{chap:metodologia}

Neste capítulo está presente outras informações relevantes antes de
entrar nos méritos de resultados. A Sessão~\ref{sec:base_de_dados}
contém a base de dados utilizados no trabalho. A sessão seguinte
(Sessão~\ref{sec:aplic_es}) revela como o algoritmo genético foi
adaptado para otimizar o problema em questão, enquanto o \acs{som}
será discutido na Sessão~\ref{ssec:som}.

% Multi-espécies
\section{Descrição da base de dados}
\label{sec:base_de_dados}

Foram utilizados três conjuntos de dados fornecidos pelo \acs{cepel},
todos amostrados em situações controladas. Serão descritos as
características de cada um deles, sendo seus códigos para
identificação \emph{Temporizado} (Subsessão~\ref{ssec:temp}),
\emph{Empilhado4} (Subsseão~\ref{ssec:emp4}) e \emph{Empilhado7}
(Subsessão~\ref{ssec:emp7}).

\subsection{Conjunto de dados \emph{Temporizado}}
\label{ssec:temp}

\FloatBarrier
O conjunto de dados \emph{Temporizado} contém apenas cinco
equipamentos:

\begin{itemize}
\item Televisão LCD;
\item Geladeira;
\item Lâmpada fluorescente 23W, 54W;
\item Ventilador.
\end{itemize}

Sua medição foi realizada com o medidor \emph{Yokogawa}, e o perfil de
consumo dos aparelhos pode ser visualizada em
\ref{fig:temporizado_overview}.
Este conjunto é o que tem a maior presença de ruído, causado pela
televisão LCD, em especial para os períodos das 00:00 às 02:00 e 04:30
às 08:00 do dia 21.

A informação no gabarito pode ser observada nas figuras
\ref{fig:temporizado_app_time}--\ref{fig:temporizado_televisao}. 
A Figura~\ref{fig:temporizado_app_time} contém a informação do consumo
temporal dos aparelhos, enquanto a
Figura~\ref{fig:temporizado_app_pie} contém o gráfico circular do
consumo estimado no gabarito para os aparelhos. As figuras
\ref{fig:temporizado_geladeira}--\ref{fig:temporizado_televisao}
contêm os transitórios dos aparelhos marcados pelo usuário durante a
criação do gabarito. Todos os eventos são movidos para obterem média
zero de forma que a figura fique uniforme e seja possível compará-los.
A informação contida nesse gráfico auxilia a identificar eventos no
gabarito que fogem do padrão, seja por erro do usuário no
preenchimento, ou por caracterizar um evento excêntrico, facilitando a
identificação desses casos. Um exemplo pode ser observado na
Figura~\ref{fig:temporizado_ventilador}, onde há a ocorrência de um
evento que foge do padrão dos outros coletados. Nessas figuras também
permitem observar a quantidade de eventos para cada alteração de
estado (indicado entre parênteses no título das subfiguras).

\begin{sidewaysfigure}[p]
\centering
\includegraphics[width=\textwidth]{imagens/Temporizado_Overview.pdf}
\caption{Perfil de consumo para o conjunto de dados \emph{Temporizado}.}
\label{fig:temporizado_overview}
\end{sidewaysfigure}

\begin{sidewaysfigure}[p]
\centering
\includegraphics[width=\textwidth]{imagens/Temporizado_AppTime.pdf}
\caption{Informação no gabarito para o conjunto de dados
\emph{Temporizado}: consumo temporal dos aparelhos.}
\label{fig:temporizado_app_time}
\end{sidewaysfigure}

\begin{sidewaysfigure}[p]
\centering
\includegraphics[width=.5\textwidth]{imagens/Temporizado_AppPie.pdf}
\caption{Informação no gabarito para o conjunto de dados
\emph{Temporizado}: gráfico circular do consumo dos aparelhos.}
\label{fig:temporizado_app_pie}
\end{sidewaysfigure}

\begin{sidewaysfigure}[p]
\centering
\includegraphics[width=\textwidth]{imagens/Temporizado_App_Geladeira.pdf}
\caption{Informação no gabarito para o conjunto de dados
\emph{Temporizado}: envoltória para as diversas variáveis para a
geladeira.}
\label{fig:temporizado_geladeira}
\end{sidewaysfigure}

\begin{sidewaysfigure}[p]
\centering
\includegraphics[width=\textwidth]{imagens/Temporizado_App_Ventilador.pdf}
\caption{Informação no gabarito para o conjunto de dados
\emph{Temporizado}: envoltória para as diversas variáveis para a
ventilador.}
\label{fig:temporizado_ventilador}
\end{sidewaysfigure}

\begin{sidewaysfigure}[p]
\centering
\includegraphics[width=\textwidth]{imagens/Temporizado_App_LF23W.pdf}
\caption{Informação no gabarito para o conjunto de dados
\emph{Temporizado}: envoltória para as diversas variáveis para a
lâmpada fluorescente 23W.}
\label{fig:temporizado_lf23}
\end{sidewaysfigure}

\begin{sidewaysfigure}[p]
\centering
\includegraphics[width=\textwidth]{imagens/Temporizado_App_LF54W.pdf}
\caption{Informação no gabarito para o conjunto de dados
\emph{Temporizado}: envoltória para as diversas variáveis para a
lâmpada fluoresecente 54W.}
\label{fig:temporizado_lf54}
\end{sidewaysfigure}

\begin{sidewaysfigure}[p]
\centering
\includegraphics[width=\textwidth]{imagens/Temporizado_App_Televisao.pdf}
\caption{Informação no gabarito para o conjunto de dados
\emph{Temporizado}: envoltória para as diversas variáveis para a
televisão.}
\label{fig:temporizado_televisao}
\end{sidewaysfigure}

\FloatBarrier

\subsection{Conjunto de dados \emph{Empilhado4}}
\label{ssec:emp4}

O conjunto de dados \emph{Empilhado4} contém os seguintes
equipamentos:

\begin{itemize}
\item Forno elétrico;
\item Lâmpada fluorescente (LF) 25 W (2 unidades), 22W (2 unidades), 15W (3
unidades), 9W;
\item Lâmpada incandescente (LI) 40W (2 unidades);
\item Televisão CRT.
\end{itemize}

Sua medição foi realizada com o medidor do \acs{cepel}. Contém a maior
quantidade de alterações de estados de aparelhos de baixo consumo (no
caso as lâmpadas fluorescentes). O
perfil de seu consumo pode ser visualizado na
Figura~\ref{fig:empilhado4_overview}.

A informação no gabarito pode ser observada nas figuras
\ref{fig:empilhado4_app_time}--\ref{fig:empilhado4_app_pie}. 
A Figura~\ref{fig:empilhado4_app_time} contém a informação do consumo
temporal dos aparelhos, enquanto a Figura~\ref{fig:empilhado4_app_pie}
contém o gráfico circular do consumo estimado no gabarito para os
aparelhos.

%Figure~\ref{fig:empilhado4_app_pie} contém o gráfico circular do
%consumo estimado no gabarito para os aparelhos. As figuras
%\ref{fig:empilhado4_geladeira}--\ref{fig:empilhado4_televisao}

\begin{sidewaysfigure}[p]
\centering
\includegraphics[width=\textwidth]{imagens/Empilhado4_Overview.pdf}
\caption{Perfil de consumo para o conjunto de dados \emph{Empilhado4}.}
\label{fig:empilhado4_overview}
\end{sidewaysfigure}

\begin{sidewaysfigure}[p]
\centering
\includegraphics[width=\textwidth]{imagens/Empilhado4_AppTime.pdf}
\caption{Informação no gabarito para o conjunto de dados
\emph{Empilhado4}: consumo temporal dos aparelhos.}
\label{fig:empilhado4_app_time}
\end{sidewaysfigure}

\begin{sidewaysfigure}[p]
\centering
\includegraphics[width=\textwidth]{imagens/Empilhado4_AppPie.png}
\caption{Informação no gabarito para o conjunto de dados
\emph{Empilhado4}: gráfico circular do consumo dos aparelhos.}
\label{fig:empilhado4_app_pie}
\end{sidewaysfigure}

%\begin{sidewaysfigure}[p]
%\centering
%\includegraphics[width=\textwidth]{imagens/Empilhado4_App_Geladeira.pdf}
%\caption{Informação no gabarito para o conjunto de dados
%\emph{Empilhado4}: envoltória para as diversas variáveis para a
%geladeira.}
%\label{fig:empilhado4_geladeira}
%\end{sidewaysfigure}
\FloatBarrier

\subsection{Conjunto de dados \emph{Empilhado7}}
\label{ssec:emp7}

O conjunto de dados \emph{Empilhado7} contém os seguintes
equipamentos:

\begin{itemize}
\item Lâmpada incadescente (LI) 60 W, 100 W;
\item Lâmpada fluorescente (LF) 20 W, 21 W, 24 W, 26 W, 28 W, 40 W;
\item Secador de cabelo;
\item Ar condicionado;
\item Sanduicheira;
\item Geladeira (obs: essa geladeira tem consumo bastante superior
àquele utilizada no arquivo \emph{Temporizado});
\item Televisão CRT.
\end{itemize}

Sua medição foi realizada com o medidor do \acs{cepel}. Uma
peculiaridade desse arquivo é a distorção dos transitórios durante o
momento que o ar condicionado está operando.

A informação no gabarito pode ser observada nas figuras
\ref{fig:empilhado7_app_time}--\ref{fig:empilhado7_app_pie}. 
A Figura~\ref{fig:empilhado7_app_time} contém a informação do consumo
temporal dos aparelhos, enquanto a Figura~\ref{fig:empilhado7_app_pie}
contém o gráfico circular do consumo estimado no gabarito para os
aparelhos.


\begin{sidewaysfigure}[p]
\centering
\includegraphics[width=\textwidth]{imagens/Empilhado7_Overview.pdf}
\caption{Perfil de consumo para o conjunto de dados \emph{Empilhado7}.}
\label{fig:empilhado7_overview}
\end{sidewaysfigure}

\begin{sidewaysfigure}[p]
\centering
\includegraphics[width=\textwidth]{imagens/Empilhado7_AppTime.pdf}
\caption{Informação no gabarito para o conjunto de dados
\emph{Empilhado7}: consumo temporal dos aparelhos.}
\label{fig:empilhado7_app_time}
\end{sidewaysfigure}

\begin{sidewaysfigure}[p]
\centering
\includegraphics[width=\textwidth]{imagens/Empilhado7_AppPie.png}
\caption{Informação no gabarito para o conjunto de dados
\emph{Empilhado7}: gráfico circular do consumo dos aparelhos.}
\label{fig:empilhado7_app_pie}
\end{sidewaysfigure}

\FloatBarrier









\section[Aplicação do ES para Otimização do Detector de Eventos]{
Aplicação do \acf{es} para Otimização do Detector de Eventos}
\label{sec:aplic_es}

Para o ajuste automático, utilizou-se o ambiente de análise detalhado
no Capítulo~\ref{chap:framework}. Entretanto, algumas questões ainda
estavam em aberto. Além dos caminhos mostrados na
Figura~\ref{fig:cepel_transitorio}, foi implementado uma nova versão
para a remoção de eventos próximos utilizando a média de seus centros
(ver p.~\pageref{text:media}). Outra maneira de remoção dos eventos
também foi adicionada utilizando o conceito de incosistência (ver
p.~\pageref{text:incosistentes}). Uma estratégia em força bruta ---
aqui se referindo a otimizar todas as possíveis configurações e
identificar a melhor convergência delas --- não parecia ser a melhor
maneira de abordar o problema. Percebeu-se a necessidade de realizar a
escolha de algumas configurações a serem testadas para reduzir a
quantidade de caminhos possíveis.

Durante a realização do \emph{Ajuste Manual}, percebeu-se que os
eventos removidos devido à incosistência eram sempre eventos de falso
alarme, mas não havia a ocorrência de eventos de detecção causados por
esse tipo de remoção. Por isso, determinou-se que a remoção de eventos de
inconsistentes seria sempre realizada. Já para o caso da remoção de
eventos devido à ruído era importante para remoção de pertubações
rápidas geradas na rede que não constituiam na mudança do patamar
operativo da rede, e com base nisso se determinou que a remoção de
eventos ruidosos sempre seria realizada. Assim, essa variável
($\Delta I_{min}$) sempre estará presente no material genético das
espécies, bem como o $\sigma$ da Gaussiana a ser utilizada no filtro
de derivada de Gaussiana e o seu valor de corte $\delta_{min}$. Já
para o caso da remoção de eventos próximos, não era possível
determinar se sua utilização era necessária, nem qual das versões de
remoção --- por média, ou sem deslocamento --- era o que mais se
adequada ao problema. No caso de não usar corte, a variável $n_{min}$
não seria utilizada, porém no caso oposto as espécies teriam gene a
mais, cujo fenótipo é inteiro. Por outro lado, sua representação é no
conjunto real, sendo necessário determinar como codificar a
informação. Simplesmente se escolheu arredondar o valor na
representação para obter o valor do fenótipo. Finalmente, também não
se sabia \emph{a priori} determinar se a ordem influenciaria na
importância, e apesar de não se esperar grandes diferenças devido a
mudança da ordem em que eles seriam removidos, decidiu-se testar ambas
configurações.  

Assim, para essas determinações, haviam cinco caminhos a serem
percorridos: 

\begin{enumerate}[label=(\Roman*)]
\item Com remoção de eventos próximos utilizando sem deslocamento
\emph{depois} de remover eventos ruidosos;
\item Com remoção de eventos próximos utilizando a média 
\emph{depois} de remover eventos ruidosos;
\item Com remoção de eventos próximos utilizando sem deslocamento
\emph{antes} de remover eventos ruidosos;
\item Com remoção de eventos próximos utilizando a média 
\emph{antes} de remover eventos ruidosos;
\item Sem remoção de eventos próximos (essa espécie tem um gene a
menos que as espécies anteriores); 
\end{enumerate}

Além da questão dos caminhos, era necessário determinar a
inicialização. Para garantir melhor convergência, empregou-se
fronteiras mínimas e máximas para cada um dos valores representados
pelos genes. A inicialização foi aleatória uniforme dentro dessas
fronteiras. Para o parâmetro da estratégia evolutiva, porém,
iniciou-se esse valor de acordo com \ref{eq:sigma_init}, garantindo
que inicialmente cada individuo tenha, em média, 95\% de chance de
explorar uma região no interior a 10\% do espaço de solução permitido. Os
limites inferiores e superiores foram determinados por
\ref{eq:sigma_min} e \ref{eq:sigma_max}, respectivamente. Os valores
de $sigma_{min,i}$ e $sigma_{max,i}$ garante que no mínimo o material
genético irá sofrer pertubações dentro da região de 0,01\% da região
de solução em 67\% dos casos, enquanto no máximo ele irá pertubar o
material genético em 30\% da região para a mesma probabilidade.

\begin{subequations}
\begin{equation}\label{eq:sigma_init}
2\sigma_{init,i}=0,1\Delta x_i
\end{equation}
\begin{equation}\label{eq:sigma_min}
\sigma_{min,i}=1\times10^{-4}\Delta x_i
\end{equation}
\begin{equation}\label{eq:sigma_max}
\sigma_{max,i}=1\times10^{+3}\sigma_{min,i}
\end{equation}
\end{subequations}

\noindent onde $\Delta x_i$ é a região do espaço dentro das fronteiras
para a i-ésima variável.

Finalmente, cabe ainda decidir qual dos métodos de competição para as
espécies será utilizado. Como a implementação do método de competição
intraespécies acaba favorecendo espécies com carga genética mais
simples de serem otimizadas ou que começaram em condições 
privilegiadas --- quando sem o mecanismo de adiamento da competição
---, irá optar-se pela competição intraespécie.

\section[Mapas Auto-Organizáveis]{\acl{som}}
\label{ssec:som}

Para a construção do modelo neural será utilizado o aprendizado não
supervisionado. O \acf{som}\footnote{Também conhecido como mapa de
\emph{Kohonen}.} \cite[cap 9]{haykin1999neural} é um modelo de rede que
agrupa eventos similares em regiões do mapa ao realizar um mapeamento
do $\Re_{n} \rightarrow \Re_{k} | k \in {1,2}$. Inicialmente, gera-se
uma grade bidimencional $m \times n$ neurônios.

Os neurônios são unidades que recebem as entradas da rede e se
movimentam no decorrer do treinamento. Cada neurônio possui uma
coordenada no espaço dada pelo vetor peso $w$. Cada neurônio é
posicionado aleatoriamente 
	 
Após a criação e a inicialização, aplica-se em todos os neurônios um
evento do banco. A Figura~\ref{fig:som_representacao} mostra esse
processo. 

\begin{figure}[h!tb]
\centering
\includegraphics[width=6cm]{imagens/kohonen.pdf}
\caption[Representação da grade bidimensional de um SOM]{
Representação da grade bidimensional de um \acl{som}. O evento $x$
aplicado em todos os neurônios (circulos) do mapa e a atuação da
função de vizinhança em torno do neurônio mais similar ao evento
(circulo em rosa claro).}
\label{fig:som_representacao}
\end{figure}

Em seguida, calcula-se a métrica Euclidiana (\ref{eq:d_i_som}) entre o
evento e cada peso (neurônio) da grade. 

\begin{equation}\label{eq:d_i_som}
d_{i} = \sqrt{\sum_{n=1}^{N}(w_{ni}-x_{n})^{2}}
\end{equation}

\noindent onde $w_{i}$ é o vetor peso do neurônio $i$ dado por $wi =
[w_{i1},w_{i2},...,w_{ni}]^{T}$ e $x$ é o vetor evento dado por $x =
[x_{1},x_{2},...,x_{n}]^{T}$ . 

O neurônio cuja métrica Euclidiana será o neurônio
vencedor, chamado de \gls{bmu}. Esse neurônio
$j$ será o centro do raio da função de vizinhança que será empregada
para atualizar os vetores $w$ ao longo do treinamento apartir de
\ref{eq:som_atualiza}.

\begin{equation}\label{eq:som_atualiza}
w_{i}(t+1)=w{i}(t)+\eta(t)h_{ij}(t)(x(t)-w_{i}(t))
\end{equation}

\noindent onde $\eta(t)$ é a taxa de aprendizado que decresce monotonicamente e
$h_{ij}(t)$ é a função de vizinhança que tem seu valor máximo no
neurônio $j$ (\acs{bmu}), diminui conforme se afasta deste e abrange os
neurônios que estão dentro do raio que diminui com relação ao tempo
(interação). 

A função de vizinhança é definida na equação \ref{eq:som_vizinhanca}.

\begin{subequations}
\begin{equation}\label{eq:som_vizinhanca}
h_{ij}=e^{\frac{-d_{ij}^{2}}{2\sigma^{2}(t)}}
\end{equation}
\begin{equation}\label{eq:sigma_t}
\sigma(t)=\sigma(0)e^{{-\frac{t}{\tau}}}
\end{equation}
\end{subequations}

\noindent onde $d_{ij}$ é a distância Euclidiana entre o peso $w_{i}$ e o
neurônio $j$ ($BMU$) e $\sigma(t)$ definido por \ref{eq:sigma_t}.

Para $\sigma(0)$ igual ao raio definido no inicio do treinamento e
$\tau$ uma constante de tempo de decaimento da função. Na
Figura~\ref{fig:vizinhanca_grade} mostra a atuação da função de
vizinhança no treinamento com o \acs{bmu} ao centro.

\begin{figure}[h!tb]
\centering
\includegraphics[width=.5\textwidth]{imagens/vizinhanca_func.png}
\caption{Atuação da função de vizinhança na grade.}
\label{fig:vizinhanca_grade}
\end{figure}

Ao invés de se atualizar os pesos da rede a cada evento (treinamento
sequencial), utiliza-se o conceito de batelada. O treinamento em
batelada atualiza os pesos após um conjunto de eventos, geralmente
toda a base, for apresentada a rede. A cada passo da interação, cada
evento da base de dados é associado ao seu \acs{bmu}. Ao final desse
processo calcula-se o somatório para cada neurônio do mapa.

\begin{equation}\label{eq:sum_centroide}
s_{i}=\sum_{k=1}^{\eta_{i}}x_{k}
\end{equation}

\noindent onde $\eta_{i}$ é o número total de eventos associados a
cada neurônio e $x_{k}$ é o vetor evento associado. 

Em seguida os pesos dos neurônios do mapa são atualizados pela
equação~\ref{eq:som_wi}, sendo $m$ o número de neurônios do mapa.

%EQ 7
\begin{equation}
\label{eq:som_wi}
w_{i} = \frac{\sum_{j=1}^{m}h_{ij}(t)s_{j}}{\sum_{j=1}^{m}\eta_{i}h_{ij}(t)} 
\end{equation}

Portanto, esse treinamento utiliza médias ponderadas para atualizar os
pesos. Esse efeito, torna o treinamento mais suave a medida que ocorre
a redução da função de vizinhança. 

\subsection{Definições de Treinamento}
\label{sec:som_treinamento}

O mapa foi definido com uma grade de tamanho de $20 \times 15$ (300
neurônios) com disposição hexagonal. Para o algoritmo de treinamento
iremos utilizar o conceito de batelada apresentado na sessão anterior.
O processo de treinamento dar-se-á em duas etapas. A primeira fase é
configurada para rodar um número longo de épocas de treinamento
definido como 2000 épocas, e utilizar um raio inicial da função de
vizinhança mais abrangente, definido como 8. A segunda etapa do
treinamento é chamada de ajuste fino. Essa etapa possui um raio
inicial de 2 e um número de épocas mais curto definido como 500
épocas. Essa segunda etapa tem como objetivo tornar os agrupamentos
formados na primeira fase mais bem definidos.

Devido a grande variância das variáveis utilizadas para compor o
treinamento da rede, foram analisados dois tipos de normalizações. A
primeira divide cada variação de potência do evento por 127V. A
segunda representa o mesmo processo, porém divide as potências pela
tensão coletada no momento da medição das variações de transição do
aparelho. Vale lembrar que um evento é composto pelas variações de
potência real, reativa e harmônica e a variação de corrente.
Totalizando quatro variáveis coletadas por evento.

\subsection{Agrupamento por Centróides dos \acl{som}}

Após o treinamento do mapa, avaliou-se o \gls{qe}, o
\gls{te} e a \gls{matrizu}. O \gls{qe} é a média
dos erros dos $N$ casos aplicados no \acs{som} dado pela diferença entre o
vetor evento $x_{k}$ e o $w_{BMU}$ do seu respectivo \gls{bmu}. O
\gls{qe} é definido pela equação \ref{eq:qe}.

\begin{equation}\label{eq:qe}
Q_{e} = \frac{1}{N}\sum_{k=1}^{N}\|x_{k}-w_{BMU}\|
\end{equation}

Esse erro avalia o quanto o mapeamento se aproxima dos padrões da
entrada. Por outro lado, o \acs{te} (\ref{eq:te}) pode ser
interpretado como o erro da copia de informação da alta dimensão para
o espaço bidimensional. Dado dois neurônios que se aproximam da
entrada $x_{k}$, $BMU_{1}$ e $BMU_{2}$ .

\begin{subequations}
\begin{equation}\label{eq:te}
T_{e} = \frac{1}{N}\sum_{k=1}^{N}u(x_{x})
\end{equation}
\begin{equation}
u(x_{k}) = \left\{\begin{array}{rl}
 1 &\text{ se } BMU_{1} \;\; \text{e} \;\; BMU_{2} \;\; \text{não são vizinhos} \\
 0 &\mbox{o.c.}\end{array}\right.
\end{equation}
\end{subequations}

Dado um neurônio $j$ e o seu respectivo peso $w_{j}$ , a \acs{matrizu}
calcula a distância dos pesos dos neurônios vizinhos com relação a
$w_{j}$. O resultado é uma imagem onde cada hexágono pode ser
interpretado como a distância calculada. 
%Cores quentes indicam que os
%pesos estão afastados e cores frias indicam que estão próximos nessas
%regiões.
	
O algorítmo de \emph{k-means} é utilizado para rotular os neurônios do
mapa, já treinado, em $n$ centróides. Embora estejam disponíveis 13
tipos de diferentes aparelhos, o algoritmo será executado até
encontrar um número de centróides que atenda a disperção dos
diferentes tipos de aparelhos e transientes no mapa. Um número pequeno
de agrupamentos pode acabar englobando vários aparelhos dentro de um
mesmo centróide.  Por outro lado, um número muito grande de centróides
pode acabar particionando em diversas partes um agrupamento que contém
um tipo de aparelho, que cria uma resolução descenessária e dificulta
o entendimento do mapa. Um bom exemplo é o conjunto de transientes de
lâmpadas fluorescentes de baixa potência que é formado por diversas
potências, menores a 25W, que na prática representam as mesmas
características.
% FIXME Avaliar a parte de cima
	
O índice de \emph{Davies-Bouldin} avalia a similaridade entre
agrupamentos e é definido pela equação \ref{eq:bouldin}

\begin{subequations}
\begin{equation}
\label{eq:bouldin}
I_{DB} = \frac{1}{C}\sum_{k=1}^{C}max_{l\neq
k}{\frac{S_{c}(Q_{k})+S_{c}(Q_{l})}{d_{ce}(Q_{k},Q_{l})}}
\end{equation}
\begin{equation} \label{eq:intracentroides}
S_{c} = \frac{1}{N_{k}}\sum_{i}^{N_{k}}\|x_{i}-c_{k}\|
\end{equation}
\begin{equation} \label{eq:distancia_centroides}
d_{ce} = \|c_{k}-c_{l}\|
\end{equation}
\end{subequations}

\noindent onde: 

\begin{description}
\item [$Q$] é um centróide; 
\item [$C$] é o número de centróides; 
\item [$S_{c}$] é a medida de similaridade intracentróides dada pela equação
\ref{eq:intracentroides};
\item [$N_{k}$] o número de eventos pertencentes a cada centróide de
centróide $c_{k}$;
\item [$d_{ce}$] é a distância entre os
centróides dada pela equação \ref{eq:distancia_centroides}. Quanto
menor for esse índice, mais separados e bem definidos se encontram os
centróides formados.
\end{description}


  \chapter{Resultados}
\label{chap:resultados}

O capítulo em questão contém os resultados para a metodologia aplicada
e sua discussão.

%\section{Descrição da base de dados}
%\label{sec:base_de_dados}
%
%Foram utilizados três conjuntos de dados fornecidos pelo \acs{cepel},
%todos amostrados em situações controladas. Serão descritos as
%características de cada um deles, sendo seus códigos para
%identificação \emph{Temporizado} (Subsessão~\ref{ssec:temp}),
%\emph{Empilhado4} (Subsseão~\ref{ssec:emp4}) e \emph{Empilhado7}
%(Subsessão~\ref{ssec:emp7}).
%
%\subsection{Conjunto de dados \emph{Temporizado}}
%\label{ssec:temp}
%
%\subsection{Conjunto de dados \emph{Empilhado4}}
%\label{ssec:emp4}
%
%\subsection{Conjunto de dados \emph{Empilhado7}}
%\label{ssec:emp7}


\section[Aplicação do ES para otimização do detector de eventos]{
Aplicação do \acf{es} para otimização do detector de eventos}
\label{sec:aplic_es}




\begin{table}[p]
\resizebox{\textwidth}{!}{
\begin{tabular}{>{\centering}m{3cm}>{\centering}m{2cm}cccccccc}
\hline \hline \hline
\multicolumn{2}{c}{\parbox[t]{5cm}{\centering Conjunto de Dados}} &
\multicolumn{2}{c}{\textbf{\acs{es} $\mathbf{1/0,9}$}} & 
\multicolumn{2}{c}{\textbf{\acs{es} $\mathbf{1/2}$}} & 
\multicolumn{2}{c}{\textbf{Manual}} & 
\multicolumn{2}{c}{\textbf{Anterior}} \tabularnewline \hline
& & 
DET & FA & 
DET & FA & 
DET & FA &
DET & FA \\
\hline\hline
\multirow{2}{3cm}{\centering\emph{Temporizado}
\footnotesize{(149~eventos)}} & {Ocorr.} & 
148 & 39 & 148 & 42 & 147 & 28 & 147 & 259\\
 & {Taxa (\%)} & 
99,3 & 26,2 & 99,3 & 28,19 & 98,7 &18,8 & 98,7 & 173,8  \\
\hline
\multirow{2}{3cm}{\centering\emph{Empilhado4}
\footnotesize{(74~eventos)}} & {Ocorr.} & 
37 & 0 & 37 & 0 & 24 & 0 & 57 & 9 \\
 & {Taxa (\%)} & 
64,9 & 0,0 & 64,9 & 0,0 & 32,4 & 0,0 & 78,0 & 12,3  \\
\hline
\multirow{2}{3cm}{\centering\emph{Empilhado7}
\footnotesize{(42~eventos)}} & {Ocorr.} & 
36 & 1 & 36 & 2 & 35 & 1 & 37 & 18 \\
 & {Taxa (\%)} & 
85,7 & 2,4 & 85,7 & 4,8 & 83,3 & 2,4 & 88,1 & 42,9  \\
\hline \hline
\end{tabular}}
\caption[Resultado para os três conjuntos de dados onde o
\acs{es} foi ajustado alimentado por todos eles.]{
Resultado para os três conjuntos de dados onde o
\acs{es} foi ajustado alimentado por todos eles. As configurações
\emph{Manual} e \emph{Anterior} se referem respectivamente aos casos
determinados empiricamente pelo autor do trabalho e pelo grupo do
\gls{cepel}, o último sendo determinado em dados sem ruídos.}
\label{tab:resultados_sem_generalizacao}
\end{table}

  \chapter{Conclusão}
\label{chap:conclusao}



%\section{Trabalhos Futuros}
%\label{sec:trabfut}



  \glsaddall{}

  \backmatter{}
  \bibliographystyle{coppe-unsrt}
  \bibliography{thesis}

  \appendix
  %\include{appenA}
\end{document}
