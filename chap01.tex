\chapter{Introdução}

% O ser humano e sua capacidade de ir além da energia endossomática
Estima-se que o ser humano consome entre 2.500 e 3.000~\acrshort{kcal}/dia sob a 
forma de alimentos, onde apenas cerca de 20\% dessa energia poderá ser 
reinvestida em atividades, dando uma capacidade biológica humana de produção
através de energia endossomática --- provida por seu metabolismo --- de 
apenas 500 a 600 \acrshort{kcal}/dia. No entanto, a capacidade do ser humano de
utilizar fluxos de energia exossomáticos --- fontes não provenientes de seu metabolismo 
--- o permite suplantar sua capacidade biológica de interagir com o meio,
tornando possível uma melhor qualidade de vida \cite{rippel}. 
Atualmente, a média de consumo energético exossomático per capita diário 
mundial supera 50.000~\acrshort{kcal} \cite{world_statics_2012}.

% A utilização de energia exossomática e os danos ao planeta
Os fluxos exossomáticos estão presentes em todas sociedades e são dependentes do
nível de desenvolvimento das mesmas. Em sociedades primitivas, os resultados dos
atos da humanidade exerciam pouca influência no meio ambiente, pois estes eram
absorvidos por aquele. Entranto, com o surgimento de sociedades industriais,
ocorreu uma mudança radical no uso dos recursos naturais e nos seus efeitos
ambientais. Com o crecimento populacional e a produção e o consumo de massa, 
baseados no uso intensivo do petróleo e da eletricidade 
como fontes energéticas, a agressão humana ao meio 
ambiente tornou-se maior de maneira que o mesmo, em muitas situações, não 
consegue mais reequilibrar-se. Os principais problemas ambientais que surgiram 
no cenario mundial tem como fator de destaque vinculados a cadeia da energia, 
da produção ao uso final \cite{rippel,jatoba}.

% A questão de energia e concentração de renda
Indo mais além, a dependência desenvolvimento-energia não está somente atrelada
ao patamar macroscópico da sociedade, mas também relacionada em como o
desenvolvimento está difundido nela. Diferentes classes sociais de uma
determinada sociedade detêm modos de consumo, bem como formas de acesso a bens e
serviços distintos. Nesse caso, os pobres não apenas consomem menos
energia do que os ricos, mas também tipos diferentes da mesma, geralmente mais
poluentes e menos eficientes como aquelas oriundas de biomassa (lenha, carvão,
etc.) \cite{rippel}. Percebe-se, portanto, que o avanço do perfil energético de
uma sociedade não indica necessariamente a sua evolução como um todo, sendo
necessário saber como a energia está difundida nela.

% A questão de sustentabilidade
% Ecologia radical
Em resposta à degradação do meio ambiente se dá a questão de
sustentabilidade. As primeiras objeções à poluição ambiental devido ao processo 
de industrialização em especial após sua aceleração no início do século XX
foi da ecologia radical, dando atenção as questões de proteção e conservação da natureza.
Há uma separação de territórios especiais para uma proteção integral e, numa
visão em larga medida romantizada da natureza, muitas vezes sem permissão 
de nenhum uso antrópico \cite{jatoba}.

% Ambientalismo moderado
A política ambiental preponderante atualmente é o discurso do ambientalismo moderado, 
surgido em meio a Crise do Petróleo na década 1970, onde, percebendo-se a
inviabilidade de sustentação do modelo econômico levando em conta o esgotamento
progressivo dos recursos naturais do planeta, se fez necessário a discussão de como 
colocar em prática as propostas da ecologia radical, mas tendo cautela de não 
necessariamente frear o crescimento econômico ou alterar substancialmente o 
modelo de desenvolvimento vigente --- trazendo o conceito de desenvolvimento
sustentável. Devido à Crise, houve uma busca mundial pela redução da 
dependência no petróleo e outras fontes fósseis em sua matriz energética, 
procurando fontes renováveis e menos poluentes que as fontes fósseis 
\cite{jatoba,epe_eficiencia_2012,rippel}. 

% A questão social
Ainda, as abordagens às questões ecológicas refletem uma intensa dinâmica
econômica, social, cultural e política, que são contraditórias entre si.
Há também uma linha do discurso ambiental que coloca a
justiça social em jogo, ressaltando que os mais vulneráveis aos problemas
ambientais são justamente os mais pobres, os quais serão os mais afetados na
hipótese de agravamento da crise ambiental \cite{jatoba}.

% Eficiencia energetica
A maneira mais efetiva, eleita pela \gls{onu}, de reduzir 
os impactos ambientais locais e globais sem prejuizos ao desenvolvimento, 
refletindo a imagem de desenvolvimento sustentável, é através da \gls{ee}
\cite{rippel,onu,dissert_cursino}.  O uso eficiente de energia deve ser entendido como 
o menor consumo possível de energia para obter uma mesma 
quantidade de produto ou serviço \cite{pne30_eff_energ}. 
As ações de \gls{ee} compreendem modificações ou aperfeiçoamentos tecnológicos ao longo
da cadeia, mas podem também resultar de uma melhor organização, conservação e
gestão energética por parte das entidades que a compõem \cite{pnef}. 

% A importância da energia elétrica
A eletricidade como recurso energético vem adquirindo importância cada vez maior na 
matriz energética mundial --- e assim nos debates englobando \gls{ee} ---
devido a, geralmente, sua maior versabilidade, eficiência, limpeza, segurança e conveniência 
quando comparando com as outras fontes energéticas. O consumo de eletricidade no uso final é de 17,7\% 
da matriz energética mundial em 2010, enquanto sua participação era de 7,3\% em 1973. Estima-se para 2035 um
crescimento de 80\% da demanda de eletricidade quando utilizando o ano base de
2008, de modo a elevar a participação da eletricidade para uma parcela de 23\%. O crescimento da demanda energética 
de eletricidade é liderado pelos países não pertencentes a \gls{ocde}, dentre
eles o Brasil, que terão uma participação de cerca de 80\% do crescimento da demanda de
eletricidade mundial \cite{iea_weo2010}. Em 2011, a porção da eletricidade na matriz 
energética brasileira é de 16,7\% \cite{ben2012}. Durante o desenvolvimento econômico, ocorre a
substituição da biomassa para eletricidade como meio de iluminação e aquecimento
no setor residêncial; a expansão do setor comercial e de serviços utilizando uma quantidade
maiores de aparelhos elétricos (como ar condicionado, iluminação e equipamentos
de \acrshort{ti}); assim como a transformação do setor industrial que
gradualmente substitui o carvão e aumenta a presença de dispositivos elétricos, gerando uma
maior demanda por eletricidade nesse setor \cite{iea_weo2010}.

Além dos beneficios ambientais, a \gls{ee} \cite{jannuzzi,epe_slides_eficiencia}: 

\begin{itemize}
\item reduz ou posterga as necessidades de investimentos em geração, transmissão 
e distribuição de energia elétrica; 
\item reduz o custo de energia para o consumidor final; 
\item contribui para a confiabilidade do sistema elétrico; 
\item traz o aumento da intensividade econômica com a redução da intensividade
energética. 
\end{itemize}

% Eficiencia energetica no mundo desenvolvido
A tendência nos países desenvolvidos é a realização de esforços 
cada vez maiores no sentindo de aumentar a \gls{ee}, 
estimando-se, por exemplo, um consumo 49\% maior nos paises da \gls{ocde} 
no período de 1973 a 1996 caso não houvessem sido adotadas medidas de 
racionalização e \gls{ee} após a Crise do Petróleo
\cite{goldemberg,epe_slides_eficiencia}.

% Eficiencia energética nos países em desenvolvimento 
Os \glspl{ped} não possuem uma capacidade de redução energética tão grande
quanto os países da \gls{ocde}, uma vez que seu consumo energético per capita é
reduzido, justamente por ser necessário o desenvolvimento para aumentar o
consumo. Por outro lado, é possível aliviar a pressão sobre a
oferta energética através da melhoria dos níveis de \gls{ee} nos
diversos setores da sociedade, contrapondo ao pensamento de que para que
haja desenvolvimento é necessário que ocorram impactos ambientais e crescimento no
consumo total de energia --- chamado de efeito \emph{leapfrogging}
\cite{goldemberg}. 

% Eficiencia energética no Brasil
Por isso, a preocupação com \gls{ee} se justifica mesmo no Brasil, 
que apresenta quase metade de sua matriz energética proveniente de fontes 
renováveis e preços de produção de energia economicamentes competitivos.
Diversas iniciativas vêm sido empreendidas, dentre elas, destacam-se
\cite{pnef,pne30_eff_energ,epe_demanda_2012,epe_slides_eficiencia}:

\begin{itemize}
\item 1931: primeiro horário de verão no Brasil;
\item 1975: criado o \gls{proalcool}, para a substituição em larga escala de
veiculos movidos a derivados de petróleo por álcool; 
\item 1981: criado o Porgrama CONSERVE, visando à promoção da conservação de
energia na indústria, ao desenvolvimento de produtos e processos energeticamente
mais eficientes, e ao estímulo à substituição de energéticos importados por
fontes alternativas autóctones;
\item 1982: aprovada as diretrizes do \gls{pme}, conjunto de ações dirigidas à
conservação de energia e à substituição de derivados de petróleo;
%\item 1982: dado as diretrizes para eficiência no \gls{pme}; 
\item 1984: o \gls{inmetro} implementou o \gls{pbe}\footnote{Seu nome inicial
foi Programa de Conservação de Energia Elétrica em Eletrodomésticos, sendo
renomeado em 1992 para a nomenclatura atual.} tendo por objetivo promover a
redução do consumo de energia em equipamentos como refrigeradores, congeladores e condicionadores
de ar domésticos;
\item 1985: foi instituído o \gls{procel}, com a finalidade de integrar as
ações visando à conservação de energia elétrica no país, sendo coordenado
executivamente pela \acrshort{eletrobras}. O \gls{procel} possibilitou uma energia acumulada de 32,9 TWh entre
1986 e 2008, reduzindo a demanda de ponta em 9.538 MW. Somente essa energia
economizada corresponde a investimentos evitados de $\text{R\$}$ 22,8 bilhões; 
\item 1991: foi instituído o \gls{conpet}, programa similar ao \gls{procel}, mas
destinado à parcela da matriz energética proveniente dos derivados do petróleo e
do gás natural, com a coordenação executiva da \acrshort{petrobras}. 
O \gls{conpet} evitou o consumo de 1030,2 milhões de litros diesel no período
2006--2010. 
\item 1998: iniciado o \gls{pee}, regulamentado pela \gls{aneel}. As concessionárias,
permissionárias e autorizadas do setor de energia elétrica devem realizar
investimentos em pesquisa e desenvolvimento promovendo a conservação de energia elétrica. 
Atualmente esse valor é de 0,50\% da receita operacional líquida das operadoras. 
Foi realizado um investimento total no \gls{pee} 
pelas concessionárias do setor elétrico na quantia de $\text{R\$}$ 2 bilhões até
o ano de 2006, onde estima-se que o programa alcançou uma econômia média de 
4.000 GWh/ano e retirando uma carga de ponta de consumo na ordem de 1.140 MW no período de 1998--2005;
\item 2001: Lei n$^\text{o}$ 10.295/2001 determina a instituição de ``níveis
máximos de consumo específico de energia, ou mínimos de \gls{ee},
de máquinas e aparelhos consumidores de energia fabricados e comercializados no
país'', sendo regulamentada pelo Decreto n$^\text{o}$4.059/2001;
\item 2004: criado a \gls{epe}, que tem entre suas diretrizes promover estudos e
produzir informações para subsidiar planos e programas de desenvolvimento
energético ambientalmente sustentável, inclusive, de \gls{ee};
\item 2006: início do \gls{proesco}, programa destinado a financiar projetos de
\gls{ee}, cuja coordenação executiva pertence ao \gls{bndes};
\item 2011: publicado as diretrizes do \gls{pnef} \cite{pnef}, que estabelece uma meta de
10\% de redução no consumo por meio de ações de \gls{ee} no período 
de 2015--2030.
\end{itemize}

% Esforços bons, mas é tudo?
Percebe-se uma evolução brasileira em \gls{ee}, tanto na legislação, 
capacitação e conhecimentos acumulados, quanto na 
coinsciência da necessidade da \gls{ee} nos diversos setores \cite{pnef}. 
Esse patrimônio, no entanto, precisa ser constantemente atualizado 
e ter sua abrangência ampliada. Através do planejamento pretende-se que
recursos possam ser melhores aplicados e os resultados venham com maior
velocidade, abrangência e amplitude. Esforços desse tipo podem ser vistos no
\gls{pnef} \cite{pnef} previsto pelo \gls{pne2030} \cite{pne30_eff_energ}, e nos
\glspl{pde} \cite{pde_2012}.

% Carência de dados
Para se realizar o planejamento é necessário informações, tais como custos e
rendimentos de equipamentos e veículos eficientes, estatísticas detalhadas de
vendas de equipamentos e veículos, assim como resultados de pesquisas sobre
posse e hábitos de usos dos equipamentos e veículos, e as respostas dos diversos
grupos de consumidores às diferentes medidas de conservação. Em função dos
custos significativos das pesquisas de campo, há precariedade das informações
disponíveis, afetando a capacidade das entidades de definir o planejamento de
estratégias de \gls{ee}. Sem a presença de uma base de dados consistente, 
não é possível modelar de forma eficiênte os programas de \gls{ee} no
planejamento da expansão do setor energético brasileiro \cite{pne30_eff_energ}. 

\section{Motivação}

A grande parcela do potencial de \gls{ee} no Brasil, de acordo com as
estimativas realizadas a partir do \gls{beu} \cite{beu}, encontra-se nos três
maiores setores por consumo final de energia, sendo eles o industrial (37,3\%), 
transportes (30,5\%) e residencial (10,2\%), que juntos representaram cerca 
de 78\% do consumo energético do país em 2011 \cite{ben2012,epe_eficiencia_2012}.

No caso do setor residencial, seu maior consumo energético é através da
eletricidade (41,7\%), que tem a tendência de continuar se elevando ao
substituir a biomassa, como lenha (27,8\%) e carvão (2,1\%) \cite{ben2012},
para iluminação e aquecimento durante o desenvolvimento do país. O estudo de 
\gls{ee} abrangendo eletricidade nesse setor é baseado em uma abordagem desagregada 
que depende do número de domícilios, a posse média e hábito de consumo específico 
dos equipamentos eletrodomésticos e rendimento médio desses equipamentos no país, 
mas que também são necessárias variáveis agregadas para o ajuste do modelo, 
sendo elas a relação entre o número de consumidores residenciais e população 
(que permite a projeção do número de consumidores a partir da projeção da população), 
e consumo médio por consumidor residencial. Também são necessários os
rendimentos médios dos equipamentos eletrodomésticos no país
\cite{epe_eficiencia_2012}. 

O rendimento médio é baseado nas tabelas do \gls{pbe} coordenado pelo \gls{inmetro}, 
enquanto as informações de posse e hábito são obtidos através de \gls{pph}. 
O \gls{procel} tem realizados esforços no intuíto de fornecer essas informações, 
onde a última e mais abrangente pesquisa de posse e hábito foi realizada entre 2004--2006 
\cite{result_procel_2005,site_pesquisas_procel}. Nela, utilizou-se cerca de 15 mil 
questionários em 18 estados e 20 concessionárias nos setores 
industrial, comercial e residencial, representando cerca de 92\% do 
mercado consumidor de energia elétrica. Apenas no segmento residencial foram
aplicados 9.847 questionários, onde os profissionais do governo preenchem um
formulário contendo 9 páginas, que além das informações de posse e hábito
realiza um levantamento das características socioeconômicas e comportamentais do
usuário, assim como a caracterização e identificação da residência.

Com isso, além do estudo de \gls{ee}, as \gls{pph} podêm ser utilizadas para: 

\begin{itemize}
\item obter informações importantes como o consumo durante o horário de pico,
fornecendo o fator de carga e fator de demanda;
\item revelar causas de perda de energia elétrica devido a conexões ilegais a
rede;
\item informar a companias de eletrodomésticos sua participação no mercado e 
o perfil de seus consumidores;
\item fazer um paralelo do desenvolvimento e das necessidades das familias 
brasileiras entre classes sociais e regiões do país, explanando as 
diferentes realidades existentes.
\end{itemize}

No entanto, as informações dos hábitos de consumo atualmente dependem
na precisão do usuário final durante o preenchimento do questionário, uma vez que
não são utilizados medições no domicilio, podendo tornar as \glspl{pph}
imprecisas. O usuário pode não se lembrar ou não saber informar os dados
referentes a uma determina pergunta, assim como omitir certas informações, 
por não confiar no profissional estranho ao seu lar ou nos intuítos da pesquisa.

Um outro recurso para se obter informações de posse e hábito é através da medição 
através de aparelhos específicos. As medições podem ser utilizadas para
complementar melhorando a precisão das \glspl{pph}, ou até mesmo substituí-las,
depositando todo o peso de identificação de posse e hábito no sistema de
medição. Afim de evitar um maior incômodo ao consumidor, é recomendável que o 
sistema não faça intrusão a sua propriedade, constituindo de um sistema de
medição \emph{não-invasivo}. 

Novas técnicas de \gls{nialm} podem ser utilizadas com essa finalidade, 
fornecendo as informações necessárias de posse e hábito do consumidor com
maior precisão. Ainda, esse dispositivo pode agilizar o processo de coleta no país,
fornecendo dados com frequência mais adequada aos estudos de \gls{ee}. 

Indo mais além, o \gls{nialm} pode ser utilizado diretamente na obtenção de 
uma melhor \gls{ee}, visto que estudos no exterior \cite{} indicam uma possível
econômia de até 12\% no consumo residencial através do fornecimento de um
retorno detalhado em tempo real ao usuário de sua utilização da eletricidade, 
com informação do consumo por equipamento e recomendações no intuíto de reduzir
o consumo de energia. Esses estudos mostram uma grande variação conforme a
cultura e perfil social do consumidor, sendo necessário um estudo para
determinar esse potencial no Brasil, entretanto, os estudos mostram que os
consumidores pertencentes a classe média são os mais aptos a reduzirem seu
consumo através do retorno da informação em detrimento aos mais pobres que não
possuem mudanças que possam ser exploradas já que muitas vezes o consumo
realizado é vital, e os mais ricos tem pouca sensibilidade aos gastos com a
energia elétrica.

\section{Objetivo}

O \gls{cepel} vem atuado no desenvolvimento de um \gls{nialm}. Este trabalho
tem como objetivo auxiliar no progresso desse dispositivo, de forma a trazer um
sistema operativo a ser aplicado em auxílio as \gls{pph}.

\section{Estrutura Capitular}

A ser preenchido.




