\chapter{Introdução}

A proposta de sustentabilidade, surgida como resposta à degradação do
meio ambiente devido ao processo de industrialização, leva, em seu
discurso, a questão das limitações tanto em termos de recursos
presentes na natureza, ou quanto da sua capacidade de absorver resíduos
de processos em ordens cada vez mais intensas. Sua vertente
predominante traz a ideia do desenvolvimento sustentável que busca
regular a demanda de recursos do meio ambiente, permitindo o
desenvolvimento sem que haja deterioração do mesmo para as gerações
futuras.

Adicionalmente, há uma correlação entre os impactos ambientais e a
necessidade energética, onde os motivos prevalescentes para suas
causas têm como fator de destaque a sua vinculação à cadeia da
energia, desde a produção ao uso final. Entretanto, essa dependência
não é decorrente apenas da intensidade, mas da eficiência energética
relacionada com o seu consumo. Entende-se como eficiência energética a
capacidade de produzir os mesmos resultados finais realizando o menor
consumo possível de energia.

A eletricidade tem uma participação cada vez maior na matriz
energética mundial devido à sua maior versabilidade, eficiência,
limpeza, segurança e conveniência, quando comparada com as outras
fontes energéticas. A eficiência energética do ponto de vista da
eletricidade traz uma série de vantagens, além de beneficios
ambientais:

\begin{itemize}
\item reduz ou posterga as necessidades de investimentos em geração, transmissão 
e distribuição de energia elétrica; 
\item reduz o custo de energia para o consumidor final; 
\item contribui para a confiabilidade do sistema elétrico; 
\item traz o aumento da atividade econômica com a redução da intensividade
energética. 
\end{itemize}

Além disso, o crescimento do consumo nos grandes centros urbanos tem
levado ao surgimento de uma nova rede elétrica --- as redes elétricas
inteligentes ---, que procura descentralizar a geração devido à
exigência excessiva da capacidade de transmissão e distribuição, de
forma que a demanda sobrecarrega as suas linhas em regiões que nem
sempre podem corresponder às necessidades através da instalação mais
linhas. 

\section{Motivação}

Uma tecnologia que tem adquirido maior interesse como forma de aliviar
a pressão de consumo nos grandes centros urbanos e de atingir maior
eficiência energética é o \gls{nilm}, seja no mundo coorporativo --- atraindo
empresas como \emph{Intel} e \emph{Belkin} --- como no meio acadêmico,
em especial nos paises desenvolvidos. 

O monitoramento não-invasivo se utiliza de um único medidor central no
fornecimento de energia da residência para identificar o consumo dos
aparelhos através dos distúrbios causados na rede elétrica pelos
mesmos. Para isso, empregam-se técnicas de processamento
de sinais, inteligência computacional e estatística para identificar
os padrões dos distúrbios e correlacioná-los com o aparelho de origem.
Dependendo da metodologia aplicada no \acs{nilm}, isso pode ser
realizado de maneira cega, ou seja, encontrando padrões recorrentes
na rede elétrica e os identificando quando eles se repetem.
A topologia mais comum pode ter sua operação resumida através destes
passos:

\begin{itemize}
\item Aquisição de dados: a eficácia do \acs{nilm} depende diretamente
da capacidade do medidor de extrair informação da rede elétrica, sendo
desejável amostragens com frequências elevadas ou com uma maior
quantidade de representações, independentes entre si;
\item Extração de características: com base na informação obtida pelo
sistema de aquisição de dados, é possível transformá-la em
características que serão utilizadas pelas etapas seguintes para obter
o consumo desagregado por equipamento;
\item Identificação de distúrbios devido à mudança de operação de um
aparelho: essa tarefa normalmente é realizada ao buscar na informação
disponível um distúrbio causado pela mudança da operação de um
aparelho elétrico;
\item Identificação do aparelho e seu estado de consumo: nessa etapa, se
processam as características do consumo de forma a reconhecer os
padrões. Geralmente essa tarefa ocorre somente quando é identificado
um distúrbio na rede, reduzindo o processamento dos dados.
\end{itemize}

As aplicações do \acs{nilm} são diversas, como monitoramento da
qualidade de energia, diagnóstico de carga, identificação de aparelhos
defeituosos ou com consumo excessivo de energia. No que se refere à
eficiência energética, o mesmo pode auxiliar destas maneiras:

\begin{itemize}
\item Auxiliar na obtenção de dados para estudos de eficiência
energética de eletricidade: no caso do setor residencial, os estudos
de eficiência energética precisam de dados com informação desagregada
por aparelho para obter melhor precisão e direcionar os esforços para
sua obtenção. Atualmente, o levantamento para o setor residencial é
realizado através de pesquisas de posse e hábito de consumo, havendo
tanto uma demora para a sua obtenção, quanto sofre de certa degeneração
devido a sua imprecisão. Essa tecnologia pode auxiliar no processo de
obtenção dessa informação, melhorando a precisão e agilizando o
processo de obtenção de dados recentes;
\item Fornecer o retorno de informação de consumo desagregado por
aparelho para o consumidor: diversas pesquisas nos países
desenvolvidos mostram que retornar uma informação mais detalhada para
o consumidor, além daquela contida na conta de energia é favorável
para incentivá-lo a tomar ações para redução do consumo e
consequentemente melhor eficiência energética. Esses
estudos indicam que quanto maior for a quantidade de informação
disponível, melhor será essa capacidade, sendo o melhor caso a
informação de consumo desagregado por aparelho em tempo-real,
justamente a capacidade da tecnologia. Porém, nesse caso, não apenas se
faz necessário somente o retorno da informação de consumo da mesma,
mas também o envolvimento de outras áreas do conhecimento como
psicologia e \emph{design} da informação para que seu emprego seja
eficiente e motive o consumidor para tomar ações sustentáveis no uso
de energia.
\end{itemize}

O \acs{nilm} pode se aproveitar das redes elétricas inteligentes para
obter a informação desagregada, facilitando a impregnação do método e
assim, maximizando seu potencial de eficiência energética ao atender
ambos casos anteriores com maior escala. Porém, isso está sujeito a limitação
da escolha da configuração dos medidores inteligentes no Brasil, de
forma que o projeto do \acs{nilm}, apesar de ser desejável sua
impregnação, deve ser realizado sem a certeza de poder utilizar a
estrutura provida pelos medidores inteligentes.

A pluralidade de técnicas aplicadas na tecnologia envolvendo o tema
revelam que o seu projeto, apesar de parecer simples, na verdade
engloba diversos desafios. O maior desafio encontrado pelos autores é
expandir a técnica para a aplicação em condições reais de operação das
redes elétricas residenciais, aonde irão estar presentes diversos
aparelhos operando simultaneamente, alguns deles com dinâmica no seu
consumo tornando complexa a identificação dos padrões deixados na rede
por outros equipamentos.

\section{Objetivo}

O \acs{cepel} vem atuando no desenvolvimento de um \acs{nilm}. Este
trabalho tem como objetivo auxiliar no progresso desse dispositivo,
sendo desenvolvido em parceria com o \acs{cepel}.

O trabalho foi realizado com o intuíto de centralizar e uniformizar,
na medida do possível, a informação relevante para o projeto dessa
tecnologia para facilitar seu desenvolvimento no Brasil quando levando
em conta o tema de eficiência energética. Para isso foi realizado um
extenso levantamento bibliográfico desde as origens das questões de
eficiência energética às diversas técnicas aplicadas para realizar a
desagregação do consumo, informação essa que não estava clara ou de
fácil acesso.

A principal intenção de aplicação da tecnologia pelo \acs{cepel} é
para o auxilio nas pesquisas de posse e hábito de eletrodomésticos, e
por isso o trabalho irá focar no setor residencial. Outro motivo para
focar apenas nesse setor é a maior dificuldade de aplicação no setor
comercial e industrial, que apresentam redes elétricas com um nível de
desafio maior do que aquelas do setor residencial. Não obstante a rede
residencial já apresenta obstáculos suficientes a serem superados,
como a aplicação do \acs{nilm} para as condições de operação mais
ativas da mesma. Nesses momentos há uma maior quantidade de
equipamentos operando, o que, possivelmente, irá adicionar dinâmica no
consumo da rede, dificultando a identificação dos eventos de transição
dos estados de operação dos aparelhos, onde poderá ocorrer a geração
de falsos eventos devido a essa dinâmica. Essa dinâmica também irá
tornar a identificação de padrões mais complexa, aonde a discriminação
terá de ocorrer com distorções em seus padrões.

\section{Contribuições do Trabalho}

Como foi dito, o \acs{nilm} constitui-se de diversas etapas, sendo
necessário tratar de todas elas para que o projeto seja aplicável.
Cada uma delas apresenta um grande nível de dificuldade, de tal modo
que nem sempre é possível tratar de ambos temas em um único trabalho.
Para dar continuidade ao projeto, é necessário haver um ambiente
único que agregue as diversas abordagens utilizadas por autores
diferentes que contribuam para o projeto, de forma que não
seja perdida informação durante a evolução do mesmo. Esse ambiente
foi implementado levando em conta, além da questão anterior, as
dificuldades encontradas no desenvolvimento da tecnologia por outros
autores.

O trabalho em questão irá tratar da etapa de identificação de
distúrbios devido à mudança de operação de um aparelho --- eventos de
transitório ---, mas, como se verá ao longo de seu corpo, diversas
considerações serão realizadas para auxiliar em futuros trabalhos.
Para tratar dessa etapa, é proposto um algoritmo genético que irá
realizar a otimização dos parâmetros empregados na metodologia
proposta pelo \acs{cepel}. A abordagem do \acs{cepel} foi extendida e
implementada em um ambiente de análise. Foram fornecidos dados pelo
\acs{cepel} com cenários de aplicação prática, contendo operação
simultânea de aparelhos com dinâmica de consumo.

Com o intuíto de estudar as características dos distúrbios,
utilizou-se \gls{som} para representá-las e compreender seu
comportamento. Essa técnica foi analisada tanto quando atuando como um
detector de eventos ou como um meio de aperfeiçoar, quanto através de
um método para validar as respostas da metodologia proposta.

\section{Estrutura Capitular}

Os Capítulos \ref{chap:ee} e \ref{chap:ee_retorno} condensam a
informação sobre as aplicações do \acs{nilm} para eficiência
energética. O Capítulo~\ref{chap:ee} irá tratar a origem do tema de
eficiência energética (Sessão~\ref{sec:ee_origem}) e prosseguir até as
suas necessidades para sua melhor obtenção no Brasil
(Sessão~\ref{sec:ee_dificuldades}), em especial ao que concerne
energia elétrica para o setor residencial --- as pesquisas de posse e
hábito de eletrodomésticos.

Por outro lado, o Capítulo~\ref{chap:ee_retorno} trata de como
expandir o potencial de eficiência energética através de um novo
programa abrangendo o tema. Antes de tratá-lo, é realizado
considerações de como são obtidos os potenciais de eficiencia
energética para o setor residencial, podendo ser considerado como uma
extensão do Capítulo~\ref{chap:ee}. Somente na
Sessão~\ref{sec:ee_res_exp} é tratado como expandir esse potencial
através do retorno de informação para o consumidor. É realizado um
resumo das diversas informações obtidas ao observar estudos no
exterior envolvendo o tema.

O Capítulo~\ref{cap:nilm} irá realizar uma introdução sobre a
tecnologia na Sessão~\ref{sec:nilm_aspec_gerais}, para então realizar
um extenso levantamento da informação envolvida do ponto de vista de
desenvolvimento, identificando os aspectos já resolvidos, as
dificuldades e as tendências para o seu futuro na
Sessão~\ref{sec:nilm_mundo}. Em sequência, as técnicas aplicadas pelo
\acs{cepel} e os trabalhos em colaboração com a \acs{coppe} serão
tratados na Sessão~\ref{sec:nilm_cepel}, aonde serão realizado o
levante das dificuldades e necessidades do projeto.

As informações levantadas na Sessão~\ref{sec:nilm_cepel} serão
resumidas e discutidas na Sessão~\ref{sec:motivacao_framework} do
Capítulo~\ref{chap:framework}. Esse capítulo dedica-se à descrição do
ambiente desenvolvido com o objetivo de melhorar a capacidade de
análise e unificar o projeto para garantir sua continuidade. Na
Sessão~\ref{sec:otimizacao} é descrito o principal módulo desse
ambiente, que será utilizado na metodologia final do trabalho para a
otimização dos parâmetros para condições ruidosas simuladas em
situações controladas.

O Capítulo~\ref{chap:metodologia} irá detalhar a base de dados
(Sessão~\ref{sec:base_de_dados}), bem como tratar a metodologia para a
aplicação do algoritmo genético (Sessão~\ref{sec:aplic_es}), e a
apresentação da informação necessária para o entendimento do \acs{som}
(Sessão~\ref{ssec:som}). Os resultados para a metodologia empregada
por este trabalho estão no Capítulo~\ref{chap:resultados}. As
conclusões do trabalho encontram-se no Capítulo~\ref{chap:conclusao}.

\glsunsetall
