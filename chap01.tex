\chapter{Introdução}

A proposta de sustentabilidade, surgida como resposta à degradação do
meio ambiente devido ao processo de industrialização, leva, em seu
discurso, a questão das limitações tanto em termos de recursos
presentes na natureza, ou quanto da sua capacidade de absorver resíduos
de processos em ordens cada vez mais intensas. Sua vertente
predominante traz a ideia do desenvolvimento sustentável, que busca
regular a demanda de recursos do meio ambiente, de modo que seja
possível o desenvolvimento sem que haja sua deterioração para as
gerações futuras.

Adicionalmente, há uma correlação entre os impactos ambientais e a
necessidade energética, onde os motivos prevalecentes para suas
causas têm como fator de destaque a sua vinculação à cadeia da
energia, desde a produção ao uso final. Entretanto, essa dependência
não é decorrente apenas da intensidade, mas da eficiência energética
relacionada com o seu consumo. Entende-se como eficiência energética a
capacidade de produzir os mesmos resultados finais realizando o menor
consumo possível de energia.

A eletricidade tem uma participação cada vez maior na matriz
energética mundial devido à sua maior versatilidade, eficiência,
limpeza, segurança e conveniência, quando comparada com as outras
fontes energéticas. A eficiência energética do ponto de vista da
eletricidade traz uma série de vantagens, além de benefícios
ambientais:

\begin{itemize}
\item reduz ou posterga as necessidades de investimentos em geração, transmissão
e distribuição de energia elétrica;
\item reduz o custo de energia para o consumidor final;
\item contribui para a confiabilidade do sistema elétrico;
\item traz o aumento da atividade econômica com a redução da
intensidade energética.
\end{itemize}

Além disso, o crescimento do consumo nos grandes centros urbanos tem
levado ao surgimento de uma nova rede elétrica --- as redes elétricas
inteligentes ---, que procura descentralizar a geração devido à
exigência excessiva da capacidade de transmissão e distribuição da
versão usual centralizada. Dependendo da regulamentação dos medidores
inteligentes --- os medidores das novas redes elétricas ---, uma série
de novas possibilidades podem ser exploradas, como:

\begin{itemize}
\item maior versatilidade de operação e planejamento das redes
elétricas devido à maior informação presente; 
\item tarifação variável conforme a demanda na rede, incentivando os
consumidores a deslocarem cargas não-essenciais para operarem em
horários fora de ponta de forma a reduzir ou postergar a necessidade
de investimentos na expansão da rede --- deslocamento de carga;
\item o retorno da informação do consumo em tempo real nas residências pode ser
explorado para obter eficiência energética, assunto melhor
debatido a seguir.
\end{itemize}

\section{Motivação}

Uma tecnologia que tem adquirido maior interesse como forma de aliviar
a pressão de consumo nos grandes centros urbanos e de atingir maior
eficiência energética é a \gls{nilm}, seja no mundo corporativo --- atraindo
empresas como \emph{Intel} e \emph{Belkin} --- como no meio acadêmico,
em especial nos países desenvolvidos.

O monitoramento não-invasivo utiliza-se de um único medidor central no
fornecimento de energia da residência para identificar o consumo dos
equipamentos através dos distúrbios causados na rede elétrica pelos
mesmos. Para isso, empregam-se técnicas de processamento
de sinais, inteligência computacional e estatística para identificar
os padrões dos distúrbios e correlacioná-los com o equipamento de origem.
Dependendo da metodologia aplicada no \acs{nilm}, isso pode ser
realizado de maneira cega, ou seja, encontrando padrões recorrentes
na rede elétrica e os identificando quando eles se repetem.
A topologia mais comum pode ter sua operação resumida através destes
passos:

\begin{itemize}
\item Aquisição de dados: a eficácia do \acs{nilm} depende diretamente
da capacidade do medidor de extrair informação da rede elétrica, sendo
desejável amostragens com frequências elevadas ou com uma maior
quantidade de representações, independentes entre si;
\item Extração de características: com base na informação obtida pelo
sistema de aquisição de dados, é possível transformá-la em
características que serão utilizadas pelas etapas seguintes para obter
o consumo desagregado por equipamento;
\item Detecção de eventos de transitório na rede devido à mudança de
operação de um equipamento: quando o \acs{nilm} utiliza a informação no
transitório na operação dos equipamentos para identificar a operação do
equipamento e estimar o seu consumo --- configuração mais comum ---, é
necessário detectar esses momentos e diferenciá-los de
ruídos causados por demais oscilações no consumo da rede;
\item Identificação da operação do equipamento e seu consumo:
nessa etapa, processam-se as características do consumo de forma a
reconhecer os padrões para estimar o consumo dos
equipamentos. Geralmente essa tarefa ocorre somente quando é
identificado um distúrbio na rede, reduzindo o processamento dos
dados.
\end{itemize}

As aplicações do \acs{nilm} são diversas, como monitoramento da
qualidade de energia, diagnóstico de carga, identificação de equipamentos
defeituosos ou com consumo excessivo de energia. No que se refere à
eficiência energética, o mesmo pode auxiliar destas maneiras:

\begin{itemize}
\item Auxiliar na obtenção de dados para estudos de eficiência
energética de eletricidade: no caso do setor residencial, os estudos
de eficiência energética precisam de dados com informação desagregada
por equipamento para obter melhor precisão e direcionar os esforços para
sua obtenção. Atualmente, o levantamento para o setor residencial é
realizado através de pesquisas de posse e hábito de consumo, havendo
tanto uma demora para a sua obtenção, quanto sofre de certa degeneração
devido à sua imprecisão. Essa tecnologia pode auxiliar no processo de
obtenção dessa informação, melhorando a precisão e agilizando a
obtenção de dados recentes;
\item Fornecer o retorno de informação de consumo desagregado por
equipamento para o consumidor: diversas pesquisas nos países
desenvolvidos mostram que retornar uma informação mais detalhada para
o consumidor --- além daquela contida na conta de energia --- é favorável
no sentido de incentivá-lo a tomar ações para redução do consumo e,
consequentemente, melhor eficiência energética. Esses
estudos indicam que quanto maior for a quantidade de informação
disponível melhor será essa capacidade, sendo o melhor caso a
informação de consumo desagregado por equipamento em tempo-real,
justamente a capacidade dessa tecnologia. Porém, nesse caso, não apenas se
faz somente necessário o retorno da informação de consumo da mesma,
mas também o envolvimento de outras áreas do conhecimento como
psicologia e \emph{design} da informação para que seu emprego seja
eficiente e motive o consumidor para tomar ações sustentáveis no uso
de energia.
\end{itemize}

O \acs{nilm} pode aproveitar-se das redes elétricas inteligentes para
obter a informação desagregada através da utilização dos medidores
inteligentes --- subordinado à capacidade de fornecer informação dos
mesmos, onde uma taxa mínima de 1~\acs{hz} é indicada nesse sentido.
Isso facilitaria a impregnação do método e, com isso, maximizaria seu
potencial de eficiência energética ao atender ambos itens anteriores
em maior escala. Porém, isso está sujeito a limitação
da escolha da configuração dos medidores inteligentes no Brasil, de
forma que o projeto do \acs{nilm}, apesar de ser desejável sua
impregnação, deve ser realizado sem a certeza de poder utilizar a
futura estrutura provida pelas redes elétricas inteligentes no país.

A pluralidade de técnicas aplicadas na tecnologia envolvendo o tema
revelam que o seu projeto, apesar de parecer simples, na verdade
engloba diversos desafios. O maior desafio geralmente encontrado pelos
autores é expandir a técnica para a aplicação em condições reais de
operação das redes elétricas residenciais, aonde estarão presentes
diversos equipamentos operando simultaneamente, alguns deles com dinâmica
no seu consumo, o que torna complexa a identificação dos padrões
deixados na rede por demais equipamentos.

\section{Objetivo}

O \acs{cepel} vem atuando no desenvolvimento de um \acs{nilm}. Este
trabalho tem como objetivo auxiliar no progresso desse dispositivo,
sendo desenvolvido em parceria com o \acs{cepel}.

A principal intenção de aplicação da tecnologia pelo \acs{cepel} é
para o auxilio nas \glspl{pph} no setor residencial. Outro motivo para
focar apenas nesse setor é a maior dificuldade de aplicação no setor
comercial e industrial, que apresentam redes elétricas com um nível de
desafio mais elevado quando comparado ao setor residencial. Não
obstante, a rede residencial já apresenta obstáculos suficientes a
serem superados, em especial nas condições de operação mais ativas da
mesma. Nesses momentos, há uma maior quantidade de equipamentos
operando, o que adiciona, potencialmente, dinâmica no consumo da rede
e, com isso, torna a tarefa a detecção dos eventos de transitório
dos estados de operação dos equipamentos não trivial pois há uma menor
relação sinal-ruído. Essa dinâmica também tornará a identificação de
padrões dos eletrodomésticos mais complexa, aonde o discriminador terá
de lidar com distorções em seus padrões. 

O trabalho atual irá expandir a metodologia proposta pelo \acs{cepel}
para operação nessas condições, propondo uma abordagem sistemática que
permita o ajuste da técnica aplicada de acordo com as condições
presentes nas redes elétricas residenciais, limitando-se ao estudo da
capacidade da metodologia proposta em detecção de eventos de
transitório. Porém, como será visto a seguir, o trabalho trouxe
diversas outras contribuições.

\section{Contribuições do Trabalho}

Durante o levantamento bibliográfico, percebeu-se um forte apelo no
exterior às questões de eficiência energética que vão além do intuito
de aplicação do \acs{nilm} nas \glspl{pph}. Diversos estudos nos
Estados Unidos e Europa Ocidental citam a capacidade do consumidor de
economizar energia ao retornar sua informação de consumo de energia
elétrica. Com o objetivo de compreender melhor como isso pode ser
realizado, este trabalho procurou explorar em maiores detalhes essas
questões, trazendo uma compilação dos estudos que parecem ser de maior
relevância envolvendo esse tema.

Como a própria questão das \glspl{pph} estão
relacionadas com a eficiência energética, ficou evidente a necessidade
de trazer no corpo do trabalho um levantamento da origem do tema ---
discurso ambiental e ecologia --- e o que tem sido feito no mundo e no
Brasil nesse sentido para tornar o assunto de mais fácil acesso para
os leitores, que podem não estar familiarizados com o tema e
importância das questões ambientais envolvidas.

Além disso, o \acs{nilm} vem sido desenvolvido desde 1992, sendo
possível encontrar uma vasta quantidade de metodologias propostas por
diversos autores. O trabalho centralizou e uniformizou, na medida do
possível, a informação relevante para o projeto dessa tecnologia para
facilitar seu desenvolvimento pelo \acs{cepel} e por demais autores
quando levando em conta o tema de eficiência energética.

Já quanto à metodologia aplicada, como foi dito, o \acs{nilm}
constitui-se de diversas etapas, sendo necessário tratar de todas elas
para que o projeto seja aplicável. Cada uma delas precisa ser estudada
e compreendida, de tal modo que nem sempre é possível tratar de
todos os pontos em um único trabalho. O \acs{cepel} propôs uma nova
abordagem para a detecção de eventos de transitório na operação dos
equipamentos, que utiliza como núcleo um filtro de derivada de Gaussiana.
Essa abordagem será explorada e estendida pelo trabalho, porém,
limitar-se-á à questão de detecção de eventos de transitório --- apenas
uma das etapas necessárias para a operação do \acs{nilm}. Indo além, é
importante notar que o objetivo do trabalho é estudar o comportamento
da metodologia proposta pelo \acs{cepel} e sua extensão realizada no
trabalho em cenários de aplicação prática. Os conjuntos de dados com
essas condições foram fornecidos pelo próprio \acs{cepel}.

A evolução da metodologia proposta pelo \acs{cepel} foi realizada em
termos de estruturação e sistematização. Também no levantamento
bibliográfico se percebeu a capacidade de complementação das técnicas
aplicadas, de forma que um ambiente único de análise não apenas
permite uma melhor compreensão das técnicas e rapidez no
desenvolvimento do projeto, mas também de explorar a capacidade de
suplementar outras técnicas, o que permite ao \gls{nilm} explorar uma
maior quantidade de equipamentos e/ou eficiência de desagregação do
consumo por equipamento.

O ajuste dos parâmetros realizados pela metodologia do \acs{cepel} era
feito empiricamente, aonde se viu a necessidade de sistematizar o
processo. Foi implementado um sistema de otimização através de um
algoritmo genético de estratégia evolutiva para auxiliar no ajuste dos
parâmetros. 

\section{Estrutura Capitular}

Os Capítulos \ref{chap:ee} e \ref{chap:ee_retorno} compilam a
informação sobre as aplicações do \acs{nilm} para eficiência
energética. O Capítulo~\ref{chap:ee} tratará a origem do tema de
eficiência energética (Seção~\ref{sec:ee_origem}) e prosseguir até as
suas necessidades para sua melhor obtenção no Brasil
(Seção~\ref{sec:ee_dificuldades}), em especial ao que concerne
energia elétrica para o setor residencial.

Por outro lado, o Capítulo~\ref{chap:ee_retorno} trata da expansão do
potencial de eficiência energética através de um novo programa
abrangendo o tema. Antes de tratá-lo, é realizado considerações de
como são obtidos os potenciais de eficiência energética para o setor
residencial na Seção~\ref{sec:ee_setor_residencial}. Somente em
seguida, na Seção~\ref{sec:ee_res_exp}, é tratado como expandir esse
potencial através do retorno de informação para o consumidor. É
feito uma compilação das informações relevantes obtidas ao
observar estudos no exterior envolvendo o tema.

O Capítulo~\ref{cap:nilm} realizará uma introdução sobre a tecnologia
na Seção~\ref{sec:nilm_aspec_gerais}, para então trazer um extenso
levantamento da informação envolvida do ponto de vista de
desenvolvimento, identificando os aspectos já resolvidos, as
dificuldades e as tendências para o seu futuro na
Seção~\ref{sec:nilm_mundo}. Em sequência, as técnicas aplicadas pelo
\acs{cepel} e os trabalhos anteriores em colaboração com a \acs{coppe}
serão tratados na Seção~\ref{sec:nilm_cepel}, aonde será realizado o
levante das dificuldades e necessidades do projeto. Nessa seção,
também está disponível a metodologia original do \acs{cepel} na
Subseção~\ref{ssec:met_cepel}.

As informações levantadas na Seção~\ref{sec:nilm_cepel} serão
resumidas e discutidas na Seção~\ref{sec:motivacao_framework} do
Capítulo~\ref{chap:framework}. Esse capítulo se dedica à descrição do
ambiente desenvolvido com o objetivo de melhorar a capacidade e
sistematizar a análise, bem como unificar o projeto para facilitar sua
continuidade. As alterações da metodologia original podem ser
encontradas ao longo desse capítulo, no entanto, na
Seção~\ref{sec:otimizacao} é descrita a principal ampliação realizada
na metodologia original, o Módulo de Otimização dos Parâmetros. 

O Capítulo~\ref{chap:metodologia} irá descrever as condições simuladas
na base de dados fornecida pelo \acs{cepel}
(Seção~\ref{sec:base_de_dados}), contendo tanto condições mais simples
com apenas acionamentos e desacionamentos, quanto condições mais
próximas às reais nas redes elétricas residenciais, com operação
simultânea de diversos equipamentos e dinâmica no consumo de alguns
equipamentos. Também é apresentado a metodologia para a aplicação do
algoritmo genético para ajuste dos parâmetros
(Seção~\ref{sec:aplic_es}). Os resultados para a metodologia empregada
por este trabalho estão no Capítulo~\ref{chap:resultados}. A conclusão
do trabalho encontra-se no Capítulo~\ref{chap:conclusao}.

\glsunsetall
