\chapter{A questão energética-ambiental}

% O ser humano e sua capacidade de ir além da energia endossomática
Estima-se que o ser humano consome entre 2500 e 3000 quilocalorias/dia sob a 
forma de alimentos \cite{hemery}. Apenas cerca de 20\% dessa energia poderá ser 
reinvestida em atividades, de forma que a capacidade humana de produção através
de energia endossomática - provida por seu metabolismo - é de apenas 500 a 600 
quilocalorias/dia. No entanto, é a especificidade do ser humano de utilizar 
fluxos de energia exossomáticos - fontes não provenientes de seu metabolismo 
- que o permite transformar o meio facilitando e melhorando sua qualidade de
vida \cite{rippel}.

% A utilização de energia exossomática e os danos ao planeta
Os fluxos exossomáticos estão presentes em todas sociedades e são dependentes do
nível de desenvolvimento das mesmas. Em sociedades primitivas, os resultados dos
atos da humanidade exerciam pouca influência no meio ambiente, pois estes eram
absorvidos por aquele. Entranto, com o surgimento de sociedades industriais,
ocorreu uma mudança radical no uso dos recursos naturais e nos seus efeitos
ambientais. Com a produção e o consumo de massa, baseados no uso intensivo do
petróleo e da eletricidade como fontes energéticas, a agressão humana ao meio 
ambiente tornou-se maior de maneira que o mesmo, em muitas situações, não 
consegue mais reequilibrar-se. Os principais problemas ambientais que surgiram 
no cenario mundial tem como fator de destaque vinculados a cadeia da energia, 
da produção ao uso final \cite{rippel,pen15_eff_energ,jatoba}. 

% A questão de sustentabilidade
% Ecologia radical
Em resposta à degradação do meio ambiente se dá a questão de
sustentabilidade. As primeiras objeções à poluição ambiental devido ao processo 
de industrialização, em especial sua aceleração no início do século XX 
foi da ecologia radical, dando atenção as questões de proteção e conservação da natureza.
Há uma separação de territórios especiais para uma proteção integral e, numa
visão em larga medida romantizada da natureza, muitas vezes sem permissão 
de nenhum uso antrópico \cite{jatoba}.

% Ambientalismo moderado
A política ambiental preponderante atualmente é o discurso do ambientalismo moderado, 
surgido em meio a Crise do Petróleo na década 1970, onde, percebendo-se a
inviabilidade de sustentação do modelo econômico levando em conta o esgotamento
progressivo dos recursos naturais do planeta, se fez necessário a discussão de como 
colocar em prática as propostas da ecologia radical, mas tendo cautela de não 
necessariamente frear o crescimento econômico ou alterar substancialmente o 
modelo de desenvolvimento vigente - trazendo o conceito de desenvolvimento
sustentável. Devido à Crise, houve uma busca mundial pela redução da 
dependência no petróleo e outras fontes fósseis em sua matriz energética, 
procurando fontes renováveis e menos poluentes que as fontes fósseis 
\cite{jatoba,eff_dec_energ_2012,rippel}. 

% Ressaltar aqui a componente social do ambientalismo 

% Eficiencia energetica
A maneira mais efetiva, eleita pela \gls{onu} \cite{onu}, de reduzir os impactos ambientais locais e globais sem
prejuizos ao desenvolvimento, refletindo a imagem de desenvolvimento
sustentável, é através da eficiência energética \cite{rippel,dissert_artur_cursino}. O uso eficiente de 
energia deve ser entendido como o menor consumo possível de energia para obter uma mesma 
quantidade de produto ou serviço \cite{pen15_eff_energ}. A melhoria pode ser obtida nos diversos
fluxos energéticos da sociedade, através da melhoria da geração de uma fonte 
energética, substituição por outra fonte mais eficiente, ou melhoria da
eficiência nos consumidores. Nas sociedades modernas a eletricidade 
como recurso energético vem adquirindo importância cada vez mais vital devido a
sua versabilidade e eficiência \cite{pen15_eff_energ,dissert_caires}, 
obtendo um lugar de destaque quando em debate a eficiência energética. 

Além dos beneficios ambientais, a eficiência energética \cite{jannuzzi,slides_eff_energetica}: 

\begin{itemize}
\item reduz ou posterga as necessidades de investimentos em geração, transmissão 
e distribuição de energia elétrica; 
\item reduz o custo de energia para o consumidor final; 
\item contribui para a confiabilidade do sistema elétrico; 
\item traz o aumento da intensividade econômica com a redução da intensividade
energética. 
\end{itemize}

% Eficiencia energetica no mundo desenvolvido
A tendência nos países desenvolvidos é a realização de esforços 
cada vez maiores no sentindo de aumentar a eficiência energética, 
estimando-se, por exemplo, um consumo 49\% maior nos paises da \gls{ocde} 
em 1996 caso não houvessem sido adotadas medidas de racionalização 
e eficiência energética após a Crise do Petróleo
\cite{goldemberg,slides_eff_energetica}.

% Eficiencia energética nos países em desenvolvimento 
Os \glspl{ped} não possuem uma capacidade de redução energética tão grande
quanto os países da \gls{ocde}, uma vez que seu consumo energético per capita é
reduzido, justamente por ser necessário o desenvolvimento para aumentar o
consumo. Por outro lado, é possível aliviar a pressão sobre a
oferta energética através da melhoria dos níveis de eficiência energética nos
diversos setores da sociedade, contrapondo ao pensamento de que para que
haja desenvolvimento é preciso que ocorram impactos ambientais e crescimento no
consumo total de energia - chamado de efeito \emph{leapfrogging}
\cite{goldemberg,dissert_maria_ines_matos}. 

% Eficiencia energética no Brasil
Assim, a preocupação com eficiência energética se justifica mesmo no Brasil, 
que apresenta quase metade de sua matriz energética proveniente de fontes 
renováveis e preços de produção de energia economicamentes competitivos.
Diversas iniciativas vêm sendo empreendidas há mais de 20 anos, dentre elas se
destacam \cite{projecao_demanda_2012,slides_eff_energetica}:

\begin{enumerate}
\item 
\end{enumerate}

% Consumo residencial bastante elevado no brasil

% Necessidade de obter o perfil de consumo para escolher equipamentos a serem
% melhorados

% Ainda, estudos no exterior mostram que o feedback de consumo poder mudar o
% habito de consumo dos consumidores. Problema, Brasil energia barata e
% dificilmente consumidores mudam habito por isso.

%
%
%
%
%
%\cite{techreport-energetico}

