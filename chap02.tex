\chapter{Eficiência Energética}
\label{chap:ee}
\glsresetall

O trabalho atual está completamente imerso no escopo de \gls{ee},
sendo realizado neste capítulo uma abordagem ao tema. 
A Sessão~\ref{sec:ee_origem} refere a origem da \gls{ee} no mundo, estando
intrinsecamente ligada ao discurso ambiental e a ideia de desenvolvimento
sustentável. Em seguida, reduz-se a abrangência para o Brasil
(Sessão~\ref{sec:ee_brasil}), apresentando logo posteriormente
(Sessão~\ref{sec:ee_dificuldades}) as dificuldades para o planejamento em \gls{ee} 
e a solução proposta levando em conta o foco do trabalho atual.

\section{Da Origem}
\label{sec:ee_origem}

% O ser humano e sua capacidade de ir além
Estima-se que o ser humano consome entre 2.500 e 3.000~\acrshort{kcal}/dia sob a 
forma de alimentos, onde apenas cerca de 20\% dessa energia poderá ser 
reinvestida em atividades, de forma que a capacidade biológica humana de produção
através de energia endossomática --- provida por seu metabolismo --- é de 
apenas 500 a 600 \acrshort{kcal}/dia. É a particularidade do ser humano de
utilizar fluxos de energia exossomáticos --- fontes não provenientes de seu metabolismo 
--- que o permite suplantar sua capacidade biológica de interagir com o meio,
tornando possível uma melhor qualidade de vida \cite{rippel}. 
Atualmente, o consumo energético exossomático médio per capita diário 
mundial supera 50.000~\acrshort{kcal} \cite{world_statics_2012}.

% A utilização de energia exossomática e os danos ao planeta
Os fluxos exossomáticos estão presentes em todas sociedades e são dependentes do
nível de desenvolvimento das mesmas. Em sociedades primitivas, os resultados dos
atos da humanidade exerciam pouca influência no meio ambiente, pois estes eram
absorvidos por aquele. Entranto, com o surgimento de sociedades industriais,
ocorreu uma mudança radical no uso dos recursos naturais e nos seus efeitos
ambientais. Com o crecimento populacional adicionados da produção e do consumo 
de massa, esses baseados no uso intensivo do petróleo e da eletricidade quanto 
fontes energéticas, a agressão humana ao meio ambiente tornou-se maior de maneira 
que o meio ambiente, em muitas situações, não consegue mais se reequilibrar. 
Os principais problemas ambientais que surgiram no cenario mundial tem 
como fator de destaque vinculados a cadeia da energia, 
desde a produção ao uso final \cite{rippel,jatoba}.

% A questão de energia e concentração de renda
Indo mais além, a dependência desenvolvimento-energia não está somente
atrelada ao patamar macroscópico da sociedade, mas também relacionada
ao modo que o desenvolvimento está difundido nela. Diferentes classes
sociais de uma determinada sociedade detêm modos de consumo, bem como
formas de acesso a bens e serviços distintos. Nesse caso, aqueles com
menor poder aquisitivo não apenas consomem menos energia, mas também
tipos diferentes da mesma, geralmente mais poluentes e menos
eficientes como aquelas oriundas de biomassa
\cite{rippel}. Percebe-se, portanto, que o avanço do perfil energético
de uma sociedade não indica necessariamente a sua evolução como um
todo, sendo necessário notar como a energia está difundida nela.

% A questão de sustentabilidade
% Ecologia radical
Em resposta à degradação do meio ambiente se dá a proposta da
sustentabilidade. As primeiras objeções à poluição ambiental devido ao processo 
de industrialização, em especial após sua aceleração no início do século XX,
foi da ecologia radical, dando atenção as questões de proteção e conservação da natureza.
Há uma separação de territórios especiais para uma proteção integral e, numa
visão em larga medida romantizada da natureza, muitas vezes sem permissão 
de nenhum uso antrópico \cite{jatoba}.

% Ambientalismo moderado
A política ambiental preponderante atualmente é o discurso do ambientalismo moderado, 
surgido em meio a Crise do Petróleo na década 1970, onde, percebendo-se a
inviabilidade de sustentação do modelo econômico levando em conta o esgotamento
progressivo dos recursos naturais do planeta, fez-se necessário a discussão de como 
colocar em prática as propostas da ecologia radical, mas tendo cautela de não 
necessariamente frear o crescimento econômico ou alterar substancialmente o 
modelo de desenvolvimento vigente --- trazendo o conceito de desenvolvimento
sustentável. Devido à Crise, houve uma busca mundial pela redução da 
dependência no petróleo e outras fontes fósseis na matriz energética, 
procurando fontes renováveis e menos poluentes que as fontes fósseis 
\cite{jatoba,epe_eficiencia_2012,rippel}. 

% A questão social
Vale aqui ressaltar que as questões ecológicas não estão ligadas apenas a
economia, mas refletem uma intensa dinâmica econômica, social, cultural e política 
que são contraditórias entre si. Há também uma linha do discurso ambiental que destaca a
justiça social, notando que os mais vulneráveis aos problemas
ambientais são justamente os mais pobres, os quais serão os mais afetados na
hipótese de agravamento da crise ambiental \cite{jatoba}.

% Eficiencia energetica
A maneira mais efetiva, eleita pela \gls{onu}, de reduzir 
os impactos ambientais locais e globais sem prejuizos ao desenvolvimento, 
refletindo a imagem de desenvolvimento sustentável, é através da \gls{ee}
\cite{rippel,onu,dissert_cursino}. O uso eficiente de energia deve ser entendido como 
o menor consumo possível de energia para obter uma mesma 
quantidade de produto ou serviço \cite{pne30_eff_energ}. 
As ações de \gls{ee} compreendem modificações ou aperfeiçoamentos tecnológicos ao longo
da cadeia, mas podem também resultar de uma melhor organização, conservação e
gestão energética por parte das entidades que a compõem \cite{pnef}. 

% A importância da energia elétrica
A eletricidade como recurso energético vem adquirindo importância cada vez maior na 
matriz energética mundial --- e assim nos debates englobando \gls{ee} ---
devido a, geralmente, sua maior versabilidade, eficiência, limpeza, segurança e conveniência 
quando comparando com as outras fontes energéticas. O consumo de eletricidade no uso final é de 17,7\% 
da matriz energética mundial em 2010, enquanto sua participação era de 7,3\% em 1973. Estima-se para 2035 um
crescimento de 80\% da demanda de eletricidade quando utilizando o ano base de
2008, de modo a elevar a participação da eletricidade para uma parcela de 23\%. O crescimento da demanda energética 
de eletricidade é liderado pelos países não pertencentes a \gls{ocde}, dentre
eles o Brasil, que terão uma participação de cerca de 80\% do crescimento da demanda de
eletricidade mundial \cite{iea_weo2010}.

Além dos beneficios ambientais, quando levando em conta o sistema
elétrico, a \gls{ee} \cite{jannuzzi,epe_slides_eficiencia}: 

\begin{itemize}
\item reduz ou posterga as necessidades de investimentos em geração, transmissão 
e distribuição de energia elétrica; 
\item reduz o custo de energia para o consumidor final; 
\item contribui para a confiabilidade do sistema elétrico; 
\item traz o aumento da intensividade econômica com a redução da intensividade
energética. 
\end{itemize}

% Eficiencia energetica no mundo desenvolvido
A tendência nos países desenvolvidos é a realização de esforços 
cada vez maiores no sentindo de aumentar a \gls{ee}, 
estimando-se, por exemplo, um consumo 49\% maior nos paises da \gls{ocde} 
no período de 1973 a 1996 caso não houvessem sido adotadas medidas de 
racionalização e \gls{ee} após a Crise do Petróleo
\cite{goldemberg,epe_slides_eficiencia}.

\section{Eficiência Energética no Brasil}
\label{sec:ee_brasil}

% Eficiencia energética nos países em desenvolvimento 
Os \glspl{ped} não possuem uma capacidade de redução energética tão grande
quanto os países da \gls{ocde}, uma vez que seu consumo energético per capita é
reduzido, justamente por ser necessário o desenvolvimento para aumentar o
consumo. Por outro lado, é possível aliviar a pressão sobre a
oferta energética através da melhoria dos níveis de \gls{ee} nos
diversos setores da sociedade, contrapondo ao pensamento de que para que
haja desenvolvimento são necessários impactos ambientais e o crescimento no
consumo total de energia --- o chamado de efeito \emph{leapfrogging}
\cite{goldemberg}, Figura~\ref{fig:leapfrogging}. 

\begin{figure}[h!t]
\centering
\includegraphics[width=.8\textwidth]{imagens/leapfrogging.pdf}
\caption[O efeito \emph{leapfrogging}.]
{O efeito \emph{leapfrogging}. Extraído de \cite{goldemberg}.}
\label{fig:leapfrogging}
\end{figure}

% Eficiencia energética no Brasil
Por isso, a preocupação com \gls{ee} se justifica mesmo no Brasil, 
que apresenta quase metade de sua matriz energética proveniente de fontes 
renováveis e preços de produção de energia economicamentes competitivos.
Diversas iniciativas vêm sido empreendidas, dentre elas, destacam-se
\cite{pnef,pne30_eff_energ,epe_demanda_2012,epe_slides_eficiencia}:

\begin{itemize}
\item 1931: primeiro horário de verão no Brasil;
\item 1975: criado o \gls{proalcool}, para a substituição em larga escala de
veiculos movidos a derivados de petróleo por álcool; 
\item 1981: criado o Programa CONSERVE, visando à promoção da conservação de
energia na indústria, ao desenvolvimento de produtos e processos energeticamente
mais eficientes, e ao estímulo à substituição de energéticos importados por
fontes alternativas autóctones\footnote{Fontes de energia produzidas e
consumidas localmente.};
\item 1982: aprovada as diretrizes do \gls{pme}, conjunto de ações dirigidas à
conservação de energia e à substituição de derivados de petróleo;
%\item 1982: dado as diretrizes para eficiência no \gls{pme}; 
\item 1984: o \gls{inmetro} implementou o \gls{pbe}\footnote{Seu nome inicial
foi Programa de Conservação de Energia Elétrica em Eletrodomésticos, sendo
renomeado em 1992 para a nomenclatura atual.} tendo por objetivo promover a
redução do consumo de energia em equipamentos como refrigeradores, congeladores e condicionadores
de ar domésticos;
\item 1985: foi instituído o \gls{procel}, com a finalidade de integrar as
ações visando à conservação de energia elétrica no país, sendo coordenado
executivamente pela \acrshort{eletrobras}. O programa possibilitou uma
economia de energia acumulada de 32,9 TWh entre
1986 e 2008, reduzindo a demanda de ponta em 9.538 MW. Somente essa energia
economizada corresponde a investimentos evitados de $\text{R\$}$ 22,8
bilhões, enquanto os recursos investidos no programa foram de aproximadamente
$\text{R\$}$1,2 bilhão, ambos para o período citado; 
\item 1991: foi instituído o \gls{conpet}, programa similar ao \gls{procel}, mas
destinado à parcela da matriz energética proveniente dos derivados do petróleo e
do gás natural, com a coordenação executiva da \acrshort{petrobras}. 
O \gls{conpet} evitou o consumo de 1030,2 milhões de litros diesel no período
2006--2010. 
\item 1998: iniciado o \gls{pee}, regulamentado pela \gls{aneel}. As concessionárias,
permissionárias e autorizadas do setor de energia elétrica devem realizar
investimentos em pesquisa e desenvolvimento promovendo a conservação de energia elétrica. 
Atualmente esse valor é de 0,50\% da receita operacional líquida das operadoras. 
Foi realizado um investimento total em \glspl{pee} 
pelas concessionárias do setor elétrico na quantia de $\text{R\$}$ 2 bilhões até
o ano de 2006, onde se estima que o programa alcançou uma econômia média de 
4.000 GWh/ano e retirando uma carga de ponta de consumo na ordem de 1.140 MW no período de 1998--2005;
\item 2001: Lei n$^\text{o}$ 10.295/2001 determina a instituição de ``níveis
máximos de consumo específico de energia, ou mínimos de \gls{ee},
de máquinas e aparelhos consumidores de energia fabricados e comercializados no
país'', sendo regulamentada pelo Decreto n$^\text{o}$4.059/2001;
\item 2004: criado a \gls{epe}, que tem entre suas diretrizes promover estudos e
produzir informações para subsidiar planos e programas de desenvolvimento
energético ambientalmente sustentável, inclusive, de \gls{ee};
\item 2006: início do \gls{proesco}, programa destinado a financiar projetos de
\gls{ee}, cuja coordenação executiva pertence ao \gls{bndes};
\item 2011: publicado as diretrizes do \gls{pnef} \cite{pnef}, que estabelece uma meta de
10\% de redução no consumo por meio de ações de \gls{ee} no período 
de 2015--2030.
\end{itemize}

% Esforços bons, mas é tudo?
Percebe-se uma evolução brasileira em \gls{ee}, tanto na legislação, 
capacitação e conhecimentos acumulados, quanto na 
coinsciência da necessidade da \gls{ee} nos diversos setores \cite{pnef}. 
Esse patrimônio, no entanto, precisa ser constantemente atualizado 
e ter sua abrangência ampliada \cite{pne30_eff_energ}. Através do planejamento pretende-se que
recursos possam ser melhores aplicados e os resultados venham com maior
velocidade, abrangência e amplitude. Esforços desse tipo podem ser vistos no
\gls{pnef} \cite{pnef}, previsto pelo \gls{pne2030} \cite{pne30_eff_energ}, e nos
\glspl{pde} \cite{pde_2020}.

% Aonde se encontra a capacidade de EE no Brasil?
A grande parcela do potencial de \gls{ee} no Brasil, de acordo com as
estimativas realizadas a partir do \gls{beu} \cite{beu}, encontra-se nos três
maiores setores por consumo final de energia, sendo eles o industrial (37,3\%), 
transportes (30,5\%) e residencial (10,2\%), que juntos representaram cerca 
de 78\% do consumo energético do país em 2011 \cite{ben2012,epe_eficiencia_2012}.

% Especificar o foco 
O escopo deste trabalho é no setor residencial, no que tange o consumo de eletricidade como
fonte energética. O consumo total dessa fonte no setor é de 41,7\% \cite{ben2012}, 
tendo a tendência de continuar se elevando com o crescimento do número de eletrodomésticos 
e a substituição da lenha (27,8\% do consumo no setor) e carvão (2,1\%) pela eletricidade, 
para iluminação e aquecimento durante o desenvolvimento do país. 

\section{Dificuldades para o Estudo em Eficiência Energética}
\label{sec:ee_dificuldades}

% E qual o problema?
Todavia, há uma grande deficiência na base de dados para os estudos de
\gls{ee} no Brasil, explicitamente indicado no \gls{pne2030}
\cite[p.~232]{pne30_eff_energ}:

\begin{quote}
As entidades encarregadas de planejar, implementar e monitorar
programas de eficiência energética necessitam de informações, tais
como custos e rendimentos de equipamentos e veículos eficientes,
estatísticas detalhadas de vendas de equipamentos e veículos e
resultados de pesquisas de campo sobre posse e hábitos de usos dos
equipamentos e veículos e sobre as respostas dos diversos grupos de
consumidores às diferentes medidas de conservação. Bancos de dados
contendo tais informações têm sido montados por diversos países
desenvolvidos desde a década de setenta. Em função dos custos
significativos das pesquisas de campo, isto não tem ocorrido na
maioria dos países em desenvolvimento, inclusive no Brasil. Existe
atualmente investimentos do \gls{procel} na área, mas que carecem
ainda de reforços e da abordagem dos demais programas nacionais. Sem
uma base de dados consistente, que inclua o levantamento de
tecnologias disponíveis ou em estudo (e análise de sua potencialidade
de mercado) e metodologia de resultados de projetos, não se podem
modelar, de forma confiável, programas de eficiência energética no
planejamento da expansão do setor energético brasileiro.
\end{quote}

% Retomando o foco: as PPHs
Os esforços do \gls{procel} citados englobam justamente o setor
residencial, em especial detalhando seu consumo de energia elétrica de
modo a fornecer informações para os estudos de \gls{ee} no setor. Eles
são o resultado da mais abrangente \gls{pph} realizada entre
2004--2006 \cite{result_procel_2005,site_pesquisas_procel}.  Nela,
utilizou-se cerca de 15 mil questionários em 18 estados e 20
concessionárias que abrangeram não apenas o setor residencial, mas
inclusive o setor industrial e comercial, representando cerca de 92\%
do mercado consumidor de energia elétrica.  Apenas no segmento
residencial foram aplicados 9.847 questionários.  

% Finalidades das PPHs
As informações na \gls{pph} para o setor residencial vão além da posse
e hábito de consumo sobre eletrodomésticos, estando presentes
informações sobre as características socioeconômicas e comportamentais
do usuário, caracterização e identificação da residência, e detalhes
sobre aquecimento de água sobretudo em relação ao banho. Desse modo,
as \glspl{pph} podem ser utilizadas com outras finalidades, como:

\begin{itemize}
\item obter informações importantes tais como o consumo durante o horário de pico,
fornecendo o fator de carga e fator de demanda;
\item revelar causas de perda de energia elétrica devido a conexões ilegais a
rede;
\item informar a companias de eletrodomésticos sua participação no mercado e 
o perfil de seus consumidores;
\item fazer um paralelo do desenvolvimento e das necessidades das familias 
brasileiras entre classes sociais e regiões do país, explanando as 
diferentes realidades existentes.
\end{itemize}

% Dificuldades nas PPHs
Atualmente, porém, os dados das \glspl{pph} dependem da precisão do
usuário final e do entrevistador durante o preenchimento do
questionário, uma vez que não são utilizados medições no domicilio,
podendo tornar as \glspl{pph} imprecisas. O consumidor pode não se
lembrar ou não saber informar os dados referentes a uma determina
pergunta, assim como omitir certas informações, por não confiar no
profissional desconhecido ou nos intuítos da pesquisa.  Pode ocorrer
também uma má manipulação por parte do entrevistador das informações,
prejudicando a qualidade do material obtido.

% A solução: NIALM
Um outro recurso para se obter informações de posse e hábito é através
da medição por meio de aparelhos específicos. As medições podem ser
utilizadas para complementar, melhorando a precisão das \glspl{pph},
ou até mesmo substituí-las, nesse caso, depositando todo o peso de
identificação de posse e hábito no sistema de medição. A fim de evitar
um maior incômodo ao consumidor, é recomendável que o sistema não faça
intrusão à sua propriedade, constituindo de um sistema de medição
\emph{não-invasivo}. Ainda, a utilização de um método intrusivo de
medição pode alterar, mesmo que inconscientemente, os hábitos de
consumo do usuário, prejudicando a qualidade dos dados.  Novas
técnicas, como o \gls{nilm}, podem ser utilizadas com essa
finalidade (Capítulo~\ref{cap:nilm}), fornecendo as informações
necessárias de posse e hábito do consumidor com maior precisão. 

% Detalhe
Ainda, vale ressaltar que as pesquisas citadas no texto do
\gls{pne2030}, escrito em 2007, como uma excessão à precariedade de
dados já se encontram defasadas atualmente, sendo um empecilho para a 
precisão dos estudos de \gls{ee}, uma vez que os dados utilizados
podem ser não fiéis a realidade brasileira atual.  Esse dispositivo,
além de fornecer melhor precisão, pode agilizar o processo de colheita
no país, fornecendo dados com frequência mais adequada aos estudos de
\gls{ee}. 
