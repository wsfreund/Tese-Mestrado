\chapter{Resultados}%
\label{chap:resultados}

A metodologia (Seção~\ref{sec:aplic_es}) foi aplicada na base de dados
(Seção~\ref{sec:base_de_dados}), ambos descritos no capítulo anterior,
para verificar a capacidade da versão multiespécie implementada do
\acs{es} (Subseção~\ref{sssec:multiespecie}) de realizar o ajuste
automático de parâmetros para os diferentes cenários simulados nos
conjuntos de dados. Como dito durante o detalhamento da metodologia, antes do
ajuste automático dos
parâmetros, também se realizou um ajuste empírico para
os novos cenários presentes no conjunto de dados \emph{Temporizado
Gabarito 1}, que representa um cenário com atuação simultânea de
cinco equipamentos e operação de um equipamento com dinâmica de carga
--- os equipamentos \acs{c5} (ver definições na
Subseção~\ref{ssec:modelos_carga}). Esse ajuste será referido como
\emph{Ajuste Manual}, enquanto o ajuste empírico realizado pelo
\acs{cepel}, que foi feito levando em consideração os conjuntos de
dados \emph{NI00***}, será referido como \emph{Ajuste CEPEL} (mais
detalhes na discussão do método, na Subseção.~\ref{ssec:met_cepel}). Por sua
vez, foram realizados três ajustes automáticos, todos seguindo o método
descrito na Seção~\ref{sec:aplic_es}. Estas são as características e objetivos
de cada um desses ajustes:

\begin{itemize}
\item \emph{ES 1}: a otimização foi realizada alimentando o \acs{es} com
os conjuntos de dados \emph{NI00***}. O objetivo é permitir uma
comparação entre o otimizador e o \emph{Ajuste CEPEL};
\item \emph{ES 2}: neste caso os conjuntos de dados alimentados para o
otimizador foram o \emph{Temporizado Gabarito 1}, \emph{Empilhado4} e
\emph{Empilhado7}. Assim, é possível observar o comportamento dos
métodos quando otimizada para os dados que representam uma
simulação mais real das características presentes em uma rede
residencial;
\item \emph{ES 3}: já a última configuração alimentou o \acs{es}
apenas com o conjunto \emph{Temporizado Gabarito 1} e
\emph{Empilhado4} para analisar a capacidade de generalização do
algoritmo.
\end{itemize}

Não obstante, aplicou-se o Detector de Patamar Elaborado desenvolvido
pelo trabalho de \citet*{nilm_cepel_alvaro} para verificar o
comportamento do mesmo nos novos dados. Porém, não foi realizado um
novo ajuste de seus parâmetros, de forma que os resultados mostrados
para esse detector não representam sua real capacidade.

Um detalhe que aqui deve ser recapitulado é a presença de dois
gabaritos para o conjunto \emph{Temporizado}. Isso se deve à presença
de um distúrbio que tem características similares a uma \acs{c3}
exibido na Figura~\ref{fig:temporizado_disturbio}, que está
presente apenas na informação do gabarito \emph{Temporizado Gabarito
2} (ver Subseção~\ref{ssec:temp} para mais detalhes).

\section{Otimização Automática dos Parâmetros}
\label{sec:otim_es}

Na Tabela~\ref{tab:resultados}, encontram-se os
resultados os ajustes comentados anteriormente no início deste
capítulo, exceto o Detector de Patamar Elaborado que será apresentado
mais adiante.

Fica evidente a maior sensibilidade da escolha dos parâmetros para o
caso \emph{Ajuste CEPEL}, para o qual ocorre uma maior capacidade de detecção,
em especial para o \emph{Empilhado4} que contém eventos de
equipamentos de menor consumo (lâmpadas fluorescentes). Porém, isso
também reflete em uma taxa de falso alarme excessiva para o conjunto
\emph{Temporizado} aonde, mesmo na sua versão com o gabarito contendo o
distúrbio descrito, ocorrem 234 eventos de falso alarme. A alta
sensibilidade dos valores também é refletida no conjunto
\emph{Empilhado7}, que, assim como o \emph{Temporizado}, contém a
presença de duas \acs{c5}. Porém, o arquivo é mais curto e conta com a
televisão operando durante apenas 30 min e ar condicionado por cerca
de 20 min, enquanto a televisão no conjunto de dados
\emph{Temporizado} opera por $\sim12$~h. Já os conjuntos
\emph{NI00***}, para os quais foi realizado o ajuste dessa
configuração, apresentam reconstrução praticamente perfeita, exceto
pela presença de dois falsos alarmes no conjunto \emph{NI00171}. A
Figura~\ref{fig:ni171_cepel} apresenta a ocorrência desses dois falsos
alarmes, que ocorrem justamente após o acionamento do computador
portátil aonde há diversas oscilações até a estabilização do consumo.
As taxa global de detecção e falso alarme para os conjuntos
\emph{NI00***} foram respectivamente de 100~\% e 1,2~\%, enquanto
esses valores de 90,9~\% e 111,7~\% para os conjuntos
\emph{Temporizado Gabarito 1}, \emph{Empilhado4} e \emph{Empilhado7}.
No caso do \emph{Temporizado} ser utilizado com a segunda versão de
seu gabarito, as taxas são respectivamente de 90,2~\% e 82,9~\%.


\begin{table}[ht!]
\resizebox{\textwidth}{!}{
\begin{tabular}{>{\centering}m{3cm}>{\centering}m{1.3cm}cccccccccc}
\hline \hline \hline
\multicolumn{2}{c}{\parbox[t]{4.3cm}{\centering Conjunto de Dados}} &
\multicolumn{2}{c}{\textbf{ES 1}} &
\multicolumn{2}{c}{\textbf{Manual}} &
\multicolumn{2}{c}{\textbf{CEPEL}} &
\multicolumn{2}{c}{\textbf{ES 2}} &
\multicolumn{2}{c}{\textbf{ES 3}}
\tabularnewline \hline
& &
DET & FA &
DET & FA &
DET & FA &
DET & FA &
DET & FA \\
\hline\hline
\multirow{2}{3cm}{\centering\emph{NI00168}
\footnotesize{(39~eventos)}} & \scriptsize{Ocorr.} &
38 & 0 &
33 & 0 &
39 & 0 &
36 & 0 &
23 & 2 \\
 & \scriptsize{Taxa (\%)} &
97,4  & 0,0 &
86,8  & 0,0 &
100,0 & 0,0 &
92,3  & 0,0 &
59,0  & 5,1 \\ \hline
\multirow{2}{3cm}{\centering\emph{NI00171}
\footnotesize{(48~eventos)}} & \scriptsize{Ocorr.} &
47 & 1 &
42 & 0 &
48 & 2 &
43 & 0 &
39 & 2 \\
 & \scriptsize{Taxa (\%)} &
97,9  & 2,1 &
89,4  & 0,0 &
100,0 & 4,2 &
89,6  & 0,0 &
83,0  & 4,2 \\ \hline
\multirow{2}{3cm}{\centering\emph{NI00173}
\footnotesize{(20~eventos)}} & \scriptsize{Ocorr.} &
20 & 1 &
19 & 1 &
20 & 0 &
20 & 1 &
20 & 0 \\
 & \scriptsize{Taxa (\%)} &
100,0 & 5,0 &
95,0  & 5,0 &
100,0 & 0,0 &
100,0 & 5,0 &
100,0 & 0,0 \\ \hline
\multirow{2}{3cm}{\centering\emph{NI00174}
\footnotesize{(8~eventos)}} & \scriptsize{Ocorr.} &
8 & 0 &
7 & 1 &
8 & 0 &
8 & 0 &
7 & 0 \\
 & \scriptsize{Taxa (\%)} &
100,0 & 0,0  &
87,5  & 12,5 &
100,0 & 0,0  &
100,0 & 0,0  &
87,5  & 0,0  \\ \hline
\multirow{2}{3cm}{\centering\emph{NI00175}
\footnotesize{(23~eventos)}} & \scriptsize{Ocorr.} &
23 & 0 &
23 & 0 &
23 & 0 &
23 & 0 &
23 & 0 \\
 & \scriptsize{Taxa (\%)} &
100,0 & 0,0 &
100,0 & 0,0 &
100,0 & 0,0 &
100,0 & 0,0 &
100,0 & 0,0 \\ \hline
\multirow{2}{3cm}{\centering\emph{NI00177}
\footnotesize{(24~eventos)}} & \scriptsize{Ocorr.} &
24 & 0 &
24 & 0 &
24 & 0 &
24 & 0 &
24 & 0 \\
 & \scriptsize{Taxa (\%)} &
100,0 & 0,0 &
100,0 & 0,0 &
100,0 & 0,0 &
100,0 & 0,0 &
100,0 & 0,0 \\ \hline
\multirow{2}{3cm}{\centering\emph{Temp. Gab. 1}
\footnotesize{(149~eventos)}} & \scriptsize{Ocorr.} &
147 & 52  &
147 & 28  &
147 & 259 &
148 & 43  &
148 & 13  \\
 & \scriptsize{Taxa (\%)} &
98,7 & 34,9  &
98,7 & 18,8  &
98,7 & 173,8 &
99,3 & 28,9  &
99,3 & 8,7 \\ \hline
\multirow{2}{3cm}{\centering\emph{Temp. Gab. 2}
\footnotesize{(211~eventos)}} & \scriptsize{Ocorr.} &
191 &  8  &
174 &  1  &
201 & 234 &
188 &  3  &
161 &  1 \\
 & \scriptsize{Taxa (\%)} &
90,5 & 3,8   &
82,5 & 0,0   &
95,3 & 110,9 &
89,1 & 1,4   &
76,3 & 0,5 \\ \hline
\multirow{2}{3cm}{\centering\emph{Empilhado4}
\footnotesize{(74~eventos)}} & \scriptsize{Ocorr.} &
42 & 1 &
24 & 0 &
56 & 10 &
37 & 0 &
34 & 0 \\
 & \scriptsize{Taxa (\%)} &
56,8 & 1,4  &
32,4 & 0,0  &
78,0 & 12,3 &
50,0 & 0,0  &
45,9 & 0,0 \\ \hline
\multirow{2}{3cm}{\centering\emph{Empilhado7}
\footnotesize{(42~eventos)}} & \scriptsize{Ocorr.} &
39 & 1  &
35 & 1  &
38 & 27 &
37 & 0  &
34 & 1 \\
 & \scriptsize{Taxa (\%)} &
92,9 & 2,4  &
83,3 & 2,4  &
90,5 & 64,3 &
88,1 & 0,0  &
80,9 & 2,4 \\
\hline \hline
\end{tabular}}
\caption[Taxa de detecção de eventos de transitório e falso
alarme para os três ajustes automáticos e os dois ajustes manuais.]{
Taxa de detecção de eventos de transitório e falso alarme
para os três ajustes automáticos e os dois ajustes manuais.  As
configurações Manual e CEPEL referem-se respectivamente aos casos
determinados empiricamente pelo autor do trabalho e pelo grupo do
\gls{cepel}, sendo o primeiro ajustado para os dados
\emph{Temporizado Gabarito 1} e
o segundo para os dados \emph{NI00***}. As configurações ES 1, ES 2 e ES
3 referem-se ao ajuste automático do \acs{es} alimentado
respectivamente pelos conjuntos: \emph{NI00***}; \emph{Temporizado
Gabarito 1}, \emph{Empilhado4} e \emph{Empilhado7}; \emph{Temporizado
Gabarito 1} e \emph{Empilhado7}. As abreviaturas DET e FA referem-se a
taxa de detecção e falso alarme, respectivamente.}
\label{tab:resultados}
\end{table}

\begin{SidewaysFigure}
\centering
\includegraphics[width=\textheight]{imagens/ni171_CepelStandard_FA.pdf}
\caption[Falsos alarmes para a configuração \emph{Ajuste CEPEL} no
conjunto de dados \emph{NI00171}.]
{Falsos alarmes para a configuração \emph{Ajuste CEPEL} no
conjunto de dados \emph{NI00171}. Os eventos marcados por etiquetas
roxas indicam as ocorrências dos falsos alarmes durante o acionamento
do computador portátil antes da estabilização de seu consumo. Observe
que há a remoção de diversos outros candidatos devido à eventos
próximos ou ruído que causariam ocorrência ainda maior de falsos
alarmes caso não houvessem suas remoções.}
\label{fig:ni171_cepel}
\end{SidewaysFigure}

A versão de \emph{Ajuste Manual} foi a menos sensível, na qual se
obteve baixa taxa de detecção para o conjunto de dados
\emph{Empilhado4} que contém as lâmpadas de baixo consumo, mas ao
mesmo tempo garantindo um falso alarme total de apenas 29 ocorrências
(28 ocorrências no \emph{Temporizado Gabarito 1} e 1 no
\emph{Empilhado7}). Levando em conta que o seu ajuste foi realizado
levando em consideração apenas o conjunto de dados \emph{Temporizado
Gabarito 1}, isso mostra que a capacidade de visualização permite
encontrar um valor que reduziria a sensibilidade do detector, sem que
fosse analisado vários casos até a obtenção dos parâmetros.  Porém,
para os conjuntos sem a presença de ruídos, essa menor sensibilidade
se reflete em perdas de detecção nos conjuntos \emph{NI00168} e
\emph{NI00171} que possuem equipamentos de menor consumo.

Antes de discutir a questão das taxas de detecção e falso alarme para
os ajustes automáticos, detém-se a convergência de suas espécies em
cada um dos casos. No ES 1, a espécie que obteve maior aptidão foi a
\ref{item:esp3}, enquanto nas configurações ES 2 e 3, foi a
Espécie~\ref{item:esp2}. As
Figuras~\ref{fig:convergencia_es_1}--\ref{fig:convergencia_es_3}
mostram a evolução das espécies em termo de aptidão para os ES 1, 2 e
3, respectivamente. Observa-se que a Espécie~\ref{item:esp3} nas
otimizações ES 2 e 3, não consegue acompanhar o processo evolutivo das
demais espécies, porém isso não ocorre na configuração ES 1, onde
todas as espécies convergem. Há indícios de que a espécie
\ref{item:esp3} não é adequada para os conjuntos de dados ruidosos,
porém é necessário a realização de mais execuções do algoritmo nessas
condições para obter mais estatística e confirmar isso, uma vez que
esses casos podem ter ocorrido devido à convergência de um máximo
local de aptidão. Porém, não foi possível aumentar essa estatística
devido à demora no tempo de execução das configurações ES 2 e 3, que
levam cerca de uma semana para convergir.

Indo além da questão da Espécie~\ref{item:esp3}, observa-se que não há
grande diferença na aptidão das demais espécies na convergência. Isso
indica que quaisquer outras configurações além da
espécie~\ref{item:esp3} é capaz de resolver igualmente bem o problema.
Um motivo para isso ocorrer é a presença de eventos próximos apenas
nos acionamentos com um pico de consumo, que causa a sensibilização e
geração de um candidato na descida de consumo após o pico. Esses
candidatos são devidamente removidos por inconsistência, método de
remoção adicionado justamente para o tratamento desses casos. Ao
observar o material genético de convergência dos ajustes automáticos,
observa-se que o valor de $n_{min}$ foi de 6, 5 e 495, para as
melhores espécies na convergência dos ES 1, 2
e 3, respectivamente. Exceto para o ES 3,
ocorrência \emph{sui generis} que será comentado mais adiante quando
se referindo a sua taxa de detecção e falso alarme, os valores de
convergência para $n_{min}$ são próximos ao limite inferior, o que
mostra que esse parâmetro não é relevante para o problema. Assim, para
os conjuntos de dados analisados, a remoção de eventos inconsistentes
é suficiente para a remoção de eventos de falso alarme que antes eram
removidos por constituírem de eventos próximos, porém, a remoção de
eventos inconsistentes é mais recomendada uma vez que ela não adiciona
tempo morto de resposta do filtro, que pode causar perda de alvo.

%A convergência para um máximo local pode
%ter ocorrido devido à função avaliada não ser continua no espaço, mas
%por diversos patamares discretos que representam valores maiores ou
%menores conforme a incidência maior ou menor de detecção, falso alarme
%e, em menor escala, eventos removidos. Como a função não é contínua,
%não há uma deriva, uma tendência que possibilite uma exploração do
%espaço de soluções gradual. Enquanto os individuos estão em cima de um
%patamar, eles não conseguem perceber qual direção localmente eles
%devem seguir, simplesmente explorando o espaço aleatóriamente em busca
%de um outro patamar na função, patamar esse que gere maiores
%ocorrências de detecção ou menores falso alarme. Quando um indivíduo
%encontrar uma região com um patamar maior, seu material genético irá
%se espalhar em sua espécie, de forma que a espécie irá subir para o
%novo patamar e voltar a explorá-lo aleatoriamente. Porém, se não
%houverem patamares no alcance da estratégia evolutiva dessa espécie,
%ela não irá conseguir deixar esse patamar, uma vez que todos os
%indivíduos que sairem do patamar serão eliminados. Pelo mesmo motivo,
%a estratégia evolutiva dessa espécie tenderá a reduzir seus valores
%de pertubação no material genético, causando a convergência para um
%máximo de aptidão local.
%
%Com o objetivo de tratar esse problema, propõe-se uma
%otimização em dois níveis, com um ciclo interno ajustando
%especificamente o limiar de corte da derivada de Gaussiana. A
%otimização no nível externo irá conter como material genético todas as
%variáveis exceto o limiar de corte, e irá otimizar com a mesma
%metodologia atual os individuos. Porém, o corte não será fixo, após
%determinar a resposta do filtro, um otimizador local irá procurar pela
%configuração ótima de regiões sensibilizadas em relação ao gabarito.
%Para isso, a mesma função pode ser considerada, com os mesmos valores
%de $\gamma_{det}$ e $\gamma_{fa}$, ou a escolha pode ser de um
%parâmetro para cada caso, permitindo encontrar mais regiões
%sensibilizadas na resposta do filtro quando em comparação com a
%resposta final do detector, justamente para explorar os cortes por
%ruído, distância de amostras e incosistência. Ao invés de
%$\gamma_{det}$ e $\gamma_{fa}$, é possível utilizar outras medidas,
%como aquelas descritas na Subseção~\ref{ssec:nilm_eff_calc}. Ao
%realizar a otimização dessa maneira, retira-se do algoritmo genético
%um grande peso para a descoberta de novas regiões, provavelmente
%garantindo um potencial de melhor convergência.

% FIXME Editar a legenda dessas figuras
\begin{figure}[!htb]
\centering
\includegraphics[width=\textwidth]{imagens/convergence_ES.pdf}
\caption[Convergência das espécies para o ajuste automático da
configuração ES 1.]{Convergência das espécies para o ajuste automático da
configuração ES 1. As espécies e suas cores são: Espécie \ref{item:esp1}:
amarelo; Espécie \ref{item:esp2}: azul; Espécie \ref{item:esp3}: rosa;
Espécie \ref{item:esp4}: verde; Espécie \ref{item:esp5}: marrom.
Observa-se que praticamente não há diferença na aptidão das espécies
na convergência.}
\label{fig:convergencia_es_1}
\end{figure}

\begin{figure}[!htb]
\centering
\includegraphics[width=\textwidth]{imagens/convergence_ES2.pdf}
\caption[Convergência das espécies para o ajuste automático da
configuração ES 2.]{Convergência das espécies para o ajuste automático da
configuração ES 2. As espécies e suas cores são: Espécie \ref{item:esp1}:
amarelo; Espécie \ref{item:esp2}: azul; Espécie \ref{item:esp3}: rosa;
Espécie \ref{item:esp4}: verde; Espécie \ref{item:esp5}: marrom. A
Espécie \ref{item:esp3} não obteve convergência como as demais
espécies.}
\label{fig:convergencia_es_2}
\end{figure}

\begin{figure}[!htb]
\centering
\includegraphics[width=\textwidth]{imagens/convergence_ES3.pdf}
\caption[Convergência das espécies para o ajuste automático da
configuração ES 3.]{Convergência das espécies para o ajuste automático da
configuração ES 3. As espécies e suas cores são: Espécie \ref{item:esp1}:
amarelo; Espécie \ref{item:esp2}: azul; Espécie \ref{item:esp3}: rosa;
Espécie \ref{item:esp4}: verde; Espécie \ref{item:esp5}: marrom. A
Espécie \ref{item:esp3} novamente não obteve convergência como as
demais espécies.}
\label{fig:convergencia_es_3}
\end{figure}

Retornando à questão das taxas de detecção e falso alarme, a
configuração ES 1 obteve eficiência compatível ao \emph{Ajuste CEPEL}
nos conjuntos \emph{NI00***}, com umas poucas diferenças. Há a perda
de alvo de um evento nos conjuntos \emph{NI00168} e \emph{NI00171} que
é causada durante o desacionamento de uma lâmpada com \emph{dimmer},
que ocorre em um processo lento e, portanto, que necessita de uma
maior sensibilidade do filtro. A ocorrência de falso alarme no
conjunto de dados é devida a um ruído próximo ao acionamento do
ventilador, causado por uma má manipulação da chave que seleciona a
velocidade do ventilador e indicado na Figura~\ref{fig:ni173_fa}.
Porém, é ao observar as taxas para os conjuntos \emph{Temporizado
Gabarito 1}, \emph{Empilhado4} e \emph{Empilhado7} que se percebe a
real vantagem do ajuste automático dos parâmetros. Apesar do mesmo não
ter tido acesso à esses conjuntos de dados, sua capacidade de
generalização foi bastante superior àquela apresentada pelo
\emph{Ajuste CEPEL}. Isso é causado pela presença do parâmetro
$\gamma_{rem}$ (ver equação~\ref{eq:regra_pontuacao}, para o cálculo
da aptidão na pp.~\pageref{eq:regra_pontuacao}), que causa uma
tendência maior para a exploração dos limites do limiar de
sensibilização. Ao invés da excessiva ocorrência de falso alarme do
\emph{Ajuste CEPEL} com 259, 10 e 27 casos para esses conjuntos,
obteve-se apenas 52, 1 e 1 casos de falso alarme.  Isso, por sua vez,
trouxe uma menor detecção no conjunto \emph{Empilhado4} que contém os
aparelhos de menor consumo. A taxa de detecção e falso alarme do ES 1
para os conjuntos \emph{NI00***} é de 98,8~\% e 1,2~\%,
respectivamente. Para os conjuntos \emph{Temporizado Gabarito 1},
\emph{Empilhado4} e \emph{Empilhado7} essas taxas são de 86,03~\% e
16,51~\%, enquanto esses valores são de 83,2 \% e 3,0 \% quando
empregando o Gabarito 2.

\begin{SidewaysFigure}
\centering
\includegraphics[width=\textheight]{imagens/173_All_fa.pdf}
\caption[Falso alarme para a configuração \emph{ES 1} e \emph{ES 2} no
conjunto de dados \emph{NI00173}.]{Falso alarme para a configuração
\emph{ES 1} e \emph{ES 2} no conjunto de dados \emph{NI00173}. O mesmo
é causado por uma má manipulação da chave que seleciona a
velocidade do ventilador.}
\label{fig:ni173_fa}
\end{SidewaysFigure}

Em seguida, alimentou-se o otimizador para realizar a otimização dos
parâmetros recebendo os conjuntos de dados ruidosos, na esperança de
que isso trouxesse a redução na taxa de falso alarme ainda elevada no
conjunto \emph{Temporizado Gabarito 1}. Na época, ainda não havia sido
observada a pertubação presente nesse conjunto de dados. A resposta
para essa configuração está na coluna ES 2 da
Tabela~\ref{tab:resultados}. Ao analisar sua taxa de eficiência,
percebeu-se que a taxa de falso alarme, apesar de sofrer redução, não
foi na ordem esperada: o valor se manteve alto, 43 ocorrências de
falso alarme nesse conjunto de dados, o que necessitou de uma
investigação mais profunda.  A descoberta foi a pertubação presente na
Figura~\ref{fig:ruido_temporizado}, e observada em mais detalhes na
Figura~\ref{fig:temporizado_disturbio}. Como foi dito durante a
descrição desse conjunto de dados, essa pertubação tem uma
característica muito marcante, contendo cerca de 2-3s e consumo de 15
W. Em vista disso, gerou-se o segundo gabarito para esse conjunto de
dados, que para o ES 2 apresentou apenas 3 falsos alarmes. Os mesmos
são causados pelo acionamento da televisão, que, como pode ser
observado na Figura~\ref{fig:temporizado_televisao}, possuem dois
estágios durante o seu acionamento, o falso alarme sendo causado
justamente durante a primeira crista de consumo. Por outro lado, ao
considerar essas pertubações como alvo, observa-se que nem todas são
detectadas, o que causa uma queda na taxa de detecção razoável. A
Figura~\ref{fig:temporizado_newTargetLoss} indica algumas perdas de
alvo quando considerando esse distúrbio como de detecção desejável.
Assim, esse é um problema ainda a ser resolvido para a aplicação do
\acs{nilm}, uma vez que da maneira que o mesmo está configurado, não é
possível obter a detecção de uma taxa razoável de eventos, nem de
evitá-los. Porém, o caminho parece ser tratar a ocorrência desses
eventos através de sua detecção. A taxa de detecção e falso alarme do
ES 2 para os conjuntos \emph{Temporizado Gabarito 1},
\emph{Empilhado4}, \emph{Empilhado7} é de 83,8 \% e 16,2 \%,
respectivamente, enquanto as mesmas taxas são de 80,12 \% e 0,9 \%
quando considerando o \emph{Temporizado Gabarito 2}. Já suas taxa
nos conjuntos \emph{NI00***} são de 98,1 \% e 0,6 \%.

\begin{SidewaysFigure}
\centering
\includegraphics[width=\textheight]{imagens/temporizadoFA.pdf}
\caption[Exemplos de falsos alarmes no conjunto de dados
\emph{Temporizado Gabarito 1}.]
{Exemplos de falsos alarmes no conjunto de dados \emph{Temporizado
Gabarito 1}. Os falsos alarmes são as caixas verdes e vermelhas
exibidas. Na subfigura inferior, são mostrados alguns pontos de
inflexão e seus valores em unidades da resposta do filtro, bem como
valores da resposta do filtro para ruídos.}
\label{fig:ruido_temporizado}
\end{SidewaysFigure}

\begin{SidewaysFigure}
\centering
\includegraphics[width=\textheight]{imagens/temp_newTargetLoss.pdf}
\caption{Exemplos de perdas de alvos do distúrbio no conjunto de dados
\emph{Temporizado Gabarito 2}.}
\label{fig:temporizado_newTargetLoss}
\end{SidewaysFigure}

Também se deteve atenção às perdas de alvo no conjunto de dados
\emph{Empilhado4}, o conjunto em que há maior dificuldade de
detectar eventos. Na Figura~\ref{fig:lampadas_emp4},
observa-se que os valores absolutos dos pontos de inflexão,
por volta de $0,006$, estão na ordem do ruído do arquivo
\emph{Temporizado} (ver Figura~\ref{fig:ruido_temporizado}), ou seja,
para detectar esses eventos é necessário descer o patamar do filtro a
um valor que causará uma excessiva ocorrência de falso alarme
nesse conjunto de dados --- caso do \emph{Ajuste CEPEL}, cujo patamar
é de $0,003$. A ocorrência de um grande número de falsos alarmes
acarretará numa perda de aptidão, tornando essas soluções
como indivíduos não desejáveis e, portanto, os mesmos serão eliminados
durante a evolução do \acs{es}.

\begin{SidewaysFigure}
\centering
\includegraphics[width=\textheight]{imagens/emp4_targetloss.pdf}
\caption[Exemplos de perdas de alvo no conjunto de dados
\emph{Empilhado4} para a configuração \emph{ES 2}] {Exemplos de perdas
de alvo no conjunto de dados \emph{Empilhado4} para a configuração
\emph{ES 2}. Os eventos não detectados tem seus pontos de inflexão e
seus valores em unidades da resposta do filtro exibidos. As etiquetas
em preto indicam os valores de suas características, aonde é possível
observar que o evento de maior potencia exibido que constitui perda de
alvo apresenta um degrau em valor absoluto de cerca de 24 W.}
\label{fig:lampadas_emp4}
\end{SidewaysFigure}

Outra otimização foi realizada, desta vez alimentando o algoritmo
genético com os conjuntos \emph{Temporizado Gabarito 1} e
\emph{Empilhado4} para verificar a capacidade de sua generalização
para as condições ruidosas. Os resultados estão representados na
coluna ES 3. Ao observar os valores obtidos para os conjuntos para os
quais essa configuração foi alimentada, percebe-se que houve uma
drástica redução do falso alarme para o conjunto \emph{Temporizado
Gabarito 1}. Esse fato se deve ao parâmetro utilizado pelo \acs{es}
para a remoção de eventos próximos, de 495 amostras, como dito
anteriormente. O valor desse parâmetro indica que há uma taxa de tempo
morto de resposta de aceitação de outro candidato de no mínimo 8,3 s.
Como geralmente não há a ocorrência de eventos causados por
acionamentos de equipamentos próximos nesses dois conjuntos, isso não
se mostrou relevante para o algoritmo genético, que percebeu uma
brecha na fronteira desse parâmetro e descobriu uma configuração que
reduz em quase 30 ocorrências a taxa de falso alarme para o
\emph{Temporizado Gabarito 1} no qual a aptidão de seus indivíduos
estava sendo analisada.

Porém, esse alto tempo morto pode causar a
perda de alvo, como observado nos conjuntos \emph{NI00168} e
\emph{NI00171}, aonde o desacionamento é seguido de um acionamento de
outro equipamento em curtos espaços de tempo, como pode ser observado
nas figuras~\ref{fig:ni00168_overview} e \ref{fig:ni00171_overview}, e
está refletido na perda de eficiência nesses conjuntos, especialmente
no \emph{NI00168}. Assim, fica evidente a necessidade de análise
pós-convergência das configurações resultantes do \acs{es}, aonde o
valor obtido de convergência representa o melhor valor obtido pelo
\acs{es} para o conjunto de dado analisado, porém, pode não
representar uma configuração aplicável à realidade, como o caso do ES
3 que explorou uma brecha na fronteira do parâmetro $n_{min}$ que
permitia a utilização de valores muito elevados.

Além disso, também se empregou a utilização do Detector de Patamar
Elaborado desenvolvido por \citet*{nilm_cepel_alvaro}. Os valores dos
parâmetros dessa configuração não foram alterados, de forma que seus
valores estão exatamente como ajustado em seu trabalho. Os conjuntos de
dados utilizados para o ajuste da técnica por esse autor não continham
a atuação de equipamentos simultaneamente e, portanto, os resultados
para os conjuntos \emph{Temporizado}, \emph{Empilhado4} e
\emph{Empilhado7} não apresentam a real capacidade de operação dessa
técnica. Ainda, como foi dito na
Seção~\ref{ssec:cepel_anteriores}, essa técnica é limitada apenas à
eventos de transitório com acréscimo de consumo. Por isso, foi gerada
uma nova tabela (Tabela~\ref{tab:det_elab_res}) contendo os resultados
das demais configurações anteriores apenas para esses eventos e
adicionada uma nova coluna com os resultados para tal detector.

\begin{table}[ht!]
\resizebox{\textwidth}{!}{
\begin{tabular}{>{\centering}m{3cm}>{\centering}m{1.3cm}cccccccccccc}
\hline \hline \hline
\multicolumn{2}{c}{\parbox[t]{4.3cm}{\centering Conjunto de Dados}} &
\multicolumn{2}{c}{\textbf{ES 1}} &
\multicolumn{2}{c}{\textbf{Manual}} &
\multicolumn{2}{c}{\textbf{CEPEL}} &
\multicolumn{2}{c}{\textbf{ES 2}} &
\multicolumn{2}{c}{\textbf{ES 3}} &
\multicolumn{2}{c}{\textbf{Det.Elab.}}
\tabularnewline \hline
\multicolumn{2}{c}{\emph{Eventos acrésc. consum.}}&
DET & FA &
DET & FA &
DET & FA &
DET & FA &
DET & FA &
DET & FA \\
\hline\hline
\multirow{2}{3cm}{\centering\emph{NI00168}
\footnotesize{(20~eventos)}} & \scriptsize{Ocorr.} &
20 & 0 &
19 & 0 &
20 & 0 &
20 & 0 &
11 & 2 &
20 & 1 \\
 & \scriptsize{Taxa (\%)} &
100,0 & 0,0 &
95,0  & 0,0 &
100,0 & 0,0 &
100,0 & 0,0 &
55,0  & 10,0 &
100,0 & 5,0 \\ \hline
\multirow{2}{3cm}{\centering\emph{NI00171}
\footnotesize{(26~eventos)}} & \scriptsize{Ocorr.} &
26 & 1 &
24 & 0 &
26 & 1 &
24 & 0 &
19 & 2 &
25 & 1 \\
 & \scriptsize{Taxa (\%)} &
100,0 & 3,8 &
92,3  & 0,0 &
100,0 & 3,8 &
92,3  & 0,0 &
73,1  & 7,7 &
96,2  & 3,4\\ \hline
\multirow{2}{3cm}{\centering\emph{NI00173}
\footnotesize{(10~eventos)}} & \scriptsize{Ocorr.} &
10 & 1 &
10 & 0 &
10 & 0 &
10 & 1 &
10 & 0 &
10 & 0 \\
 & \scriptsize{Taxa (\%)} &
100,0 & 10,0 &
100,0 & 0,0  &
100,0 & 0,0  &
100,0 & 10,0 &
100,0 & 0,0  &
100,0 & 0,0 \\ \hline
\multirow{2}{3cm}{\centering\emph{NI00174}
\footnotesize{(4~eventos)}} & \scriptsize{Ocorr.} &
4 & 0 &
4 & 0 &
4 & 0 &
4 & 0 &
3 & 0 &
4 & 1 \\
 & \scriptsize{Taxa (\%)} &
100,0 & 0,0 &
100,0 & 0,0 &
100,0 & 0,0 &
100,0 & 0,0 &
75,0  & 0,0 &
100,0 & 25,0 \\ \hline
\multirow{2}{3cm}{\centering\emph{NI00175}
\footnotesize{(12~eventos)}} & \scriptsize{Ocorr.} &
12 & 0 &
12 & 0 &
12 & 0 &
12 & 0 &
12 & 0 &
12 & 1 \\
 & \scriptsize{Taxa (\%)} &
100,0 & 0,0 &
100,0 & 0,0 &
100,0 & 0,0 &
100,0 & 0,0 &
100,0 & 0,0 &
100,0 & 8,3 \\ \hline
\multirow{2}{3cm}{\centering\emph{NI00177}
\footnotesize{(12~eventos)}} & \scriptsize{Ocorr.} &
12 & 0 &
12 & 0 &
12 & 0 &
12 & 0 &
12 & 0 &
12 & 1 \\
 & \scriptsize{Taxa (\%)} &
100,0 & 0,0 &
100,0 & 0,0 &
100,0 & 0,0 &
100,0 & 0,0 &
100,0 & 0,0 &
100,0 & 8,3 \\ \hline
\multirow{2}{3cm}{\centering\emph{Temp. Gab. 1}
\footnotesize{(74~eventos)}} & \scriptsize{Ocorr.} &
72 & 29 &
73 & 12 &
71 & 144 &
73 & 21 &
71 & 7  &
71 & 62 \\
 & \scriptsize{Taxa (\%)} &
97,3 & 39,0 &
98,6 & 16,2 &
95,9 & 194,6 &
98,6 & 28,4 &
95,9 & 9,5 &
95,9 & 83,8 \\ \hline
\multirow{2}{3cm}{\centering\emph{Temp. Gab. 2}
\footnotesize{(105~eventos)}} & \scriptsize{Ocorr.} &
72  & 29  &
85  & 0   &
102 & 116 &
91  & 3   &
77  & 1   &
99 & 34 \\
 & \scriptsize{Taxa (\%)} &
97,3 & 39,2  &
81,0 & 0,0   &
97,1 & 110,5 &
86,7 & 2,9 &
73,3 & 0,9 &
94,3 & 32,4 \\ \hline
\multirow{2}{3cm}{\centering\emph{Empilhado4}
\footnotesize{(40~eventos)}} & \scriptsize{Ocorr.} &
23 & 1 &
18 & 0 &
28 & 8 &
24 & 0 &
22 & 0 &
31 & 1 \\
 & \scriptsize{Taxa (\%)} &
57,5 & 2,5  &
45,0 & 0,0  &
70,0 & 20,0 &
60,0 & 0,0  &
55,0 & 0,0  &
77,5 & 2,5 \\ \hline
\multirow{2}{3cm}{\centering\emph{Empilhado7}
\footnotesize{(24~eventos)}} & \scriptsize{Ocorr.} &
22 & 0  &
22 & 0  &
21 & 16 &
22 & 0  &
18 & 2  &
23 & 8 \\
 & \scriptsize{Taxa (\%)} &
91,7 & 0,0  &
91,7 & 0,0  &
87,5 & 66,7 &
91,7 & 0,0  &
75,0 & 8,3  &
95,8 & 33,3 \\
\hline \hline
\end{tabular}}
\caption[Taxa de detecção de eventos de transitório e falso
alarme para os três ajustes automáticos, os dois ajustes manuais e o
Detector de Patamar Elaborado.]{Taxa de detecção de eventos
de transitório e falso alarme para os três ajustes automáticos, os
dois ajustes manuais e o Detector de Patamar Elaborado. Apenas os
eventos com acréscimo de consumo foram considerados devido à
limitação do Detector de Patamar Elaborado em detectar apenas esse
tipo de eventos. As abreviaturas DET e FA referem-se a
taxa de detecção e falso alarme, respectivamente.}
\label{tab:det_elab_res}
\end{table}

Observa-se que o Detector de Patamar Elaborado, mesmo não ajustado
para os conjuntos de dados analisados, comportou-se razoavelmente bem
nos mesmos. Essa técnica obteve boa taxa de detecção, inclusive
obtendo a taxa mais alta no conjunto \emph{Empilhado4} e não
detectando apenas 6 eventos no conjunto \emph{Temporizado Gabarito 2}.
A taxa de falso alarme no último conjunto citado foi de 34 eventos,
próxima àquela obtida pelo ES 1, indicando o potencial dessa técnica.
Porém, o conjunto \emph{Empilhado7} contém uma maior taxa de falsos
alarmes para essa técnica, que ocorrem durante a operação da televisão
CRT. Além disso, há a presença de alguns falsos alarmes nos conjuntos
\emph{NI00***} que não ocorrem na método com o filtro de Gaussiana.
Esses falsos alarmes foram levantados e identificados pela operação
das chaves do ventilador e secador de cabelo, que são do tipo
``interrompa e então faça'' (do inglês \emph{break then make}), aonde
há a interrupção do consumo do equipamento quando alterando seu estado
de operação. Um exemplo de distúrbio que causa falso alarme de um
acionamento na operação dessa chave pode ser observado na
Figura~\ref{fig:ni00174_overview}, onde o segundo desacionamento nesse
conjunto causa um evento de acionamento pelo Detector de Patamar
Elaborado. Talvez seja possível trabalhar este método de forma a
eliminar esse problema, que aliado com o ajuste de seus parâmetros,
pode se tornar uma técnica bastante adequada para a detecção de
eventos. O ajuste de parâmetros pode ser feito de maneira automática,
utilizando um método similar àquela empregada neste trabalho.

Finalmente, como foi dito no Capítulo~\ref{cap:nilm}, é necessário
indicar a capacidade de reconstrução de energia do \acs{nilm}, porém, o
mesmo não faria sentido nas simulações em laboratório, que não
apresentam a operação dos aparelhos nos tempos que normalmente ocorrem
nas residencias. Indo além, os dados de residência disponíveis (ver
Figura~\ref{fig:casa_real}) não tinham a informação necessária para a
construção do gabarito. Assim, as taxas aqui disponibilizadas não
revelam a real capacidade de reconstrução de energia nas residências
brasileiras, mas servem como um estudo inicial nesse sentido.
Espera-se nos estudos posteriores empregar a análise em dados
representativos, de forma a trazer as taxas já calculadas em termos de
energia e capacidade de desagregação de consumo do \acs{nilm}.


