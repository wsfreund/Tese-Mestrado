\chapter{Resultados}
\label{chap:resultados}

O capítulo em questão contém os resultados para a metodologia aplicada
e sua discussão.

%\section{Descrição da base de dados}
%\label{sec:base_de_dados}
%
%Foram utilizados três conjuntos de dados fornecidos pelo \acs{cepel},
%todos amostrados em situações controladas. Serão descritos as
%características de cada um deles, sendo seus códigos para
%identificação \emph{Temporizado} (Subsessão~\ref{ssec:temp}),
%\emph{Empilhado4} (Subsseão~\ref{ssec:emp4}) e \emph{Empilhado7}
%(Subsessão~\ref{ssec:emp7}).
%
%\subsection{Conjunto de dados \emph{Temporizado}}
%\label{ssec:temp}
%
%\subsection{Conjunto de dados \emph{Empilhado4}}
%\label{ssec:emp4}
%
%\subsection{Conjunto de dados \emph{Empilhado7}}
%\label{ssec:emp7}


\section[Aplicação do ES para otimização do detector de eventos]{
Aplicação do \acf{es} para otimização do detector de eventos}
\label{sec:aplic_es}




\begin{table}[p]
\resizebox{\textwidth}{!}{
\begin{tabular}{>{\centering}m{3cm}>{\centering}m{2cm}cccccccc}
\hline \hline \hline
\multicolumn{2}{c}{\parbox[t]{5cm}{\centering Conjunto de Dados}} &
\multicolumn{2}{c}{\textbf{\acs{es} $\mathbf{1/0,9}$}} & 
\multicolumn{2}{c}{\textbf{\acs{es} $\mathbf{1/2}$}} & 
\multicolumn{2}{c}{\textbf{Manual}} & 
\multicolumn{2}{c}{\textbf{Anterior}} \tabularnewline \hline
& & 
DET & FA & 
DET & FA & 
DET & FA &
DET & FA \\
\hline\hline
\multirow{2}{3cm}{\centering\emph{Temporizado}
\footnotesize{(149~eventos)}} & {Ocorr.} & 
148 & 39 & 148 & 42 & 147 & 28 & 147 & 259\\
 & {Taxa (\%)} & 
99,3 & 26,2 & 99,3 & 28,19 & 98,7 &18,8 & 98,7 & 173,8  \\
\hline
\multirow{2}{3cm}{\centering\emph{Empilhado4}
\footnotesize{(74~eventos)}} & {Ocorr.} & 
37 & 0 & 37 & 0 & 24 & 0 & 57 & 9 \\
 & {Taxa (\%)} & 
64,9 & 0,0 & 64,9 & 0,0 & 32,4 & 0,0 & 78,0 & 12,3  \\
\hline
\multirow{2}{3cm}{\centering\emph{Empilhado7}
\footnotesize{(42~eventos)}} & {Ocorr.} & 
36 & 1 & 36 & 2 & 35 & 1 & 37 & 18 \\
 & {Taxa (\%)} & 
85,7 & 2,4 & 85,7 & 4,8 & 83,3 & 2,4 & 88,1 & 42,9  \\
\hline \hline
\end{tabular}}
\caption[Resultado para os três conjuntos de dados onde o
\acs{es} foi ajustado alimentado por todos eles.]{
Resultado para os três conjuntos de dados onde o
\acs{es} foi ajustado alimentado por todos eles. As configurações
\emph{Manual} e \emph{Anterior} se referem respectivamente aos casos
determinados empiricamente pelo autor do trabalho e pelo grupo do
\gls{cepel}, o último sendo determinado em dados sem ruídos.}
\label{tab:resultados_sem_generalizacao}
\end{table}
