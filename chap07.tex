\chapter{Resultados}
\label{chap:resultados}

O capítulo em questão contém os resultados e discussão da proposta de
otimização automática dos parâmetros do filtro de núcleo de Gaussiana
%e da aplicação de \gls{som}, ambos com o intuíto de realizar a
para a
detecção de eventos. %No primeiro caso, serão utilizados os três
Serão utilizados os três
conjuntos de dados descritos Seção~\ref{sec:base_de_dados} que serão
otimizados pelo algoritmo genético de estratégia evolutiva exposto na
Seção~\ref{ssec:es}. A metodologia empregada está descrita na
Seção~\ref{sec:aplic_es}. 

\section{Otimização Automática dos Parâmetros}
\label{sec:otim_es}

Determinou-se, antes de realizar a análise por otimização, os
parâmetros do detector de eventos através da capacidade de
visualização que o ambiente de análise permite. Os resultados para
essa análise serão referidos como \emph{Ajuste Manual}, determinados
pelo autor deste trabalho. Já os resultados para o ajuste realizado
pelo \acs{cepel}, realizado para a uma base de dados sem ruído, será
referida por CEPEL.

Executou-se o algoritmo de duas maneiras, a configuração \acs{es} $1/0,9$
onde foi feita a escolha de $\gamma_{fa}=-0,9$ e a configuração
\acs{es} $1/2$, onde empregou-se $\gamma_{fa}=-2$. Em ambos casos, os
parâmetros de detecção e de eventos removidos foram utilizados com os
respectivos valores: $\gamma_{det}=1$ e $\gamma_{rem}=-0,05$. Esses
parâmetros são utilizados pela equação~\ref{eq:regra_pontuacao}. A
ideia inicial para ter escolhido um valor não intuitivo de -0,9 na
primeira configuração foi a tentativa de gerar uma ligeira vantagem para
a detecção nesse indíviduo. Já o valor de $-2$ foi utilizado
posteriormente com o intuíto de reduzir o falso alarme. 

\begin{table}[ht!]
\resizebox{\textwidth}{!}{
\begin{tabular}{>{\centering}m{3cm}>{\centering}m{1.3cm}cccccccccc}
\hline \hline \hline
\multicolumn{2}{c}{\parbox[t]{4.3cm}{\centering Conjunto de Dados}} &
\multicolumn{2}{c}{\textbf{\acs{es}}} & 
\multicolumn{2}{c}{\textbf{Manual}} & 
\multicolumn{2}{c}{\textbf{CEPEL}} &
\multicolumn{2}{c}{\textbf{ES 2}} &
\multicolumn{2}{c}{\textbf{ES 3}} 
\tabularnewline \hline
& & 
DET & FA & 
DET & FA & 
DET & FA & 
DET & FA & 
DET & FA \\
\hline\hline
\multirow{2}{3cm}{\centering\emph{NI00168}
\footnotesize{(39~eventos)}} & \scriptsize{Ocorr.} & 
38 & 0 & 
33 & 0 & 
39 & 0 &
36 & 0 &
39 & 0 \\
 & \scriptsize{Taxa (\%)} & 
97,4  & 0,0 &  
86,8  & 0,0 & 
100,0 & 0,0 &
92,3  & 0,0 &
100,0 & 0,0 \\ \hline
\multirow{2}{3cm}{\centering\emph{NI00171}
\footnotesize{(48~eventos)}} & \scriptsize{Ocorr.} & 
47 & 1 & 
42 & 0 & 
48 & 2 &
43 & 0 &
39 & 0 \\
 & \scriptsize{Taxa (\%)} & 
97,9  & 2,1 &  
89,4  & 0,0 &
100,0 & 4,2 &
89,6  & 0,0 &
83,0  & 0,0 \\ \hline
\multirow{2}{3cm}{\centering\emph{NI00173}
\footnotesize{(20~eventos)}} & \scriptsize{Ocorr.} & 
20 & 1 & 
19 & 1 & 
20 & 0 &
20 & 1 &
20 & 0 \\
 & \scriptsize{Taxa (\%)} & 
100,0 & 5,0 & 
95,0  & 5,0 &
100,0 & 0,0 &
100,0 & 5,0 &
100,0 & 0,0 \\ \hline
\multirow{2}{3cm}{\centering\emph{NI00174}
\footnotesize{(8~eventos)}} & \scriptsize{Ocorr.} & 
8 & 0 & 
7 & 1 &
8 & 0 &
8 & 0 &
7 & 0 \\
 & \scriptsize{Taxa (\%)} & 
100,0 & 0,0  & 
87,5  & 12,5 &
100,0 & 0,0  &
100,0 & 0,0  &
87,5  & 0,0  \\ \hline
\multirow{2}{3cm}{\centering\emph{NI00175}
\footnotesize{(23~eventos)}} & \scriptsize{Ocorr.} & 
23 & 0 & 
23 & 0 &
23 & 0 &
23 & 0 &
23 & 0 \\
 & \scriptsize{Taxa (\%)} & 
100,0 & 0,0 &
100,0 & 0,0 & 
100,0 & 0,0 &
100,0 & 0,0 &
100,0 & 0,0 \\ \hline
\multirow{2}{3cm}{\centering\emph{NI00177}
\footnotesize{(24~eventos)}} & \scriptsize{Ocorr.} & 
24 & 0 & 
24 & 0 &
24 & 0 &
24 & 0 &
24 & 0 \\
 & \scriptsize{Taxa (\%)} & 
100,0 & 0,0 & 
100,0 & 0,0 & 
100,0 & 0,0 &
100,0 & 0,0 &
100,0 & 0,0 \\ \hline
\multirow{2}{3cm}{\centering\emph{Temp. Gab. 1}
\footnotesize{(149~eventos)}} & \scriptsize{Ocorr.} & 
147 & 52  & 
147 & 28  & 
147 & 259 &
148 & 43  &
148 & 13  \\
 & \scriptsize{Taxa (\%)} & 
98,7 & 34,9  & 
98,7 & 18,8  & 
98,7 & 173,8 &
99,3 & 28,9  &
99,3 & 8,7 \\ \hline
\multirow{2}{3cm}{\centering\emph{Temp. Gab. 2}
\footnotesize{(211~eventos)}} & \scriptsize{Ocorr.} & 
191 &  8  & 
174 &  1  & 
201 & 234 &
188 &  3  &
161 &  1 \\
 & \scriptsize{Taxa (\%)} & 
90,5 & 3,8   & 
82,5 & 0,0   & 
95,3 & 110,9 &
89,1 & 1,4   &
76,3 & 0,5 \\ \hline
\multirow{2}{3cm}{\centering\emph{Empilhado4}
\footnotesize{(74~eventos)}} & \scriptsize{Ocorr.} & 
42 & 1 & 
24 & 0 & 
56 & 10 &
37 & 0 &
34 & 0 \\
 & \scriptsize{Taxa (\%)} & 
56,8 & 1,4  & 
32,4 & 0,0  & 
78,0 & 12,3 &
50,0 & 0,0  &
45,9 & 0,0 \\ \hline
\multirow{2}{3cm}{\centering\emph{Empilhado7}
\footnotesize{(42~eventos)}} & \scriptsize{Ocorr.} & 
39 & 1  & 
35 & 1  & 
38 & 27 &
37 & 0  &
34 & 1 \\
 & \scriptsize{Taxa (\%)} & 
92,9 & 2,4  & 
83,3 & 2,4  & 
90,5 & 64,3 & 
88,1 & 0,0  &
80,9 & 2,4 \\
\hline \hline
\end{tabular}}
\caption[Probabilidade de detecção de eventos de transitório para os
três conjuntos de dados, onde o \acs{es} foi ajustado alimentado por
todos eles.]{Probabilidade de detecção de eventos de transitório para
os três conjuntos de dados onde o \acs{es} foi ajustado alimentado por
todos eles. As configurações \emph{Ajuste Manual} e CEPEL 
se referem respectivamente aos casos determinados
empiricamente pelo autor do trabalho e pelo grupo do \gls{cepel}, o
último sendo determinado em dados sem ruídos.}
\label{tab:resultados}
\end{table}

Na Tabela~\ref{tab:resultados}, encontram-se os
resultados para essas configurações. Fica evidente a maior
sensibilidade da escolha dos patamares para o caso CEPEL, 
caso para o qual ocorre uma maior capacidade de detecção,
em especial para o \emph{Empilhado4} que contém eventos de
equipamentos de menor consumo (lâmpadas fluorescentes). Porém, isso
também reflete em uma taxa de falso alarme excessiva para o conjunto
\emph{Temporizado}, que contém uma \acs{c5}. Uma
quantidade de falsos positivos dessa ordem não justifica a
identificação de lâmpadas de menor porte contidos no conjunto
\emph{Empilhado4}. A alta sensibilidade dos valores também é refletida
no conjunto \emph{Empilhado7}, que, assim como o \emph{Temporizado},
contém a presença de duas \acs{c5}. Porém, o arquivo é mais curto e
conta com a televisão operando durante apenas 30 min e ar condicionado
por cerca de 20 min, enquanto a televisão no conjunto de dados 
\emph{Temporizado} opera por $\sim12$~h.

A versão de \emph{Ajuste Manual} foi a menos sensível, obtendo menor
detecção para o conjunto de dados \emph{Empilhado4}, mas ao mesmo
tempo garantindo um falso alarme total de apenas 29 ocorrências (28
ocorrências no \emph{Temporizado} e 1 no \emph{Empilhado7}).
Isso mostra que a capacidade de visualização permite encontrar um
valor que reduziria a sensibilidade do detector, sem que fosse
analisado vários casos até sua obtenção.

\begin{figure}[!htb]
\centering
\includegraphics[width=\textwidth]{imagens/convergencia_es_1-9.pdf}
\caption[Convergência para o ajuste automático através da regra
$1/0,9$.] {Convergência para o ajuste automático através da regra
$1/0,9$. As espécies e suas cores são: Espécie \ref{item:esp1}:
amarelo, Espécie \ref{item:esp2}: azul, Espécie \ref{item:esp3}: rosa,
Espécie \ref{item:esp4}: verde, Espécie \ref{item:esp5}: marrom.}
\label{fig:convergencia_es_1}
\end{figure}

\begin{figure}[!htb]
\centering
\includegraphics[width=\textwidth]{imagens/convergencia_es_1-2.pdf}
\caption[Convergência para o ajuste automático através da regra
$1/2$.] {Convergência para o ajuste automático através da regra
$1/2$. As espécies e suas cores são: Espécie \ref{item:esp1}:
amarelo, Espécie \ref{item:esp2}: azul, Espécie \ref{item:esp3}: rosa,
Espécie \ref{item:esp4}: verde, Espécie \ref{item:esp5}: marrom.}
\label{fig:convergencia_es_2}
\end{figure}

No \acs{es} $1/0,9$, a espécie que obteve maior aptidão foi a
\ref{item:esp1}, enquanto na configuração \acs{es} $1/2$, foi a
Espécie~\ref{item:esp2}. As figuras~\ref{fig:convergencia_es_1} e
\ref{fig:convergencia_es_2} mostram a evolução das espécies para o
\acs{es} $1/0,9$ e \acs{es} $1/2$. O primeiro fator que se ressalta em
ambas as figuras é a evolução mais rápida da espécie com material
genético mais simples --- a Espécie~\ref{item:esp5}  ---, mostrando
que o grau de liberdade a menos (essa espécie tem um gene a menos: o
tamanho da janela) contribui ao facilitar o ajuste da estratégia
evolutiva. Porém, isso não significa que a espécie é mais apta para a
solução do problema, apenas que ela irá evoluir com maior facilidade
que as demais espécies. Observa-se que a Espécie~\ref{item:esp3} da
configuração \acs{es} $1/0,9$, assim como as espécies \ref{item:esp2}
e \ref{item:esp3} da configuração \acs{es} $1/2$, não conseguem
acompanhar o processo evolutivo das demais espécies. Como a espécie
\ref{item:esp2} converge para a outra configuração testada, e, no
caso, obtém a melhor aptidão para a configuração, há um indicativo de
que, na verdade, essas espécies, que não acompanharam a evolução das
demais espécies, acabaram convergindo para máximos locais de aptidão.

Quanto à questão de não convergência de algumas espécies, a mesma pode
ter ocorrido devido à função avaliada não ser continua no espaço, mas
por diversos patamares discretos que representam valores maiores ou
menores conforme a incidência maior ou menor de detecção, falso alarme
e, em menor escala, eventos removidos. Como a função não é contínua,
não há uma deriva, uma tendência que possibilite uma exploração do
espaço de soluções gradual. Enquanto os individuos estão em cima de um
patamar, eles não conseguem perceber qual direção localmente eles
devem seguir, simplesmente explorando o espaço aleatóriamente em busca
de um outro patamar na função, patamar esse que gere maiores
ocorrências de detecção ou menores falso alarme. Quando um indivíduo
encontrar uma região com um patamar maior, seu material genético irá
se espalhar em sua espécie, de forma que a espécie irá subir para o
novo patamar e voltar a explorá-lo aleatoriamente. Porém, se não
houverem patamares no alcance da estratégia evolutiva dessa espécie,
ela não irá conseguir deixar esse patamar, uma vez que todos os
indivíduos que sairem do patamar serão eliminados. Pelo mesmo motivo,
a estratégia evolutiva dessa espécie tenderá a reduzir seus valores
de pertubação no material genético, causando a convergência para um
máximo de aptidão local. 

Com o objetivo de tratar esse problema, propõe-se uma
otimização em dois níveis, com um ciclo interno ajustando
especificamente o limiar de corte da derivada de Gaussiana. A
otimização no nível externo irá conter como material genético todas as
variáveis exceto o limiar de corte, e irá otimizar com a mesma
metodologia atual os individuos. Porém, o corte não será fixo, após
determinar a resposta do filtro, um otimizador local irá procurar pela
configuração ótima de regiões sensibilizadas em relação ao gabarito.
Para isso, a mesma função pode ser considerada, com os mesmos valores
de $\gamma_{det}$ e $\gamma_{fa}$, ou a escolha pode ser de um
parâmetro para cada caso, permitindo encontrar mais regiões
sensibilizadas na resposta do filtro quando em comparação com a
resposta final do detector, justamente para explorar os cortes por
ruído, distância de amostras e incosistência. Ao invés de
$\gamma_{det}$ e $\gamma_{fa}$, é possível utilizar outras medidas,
como aquelas descritas na Subseção~\ref{ssec:nilm_eff_calc}. Ao
realizar a otimização dessa maneira, retira-se do algoritmo genético
um grande peso para a descoberta de novas regiões, provavelmente
garantindo um potencial de melhor convergência.

Além disso, ao analisar a taxa de falso alarme do arquivo
\emph{Temporizado}, percebeu-se que os mesmos decorrem
de uma pertubação na rede que se assemelha muito ao
transitório de um equipamento, Figura~\ref{fig:ruido_temporizado}.
Esssa pertubação tem uma característica muito marcante, contendo cerca
de 2-3s e consumo de 15 W. Não foi possível determinar qual
equipamento que causa esse distúrbio, mas nitidamente esses degraus
apresentam características semelhantes àquelas causadas por \glspl{c2}
e \glspl{c3}. Agora fica a questão de como o gabarito deveria ser
construído, se considerando essas alterações como eventos a serem
detectados, ou se elas continuam sendo ruído. Já para a
Figure~\ref{fig:lampadas_emp4}, 
observa-se que os valores absolutos dos pontos de inflexão,
por volta de $0,006$, já estão na ordem do ruído do arquivo
\emph{Temporizado} (ver Figura~\ref{fig:ruido_temporizado}), ou seja,
para detectar esses eventos é necessário descer o patamar do filtro a
um valor que irá causar uma excessiva ocorrência de falso alarme
nesse conjunto de dados --- caso do filtro anteriormente utilizado pelo
\acs{cepel}, cujo patamar é de $0,003$. A ocorrência de um grande
número de falsos alarmes irá acarretar numa perda de aptidão, tornando
esses individuos soluções não desejáveis e, portanto, os mesmos serão
eliminados.

Grande parte dos falsos alarmes ocorridos para o \emph{Temporizado} no
caso dos \glspl{es} são devidos a essa mudança de operação
desconhecida. Contabilizou-se a ocorrência desses casos no conjunto de
dados \emph{Temporizado}, aonde 36 ocorrências desse fenômeno foram
aceitas pelo filtro de núcleo de Gaussiana. Caso esses eventos fossem
contabilizados como alvos, somente 3 falsos alarmes seriam
contabilizados. Assim, ao contabilizar esses eventos como parte do
gabarito, a taxa de falso alarme reduziria para 2,0 \%, mas também
seria necessário realizar novamente o cálculo para a taxa de deteção,
uma vez que agora esses eventos são alvos. Percebe-se, também, que uma
nova otimização, considerando esses eventos como alvos poderia
melhorar as taxas para essa nova representação, dado que essas
possíveis representações de eventos de transitório estavam sendo
evitados pelo otimizador.

\begin{sidewaysfigure}[p]
\centering
\includegraphics[width=\textwidth]{imagens/temporizadoFA.pdf}
\caption[Exemplos de falsos alarmes no conjunto de dados \emph{Temporizado}.]
{Exemplos de falsos alarmes no conjunto de dados \emph{Temporizado}. Os
falsos alarmes são as caixas verdes e vermelhas exibidas. Na subfigura
inferior, são mostrados alguns pontos de inflexão e seus valores em
unidades da resposta do filtro, bem como valores da resposta do filtro
para ruídos.}
\label{fig:ruido_temporizado}
\end{sidewaysfigure}

\begin{sidewaysfigure}[p]
\centering
\includegraphics[width=\textwidth]{imagens/emp4_evtsPerdidos.pdf}
\caption[Exemplos de perdas de alvo no conjunto de dados Empilhado4]
{Exemplos de perdas de alvo no conjunto de dados \emph{Empilhado4}. Os
eventos não detectados tem seus pontos de inflexão e seus valores em
unidades da resposta do filtro exibidos.}
\label{fig:lampadas_emp4}
\end{sidewaysfigure}

Observa-se, ainda, que a configuração com maior penalidade para falso
alarme obteve uma falso alarme a menos para o \emph{Temporizado}, mas
também perdeu um evento de detecção do arquivo \emph{Empilhado7}.
Essa diferença é muito pequena para inferir em apenas uma otimização
se haveria realmente uma diferença entre as configurações, no entanto,
ela mostra uma pequena tendência de reduzir o falso alarme, aceitando
deterioramento na detecção. Atribui-se esse fato
justamente por estar explorando o limite da fronteira de ruído, sendo
necessário um valor maior a -2 pontos em $\gamma_{fa}$ para os
individuos que exploram a região de ruído mencionada anteriormente
sejam interessantes. 


\begin{table}[ht!]
\resizebox{\textwidth}{!}{
\begin{tabular}{>{\centering}m{3cm}>{\centering}m{1.3cm}cccccccccccc}
\hline \hline \hline
\multicolumn{2}{c}{\parbox[t]{4.3cm}{\centering Conjunto de Dados}} &
\multicolumn{2}{c}{\textbf{\acs{es}}} & 
\multicolumn{2}{c}{\textbf{Manual}} & 
\multicolumn{2}{c}{\textbf{Anterior}} &
\multicolumn{2}{c}{\textbf{ES 2}} &
\multicolumn{2}{c}{\textbf{ES 3}} &
\multicolumn{2}{c}{\textbf{Det.Elab.}} 
\tabularnewline \hline
\multicolumn{2}{c}{\emph{Eventos acrésc. consum.}}& 
DET & FA & 
DET & FA & 
DET & FA & 
DET & FA & 
DET & FA & 
DET & FA \\
\hline\hline
\multirow{2}{3cm}{\centering\emph{NI00168}
\footnotesize{(20~eventos)}} & \scriptsize{Ocorr.} & 
20 & 0 & 
19 & 0 & 
20 & 0 &
20 & 0 &
11 & 2 &
20 & 1 \\
 & \scriptsize{Taxa (\%)} & 
100,0 & 0,0 &  
95,0  & 0,0 & 
100,0 & 0,0 &
100,0 & 0,0 &
55,0  & 10,0 &
100,0 & 5,0 \\ \hline
\multirow{2}{3cm}{\centering\emph{NI00171}
\footnotesize{(26~eventos)}} & \scriptsize{Ocorr.} & 
26 & 1 & 
24 & 0 & 
26 & 1 &
24 & 0 &
19 & 2 &
25 & 1 \\
 & \scriptsize{Taxa (\%)} & 
100,0 & 3,8 &  
92,3  & 0,0 &
100,0 & 3,8 &
92,3  & 0,0 &
73,1  & 7,7 &
96,2  & 3,4\\ \hline
\multirow{2}{3cm}{\centering\emph{NI00173}
\footnotesize{(10~eventos)}} & \scriptsize{Ocorr.} & 
10 & 1 & 
10 & 0 & 
10 & 0 &
10 & 1 &
10 & 0 &
10 & 0 \\
 & \scriptsize{Taxa (\%)} & 
100,0 & 10,0 & 
100,0 & 0,0  &
100,0 & 0,0  &
100,0 & 10,0 &
100,0 & 0,0  &
100,0 & 0,0 \\ \hline
\multirow{2}{3cm}{\centering\emph{NI00174}
\footnotesize{(4~eventos)}} & \scriptsize{Ocorr.} & 
4 & 0 & 
4 & 0 &
4 & 0 &
4 & 0 &
3 & 0 &
4 & 1 \\
 & \scriptsize{Taxa (\%)} & 
100,0 & 0,0 & 
100,0 & 0,0 &
100,0 & 0,0 &
100,0 & 0,0 &
75,0  & 0,0 &
100,0 & 25,0 \\ \hline
\multirow{2}{3cm}{\centering\emph{NI00175}
\footnotesize{(12~eventos)}} & \scriptsize{Ocorr.} & 
12 & 0 & 
12 & 0 &
12 & 0 &
12 & 0 &
12 & 0 &
12 & 1 \\
 & \scriptsize{Taxa (\%)} & 
100,0 & 0,0 &
100,0 & 0,0 & 
100,0 & 0,0 &
100,0 & 0,0 &
100,0 & 0,0 &
100,0 & 8,3 \\ \hline
\multirow{2}{3cm}{\centering\emph{NI00177}
\footnotesize{(12~eventos)}} & \scriptsize{Ocorr.} & 
12 & 0 & 
12 & 0 &
12 & 0 &
12 & 0 &
12 & 0 &
12 & 1 \\
 & \scriptsize{Taxa (\%)} & 
100,0 & 0,0 & 
100,0 & 0,0 & 
100,0 & 0,0 &
100,0 & 0,0 &
100,0 & 0,0 &
100,0 & 8,3 \\ \hline
\multirow{2}{3cm}{\centering\emph{Temp. Gab. 1}
\footnotesize{(74~eventos)}} & \scriptsize{Ocorr.} & 
72 & 29 & 
73 & 12 & 
71 & 144 &
73 & 21 &
71 & 7  &
71 & 62 \\
 & \scriptsize{Taxa (\%)} & 
97,3 & 39,0 & 
98,6 & 16,2 & 
95,9 & 194,6 &
98,6 & 28,4 &
95,9 & 9,5 &
95,9 & 83,8 \\ \hline
\multirow{2}{3cm}{\centering\emph{Temp. Gab. 2}
\footnotesize{(105~eventos)}} & \scriptsize{Ocorr.} & 
72  & 29  & 
85  & 0   & 
102 & 116 &
91  & 3   &
77  & 1   &
99 & 34 \\
 & \scriptsize{Taxa (\%)} & 
97,3 & 39,2  & 
81,0 & 0,0   & 
97,1 & 110,5 &
86,7 & 2,9 &
73,3 & 0,9 &
94,3 & 32,4 \\ \hline
\multirow{2}{3cm}{\centering\emph{Empilhado4}
\footnotesize{(40~eventos)}} & \scriptsize{Ocorr.} & 
23 & 1 & 
18 & 0 & 
28 & 8 &
24 & 0 &
22 & 0 &
31 & 1 \\
 & \scriptsize{Taxa (\%)} & 
57,5 & 2,5  & 
45,0 & 0,0  & 
70,0 & 20,0 &
60,0 & 0,0  &
55,0 & 0,0  &
77,5 & 2,5 \\ \hline
\multirow{2}{3cm}{\centering\emph{Empilhado7}
\footnotesize{(24~eventos)}} & \scriptsize{Ocorr.} & 
22 & 0  & 
22 & 0  & 
21 & 16 &
22 & 0  &
18 & 2  &
23 & 8 \\
 & \scriptsize{Taxa (\%)} & 
91,7 & 0,0  & 
91,7 & 0,0  & 
87,5 & 66,7 & 
91,7 & 0,0  &
75,0 & 8,3  &
95,8 & 33,3 \\
\hline \hline
\end{tabular}}
\end{table}

