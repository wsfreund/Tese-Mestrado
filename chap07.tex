\chapter{Resultados}
\label{chap:resultados}

O capítulo em questão contém os resultados para a metodologia aplicada
e sua discussão. São utilizados os três conjuntos de dados descritos em
~\ref{sec:base_de_dados}. Na Sessão~\ref{sec:otim_es} trata-se a otimização
automática dos parâmetros através do algoritmo genético de estratégia
evolutiva exposto na Sessão~\ref{ssec:es} e com a metodologia
determinada na Sessão~\ref{sec:aplic_es}. Em seguida, será tratado o
estudo da representação do espaço das características extraídas para o
mesmo através de \acs{som}. Ele será aplicado como um detector de
eventos ou aplicado em série para atuar após o filtro de derivada de
Gaussiana na Subsessão~\ref{sec:som_e_es}.

\section{Otimização Automática dos Parâmetros}
\label{sec:otim_es}

Antes de realizar a análise por otimização, foi feita a determinação
dos parâmetros do detector de eventos através da capacidade de
visualização que o ambiente de análise permitia. Os resultados para
essa análise serão referidos como \emph{Ajuste Manual}, determinados
pelo autor deste trabalho. Já os resultados para o ajuste manual
realizado pelo \acs{cepel} antes da para a uma base de dados sem ruído
será referida através de \emph{Ajuste Anterior}.

Assim, executou-se o algoritmo de duas maneiras, uma escolhendo
$\gamma_{fa}=-0,9$ --- referido como \emph{\acs{es} $1/0,9$} --- e 
na segunda para $\gamma_{fa}=-2$ --- referido como \emph{\acs{es}
$1/2$} (ver \ref{eq:regra_pontuacao}). Nos dois casos $\gamma_{det}=1$
e $\gamma_{rem}=-0,05$. A ideia inicial para ter escolhido um valor
não intuitivo de -0,9 foi a tentativa de dar uma ligeira vantagem para a
detecção. Já o valor de $-2$ foi utilizado posteriormente com o
intuíto de reduzir o falso alarme. 

\begin{table}[ht!]
\resizebox{\textwidth}{!}{
\begin{tabular}{>{\centering}m{3cm}>{\centering}m{2cm}cccccccc}
\hline \hline \hline
\multicolumn{2}{c}{\parbox[t]{5cm}{\centering Conjunto de Dados}} &
\multicolumn{2}{c}{\textbf{\acs{es} $\mathbf{1/0,9}$}} & 
\multicolumn{2}{c}{\textbf{\acs{es} $\mathbf{1/2}$}} & 
\multicolumn{2}{c}{\textbf{Ajuste Manual}} & 
\multicolumn{2}{c}{\textbf{A. Anterior}} \tabularnewline \hline
& & 
DET & FA & 
DET & FA & 
DET & FA &
DET & FA \\
\hline\hline
\multirow{2}{3cm}{\centering\emph{Temporizado}
\footnotesize{(149~eventos)}} & {Ocorr.} & 
148 & 39 & 148 & 42 & 147 & 28 & 147 & 259\\
 & {Taxa (\%)} & 
99,3 & 26,2 & 99,3 & 28,19 & 98,7 &18,8 & 98,7 & 173,8  \\
\hline
\multirow{2}{3cm}{\centering\emph{Empilhado4}
\footnotesize{(74~eventos)}} & {Ocorr.} & 
37 & 0 & 37 & 0 & 24 & 0 & 57 & 9 \\
 & {Taxa (\%)} & 
64,9 & 0,0 & 64,9 & 0,0 & 32,4 & 0,0 & 78,0 & 12,3  \\
\hline
\multirow{2}{3cm}{\centering\emph{Empilhado7}
\footnotesize{(42~eventos)}} & {Ocorr.} & 
36 & 1 & 36 & 2 & 35 & 1 & 37 & 18 \\
 & {Taxa (\%)} & 
85,7 & 2,4 & 85,7 & 4,8 & 83,3 & 2,4 & 88,1 & 42,9  \\
\hline \hline
\end{tabular}}
\caption[Resultado para os três conjuntos de dados onde o
\acs{es} foi ajustado alimentado por todos eles.]{
Resultado para os três conjuntos de dados onde o
\acs{es} foi ajustado alimentado por todos eles. As configurações
\emph{Ajuste Manual} e \emph{Ajuste Anterior} se referem respectivamente aos casos
determinados empiricamente pelo autor do trabalho e pelo grupo do
\gls{cepel}, o último sendo determinado em dados sem ruídos.}
\label{tab:resultados_sem_generalizacao}
\end{table}

% 83,4 

Na Tabela~\ref{tab:resultados_sem_generalizacao} encontram-se os
resultados para essas configurações. Fica evidente a maior
sensibilidade da escolha dos patamares para o caso anterior, onde há
uma maior capacidade de detecção, em especial para o \emph{Empilhado4}
que contém eventos de equipamentos de menor consumo (lâmpadas
fluorescentes) que são ignorados pelas configurações dos \acs{es}.
Porém, isso também reflete em uma taxa de falso alarme excessiva para
o conjunto \emph{Temporizado}, aonde há a presença de maior quantidade
de ruído. Uma quantidade de falsos positivos nessa ordem não justifica
a identificação de equipamentos de menor porte contidos no conjunto
\emph{Empilhado4}. A alta sensibilidade dos valores também é refletida
no conjunto \emph{Empilhado7}, que assim como o \emph{Temporizado}
contém a presença de uma \acs{c5}, porém o arquivo é mais curto ---
apenas uma hora quando em comparação com o \emph{Temporizado} cuja
duração é superior à um dia ---, não dando uma margem tão grande de
falsos alarmes quanto no outro caso.

Ao comparar a convergência dos \acs{es}, percebe-se que os mesmos
convergiram praticamente para a mesma região, porém, a versão que
deveria obter menor falso alarme por pontuar suas ocorrências como uma
maior penalidade, o caso \acs{es}$ 1/2$, obteve um falso alarme maior
do que a sua versão com penalidade menor. A isso se atribuiu o fato
deles estarem explorando os limites da ruído do conjunto de dados
\emph{Temporizado}, que, ao realizar uma pequena mudança no sentido de
melhorar a detecção, irá gerar uma quantidade grande na geração de
falsos alarmes. E no caso do filtro acabar convergindo para uma
determinada janela, a mudança para uma versão com valores ajustados de
corte para a nova janela depende de duas mutações corretas, sendo
dificeis de ocorrerem em especial para a condição para a qual o
\acs{es} irá reduzir a pertubação de sua estratégia evolutiva por ter
encontrado uma região proeminente que será difícil de escapar. Por
isso, a diferença entre as convergências na verdade é causada apenas
para a configuração de janela que o filtro acabou escolhendo. Em
decorrência desse fato observado, percebeu-se que era necessário
realizar uma mudança no futuro do código, realizando uma otimização
em um ciclo interno do corte do filtro após determinada a janela, para
garantir melhor capacidade de convergência.

A versão de \emph{Ajuste Manual} foi a menos sensível, obtendo menor
detecção para o conjunto de dados \emph{Empilhado4}, mas ao mesmo
tempo garantindo um falso alarme total de apenas 29 ocorrências.
Isso mostra que com a capacidade de visualização foi possível prever
um valor que reduziria a sensibilidade do detector, sem que fosse
analisado vários casos até obtê-lo.

\begin{figure}[!htb]
\centering
\includegraphics[width=\textwidth]{imagens/convergencia_es_1-9.pdf}
\caption[Convergência para o ajuste automática através da regra
$1/0,9$.] {Convergência para o ajuste automática através da regra
$1/0,9$. As espécies são: remoção de eventos próximos estático após
eventos ruidosos (amarelo), remoção de eventos próximos pela sua média
após eventos ruidosos (azul), remoção de eventos próximos estatíco
antes de remoção de eventos ruidosos (rosa), remoção de eventos
próximos pela sua média antes eventos ruidosos (verde) e apenas
remoção de eventos ruidosos (marrom). A caixa de texto mostra a
convergência para o melhor caso: remoção de eventos próximos pela sua
média após eventos ruidosos (azul).}
\label{fig:convergencia_es_1}
\end{figure}

Em ambos os casos testados para o \acs{es}, a configuração de remoção
de eventos próximos pela sua média após a remoção de eventos ruidosos utilizando a
média se mostrou mais apta para a solução do problema, sendo por esse
motivo indicada. Na Figura~\ref{fig:convergencia_es_1} está a evolução
das espécies para o \emph{\acs{es} $1/0,9$}. Novamente a espécie de
menor complexidade genética (marrom) foi a que obteve convergência
mais rápida, porém, as outras espécies de maior carga genética, ao
evoluir, ultrapassam-na por volta da 30$^{\underline{a}}$ geração. A espécie rosa
não consegue acompanhar o processo evolutivo, indicando,
possivelmente, a sua convergência para um máximo local.

Entrando em mais detalhes quanto a discussão das ocorrências de falso
alarme e detecção, compara-se as Figura~\ref{fig:ruido_temporizado} e
Figura~\ref{fig:lampadas_emp4}. Na verdade, os eventos devido à falso
alarme no conjunto de dados \emph{Temporizado} ocorrem por causa de
uma configuração que se assemlha muito à uma mudança de estado com a
forma de um degrau de potência de cerca de 15 W que dura cerca de 2 à 
3s. No momento não se sabe qual equipamento que causa esse distúrbio,
mas nitidamente esses degraus apresentam exatamente as características
de uma mudança discreta no consumo. Agora fica a questão de como o
gabarito deveria ser montado, se considerando essas alterações como
eventos a serem detectadas, ou se elas continuam sendo ruído. 
Já para a segunda figura, observa-se que os valores
absolutos dos pontos de inflexão, por volta de $0,006$, já estão na
ordem do ruído do arquivo \emph{Temporizado}, ou seja, para detectar
esses eventos é necessário descer o patamar do filtro a um valor que
irá causar uma excessiva ocorrências de falso alarme nesse arquivo ---
caso do filtro anteriormente utilizado pelo \acs{cepel}, cujo patamar
é de $0,003$. A ocorrência de um grande número de falso alarmes irá
acarretar numa perda de aptidão, tornando esses individuos soluções
não desejáveis, cujos individuos que exploraram esses eventos não
apresentaram boa aptidão e foram eliminados.

Assim, grande parte dos falsos alarmes ocorridos para o
\emph{Temporizado} no caso dos \glspl{es} são devidos a essa mudança
de operação desconhecida de um equipamento e não a falso alarme devido à
dinâmica de operação da televisão LCD. Apenas como referência,
contabilizou-se a ocorrência desses casos no conjunto de dados
\emph{Temporizado} utilizando a informação gráfica, observando 36
ocorrências desse tipo desse tipo de mudanças de estado marcadas como
evento, e portanto contabilizadas como falso alarme. Caso esses
eventos fossem contabilizados como alvos, somente 3 falso alarmes
seriam contabilizados. Mas a questão de como tratar esse
problema fica aberta a discussão.

\begin{sidewaysfigure}[p]
\centering
\includegraphics[width=\textwidth]{imagens/temporizadoFA.pdf}
\caption[Exemplos de falsos alarme no conjunto de dados \emph{Temporizado}.]
{Exemplos de falsos alarme no conjunto de dados \emph{Temporizado}. Os
falsos alarmes são as caixas verdes e vermelhas exibidas. Na subfigura
inferior, são mostrados os pontos de inflexão e seus valores em
unidades da resposta do filtro.}
\label{fig:ruido_temporizado}
\end{sidewaysfigure}

\begin{sidewaysfigure}[p]
\centering
\includegraphics[width=\textwidth]{imagens/emp4_evtsPerdidos.pdf}
\caption[Exemplos de perdas de alvo no conjunto de dados Empilhado4]
{Exemplos de perdas de alvo no conjunto de dados \emph{Empilhado4}. Os
eventos não detectados tem seus pontos de inflexão e seus valores em
unidades da resposta do filtro exibidos.}
\label{fig:lampadas_emp4}
\end{sidewaysfigure}

\section{\acl{som}}
\label{sec:som_resultadados}

Foram feitos dois treinamentos com pré-processamento dos dados
distintos. O primeiro realiza uma normalização nas variáveis de
potência do evento dividindo-as por 127V. A segunda por sua vez
realiza o mesmo processo porém divide as potências pela tensão medida
no instante da aquisição. Para a comparação dos treinamentos foram
avaliados os \gls{qe} e \gls{te} para cada uma das normalizações. A
Tabela~\ref{tab:te_qe_norm} exibe a comparação entre os erros.
	
%TABELA 1
\begin{table}[!htb]
\centering
\begin{tabular}{ccc}
\hline \hline
Normalização & $Qe$ & $Te$\\
\hline \hline
divisão por 127V & 0,0082 & 0,1321 \\
\hline
divisão pela tensão medida & 0,0078 & 0,1019 \\
\hline
\hline
\end{tabular}
\caption{Valores dos erros de mapeamento para cada normalização.}
\label{tab:te_qe_norm}
\end{table}	
	
Conclui-se que a segunda normalização obteve erros menores quando
comparada com a primeira. Para o \gls{qe} essa
diferença é de 0,0004. Embora pequeno, esse valor indica que a segunda
normalização obteve uma maior aproximação da nuvem dos eventos. No
caso do \gls{te}, a diferença é de 0,0302. Indicando que a segunda
normalização obteve mais sucesso ao copiar as informações que estão no
espaço de 4 dimensões para o espaço bidimensional do mapa. Tendo em
vista esses resultados, a segunda normalização será utilizada no
processo de treinamento do \acs{som}.

Na Figura~\ref{fig:matrizu_res} é apresentada a \gls{matrizu} do mapa para o
pré-processamento dos dados escolhido. A \gls{matrizu} representa as
distâncias dos agrupamentos formados pelo \gls{som} que estão no
espaço de alta dimensão copiadas para o espaço bidimensional. Cores
frias (próximas do azul escuro) representam uma próximidade dos
neurônios do mapa naquela região. Por outro lado cores quentes
(próximas do vermelho) representam maiores distâncias entre os
neurônios na região.

\begin{figure}[!htb]
\centering
\includegraphics[width=9cm]{imagens/umatrix.pdf}
\caption[Matriz-U formada pelo treinamento do SOM]
{\acs{matrizu} formada pelo treinamento do \acs{som}.}
\label{fig:matrizu_res}
\end{figure}

Na \acs{matrizu} são encontradas duas regiões quentes. Uma na parte
superior esquerda do mapa e outra na parte inferior direita. Essas
regiões indicam que existem eventos que são bem diferentes dos eventos
que marcaram nas regiões em azul do mapa. De fato, transições como as
do chuveiro elétrico ou forno possuem uma mudança de patamar bastante
discriminante quando comparadas com as lâmpadas ou as televisões,
sendo então eventos bem distantes no espaço multidimensional. Para as
regiões de cores frias existem duas quebras. Uma na região superior
direita e uma na região inferior esquerda do mapa indicando a formação
de um agrupamento. Nas laterais do mapa formaram-se duas ilhas no lado
direito e uma outra formação no lado esquerdo. No centro do mapa não
existe nenhum corte bem definido. Indicando que nessa região podem
existir transições de aparelhos que estão muito próximos no espaço
multidimensional.
	
A Figura~\ref{fig:allhits} representa o comportamento de cada variável
utilizada no treinamento no espaço bidimensional. Repare que existem
regiões quentes e frias em cada um dos mapas gerados. Cores azuis
representam uma concentração de transições de valores negativos bem
elevados. Por outro lado cores avermelhadas representam uma
concentração de transições de valores positivos elevados.	
	
\begin{figure}[!htb]
\centering
\includegraphics[width=.9\textwidth]{imagens/all_comp_2.pdf}
\caption{Disperção de energia das características utilizadas no SOM.}
\label{fig:allhits}
\end{figure}
	
O algoritmo de \emph{k-means} foi executado até encontrar um número
$k$ de agrupamentos que representa-se bem a distribuição de ativações
dos aparelhos pelo mapa. Para valores de $k$ pequenos formam gerados
agrupamentos nas pontas do mapa, indicados pelas regiões quentes
vistas pela \acs{matrizu}, e poucos agrupamentos na região central e
lateral do mapa. Em contra partida, valores de $k$ bem elevados
estabilizam a formação de agrupamentos na região periférica do mapa
mas quebram agrupamentos em diversos subagrupamentos na região
central do mapa. Como no caso das lâmpadas fluorescentes de baixa
potência, que embora existam diversos tipos no conjunto de
treinamento, ambas possuem na prática o mesmo comportamento.
	
A Figura~\ref{fig:agrup_kmeans}, por sua vez, mostra os agrupamentos
formados pelo algoritmo de \emph{k-means}. Ao todo temos 16 formações
encontradas com $I_{DB}$ igual a 0,59. Cada agrupamento foi rotulado
de -8 a 8 e a escala de cores representa a potência real média de cada
agrupamento.  Ou seja, cores quentes próximas do vinho e cores frias
próximas do azul escuro indicam uma concentração de aparelhos que
possuem variações de potencia em módulo elevadas.
	
\begin{figure}[!htb]
  \begin{center}
    \begin{subfigure}[c]{7.5cm}
      \includegraphics[width=7.5cm]{imagens/kmeans_cluster.pdf}
      \caption{Agrupamentos formados pelo k-means}
      \label{fig:kmeans_cluster}
    \end{subfigure}
    \begin{subfigure}[c]{7.5cm}
      \includegraphics[width=7.5cm]{imagens/idb.pdf}
      \caption{Índice de Davies Bouldin.}
      \label{fig:bouldin}
    \end{subfigure}
  \end{center}
\caption[Agrupamentos formados pelo algoritmo de k-means.]{
Agrupamentos formados pelo algoritmo de \emph{k-means}. Os
agrupamentos foram pintados visando uma escala de consumo. A numeração
indica o rótulo do agrupamento.}
\label{fig:agrup_kmeans}
\end{figure}

A Tabela~\ref{tab:media_pot_real}  exibe a média das potências reais
em cada um dos 16 agrupamentos formados pelo \emph{k-means}.	
	

%TABELA 1
\begin{table}[!htb]
\centering
\begin{tabular}{ccccccccc}
\hline \hline
\multicolumn{9}{c}{Média das potências reais}\\
\hline \hline
Agrupamento & -8 & -7 & -6 & -5 & -4 & -3 & -2 & -1 \\
Média $VA/V$ & -17,39 & -9,98 & -5,04 & -2,31 & -1,05 & -0,38 & -0,37
& -0,35\\
\hline
Agrupamentos & 1 & 2 & 3 & 4 & 5 & 6 & 7 & 8\\
Média $VA/V$ & -0,18 & 0,17 & 0,40 & 0,44 & 0,88 & 2,28 & 6,69 &
14,52\\
\hline
\hline
\end{tabular}
\caption{Média das variações de potência real em cada agrupemanto.}
\label{tab:media_pot_real}
\end{table} 


A dispersão de transições para cada um dos aparelhos será mostrada nas
figuras \ref{fig:hits_altapot} e \ref{fig:baixapot}. Cada neurônio do
mapa foi pintado de acordo com a transição dos aparelhos que o
ativaram. Na maioria dos casos, a cor vermelha representa um
desligamento e a cor verde um ligamento.  Algumas transições de
aparelhos com mais de dois estados, como o ar condicionado, foram
representadas com cores diferenciadas.  Para cada neurônio ainda temos
o número de eventos que o ativaram (primeira linha dentro do hexágono)
e o agrupamento pertencente (segunda linha do hexágono). A
Figura~\ref{fig:hits_altapot} mostra a dispersão dos eventos para o
secador, a sanduicheira, o forno e o chuveiro elétrico --- aparelhos
de alto consumo. Os quatro tipos de aparelhos observados ativaram
neurônios localizados no canto superior esquerdo e no canto inferior
direito do mapa. Essas regiões englobam agrupamentos de alto consumo,
o que já era esperado para esses aparelhos. Outro ponto é a
localização dos tipos de transições no mapa. Repare que as transições
de ligamento (cor verde) marcaram na região inferior e as de
desligamento (cor vermelha) na região superior do mapa. Indicando um
espelhamento na localização desses eventos. 
	
\begin{figure}[!htb]
  \begin{center}

    \begin{subfigure}[c]{\textwidth}
      \begin{subfigure}[c]{8cm}
        \includegraphics[width=8cm]{imagens/hits_Secador.pdf}
        \caption{Dispersão do secador}
        \label{fig:hits_secador}
      \end{subfigure}
      \begin{subfigure}[c]{8cm}
        \includegraphics[width=8cm]{imagens/hits_Sandch.pdf}
        \caption{Dispersão da sanduicheira}
        \label{fig:hits_sanduicheira}
      \end{subfigure}
    \end{subfigure}

    \begin{subfigure}[c]{\textwidth}
      \begin{subfigure}[c]{8cm}
        \includegraphics[width=8cm]{imagens/hits_ChuvElet.pdf}
        \caption{Dispersão do chuveiro elétrico}
        \label{fig:hits_chuveiro}
      \end{subfigure}
      \begin{subfigure}[c]{8cm}
        \includegraphics[width=8cm]{imagens/hits_Forno.pdf}
        \caption{Dispersão do forno}
        \label{fig:hits_forno}
      \end{subfigure}
    \end{subfigure}

  \end{center}
\caption{Dispersão dos aparelhos de alto consumo no mapa. A cor
vermelha representa eventos de desligamento e a cor verde eventos de
ligamento.}
\label{fig:hits_altapot}
\end{figure}

A Figura~\ref{fig:eletronicos} mostra a dispersão dos eventos para a
televisão e para os 3 tipos de lâmpadas utilizadas no experimento.
Eventos como televisão, lâmpadas fluorescentes de baixa potência e as
incandescentes foram alocadas no centro do mapa. Indicando ativações
de aparelhos de baixo consumo como se pode ver pelos agrupamentos
formados e a Tabela~\ref{tab:media_pot_real}. Outro ponto é a
aproximidade desses aparelhos nessa região. 

\begin{figure}[h!t]
  \begin{center}
    \begin{subfigure}[c]{\textwidth}
      \begin{subfigure}[c]{8cm}
        \includegraphics[width=8cm]{imagens/hits_Televisao.pdf}
        \caption{Dispersão da televisão}
        \label{fig:hits_televisao}
      \end{subfigure}
      \begin{subfigure}[c]{8cm}
        \includegraphics[width=8cm]{imagens/hits_LFLow.pdf}
        \caption{Dispersão das lâmpadas fluorescentes\\ de baixa
  potência.}
        \label{fig:hits_lf_ap}
      \end{subfigure}
    \end{subfigure}

    \begin{subfigure}[c]{\textwidth}
      \begin{subfigure}[c]{8cm}
        \includegraphics[width=8cm]{imagens/hits_LFHigh.pdf}
        \caption{Dipersão das lâmpadas fluorescentes\\ de alta potência.}
        \label{fig:hits_lw_bp}
      \end{subfigure}
      \begin{subfigure}[c]{8cm}
        \includegraphics[width=8cm]{imagens/hits_LIHigh.pdf}
        \caption{Dispersão das lâmpadas incandescentes\\ de alta potência}
        \label{fig:hits_li_ap}
      \end{subfigure}
    \end{subfigure}
  \end{center}
\caption{A cor vermelha representa eventos de desligamento e a cor
verde eventos de ligamento. Para a televisão a cor cinza representa
uma mudança do estado desligado para uma imagem na tela e a cor roxa
representa uma mudança do estado ligado para imagem (congelamento de
sinal, erro de recepção, etc)}
\label{fig:eletronicos}
\end{figure}

Diferentemente dos aparelhos de alta potência mostrados na
Figura~\ref{fig:baixapot}, os de baixa potência como as lâmpadas e as
televisores possuem características muito próximas. Apenas as lâmpadas
fluoerescentes de alta potência se destacaram fora da região central
do mapa devido ao seu consumo elevado quando comparadas com as outras
lâmpadas.

\begin{figure}[h!tb]
  \begin{center}
    \begin{subfigure}[c]{\textwidth}
      \begin{subfigure}[c]{8cm}
        \includegraphics[width=8cm]{imagens/hits_Geladeira.pdf}
        \caption{Dispersão da geladeira.}
        \label{fig:hits_geladeira}
      \end{subfigure}
      \begin{subfigure}[c]{8cm}
        \includegraphics[width=8cm]{imagens/hits_AllShutDown.pdf}
        \caption{Ativação para o evento de desligamento\\ geral.}
        \label{fig:hits_allshutdown}
      \end{subfigure}
    \end{subfigure}

    \begin{subfigure}[c]{\textwidth}
      \begin{subfigure}[c]{8cm}
        \includegraphics[width=8cm]{imagens/hits_Ventilador.pdf}
        \caption{Dipersão do ventilador.}
        \label{fig:hits_ventilador}
      \end{subfigure}
      \begin{subfigure}[c]{8cm}
        \includegraphics[width=8cm]{imagens/hits_TelevisaoCRT.pdf}
        \caption{Ativação para uma televisão CRT.}
        \label{fig:televisao_CRT}
      \end{subfigure}
    \end{subfigure}
  \end{center}
\caption{A cor vermelha representa eventos de desligamento e a cor
verde eventos de ligamento. Na dispersão das geladeiras, eventos
marcados com a cor laranja representam uma mudança do estado de
standby para o ligado e cor amarela representa a mudança do estado
desligado para o standby.}
\label{fig:baixapot}
\end{figure}

As transições para as geladeiras formaram dois agrupamentos bem
definidos. O primeiro localizado no canto superior direito do mapa é
formado o eventos de transição de desligamento. O segundo localizado
na região inferior esquerda do mapa é formado basicamente por eventos
de ligamento. Existem ainda geladeiras diferenciadas neste grupo.
Algumas delas formadas por geladeiras com alto consumo, processadores
integrados e fazedores de gelo. Essas geladeiras podem ser vistas nas
transições pintadas por cores diferenciadas como o laranja e amarelo
que possuem um estado de stanby a mais. Os ventiladores formaram
agrupamentos bem definidos nas regiões laterais do mapa. Também com o
mesmo perfil de espelhamento já visto nos mapas anteriores. 
Por sua vez, o evento nomeado como \emph{AllShutDown} marcou em um
agrupamento de potência perto de aparelhos como forno e chuveiro
elétrico. A televisão CRT (ou de tubo) possui apenas um evento de
ligamento que está localizado na mesma região de ligamento das
lâmpadas fluorescentes de baixa potência. 

\begin{figure}[!htb]
\centering
\includegraphics[width=8.5cm]{imagens/hits_ArCondicionado.pdf}
\caption{Disperção do evento ar condicionado no mapa.}
\label{fig:hit_ac}
\end{figure}

Já na Figura~\ref{fig:hit_ac} apresenta a localização do ar condicionado
no mapa. Vale lembra que o ar condicionado possui quatro tipos de
transição.  A \gls{fsm} do ar condicionado é formada por quatro
transições. A primeira representa a mudança do estado desligado para a
ventilação. A segunda, no momento em que o compressor arma, vai do
estado de ventilação para compressão. A terceira segue o sentido
contrário que vai da compressão para ventilação. E por último da
ventilação para o desligado. Sendo as duas primeiras transições
marcadas pela súbida de estado de operação da máquina e as duas
últimas os de descida. Na distribuição do mapa podemos ver que as
ativações que envolvem o estágio final da máquina, o de compressão,
estão localizadas próximo as regiões de aparelhos de alto consumo.
Transições que envolvem o estado de desligado e ventilação em ambos os
sentidos marcaram próximo as regiões laterais do mapa. Indicando uma
aproximação com os agrupamentos formados pelos ventiladores. 

Por fim, a Tabela~\ref{tab:porc_aparelho} contém a porcentagem de cada
aparelho em cada um dos agrupamentos formados. Os aparelhos foram
ordenados por escala de consumo e cada barra de porcentagem foi
pintada fazendo referência a cor do respectivo agrupamento.
Observa-se que agrupamentos de baixa potência na região central do
mapa, rótulados como 1 e 2, possuem uma forte mistura entre
televisores e lâmpadas fluorescentes. Aparelhos como forno, chuveiro e
sanduicheira tiveram suas localizações no topo da tabela. Tendo pouca
ou nenhuma mistura com outros aparelhos. Os agrupamentos localizados
na região lateral do mapa, rotulados como -2 e 3, possuem uma
distribuição uniforme de ventiladores, 50$\%$ para cada agrupamento, e
uma pequena mistura com as lâmpadas incandescentes.

Embora o mapeamento do \acs{som} não tenha gerado somente agrupamentos
com um único tipo de aparelho, suas marcações quanto ao tipo de
transição foram feitas quase simétricas. Apenas a região central do
mapa, mais especificamente os agrupamentos 1 e 2, é que está sujeita
há algum tipo de erro devido a enorme região de confusão entre as
lâmpadas e televisores.
	
Tendo em vista essa propriedade de separação quanto as transições
dos aparelhos gerados no mapa, utilizar-se-á o \acs{som} na marcação de
transições e compara-lo com o desempenho do algoritmo genético. Uma
ação conjunta entre \acs{som} e o algoritmo genético tambem será sugerida na
próxima sessão. 

% FIXME
\begin{landscape}
\begin{table}[p]
\centering
\includegraphics[height=1.3\textheight]{imagens/table.pdf}
\caption{Porcentagem de cada aparelho em cada cluster formado.}
\label{tab:porc_aparelho}
\end{table}
\end{landscape}

\section{Análise de ambas estratégias}
\label{sec:som_e_es}

Nesta sessão será apresentado os resultados de detecção de transição
do \acs{som} e compará-los com os do algoritmo genético. Também será
apresentado os resultados em conjunto. Utilizando o \acs{som} para diminuir
os falsos alarmes gerados pelo algoritmo genético. Os resultados serão
apresentados para cada um dos três conjuntos de dados fornecidos pelo
\acs{cepel}.
	
Ao injetar ruído na entrada da rede neural, percebeu-se que as
ativações geradas no mapa se concentravam na região de fronteria entre
os agrupamentos 1 e 2. De fato essas regiões apresentam valores de
variação de potência real média próximas de zero. Sendo o agrupamento
1 com média de -0,18 VA/V e o agrupamento 2 com 0,17 VA/V. Visto isso
definimos a região de fronteira desses dois agrupamentos como zona de
ruído ou zona morta. Entradas que ativem essa região não irão disparar
a marcação do evento na linha do tempo. A Figura~\ref{fig:regiao_morta}
mostra a zona morta do mapa pintada de preto.
	
\begin{figure}[!htb]
\centering
\includegraphics[width=8cm]{imagens/deadZone.pdf}
\caption{Os hexágonos de cor preta representam a zona sujeita a
ruído definida como zona morta do mapa.}
\label{fig:regiao_morta}
\end{figure}

O algoritmo irá analisar se o mesmo ativou ou não a zona morta quando
propagado um evento na rede. Caso positivo, a marcação do evento não
dispara e ele aguarda até a próxima entrada. Porém se o evento ativou
uma região fora da zona morta, o algoritmo começa a marcar o sinal e a
salvar a númeração das amostras até que as ativações novamente voltem
para a zona morta do mapa.
	
Como em geral uma transição tem entre 250 a 300 amostras de variação,
será realizado um corte de eventos ao final da execução do algoritmo.
Eventos que tenham marcado o sinal com menos de 150 amostras serão
considerados falhas de aquisição e serão excluidos da análise.	

As figuras \ref{fig:som_es_temp}, \ref{fig:som_es_emp4} e
\ref{fig:som_es_emp7} exibem a marcação dos eventos nos conjuntos
de dados \emph{Temporizado}, \emph{Empilhado4} e \emph{Empilhado7}
respectivamente para o \acs{es} e a rede neural \acs{som}.
Marcações em verde significam uma transição verdadeira dentro da marca
gerada pelos algoritmos, as marcações em azul representam a
localização das transições do conjunto de dados e as marcações em
vermelho representam os falsos alarmes gerados.

	
\begin{sidewaysfigure}[p]
\centering
\includegraphics[width=23cm]{imagens/TemporizadoDeltas_nc2_150to1000.pdf}
\caption{Marcação das transições para o algoritmo genético e a rede
neural SOM pare os dados do \emph{Temporizado}. Linhas em verde significam
verdadeiro positivo,as vermelhas o falso alarme e as azuls a verdade.}
\label{fig:som_es_temp}
\end{sidewaysfigure}

\begin{sidewaysfigure}[p]
\centering
\includegraphics[width=23cm]{imagens/Empilhado4Deltas_nc2_150to1000.pdf}
\caption{Marcação das transições para o algoritmo genético e a rede
neural SOM para os dados do \emph{Empilhado4}. Linhas em verde significam
verdadeiro positivo,as vermelhas o falso alarme e as azuls a verdade.}
\label{fig:som_es_emp4}
\end{sidewaysfigure}

\begin{sidewaysfigure}[p]
\centering
\includegraphics[width=23cm]{imagens/Empilhado7Deltas_nc2_150to1000.pdf}
\caption{Marcação das transições para o algoritmo genético e a rede
neural SOM para os dados do \emph{Empilhado7}. Linhas em verde significam
verdadeiro positivo,as vermelhas o falso alarme e as azuls a verdade.}
\label{fig:som_es_emp7}
\end{sidewaysfigure}
	
Foi analisado, também, uma versão conjunta dos dois algoritmos. Como a
quantidade de falsos alarmes gerados pelo algoritmo génetico é muito
menor que os gerados pelo \acs{som}, o algoritmo combinado irá atuar encima
da resposta do genético. Eventos em que o genético tenha marcado mas o
\acs{som} não marcou serão descartados no final. Essa tentativa combinada
tem o objetivo de reduzir os falsos alarmes gerados pelo genético. A
Tabela~\ref{tab:det_fa_es_som} mostra a quantidade de detecções e
falsos alarmes para cada um dos algoritmos testados encima de cada um
dos banco de dados.

\begin{table}[!htb]
\centering
\begin{tabular}{ccccccc}
\hline
\hline
%\toprule
 &\multicolumn{2}{c}{Genético} & \multicolumn{2}{c}{Combinado} & \multicolumn{2}{c}{SOM}\\
Dados & DET & FA & DET & FA & DET & FA\\
\hline
Temporizado (149 eventos) & 148 & 42 & 148 & 41 & 149 & 236\\
Empilhado4 (74 eventos) & 37 & 0 & 35 & 0 & 55 & 1\\
Empilhado7 (42 eventos) & 37 & 1 & 35 & 1 & 38 & 8\\
\hline
\hline

\hline
\hline
\end{tabular}
\caption{Valores de detecção e falsos alarmes para cada um dos
algorítmos testados.}
\label{tab:det_fa_es_som}
\end{table}	

Analisando somente o \acs{som}, o conjunto \emph{Temporizado} disparou uma
quantidade de falsos alarmes quase 6 vezes a mais que o algoritmo
genético. Esse problema deve-se a sensibilidade do \acs{som} para faixas de
baixa potência presente no ruído deste conjunto de dados. Sua
sensibilidade permitiu acertar todos os eventos do conjunto de dados a
troco de uma quantidade de falsos alarmes enorme. Por sua vez, o
algoritmo genético obteve melhores resultados errando apenas um evento
e marcando apenas 43 falsos alarmes. O algoritmo combinado só foi
capaz de excluir apenas um falso alarme do algoritmo genético.

Para o conjunto de dados \emph{Empilhado4}, o \acs{som} obteve os melhores
resultados de detecção quando comparado com o genético. Esse resultado
deve-se a sensibilidade do \acs{som} para eventos como lâmpadas de baixa
potência e a pouca presença de ruído nas amostras. Para o algoritmo
combinado o \acs{som} excluiu dois eventos verdadeiros detectados pelo
genético.
	
Para o conjunto \emph{Empilhado7}, o \acs{som} obteve um total de 38 detecções
verdadeiras e 8 falsos alarmes contra 37 detecções e 1 falso alarme do
algoritmo genético. O algoritmo combinado retirou 2 detecções do
genético e nenhum falso alarme. As duas marcações eram lâmpadas
fluorescentes de alta e baixa potência localizadas no período em que o
ar condicionado está ligando.
	
Conclui-se que o algoritmo genético possui uma maior especificidade
para eventos de transição gerando em todos os casos números maiores
que 50\% do conjunto de dado e valores de falso alarme bem abaixo dos
gerados pelo algoritmo do \acs{som}.

