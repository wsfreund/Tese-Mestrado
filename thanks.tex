\chapter*{Agradecimentos}

Agradeço aos meus pais por tornar isso tudo possível. Vocês que
tiveram o maior peso para que isso se transformasse em realidade.
Agradeço de coração por todo esforço e trabalho que tiveram com o
intuíto de me verem chegar aqui. Aos meus avós por todo o carinho e
apoio que sempre me deram, sempre me incentivando para conseguir
atingir meus sonhos. Ao meu irmão mais novo que sacríficou seu tempo
livre fazendo algumas de minhas tarefas enquanto eu estava
trabalhando. Ao meu outro irmão pelas risadas.

Ao meu orientador, Seixas, que me deu suporte e me guiou neste
trabalho, apesar de todas condições atípicas. Seus conselhos foram
muito valiosos, sua ajuda foi muito mais do que essêncial. Aos outros
orientadores que tive, Torres e Damazio, suas orientações ajudaram a
tornar-me o que sou hoje.

Aos engenheiros do CATE no \acl{cepel}, João, Aroldo, Guilherme e
Victor por toda a ajuda e compreensão durante o tempo de trabalho. A
convivência com vocês foi muito agradável! Também à Beth por sempre
ter sido prestativa. Ao Alvaro pela informação necessária
para a construção dos gabaritos dos conjuntos de dados
\emph{Empilhado4} e \emph{Empilhado7}.

Ao pessoal do \acs{lps}: Balabram, Junior, Moura, Grael, Rodrigo, João
Victor, Diego e Hellen. Vocês fazem o laboratório ser o que ele é, sem
vocês ele não tem graça alguma. À Ana e Talia, vocês fazem muita falta
lá. Agradeço em especial ao João Victor pelo suporte providencial no
momento em que precisei, espero ter retornado da maneira possível com
o máximo de conhecimento em troca. Desculpas ao Grael por todo
incomodo causado, muito obrigado por fazer o melhor que podia. Ao
Pedro que ajudou com parte do material utilizado para o
levantamento bibliográfico das técnicas empregadas no \acs{nilm}.
Aos meus orientados, Diego e Hellen, me desculpem se exagerei na
cobrança: corram atrás de seus futuros que o \emph{Walhalla} espera
por vocês!

À toda ajuda, paciência e instruções oferecidas pela Dani do
\acs{peecoppe}. Aos professores da \acs{coppe} por todo seu
empenho, ética e dedicação.

Finalmente, agradeço a todos meus amigos que escutaram diversas vezes:
``não posso, tenho que terminar o mestrado'', mas nunca deixaram de
compreender o momento em que estava passando. Hoje chegou o dia!

