\chapter{Conclusão}
\label{chap:conclusao}

Este trabalho trouxe um extenso levantamento bibliográfico, desde um
detalhamento das aplicações do \acs{nilm} para a obtenção de uma
melhor \acs{ee}, às diversas técnicas aplicadas por outros autores
em busca da obtenção de uma estimativa do consumo desagregado. A
primeira parte do levantamento, em relação à aplicação do \acs{nilm}
para \acs{ee}, detalhou um novo programa de \acs{ee} proposto em
países desenvolvidos como \acs{eua} e outros países na Europa
Ocidental que permite a redução do consumo de energia elétrica no
setor residencial através do retorno da informação (estimada) de
consumo, aonde o \acs{nilm} pode ter papel fundamental de tornar essa
informação disponível com baixo custo. A intenção desses autores é
tentativa da união da infraestrutura que será disponibilizada pelas
redes elétricas inteligentes (\emph{smart grid}) como sistema de
aquisição de dados --- os medidores inteligentes (\emph{smart
meters}). Além disso, o mesmo pode auxiliar na obtenção de uma
informação com melhor qualidade e maior frequência para os estudos em
\acs{ee} realizados por entidades governamentais, aonde o mesmo pode
auxiliar na informação obtida através das \glspl{pph} ou, até mesmo,
futuramente substituí-la.

O levantamento das técnicas permitiu a realização de diversas
considerações sobre os pontos envolvidos no projeto dessa tecnologia,
aonde essa informação foi organizada e debatida na primeira seção
Capítulo~\ref{cap:nilm}. Isso permitiu uma melhor compreensão do
projeto como um todo, o que levou ao desenvolvimento de um ambiente
para atender às especificações observadas. Dentre elas, destaca-se a
necessidade de um ambiente único de análise para a evolução do
projeto, que pode utilizar diversas técnicas e características
diferentes durante o seu progresso, com o intuito de melhorar a
capacidade de reconstrução de energia do \acs{nilm}, que inclusive
pode alterar a configuração de seu sistema de aquisição de dados e,
portanto, precisa ser capaz de representar os dados e análises em uma
única maneira. Esse ambiente permite analisar o comportamento do
\acs{nilm} antes que o mesmo seja colocado em operação.

A implementação desse ambiente adaptou a metodologia proposta pelo
\acs{cepel} para detecção de eventos de transitório, passo necessário
para a exploração da informação deixada por equipamentos durante a
alteração de seu estado operativo e que foi atacado pelo grupo do
\acs{cepel}. A sua metodologia emprega um filtro de derivada de
Gaussiana para a geração de candidatos, que ainda passam por outros
dois cortes antes de serem aceitos como um evento de transitório: um
para o valor mínimo de degrau de corrente que, caso seja
demasiadamente pequeno, será removido por constituir um evento de
ruído; e outro para candidatos próximos, adicionando um tempo morto de
resposta do filtro antes que um novo candidato possa ser aceito.

Na versão da metodologia presente no ambiente de análise,
adicionaram-se diversos avanços operacionais, alguns exemplos: 
leitura de redes elétricas com até três fases, transformação dos dados
em um formato único para os dois medidores sendo utilizados;
segmentação da informação em memória para permitir a leitura de dados
longos; redução da necessidade de leitura em disco através do
armazenamento das bordas dos arquivos em memória transitória;
robustez indicando capacidade de recuperação de dados com falhas no
medidor ou sua sobrecarga; aumento da velocidade da análise através da
remoção de amostras irrelevantes na janela do filtro em convolução;
melhor capacidade de interpretação gráfica dos dados e das análises; etc.

Porém, as maiores contribuições foram: 

\begin{itemize}
\item Construção do gabarito: uma das dificuldades para o \acs{nilm} é
obter uma estimativa do consumo desagregado por aparelho a ser
identificado pelas técnicas aplicadas no \acs{nilm}. O ambiente tratou
desse caso fornecendo uma interface gráfica para o usuário, que
permitiu a criação da informação estimada a ser desagregada nos dados.
Após completar o seu preenchimento para um determinado conjunto de
dados, esse ambiente facilita a compreensão da informação contida no
mesmo através de informações gráficas;
\item Remoção de eventos por inconsistência: a mesma mostrou-se
suficiente nos conjuntos de dados analisados para a remoção de
candidatos a eventos de transitório antes removidos pela adição de
tempo morto em resposta do detector pela metodologia do \acs{cepel},
esses eventos causados por picos no consumo durante acionamento de
equipamentos --- como aqueles contendo motores. A inconsistência
ocorre quando a resposta do filtro tem sinal diferente àquele presente
no degrau (diferença entre pós e pré transitório) de corrente extraído
durante a geração do candidato à evento;
\item Capacidade de ajuste automático dos parâmetros: realizado
através de uma adaptação do \acl{es} inspirada nas versões
multiespécies de otimização multiobjetivo, porém para permitir uma
dinâmica durante a evolução das espécies que possibilita separar
maior capacidade computacional às espécies que melhor se adaptam para
a solução do problema. O otimizador também apresenta outras capacidades:
\begin{itemize}
\item Determinação do ponto de operação: através do ajuste dos
parâmetros livres para o cálculo da aptidão, é possível priorizar
maiores taxas de detecção ou falso alarme, porém essa capacidade não
pode ser avaliada neste trabalho; 
\item Recuperar o processo caso ocorra interrupção não planejada: o
algoritmo genético implementado é capaz de recuperar a evolução de um
processo perdido, o que é especialmente importante durante a
otimização de conjuntos com grande quantidade de dados.
\end{itemize}
\end{itemize}

Três ajustes utilizando o otimizador foram realizados:

\begin{itemize}
\item ES 1: antes de perseguir a aplicação do otimizador para os dados
simulando cenários de aplicação prática, aplicou-se a otimização para
os dados com condições mais simples --- os conjuntos \emph{NI00***}.
O otimizador apresentou menor falso alarme nos dados simulando
cenários de aplicação prática. Atribuiu-se essa capacidade ao
parâmetro $\gamma_{rem}$ utilizado para o cálculo da aptidão dos
indivíduos. Seu emprego permitiu explorar os limites do limiar de
sensibilização para a geração de candidatos a eventos de transitório,
o que reduziu a presença de falsos alarmes nos conjuntos representando
cenários de aplicação reais, mesmo que eles não tenham sido vistos
pelo otimizador;
\item ES 2: como o objetivo do trabalho foi a extensão da técnica para
condições mais próximas àquelas presentes nas redes elétricas
residenciais --- o ajuste de parâmetros realizado pelo \acs{cepel} foi
realizado para dados com pouca presença de sinal ruído e operação de
poucos (no máximo dois) equipamentos simultaneamente ---, aplicou-se o
ajuste automático para essas condições. A redução do falso alarme para
o conjunto \emph{Temporizado}, o conjunto com maior incidência de
falsos alarmes, não correspondeu às expectativas, o que durante a
investigação causou a descoberta de um distúrbio causado durante a
operação da televisão LCD presente em tal conjunto. Ao considerar esse
distúrbio como alvo, percebeu-se uma redução na taxa de detecção, o
que mostra que nem todos eventos desse distúrbio são identificados
pela metodologia. Assim, esse distúrbio ainda precisa ser melhor
compreendido antes da aplicação do \acs{nilm};
\item ES 3: para analisar a capacidade de generalização, ao invés de
empregar os três conjuntos de dados simulando cenários de aplicação
prática, realizou-se a otimização com apenas dois desses conjuntos.
Essa configuração permitiu identificar uma falha na fronteira
determinada para o parâmetro responsável pelo tempo morto de resposta
que permitia um limite superior de até 8,3 s aonde todos os candidatos
seriam removidos caso um evento de transitório fosse confirmado.
Apesar desse valor não ser justificável, para os dois conjuntos de
dados analisados esse valor se constituía de uma vantagem por permitir
a redução do falso alarme em detrimento de pouca perda na detecção, e
o algoritmo genético explorou essa falha ao percebê-la. Assim, ainda
que o ajuste seja realizado automaticamente, é necessário a realização
de critica pós-convergência para identificação de possíveis problemas
como esse. 
\end{itemize}

As taxas de detecção de falso alarme para o ajuste realizado pelo
\acs{cepel} eram respectivamente de 100~\% e 1,2~\% para os dados
\emph{NI00***}, enquanto esses valores foram de 90,2~\% e 82,9~\% para
os dados com cenários de aplicação prática. A configuração ES~1
alterou os valores para os cenários de aplicação prática para 83,18 e
3,0~\%, bem mais próximos a necessidade de aplicação do \acs{nilm},
enquanto seus valores para os conjuntos de dados \emph{NI00***}
permaneceram praticamente constantes, em 98,8~\% e 1,2~\%. Os valores
para os dados \emph{NI00***} também permaneceram na mesma faixa para
o ES 2, porém houve sua taxa para os dados com cenários de aplicação
prática foram 80,1~\% e 0,9~\%, reduzindo o falso alarme em detrimento
de uma menor taxa de detecção em comparação à configuração anterior.
Portanto, além de todas facilidades e capacidades já referidas, o
ambiente trouxe sistematização do processo de determinação dos
parâmetros, que podem ser ajustados automaticamente para novas
condições ainda não conhecidas presentes nas redes elétricas
residenciais. 

%A função do \acs{nilm} é informar o consumo de energia desagregado por
%aparelho, e não apenas identificar o mesmo. Além disso, a grande
%dificuldade para a implementação da tecnologia é quando a rede
%elétrica das residências está com maior operação de aparelhos, o que,
%dependendo de sua dinâmica operativa, irá causar ruído na rede.  Há
%ainda a questão da ocorrência de aparelhos desconhecidos que podem
%afetar a eficácia das técnicas. Por isso, é necessário determinar como
%a presença dos mesmos irá afetar a eficácia no \acs{nilm}. E, por esse
%mesmo motivo, é desejável que o \acs{nilm} tenha a capacidade de
%identificar aparelhos desconhecidos, adicionando-os ao seu catálogo e
%permitindo sua futura identificação de alguma maneira.
%
%Devido à percepção de que o projeto irá se prolongar bem além deste
%trabalho, viu-se a necessidade de realizar um ambiente para integrar
%as técnicas aplicadas no \acs{nilm} permitindo que elas sejam
%utilizadas no futuro em um único ambiente agregando toda a informação.
%Bem como também facilitar a continuação do projeto no rumo da
%discriminação dos aparelhos e de seus estados operativos, que é o
%próximo passo para tornar o desenvolvimento da tecnologia.
%Esse ambiente também pode auxiliar caso o \acs{cepel} opte por tomar
%outras direções no projeto --- como na possibilidade de impregnar o
%método atuando nos medidores inteligentes, mas nesse caso se limitando
%a amostragens mais baixas, ou elevar a amostragem com o objetivo de
%obter melhores eficácias ---, sendo necessário apenas ajustar os
%módulos para atender as novas configurações de operação do \acs{nilm}
%facilitando, assim, possíveis desdobramentos. 


%Esse ambiente permitiu avaliar a proposta do \acs{cepel} de um filtro
%de núcleo de Gaussiana para os três conjuntos de dados oferecidos pelo
%mesmo, onde estão representados possíveis cenários de aplicação real,
%contendo a operação simultânea de aparelhos na rede que a tornam um
%local mais complexo para operação da tecnologia. Nesses conjuntos de
%dados, aplicou-se um otimizador por algoritmo genético para obter os
%parâmetros ótimos para as regras de avaliação determinadas pelo
%usuário. Obteve-se taxa de detecção superior a 83\%, podendo ser
%considerada boa. A taxa de falso alarme significativa, próxima a 17\%,
%é causada em sua maioria pelo arquivo \emph{Temporizado}, aonde há um
%distúrbio semelhante à operação de um equipamento não conhecido, e por
%esse mesmo motivo, não estando presente no gabarito do \acs{nilm}.
%Como consequência, há um aumento da taxa de falso alarme. Nesses
%casos, seria importante ter, além dos rótulos dos tempos de alteração
%de estados realizadas nos equipamentos, a informação de submedição dos
%aparelhos para garantir a construção de um gabarito fiel à realidade
%dos conjuntos de dados. 

%Porém, foi constatado que a 
%de baixo consumo, onde muitos outros autores desconsideram aparelhos
%abaixo de uma determinada faixa de consumo, como 150 W, ou seja, o
%conjunto de dados apresenta uma dificuldade bastante elevada se o
%intuíto for detectar todos os eventos --- o caso para as taxas
%apresentadas. Ao alterar o ponto de operação elevando a penalidade
%para ocorrências de falso alarme, obterá-se uma resposta menos
%sensível aos ruídos presentes nesses dados.


\section{Trabalhos Futuros}
\label{sec:trabfut}

O trabalho tratou apenas de uma das etapas necessárias para o
desenvolvimento da tecnologia, o próximo passo é tratar do tema de
discriminação e identificação do estado operativo dos aparelhos.
Foram realizadas considerações sobre o tema das escolhas das
características neste trabalho, que pode auxiliar próximos trabalhos
em determinar quais delas serão utilizadas. Deseja-se aqui ressaltar a
característica que se sobressaiu no levantamento bibliográfico, a
estatística de uso. Apesar de não ser possível realizar a
discriminação apenas utilizando a envoltória como característica
única, uma vez que ela só é discriminante para mudanças de estados que
causam acréscimo no consumo, a mesma mostrou-se discriminante nos
estudos anteriores em colaboração com o \acs{cepel} e podem ser
empregadas para auxiliar na discriminação. Propõe-se que a
implementação seja realizada no ambiente de análise pela questão de
unificação e aproveitamento das ferramentas, além de toda a capacidade
já oferecida por esse ambiente.

Outra questão importante é a utilização das medidas que levam em sua
consideração a reconstrução em energia para a determinação da eficácia
do \acs{nilm}, bem como de conjuntos de dados reais. Também quanto a
questão dos dados, é interessante a utilização dos conjuntos
\acs{redd} e \acs{blued} para possibilitar a técnica ser comparada em
nível internacional. As medidas em termos de energia podem ser
adaptadas também para o caso de detecção de eventos, que devem ser
utilizadas com a evolução do trabalho.

A otimização automática de parâmetros também tem pontos que precisam
ser trabalhados. Primeiro, a versão do \acs{es} implementada ainda não
está completa, sendo necessário adicionar um método que permita a
competição dos indivíduos somente após atingir um estágio de
pré-evolução. Ainda, é necessário avaliar a capacidade do \acs{es} de
determinar o ponto de operação através do ajuste dos parâmetros livres
($\gamma_{det}$ e $\gamma_{fa}$) no cálculo da aptidão. Por
fim, outros algoritmos podem ser analisados, como Inteligência de
Enxames.

Quanto a técnica para detecção de eventos, podem ser aplicados outros
métodos além do filtro de núcleo de Gaussiana. Mesmo sem alterar a
ideia de convolução de um filtro, pode-se aplicar outra com uma função
simétrica no sinal para a detecção de eventos. Além disso,
durante este trabalho também se avaliou a capacidade da técnica
proposta pelo estudo anterior realizado pela colaboração
\acs{coppe}--\acs{cepel}, o Detector de Patamar Elaborado.
Percebeu-se que o mesmo apresenta um potencial para ser estendido para
os novos cenários, sendo necessário realizar o ajuste de seus
parâmetros. Nesse caso, também será necessário a sua adaptação para
não causar falsos alarmes durante a atuação de chaves \emph{break then
make}, presentes em alguns equipamentos como ventiladores e secadores
de cabelo. Outra possibilidade seria a aplicação da \gls{tw}, que como
observado durante o levantamento bibliográfico, tem aplicação
bem-sucedidada para a detecção de transitórios em redes elétricas.
Ainda, o patamar não precisa ser determinado como um valor fixo, é
possível estender essa metodologia para a utilização de patamares
adaptativos, tendo um valor de sensibilização conforme o nível de
dinâmica presente na rede elétrica.



