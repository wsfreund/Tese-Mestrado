\chapter{Conclusão}
\label{chap:conclusao}

O \acs{nilm} é uma técnica com potencial para obtenção de uma melhor
\acs{ee}, podendo auxiliar diretamente na obtenção de uma menor
intensidade energética ou facilitar no estudo realizados de \acs{ee}
para o setor residencial através da oferta de dados com melhor
qualidade e frequência.

A função do \acs{nilm} é informar o consumo de energia desagregado por
aparelho, e não apenas identificar o mesmo. Além disso, a grande
dificuldade para a implementação da tecnologia é quando a rede
elétrica das residências está com maior operação de aparelhos, o que,
dependendo de sua dinâmica operativa, irá causar ruído na rede.  Há
ainda a questão da ocorrência de aparelhos desconhecidos, sendo
necessário entender como a presença dos mesmos irá afetar a eficácia
no \acs{nilm}. E, por esse mesmo motivo, é desejável que o \acs{nilm}
tenha a capacidade de identificar aparelhos desconhecidos,
adicionando-os ao seu catálogo e permitindo sua futura identificação
de alguma maneira.

Devido â percepção de que o projeto irá se prolongar bem além deste
trabalho, viu-se a necessidade de realizar um ambiente para integrar
as técnicas aplicadas no \acs{nilm} permitindo que elas sejam
utilizadas no futuro em um único ambiente agregando toda a informação.
Inclusive para facilitar a continuação do projeto no rumo da
discriminação dos aparelhos e de seus estados operativos que é o
próximo passo para tornar o desenvolvimento da tecnologia.
Esse ambiente também pode auxiliar caso o \acs{cepel} opte por tomar
outras direções no projeto --- como na possibilidade de impregnar o
método atuando nos medidores inteligentes, mas nesse caso se limitando
a amostragens mais baixas, ou elevar a amostragem com o objetivo de
obter melhores eficácias ---, sendo necessário apenas ajustar os
módulos para atender as novas configurações de operação do \acs{nilm}
facilitando, assim, possíveis desdobramentos. 

Uma das dificuldades para o \acs{nilm} é obter uma estimativa do
consumo desagregado por aparelho para permitir a otimização das
técnicas e o cálculo de eficiência do \acs{nilm}. O ambiente tratou
desse caso fornecendo uma interface gráfica para o usuário, que
permitiu a criação da informação estimada a ser desagregada nos dados.
Após completar o seu preenchimento para um determinado conjunto de
dados, esse ambiente facilita a compreensão da informação contida no
mesmo através de informações gráficas. Apenas utilizando essa
informação já foi possível determinar um valor adequado a ser
empregado na metodologia original do \acs{cepel}.

Esse ambiente permitiu avaliar a proposta do \acs{cepel} de um filtro
de núcleo de Gaussiana para os três conjuntos de dados oferecidos pelo
mesmo, que contém cenários possíveis de aplicação real representando a
operação simultânea de aparelhos na rede que a tornam um local mais
complexo para operação da tecnologia. Nesses conjuntos de dados
aplicou-se um otimizador por algoritmo genético para obter os
parâmetros ótimos para as regras de avaliação determinadas pelo
usuário. Obteve-se taxa de detecção superiores a 84\%, podendo ser
considerada boa. A taxa de falso alarme alta, próxima à 16\%, é
causada em sua maioria pelo arquivo Temporizado, aonde há uma
alteração de estado de equipamento não conhecida nem presente no
arquivo de gabarito, sendo contabilizada como falso alarme. Fica
evidente, assim, a importância de ter em mãos tanto a informação dos
momentos de estados programados, quanto da submedição dos aparelhos
para garantir a construção de um gabarito fiel à realidade dos
conjuntos de dados. 

%Porém, foi constatado que a 
%de baixo consumo, onde muitos outros autores desconsideram aparelhos
%abaixo de uma determinada faixa de consumo, como 150 W, ou seja, o
%conjunto de dados apresenta uma dificuldade bastante elevada se o
%intuíto for detectar todos os eventos --- o caso para as taxas
%apresentadas. Ao alterar o ponto de operação elevando a penalidade
%para ocorrências de falso alarme, obterá-se uma resposta menos
%sensível aos ruídos presentes nesses dados.

Para auxiliar na compreensão do problema, também se empregou \acs{som}
com o objetivo de melhor compreender as características dos distúrbios
devido à alteração de um estado de consumo de um aparelho. Estudou-se,
também, a possibilidade de empregá-los para aperfeiçoar a metodologia
anterior. Apesar do \acs{som} não ter mostrado a capacidade de
reproduzir a eficiência do filtro de núcleo de Gaussiana, nem auxiliar
de maneira efetiva na eliminação da alta taxa de falso alarme, ele
mostrou capacidade discriminante bem como aptidão para sintetizar a
informação em um degradê de potência.

\section{Trabalhos Futuros}
\label{sec:trabfut}

O trabalho tratou apenas de uma das etapas necessárias para o
desenvolvimento da tecnologia, sendo ainda preciso tratar do tema de
discriminação e identificação do estado operativo dos aparelhos.
Foi realizada uma abordagem sobre a escolha das características, que
pode auxiliar próximos trabalhos para determinar o caminho que será
seguido. Aqui se ressalta a característica que se sobressaiu, a
estatística de uso. Também não é possível realizar a discriminação
apenas utilizando a envoltória como única característica uma vez que
ela só é discriminante para mudanças de estados que causam crescimento
na potência. Propõe-se que a implementação seja realizada no ambiente
de análise, pela questão de unificação e aproveitamento das
ferramentas, além de toda a capacidade já oferecida por esse ambiente.

Conforme a evolução da técnica para discriminação, será necessário
determinar o ponto operativo do detector de evento para garantir uma
melhor capacidade do \acs{nilm}. É preciso determinar o quanto os
falsos alarmes irão afetar a eficácia do \acs{nilm} de desagregar a
informação, sendo necessário estudar o caso para determinar quais são
as taxas de falso alarme que são aceitáveis sem deteriorar a operação.

Quanto ao otimizador por estratégia evolutiva, pretende-se ajustar os
métodos de competição para que a mesma só comece após algumas
gerações, estratégia importante para garantir que as espécies ajustem
suas estratégias evolutivas e se desenvolvam minimamente antes de
competir, para assim garantir uma competição justa entre elas. 

